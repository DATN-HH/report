% % \subsection{Động lực}

% % Việc phát triển hệ thống quản lý đặt món trong nhà hàng xuất phát từ nhu cầu thực tế và những lợi ích mà nó mang lại:

% % \begin{enumerate}
% %     \item Xu hướng chuyển đổi số trong ngành dịch vụ ăn uống: Ngày nay, việc ứng dụng công nghệ vào các lĩnh vực kinh doanh không chỉ là một lựa chọn mà đã trở thành một yêu cầu bắt buộc. Các nhà hàng cần áp dụng các giải pháp hiện đại để nâng cao hiệu quả hoạt động và mang lại trải nghiệm tốt hơn cho khách hàng.

% %     \item Tăng trải nghiệm khách hàng: Một hệ thống quản lý đặt món tự động giúp khách hàng dễ dàng tiếp cận thực đơn, tùy chỉnh các lựa chọn, đặt món và thanh toán nhanh chóng. Điều này góp phần tạo ấn tượng tích cực và giữ chân khách hàng lâu dài.

% %     \item Giảm áp lực cho nhân viên: Nhân viên nhà hàng thường gặp khó khăn khi phải xử lý nhiều đơn hàng cùng lúc, đặc biệt vào giờ cao điểm. Hệ thống tự động hóa giúp giảm thiểu sai sót và tăng năng suất làm việc của nhân viên.

% %     \item Tăng tính cạnh tranh trên thị trường: Với sự phát triển của các mô hình nhà hàng thông minh, việc áp dụng công nghệ không chỉ giúp tối ưu hóa hoạt động mà còn tạo lợi thế cạnh tranh với các đối thủ.

% %     \item Hỗ trợ quản lý hiệu quả: Một hệ thống hiện đại không chỉ dừng lại ở việc quản lý đặt món mà còn cung cấp các công cụ phân tích, báo cáo giúp nhà quản lý đưa ra các quyết định kinh doanh chiến lược dựa trên dữ liệu thực tế.
% % \end{enumerate}

% % Động lực từ thực tiễn và xu hướng công nghệ chính là yếu tố thúc đẩy việc thực hiện đề tài này, nhằm giải quyết các vấn đề hiện tại và hướng đến sự phát triển bền vững.

% \subsection{Động lực}

% Việc phát triển hệ thống quản lý đặt món trong nhà hàng xuất phát từ nhu cầu thực tế và những lợi ích mà nó mang lại, đặc biệt trong bối cảnh ngành công nghiệp nhà hàng tại Việt Nam đang chuyển đổi số mạnh mẽ như đã đề cập:

% \begin{enumerate}
%     \item \textbf{Xu hướng chuyển đổi số trong ngành dịch vụ ăn uống}: Như đã trình bày, sự tích hợp công nghệ, chẳng hạn như hệ thống POS và dịch vụ đặt món trực tuyến, đã trở thành yếu tố then chốt giúp nhà hàng nâng cao hiệu quả và đáp ứng nhu cầu khách hàng. Việc áp dụng một hệ thống quản lý đặt món hiện đại là một bước đi cần thiết để theo kịp xu hướng này.
    
%     \item \textbf{Tăng trải nghiệm khách hàng}: Hệ thống tự động hóa giúp khách hàng dễ dàng tiếp cận thực đơn, tùy chỉnh món ăn, đặt hàng nhanh chóng và thanh toán tiện lợi qua các nền tảng số, như đã thấy trong sự phát triển của dịch vụ giao đồ ăn trực tuyến. Điều này không chỉ cải thiện trải nghiệm mà còn xây dựng lòng trung thành của khách hàng.
    
%     \item \textbf{Giảm áp lực cho nhân viên}: Trong môi trường cạnh tranh và áp lực cao của ngành nhà hàng, nhân viên thường phải xử lý nhiều đơn hàng cùng lúc. Hệ thống quản lý đặt món giúp giảm thiểu sai sót, tối ưu hóa quy trình phục vụ, từ đó nâng cao năng suất làm việc, như đã đề cập trong các hạn chế của phương pháp quản lý truyền thống.
    
%     \item \textbf{Tăng tính cạnh tranh trên thị trường}: Với sự gia tăng của các mô hình nhà hàng thông minh và kênh giao đồ ăn trực tuyến như Shopee Food hay Grab Food, việc áp dụng công nghệ giúp nhà hàng tạo lợi thế cạnh tranh, thu hút khách hàng mới và duy trì vị thế trên thị trường.
    
%     \item \textbf{Hỗ trợ quản lý hiệu quả}: Hệ thống không chỉ tự động hóa quy trình đặt món mà còn cung cấp các công cụ phân tích doanh thu, quản lý kho và tối ưu hóa lịch làm việc, giúp nhà quản lý đưa ra quyết định chiến lược dựa trên dữ liệu thực tế, như vai trò của hệ thống POS đã được đề cập.
% \end{enumerate}

% Động lực từ nhu cầu thực tiễn, xu hướng công nghệ và những thách thức trong vận hành nhà hàng chính là yếu tố thúc đẩy việc thực hiện đề tài này, nhằm mang lại giải pháp toàn diện và bền vững.



\subsection{Động lực}

Việc nghiên cứu và phát triển một hệ thống quản lý đặt món hiện đại cho nhà hàng không chỉ là một đề xuất cải tiến mà còn xuất phát từ những động lực cấp thiết và cơ hội chiến lược, bám sát vào thực trạng và xu hướng của ngành dịch vụ ẩm thực tại Việt Nam như đã được phân tích chi tiết trong phần giới thiệu:

\begin{enumerate}
    \item \textbf{Yêu cầu cấp bách từ làn sóng Chuyển đổi số trong ngành Food\&Beverage:} Như đã nêu, ngành dịch vụ thực phẩm Việt Nam (đạt 26,9 tỷ USD năm 2023) đang chứng kiến sự chuyển mình mạnh mẽ nhờ công nghệ. Sự phổ biến của các hệ thống POS tiên tiến và sự bùng nổ của thị trường đặt món trực tuyến (đạt 1,4 tỷ USD, tăng 30\% năm 2023 \cite{USDA}) không chỉ là xu hướng mà đã trở thành một phần quan trọng trong mô hình kinh doanh hiện đại. Việc không tích hợp công nghệ, cụ thể là một hệ thống quản lý đặt món hiệu quả, đồng nghĩa với việc nhà hàng tự đặt mình vào thế bất lợi, có nguy cơ bị tụt hậu so với các đối thủ đang nhanh chóng thích nghi và khai thác lợi ích từ công nghệ số. Do đó, phát triển hệ thống này là một bước đi tất yếu để hòa nhập và cạnh tranh.

    \item \textbf{Khắc phục những hạn chế cố hữu của phương pháp quản lý truyền thống:} Quy trình vận hành dựa trên ghi chép thủ công và giao tiếp trực tiếp, như đã đề cập, tiềm ẩn nhiều rủi ro: sai sót trong ghi nhận đơn hàng, nhầm lẫn trong quá trình chuyển thông tin đến bếp, khó khăn trong việc theo dõi lịch sử và sở thích khách hàng, và lãng phí thời gian của cả nhân viên lẫn khách hàng. Những sai sót này không chỉ gây thất thoát doanh thu mà còn làm tổn hại đến uy tín thương hiệu và sự hài lòng của khách hàng. Một hệ thống quản lý đặt món tự động hóa sẽ giải quyết các nút thắt này, đảm bảo tính chính xác, minh bạch và hiệu quả cho toàn bộ quy trình từ lúc khách chọn món đến khi hoàn tất thanh toán.

    \item \textbf{Nâng tầm trải nghiệm khách hàng trong kỷ nguyên số và tiện ích:} Khách hàng hiện đại, đặc biệt tại các đô thị, ngày càng đề cao sự tiện lợi, tốc độ và khả năng kiểm soát trong trải nghiệm dịch vụ. Họ đã quen với các giao dịch số và mong muốn quy trình đặt món tại nhà hàng cũng mượt mà như vậy. Hệ thống quản lý đặt món hiện đại đáp ứng kỳ vọng này bằng cách cho phép khách hàng dễ dàng xem thực đơn (có thể kèm hình ảnh, mô tả chi tiết), tùy chỉnh món ăn theo sở thích cá nhân, đặt hàng nhanh chóng (có thể ngay tại bàn qua thiết bị di động hoặc kiosk) và thanh toán thuận tiện qua nhiều hình thức điện tử. Việc giảm thiểu thời gian chờ đợi, đảm bảo đơn hàng chính xác và mang lại sự chủ động cho khách hàng chính là yếu tố then chốt để tạo ấn tượng tích cực, xây dựng lòng trung thành và khuyến khích họ quay trở lại.

    \item \textbf{Tối ưu hóa hiệu quả vận hành và tăng cường năng lực cạnh tranh trong bối cảnh thách thức:} Ngành nhà hàng đang đối mặt với áp lực không nhỏ từ lạm phát, chi phí nguyên liệu và vận hành tăng cao (thể hiện qua con số hơn 30.000 cơ sở đóng cửa \cite{USDA}), cùng với tình trạng thiếu hụt lao động có kỹ năng. Trong bối cảnh đó, việc tối ưu hóa nguồn lực và quy trình là yếu tố sống còn. Hệ thống quản lý đặt món giúp giảm tải đáng kể công việc thủ công cho nhân viên phục vụ và thu ngân, giảm thiểu sai sót do con người, cho phép phân bổ nhân sự hiệu quả hơn. Quan trọng hơn, hệ thống cung cấp dữ liệu vận hành chi tiết (tương tự vai trò phân tích của POS), giúp nhà quản lý nắm bắt xu hướng tiêu dùng, quản lý tốt hơn, đưa ra các quyết định về giá cả, thực đơn và chiến lược kinh doanh dựa trên bằng chứng cụ thể, từ đó nâng cao hiệu quả và tạo lợi thế cạnh tranh sắc bén.
\end{enumerate}

Tóm lại, động lực cốt lõi thúc đẩy việc thực hiện đề tài này là sự cộng hưởng mạnh mẽ giữa nhu cầu giải quyết các bài toán thực tiễn trong quản lý nhà hàng, yêu cầu phải bắt kịp và tận dụng xu thế chuyển đổi số không thể đảo ngược, cùng với khát vọng nâng cao chất lượng dịch vụ, tối ưu hóa hiệu quả hoạt động và xây dựng lợi thế cạnh tranh bền vững cho các doanh nghiệp trong ngành ẩm thực Việt Nam.