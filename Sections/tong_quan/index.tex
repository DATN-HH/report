\section{TỔNG QUAN VỀ HỆ THỐNG}

\subsection{Bối cảnh kinh doanh (Business context)}
Trong bối cảnh ngành công nghiệp thực phẩm và dịch vụ ngày càng phát triển và cạnh tranh khốc liệt, việc ứng dụng công nghệ vào quản lý nhà hàng đã trở thành một xu hướng tất yếu. Hệ thống quản lý chuỗi nhà hàng là một giải pháp kỹ thuật số được thiết kế để hỗ trợ và tối ưu hóa các hoạt động vận hành hàng ngày của nhà hàng (đặc biệt là nhà hàng fine dining), từ đặt bàn, xử lý đơn hàng, thanh toán cho đến quản lý nhân sự. Một hệ thống quản lý chuỗi nhà hàng hiệu quả không chỉ giúp nâng cao trải nghiệm của khách hàng mà còn cải thiện hiệu suất làm việc của nhân viên và tối ưu hóa quản lý doanh thu. 

% \subsubsection{Mục tiêu và phạm vi hệ thống}
% Hệ thống quản lý nhà hàng \textbf{\textit{Menu+}} được phát triển nhằm mục đích đơn giản hóa và nâng cao hiệu quả các hoạt động vận hành hàng ngày trong một nhà hàng fine dining. Các chức năng chính của \textbf{\textit{Menu+}} bao gồm:
% \begin{itemize}
%     \item \textbf{Đặt bàn trực tuyến và quản lý bàn}: Giúp khách hàng dễ dàng đặt chỗ và hỗ trợ nhân viên theo dõi trạng thái bàn một cách chính xác.
%      \item \textbf{Quản lý bàn}: Hỗ trợ nhân viên theo dõi trạng thái bàn (trống, đã đặt, đang phục vụ) và thực hiện các thao tác như chuyển bàn khi cần thiết.
%     \item \textbf{Xử lý đơn hàng}: Cho phép nhân viên nhập đơn hàng, tùy chỉnh theo yêu cầu của khách và gửi trực tiếp đến hệ thống hiển thị bếp (Kitchen Display System - KDS).
%     \item \textbf{Thanh toán và quản lý doanh thu}: Tích hợp các phương thức thanh toán đa dạng, tạo điều kiện thuận lợi cho cả khách hàng và nhà hàng.
%     \item \textbf{Quản lý quan hệ khách hàng (CRM) và nhân sự}: Lưu trữ thông tin khách hàng, quản lý lịch làm việc của nhân viên và cung cấp các báo cáo chi tiết về hoạt động kinh doanh.
% \end{itemize}
% \textbf{\textit{Menu+}} linh hoạt và thân thiện với người dùng đóng vai trò quan trọng trong việc giúp các nhà hàng, đặc biệt là những nhà hàng mới khởi nghiệp, mở rộng quy mô hoạt động một cách hiệu quả.

% \subsubsection{Ràng buộc kinh doanh (Business Constraint)}

% exclude

\subsubsection{Tổng quan hệ thống quản lý nhà hàng}

Dưới đây là mô tả tổng quan về các thành phần cấu thành hệ thống quản lý chuỗi nhà hàng, với các chức năng thiết yếu nhằm hỗ trợ việc vận hành và điều phối các hoạt động trong một chuỗi nhà hàng hiện đại. Những hệ thống con này đóng vai trò quan trọng trong việc tối ưu hóa hiệu quả công việc, nâng cao chất lượng dịch vụ và đảm bảo sự hoạt động trơn tru của toàn bộ chuỗi.

\begin{figure}[H]
    \centering
    \includegraphics[width=15cm]{Images/so-do-he-thong.png}
    \vspace{0.5cm}
    \caption{Sơ đồ tổng quan về một phần mềm Quản lý Nhà hàng}
    \label{fig:my_label}
\end{figure}

\begin{itemize}
    \item \textbf{Hệ thống Quản lý Đơn hàng (Order Management System - OMS)}: Đóng vai trò quan trọng trong việc tiếp nhận và xử lý các đơn đặt món từ khách hàng. Khi khách hàng đặt món, thông tin đơn hàng sẽ được hệ thống OMS ghi nhận và phân phối đến các bộ phận liên quan trong nhà hàng, như bếp để chế biến món ăn, thu ngân để xử lý thanh toán, và đội ngũ giao hàng nếu có. Hệ thống OMS không chỉ giúp theo dõi tình trạng của từng đơn hàng mà còn quản lý các thông tin liên quan đến khách hàng và việc giao nhận món ăn. Đặc biệt, OMS sẽ giúp nhà quản lý theo dõi hiệu suất của đơn hàng và cập nhật kịp thời trạng thái để khách hàng có thể nhận được món ăn một cách nhanh chóng. Mọi thay đổi về đơn hàng sẽ được cập nhật liên tục để các bộ phận liên quan có thể xử lý ngay khi có sự thay đổi, ví dụ như khi khách hàng hủy đơn hoặc có yêu cầu thay đổi món.

    \item \textbf{Hệ thống Quản lý Bếp (Kitchen Management System - KMS)}: Tập trung vào việc quản lý quá trình chế biến món ăn từ khi nhận được đơn hàng từ OMS cho đến khi món ăn hoàn thành và sẵn sàng phục vụ khách. Thông qua KMS, nhà bếp có thể lên kế hoạch chế biến cho từng món ăn và theo dõi hiệu quả làm việc của từng đầu bếp. Mỗi đơn hàng sẽ được chia thành các công đoạn nhỏ để các nhân viên bếp dễ dàng theo dõi và thực hiện. Hệ thống này cũng đóng vai trò trong việc quản lý chất lượng món ăn, đảm bảo rằng món ăn được chế biến đúng quy trình và đạt tiêu chuẩn chất lượng. KMS cũng kết nối với các hệ thống khác như Inventory Management System (IMS) để tự động kiểm tra và yêu cầu bổ sung nguyên liệu khi kho hàng thiếu hụt.

    \item \textbf{Hệ thống Quản lý Kho (Inventory Management System - IMS)}: Quản lý và giám sát tình trạng nguyên liệu trong kho của các chi nhánh nhà hàng. IMS giúp nhà quản lý kiểm soát số lượng nguyên liệu, hạn sử dụng và các mặt hàng còn lại trong kho để đảm bảo rằng luôn có đủ nguyên liệu cho việc chế biến món ăn. Hệ thống này sẽ tự động thông báo khi lượng nguyên liệu sắp hết hoặc sắp hết hạn sử dụng, giúp bộ phận kho có thể lên kế hoạch nhập hàng bổ sung từ các nhà cung cấp. IMS không chỉ giúp đảm bảo việc cung cấp nguyên liệu kịp thời cho KMS, mà còn giúp giảm thiểu tình trạng thiếu hụt nguyên liệu, qua đó đảm bảo việc phục vụ khách hàng không bị gián đoạn. Hệ thống này cũng giúp tối ưu hóa việc sử dụng nguyên liệu, giảm lãng phí và giúp duy trì chi phí vận hành hợp lý.

    \item \textbf{Hệ thống Quản lý Thanh toán (Payment Management System - PMS)}: Chịu trách nhiệm xử lý tất cả các giao dịch thanh toán của khách hàng sau khi món ăn được giao đến bàn hoặc giao tận nơi. Khi món ăn hoàn thành và khách hàng chuẩn bị thanh toán, hệ thống PMS sẽ tự động tính toán giá trị đơn hàng, bao gồm thuế, chiết khấu, và các khoản phí khác nếu có. Hệ thống này hỗ trợ nhiều phương thức thanh toán khác nhau, từ tiền mặt, thẻ tín dụng, thẻ ghi nợ, cho đến các phương thức thanh toán trực tuyến. Sau khi thanh toán hoàn tất, PMS sẽ in hóa đơn cho khách và cập nhật dữ liệu doanh thu vào hệ thống tài chính. Hệ thống này cũng đồng bộ hóa dữ liệu với các hệ thống báo cáo để nhà quản lý có cái nhìn tổng quan về tình hình tài chính của từng chi nhánh. PMS còn có khả năng theo dõi và phân tích các xu hướng chi tiêu của khách hàng để đưa ra các chiến lược giá hợp lý và tăng trưởng doanh thu.

    \item \textbf{Hệ thống Quản lý Nhân sự (Human Resource Management System - HRMS)}: Quản lý thông tin nhân viên, lịch làm việc và hiệu suất công việc của các nhân viên trong chuỗi nhà hàng. Hệ thống này theo dõi số giờ làm việc, ca làm việc của nhân viên, và các yếu tố liên quan đến chấm công, bảo hiểm, tiền lương. Ngoài ra, HRMS còn hỗ trợ việc phân công công việc cho các nhân viên bếp, thu ngân, và các bộ phận khác, giúp việc tổ chức công việc trở nên khoa học và hợp lý. Các tính năng như theo dõi kỳ nghỉ, đào tạo và phát triển nhân viên cũng được HRMS đảm bảo. Một trong những chức năng quan trọng của HRMS là giúp tối ưu hóa quy trình tuyển dụng, giúp tuyển chọn nhân viên phù hợp với yêu cầu công việc. Ngoài ra, HRMS còn hỗ trợ các bộ phận quản lý nhân sự của từng chi nhánh trong việc đánh giá hiệu suất làm việc và cải thiện chất lượng nhân sự.

    \item \textbf{Hệ thống Báo cáo \& Phân tích (Reporting \& Analytics System - R\&A)}: Quản lý có cái nhìn tổng quan về hiệu suất của toàn chuỗi nhà hàng. Hệ thống này thu thập và tổng hợp dữ liệu từ các hệ thống khác nhau như OMS, KMS, IMS, PMS và HRMS để tạo ra các báo cáo chi tiết. Những báo cáo này không chỉ về tình hình doanh thu, mà còn cung cấp các thông tin liên quan đến hiệu quả công việc của nhân viên, tình trạng kho nguyên liệu, mức độ hài lòng của khách hàng, và nhiều yếu tố khác. Thông qua phân tích dữ liệu, nhà quản lý có thể đưa ra những quyết định chiến lược giúp cải thiện quy trình hoạt động, tối ưu hóa chi phí và nâng cao chất lượng dịch vụ. Hệ thống cũng hỗ trợ tạo ra các báo cáo tài chính, giúp các chi nhánh có thể theo dõi doanh thu và chi phí một cách chi tiết.

    \item \textbf{Hệ thống Quản lý Chuỗi Cung ứng (Supply Chain Management System - SCM)}: Quản lý và điều phối mối quan hệ với các nhà cung cấp, đảm bảo nguồn nguyên liệu luôn được cung cấp đầy đủ và đúng chất lượng. SCM theo dõi tình trạng đơn hàng từ khi nguyên liệu được đặt hàng cho đến khi nhận được hàng và đưa vào kho. Hệ thống này không chỉ giúp kiểm soát chất lượng nguồn cung mà còn tối ưu hóa quá trình giao nhận nguyên liệu, giảm thiểu tình trạng thiếu hụt hoặc tồn đọng hàng hóa trong kho. Bằng cách kết hợp dữ liệu từ IMS và KMS, SCM có thể dự báo nhu cầu nguyên liệu cho các món ăn, từ đó có kế hoạch đặt hàng hiệu quả hơn.
    
    \item \textbf{Hệ thống Quản lý Khách hàng (Customer Relationship Management - CRM)}: Lưu trữ và quản lý thông tin của khách hàng, bao gồm các thông tin cá nhân, lịch sử đơn hàng và các thói quen tiêu dùng. Hệ thống CRM giúp nhà hàng không chỉ lưu giữ thông tin khách hàng mà còn tạo dựng mối quan hệ lâu dài với khách hàng thông qua các chương trình khách hàng thân thiết và các ưu đãi cá nhân hóa. Bằng cách phân tích dữ liệu từ CRM, nhà hàng có thể hiểu rõ hơn về nhu cầu và sở thích của khách hàng, từ đó đề xuất các món ăn phù hợp, tạo ra những trải nghiệm cá nhân hóa. Hệ thống này còn hỗ trợ việc gửi các thông báo về chương trình khuyến mãi, sự kiện đặc biệt, hoặc thông tin về các sản phẩm mới đến khách hàng, giúp duy trì và phát triển mối quan hệ với khách hàng cũ và thu hút khách hàng mới.

    \item \textbf{Hệ thống Quản lý Marketing \& Khuyến mãi (Marketing \& Promotion Management System)}: Lên kế hoạch, triển khai và theo dõi các chiến dịch marketing và chương trình khuyến mãi. Hệ thống này hỗ trợ việc tạo ra các chiến dịch quảng bá các món ăn, khuyến mãi theo mùa hoặc các chương trình giảm giá đặc biệt cho khách hàng. Marketing \& Promotion Management System cung cấp các công cụ để tạo mã giảm giá, quản lý các chương trình khuyến mãi và theo dõi hiệu quả của từng chiến dịch. Hệ thống này còn giúp phân tích dữ liệu khách hàng từ CRM, từ đó xác định đối tượng mục tiêu cho các chiến dịch marketing, giúp tăng tỷ lệ chuyển đổi và tối ưu hóa chi phí marketing. Ngoài ra, các chiến dịch và khuyến mãi cũng có thể được tích hợp với hệ thống thanh toán để khách hàng có thể dễ dàng sử dụng các ưu đãi khi thanh toán.

    \item \textbf{Hệ thống Quản lý Phản hồi \& Khiếu nại (Feedback \& Complaint Management System)}: Nhận và xử lý các phản hồi, khiếu nại từ khách hàng. Việc lắng nghe và giải quyết nhanh chóng các vấn đề của khách hàng giúp nhà hàng cải thiện chất lượng dịch vụ và tạo dựng lòng tin của khách hàng. Hệ thống này giúp theo dõi các khiếu nại về món ăn, thái độ phục vụ, không gian nhà hàng hoặc các vấn đề khác. Sau khi nhận được phản hồi hoặc khiếu nại từ khách hàng, hệ thống sẽ tự động phân loại và chuyển đến các bộ phận liên quan để xử lý, từ đó giúp khách hàng cảm thấy hài lòng hơn với dịch vụ của nhà hàng. Hệ thống này cũng có chức năng theo dõi các phản hồi tích cực để có thể ghi nhận và thưởng cho những nhân viên hoặc bộ phận có đóng góp xuất sắc.

    \item \textbf{Hệ thống Quản lý Chuỗi Nhà hàng (Chain Management System - CMS)}: Trung tâm quản lý tổng thể của chuỗi các chi nhánh nhà hàng. Hệ thống này giúp giám sát các hoạt động của tất cả các chi nhánh từ một hệ thống tập trung, bao gồm việc theo dõi doanh thu, tồn kho, nhân sự, cũng như các hoạt động vận hành khác của từng chi nhánh. CMS không chỉ giúp đồng bộ hóa các quy trình giữa các chi nhánh mà còn cung cấp các báo cáo tài chính và hoạt động chi tiết để hỗ trợ các quyết định quản lý chiến lược. Hệ thống này tích hợp dữ liệu từ các hệ thống khác như PMS, KMS, IMS, giúp nhà quản lý chuỗi có thể theo dõi tình trạng của từng chi nhánh và đưa ra các quyết định kịp thời để tối ưu hóa hoạt động của toàn chuỗi.

    \item \textbf{Hệ thống Quản lý Giao hàng (Delivery Management System)}: Quản lý các đơn hàng giao tận nơi, bao gồm việc điều phối các nhân viên giao hàng, theo dõi tình trạng đơn hàng và tối ưu hóa thời gian giao hàng. Hệ thống này giúp các nhân viên giao hàng nhận được thông tin đơn hàng một cách nhanh chóng, biết rõ địa chỉ giao hàng và các yêu cầu đặc biệt của khách hàng (nếu có). Hệ thống còn giúp theo dõi trạng thái của đơn hàng từ khi rời khỏi nhà hàng cho đến khi giao đến tay khách hàng, đồng thời cung cấp các công cụ để quản lý các tuyến đường giao hàng sao cho hiệu quả và tiết kiệm thời gian. Việc tích hợp hệ thống này với OMS giúp cập nhật trạng thái của đơn hàng cho khách hàng trong thời gian thực và đảm bảo dịch vụ giao hàng nhanh chóng, chính xác.
\end{itemize}

\subsubsection{Chính sách vận hành (Policy)}
Các nhà hàng hiện đại ngày nay thường áp dụng nhiều chính sách linh hoạt nhằm nâng cao chất lượng phục vụ và giảm thiểu các rủi ro trong kinh doanh. Các chính sách phổ biến được áp dụng có thể kể tới như sau:

\begin{itemize}
    \item \textbf{Đa dạng hóa phương thức thanh toán}: Chính sách này tập trung vào việc cung cấp đa dạng các hình thức thanh toán, như tiền mặt, thẻ tín dụng, ví điện tử và mã QR. Bên cạnh đó, các nhà hàng cũng triển khai hệ thống quản lý thanh toán tự động để rút ngắn thời gian xử lý, giảm thiểu sai sót và tăng trải nghiệm hài lòng của khách hàng. Chính sách này đã được các nhà hàng như \textit{Ngưu Phồn} và \textit{Cơm niêu Sài Gòn} áp dụng rất hiệu quả.
    
    \item \textbf{Yêu cầu đặt cọc khi đặt bàn}: Để giảm thiểu tổn thất khi khách hàng hủy đặt bàn vào phút chót hoặc không đến nhà hàng mà không báo trước, nhiều nhà hàng áp dụng chính sách yêu cầu khách hàng đặt cọc trước một khoản tiền nhất định, thường dao động từ 10\% đến 20\% tổng giá trị bàn tiệc. Khoản đặt cọc này sẽ được khấu trừ vào hóa đơn thanh toán hoặc bị giữ lại nếu khách hủy đặt bàn trễ hơn thời gian quy định. Một số nhà hàng áp dụng thành công chính sách này là \textit{Nhà Hàng Phúc Thành} và \textit{Vân Nghĩa Palace}.
    
    \item \textbf{Xác nhận đặt bàn trước ngày hẹn}: Chính sách này yêu cầu nhân viên nhà hàng chủ động liên hệ với khách hàng trước ngày đặt bàn để xác nhận lại thông tin và nhắc nhở khách đến đúng giờ. Điều này không chỉ giúp giảm tình trạng khách quên hay thay đổi kế hoạch mà không thông báo, mà còn giúp nhà hàng quản lý hiệu quả hơn trong việc chuẩn bị dịch vụ. Nhà hàng áp dụng tiêu biểu chính sách này là \textit{Nhà Hàng Phúc Thành}.
    
    \item \textbf{Chính sách hủy đặt bàn và hoàn tiền rõ ràng}: Các nhà hàng thường đưa ra quy định chi tiết về thời hạn hủy đặt bàn và mức phí áp dụng khi hủy. Điều này giúp khách hàng hiểu rõ trách nhiệm và quyền lợi của mình khi sử dụng dịch vụ, đồng thời tạo ra sự minh bạch và uy tín trong hoạt động kinh doanh. Chính sách này được áp dụng rõ ràng tại \textit{Nhà Hàng Ocean Bay Vũng Tàu}.
    
    \item \textbf{Điều kiện hủy đơn hàng}: Chính sách này quy định rõ ràng các điều kiện và thời điểm cụ thể khách hàng được phép hủy đơn hàng, thường là trước khi nhà hàng xác nhận và bắt đầu chuẩn bị món ăn. Điều này giúp nhà hàng giảm thiểu tổn thất về nguyên vật liệu và công sức chế biến không cần thiết. Một ví dụ cụ thể về chính sách này là tại nhà hàng \textit{Patyko}.
    
    \item \textbf{Phí hủy đơn hàng}: Để bảo vệ quyền lợi và giảm thiểu thiệt hại khi khách hàng hủy đơn sau khi nhà hàng đã bắt đầu chế biến, nhiều nhà hàng đặt ra quy định thu phí hủy hoặc không hoàn lại tiền đặt cọc trong một số trường hợp nhất định. Chính sách này nhằm bù đắp một phần chi phí nguyên vật liệu và nhân công đã bỏ ra. Điển hình trong việc áp dụng chính sách này là \textit{Nhà Hàng Khoái}.
    
    \item \textbf{Chính sách đổi trả sản phẩm}: Để đảm bảo quyền lợi khách hàng và duy trì chất lượng dịch vụ, các nhà hàng thường áp dụng chính sách đổi trả rõ ràng. Khách hàng được quyền đổi hoặc trả lại món ăn trong các trường hợp món bị lỗi, hỏng, không thể sử dụng hoặc không đảm bảo vệ sinh an toàn thực phẩm. Chính sách này giúp tạo dựng lòng tin và gia tăng uy tín của nhà hàng. Một số nhà hàng nổi bật áp dụng thành công chính sách này là \textit{Sườn Mười} và \textit{Nhà Hàng Khoái}.
    
\end{itemize}



\begin{table}[H]
\centering
\begin{tabular}{|p{4cm}|p{8cm}|p{4cm}|}
\hline
\textbf{Tên chính sách} & \textbf{Mô tả} & \textbf{Nhà hàng áp dụng} \\
\hline
Đa dạng hóa phương thức thanh toán & Cung cấp nhiều phương thức thanh toán như tiền mặt, thẻ tín dụng, ví điện tử và mã QR, đồng thời áp dụng hệ thống quản lý thanh toán tự động để giảm thiểu sai sót và tăng tốc độ phục vụ. & Ngưu Phồn, Cơm niêu Sài Gòn \\
\hline
Yêu cầu đặt cọc khi đặt bàn & Để giảm thiểu tình trạng khách hàng hủy bàn vào phút chót hoặc không đến mà không báo trước, nhiều nhà hàng yêu cầu khách đặt cọc trước một khoản tiền, thường khoảng 10-20\% tổng giá trị bàn tiệc. & Nhà Hàng Phúc Thành, Vân Nghĩa Palace \\
\hline
Xác nhận đặt bàn trước ngày hẹn & Trước ngày đặt bàn, nhân viên nhà hàng nên gọi điện hoặc nhắn tin xác nhận lại với khách hàng để nhắc nhở và đảm bảo họ sẽ đến. & Nhà Hàng Phúc Thành \\
\hline
Chính sách hủy và hoàn tiền rõ ràng & Nhà hàng cần quy định rõ ràng về thời hạn hủy đặt bàn và mức phí hủy, giúp khách hàng nắm rõ quyền lợi và trách nhiệm của mình. & Nhà Hàng Ocean Bay Vũng Tàu \\
\hline
Điều kiện hủy đơn hàng & Quy định rõ ràng về thời điểm và điều kiện khách hàng có thể hủy đơn hàng, đại khái hơn là trước khi nhà hàng xác nhận và lên đơn hàng. & Patyko \\
\hline
Phí hủy đơn hàng & Áp dụng phí hủy hoặc không hoàn tiền đặt cọc nếu khách hủy sau một thời điểm nhất định, giúp bù đắp chi phí nguyên liệu và công sức chuẩn bị. & Nhà Hàng Khoái \\
\hline
Chính sách đổi trả sản phẩm & Chấp nhận đổi, trả các sản phẩm bị lỗi, hỏng, không thể sử dụng hoặc không đảm bảo vệ sinh an toàn thực phẩm. & Sườn Mười, Nhà Hàng Khoái \\
\hline
\end{tabular}
\caption{Tổng hợp các chính sách, mô tả và tham khảo từ các nhà hàng hoặc hệ thống liên quan}
\end{table}

\subsection{Người dùng và mục đích hệ thống Menu+}

Mục đích của hệ thống quản lý nhà hàng \textbf{\textit{Menu+}} là cung cấp một giải pháp toàn diện nhằm tối ưu hóa và nâng cao hiệu quả các hoạt động vận hành hàng ngày trong một nhà hàng fine dining. Hệ thống được phát triển với mục tiêu hỗ trợ các nhóm người dùng chính trong nhà hàng, bao gồm khách hàng, thu ngân, quản lý, quản trị viên, bếp trưởng, nhân viên phục vụ và nhân viên vệ sinh.

\begin{figure}[H]
    \centering
    \includegraphics[width=15cm]{Images/OMS-Page-2.png}
    \vspace{0.5cm}
    \caption{Các người dùng trong hệ thống}
    \label{fig:my_label}
\end{figure}

% \begin{table}[h!]
% \centering
% \begin{tabular}{|p{3cm}|p{3cm}|p{9cm}|}
% \hline
% \textbf{Mã đối tượng} & \textbf{Tên} & \textbf{Mô tả} \\ \hline
% US-01 & Customer (Khách hàng) & Xem menu trực tuyến, đặt bàn hoặc order trực tiếp tại nhà hàng, thanh toán và sử dụng mã khuyến mãi. \\ \hline
% US-02 & Cashier (Thu ngân) & Thực hiện xác nhận thanh toán, gộp hoặc tách bill, hỗ trợ khách hàng sử dụng mã khuyến mãi, nhập số tiền nhận và tính tiền thối lại. \\ \hline
% US-03 & Manager (Quản lý) & Quản lý order, doanh thu, nhân sự, chia ca, thêm nhân sự mới, và điều hành chung hệ thống nhà hàng. \\ \hline
% US-04 & Administrator (Quản trị viên) & Quản lý và duy trì hệ thống, cấu hình menu, điều chỉnh các thông tin liên quan đến hệ thống chung. \\ \hline
% US-05 & Chef (Bếp trưởng) & Xem Kitchen Display System (KDS), xác nhận món ăn từ trạng thái "not ready", chuyển sang "cook" và cuối cùng là "ready to serve". \\ \hline
% US-06 & Waiter (Nhân viên phục vụ) & Giúp khách hàng đặt bàn, order món, lấy món ăn từ bếp, phục vụ món ăn, chuyển bàn, thực hiện self-order cho khách, cập nhật trạng thái món ăn là đã phục vụ. \\ \hline
% US-07 & Cleaning Staff (Nhân viên vệ sinh) & Dọn dẹp bàn ăn sau khi khách rời đi, cập nhật tình trạng bàn trên hệ thống để sẵn sàng phục vụ khách tiếp theo. \\ \hline
% \end{tabular}
% \caption{Bảng tổng hợp đối tượng và chức năng trong hệ thống quản lý nhà hàng}
% \label{tab:restaurant_objects}
% \end{table}

\textbf{Các chức năng chính của Menu+}

Hệ thống Menu+ bao gồm các chức năng chính giúp các nhóm người dùng trên thực hiện các công việc hàng ngày một cách hiệu quả và nhanh chóng:

\begin{itemize}


    \item Đặt bàn trực tuyến và quản lý bàn: Giúp khách hàng dễ dàng đặt chỗ và hỗ trợ nhân viên theo dõi trạng thái bàn (trống, đã đặt, đang phục vụ).

    \item Quản lý bàn: Cung cấp công cụ cho nhân viên theo dõi và quản lý bàn trong nhà hàng.

    \item Xử lý đơn hàng: Cho phép nhân viên nhập và gửi đơn hàng đến hệ thống Kitchen Display System (KDS) cho bếp.

    \item Thanh toán và quản lý doanh thu: Tích hợp các phương thức thanh toán đa dạng, tạo thuận tiện cho cả khách hàng và nhà hàng.

    \item Quản lý quan hệ khách hàng (CRM) và nhân sự: Lưu trữ thông tin khách hàng, quản lý lịch làm việc của nhân viên và cung cấp các báo cáo chi tiết về hoạt động kinh doanh.
\end{itemize}

\subsection{Đặc tả chức năng}
\begin{longtable}{|m{1.5cm}|m{3.5cm}|m{4.5cm}|m{5cm}|}
\hline
\textbf{Mã} & \textbf{Tên Người Dùng} & \textbf{Vai Trò Thực Tế} & \textbf{Mô Tả Ngắn} \\
\hline
\endhead % Header cho các trang tiếp theo

\hline
\endfoot % Footer cho bảng

\hline
\endlastfoot % Footer cho trang cuối cùng

US-01 & Quản lý nhà hàng & Chủ nhà hàng hoặc người quản lý cấp cao & Quản lý tổng thể hoạt động, cấu hình hệ thống (giá bàn, \% đặt cọc, thời gian gọi bot), xem báo cáo toàn diện, quản lý nhân viên và thực đơn. \\
\hline
US-02 & Nhân viên phục vụ & Waiter/Waitress & Sử dụng POS để nhận đơn hàng tại bàn, quản lý trạng thái bàn, xử lý thanh toán (bao gồm cả việc trừ tiền đặt cọc), tương tác với hệ thống bếp. \\
\hline
US-03 & Nhân viên lễ tân & Host/Hostess (Có thể là Nhân viên phục vụ đảm nhiệm) & Quản lý sơ đồ tầng, trạng thái bàn, nhận và quản lý đặt chỗ trực tiếp hoặc qua điện thoại (ít tương tác hệ thống hơn khách hàng tự đặt). \\
\hline
US-04 & Nhân viên bếp & Chef, Cook, Kitchen Assistant & Tương tác chính với Màn hình hiển thị Bếp (KDS) hoặc máy in phiếu bếp để xem chi tiết đơn hàng (bao gồm món đặt trước) và cập nhật trạng thái chuẩn bị. \\
\hline
US-05 & Nhân viên thu ngân & Cashier (Có thể là Nhân viên phục vụ hoặc Quản lý) & Chịu trách nhiệm đóng/mở phiên POS, đối soát tiền mặt cuối ngày, xử lý các giao dịch thanh toán. \\
\hline
US-06 & Kế toán & Accountant/Bookkeeper & Truy cập dữ liệu bán hàng đã được tổng hợp, báo cáo tài chính, quản lý công nợ (nếu có), đối soát doanh thu và tiền đặt cọc. \\
\hline
US-07 & Nhân viên (Chung) & Bất kỳ nhân viên nào cần xem lịch làm việc & Xem lịch làm việc cá nhân được phân công qua Employee Portal, có thể có quyền yêu cầu đổi ca hoặc báo nghỉ. \\
\hline
US-08 & Khách hàng & Customer/Guest & Tương tác qua giao diện web/app để đặt bàn, đặt món ăn trước, thanh toán đặt cọc, đặt hàng mang về hoặc giao hàng. \\
\hline
US-09 & Nhân viên hỗ trợ/ Vận hành & Support/Operations Staff & Tiếp nhận và xử lý các yêu cầu hỗ trợ từ khách hàng. \\
\hline
US-10 & Quản trị viên Hệ thống & System Administrator (Có thể là Quản lý nhà hàng hoặc IT) & Thực hiện các cấu hình kỹ thuật sâu, quản lý tích hợp (Shipday, Bot), quản lý tài khoản người dùng và phân quyền chi tiết. \\
\hline
\caption{Danh sách Người dùng Hệ thống} \label{tab:users} \\

\end{longtable}

\newpage % Ngắt trang



\begin{longtable}{|m{2.5cm}|m{2.5cm}|m{5cm}|m{5cm}|}
\hline
\textbf{Người dùng} & \textbf{Mã chức năng} & \textbf{Tên chức năng} & \textbf{Mô tả ngắn} \\
\hline
\endhead % Header cho các trang tiếp theo

\midrule
\endfoot % Footer cho bảng

\bottomrule
\endlastfoot % Footer cho trang cuối cùng

% === US-01: Quản lý nhà hàng ===
\multicolumn{4}{|l|}{\textbf{US-01: Quản lý nhà hàng}} \\ \hline
\multirow{15}{=}[2pt]{US-01: Quản lý nhà hàng} & FR-MD01-01 & Tạo ca làm việc & Cho phép định nghĩa các ca làm việc mới (thời gian, vai trò, số lượng). \\
& FR-MD01-02 & Gán nhân viên vào ca & Chỉ định nhân viên cụ thể cho các vị trí trống trong ca làm việc. \\
& FR-MD01-03 & Xem lịch biểu Gantt & Hiển thị lịch làm việc dạng Gantt, lọc theo vai trò, xem theo ngày/tuần/tháng. \\
& FR-MD01-04 & (Kích hoạt) Phát hiện trùng lịch & Kích hoạt hệ thống kiểm tra trùng lịch khi gán nhân viên. \\
& FR-MD01-05 & Xuất bản và Thông báo lịch & Công khai lịch làm việc và kích hoạt gửi thông báo đến nhân viên. \\
& FR-MD01-07 & Sao chép lịch tuần & Sao chép nhanh lịch làm việc của một tuần sang tuần khác. \\
& FR-MD01-08 & Quản lý vai trò công việc & Định nghĩa các vai trò công việc trong nhà hàng. \\
& FR-MD01-10 & Xem lịch theo vai trò & Lọc và xem lịch làm việc của các nhân viên theo vai trò cụ thể. \\ \cline{2-4}
& FR-MD02-01 & Tạo Sản phẩm Mới (Món ăn/Đồ uống) & Thêm món ăn, đồ uống mới vào hệ thống (tên, giá, loại...). \\
& FR-MD02-02 & Chỉnh sửa Thông tin Sản phẩm & Cập nhật chi tiết sản phẩm đã có (giá, mô tả, ảnh...). \\
& FR-MD02-03 & Lưu trữ/Hủy kích hoạt Sản phẩm & Ẩn/Hiện sản phẩm khỏi các giao dịch mà không xóa hẳn. \\
& FR-MD02-04 & Quản lý Danh mục Sản phẩm POS & Tạo, sửa, xóa, sắp xếp các danh mục hiển thị trên POS. \\
& FR-MD02-05 & Định nghĩa Thuộc tính \& Giá trị (cho Biến thể) & Định nghĩa các đặc tính (size, độ cay) và lựa chọn (S, M, L). (Có thể là US-10) \\
& FR-MD02-06 & Cấu hình Biến thể Sản phẩm & Áp dụng thuộc tính/giá trị vào sản phẩm gốc để tạo biến thể, cấu hình giá riêng. \\
& FR-MD02-07 & Thiết lập Loại Sản phẩm & Xác định loại sản phẩm (Consumable, Stockable, Service). \\
& FR-MD02-08 & Cấu hình Hiển thị trên POS & Đánh dấu sản phẩm bán trên POS và gán vào danh mục POS. \\
& FR-MD02-09 & Quản lý Hình ảnh Sản phẩm & Tải lên, thay thế, xóa hình ảnh đại diện cho sản phẩm. \\
& FR-MD02-10 & Cấu hình In Bếp/Hiển thị KDS & Chỉ định danh mục sản phẩm nào gửi đến máy in/KDS nào. \\
& FR-MD02-11 & Định nghĩa Sản phẩm Tùy chọn/Phụ thu & Tạo các sản phẩm nhỏ (Extra cheese) để dùng làm modifier trên POS. \\ \cline{2-4}
& FR-MD03-11 & Cấu hình Tham số Đặt chỗ & Thiết lập giờ hoạt động, giới hạn khách, quy tắc đặt cọc, giá bàn... \\
& FR-MD03-12 & Xem Danh sách Đặt chỗ & Xem danh sách tổng hợp các lượt đặt chỗ và trạng thái. \\
& FR-MD03-13 & Xem Chi tiết Đặt chỗ & Xem thông tin chi tiết của một lượt đặt chỗ cụ thể. \\
& FR-MD03-14 & Tạo/Sửa Đặt chỗ Thủ công & Tạo hoặc sửa đặt chỗ cho khách qua kênh offline (điện thoại...). \\
& FR-MD03-15 & Quản lý Trạng thái Đặt chỗ & Thay đổi trạng thái đặt chỗ (Xác nhận, Đã đến, Hủy...). \\
& FR-MD03-16 & Xem Danh sách Món đặt trước & Xem các món cần chuẩn bị cho các đặt chỗ sắp tới. \\ \cline{2-4}
& FR-MD04-05 & Cấu hình Dịch vụ Bot Call & Cấu hình tham số tích hợp Bot Call (số ngày gọi, kịch bản, số hỗ trợ...). (Có thể là US-10) \\ \cline{2-4}
& FR-MD05-01 & Mở phiên làm việc POS & Bắt đầu phiên làm việc POS, nhập tiền mặt đầu ca. (Thường là US-05) \\
& FR-MD05-13 & Đóng Phiên làm việc POS & Kết thúc phiên làm việc, tổng kết, đối chiếu tiền mặt. (Thường là US-05) \\
& FR-MD05-14 & Chuyển bàn/Ghép bàn & Di chuyển hoặc gộp đơn hàng giữa các bàn. \\
& FR-MD05-15 & Hủy món/Hủy đơn (Void) & Hủy bỏ món hoặc đơn hàng (có thể cần quyền quản lý). \\ \cline{2-4}
& FR-MD09-03 & Xem Báo cáo Doanh thu Phiên POS & Xem báo cáo tổng kết chi tiết của các phiên POS đã đóng. \\
& FR-MD09-04 & Xem Báo cáo Bán hàng theo Sản phẩm/Danh mục & Xem thống kê số lượng, doanh thu theo từng món ăn/danh mục. \\
& FR-MD09-05 & Xem Báo cáo Hiệu suất Nhân viên (POS) & Xem doanh thu, số đơn hàng theo từng nhân viên POS. \\
& FR-MD09-06 & Xem Báo cáo Tiền đặt cọc & Xem báo cáo tổng hợp tình hình thu, sử dụng, mất cọc. \\
& FR-MD09-07 & Xem Báo cáo Doanh thu theo Loại hình & Phân tích doanh thu theo Eat-in, Takeout, Delivery. \\
& FR-MD09-08 & Xuất dữ liệu Báo cáo & Xuất dữ liệu báo cáo ra file Excel/CSV. \\ \cline{2-4}
& FR-MD10-04 & Cấu hình Chung của Hệ thống & Cấu hình thông tin công ty, logo, tiền tệ, email... (Có thể là US-10) \\
& FR-MD10-05 & Cấu hình Tích hợp Bên thứ ba & Quản lý API keys cho Cổng thanh toán, Bot Call, Shipday. (Có thể là US-10) \\
& FR-MD10-06 & Cấu hình Tham số Nghiệp vụ Đặc thù & Cấu hình tỷ lệ cọc, giá bàn, số ngày gọi bot... (Có thể là US-10) \\
\hline

% === US-02: Nhân viên phục vụ ===
\multicolumn{4}{|l|}{\textbf{US-02: Nhân viên phục vụ}} \\ \hline
\multirow{12}{=}[2pt]{US-02: Nhân viên phục vụ} & FR-MD05-02 & Truy cập Sơ đồ tầng \& Chọn bàn & Xem trạng thái bàn và chọn bàn để phục vụ. \\
& FR-MD05-03 & Bắt đầu/Mở đơn hàng tại bàn & Mở giao diện đơn hàng cho bàn đã chọn. \\
& FR-MD05-04 & Tải và Xác nhận Món ăn Đặt trước & Xem và xác nhận các món khách đã đặt trước online. \\
& FR-MD05-05 & Thêm món ăn/đồ uống vào đơn hàng & Nhận order tại bàn và thêm món vào POS. \\
& FR-MD05-06 & Xử lý Yêu cầu đặc biệt/Ghi chú bếp & Thêm ghi chú (ít cay, dị ứng...) vào món ăn/đơn hàng. \\
& FR-MD05-07 & Gửi đơn hàng xuống Bếp/Bar & Gửi thông tin món cần chuẩn bị đến bếp/bar. \\
& FR-MD05-08 & Yêu cầu/In Hóa đơn Tạm tính & In bill tạm tính cho khách kiểm tra. \\
& FR-MD05-09 & (Kích hoạt) Áp dụng Tiền Đặt cọc vào Hóa đơn & Kích hoạt việc trừ cọc khi vào màn hình thanh toán. \\
& FR-MD05-10 & Tách hóa đơn (Split Bill) & Chia hóa đơn cho khách thanh toán riêng. \\
& FR-MD05-11 & Xử lý Thanh toán & Nhận thanh toán từ khách (tiền mặt, thẻ...), xử lý tiền boa. \\
& FR-MD05-12 & Đóng Đơn hàng và Bàn & Đóng đơn hàng và giải phóng bàn sau khi khách thanh toán. \\
& FR-MD05-14 & Chuyển bàn/Ghép bàn & Di chuyển hoặc gộp đơn hàng giữa các bàn. \\
& FR-MD05-15 & Hủy món/Hủy đơn (Void) & Hủy món/đơn (có thể cần quyền hoặc xác nhận của quản lý). \\ \cline{2-4}
& FR-MD06-01 & Chọn Chế độ Bán Mang về & Chuyển sang giao diện bán mang về trên POS. \\
& FR-MD06-02 & Tạo Đơn hàng Mang về & Khởi tạo đơn hàng mới cho khách mang về. \\
& FR-MD06-03 & (Tùy chọn) Liên kết Khách hàng & Gán đơn hàng mang về với khách hàng (nếu cần). \\
& FR-MD06-04 & Thêm món vào Đơn hàng Mang về & Thêm món vào đơn hàng mang về. \\
& FR-MD06-05 & Xử lý Ghi chú cho Đơn Mang về & Thêm ghi chú cho đơn hàng mang về. \\
& FR-MD06-06 & Gửi đơn Mang về xuống Bếp/Bar & Gửi món của đơn mang về xuống bếp/bar. \\
& FR-MD06-07 & (Kích hoạt) Áp dụng Đặt cọc (Nếu Đặt trước Online) & Kích hoạt trừ cọc cho đơn takeout đặt online. \\
& FR-MD06-08 & Thanh toán Đơn hàng Mang về & Nhận thanh toán cho đơn hàng mang về tại quầy. \\
& FR-MD06-09 & (Nhận) In Hóa đơn/Phiếu thu Mang về & Nhận hóa đơn in ra để đưa khách. \\
& FR-MD06-10 & Đóng Đơn hàng Mang về & Đóng đơn hàng mang về sau khi thanh toán/giao hàng. \\ \cline{2-4}
& FR-MD07-01 & Chọn Chế độ Giao hàng & Chuyển sang giao diện xử lý đơn giao hàng. \\
& FR-MD07-02 & Tạo/Mở Đơn hàng Giao hàng & Tạo/mở đơn hàng giao đi. \\
& FR-MD07-03 & Liên kết/Nhập Thông tin Khách hàng Giao hàng & Nhập/chọn thông tin khách và địa chỉ giao hàng. \\
& FR-MD07-04 & Thêm món vào Đơn hàng Giao hàng & Thêm món vào đơn hàng giao đi. \\
& FR-MD07-05 & Xử lý Ghi chú cho Đơn Giao hàng & Thêm ghi chú cho món hoặc cho tài xế. \\
& FR-MD07-06 & Gửi đơn Giao hàng xuống Bếp/Bar & Gửi món của đơn giao hàng xuống bếp/bar. \\
& FR-MD07-07 & (Kích hoạt) Áp dụng Đặt cọc/Thanh toán Trước & Kích hoạt trừ cọc/thanh toán trước cho đơn giao hàng. \\
& FR-MD07-08 & Xác nhận và Gửi Đơn hàng sang Shipday & Gửi thông tin đơn hàng qua Shipday để điều phối giao hàng. \\
& FR-MD07-10 & Xử lý Thanh toán Đơn hàng Giao hàng (Nếu COD) & Ghi nhận tiền COD tài xế nộp lại. \\
& FR-MD07-11 & In Hóa đơn/Phiếu Giao hàng & In phiếu giao hàng/hóa đơn cho tài xế và khách. \\
& FR-MD07-12 & Đóng Đơn hàng Giao hàng & Đóng đơn hàng giao đi sau khi hoàn tất. \\
\hline

% === US-03: Nhân viên lễ tân ===
\multicolumn{4}{|l|}{\textbf{US-03: Nhân viên lễ tân}} \\ \hline
\multirow{5}{=}[2pt]{US-03: Nhân viên lễ tân} & FR-MD03-12 & Xem Danh sách Đặt chỗ & Xem danh sách tổng hợp các lượt đặt chỗ và trạng thái. \\
& FR-MD03-13 & Xem Chi tiết Đặt chỗ & Xem thông tin chi tiết của một lượt đặt chỗ cụ thể. \\
& FR-MD03-14 & Tạo/Sửa Đặt chỗ Thủ công & Tạo hoặc sửa đặt chỗ cho khách qua kênh offline (điện thoại...). \\
& FR-MD03-15 & Quản lý Trạng thái Đặt chỗ & Thay đổi trạng thái đặt chỗ (Xác nhận, Đã đến, Hủy...). \\ \cline{2-4}
& FR-MD05-02 & Truy cập Sơ đồ tầng \& Chọn bàn & Xem trạng thái bàn và xếp khách vào bàn (có thể dùng POS). \\
\hline

% === US-04: Nhân viên bếp ===
\multicolumn{4}{|l|}{\textbf{US-04: Nhân viên bếp}} \\ \hline
\multirow{5}{=}[2pt]{US-04: Nhân viên bếp} & FR-MD03-16 & Xem Danh sách Món đặt trước & Xem các món cần chuẩn bị cho các đặt chỗ sắp tới. \\ \cline{2-4}
& FR-MD08-02 & Xem Đơn hàng trên KDS & Xem các đơn hàng/phiếu chờ xử lý trên màn hình KDS. \\
& FR-MD08-03 & Thay đổi Trạng thái Món ăn/Đơn hàng trên KDS & Đánh dấu món/đơn hàng đang làm, đã xong trên KDS. \\
& FR-MD08-04 & Xem Chi tiết Món ăn trên KDS & Xem chi tiết món ăn, biến thể, ghi chú trên KDS. \\
& FR-MD08-05 & (Tùy chọn) Sắp xếp/Ưu tiên Đơn hàng trên KDS & Sắp xếp hoặc đánh dấu ưu tiên các đơn hàng trên KDS. \\
& FR-MD08-07 & Nhận và Xử lý Phiếu in Bếp & Nhận phiếu in từ máy in và thực hiện chế biến (thủ công). \\
\hline

% === US-05: Nhân viên thu ngân ===
\multicolumn{4}{|l|}{\textbf{US-05: Nhân viên thu ngân}} \\ \hline
\multirow{11}{=}[2pt]{US-05: Nhân viên thu ngân} & FR-MD05-01 & Mở phiên làm việc POS & Bắt đầu phiên làm việc POS, nhập tiền mặt đầu ca. \\
& FR-MD05-11 & Xử lý Thanh toán & Nhận thanh toán từ khách (tiền mặt, thẻ...), xử lý tiền boa. (Có thể là US-02) \\
& FR-MD05-13 & Đóng Phiên làm việc POS & Kết thúc phiên làm việc, tổng kết, đối chiếu tiền mặt. \\ \cline{2-4}
& FR-MD06-01 & Chọn Chế độ Bán Mang về & Chuyển sang giao diện bán mang về trên POS. \\
& FR-MD06-02 & Tạo Đơn hàng Mang về & Khởi tạo đơn hàng mới cho khách mang về. \\
& FR-MD06-03 & (Tùy chọn) Liên kết Khách hàng & Gán đơn hàng mang về với khách hàng (nếu cần). \\
& FR-MD06-04 & Thêm món vào Đơn hàng Mang về & Thêm món vào đơn hàng mang về. \\
& FR-MD06-05 & Xử lý Ghi chú cho Đơn Mang về & Thêm ghi chú cho đơn hàng mang về. \\
& FR-MD06-06 & Gửi đơn Mang về xuống Bếp/Bar & Gửi món của đơn mang về xuống bếp/bar. \\
& FR-MD06-08 & Thanh toán Đơn hàng Mang về & Nhận thanh toán cho đơn hàng mang về tại quầy. \\
& FR-MD06-09 & (Nhận) In Hóa đơn/Phiếu thu Mang về & Nhận hóa đơn in ra để đưa khách. \\
& FR-MD06-10 & Đóng Đơn hàng Mang về & Đóng đơn hàng mang về sau khi thanh toán/giao hàng. \\ \cline{2-4}
& FR-MD07-01 & Chọn Chế độ Giao hàng & Chuyển sang giao diện xử lý đơn giao hàng. \\
& FR-MD07-02 & Tạo/Mở Đơn hàng Giao hàng & Tạo/mở đơn hàng giao đi. \\
& FR-MD07-03 & Liên kết/Nhập Thông tin Khách hàng Giao hàng & Nhập/chọn thông tin khách và địa chỉ giao hàng. \\
& FR-MD07-04 & Thêm món vào Đơn hàng Giao hàng & Thêm món vào đơn hàng giao đi. \\
& FR-MD07-05 & Xử lý Ghi chú cho Đơn Giao hàng & Thêm ghi chú cho món hoặc cho tài xế. \\
& FR-MD07-06 & Gửi đơn Giao hàng xuống Bếp/Bar & Gửi món của đơn giao hàng xuống bếp/bar. \\
& FR-MD07-08 & Xác nhận và Gửi Đơn hàng sang Shipday & Gửi thông tin đơn hàng qua Shipday để điều phối giao hàng. \\
& FR-MD07-10 & Xử lý Thanh toán Đơn hàng Giao hàng (Nếu COD) & Ghi nhận tiền COD tài xế nộp lại. \\
& FR-MD07-11 & In Hóa đơn/Phiếu Giao hàng & In phiếu giao hàng/hóa đơn cho tài xế và khách. \\
& FR-MD07-12 & Đóng Đơn hàng Giao hàng & Đóng đơn hàng giao đi sau khi hoàn tất. \\
\hline

% === US-06: Kế toán ===
\multicolumn{4}{|l|}{\textbf{US-06: Kế toán}} \\ \hline
\multirow{5}{=}[2pt]{US-06: Kế toán} & FR-MD09-03 & Xem Báo cáo Doanh thu Phiên POS & Xem báo cáo tổng kết chi tiết của các phiên POS đã đóng. \\
& FR-MD09-04 & Xem Báo cáo Bán hàng theo Sản phẩm/Danh mục & Xem thống kê số lượng, doanh thu theo từng món ăn/danh mục. \\
& FR-MD09-06 & Xem Báo cáo Tiền đặt cọc & Xem báo cáo tổng hợp tình hình thu, sử dụng, mất cọc. \\
& FR-MD09-07 & Xem Báo cáo Doanh thu theo Loại hình & Phân tích doanh thu theo Eat-in, Takeout, Delivery. \\
& FR-MD09-08 & Xuất dữ liệu Báo cáo & Xuất dữ liệu báo cáo ra file Excel/CSV. \\
\hline

% === US-07: Nhân viên (Chung) ===
\multicolumn{4}{|l|}{\textbf{US-07: Nhân viên (Chung)}} \\ \hline
\multirow{2}{=}[2pt]{US-07: Nhân viên (Chung)} & FR-MD01-06 & Xem lịch cá nhân & Xem lịch làm việc đã được xuất bản của bản thân. \\
& FR-MD01-09 & Đánh dấu không sẵn sàng & Thông báo cho quản lý về thời gian không thể làm việc. \\
\hline

% === US-08: Khách hàng ===
\multicolumn{4}{|l|}{\textbf{US-08: Khách hàng}} \\ \hline
\multirow{9}{=}[2pt]{US-08: Khách hàng} & FR-MD03-01 & Xem Giao diện Đặt chỗ & Truy cập và xem giao diện đặt chỗ online. \\
& FR-MD03-02 & Chọn Thông tin Đặt bàn & Chọn ngày, giờ, số lượng người đặt bàn. \\
& FR-MD03-03 & (Tùy chọn) Chọn Bàn cụ thể & Xem sơ đồ tầng và chọn bàn trống (nếu được phép). \\
& FR-MD03-04 & Xem Thực đơn & Xem thực đơn online để chọn món đặt trước. \\
& FR-MD03-05 & Chọn Món ăn Đặt trước & Thêm món ăn/đồ uống vào giỏ hàng đặt trước. \\
& FR-MD03-06 & Xem Tóm tắt Đặt chỗ & Xem lại thông tin đặt bàn, món ăn, tiền cọc dự kiến. \\
& FR-MD03-07 & Nhập Thông tin Khách hàng & Cung cấp Tên, SĐT, Email. \\
& FR-MD03-09 & Thanh toán Đặt cọc & Thực hiện thanh toán tiền đặt cọc qua cổng thanh toán. \\
& FR-MD03-10 & (Nhận) Xác nhận Đặt chỗ & Nhận email/SMS xác nhận sau khi thanh toán thành công. \\
& FR-MD03-17 & Xem Lịch sử/Chi tiết Đặt chỗ Cá nhân & Xem lại các đặt chỗ đã thực hiện qua tài khoản. \\ \cline{2-4}
& FR-MD04-02 & (Tương tác) Thực hiện Cuộc gọi và Tương tác Khách hàng & Nghe máy và bấm phím 1, 0, 2 khi nhận cuộc gọi từ Bot. \\
\hline

% === US-09: Nhân viên hỗ trợ/ Vận hành ===
\multicolumn{4}{|l|}{\textbf{US-09: Nhân viên hỗ trợ/ Vận hành}} \\ \hline
\multirow{1}{=}[2pt]{US-09: Nhân viên hỗ trợ/ Vận hành} & FR-MD04-03 & (Tiếp nhận) Xử lý Phản hồi Khách hàng từ Bot Call & Nhận cuộc gọi chuyển tiếp từ Bot khi khách bấm phím 2 để hỗ trợ. \\
&&&
\tabularnewline\hline
% === US-10: Quản trị viên Hệ thống ===
\multicolumn{4}{|l|}{\textbf{US-10: Quản trị viên Hệ thống}} \\ \hline
\multirow{7}{=}[2pt]{US-10: Quản trị viên Hệ thống} & FR-MD02-05 & Định nghĩa Thuộc tính \& Giá trị (cho Biến thể) & Định nghĩa các đặc tính (size, độ cay) và lựa chọn (S, M, L). (Có thể là US-01) \\ \cline{2-4}
& FR-MD04-05 & Cấu hình Dịch vụ Bot Call & Cấu hình tham số tích hợp Bot Call. (Có thể là US-01) \\ \cline{2-4}
& FR-MD07-13 & Cấu hình Tích hợp Shipday & Cấu hình tham số kết nối API Shipday. (Có thể là US-01) \\ \cline{2-4}
& FR-MD10-01 & Quản lý Người dùng (Nhân viên) & Tạo, sửa, vô hiệu hóa tài khoản người dùng nhân viên. \\
& FR-MD10-02 & Quản lý Nhóm Quyền & Xem, tạo, sửa, xóa các nhóm quyền truy cập. \\
& FR-MD10-03 & Phân quyền Truy cập cho Người dùng & Gán người dùng vào các nhóm quyền phù hợp. \\
& FR-MD10-04 & Cấu hình Chung của Hệ thống & Cấu hình thông tin công ty, logo, tiền tệ, email... (Có thể là US-01) \\
& FR-MD10-05 & Cấu hình Tích hợp Bên thứ ba & Quản lý API keys cho Cổng thanh toán, Bot Call, Shipday. (Có thể là US-01) \\
& FR-MD10-06 & Cấu hình Tham số Nghiệp vụ Đặc thù & Cấu hình tỷ lệ cọc, giá bàn, số ngày gọi bot... (Có thể là US-01) \\
& FR-MD10-07 & Xem Nhật ký Hệ thống (Logs) & Xem log hệ thống để theo dõi và khắc phục sự cố. \\
\hline
\caption{Phân công Chức năng theo Người dùng} \label{tab:user_function_map} \\

\end{longtable}

\newpage


\begin{longtable}{|m{1.5cm}|m{4.5cm}|m{9cm}|}
\hline
\textbf{Mã} & \textbf{Tên Module} & \textbf{Mô Tả} \\
\hline
\endhead % Header cho các trang tiếp theo

\hline
\endfoot % Footer cho bảng

\hline
\endlastfoot % Footer cho trang cuối cùng

MD-01 & Quản lý Lịch làm việc (Scheduling) & Bao gồm các chức năng: tạo ca làm việc, gán nhân viên theo vai trò, hiển thị dạng Gantt, kiểm tra trùng lịch, xuất bản và thông báo lịch trình cho nhân viên. \\
\hline
MD-02 & Quản lý Thực đơn \& Sản phẩm (Menu \& Product) & Quản lý danh sách món ăn, đồ uống dưới dạng sản phẩm. Bao gồm tạo mới, chỉnh sửa giá, mô tả, hình ảnh, phân loại theo danh mục POS, quản lý biến thể (ví dụ: size S/M/L), quản lý tồn kho cho các mặt hàng cụ thể (vd: rượu chai). \\
\hline
MD-03 & Quản lý Đặt chỗ \& Đặt món trước (Booking \& Pre-order) & Cung cấp giao diện cho khách hàng (web/app) để chọn thời gian, số lượng người, chọn bàn (nếu có), và tùy chọn đặt trước các món ăn từ thực đơn. Tính toán và xử lý thanh toán tiền đặt cọc (15\% giá trị bàn + 15\% giá trị món ăn đặt trước). Quản lý trạng thái các lượt đặt chỗ. \\
\hline
MD-04 & Xác nhận Tự động qua Bot (Automated Bot Confirmation) & Tự động kích hoạt cuộc gọi thoại tới khách hàng trước N ngày (cấu hình được) so với ngày đặt chỗ. Phát thông báo và xử lý lựa chọn của khách (1: Xác nhận, 0: Hủy - cập nhật trạng thái đặt chỗ và ghi nhận mất cọc, 2: Chuyển hướng cuộc gọi đến Nhân viên hỗ trợ). \\
\hline
MD-05 & Quản lý Bán hàng Tại chỗ (POS - Eat-in) & Giao diện POS cho nhân viên phục vụ, quản lý sơ đồ tầng, trạng thái bàn, nhận đơn hàng tại bàn (bao gồm cả món khách đã đặt trước), xử lý yêu cầu đặc biệt (ghi chú bếp), tách/gộp hóa đơn, áp dụng tiền đặt cọc đã thanh toán vào hóa đơn cuối cùng, xử lý thanh toán đa phương thức, quản lý tiền boa. \\
\hline
MD-06 & Quản lý Bán mang về (POS - Takeout) & Cung cấp một giao diện/luồng riêng biệt trên POS (ví dụ: nút "Takeout") để tạo đơn hàng mang về. Không yêu cầu chọn bàn. Có thể liên kết với khách hàng nếu họ đã đăng nhập hoặc cung cấp thông tin. Áp dụng tiền đặt cọc nếu đơn hàng được đặt trước qua kênh online. Xử lý thanh toán. \\
\hline
MD-07 & Quản lý Giao hàng (POS - Delivery) & Cung cấp một giao diện/luồng riêng biệt trên POS (ví dụ: nút "Delivery"). Yêu cầu thông tin khách hàng (đã đăng nhập từ web/app đặt hàng). Tích hợp với Shipday: gửi thông tin đơn hàng (địa chỉ, chi tiết món, thông tin khách) sang Shipday để điều phối tài xế, nhận lại cập nhật trạng thái giao hàng từ Shipday. Áp dụng tiền đặt cọc đã thanh toán. Xử lý thanh toán (có thể là online trước hoặc COD tùy cấu hình). \\
\hline
MD-08 & Tích hợp Bếp (Kitchen Integration) & Truyền thông tin đơn hàng (từ Eat-in, Takeout, Delivery - bao gồm món đặt trước và ghi chú) tới khu vực bếp thông qua Màn hình hiển thị Bếp (KDS) hoặc máy in phiếu bếp. Cho phép nhân viên bếp cập nhật trạng thái món ăn (đang làm, đã xong). \\
\hline
MD-09 & Quản lý Phiên \& Báo cáo (Session \& Reporting) & Quản lý việc mở và đóng phiên làm việc trên POS, đối soát tiền mặt. Cung cấp các báo cáo về doanh thu (theo loại hình: Eat-in, Takeout, Delivery), sản phẩm bán chạy, hiệu suất nhân viên, tiền đặt cọc, trạng thái đơn hàng giao, tích hợp dữ liệu. \\
\hline
MD-10 & Quản lý Hệ thống \& Người dùng (System \& User) & Quản lý tài khoản người dùng (tạo, sửa, xóa), phân quyền truy cập chi tiết cho từng vai trò vào các module và chức năng. Quản lý các cấu hình chung như tỷ lệ đặt cọc, giá trị bàn, thời gian gọi bot, cấu hình tích hợp bên thứ ba (API keys,...). \\
\hline
\caption{Phân chia Module Hệ thống} \label{tab:modules} \\

\end{longtable}

\subsubsection{Module MD-01: Quản lý Lịch làm việc (Scheduling)}


\begin{longtable}{|m{2cm}|m{2.5cm}|m{2cm}|m{4cm}|m{4.5cm}|}
\caption{Danh sách Yêu cầu Chức năng cho Module MD-01: Quản lý Lịch làm việc} \label{tab:fr_md01} \\
\hline
\textbf{Mã Module} & \textbf{Mã Yêu cầu CN} & \textbf{Mã Người dùng} & \textbf{Tên Chức năng} & \textbf{Mô tả Ngắn} \\
\hline
\endhead % Header cho các trang tiếp theo

\hline
\endfoot % Footer cho bảng

\hline
\endlastfoot % Footer cho trang cuối cùng

MD-01 & FR-MD01-01 & US-01 & Tạo ca làm việc & Cho phép Quản lý nhà hàng định nghĩa các ca làm việc mới (thời gian bắt đầu, kết thúc, ngày, vai trò cần thiết, số lượng nhân viên cần cho vai trò đó). \\
\hline
MD-01 & FR-MD01-02 & US-01 & Gán nhân viên vào ca & Cho phép Quản lý nhà hàng chỉ định nhân viên cụ thể cho các vị trí trống trong ca làm việc đã tạo, dựa trên vai trò phù hợp của nhân viên. \\
\hline
MD-01 & FR-MD01-03 & US-01 & Xem lịch biểu Gantt & Hiển thị lịch làm việc của tất cả nhân viên hoặc lọc theo vai trò dưới dạng biểu đồ Gantt trực quan theo dòng thời gian (ngày, tuần, tháng). \\
\hline
MD-01 & FR-MD01-04 & US-01 (Trigger), System (Action) & Phát hiện trùng lịch & Hệ thống tự động kiểm tra và đưa ra cảnh báo trực quan (ví dụ: đổi màu ca bị trùng) nếu một nhân viên được gán vào hai ca làm việc có thời gian trùng nhau. \\
\hline
MD-01 & FR-MD01-05 & US-01 & Xuất bản và Thông báo lịch & Cho phép Quản lý nhà hàng công khai (publish) lịch làm việc đã xếp và kích hoạt hệ thống tự động gửi thông báo (ví dụ: email, thông báo trong ứng dụng) đến từng nhân viên về lịch trình cá nhân của họ. \\
\hline
MD-01 & FR-MD01-06 & US-07 & Xem lịch cá nhân & Cho phép Nhân viên xem lịch làm việc đã được xuất bản của riêng mình thông qua cổng thông tin nhân viên hoặc ứng dụng di động. \\
\hline
MD-01 & FR-MD01-07 & US-01 & Sao chép lịch tuần & Cho phép Quản lý nhà hàng sao chép nhanh lịch làm việc của một tuần (hoặc khoảng thời gian tùy chọn) sang tuần kế tiếp để tiết kiệm thời gian lập lịch. \\
\hline
MD-01 & FR-MD01-08 & US-01 & Quản lý vai trò công việc & Cho phép Quản lý nhà hàng định nghĩa các vai trò công việc trong nhà hàng (ví dụ: Bếp trưởng, Phục vụ, Pha chế, Lễ tân) để sử dụng khi tạo ca và gán nhân viên. \\
\hline
MD-01 & FR-MD01-09 & US-07 & Đánh dấu không sẵn sàng & Cho phép Nhân viên (nếu được cấu hình) đánh dấu các khoảng thời gian không sẵn sàng làm việc (ví dụ: nghỉ phép, bận việc riêng) để Quản lý nhà hàng xem xét khi xếp lịch. \\ % Thêm mới dựa trên case study
\hline
MD-01 & FR-MD01-10 & US-01 & Xem lịch theo vai trò & Cho phép Quản lý nhà hàng lọc và xem lịch làm việc chỉ của các nhân viên thuộc một vai trò cụ thể (ví dụ: xem tất cả ca của Bếp trưởng). \\ % Làm rõ FR-MD01-03
\hline

\end{longtable}


\begin{figure}[H]
	\centering
	\includegraphics[width=15cm]{Sections/tong_quan/functional_spec/img/ucd01.png}

     \vspace{0.5cm}
    \caption{Use case diagram cho Module MD-01: Quản lý Lịch làm việc (Scheduling)}
\end{figure}

\subsubsubsection{FR-MD01-01: Tạo ca làm việc mới}

\begin{figure}[H]
	\centering
	\includegraphics[width=15cm]{Sections/tong_quan/functional_spec/img/Screenshot 2025-04-30 at 20.11.53.png}

     \vspace{0.5cm}
    \caption{Quy trình Tạo ca làm việc mới (UC-MD-1-01)}
\end{figure}

\subsubsubsection{FR-MD01-02: Gán nhân viên vào ca làm việc}
\begin{figure}[H]
	\centering
	\includegraphics[width=15cm]{Sections/tong_quan/functional_spec/img/1.2.png}

     \vspace{0.5cm}
    \caption{Quy trình Gán nhân viên vào ca làm việc (UC-MD-1-02)}
\end{figure}


\subsubsubsection{FR-MD01-03: Xem lịch biểu Gantt}

\begin{figure}[H]
	\centering
	\includegraphics[width=15cm]{Sections/tong_quan/functional_spec/img/1.3.png}

     \vspace{0.5cm}
    \caption{Quy trình Xem lịch biểu Gantt (UC-MD-1-03)}
\end{figure}

\subsubsubsection{FR-MD01-05: Phát hiện và Cảnh báo Trùng lịch}

\begin{figure}[H]
	\centering
	\includegraphics[width=15cm]{Sections/tong_quan/functional_spec/img/1.4.png}

     \vspace{0.5cm}
    \caption{Quy trình  Phát hiện và Cảnh báo Trùng lịch (UC-MD-1-04)}
\end{figure}

\subsubsubsection{FR-MD01-05: Xuất bản và Thông báo Lịch làm việc}


\subsubsubsection{FR-MD01-06: Xem lịch làm việc cá nhân}

\begin{figure}[H]
	\centering
	\includegraphics[width=15cm]{Sections/tong_quan/functional_spec/img/1.6.png}

     \vspace{0.5cm}
    \caption{Quy trình Xem lịch làm việc cá nhân (UC-MD-1-06)}
\end{figure}

\subsubsubsection{FR-MD01-07: Sao chép lịch tuần}

\begin{figure}[H]
	\centering
	\includegraphics[width=15cm]{Sections/tong_quan/functional_spec/img/1.7.png}

     \vspace{0.5cm}
    \caption{Quy trình Sao chép lịch tuần (UC-MD-1-07)}
\end{figure}

\subsubsubsection{FR-MD01-08: Quản lý vai trò công việc}

\begin{figure}[H]
	\centering
	\includegraphics[width=15cm]{Sections/tong_quan/functional_spec/img/1.8.1.png}

     \vspace{0.5cm}
    \caption{Quy trình Tạo mới Vai trò công việc (UC-MD-1-08 - Create)}
\end{figure}
\begin{figure}[H]
	\centering
	\includegraphics[width=15cm]{Sections/tong_quan/functional_spec/img/1.8.2.png}

     \vspace{0.5cm}
    \caption{Quy trình Xóa Vai trò công việc (UC-MD-1-08 - Delete)}
\end{figure}
\begin{figure}[H]
	\centering
	\includegraphics[width=15cm]{Sections/tong_quan/functional_spec/img/1.8.3.png}

     \vspace{0.5cm}
    \caption{Quy trình Sửa Vai trò công việc (UC-MD-1-08 - Edit)}
\end{figure}

\subsubsubsection{FR-MD01-09: Đánh dấu không sẵn sàng làm việc}

\begin{figure}[H]
	\centering
	\includegraphics[width=15cm]{Sections/tong_quan/functional_spec/img/1.9.png}

     \vspace{0.5cm}
    \caption{Quy trình Đánh dấu không sẵn sàng làm việc (UC-MD-1-09)}
\end{figure}

\subsubsubsection{Use Case UC-MD01-10: Xem lịch theo vai trò}

\subsubsubsection{MVP (Minimum viable product) và Screen Flow}

\begin{figure}[H]
	\centering
	\includegraphics[width=15cm]{Sections/tong_quan/functional_spec/img/proto1.1.png}

     \vspace{0.5cm}
    \caption{Trang Dashboard (1)}
\end{figure}
\begin{figure}[H]
	\centering
	\includegraphics[width=15cm]{Sections/tong_quan/functional_spec/img/proto1.2.png}

     \vspace{0.5cm}
    \caption{Trang Dashboard (2)}
\end{figure}
\begin{figure}[H]
	\centering
	\includegraphics[width=15cm]{Sections/tong_quan/functional_spec/img/proto1.3.png}

     \vspace{0.5cm}
    \caption{Trang Dashboard (3)}
\end{figure}
\begin{figure}[H]
	\centering
	\includegraphics[width=15cm]{Sections/tong_quan/functional_spec/img/proto1.4.png}

     \vspace{0.5cm}
    \caption{Trang Dashboard (4)}
\end{figure}
\begin{figure}[H]
	\centering
	\includegraphics[width=15cm]{Sections/tong_quan/functional_spec/img/proto1.5.png}

     \vspace{0.5cm}
    \caption{Trang Dashboard (5)}
\end{figure}
\begin{figure}[H]
	\centering
	\includegraphics[width=15cm]{Sections/tong_quan/functional_spec/img/proto1.6.png}

     \vspace{0.5cm}
    \caption{Trang Staff Management}
\end{figure}
\begin{figure}[H]
	\centering
	\includegraphics[width=15cm]{Sections/tong_quan/functional_spec/img/proto1.7.png}

     \vspace{0.5cm}
    \caption{Trang Shift Management}
\end{figure}
\begin{figure}[H]
	\centering
	\includegraphics[width=15cm]{Sections/tong_quan/functional_spec/img/proto1.8.png}

     \vspace{0.5cm}
    \caption{Trang Settings (1)}
\end{figure}
\begin{figure}[H]
	\centering
	\includegraphics[width=15cm]{Sections/tong_quan/functional_spec/img/proto1.9.png}

     \vspace{0.5cm}
    \caption{Trang Settings (2)}
\end{figure}
\begin{figure}[H]
	\centering
	\includegraphics[width=15cm]{Sections/tong_quan/functional_spec/img/proto1.10.png}

     \vspace{0.5cm}
    \caption{Trang Settings (3)}
\end{figure}

\textbf{User Flow:}

\begin{figure}[H]
	\centering
	\includegraphics[width=15cm]{Sections/tong_quan/functional_spec/img/dashboard1.png}

     \vspace{0.5cm}
    \caption{User flow cho trang Dashboard}
\end{figure}
\begin{figure}[H]
	\centering
	\includegraphics[width=15cm]{Sections/tong_quan/functional_spec/img/staffmanage1.png}

     \vspace{0.5cm}
    \caption{User flow cho trang Staff Management}
\end{figure}
\begin{figure}[H]
	\centering
	\includegraphics[width=15cm]{Sections/tong_quan/functional_spec/img/shiftmanage1.png}

     \vspace{0.5cm}
    \caption{User flow cho trang Staff Management}
\end{figure}
\begin{figure}[H]
	\centering
	\includegraphics[width=15cm]{Sections/tong_quan/functional_spec/img/setting1.png}

     \vspace{0.5cm}
    \caption{User flow cho trang Settings}
\end{figure}
\subsubsection{Module MD-02: Quản lý Thực đơn \& Sản phẩm}
\begin{longtable}{|m{2cm}|m{2.5cm}|m{2cm}|m{4cm}|m{4.5cm}|}
\caption{Danh sách Yêu cầu Chức năng cho Module MD-02: Quản lý Thực đơn \& Sản phẩm} \label{tab:fr_md02} \\
\hline
\textbf{Mã Module} & \textbf{Mã Yêu cầu CN} & \textbf{Mã Người dùng} & \textbf{Tên Chức năng} & \textbf{Mô tả Ngắn} \\
\hline
\endhead % Header cho các trang tiếp theo

\hline
\endfoot % Footer cho bảng

\hline
\endlastfoot % Footer cho trang cuối cùng

MD-02 & FR-MD02-01 & US-01 & Tạo Sản phẩm Mới (Món ăn/Đồ uống) & Cho phép Quản lý nhà hàng thêm một món ăn, đồ uống, hoặc dịch vụ mới vào hệ thống với các thông tin cơ bản (tên, giá, loại). \\
\hline
MD-02 & FR-MD02-02 & US-01 & Chỉnh sửa Thông tin Sản phẩm & Cho phép Quản lý nhà hàng cập nhật các chi tiết của một sản phẩm đã tồn tại (ví dụ: thay đổi giá bán, mô tả, cập nhật hình ảnh, gán lại danh mục). \\
\hline
MD-02 & FR-MD02-03 & US-01 & Lưu trữ/Hủy kích hoạt Sản phẩm & Cho phép Quản lý nhà hàng ẩn một sản phẩm khỏi các giao dịch (ví dụ: POS, đặt hàng online) mà không xóa hẳn dữ liệu lịch sử. \\
\hline
MD-02 & FR-MD02-04 & US-01 & Quản lý Danh mục Sản phẩm POS & Cho phép Quản lý nhà hàng tạo, sửa, xóa và sắp xếp thứ tự các danh mục được sử dụng để nhóm sản phẩm trên giao diện POS (ví dụ: Khai vị, Món chính, Tráng miệng, Đồ uống). \\
\hline
MD-02 & FR-MD02-05 & US-01 / US-10 & Định nghĩa Thuộc tính \& Giá trị (cho Biến thể) & Cho phép người dùng định nghĩa các thuộc tính (ví dụ: Kích cỡ, Độ cay, Loại topping) và các giá trị tương ứng cho từng thuộc tính (ví dụ: S, M, L; Ít cay, Cay vừa, Cay nhiều; Phô mai, Thịt nguội). \\
\hline
MD-02 & FR-MD02-06 & US-01 & Cấu hình Biến thể Sản phẩm & Cho phép Quản lý nhà hàng áp dụng các thuộc tính đã định nghĩa vào một sản phẩm gốc để tạo ra các biến thể (ví dụ: Cà phê size S, Cà phê size L), quản lý giá và mã SKU riêng cho từng biến thể nếu cần. \\
\hline
MD-02 & FR-MD02-07 & US-01 & Thiết lập Loại Sản phẩm & Cho phép Quản lý nhà hàng xác định loại sản phẩm (Consumable, Stockable, Service) để quyết định cách hệ thống quản lý tồn kho (nếu là Stockable) hoặc không theo dõi tồn kho. \\
\hline
MD-02 & FR-MD02-08 & US-01 & Cấu hình Hiển thị trên POS & Cho phép Quản lý nhà hàng đánh dấu sản phẩm có sẵn sàng để bán trên POS hay không và gán sản phẩm vào (các) danh mục POS phù hợp. \\
\hline
MD-02 & FR-MD02-09 & US-01 & Quản lý Hình ảnh Sản phẩm & Cho phép Quản lý nhà hàng tải lên và quản lý hình ảnh đại diện cho sản phẩm, hiển thị trên POS hoặc các kênh bán hàng khác. \\
\hline
MD-02 & FR-MD02-10 & US-01 & Cấu hình In Bếp/Hiển thị KDS & Cho phép Quản lý nhà hàng chỉ định danh mục sản phẩm nào sẽ được gửi đến máy in bếp hoặc màn hình KDS cụ thể khi đặt hàng qua POS. (Có thể liên quan cấu hình POS/IoT). \\
\hline
MD-02 & FR-MD02-11 & US-01 & Định nghĩa Sản phẩm Tùy chọn/Phụ thu & Cho phép Quản lý nhà hàng tạo các sản phẩm nhỏ (ví dụ: Extra cheese, Thêm sốt) để dùng làm tùy chọn có tính phí khi khách hàng yêu cầu thêm vào món chính (sử dụng trong cấu hình POS modifiers). \\
\hline

\end{longtable}

\subsubsubsection{FR-MD01-01: Tạo ca làm việc mới}

\begin{figure}[H]
	\centering
	\includegraphics[width=15cm]{Sections/tong_quan/functional_spec/img/2.1.png}

     \vspace{0.5cm}
    \caption{Quy trình Tạo ca làm việc mới (UC-MD-1-01)}
\end{figure}

\subsubsubsection{FR-MD01-01: Tạo ca làm việc mới}

\begin{figure}[H]
	\centering
	\includegraphics[width=15cm]{Sections/tong_quan/functional_spec/img/2.2.png}

     \vspace{0.5cm}
    \caption{Quy trình Tạo ca làm việc mới (UC-MD-1-01)}
\end{figure}
\subsubsubsection{FR-MD01-01: Tạo ca làm việc mới}

\begin{figure}[H]
	\centering
	\includegraphics[width=15cm]{Sections/tong_quan/functional_spec/img/2.3.1.png}

     \vspace{0.5cm}
    \caption{Quy trình Tạo ca làm việc mới (UC-MD-1-01)}
\end{figure}

\begin{figure}[H]
	\centering
	\includegraphics[width=15cm]{Sections/tong_quan/functional_spec/img/2.3.2.png}

     \vspace{0.5cm}
    \caption{Quy trình Tạo ca làm việc mới (UC-MD-1-01)}
\end{figure}
\subsubsubsection{FR-MD01-01: Tạo ca làm việc mới}

\begin{figure}[H]
	\centering
	\includegraphics[width=15cm]{Sections/tong_quan/functional_spec/img/2.4.1.png}

     \vspace{0.5cm}
    \caption{Quy trình Tạo ca làm việc mới (UC-MD-1-01)}
\end{figure}

\begin{figure}[H]
	\centering
	\includegraphics[width=15cm]{Sections/tong_quan/functional_spec/img/2.4.png}

     \vspace{0.5cm}
    \caption{Quy trình Tạo ca làm việc mới (UC-MD-1-01)}
\end{figure}
\subsubsubsection{FR-MD01-01: Tạo ca làm việc mới}

\begin{figure}[H]
	\centering
	\includegraphics[width=15cm]{Sections/tong_quan/functional_spec/img/2.5.png}

     \vspace{0.5cm}
    \caption{Quy trình Tạo ca làm việc mới (UC-MD-1-01)}
\end{figure}
\begin{figure}[H]
	\centering
	\includegraphics[width=15cm]{Sections/tong_quan/functional_spec/img/2.5.1.png}

     \vspace{0.5cm}
    \caption{Quy trình Tạo ca làm việc mới (UC-MD-1-01)}
\end{figure}
\begin{figure}[H]
	\centering
	\includegraphics[width=15cm]{Sections/tong_quan/functional_spec/img/2.5.2.png}

     \vspace{0.5cm}
    \caption{Quy trình Tạo ca làm việc mới (UC-MD-1-01)}
\end{figure}
\subsubsubsection{FR-MD01-01: Tạo ca làm việc mới}

\begin{figure}[H]
	\centering
	\includegraphics[width=15cm]{Sections/tong_quan/functional_spec/img/2.6.png}

     \vspace{0.5cm}
    \caption{Quy trình Tạo ca làm việc mới (UC-MD-1-01)}
\end{figure}
\subsubsubsection{FR-MD01-01: Tạo ca làm việc mới}


\subsubsubsection{FR-MD01-01: Tạo ca làm việc mới}

\begin{figure}[H]
	\centering
	\includegraphics[width=15cm]{Sections/tong_quan/functional_spec/img/2.8.png}

     \vspace{0.5cm}
    \caption{Quy trình Tạo ca làm việc mới (UC-MD-1-01)}
\end{figure}
\subsubsubsection{FR-MD01-01: Tạo ca làm việc mới}

\begin{figure}[H]
	\centering
	\includegraphics[width=15cm]{Sections/tong_quan/functional_spec/img/2.9.png}

     \vspace{0.5cm}
    \caption{Quy trình Tạo ca làm việc mới (UC-MD-1-01)}
\end{figure}
\subsubsubsection{FR-MD01-01: Tạo ca làm việc mới}

\begin{figure}[H]
	\centering
	\includegraphics[width=15cm]{Sections/tong_quan/functional_spec/img/2.10.png}

     \vspace{0.5cm}
    \caption{Quy trình Tạo ca làm việc mới (UC-MD-1-01)}
\end{figure}

\subsubsubsection{FR-MD01-01: Tạo ca làm việc mới}



\subsubsection{Module MD-03: Quản lý Đặt chỗ \& Đặt món trước}

\begin{longtable}{|m{2cm}|m{2.5cm}|m{2cm}|m{4.5cm}|m{4cm}|}
\caption{Danh sách Yêu cầu Chức năng cho Module MD-03: Quản lý Đặt chỗ \& Đặt món trước} \label{tab:fr_md03} \\
\hline
\textbf{Mã Module} & \textbf{Mã Yêu cầu CN} & \textbf{Mã Người dùng} & \textbf{Tên Chức năng} & \textbf{Mô tả Ngắn} \\
\hline
\endhead % Header cho các trang tiếp theo

\hline
\endfoot % Footer cho bảng

\hline
\endlastfoot % Footer cho trang cuối cùng

MD-03 & FR-MD03-01 & US-08 & Xem Giao diện Đặt chỗ & Hiển thị giao diện cho phép khách hàng xem các khung giờ/bàn còn trống và bắt đầu quá trình đặt chỗ. \\
\hline
MD-03 & FR-MD03-02 & US-08 & Chọn Thông tin Đặt bàn & Cho phép khách hàng chọn ngày, giờ, số lượng người cho lượt đặt bàn mong muốn. \\
\hline
MD-03 & FR-MD03-03 & US-08 & Chọn Bàn cụ thể & Nếu được cấu hình, cho phép khách hàng xem sơ đồ bàn và chọn một bàn cụ thể còn trống phù hợp với số lượng người. \\
\hline
MD-03 & FR-MD03-04 & US-08 & Xem Thực đơn & Hiển thị thực đơn (các sản phẩm từ MD-02) để khách hàng lựa chọn món ăn muốn đặt trước. \\
\hline
MD-03 & FR-MD03-05 & US-08 & Chọn Món ăn Đặt trước & Cho phép khách hàng thêm các món ăn/đồ uống từ thực đơn vào giỏ hàng đặt trước của lượt đặt chỗ. Bao gồm chọn biến thể (nếu có). \\
\hline
MD-03 & FR-MD03-06 & US-08 & Xem Tóm tắt Đặt chỗ & Hiển thị thông tin tổng hợp về lượt đặt chỗ: ngày giờ, số người, bàn (nếu chọn), danh sách món đặt trước, và số tiền đặt cọc dự kiến. \\
\hline
MD-03 & FR-MD03-07 & US-08 & Nhập Thông tin Khách hàng & Thu thập thông tin bắt buộc của khách hàng (Tên, Số điện thoại, Email) để hoàn tất đặt chỗ. \\
\hline
MD-03 & FR-MD03-08 & System / US-08 & Tính toán Tiền Đặt cọc & Hệ thống tự động tính toán số tiền đặt cọc dựa trên cấu hình: 15\% giá bàn + 15\% tổng giá trị món ăn đặt trước. \\
\hline
MD-03 & FR-MD03-09 & US-08 & Thanh toán Đặt cọc & Tích hợp với cổng thanh toán để khách hàng thực hiện thanh toán số tiền đặt cọc đã được tính toán. \\
\hline
MD-03 & FR-MD03-10 & System / US-08 & Xác nhận Đặt chỗ & Sau khi thanh toán thành công, hệ thống tạo bản ghi đặt chỗ, gửi email/SMS xác nhận cho khách hàng và cập nhật trạng thái bàn/lịch. \\
\hline
MD-03 & FR-MD03-11 & US-01 / US-10 & Cấu hình Tham số Đặt chỗ & Cho phép quản lý cấu hình các tùy chọn đặt chỗ: giờ hoạt động, thời gian mỗi lượt đặt, số lượng khách tối thiểu/tối đa, đặt trước bao lâu, giá trị từng bàn (cho việc tính cọc). \\
\hline
MD-03 & FR-MD03-12 & US-01 / US-03 & Xem Danh sách Đặt chỗ & Hiển thị danh sách tất cả các lượt đặt chỗ (online và offline) với các thông tin cơ bản và trạng thái (Chờ xác nhận, Đã xác nhận, Đã hủy...). \\
\hline
MD-03 & FR-MD03-13 & US-01 / US-03 & Xem Chi tiết Đặt chỗ & Cho phép xem thông tin chi tiết của một lượt đặt chỗ cụ thể: thông tin khách, bàn, món đặt trước, trạng thái thanh toán cọc, lịch sử thay đổi. \\
\hline
MD-03 & FR-MD03-14 & US-01 / US-03 & Tạo/Sửa Đặt chỗ Thủ công & Cho phép nhân viên (Lễ tân, Quản lý) tạo hoặc chỉnh sửa một lượt đặt chỗ trong hệ thống (ví dụ: cho khách gọi điện thoại), bao gồm cả việc nhập món đặt trước và ghi nhận đặt cọc (nếu có). \\
\hline
MD-03 & FR-MD03-15 & US-01 / US-03 & Quản lý Trạng thái Đặt chỗ & Cho phép nhân viên thay đổi trạng thái của một lượt đặt chỗ (ví dụ: Xác nhận thủ công, Đánh dấu đã đến, Hủy bỏ). \\
\hline
MD-03 & FR-MD03-16 & US-04 / US-01 & Xem Danh sách Món đặt trước & Cung cấp giao diện (có thể là báo cáo hoặc màn hình riêng) cho bộ phận bếp/quản lý xem trước các món ăn cần chuẩn bị cho các lượt đặt chỗ sắp tới. \\
\hline
MD-03 & FR-MD03-17 & US-08 & Xem Lịch sử/Chi tiết Đặt chỗ Cá nhân & Cho phép khách hàng đã đăng nhập xem lại các lượt đặt chỗ đã thực hiện và trạng thái của chúng. \\
\hline


\end{longtable}


\subsubsection{Module MD-04: Xác nhận Tự động qua Bot}

\begin{longtable}{|m{2cm}|m{2.5cm}|m{2.5cm}|m{4.5cm}|m{3.5cm}|}
\caption{Danh sách Yêu cầu Chức năng cho Module MD-04: Xác nhận Tự động qua Bot} \label{tab:fr_md04} \\
\hline
\textbf{Mã Module} & \textbf{Mã Yêu cầu CN} & \textbf{Mã Người dùng} & \textbf{Tên Chức năng} & \textbf{Mô tả Ngắn} \\
\hline
\endhead % Header cho các trang tiếp theo

\hline
\endfoot % Footer cho bảng

\hline
\endlastfoot % Footer cho trang cuối cùng

MD-04 & FR-MD04-01 & System & Lên lịch và Kích hoạt Cuộc gọi Xác nhận & Tự động xác định các đặt chỗ 'Đã xác nhận' sắp diễn ra và lên lịch kích hoạt cuộc gọi xác nhận N ngày trước ngày đặt (N cấu hình được). \\
\hline
MD-04 & FR-MD04-02 & System (Bot Service), US-08 (Tương tác) & Thực hiện Cuộc gọi và Tương tác Khách hàng & Tích hợp với dịch vụ Bot Call bên ngoài để thực hiện cuộc gọi đến SĐT khách hàng, phát thông điệp và nhận lựa chọn (phím 1, 0, 2). \\
\hline
MD-04 & FR-MD04-03 & System, US-09 (Tiếp nhận cuộc gọi hỗ trợ) & Xử lý Phản hồi Khách hàng từ Bot Call & Cập nhật trạng thái đặt chỗ và thực hiện hành động tương ứng (xác nhận lại, hủy bỏ & giải phóng bàn, chuyển cuộc gọi hỗ trợ) dựa trên phím khách hàng đã bấm. \\
\hline
MD-04 & FR-MD04-04 & System & Ghi nhận Kết quả Cuộc gọi & Lưu trữ lại kết quả của mỗi cuộc gọi Bot Call (thành công, thất bại, không liên lạc được, lựa chọn của khách) vào thông tin đặt chỗ hoặc nhật ký hệ thống. \\
\hline
MD-04 & FR-MD04-05 & US-01 / US-10 & Cấu hình Dịch vụ Bot Call & Cho phép cấu hình các tham số tích hợp Bot Call: số ngày N gọi trước, nội dung kịch bản thoại, số điện thoại chuyển tiếp hỗ trợ, API key/credentials của dịch vụ Bot Call. \\
\hline

\end{longtable}


\subsubsection{Module MD-05: Quản lý Bán hàng Tại chỗ (POS - Eat-in)}


\begin{longtable}{|m{2cm}|m{2.5cm}|m{2.5cm}|m{4.5cm}|m{4cm}|}
\caption{Danh sách Yêu cầu Chức năng cho Module MD-05: Quản lý Bán hàng Tại chỗ (POS - Eat-in)} \label{tab:fr_md05} \\
\hline
\textbf{Mã Module} & \textbf{Mã Yêu cầu CN} & \textbf{Mã Người dùng} & \textbf{Tên Chức năng} & \textbf{Mô tả Ngắn} \\
\hline
\endhead % Header cho các trang tiếp theo

\hline
\endfoot % Footer cho bảng

\hline
\endlastfoot % Footer cho trang cuối cùng

MD-05 & FR-MD05-01 & US-05, US-01 & Mở phiên làm việc POS & Cho phép nhân viên thu ngân/quản lý bắt đầu một phiên làm việc mới trên POS, nhập số tiền mặt ban đầu (nếu có kiểm soát tiền mặt). \\
\hline
MD-05 & FR-MD05-02 & US-02, US-03 & Truy cập Sơ đồ tầng \& Chọn bàn & Hiển thị sơ đồ tầng trực quan, cho phép nhân viên xem trạng thái bàn (trống, đang có khách, đã đặt trước) và chọn một bàn cụ thể. \\
\hline
MD-05 & FR-MD05-03 & US-02 & Bắt đầu/Mở đơn hàng tại bàn & Khởi tạo một đơn hàng mới hoặc mở lại đơn hàng đang hoạt động khi nhân viên chọn một bàn đang có khách hoặc xếp khách mới vào bàn trống. \\
\hline
MD-05 & FR-MD05-04 & US-02, System & Tải và Xác nhận Món ăn Đặt trước & Nếu bàn có liên kết với một đặt chỗ (từ MD-03) có món đặt trước, hệ thống tự động tải các món này vào đơn hàng POS để nhân viên xác nhận với khách và gửi bếp. \\
\hline
MD-05 & FR-MD05-05 & US-02 & Thêm món ăn/đồ uống vào đơn hàng & Cho phép nhân viên chọn các món ăn/đồ uống từ giao diện menu POS (theo danh mục - FR-MD02-04) và thêm vào đơn hàng hiện tại của bàn. \\
\hline
MD-05 & FR-MD05-06 & US-02 & Xử lý Yêu cầu đặc biệt/Ghi chú bếp & Cho phép nhân viên đính kèm ghi chú tùy chỉnh hoặc chọn các tùy chọn/ghi chú được định sẵn (liên quan FR-MD02-11) cho từng món ăn hoặc toàn bộ đơn hàng để gửi xuống bếp. \\
\hline
MD-05 & FR-MD05-07 & US-02 & Gửi đơn hàng xuống Bếp/Bar & Gửi thông tin các món ăn mới thêm (hoặc các món đặt trước đã xác nhận) đến máy in bếp/bar hoặc màn hình KDS tương ứng (theo cấu hình FR-MD02-10). \\
\hline
MD-05 & FR-MD05-08 & US-02 & Yêu cầu/In Hóa đơn Tạm tính & Cho phép nhân viên tạo và in ra hóa đơn tạm tính (pro-forma invoice / bill) cho khách hàng kiểm tra trước khi thanh toán. \\
\hline
MD-05 & FR-MD05-09 & US-02, System & Áp dụng Tiền Đặt cọc vào Hóa đơn & Trước khi tính tiền thanh toán cuối cùng, hệ thống tự động xác định và áp dụng (trừ đi) số tiền đặt cọc mà khách hàng đã thanh toán trước đó (từ MD-03) vào tổng hóa đơn. \\
\hline
MD-05 & FR-MD05-10 & US-02 & Tách hóa đơn (Split Bill) & Cung cấp chức năng tách hóa đơn của một bàn thành nhiều hóa đơn nhỏ hơn (theo người hoặc theo món ăn), có xem xét việc phân bổ tiền đặt cọc đã áp dụng. \\
\hline
MD-05 & FR-MD05-11 & US-02, US-05 & Xử lý Thanh toán & Cho phép nhận thanh toán từ khách hàng bằng nhiều phương thức (tiền mặt, thẻ, ví điện tử...), xử lý tiền boa, và ghi nhận giao dịch vào hệ thống sau khi đã trừ tiền đặt cọc. \\
\hline
MD-05 & FR-MD05-12 & US-02 & Đóng Đơn hàng và Bàn & Hoàn tất đơn hàng sau khi thanh toán thành công và cập nhật trạng thái bàn thành trống (sẵn sàng cho khách tiếp theo). \\
\hline
MD-05 & FR-MD05-13 & US-05, US-01 & Đóng Phiên làm việc POS & Kết thúc phiên làm việc POS, hệ thống tổng kết doanh thu theo từng phương thức thanh toán, đối chiếu tiền mặt và chuẩn bị dữ liệu cho bộ phận kế toán. \\
\hline
MD-05 & FR-MD05-14 & US-01, US-02 & Chuyển bàn/Ghép bàn & Cho phép nhân viên chuyển đơn hàng của khách từ bàn này sang bàn khác hoặc ghép nhiều bàn/đơn hàng lại với nhau. \\
\hline
MD-05 & FR-MD05-15 & US-01, US-02 & Hủy món/Hủy đơn (Void) & Cho phép nhân viên (có thể cần quyền quản lý) hủy bỏ một món đã gọi hoặc toàn bộ đơn hàng với lý do cụ thể, có ghi nhận vào hệ thống. \\
\hline

\end{longtable}

\subsubsection{Module MD-06: Quản lý Bán mang về (POS - Takeout)}

Module Quản lý Bán mang về (MD-06) là một phần mở rộng hoặc một chế độ hoạt động chuyên biệt của hệ thống Point of Sale (POS), được thiết kế để phục vụ nhu cầu của khách hàng mua đồ ăn, thức uống để mang đi (Takeout/Takeaway). Module này tập trung vào việc xử lý nhanh chóng các đơn hàng không yêu cầu quản lý bàn, từ việc tạo đơn, chọn món, cho đến thanh toán và hoàn tất giao dịch.

\begin{figure}[H]
    \centering
    \includegraphics[width=15cm]{Sections/tong_quan/functional_spec/img/uc6.png}
    \vspace{0.5cm}
    \caption{Use case diagram cho Module MD-06}
    \label{fig:my_label}
\end{figure}

\begin{longtable}{|m{2cm}|m{2.5cm}|m{2.5cm}|m{4.5cm}|m{4cm}|}
\caption{Danh sách Yêu cầu Chức năng cho Module MD-06: Quản lý Bán hàng Mang về (POS - Takeout)} \label{tab:fr_md06_revised_v3} \\
\hline
\textbf{Mã Module} & \textbf{Mã Yêu cầu CN} & \textbf{Mã Người dùng} & \textbf{Tên Chức năng} & \textbf{Mô tả Ngắn} \\
\hline
\endhead % Header cho các trang tiếp theo
\hline
\endfoot % Footer cho bảng
\hline
\endlastfoot % Footer cho trang cuối cùng

MD-06 & FR-MD06-01 & US-02, US-05 & Chọn Chế độ Bán Mang về & Nhân viên chọn chế độ/giao diện riêng trên POS cho đơn mang về. \\
\hline
MD-06 & FR-MD06-02 & US-02, US-05 & Tạo Đơn hàng Mang về Mới & Nhân viên khởi tạo một đơn hàng mới trong chế độ mang về. \\
\hline
MD-06 & FR-MD06-03 & US-02, US-05 & Gán Khách hàng vào Đơn Mang về & Nhân viên tìm và liên kết đơn mang về với khách hàng có sẵn hoặc tạo mới. \\
\hline
MD-06 & FR-MD06-04 & US-02, US-05 & Thêm Món vào Đơn hàng Mang về & Nhân viên thêm món ăn/đồ uống vào đơn hàng mang về. (Hành động tương tự FR-MD05-05, FR-MD05-06). \\
\hline
MD-06 & FR-MD06-05 & US-02, US-05 & Thêm Ghi chú cho Đơn Mang về & Nhân viên thêm ghi chú đặc biệt cho món hoặc cả đơn mang về. (Hành động tương tự FR-MD05-07). \\
\hline
MD-06 & FR-MD06-06 & US-02, US-05 & Gửi Yêu cầu Chuẩn bị Đơn Mang về (Bếp/Bar) & Nhân viên gửi thông tin món cần chuẩn bị đến bếp/bar, có đánh dấu "Takeout". (Hành động tương tự FR-MD05-08). \\
\hline
MD-06 & FR-MD06-07 & US-02, US-05 & Xác nhận và Tiến hành Thanh toán Đơn Mang về (có xem xét Cọc/Trả trước) & Nhân viên vào màn hình thanh toán, nơi hệ thống đã tự động áp dụng cọc/trả trước (nếu đơn hàng được đặt online và có trả trước). \\
\hline
MD-06 & FR-MD06-08 & US-02, US-05 & Thực hiện Thanh toán Tiền mặt cho Đơn Mang về & Nhân viên nhận tiền mặt và ghi nhận thanh toán. (Hành động tương tự FR-MD05-12). \\
\hline
MD-06 & FR-MD06-09 & US-02, US-05 & Ghi nhận Thanh toán bằng Phương thức Khác (Không Thẻ) cho Đơn Mang về & Nhân viên ghi nhận thanh toán bằng các phương thức khác được hỗ trợ (ví dụ: ví điện tử nếu có). \\
\hline
MD-06 & FR-MD06-10 & US-02, US-05 & Thực hiện Thanh toán Đơn Mang về bằng Nhiều Phương thức (Không Thẻ) & Nhân viên nhận thanh toán bằng cách kết hợp nhiều phương thức được hỗ trợ. (Hành động tương tự FR-MD05-13). \\
\hline
MD-06 & FR-MD06-11 & US-02, US-05 & In Hóa đơn/Biên lai cho Đơn Mang về & Nhân viên kích hoạt in hóa đơn/biên lai sau khi thanh toán. (Hành động tương tự FR-MD05-15). \\
\hline
MD-06 & FR-MD06-12 & US-02, US-05 & Hoàn tất Đơn hàng Mang về & Nhân viên đóng đơn hàng mang về sau khi khách đã thanh toán và nhận hàng. (Hành động tương tự FR-MD05-16). \\
\hline

\end{longtable}


\subsubsubsection{Mục tiêu và Phạm vi}
\label{sssec:md06_objectives_scope}
Mục tiêu chính của module MD-06 là:
\begin{itemize}
    \item \textbf{Xử lý nhanh đơn hàng mang về:} Cung cấp một quy trình tinh gọn cho nhân viên để tiếp nhận và xử lý các đơn hàng mang đi một cách hiệu quả, giảm thời gian chờ đợi cho khách hàng.
    \item \textbf{Quản lý đơn hàng không cần bàn:} Cho phép tạo và quản lý các đơn hàng mà không cần liên kết với một bàn cụ thể trong nhà hàng.
    \item \textbf{Tích hợp với quy trình chuẩn bị:} Đảm bảo thông tin đơn hàng mang về được gửi chính xác xuống bếp/bar, có phân biệt rõ ràng với đơn ăn tại bàn để bộ phận chuẩn bị có quy trình đóng gói phù hợp.
    \item \textbf{Linh hoạt trong việc gán khách hàng (tùy chọn):} Cho phép liên kết đơn hàng với thông tin khách hàng để tiện theo dõi hoặc áp dụng các chương trình khuyến mãi.
    \item \textbf{Hỗ trợ thanh toán đa dạng:} Cho phép khách hàng thanh toán bằng nhiều phương thức khác nhau.
    \item \textbf{Xử lý các khoản trả trước/đặt cọc:} Tự động áp dụng các khoản tiền khách hàng có thể đã thanh toán trước khi đặt hàng mang về qua các kênh khác (ví dụ: website, ứng dụng).
\end{itemize}
Phạm vi của module bao gồm từ việc nhân viên chọn chế độ bán mang về, tạo đơn hàng, thêm món, gửi yêu cầu chuẩn bị, cho đến khi xử lý thanh toán và hoàn tất đơn hàng. Module này không bao gồm các chức năng quản lý bàn hoặc đặt chỗ phức tạp như ở module ăn tại bàn (MD-05) hay đặt chỗ (MD-03).

\subsubsubsection{Đối tượng Sử dụng Chính}
\label{sssec:md06_primary_users}
Các đối tượng người dùng chính tương tác với module này bao gồm:
\begin{itemize}
    \item \textbf{US-02 (Nhân viên phục vụ):} Có thể trực tiếp nhận đơn hàng mang về từ khách tại quầy.
    \item \textbf{US-05 (Nhân viên thu ngân):} Thường là người chính xử lý các đơn hàng mang về, bao gồm việc tạo đơn, nhận thanh toán và hoàn tất giao dịch.
    \item \textbf{US-01 (Quản lý nhà hàng):} Có thể sử dụng các chức năng này và giám sát hoạt động bán mang về.
\end{itemize}
Khách hàng (US-08) là người yêu cầu dịch vụ mang về và cung cấp thông tin đơn hàng.

\subsubsubsection{Các Chức năng Chính}
\label{sssec:md06_key_functionalities}
Module MD-06 cung cấp các chức năng cần thiết để quản lý hiệu quả quy trình bán mang về, được mô tả chi tiết qua các Use Case sau:

\begin{itemize}
    \item \textbf{Khởi tạo và Quản lý Đơn hàng Mang về (UC-MD06-01 đến UC-MD06-03):}
    \begin{itemize}
        \item Cho phép nhân viên chủ động chọn hoặc chuyển sang chế độ hoạt động dành riêng cho bán mang về trên giao diện POS (UC-MD06-01).
        \item Hệ thống tự động hoặc nhân viên khởi tạo một đơn hàng POS mới, được đánh dấu là loại hình "Mang về" (UC-MD06-02).
        \item (Tùy chọn) Cho phép nhân viên tìm kiếm khách hàng đã có hoặc tạo nhanh thông tin khách hàng mới để liên kết với đơn hàng mang về (UC-MD06-03).
    \end{itemize}

    \item \textbf{Thao tác trên Đơn hàng Mang về (UC-MD06-04 đến UC-MD06-06):}
    \begin{itemize}
        \item Thêm các món ăn/đồ uống vào đơn hàng mang về, bao gồm việc chọn biến thể và điều chỉnh số lượng (UC-MD06-04, tương tự UC-MD05-05 và UC-MD05-06).
        \item Thêm các ghi chú hoặc yêu cầu đặc biệt của khách (ví dụ: về đóng gói, khẩu vị) cho từng món hoặc toàn bộ đơn hàng mang về (UC-MD06-05, tương tự UC-MD05-07).
        \item Gửi thông tin các món đã chọn của đơn hàng mang về xuống các máy in bếp/bar hoặc màn hình KDS, có chỉ dẫn rõ đây là đơn mang về (UC-MD06-06, tương tự UC-MD05-08 nhưng có thêm thông tin "Takeout").
    \end{itemize}

    \item \textbf{Xử lý Thanh toán cho Đơn hàng Mang về (UC-MD06-07 đến UC-MD06-12):}
    \begin{itemize}
        \item Trước khi vào màn hình thanh toán, hệ thống tự động kiểm tra và áp dụng (trừ đi) các khoản tiền đặt cọc hoặc thanh toán trước mà khách hàng có thể đã thực hiện cho đơn hàng mang về đó (UC-MD06-07).
        \item Thực hiện thanh toán bằng tiền mặt cho đơn hàng mang về (UC-MD06-08, tương tự UC-MD05-12).
        \item Ghi nhận thanh toán bằng các phương thức khác được hỗ trợ (không bao gồm thẻ ngân hàng qua terminal tích hợp) cho đơn hàng mang về (UC-MD06-09, tương tự UC-MD05-13).
        \item Thực hiện thanh toán cho đơn hàng mang về bằng cách kết hợp nhiều phương thức khác nhau (không bao gồm thẻ) (UC-MD06-10, tương tự UC-MD05-14).
        \item Sau khi nhận đủ thanh toán, cho phép nhân viên kích hoạt (hoặc hệ thống tự động) in hóa đơn/biên lai cuối cùng cho đơn hàng mang về (UC-MD06-11, tương tự UC-MD05-16).
        \item Hoàn tất và đóng đơn hàng mang về trong hệ thống sau khi khách đã thanh toán và nhận hàng (UC-MD06-12, tương tự UC-MD05-17).
    \end{itemize}
\end{itemize}

\subsubsubsection{Tóm tắt Luồng Hoạt động Tổng thể}
\label{sssec:md06_overall_workflow}
Luồng hoạt động điển hình trong module Bán mang về (POS - Takeout) diễn ra như sau:
\begin{enumerate}
    \item \textbf{Chuyển sang chế độ mang về:} Nhân viên Chọn Chế độ Bán Mang về (UC-MD06-01) trên giao diện POS.
    \item \textbf{Tạo đơn hàng:} Hệ thống tự động hoặc nhân viên Tạo Đơn hàng Mang về Mới (UC-MD06-02).
    \item \textbf{(Tùy chọn) Gán khách hàng:} Nhân viên Gán Khách hàng vào Đơn Mang về (UC-MD06-03).
    \item \textbf{Nhập món ăn:}
        \begin{itemize}
            \item Nhân viên Thêm Món vào Đơn hàng Mang về (UC-MD06-04).
            \item Thêm Ghi chú cho Đơn Mang về (UC-MD06-05) nếu khách có yêu cầu đặc biệt.
        \end{itemize}
    \item \textbf{Gửi yêu cầu chuẩn bị:} Nhân viên Gửi Yêu cầu Chuẩn bị Đơn Mang về (Bếp/Bar) (UC-MD06-06), phiếu gửi đi có ghi rõ là "Takeout".
    \item \textbf{Tiến hành thanh toán:}
        \begin{itemize}
            \item Nhân viên chọn thanh toán, hệ thống Xác nhận và Tiến hành Thanh toán Đơn Mang về, có xem xét và tự động áp dụng Cọc/Trả trước nếu có (UC-MD06-07).
            \item Nhân viên nhận thanh toán bằng một hoặc nhiều phương thức: Tiền mặt (UC-MD06-08), Phương thức Khác (Không Thẻ) (UC-MD06-09), hoặc kết hợp Nhiều Phương thức (Không Thẻ) (UC-MD06-10).
        \end{itemize}
    \item \textbf{Hoàn tất giao dịch:}
        \begin{itemize}
            \item Sau khi thanh toán đủ, nhân viên In Hóa đơn/Biên lai cho Đơn Mang về (UC-MD06-11).
            \item Cuối cùng, nhân viên Hoàn tất Đơn hàng Mang về (UC-MD06-12) trong hệ thống.
        \end{itemize}
\end{enumerate}
Module MD-06 đảm bảo rằng các đơn hàng mang về được xử lý một cách nhanh chóng, chính xác và hiệu quả, đáp ứng nhu cầu của cả khách hàng và nhà hàng.


\subsubsection{Module MD-07: Quản lý Giao hàng (POS - Delivery)}

\begin{longtable}{|m{2cm}|m{2.5cm}|m{2.5cm}|m{4.5cm}|m{4cm}|}
\caption{Danh sách Yêu cầu Chức năng cho Module MD-07: Quản lý Giao hàng (POS - Delivery)} \label{tab:fr_md07} \\
\hline
\textbf{Mã Module} & \textbf{Mã Yêu cầu CN} & \textbf{Mã Người dùng} & \textbf{Tên Chức năng} & \textbf{Mô tả Ngắn} \\
\hline
\endhead % Header cho các trang tiếp theo

\hline
\endfoot % Footer cho bảng

\hline
\endlastfoot % Footer cho trang cuối cùng

MD-07 & FR-MD07-01 & US-02, US-05 & Chọn Chế độ Giao hàng & Cho phép nhân viên chọn một chế độ/giao diện riêng biệt trên POS dành cho việc xử lý các đơn hàng giao đi (Delivery). \\
\hline
MD-07 & FR-MD07-02 & US-02, US-05 & Tạo/Mở Đơn hàng Giao hàng & Khởi tạo đơn hàng mới hoặc mở lại đơn hàng giao hàng đang chờ xử lý. Yêu cầu liên kết với thông tin khách hàng (địa chỉ giao, SĐT). \\
\hline
MD-07 & FR-MD07-03 & US-02, US-05 & Liên kết/Nhập Thông tin Khách hàng Giao hàng & Yêu cầu bắt buộc tìm kiếm/chọn khách hàng đã có (với địa chỉ) hoặc nhập thông tin khách hàng mới bao gồm Tên, SĐT và Địa chỉ giao hàng chi tiết. \\
\hline
MD-07 & FR-MD07-04 & US-02, US-05 & Thêm món vào Đơn hàng Giao hàng & Cho phép nhân viên thêm các món ăn/đồ uống vào đơn hàng giao đi. Tương tự UC-MD05-05 / UC-MD06-04. \\
\hline
MD-07 & FR-MD07-05 & US-02, US-05 & Xử lý Ghi chú cho Đơn Giao hàng & Cho phép thêm ghi chú đặc biệt (ví dụ: "Giao vào giờ nghỉ trưa", "Gọi trước khi đến"). Tương tự UC-MD05-06 / UC-MD06-05. \\
\hline
MD-07 & FR-MD07-06 & US-02, US-05 & Gửi đơn Giao hàng xuống Bếp/Bar & Gửi thông tin món cần chuẩn bị xuống bếp/bar, đánh dấu là đơn "Delivery". Tương tự UC-MD05-07 / UC-MD06-06. \\
\hline
MD-07 & FR-MD07-07 & US-02, US-05, System & Áp dụng Đặt cọc/Thanh toán Trước (Nếu có) & Nếu đơn hàng giao đi được đặt online và đã thanh toán trước (toàn bộ hoặc đặt cọc), hệ thống cần áp dụng khoản đã thanh toán này. Logic tương tự UC-MD05-09 / UC-MD06-07. \\
\hline
MD-07 & FR-MD07-08 & US-02, US-05 & Xác nhận và Gửi Đơn hàng sang Shipday & Sau khi đơn hàng sẵn sàng, nhân viên xác nhận và kích hoạt việc gửi thông tin đơn hàng (chi tiết món, thông tin khách hàng, địa chỉ giao) sang hệ thống Shipday qua API. \\
\hline
MD-07 & FR-MD07-09 & System (Odoo/Shipday) & Nhận và Hiển thị Trạng thái Giao hàng từ Shipday & Hệ thống Odoo nhận cập nhật trạng thái giao hàng (ví dụ: Đã gán tài xế, Đang giao, Đã giao thành công, Giao thất bại) từ Shipday (qua webhook) và hiển thị trên chi tiết đơn hàng POS/Backend. \\
\hline
MD-07 & FR-MD07-10 & US-02, US-05 & Xử lý Thanh toán Đơn hàng Giao hàng (Nếu COD) & Nếu đơn hàng thanh toán khi nhận hàng (COD), cho phép nhân viên ghi nhận thanh toán sau khi nhận được tiền từ tài xế giao hàng. \\
\hline
MD-07 & FR-MD07-11 & US-02, US-05 & In Hóa đơn/Phiếu Giao hàng & In hóa đơn/phiếu giao hàng chứa thông tin chi tiết đơn hàng, thông tin khách hàng, địa chỉ giao để tài xế sử dụng và giao cho khách. \\
\hline
MD-07 & FR-MD07-12 & US-02, US-05, System & Đóng Đơn hàng Giao hàng & Hoàn tất và đóng đơn hàng giao đi sau khi đã giao thành công và (nếu COD) đã nhận đủ thanh toán. \\
\hline
MD-07 & FR-MD07-13 & US-01 / US-10 & Cấu hình Tích hợp Shipday & Cho phép cấu hình các tham số kết nối API giữa Odoo và Shipday (API Key, Endpoint...), và các quy tắc đồng bộ dữ liệu. \\
\hline

\end{longtable}


\subsubsection{Module MD-08: Tích hợp Bếp (Kitchen Integration)}
Module Quản lý Giao hàng (MD-07) là một thành phần quan trọng của hệ thống Point of Sale (POS), được thiết kế đặc biệt để hỗ trợ nhà hàng quản lý các đơn hàng mà khách yêu cầu giao đến một địa chỉ cụ thể. Module này tập trung vào việc thu thập thông tin khách hàng và địa chỉ giao hàng, xử lý đơn hàng, và đặc biệt là tích hợp với dịch vụ quản lý giao hàng của bên thứ ba (trong trường hợp này là Shipday) để tự động hóa việc gửi yêu cầu giao hàng và theo dõi trạng thái.

Module Tích hợp Bếp (MD-08) đóng vai trò cầu nối quan trọng giữa bộ phận phục vụ (thông qua hệ thống POS) và bộ phận bếp/bar. Mục tiêu chính của module này là đảm bảo thông tin đơn hàng được truyền tải một cách chính xác, kịp thời và hiệu quả đến các nhân viên bếp, giúp họ chuẩn bị món ăn đúng theo yêu cầu và tối ưu hóa quy trình làm việc trong bếp. Module này có thể được triển khai dưới dạng Màn hình Hiển thị Bếp (Kitchen Display System - KDS) hoặc thông qua việc sử dụng máy in bếp truyền thống.


\begin{figure}[H]
    \centering
    \includegraphics[width=15cm]{Sections/tong_quan/functional_spec/img/uc8.png}
    \vspace{0.5cm}
    \caption{Use case diagram cho Module MD-08}
    \label{fig:my_label}
\end{figure}

\begin{longtable}{|m{2cm}|m{2.5cm}|m{2.5cm}|m{4.5cm}|m{4cm}|}
\caption{Danh sách Yêu cầu Chức năng cho Module MD-08: Tích hợp Bếp (Kitchen Integration)} \label{tab:fr_md08_revised_v2} \\
\hline
\textbf{Mã Module} & \textbf{Mã Yêu cầu CN} & \textbf{Mã Người dùng} & \textbf{Tên Chức năng} & \textbf{Mô tả Ngắn} \\
\hline
\endhead % Header cho các trang tiếp theo
\hline
\endfoot % Footer cho bảng
\hline
\endlastfoot % Footer cho trang cuối cùng

MD-08 & FR-MD08-01 & US-04 & Xem Đơn hàng/Món ăn Mới trên KDS/Máy in Bếp & Nhân viên bếp xem các đơn hàng/món ăn mới được gửi đến KDS hoặc nhận phiếu in từ máy in bếp. (Việc gửi đi là kết quả của FR-MD05-08, FR-MD06-06, FR-MD07-06). \\
\hline
MD-08 & FR-MD08-02 & US-04 & Xem Chi tiết Yêu cầu Món ăn trên KDS/Phiếu in & Nhân viên bếp đọc thông tin chi tiết của từng món cần chuẩn bị: tên, số lượng, biến thể, ghi chú đặc biệt. \\
\hline
MD-08 & FR-MD08-03 & US-04 & Cập nhật Trạng thái Chế biến Món ăn trên KDS & Nhân viên bếp tương tác với KDS để đánh dấu trạng thái chế biến của món ăn (ví dụ: Bắt đầu làm, Đã xong). \\
\hline
MD-08 & FR-MD08-04 & US-04 & Đánh dấu Hoàn thành Toàn bộ Đơn hàng/Phiếu trên KDS & Nhân viên bếp đánh dấu toàn bộ các món trong một đơn hàng/phiếu đã được chuẩn bị xong trên KDS. \\
\hline
MD-08 & FR-MD08-05 & US-04 & Sắp xếp/Đánh dấu Ưu tiên Đơn hàng trên KDS & Nhân viên bếp thay đổi thứ tự hoặc đánh dấu ưu tiên cho các đơn hàng/phiếu trên KDS. \\
\hline
MD-08 & FR-MD08-06 & US-02/US-05 & Xem Cập nhật Trạng thái Món ăn từ Bếp trên POS & Nhân viên phục vụ/thu ngân xem được thông tin món nào đã sẵn sàng từ bếp (nếu KDS có gửi cập nhật về POS). \\
\hline

\end{longtable}


\subsubsubsection{Mục tiêu và Phạm vi}
\label{sssec:md08_objectives_scope}
Mục tiêu chính của module MD-08 là:
\begin{itemize}
    \item \textbf{Truyền tải chính xác yêu cầu món ăn:} Đảm bảo mọi chi tiết của đơn hàng (tên món, số lượng, biến thể, ghi chú đặc biệt) được gửi từ POS đến bếp một cách đầy đủ và không sai sót.
    \item \textbf{Tối ưu hóa quy trình làm việc trong bếp:} Giúp nhân viên bếp dễ dàng tiếp nhận, xem, quản lý và theo dõi tiến độ chuẩn bị các món ăn.
    \item \textbf{Giảm thiểu sai sót và nhầm lẫn:} Hạn chế việc trao đổi thông tin bằng miệng hoặc giấy tờ dễ thất lạc, từ đó giảm lỗi trong quá trình chế biến.
    \item \textbf{Cải thiện thời gian phục vụ:} Giúp bếp nhận yêu cầu nhanh hơn và quản lý thứ tự ưu tiên hiệu quả hơn (đặc biệt với KDS).
    \item \textbf{(Nếu dùng KDS) Cung cấp khả năng theo dõi và cập nhật trạng thái:} Cho phép nhân viên bếp đánh dấu trạng thái chế biến (đang làm, đã xong) và (tùy chọn) đồng bộ thông tin này ngược lại cho nhân viên phục vụ.
    \item \textbf{Hỗ trợ định tuyến thông minh:} Đảm bảo các món ăn được gửi đến đúng trạm chuẩn bị (ví dụ: món chính gửi bếp chính, đồ uống gửi quầy bar) nếu nhà hàng có nhiều khu vực bếp/bar.
\end{itemize}
Phạm vi của module bao gồm việc tiếp nhận yêu cầu món ăn từ hệ thống POS (MD-05, MD-06, MD-07), hiển thị thông tin chi tiết cho nhân viên bếp, và (nếu sử dụng KDS) cho phép nhân viên bếp tương tác để cập nhật trạng thái chế biến. Nó không bao gồm việc quản lý công thức, định lượng nguyên vật liệu, hay các chức năng quản lý kho chi tiết (thuộc các module khác).

\subsubsubsection{Đối tượng Sử dụng Chính}
\label{sssec:md08_primary_users}
Đối tượng người dùng chính của module này là:
\begin{itemize}
    \item \textbf{US-04 (Nhân viên bếp):} Là người trực tiếp sử dụng KDS hoặc nhận phiếu in từ máy in bếp để xem yêu cầu, chuẩn bị món ăn, và (nếu có KDS) cập nhật trạng thái chế biến.
\end{itemize}
Các đối tượng khác tương tác gián tiếp:
\begin{itemize}
    \item \textbf{US-02 (Nhân viên phục vụ) / US-05 (Nhân viên thu ngân):} Là người gửi yêu cầu chuẩn bị món từ POS. Họ cũng có thể (tùy chọn) nhận được cập nhật trạng thái món ăn từ KDS (UC-MD08-06).
    \item \textbf{US-01 (Quản lý nhà hàng) / US-10 (Quản trị viên Hệ thống):} Chịu trách nhiệm cấu hình máy in bếp, KDS, và các quy tắc định tuyến.
\end{itemize}

\subsubsubsection{Các Chức năng Chính}
\label{sssec:md08_key_functionalities}
Module MD-08 cung cấp các chức năng thiết yếu cho việc vận hành bếp, được mô tả chi tiết qua các Use Case sau:

\begin{itemize}
    \item \textbf{Tiếp nhận và Hiển thị Yêu cầu (UC-MD08-01, UC-MD08-02):}
    \begin{itemize}
        \item Nhân viên bếp xem các đơn hàng/món ăn mới xuất hiện trên Màn hình Hiển thị Bếp (KDS) hoặc nhận phiếu yêu cầu được in ra từ máy in bếp (UC-MD08-01).
        \item Nhân viên bếp xem thông tin chi tiết của từng yêu cầu món ăn, bao gồm tên món, số lượng, các tùy chọn biến thể và ghi chú đặc biệt (UC-MD08-02).
    \end{itemize}

    \item \textbf{Quản lý Trạng thái Chế biến trên KDS (UC-MD08-03, UC-MD08-04):} (Áp dụng nếu sử dụng KDS)
    \begin{itemize}
        \item Nhân viên bếp cập nhật trạng thái chế biến của từng món ăn cụ thể trên KDS (ví dụ: "Đang làm", "Đã xong") (UC-MD08-03).
        \item Nhân viên bếp đánh dấu hoàn thành toàn bộ một đơn hàng/phiếu trên KDS khi tất cả các món trong đó đã được chuẩn bị xong (UC-MD08-04).
    \end{itemize}

    \item \textbf{Tối ưu hóa và Đồng bộ hóa (UC-MD08-05, UC-MD08-06):} (Chủ yếu áp dụng cho KDS)
    \begin{itemize}
        \item (Tùy chọn) Nhân viên bếp có thể sắp xếp lại thứ tự hoặc đánh dấu ưu tiên cho các đơn hàng/phiếu trên KDS để quản lý công việc hiệu quả hơn (UC-MD08-05).
        \item (Tùy chọn) Hệ thống cho phép nhân viên phục vụ trên POS xem được thông tin cập nhật về trạng thái món ăn ("Đã xong") từ KDS (UC-MD08-06).
    \end{itemize}
\end{itemize}

\subsubsubsection{Tóm tắt Luồng Hoạt động Tổng thể}
\label{sssec:md08_overall_workflow}
Luồng hoạt động chính trong module Tích hợp Bếp thường diễn ra như sau:
\begin{enumerate}
    \item \textbf{Nhận yêu cầu từ POS:}
        \begin{itemize}
            \item Khi nhân viên phục vụ gửi yêu cầu chuẩn bị món từ POS (UC-MD05-08, UC-MD06-06, UC-MD07-06), thông tin được chuyển đến bếp.
            \item Nhân viên bếp Xem Đơn hàng/Món ăn Mới trên KDS hoặc Nhận Phiếu in Bếp (UC-MD08-01).
        \end{itemize}
    \item \textbf{Xem chi tiết và chuẩn bị:}
        \begin{itemize}
            \item Nhân viên bếp Xem Chi tiết Yêu cầu Món ăn trên KDS/Phiếu in (UC-MD08-02) để nắm rõ các yêu cầu về món, biến thể, và ghi chú.
            \item Nhân viên bếp tiến hành chuẩn bị món ăn.
        \end{itemize}
    \item \textbf{Cập nhật trạng thái (Nếu dùng KDS):}
        \begin{itemize}
            \item Trong quá trình chuẩn bị, nhân viên bếp Cập nhật Trạng thái Chế biến Món ăn trên KDS (UC-MD08-03), ví dụ: chuyển từ "Chờ" sang "Đang làm".
            \item (Tùy chọn) Nhân viên bếp có thể Sắp xếp/Đánh dấu Ưu tiên Đơn hàng trên KDS (UC-MD08-05) nếu cần.
            \item Khi tất cả các món trong một đơn/phiếu đã xong, nhân viên bếp Đánh dấu Hoàn thành Toàn bộ Đơn hàng/Phiếu trên KDS (UC-MD08-04).
        \end{itemize}
    \item \textbf{(Tùy chọn) Đồng bộ về POS (Nếu dùng KDS và có cấu hình):}
        \begin{itemize}
            \item Hệ thống cho phép nhân viên phục vụ Xem Cập nhật Trạng thái Món ăn từ Bếp trên POS (UC-MD08-06), giúp họ biết món nào đã sẵn sàng để phục vụ.
        \end{itemize}
\end{enumerate}
Module MD-08 giúp số hóa và tối ưu hóa giao tiếp giữa bộ phận phục vụ và bếp, góp phần nâng cao hiệu suất và chất lượng dịch vụ của nhà hàng.




\subsubsection{Module MD-09: Quản lý Phiên \& Báo cáo}

Module Quản lý Phiên \& Báo cáo (MD-09) là một thành phần thiết yếu trong hệ thống quản lý nhà hàng, tập trung vào việc cung cấp các công cụ để theo dõi, tổng kết và phân tích hoạt động bán hàng từ các phiên làm việc Point of Sale (POS). Module này cho phép Quản lý nhà hàng và Kế toán truy cập vào dữ liệu lịch sử, xem xét hiệu suất kinh doanh, đối soát tài chính, và đưa ra các quyết định dựa trên số liệu thực tế.




\begin{figure}[H]
    \centering
    \includegraphics[width=15cm]{Sections/tong_quan/functional_spec/img/uc9.png}
    \vspace{0.5cm}
    \caption{Use case diagram cho Module MD-09}
    \label{fig:my_label}
\end{figure}

\begin{longtable}{|m{2cm}|m{2.5cm}|m{2cm}|m{4.5cm}|m{4cm}|}
\caption{Danh sách Yêu cầu Chức năng cho Module MD-09: Quản lý Phiên \& Báo cáo} \label{tab:fr_md09_revised_v2} \\
\hline
\textbf{Mã Module} & \textbf{Mã Yêu cầu CN} & \textbf{Mã Người dùng} & \textbf{Tên Chức năng} & \textbf{Mô tả Ngắn} \\
\hline
\endhead % Header cho các trang tiếp theo
\hline
\endfoot % Footer cho bảng
\hline
\endlastfoot % Footer cho trang cuối cùng

MD-09 & FR-MD09-01 & US-01/US-06 & Xem Danh sách các Phiên POS đã Đóng & Cho phép Quản lý/Kế toán xem danh sách các phiên POS đã được đóng. \\
\hline
MD-09 & FR-MD09-02 & US-01/US-06 & Xem Chi tiết một Phiên POS đã Đóng (Báo cáo Doanh thu Phiên) & Cung cấp báo cáo chi tiết về doanh thu, thanh toán, tiền mặt... của một phiên POS cụ thể đã đóng. \\
\hline
MD-09 & FR-MD09-03 & US-01/US-06 & Xem Báo cáo Tổng hợp Bán hàng theo Sản phẩm/Danh mục POS & Thống kê số lượng và doanh thu của từng sản phẩm/danh mục POS trong một khoảng thời gian. \\
\hline
MD-09 & FR-MD09-04 & US-01 & Xem Báo cáo Hiệu suất Bán hàng của Nhân viên POS & Thống kê doanh thu hoặc số lượng đơn hàng do từng nhân viên xử lý trên POS. \\
\hline
MD-09 & FR-MD09-05 & US-01/US-06 & Xem Báo cáo Quản lý Tiền Đặt cọc & Báo cáo tổng hợp tình hình thu, sử dụng, và mất cọc từ các lượt đặt chỗ. \\
\hline
MD-09 & FR-MD09-06 & US-01/US-06 & Xem Báo cáo Doanh thu theo Loại hình Đơn hàng & Phân tích doanh thu dựa trên các loại hình: Ăn tại chỗ, Mang về, Giao hàng. \\
\hline
MD-09 & FR-MD09-07 & US-01/US-06 & Xuất Dữ liệu từ các Báo cáo & Cho phép xuất dữ liệu báo cáo ra định dạng Excel/CSV. \\
\hline
MD-09 & FR-MD09-08 & US-01/US-06 & Xem Báo cáo Thanh toán theo Phương thức & Thống kê tổng số tiền thu được theo từng phương thức thanh toán (Tiền mặt, Ví điện tử...) trong một khoảng thời gian. \\
\hline
MD-09 & FR-MD09-09 & US-01 & Xem Báo cáo các Khoản Giảm giá/Khuyến mãi đã Áp dụng & Thống kê tổng giá trị giảm giá đã được áp dụng trên các đơn hàng trong một khoảng thời gian. \\
\hline
MD-09 & FR-MD09-10 & US-01 & Xem Báo cáo các Đơn hàng/Món ăn đã Hủy (Void) & Thống kê số lượng và giá trị các đơn hàng hoặc món ăn đã bị hủy trên POS, có thể kèm lý do. \\
\hline
MD-09 & FR-MD09-11 & US-01/US-06 & (Tùy chọn) Thiết lập và Lên lịch Gửi Báo cáo Tự động & Cho phép cấu hình để hệ thống tự động gửi một số báo cáo nhất định (ví dụ: báo cáo doanh thu ngày) đến email của người quản lý theo lịch. \\
\hline

\end{longtable}

\subsubsubsection{Mục tiêu và Phạm vi}
\label{sssec:md08_objectives_scope}
Mục tiêu chính của module MD-08 là:
\begin{itemize}
    \item \textbf{Truyền tải chính xác yêu cầu món ăn:} Đảm bảo mọi chi tiết của đơn hàng (tên món, số lượng, biến thể, ghi chú đặc biệt) được gửi từ POS đến bếp một cách đầy đủ và không sai sót.
    \item \textbf{Tối ưu hóa quy trình làm việc trong bếp:} Giúp nhân viên bếp dễ dàng tiếp nhận, xem, quản lý và theo dõi tiến độ chuẩn bị các món ăn.
    \item \textbf{Giảm thiểu sai sót và nhầm lẫn:} Hạn chế việc trao đổi thông tin bằng miệng hoặc giấy tờ dễ thất lạc, từ đó giảm lỗi trong quá trình chế biến.
    \item \textbf{Cải thiện thời gian phục vụ:} Giúp bếp nhận yêu cầu nhanh hơn và quản lý thứ tự ưu tiên hiệu quả hơn (đặc biệt với KDS).
    \item \textbf{(Nếu dùng KDS) Cung cấp khả năng theo dõi và cập nhật trạng thái:} Cho phép nhân viên bếp đánh dấu trạng thái chế biến (đang làm, đã xong) và (tùy chọn) đồng bộ thông tin này ngược lại cho nhân viên phục vụ.
    \item \textbf{Hỗ trợ định tuyến thông minh:} Đảm bảo các món ăn được gửi đến đúng trạm chuẩn bị (ví dụ: món chính gửi bếp chính, đồ uống gửi quầy bar) nếu nhà hàng có nhiều khu vực bếp/bar.
\end{itemize}

\subsubsubsection{Mục tiêu và Phạm vi}
\label{sssec:md09_objectives_scope}
Mục tiêu chính của module MD-09 là:
\begin{itemize}
    \item \textbf{Cung cấp thông tin tổng kết phiên POS:} Cho phép xem lại chi tiết tài chính và hoạt động của từng phiên POS đã đóng.
    \item \textbf{Phân tích hiệu quả bán hàng:} Cung cấp các báo cáo đa dạng về doanh thu theo sản phẩm, danh mục, nhân viên, và loại hình đơn hàng.
    \item \textbf{Hỗ trợ đối soát tài chính:} Cung cấp báo cáo về các phương thức thanh toán và quản lý tiền đặt cọc, giúp kế toán dễ dàng đối soát.
    \item \textbf{Kiểm soát thất thoát và hoạt động bất thường:} Thông qua báo cáo về các đơn hàng/món ăn đã hủy.
    \item \textbf{Tăng cường khả năng ra quyết định dựa trên dữ liệu:} Cung cấp thông tin đầu vào quan trọng cho việc lập kế hoạch kinh doanh, marketing, và quản lý nhân sự.
    \item \textbf{(Tùy chọn) Tự động hóa việc gửi báo cáo:} Giúp các nhà quản lý nhận được thông tin quan trọng một cách định kỳ mà không cần thao tác thủ công.
\end{itemize}
Phạm vi của module bao gồm việc hiển thị danh sách các phiên POS đã đóng, cung cấp báo cáo chi tiết cho từng phiên, tạo các báo cáo tổng hợp và phân tích về doanh thu, sản phẩm, nhân viên, thanh toán, giảm giá, hủy đơn, và (tùy chọn) thiết lập cơ chế gửi báo cáo tự động. Dữ liệu đầu vào cho module này chủ yếu đến từ các giao dịch được ghi nhận trong các module POS (MD-05, MD-06, MD-07) và Đặt chỗ (MD-03, liên quan đến đặt cọc).

\subsubsubsection{Đối tượng Sử dụng Chính}
\label{sssec:md09_primary_users}
Các đối tượng người dùng chính tương tác với module này bao gồm:
\begin{itemize}
    \item \textbf{US-01 (Quản lý nhà hàng):} Là người dùng thường xuyên nhất, sử dụng các báo cáo để theo dõi tình hình kinh doanh, đánh giá hiệu suất, kiểm soát hoạt động và đưa ra các quyết định chiến lược.
    \item \textbf{US-06 (Kế toán):} Sử dụng các báo cáo (đặc biệt là báo cáo chi tiết phiên, báo cáo thanh toán theo phương thức, báo cáo tiền đặt cọc) để thực hiện công việc đối soát tài chính, hạch toán kế toán.
\end{itemize}

\subsubsubsection{Các Chức năng Chính}
\label{sssec:md09_key_functionalities}
Module MD-09 cung cấp một loạt các chức năng báo cáo và quản lý dữ liệu lịch sử, được mô tả chi tiết qua các Use Case sau:

\begin{itemize}
    \item \textbf{Quản lý và Xem lại Phiên POS (UC-MD09-01, UC-MD09-02):}
    \begin{itemize}
        \item Cho phép xem danh sách các phiên làm việc POS đã được đóng, với các thông tin tóm tắt và khả năng lọc/tìm kiếm (UC-MD09-01).
        \item Xem báo cáo chi tiết tổng kết của một phiên POS cụ thể đã đóng, bao gồm doanh thu, chi tiết thanh toán, và đối chiếu tiền mặt (nếu có) (UC-MD09-02).
    \end{itemize}

    \item \textbf{Báo cáo Phân tích Bán hàng (UC-MD09-03, UC-MD09-04, UC-MD09-06, UC-MD09-08, UC-MD09-09, UC-MD09-10):}
    \begin{itemize}
        \item Xem báo cáo thống kê về số lượng bán ra và doanh thu của từng sản phẩm hoặc theo từng danh mục sản phẩm POS (UC-MD09-03).
        \item Xem báo cáo về hiệu suất bán hàng của từng nhân viên POS, bao gồm tổng doanh thu và số lượng đơn hàng (UC-MD09-04).
        \item Xem báo cáo phân tích tổng doanh thu theo từng loại hình đơn hàng (Ăn tại chỗ, Mang về, Giao hàng) (UC-MD09-06).
        \item Xem báo cáo thống kê tổng số tiền đã thu được qua từng phương thức thanh toán khác nhau (UC-MD09-08).
        \item Xem báo cáo về tổng giá trị các khoản giảm giá hoặc chương trình khuyến mãi đã được áp dụng (UC-MD09-09).
        \item Xem báo cáo thống kê về các món ăn hoặc toàn bộ đơn hàng đã bị hủy (voided) trên POS (UC-MD09-10).
    \end{itemize}

    \item \textbf{Báo cáo Quản lý Đặc thù (UC-MD09-05):}
    \begin{itemize}
        \item Xem báo cáo tổng hợp về tình hình thu và sử dụng tiền đặt cọc từ các lượt đặt chỗ (UC-MD09-05).
    \end{itemize}

    \item \textbf{Tiện ích Báo cáo (UC-MD09-07, UC-MD09-11):}
    \begin{itemize}
        \item Cho phép xuất dữ liệu từ các giao diện báo cáo ra các định dạng tệp phổ biến như Excel hoặc CSV (UC-MD09-07).
        \item (Tùy chọn) Cho phép thiết lập và lên lịch để hệ thống tự động tạo và gửi một số loại báo cáo nhất định đến email theo định kỳ (UC-MD09-11).
    \end{itemize}
\end{itemize}

\subsubsubsection{Tóm tắt Luồng Hoạt động Tổng thể}
\label{sssec:md09_overall_workflow}
Luồng hoạt động chính trong module Quản lý Phiên & Báo cáo thường diễn ra như sau:
\begin{enumerate}
    \item \textbf{Truy cập thông tin phiên đã đóng:}
        \begin{itemize}
            \item Người dùng (Quản lý/Kế toán) Xem Danh sách các Phiên POS đã Đóng (UC-MD09-01).
            \item Từ danh sách, người dùng chọn một phiên cụ thể để Xem Chi tiết một Phiên POS đã Đóng (Báo cáo Doanh thu Phiên) (UC-MD09-02).
        \end{itemize}
    \item \textbf{Xem các báo cáo phân tích và tổng hợp:}
        \begin{itemize}
            \item Người dùng lựa chọn và xem các loại báo cáo khác nhau tùy theo nhu cầu phân tích:
                \begin{itemize}
                    \item Báo cáo Tổng hợp Bán hàng theo Sản phẩm/Danh mục POS (UC-MD09-03).
                    \item Báo cáo Hiệu suất Bán hàng của Nhân viên POS (UC-MD09-04).
                    \item Báo cáo Quản lý Tiền Đặt cọc (UC-MD09-05).
                    \item Báo cáo Doanh thu theo Loại hình Đơn hàng (UC-MD09-06).
                    \item Báo cáo Thanh toán theo Phương thức (UC-MD09-08).
                    \item Báo cáo các Khoản Giảm giá/Khuyến mãi đã Áp dụng (UC-MD09-09).
                    \item Báo cáo các Đơn hàng/Món ăn đã Hủy (Void) (UC-MD09-10).
                \end{itemize}
            \item Trong quá trình xem các báo cáo này, người dùng có thể cần Xuất Dữ liệu từ các Báo cáo (UC-MD09-07) ra file để lưu trữ hoặc phân tích thêm.
        \end{itemize}
    \item \textbf{(Tùy chọn) Thiết lập gửi báo cáo tự động:}
        \begin{itemize}
            \item Nếu có nhu cầu, người dùng có thể Thiết lập và Lên lịch Gửi Báo cáo Tự động (UC-MD09-11) cho các báo cáo quan trọng.
        \end{itemize}
\end{enumerate}
Module MD-09 cung cấp cái nhìn sâu sắc về hoạt động kinh doanh của nhà hàng, hỗ trợ việc đưa ra các quyết định quản lý dựa trên dữ liệu chính xác và kịp thời, đồng thời đảm bảo tính minh bạch và kiểm soát tài chính.



\subsubsection{Module MD-10: Quản lý Hệ thống \& Người 
dùng}
Module Quản lý Hệ thống \& Người dùng (MD-10) đóng vai trò là hạt nhân quản trị và cấu hình của toàn bộ hệ thống nhà hàng. Module này cung cấp các công cụ thiết yếu cho Quản trị viên hệ thống và Quản lý nhà hàng để quản lý tài khoản người dùng (nhân viên và khách hàng), phân quyền truy cập, thiết lập các thông số hoạt động chung của nhà hàng, cấu hình các tích hợp với dịch vụ của bên thứ ba (như cổng thanh toán, dịch vụ bot call, dịch vụ giao hàng), và theo dõi hoạt động hệ thống. Sự ổn định, bảo mật và khả năng tùy biến của module này là nền tảng cho hoạt động hiệu quả của tất cả các module nghiệp vụ khác.


\begin{figure}[H]
    \centering
    \includegraphics[width=15cm]{Sections/tong_quan/functional_spec/img/uc10.png}
    \vspace{0.5cm}
    \caption{Use case diagram cho Module MD-10}
    \label{fig:my_label}
\end{figure}

\begin{longtable}{|m{2cm}|m{2.5cm}|m{2.5cm}|m{4.5cm}|m{4cm}|}
\caption{Danh sách Yêu cầu Chức năng cho Module MD-10: Quản lý Hệ thống, Người dùng \& Xác thực} 
\hline
\textbf{Mã Module} & \textbf{Mã Yêu cầu CN} & \textbf{Mã Người dùng} & \textbf{Tên Chức năng} & \textbf{Mô tả Ngắn} \
\hline
\endhead % Header cho các trang tiếp theo
\midrule
\endfoot % Footer cho bảng
\bottomrule
\endlastfoot % Footer cho trang cuối cùng
MD-10 & FR-MD10-01 & US-10 & Tạo mới Tài khoản Người dùng (Nhân viên) & Quản trị viên tạo tài khoản đăng nhập mới cho nhân viên, bao gồm thông tin cơ bản. \
\hline
MD-10 & FR-MD10-02 & US-10 & Xem Danh sách Người dùng & Quản trị viên xem danh sách các tài khoản người dùng hiện có trong hệ thống. \
\hline
MD-10 & FR-MD10-03 & US-10 & Sửa Thông tin Tài khoản Người dùng & Quản trị viên cập nhật các thông tin cá nhân, liên hệ của một tài khoản người dùng. \
\hline
MD-10 & FR-MD10-04 & US-10 & Gán/Gỡ bỏ Nhóm Quyền cho Người dùng & Quản trị viên thay đổi các nhóm quyền truy cập mà một người dùng thuộc về. \
\hline
MD-10 & FR-MD10-05 & US-10 & Vô hiệu hóa Tài khoản Người dùng & Quản trị viên tạm khóa (archive) khả năng đăng nhập của một tài khoản người dùng. \
\hline
MD-10 & FR-MD10-06 & US-10 & Kích hoạt lại Tài khoản Người dùng đã Vô hiệu hóa & Quản trị viên mở lại (unarchive) khả năng đăng nhập cho một tài khoản đã bị khóa. \
\hline
MD-10 & FR-MD10-07 & US-10 & Đặt lại Mật khẩu cho Người dùng (bởi Admin) & Quản trị viên hỗ trợ đặt lại mật khẩu cho người dùng (gửi link hoặc đặt trực tiếp). \
\hline
MD-10 & FR-MD10-08 & US-10 & Xem Chi tiết một Nhóm Quyền Truy cập & Quản trị viên xem chi tiết cấu hình và các quyền hạn của một nhóm quyền cụ thể. \
\hline
MD-10 & FR-MD10-09 & US-10, US-01 & Cấu hình Thông tin Chung của Nhà hàng/Công ty & Quản trị viên/Quản lý thiết lập tên, địa chỉ, logo, tiền tệ mặc định cho nhà hàng/công ty. \
\hline
MD-10 & FR-MD10-10 & US-10, US-01 & Cấu hình Máy chủ Gửi Email (Outgoing Email Server) & Quản trị viên/Quản lý thiết lập thông tin máy chủ SMTP để hệ thống có thể gửi email đi. \
\hline
MD-10 & FR-MD10-11 & US-10, US-01 & Cấu hình Tích hợp Cổng Thanh toán & Quản trị viên/Quản lý nhập API keys và các tham số cho các cổng thanh toán trực tuyến. \
\hline
MD-10 & FR-MD10-12 & US-10, US-01 & Cấu hình Tích hợp Dịch vụ Bot Call & Quản trị viên/Quản lý nhập API keys và các tham số vận hành cho dịch vụ Bot Call. \
\hline
MD-10 & FR-MD10-13 & US-10, US-01 & Cấu hình Tích hợp Dịch vụ Giao hàng (Shipday) & Quản trị viên/Quản lý nhập API keys và các tham số vận hành cho dịch vụ Shipday. \
\hline
MD-10 & FR-MD10-14 & US-10, US-01 & Cấu hình Tham số Nghiệp vụ Đặc thù cho Đặt chỗ & Quản trị viên/Quản lý thiết lập các quy tắc kinh doanh như tỷ lệ đặt cọc, giá trị bàn, số ngày gọi bot... \
\hline
MD-10 & FR-MD10-15 & US-10 & Xem Nhật ký Hoạt động Hệ thống (Logs) & Quản trị viên xem các bản ghi log hệ thống để theo dõi, chẩn đoán lỗi và hoạt động. \
\hline
MD-10 & FR-MD10-16 & US-01, US-02, US-03, US-04, US-05, US-06, US-07, US-09, US-10 & Đăng nhập vào Hệ thống & Người dùng cung cấp thông tin xác thực (email/mật khẩu) để truy cập hệ thống. \
\hline
MD-10 & FR-MD10-17 & US-01, US-02, US-03, US-04, US-05, US-06, US-07, US-09, US-10 & Đăng xuất khỏi Hệ thống & Người dùng kết thúc phiên làm việc hiện tại và thoát khỏi hệ thống an toàn. \
\hline
MD-10 & FR-MD10-18 & US-01, US-02, US-03, US-04, US-05, US-06, US-07, US-09, US-10 & Người dùng Yêu cầu Đặt lại Mật khẩu (Quên Mật khẩu) & Người dùng tự yêu cầu hệ thống gửi hướng dẫn đặt lại mật khẩu qua email khi họ quên mật khẩu. \
\hline
MD-10 & FR-MD10-19 & US-08 & Khách hàng Tự Đăng ký Tài khoản & Khách hàng tự tạo tài khoản người dùng trên giao diện web/app để đặt chỗ, quản lý thông tin. \
\hline
\end{longtable}
    

% Bỏ comment và sửa đường dẫn nếu bạn có file ảnh uc10.png
% \begin{figure}[H]
%     \centering
%     % \includegraphics[width=15cm]{Sections/tong_quan/functional_spec/img/uc10.png.png} % Thay thế bằng đường dẫn thực tế đến file ảnh của bạn
%     \caption{Sơ đồ Use Case cho Module MD-10 (Ví dụ)} % Sửa caption nếu cần
%     \label{fig:uc_diagram_md10_full} % Sửa label nếu cần
% \end{figure}

\subsubsubsection{Mục tiêu và Phạm vi}
\label{sssec:md10_objectives_scope_full}
Mục tiêu chính của module MD-10 là:
\begin{itemize}
    \item \textbf{Quản lý tài khoản người dùng toàn diện:} Cho phép tạo, sửa đổi, vô hiệu hóa, kích hoạt lại và quản lý mật khẩu cho tất cả các loại tài khoản người dùng (nhân viên và khách hàng).
    \item \textbf{Kiểm soát truy cập chặt chẽ:} Thông qua việc quản lý các nhóm quyền và gán người dùng vào các nhóm phù hợp, đảm bảo nguyên tắc phân quyền tối thiểu và bảo mật dữ liệu.
    \item \textbf{Cấu hình thông tin và hoạt động chung của nhà hàng:} Thiết lập các thông tin định danh của nhà hàng (tên, địa chỉ, logo), đơn vị tiền tệ, và các tham số nghiệp vụ đặc thù ảnh hưởng đến hoạt động của các module khác (ví dụ: quy tắc đặt cọc cho module đặt chỗ).
    \item \textbf{Quản lý tích hợp với các dịch vụ bên thứ ba:} Cho phép cấu hình và quản lý kết nối API với các dịch vụ thiết yếu như máy chủ gửi email, cổng thanh toán trực tuyến, dịch vụ bot call, và dịch vụ quản lý giao hàng.
    \item \textbf{Cung cấp cơ chế xác thực an toàn:} Đảm bảo quy trình đăng nhập, đăng xuất và đặt lại mật khẩu được thực hiện một cách an toàn và hiệu quả.
    \item \textbf{Theo dõi và chẩn đoán hệ thống:} Cung cấp khả năng xem nhật ký hoạt động của hệ thống để hỗ trợ việc giám sát, chẩn đoán và khắc phục sự cố.
\end{itemize}
Phạm vi của module bao gồm toàn bộ vòng đời quản lý người dùng, quản lý phân quyền, cấu hình các thông số hệ thống cơ bản và các tích hợp quan trọng, cũng như các chức năng xác thực người dùng và theo dõi log hệ thống.

\subsubsubsection{Đối tượng Sử dụng Chính}
\label{sssec:md10_primary_users_full}
Module này phục vụ nhiều nhóm đối tượng người dùng với các vai trò khác nhau:
\begin{itemize}
    \item \textbf{US-10 (Quản trị viên Hệ thống):} Là người dùng có quyền hạn cao nhất, chịu trách nhiệm chính trong việc tạo và quản lý tài khoản người dùng (nhân viên), phân quyền, cấu hình các tích hợp kỹ thuật (máy chủ email, API), và xem log hệ thống.
    \item \textbf{US-01 (Quản lý nhà hàng):} Có thể được cấp quyền để cấu hình một số thông tin chung của nhà hàng, các tham số nghiệp vụ, và một số tích hợp dịch vụ (cổng thanh toán, bot call, giao hàng). Cũng có thể có quyền quản lý một số khía cạnh của tài khoản nhân viên.
    \item \textbf{US-01, US-02, US-03, US-04, US-05, US-06, US-07, US-09 (Tất cả Nhân viên có tài khoản):} Là người dùng của các chức năng xác thực cơ bản như Đăng nhập, Đăng xuất, và Yêu cầu Đặt lại Mật khẩu khi quên.
    \item \textbf{US-08 (Khách hàng):} Là người dùng của chức năng Tự Đăng ký Tài khoản trên giao diện web/app của nhà hàng.
\end{itemize}

\subsubsubsection{Các Chức năng Chính}
\label{sssec:md10_key_functionalities_full}
Module MD-10 cung cấp một bộ các chức năng quản trị và cấu hình hệ thống toàn diện, được mô tả chi tiết qua các Use Case sau:

\begin{itemize}
    \item \textbf{Quản lý Tài khoản Người dùng (Nhân viên) (UC-MD10-01 đến UC-MD10-07):}
    \begin{itemize}
        \item Tạo mới tài khoản đăng nhập cho nhân viên (UC-MD10-01).
        \item Xem danh sách tất cả các tài khoản người dùng trong hệ thống (UC-MD10-02).
        \item Sửa đổi thông tin cá nhân và liên hệ của một tài khoản người dùng (UC-MD10-03).
        \item Gán hoặc gỡ bỏ các Nhóm Quyền truy cập cho một người dùng để xác định phạm vi hoạt động (UC-MD10-04).
        \item Vô hiệu hóa (tạm khóa) khả năng đăng nhập của một tài khoản người dùng (UC-MD10-05).
        \item Kích hoạt lại một tài khoản người dùng đã bị vô hiệu hóa trước đó (UC-MD10-06).
        \item Quản trị viên hỗ trợ đặt lại mật khẩu cho một người dùng (UC-MD10-07).
    \end{itemize}

    \item \textbf{Quản lý Phân quyền (UC-MD10-08):}
    \begin{itemize}
        \item Xem chi tiết cấu hình và các quyền hạn cụ thể của một Nhóm Quyền truy cập đã tồn tại (UC-MD10-08).
    \end{itemize}

    \item \textbf{Cấu hình Hệ thống và Tích hợp (UC-MD10-09 đến UC-MD10-14):}
    \begin{itemize}
        \item Thiết lập các thông tin chung của nhà hàng/công ty như tên, địa chỉ, logo, tiền tệ (UC-MD10-09).
        \item Cấu hình thông tin máy chủ SMTP để hệ thống có thể gửi email (UC-MD10-10).
        \item Nhập API keys và các tham số để tích hợp với các cổng thanh toán trực tuyến (UC-MD10-11).
        \item Nhập API keys và các tham số vận hành để tích hợp với dịch vụ Bot Call (UC-MD10-12, liên quan FR-MD04-05).
        \item Nhập API keys và các tham số vận hành để tích hợp với dịch vụ quản lý giao hàng Shipday (UC-MD10-13, liên quan FR-MD07-15).
        \item Thiết lập các quy tắc và tham số nghiệp vụ đặc thù cho chức năng Đặt chỗ (ví dụ: tỷ lệ đặt cọc, giá trị bàn, số ngày gọi bot) (UC-MD10-14, liên quan FR-MD03-15 và FR-MD04-05).
    \end{itemize}

    \item \textbf{Theo dõi và Bảo trì Hệ thống (UC-MD10-15):}
    \begin{itemize}
        \item Cho phép Quản trị viên xem các bản ghi nhật ký hoạt động của hệ thống để theo dõi và chẩn đoán lỗi (UC-MD10-15).
    \end{itemize}

    \item \textbf{Xác thực Người dùng (UC-MD10-16 đến UC-MD10-19):}
    \begin{itemize}
        \item Cho phép tất cả người dùng có tài khoản (nhân viên, khách hàng) đăng nhập vào hệ thống bằng tên đăng nhập/email và mật khẩu (UC-MD10-16).
        \item Cho phép người dùng đã đăng nhập kết thúc phiên làm việc và thoát khỏi hệ thống một cách an toàn (UC-MD10-17).
        \item Cho phép người dùng tự yêu cầu hệ thống gửi hướng dẫn đặt lại mật khẩu qua email khi họ quên mật khẩu (UC-MD10-18).
        \item Cho phép khách hàng mới tự tạo tài khoản người dùng trên giao diện web/app của nhà hàng (UC-MD10-19).
    \end{itemize}
\end{itemize}

\subsubsubsection{Tóm tắt Luồng Hoạt động Tổng thể}
\label{sssec:md10_overall_workflow_full}
Luồng hoạt động trong module MD-10 rất đa dạng, phụ thuộc vào vai trò của người dùng và nhu cầu cụ thể. Một số luồng chính bao gồm:
\begin{enumerate}
    \item \textbf{Thiết lập hệ thống ban đầu hoặc khi có thay đổi lớn:}
        \begin{itemize}
            \item Quản trị viên/Quản lý Cấu hình Thông tin Chung của Nhà hàng (UC-MD10-09).
            \item Cấu hình Máy chủ Gửi Email (UC-MD10-10).
            \item Cấu hình Tích hợp Cổng Thanh toán (UC-MD10-11).
            \item Cấu hình Tích hợp Dịch vụ Bot Call (UC-MD10-12).
            \item Cấu hình Tích hợp Dịch vụ Giao hàng (Shipday) (UC-MD10-13).
            \item Cấu hình Tham số Nghiệp vụ Đặc thù cho Đặt chỗ (UC-MD10-14).
        \end{itemize}
    \item \textbf{Quản lý tài khoản nhân viên:}
        \begin{itemize}
            \item Quản trị viên Tạo mới Tài khoản Người dùng (Nhân viên) (UC-MD10-01).
            \item Gán/Gỡ bỏ Nhóm Quyền cho Người dùng (UC-MD10-04) (có thể tham khảo UC-MD10-08 để hiểu rõ nhóm quyền).
            \item Khi cần, Sửa Thông tin Tài khoản Người dùng (UC-MD10-03).
            \item Hỗ trợ Đặt lại Mật khẩu cho Người dùng (bởi Admin) (UC-MD10-07) nếu nhân viên yêu cầu.
            \item Vô hiệu hóa Tài khoản Người dùng (UC-MD10-05) khi nhân viên nghỉ việc, hoặc Kích hoạt lại Tài khoản Người dùng đã Vô hiệu hóa (UC-MD10-06) khi cần.
            \item Quản trị viên thường xuyên Xem Danh sách Người dùng (UC-MD10-02) để kiểm tra.
        \end{itemize}
    \item \textbf{Quy trình xác thực của tất cả người dùng:}
        \begin{itemize}
            \item Người dùng (nhân viên hoặc khách hàng đã đăng ký) Đăng nhập vào Hệ thống (UC-MD10-16).
            \item Nếu quên mật khẩu, người dùng Yêu cầu Đặt lại Mật khẩu (Quên Mật khẩu) (UC-MD10-18).
            \item Sau khi làm việc xong, người dùng Đăng xuất khỏi Hệ thống (UC-MD10-17).
        \end{itemize}
    \item \textbf{Khách hàng tự đăng ký (nếu có kênh online):}
        \begin{itemize}
            \item Khách hàng mới Tự Đăng ký Tài khoản (UC-MD10-19) trên website/app.
        \end{itemize}
    \item \textbf{Theo dõi và bảo trì hệ thống:}
        \begin{itemize}
            \item Quản trị viên Xem Nhật ký Hoạt động Hệ thống (Logs) (UC-MD10-15) để chẩn đoán sự cố hoặc theo dõi hoạt động.
        \end{itemize}
\end{enumerate}
Module MD-10 là xương sống đảm bảo cho toàn bộ hệ thống nhà hàng có thể vận hành một cách trơn tru, an toàn và tuân thủ các quy định, chính sách đã đặt ra.



\subsubsection{Module MD-11: Quản lý Quan hệ Khách hàng (CRM)}
Module Quản lý Quan hệ Khách hàng (MD-11) là một công cụ chiến lược giúp nhà hàng xây dựng và duy trì mối quan hệ bền chặt với khách hàng. Module này tập trung vào việc thu thập, lưu trữ, quản lý thông tin khách hàng, theo dõi lịch sử tương tác, phân loại khách hàng, quản lý các chương trình khuyến mãi/voucher, và thu thập phản hồi/đánh giá từ khách hàng. Mục tiêu cuối cùng là nâng cao sự hài lòng của khách hàng, tăng cường lòng trung thành và thúc đẩy doanh thu.



\begin{figure}[H]
    \centering
    \includegraphics[width=15cm]{Sections/tong_quan/functional_spec/img/uc11.png}
    \vspace{0.5cm}
    \caption{Use case diagram cho Module MD-11}
    \label{fig:my_label}
\end{figure}

\begin{longtable}{|m{2cm}|m{2.5cm}|m{2.5cm}|m{4.5cm}|m{4cm}|}
\caption{Danh sách Yêu cầu Chức năng cho Module MD-11: Quản lý Quan hệ Khách hàng (CRM)} \label{tab:fr_md11_crm_marketing_revised_in_codeblock} \\
\hline
\textbf{Mã Module} & \textbf{Mã Yêu cầu CN} & \textbf{Mã Người dùng} & \textbf{Tên Chức năng} & \textbf{Mô tả Ngắn} \\
\hline
\endhead % Header cho các trang tiếp theo
\midrule
\endfoot % Footer cho bảng
\bottomrule
\endlastfoot % Footer cho trang cuối cùng

% === Quản lý Hồ sơ Khách hàng (CRM View) - Tách nhỏ ===
MD-11 & FR-MD11-01 & US-01, US-03, US-10 & Tạo mới Hồ sơ Khách hàng (CRM) & Nhân viên/Quản lý tạo hồ sơ mới cho khách hàng với các thông tin chi tiết. \\
\hline
MD-11 & FR-MD11-02 & US-01, US-03, US-10 & Xem Danh sách Hồ sơ Khách hàng (CRM) & Xem danh sách tất cả khách hàng trong hệ thống CRM, có thể tìm kiếm, lọc. \\
\hline
MD-11 & FR-MD11-03 & US-01, US-03, US-10 & Xem Chi tiết Hồ sơ Khách hàng (CRM) & Xem toàn bộ thông tin chi tiết của một khách hàng cụ thể (thông tin cá nhân, lịch sử, sở thích...). \\
\hline
MD-11 & FR-MD11-04 & US-01, US-03, US-10 & Sửa Thông tin Hồ sơ Khách hàng (CRM) & Cập nhật, chỉnh sửa các thông tin trong hồ sơ của một khách hàng đã có. \\
\hline
MD-11 & FR-MD11-05 & US-01, US-10 & Xóa/Lưu trữ Hồ sơ Khách hàng (CRM) & Xóa (nếu chưa có giao dịch) hoặc lưu trữ (ẩn đi) hồ sơ khách hàng không còn hoạt động. \\
\hline
MD-11 & FR-MD11-06 & US-01, US-06, US-10 & Phân loại/Gắn thẻ Khách hàng & Phân loại khách hàng (VIP, thường xuyên...) hoặc gắn thẻ (tag) cho khách hàng để phục vụ marketing, chăm sóc. \\
\hline
MD-11 & FR-MD11-07 & US-01, US-06, US-10 & Xem Lịch sử Tương tác/Đặt chỗ của Khách hàng & Truy cập toàn bộ lịch sử đặt chỗ, hóa đơn, phản hồi, ghi chú tương tác của một khách hàng cụ thể. \\
\hline

% === Quản lý Voucher/Khuyến mãi ===
MD-11 & FR-MD11-08 & US-01, US-10 & Tạo mới Chương trình Khuyến mãi/Voucher & Định nghĩa các chương trình khuyến mãi (giảm giá \%, giảm tiền cố định, mua X tặng Y) hoặc tạo mã voucher. \\
\hline
MD-11 & FR-MD11-09 & US-01, US-10 & Thiết lập Điều kiện Áp dụng Khuyến mãi/Voucher & Cấu hình điều kiện: thời gian, giá trị đơn tối thiểu, sản phẩm/danh mục áp dụng, số lần sử dụng... \\
\hline
MD-11 & FR-MD11-10 & US-01, US-10 & Quản lý Danh sách Mã Voucher & Xem danh sách mã voucher đã tạo, trạng thái (đã dùng, còn hạn), có thể xuất mã hàng loạt, vô hiệu hóa voucher. \\
\hline
MD-11 & FR-MD11-11 & US-02, US-05 & Áp dụng Khuyến mãi/Voucher vào Đơn hàng POS & Nhân viên POS nhập mã voucher hoặc chọn chương trình khuyến mãi đủ điều kiện để áp dụng cho đơn hàng. \\
\hline
MD-11 & FR-MD11-12 & US-08 & Khách hàng Sử dụng Voucher khi Đặt chỗ Online & Khách hàng nhập mã voucher hợp lệ vào đơn đặt chỗ/đặt món online để được giảm giá. \\
\hline
MD-11 & FR-MD11-13 & US-01 & Xem Báo cáo Hiệu quả Khuyến mãi/Voucher & Thống kê số lần sử dụng, tổng giá trị giảm giá của từng chương trình/voucher. (Mở rộng từ UC-MD09-09) \\
\hline

% === Thu thập & Quản lý Đánh giá/Phản hồi ===
MD-11 & FR-MD11-14 & US-08 & Khách hàng Gửi Đánh giá/Review sau Khi sử dụng Dịch vụ & Khách hàng (có thể qua email mời hoặc link trên hóa đơn) gửi đánh giá về chất lượng món ăn, dịch vụ. \\
\hline
MD-11 & FR-MD11-15 & US-01, US-09 & Xem và Quản lý Đánh giá/Phản hồi của Khách hàng & Quản lý xem các đánh giá, có thể phản hồi, phân loại (tích cực, tiêu cực), hoặc đánh dấu đã xử lý. \\
\hline
MD-11 & FR-MD11-16 & US-01 & (Tùy chọn) Hiển thị Đánh giá Tích cực trên Website & Chọn lọc và hiển thị các đánh giá tốt của khách hàng trên trang web nhà hàng (nếu muốn). \\
\hline
% Các chức năng khác như Loyalty, Email Marketing sẽ được thêm vào sau nếu cần
\end{longtable}

\subsubsubsection{Mục tiêu và Phạm vi}
\label{sssec:md11_objectives_scope}
Mục tiêu chính của module MD-11 là:
\begin{itemize}
    \item \textbf{Xây dựng cơ sở dữ liệu khách hàng toàn diện:} Lưu trữ thông tin liên hệ, lịch sử giao dịch, sở thích và các ghi chú quan trọng về từng khách hàng.
    \item \textbf{Phân loại và phân khúc khách hàng:} Cho phép gắn thẻ, phân loại khách hàng (ví dụ: VIP, khách thường xuyên) để phục vụ các chiến dịch chăm sóc và marketing cá nhân hóa.
    \item \textbf{Quản lý chương trình khuyến mãi và voucher hiệu quả:} Tạo, cấu hình điều kiện áp dụng, theo dõi việc sử dụng và đánh giá hiệu quả của các chương trình khuyến mãi và mã voucher.
    \item \textbf{Thu thập và quản lý phản hồi của khách hàng:} Cung cấp kênh cho khách hàng gửi đánh giá và cho phép nhà hàng xem xét, phân tích các phản hồi này.
    \item \textbf{Tăng cường tương tác và chăm sóc khách hàng:} Cung cấp thông tin để nhân viên có thể phục vụ khách hàng tốt hơn và xây dựng mối quan hệ lâu dài.
    \item \textbf{Hỗ trợ các hoạt động marketing:} Cung cấp dữ liệu khách hàng cho việc tạo các chiến dịch email marketing hoặc các hoạt động quảng bá khác (có thể tích hợp với module marketing riêng).
\end{itemize}
Phạm vi của module bao gồm việc quản lý hồ sơ khách hàng từ khi tạo mới, cập nhật thông tin, theo dõi lịch sử, cho đến việc thiết kế và quản lý các chương trình khuyến mãi/voucher, cũng như quy trình thu thập và xem xét đánh giá từ khách hàng.

\subsubsubsection{Đối tượng Sử dụng Chính}
\label{sssec:md11_primary_users}
Các đối tượng người dùng chính tương tác với module này bao gồm:
\begin{itemize}
    \item \textbf{US-01 (Quản lý nhà hàng):} Là người dùng chính, chịu trách nhiệm tạo và quản lý các chương trình khuyến mãi, xem báo cáo hiệu quả, quản lý hồ sơ khách hàng VIP, và xem xét các đánh giá của khách hàng.
    \item \textbf{US-03 (Nhân viên lễ tân):} Thường xuyên tạo mới và cập nhật hồ sơ khách hàng, có thể xem lịch sử đặt chỗ của khách để hỗ trợ tốt hơn.
    \item \textbf{US-06 (Kế toán):} Có thể cần xem thông tin khách hàng liên quan đến hóa đơn hoặc các chương trình khách hàng thân thiết.
    \item \textbf{US-10 (Quản trị viên Hệ thống):} Có thể tham gia vào việc cấu hình ban đầu của module CRM, tạo các trường tùy chỉnh hoặc thiết lập các quy tắc tự động (nếu có).
    \item \textbf{US-08 (Khách hàng):} Là người cung cấp thông tin cá nhân, sử dụng voucher khi đặt chỗ online, và gửi đánh giá/review về dịch vụ.
    \item \textbf{US-02 (Nhân viên phục vụ) / US-05 (Nhân viên thu ngân):} Áp dụng các chương trình khuyến mãi hoặc mã voucher cho đơn hàng tại POS.
\end{itemize}

\subsubsubsection{Các Chức năng Chính}
\label{sssec:md11_key_functionalities}
Module MD-11 cung cấp một bộ các chức năng tập trung vào việc quản lý và tăng cường mối quan hệ với khách hàng:

\begin{itemize}
    \item \textbf{Quản lý Hồ sơ Khách hàng (UC-MD11-01 đến UC-MD11-07):}
    \begin{itemize}
        \item Tạo mới một hồ sơ khách hàng trong hệ thống CRM với các thông tin liên hệ cơ bản (UC-MD11-01).
        \item Xem danh sách tất cả các hồ sơ khách hàng với khả năng tìm kiếm và lọc (UC-MD11-02).
        \item Xem thông tin chi tiết đầy đủ của một hồ sơ khách hàng, bao gồm lịch sử giao dịch và tương tác (UC-MD11-03).
        \item Sửa đổi và cập nhật các thông tin trong một hồ sơ khách hàng đã tồn tại (UC-MD11-04).
        \item Xóa vĩnh viễn (nếu không có giao dịch) hoặc lưu trữ (ẩn đi) các hồ sơ khách hàng không còn phù hợp (UC-MD11-05).
        \item Gán các thẻ (tags) hoặc phân loại khách hàng (ví dụ: VIP, Thường xuyên) để phục vụ việc nhóm và phân tích (UC-MD11-06).
        \item Truy cập và xem lại toàn bộ lịch sử các hoạt động và giao dịch liên quan đến một khách hàng cụ thể (UC-MD11-07).
    \end{itemize}

    \item \textbf{Quản lý Chương trình Khuyến mãi và Voucher (UC-MD11-08 đến UC-MD11-13):}
    \begin{itemize}
        \item Định nghĩa một chương trình khuyến mãi mới hoặc một lô mã voucher mới, bao gồm loại hình và giá trị khuyến mãi (UC-MD11-08).
        \item Thiết lập các điều kiện và quy tắc chi tiết để chương trình/voucher có thể được áp dụng (ví dụ: thời gian hiệu lực, giá trị đơn hàng tối thiểu, sản phẩm áp dụng) (UC-MD11-09).
        \item Xem danh sách các mã voucher đã tạo, kiểm tra trạng thái, và có thể thực hiện xuất file hoặc vô hiệu hóa mã (UC-MD11-10).
        \item Nhân viên tại POS áp dụng một chương trình khuyến mãi hoặc nhập mã voucher hợp lệ vào đơn hàng của khách (UC-MD11-11).
        \item Khách hàng tự nhập và sử dụng mã voucher hợp lệ khi thực hiện đặt chỗ hoặc đặt món trước trực tuyến (UC-MD11-12).
        \item Xem báo cáo thống kê chi tiết về tình hình sử dụng và hiệu quả của các chương trình khuyến mãi hoặc voucher (UC-MD11-13).
    \end{itemize}

    \item \textbf{Quản lý Đánh giá và Phản hồi từ Khách hàng (UC-MD11-14, và sẽ có các UC tiếp theo như UC-MD11-15, UC-MD11-16):}
    \begin{itemize}
        \item Cho phép khách hàng gửi các ý kiến đánh giá, nhận xét, hoặc xếp hạng về dịch vụ của nhà hàng thông qua các kênh trực tuyến (UC-MD11-14).
        \item (Dự kiến) Cho phép Quản lý nhà hàng xem danh sách các đánh giá đã nhận được.
        \item (Dự kiến) Cho phép Quản lý nhà hàng xem chi tiết một đánh giá cụ thể và có thể thực hiện các hành động phản hồi.
    \end{itemize}
\end{itemize}

\subsubsubsection{Tóm tắt Luồng Hoạt động Tổng thể}
\label{sssec:md11_overall_workflow}
Luồng hoạt động trong module Quản lý Quan hệ Khách hàng (CRM) thường bao gồm các giai đoạn và quy trình sau:
\begin{enumerate}
    \item \textbf{Thu thập và Quản lý Thông tin Khách hàng:}
        \begin{itemize}
            \item Nhân viên Tạo mới Hồ sơ Khách hàng (CRM) (UC-MD11-01) khi có khách mới hoặc nhập dữ liệu.
            \item Nhân viên thường xuyên Xem Danh sách Hồ sơ Khách hàng (UC-MD11-02), Xem Chi tiết Hồ sơ Khách hàng (UC-MD11-03) để tra cứu, và Sửa Thông tin Hồ sơ Khách hàng (UC-MD11-04) khi cần cập nhật.
            \item Thực hiện Phân loại/Gắn thẻ Khách hàng (UC-MD11-06) để phục vụ các mục đích khác nhau.
            \item Khi cần, Xóa/Lưu trữ Hồ sơ Khách hàng (CRM) (UC-MD11-05).
            \item Xem Lịch sử Tương tác/Đặt chỗ của Khách hàng (UC-MD11-07) để hiểu rõ hơn về khách.
        \end{itemize}
    \item \textbf{Triển khai và Quản lý Chương trình Khuyến mãi/Voucher:}
        \begin{itemize}
            \item Quản lý Tạo mới Chương trình Khuyến mãi/Voucher (UC-MD11-08).
            \item Thiết lập Điều kiện Áp dụng Khuyến mãi/Voucher (UC-MD11-09) chi tiết cho từng chương trình.
            \item (Nếu là voucher) Quản lý Danh sách Mã Voucher (UC-MD11-10), có thể xuất file hoặc vô hiệu hóa mã.
            \item Nhân viên tại POS Áp dụng Khuyến mãi/Voucher vào Đơn hàng POS (UC-MD11-11) cho khách.
            \item Khách hàng Sử dụng Voucher khi Đặt chỗ Online (UC-MD11-12) trên website/app.
            \item Quản lý định kỳ Xem Báo cáo Hiệu quả Khuyến mãi/Voucher (UC-MD11-13) để đánh giá.
        \end{itemize}
    \item \textbf{Thu thập và Xử lý Phản hồi Khách hàng:}
        \begin{itemize}
            \item Khách hàng Gửi Đánh giá/Review sau Khi sử dụng Dịch vụ (UC-MD11-14).
            \item (Dự kiến) Quản lý xem xét các đánh giá này và có thể phản hồi hoặc thực hiện các hành động cải thiện dịch vụ.
        \end{itemize}
\end{enumerate}
Module MD-11 giúp nhà hàng không chỉ quản lý giao dịch mà còn xây dựng mối quan hệ ý nghĩa với khách hàng, từ đó tạo lợi thế cạnh tranh và phát triển bền vững.




% \subsection{Yêu cầu chất lương}

% % \subsubsection{Yêu cầu chức năng}

% Trong phạm vi đồ án này, chúng em sẽ tập trung vào các đối tượng chính sử dụng hệ thống, bao gồm Khách hàng, Nhân viên phục vụ, Nhân viên thu ngân, Nhân viên chăm sóc khách hàng, Nhân viên bếp, Quản lý chi nhánh và Quản lý tổng, nhằm đảm bảo hệ thống đáp ứng tốt nhu cầu vận hành và trải nghiệm của từng vai trò.

% \begin{figure}[H]
%     \centering
%     \includegraphics[width=15cm]{Images/nguoi-dung-he-thong.png}
%     \vspace{0.5cm}
%     \caption{Các đối tượng người dùng của hệ thống}
%     \label{fig:my_label}
% \end{figure}

% \textbf{Đối với Khách hàng}
% \begin{itemize}
%     \item Đăng ký, đăng nhập
%     \item Gửi yêu cầu đặt bàn trực tuyến, xem được tổng quan vị trí, trạng thái bàn thông qua các sơ đồ
%     \item Hủy yêu cầu đặt bàn cho đến khi trước thời gian hẹn 2 tiếng
%     \item Quét QR để truy cập vào menu và đặt món, không cần gọi nhân viên
%     \item Xem danh sách món ăn và đồ uống, kèm theo hình ảnh, mô tả và giá cả
%     \item Tìm kiếm món ăn theo tên hoặc danh mục, giá cả
%     \item Chọn món, số lượng và tùy chọn (size, topping, gia vị)
%     \item Gợi ý món ăn phù hợp dựa trên lịch sử đơn hàng và số liệu phân tích
%     \item Thêm món vào giỏ hàng
%     \item Xem lại giỏ hàng trước khi xác nhận đặt món
%     \item Đặt món nhiều lần trong cùng một lượt sử dụng tại nhà hàng, để có thể thay đổi món ăn trong suốt quá trình ăn
%     \item Gửi yêu cầu hủy đơn hoặc hủy/thay đổi các món cụ thể
%     \item Xem lại các đơn hàng đã đặt trước đó
%     \item Xem chi tiết các món đã đặt, tổng tiền và các khuyến mãi (nếu có)
%     \item Gửi yêu cầu thanh toán bằng cách quét mã QR, hoặc thao tác trên ứng dụng
%     \item Nhận thông báo về các khuyến mãi và ưu đãi
%     \item Gửi phản hồi về món ăn hoặc dịch vụ
%     \item Gửi khiếu nại nếu có sự cố
%     \item Trò chuyện trực tiếp với nhân viên chăm sóc khách hàng
% \end{itemize}

% \textbf{Đối với Phục vụ}
% \begin{itemize}
%     \item Xem danh sách trạng thái bàn
%     \item Hỗ trợ khách hàng đặt món, thanh toán
%     \item Xem danh sách đơn hàng của khách
%     \item Nhận thông báo khi món ăn đã sẵn sàng
%     \item Xử lý yêu cầu hủy đơn hoặc hủy/thay đổi món của khách
%     \item Chuyển vị trí đơn hàng sang bàn khác.
%     \item Đặt lại trạng thái bàn
% \end{itemize}

% \textbf{Đối với Thu ngân}
% \begin{itemize}
%     \item Nhận thông báo khi có yêu cầu thanh toán
%     \item Xem chi tiết đơn hàng và tổng hóa đơn
%     \item Áp dụng khuyến mãi và giảm giá khi thanh toán
%     \item Quản lý các phương thức thanh toán
%     \item Nhận thông báo về thanh toán thành công với phương thức quét QR, quẹt thẻ
%     \item Cập nhật trạng thái thanh toán với phương thức thanh toán tiền mặt
%     \item In hóa đơn cho khách hàng
%     \item Xử lý yêu cầu hoàn tiền (vấn đề phát sinh)
%     \item Tạo báo cáo doanh thu hàng ngày
% \end{itemize}

% \textbf{Đối với Nhân viên chăm sóc khách hàng}
% \begin{itemize}
%     \item Tiếp nhận và xử lý yêu cầu từ khách hàng
%     \item Theo dõi và quản lý khiếu nại \& phản hồi từ khách hàng
%     \item Trả lời tin nhắn trực tuyến với khách hàng
%     \item Cập nhật thông tin khách hàng
%     \item Theo dõi và nhắc nhở khách hàng về các chương trình ưu đãi qua tài khoản và email
% \end{itemize}

% \textbf{Đối với Nhân viên bếp}
% \begin{itemize}
%     \item Nhận đơn hàng từ hệ thống quản lý đơn hàng
%     \item Xem tất cả danh sách món ăn được sắp xếp theo thứ tự ưu tiên
%     \item Xem được các yêu cầu đặc biệt của từng món ăn cần làm
%     \item Cập nhật trạng thái chế biến của từng món ăn (chưa làm, đang chế biến, hoàn thành...)
% \end{itemize}

% \textbf{Đối với Quản lý chi nhánh}
% \begin{itemize}
%     \item Thêm/Chỉnh sửa sơ đồ nhà hàng
%     \item Xử lý, xác nhận các yêu cầu đặt bàn của khách hàng tại chi nhánh
%     \item Xem báo cáo kinh doanh và doanh thu của chi nhánh
%     \item Điều chỉnh và thiết lập các chương trình khuyến mãi, marketing chi nhánh
%     \item Theo dõi các phản hồi của khách hàng
%     \item Xem danh sách nhân viên
%     \item Phân chia công việc cho các tài khoản nhân viên
% \end{itemize}

% \textbf{Đối với Quản lý tổng}
% \begin{itemize}
%     \item Quản lý thông tin chi nhánh
%     \item Thêm/Xóa chi nhánh
%     \item Quản lý các tài khoản nhân viên và khách hàng
%     \item Xem báo cáo tổng quan tất cả các chi nhánh
%     \item Xem báo cáo chi tiết của một chi nhánh tổng
% \end{itemize}

% % User Story là một kỹ thuật phát triển phần mềm tập trung vào nhu cầu của người dùng trong quá trình sử dụng sản phẩm. Mục đích của User Story là giúp các nhà phát triển phần mềm hiểu rõ những tính năng cốt lõi của sản phẩm và xác định được các chức năng cần thiết để hiện thực đầu tiên của ứng dụng.\\

% % Để đưa ra danh sách User Story của ứng dụng, nhóm chúng em đã tiến hành thảo luận, nghiên cứu và phân tích yêu cầu của người dùng. Chúng tôi bắt đầu xây dựng hệ thống tuyển dụng bằng cách xác định các tính năng cơ bản cần phải có trong ứng dụng để đáp ứng nhu cầu và mong muốn của người dùng.\\

% % \textbf{Một số user story cơ bản như sau:}
% % \begin{itemize}
% %     \item Đối với Khách hàng
% %     \begin{enumerate}
% %         \item Là Khách Hàng, tôi muốn quét mã QR trên bàn để truy cập vào menu và đặt món.
% %         \item Là Khách Hàng, tôi muốn xem danh sách các món ăn và đồ uống, kèm theo hình ảnh, mô tả và giá cả.
% %         \item Là Khách Hàng, tôi muốn tìm kiếm món ăn theo tên hoặc danh mục.
% %         \item Là Khách Hàng, tôi muốn chọn món, số lượng và các tùy chọn (ví dụ: size, topping, gia vị).
% %         \item Là Khách Hàng, tôi muốn thêm các món đã chọn vào giỏ hàng.
% %         \item Là Khách Hàng, tôi muốn xem lại giỏ hàng trước khi xác nhận đặt món.
% %         \item Là Khách Hàng, tôi muốn đặt món nhiều lần trong cùng một lượt sử dụng tại nhà hàng.
% %         \item Là Khách Hàng, tôi muốn xem lại các order đã đặt trước đó (nếu đã đăng nhập).
% %         \item Là Khách Hàng, tôi muốn xem chi tiết các món đã đặt, tổng tiền và các khuyến mãi (nếu có).
% %         \item Là Khách Hàng, tôi muốn thanh toán bằng cách quét mã QR hoặc thanh toán tiền mặt.
% %         \item Là Khách Hàng, tôi muốn xem lại hóa đơn sau khi đã thanh toán.
% %         \item Là Khách Hàng, tôi muốn đăng ký tài khoản thành viên để tham gia chương trình khách hàng thân thiết.
% %         \item Là Khách Hàng, tôi muốn đăng nhập để xem lịch sử đặt món, nhận khuyến mãi và các ưu đãi khác.
% %     \end{enumerate}
% %     \item Đối với Nhân Viên Thu Ngân/Phục Vụ
% %     \begin{enumerate}
% %         \item Là Nhân Viên Thu Ngân/Phục Vụ, tôi muốn xem danh sách các order mới và đang chờ xử lý.
% %         \item Là Nhân Viên Thu Ngân/Phục Vụ, tôi muốn xem chi tiết các order của khách hàng.
% %         \item Là Nhân Viên Thu Ngân/Phục Vụ, tôi muốn gộp các order của một bàn thành một hóa đơn duy nhất.
% %         \item Là Nhân Viên Thu Ngân/Phục Vụ, tôi muốn xóa bỏ các order nếu cần.
% %         \item Là Nhân Viên Thu Ngân/Phục Vụ, tôi muốn xác nhận thanh toán bằng QR.
% %         \item Là Nhân Viên Thu Ngân/Phục Vụ, tôi muốn xác nhận thanh toán bằng tiền mặt (tự thao tác).
% %         \item Là Nhân Viên Thu Ngân/Phục Vụ, tôi muốn in hóa đơn cho khách hàng (có mã QR để thanh toán).
% %         \item Là Nhân Viên Thu Ngân/Phục Vụ, tôi muốn check-in (chấm công) đầu ca và check-out (chấm công) cuối ca.
% %         \item Là Nhân Viên Thu Ngân/Phục Vụ, tôi muốn chuyển order từ bàn này sang bàn khác (nếu khách hàng muốn đổi bàn).
% %     \end{enumerate}
% %     \item Đối với Quản Lý Chi Nhánh
% %     \begin{enumerate}
% %         \item Là Quản Lý Chi Nhánh, tôi muốn nhập kho, xem danh sách các nguyên liệu trong kho.
% %         \item Là Quản Lý Chi Nhánh, tôi muốn theo dõi số lượng còn lại của từng nguyên liệu.
% %         \item Là Quản Lý Chi Nhánh, tôi muốn xem báo cáo nhập/xuất kho hàng ngày.
% %         \item Là Quản Lý Chi Nhánh, tôi muốn xem danh sách nhân viên.
% %         \item Là Quản Lý Chi Nhánh, tôi muốn phân công ca làm cho nhân viên.
% %         \item Là Quản Lý Chi Nhánh, tôi muốn xem lịch sử check-in/check-out của nhân viên.
% %         \item Là Quản Lý Chi Nhánh, tôi muốn xem các báo cáo về doanh thu và số lượng món ăn bán được trong chi nhánh.
% %         \item Là Quản Lý Chi Nhánh, tôi muốn xem các báo cáo về kho nguyên liệu.
% %         \item Là Quản Lý Chi Nhánh, tôi muốn xem danh sách bàn đang có.
% %         \item Là Quản Lý Chi Nhánh, tôi muốn sắp xếp bàn cho khách hàng và quản lý bàn trống.
% %     \end{enumerate}
% %     \item Đối với Quản Lý Tổng
% %     \begin{enumerate}
% %         \item Là Quản Lý Tổng, tôi muốn xem tổng doanh thu của tất cả các chi nhánh.
% %         \item Là Quản Lý Tổng, tôi muốn xem chi tiết doanh thu của từng chi nhánh, theo ngày, tuần, tháng, năm.
% %         \item Là Quản Lý Tổng, tôi muốn xem các báo cáo tổng quan về hoạt động của nhà hàng.
% %         \item Là Quản Lý Tổng, tôi muốn xuất báo cáo tổng quan về doanh thu, chi phí và lợi nhuận.
% %         \item Là Quản Lý Tổng, tôi muốn xem danh sách tất cả các chi nhánh.
% %         \item Là Quản Lý Tổng, tôi muốn thêm, sửa hoặc xóa thông tin chi nhánh.
% %         \item Là Quản Lý Tổng, tôi muốn xem các báo cáo tổng quan từ các chi nhánh.
% %         \item Là Quản Lý Tổng, tôi muốn thêm, chỉnh sửa hoặc xóa các món ăn và đồ uống trong thực đơn.
% %         \item Là Quản Lý Tổng, tôi muốn thay đổi giá cả, mô tả và hình ảnh của món ăn.
% %     \end{enumerate}
% %     \item Các Tính Năng Mở Rộng (Optional)
% %     \begin{enumerate}
% %         \item Là Khách Hàng, tôi muốn hệ thống gợi ý món ăn dựa trên lịch sử đặt món của mình.
% %         \item Là Quản Lý (Tổng/Chi Nhánh), tôi muốn tạo và quản lý các chương trình khuyến mãi, giảm giá.
% %         \item Là Khách Hàng, tôi muốn đặt bàn trước thông qua ứng dụng.
% %         \item Là Khách Hàng, tôi muốn đánh giá chất lượng món ăn và dịch vụ của nhà hàng.
% %         \item Là Quản Lý (Tổng/Chi Nhánh), tôi muốn hệ thống tự động xuất dữ liệu sang hệ thống kế toán.
% %     \end{enumerate}
% % \end{itemize}

% % \textbf{Một số usecase diagram cơ bản như sau:}

% % \begin{figure}[H]
% %     \centering
% %     \includegraphics[width=15cm]{Images/us-dat-mon.png}
% %     \vspace{0.5cm}
% %     \caption{Các use case liên quan đến đặt món}
% %     \label{fig:my_label}
% % \end{figure}

% % \begin{figure}[H]
% %     \centering
% %     \includegraphics[width=15cm]{Images/us-tai-khoan.png}
% %     \vspace{0.5cm}
% %     \caption{Các use case liên quan đến quản lý tài khoản}
% %     \label{fig:my_label}
% % \end{figure}

% % \begin{figure}[H]
% %     \centering
% %     \includegraphics[width=15cm]{Images/us-quan-ly.png}
% %     \vspace{0.5cm}
% %     \caption{Các use case liên quan đến quản lý}
% %     \label{fig:my_label}
% % \end{figure}

% % \begin{enumerate}[(a)]
% %     \item Khách hàng
% %     \begin{itemize}
% %         \item Quét mã QR
% %         \begin{itemize}
% %             \item Khách hàng quét mã QR trên bàn để truy cập vào menu và đặt món.
% %         \end{itemize}
% %         \item Xem menu
% %         \begin{itemize}
% %             \item Xem danh sách các món ăn và đồ uống, kèm theo hình ảnh, mô tả, giá cả.
% %             \item Tìm kiếm món ăn theo tên hoặc danh mục.
% %         \end{itemize}
% %         \item Đặt món
% %         \begin{itemize}
% %             \item Chọn món, số lượng, tùy chọn (ví dụ: size, topping, gia vị)
% %             \item Thêm món vào giỏ hàng.
% %             \item Xem lại giỏ hàng trước khi xác nhận đặt món.
% %             \item Đặt món nhiều lần trong cùng một lượt sử dụng.
% %         \end{itemize}
% %         \item Xem lịch sử đặt món
% %         \begin{itemize}
% %             \item Xem lại các order đã đặt trước đó (nếu đã đăng nhập).
% %         \end{itemize}
% %         \item Xem hóa đơn
% %         \begin{itemize}
% %             \item Xem chi tiết các món đã đặt, tổng tiền, các khuyến mãi (nếu có).
% %         \end{itemize}
% %         \item Thanh toán
% %         \begin{itemize}
% %             \item Thanh toán bằng cách quét mã QR hoặc thanh toán tiền mặt.
% %             \item Xem lại hóa đơn sau khi đã thanh toán.
% %         \end{itemize}
% %         \item Đăng ký/Đăng nhập (Tùy chọn)
% %         \begin{itemize}
% %             \item Đăng ký tài khoản thành viên để tham gia chương trình khách hàng thân thiết.
% %             \item Đăng nhập để xem lịch sử đặt món, nhận khuyến mãi và các ưu đãi khác.
% %         \end{itemize}
% %     \end{itemize}
% %     \item Nhân viên thu ngân/phục vụ
% %     \begin{itemize}
% %         \item Quản lý order
% %         \begin{itemize}
% %             \item Xem danh sách các order mới và đang chờ xử lý.
% %             \item Xem chi tiết các order của khách hàng.
% %         \end{itemize}
% %         \item Gộp hóa đơn
% %         \begin{itemize}
% %             \item Gộp các order của một bàn thành một hóa đơn duy nhất.
% %             \item Xóa bỏ các order (nếu cần)
% %         \end{itemize}
% %         \item Xác nhận thanh toán
% %         \begin{itemize}
% %             \item Xác nhận thanh toán bằng QR.
% %             \item Xác nhận thanh toán bằng tiền mặt (tự thao tác).
% %         \end{itemize}
% %         \item In hóa đơn
% %         \begin{itemize}
% %             \item In hóa đơn cho khách hàng (có mã QR để thanh toán).
% %         \end{itemize}
% %         \item Check-in/Check-out
% %         \begin{itemize}
% %             \item Chấm công đầu ca và cuối ca.
% %         \end{itemize}
% %         \item Chuyển Order
% %         \begin{itemize}
% %             \item Có thể chuyển order từ bàn này sang bàn khác (nếu khách hàng muốn đổi bàn)
% %         \end{itemize}
% %     \end{itemize}
% %     \item Chức năng cho quản lý chi nhánh
% %     \begin{itemize}
% %         \item Quản lý kho nguyên liệu
% %         \begin{itemize}
% %             \item Nhập kho, xem danh sách các nguyên liệu trong kho.
% %             \item Theo dõi số lượng còn lại của từng nguyên liệu.
% %             \item Báo cáo nhập/xuất kho hàng ngày.
% %         \end{itemize}
% %         \item Quản lý nhân viên
% %         \begin{itemize}
% %             \item Xem danh sách nhân viên.
% %             \item Phân công ca làm.
% %             \item Xem lịch sử check-in/check-out của nhân viên.
% %         \end{itemize}
% %         \item Xem báo cáo
% %         \begin{itemize}
% %             \item Xem các báo cáo về doanh thu, số lượng món ăn bán được trong chi nhánh.
% %             \item Xem các báo cáo về kho nguyên liệu.
% %         \end{itemize}
% %         \item Quản lý bàn
% %         \begin{itemize}
% %             \item Xem danh sách bàn đang có.
% %             \item Có thể sắp xếp bàn cho khách hàng, quản lý bàn trống.
% %         \end{itemize}
% %     \end{itemize}
% %     \item Chức năng cho quản lý tổng
% %     \begin{itemize}
% %         \item Quản lý doanh thu
% %         \begin{itemize}
% %             \item Xem tổng doanh thu của tất cả các chi nhánh.
% %             \item Xem chi tiết doanh thu của từng chi nhánh, theo ngày, tuần, tháng, năm.
% %         \end{itemize}
% %         \item Xem báo cáo
% %         \begin{itemize}
% %             \item Xem các báo cáo tổng quan về hoạt động của nhà hàng.
% %             \item Xuất báo cáo tổng quan về doanh thu, chi phí, lợi nhuận.
% %         \end{itemize}
% %         \item Quản lý chi nhánh
% %         \begin{itemize}
% %             \item Xem danh sách tất cả các chi nhánh.
% %             \item Thêm/sửa/xóa thông tin chi nhánh.
% %             \item Xem các báo cáo tổng quan từ các chi nhánh.
% %         \end{itemize}
% %         \item Quản lý thực đơn
% %         \begin{itemize}
% %             \item Thêm, chỉnh sửa, xóa các món ăn và đồ uống trong thực đơn.
% %             \item Thay đổi giá cả, mô tả, hình ảnh của món ăn.
% %         \end{itemize}
% %     \end{itemize}
% %     \item Mở rộng (có thể làm nếu kịp thời gian)
% %     \begin{itemize}
% %         \item Hệ thống gợi ý món ăn: Dựa trên lịch sử đặt món của khách hàng, hệ thống có thể gợi ý các món ăn phù hợp.
% %         \item Hệ thống quản lý khuyến mãi: Cho phép tạo và quản lý các chương trình khuyến mãi, giảm giá.
% %         \item Hệ thống đặt bàn: Cho phép khách hàng đặt bàn trước thông qua ứng dụng.
% %         \item Hệ thống quản lý đánh giá: Cho phép khách hàng đánh giá chất lượng món ăn và dịch vụ của nhà hàng.
% %         \item Tích hợp với hệ thống kế toán: Tự động xuất dữ liệu sang hệ thống kế toán.
% %     \end{itemize}
% % \end{enumerate}

% \subsubsection{Yêu cầu phi chức năng}
% \begin{itemize}
%     \item Bảo mật:
%     \begin{itemize}
%         \item Dữ liệu người dùng và dữ liệu giao dịch phải được bảo vệ.
%         \item Hệ thống sử dụng HTTPS để đảm bảo an toàn cho quá trình truyền dữ liệu.
%     \end{itemize}

%     \item Hiệu năng:
%     \begin{itemize}
%         \item Hệ thống phải có tốc độ xử lý nhanh và ổn định.
%         \item Khả năng đáp ứng nhanh chóng khi nhiều người dùng truy cập cùng một lúc.    
%     \end{itemize}

%     \item Tính khả dụng:
%     \begin{itemize}
%         \item Hệ thống hoạt động ổn định và có thời gian uptime cao.
%         \item Giao diện thân thiện và dễ sử dụng trên cả web và mobile.
%     \end{itemize}

%     \item Khả năng mở rộng:
%     \begin{itemize}
%         \item Hệ thống có thể mở rộng để hỗ trợ thêm nhiều chi nhánh và người dùng.
%         \item Dễ dàng thêm các tính năng mới khi cần thiết.
%     \end{itemize}

% \end{itemize}











