\section{ĐÁNH GIÁ}
\subsection{Đánh giá hệ thống }

Đánh giá hệ thống giúp đảm bảo rằng ứng dụng hoạt động mượt mà và hiệu quả, không gặp vấn đề về hiệu suất hay tốc độ phản hồi. Điều này làm tăng trải nghiệm người dùng và duy trì sự hài lòng. Bên cạnh đó, việc này sẽ giúp đảm bảo rằng các chức năng và tính năng của ứng dụng đáp ứng đúng nhu cầu của người dùng. Việc này giữ cho ứng dụng hữu ích và thú vị cho người sử dụng.	

\subsubsection{Về mặt kỹ thuật}

\begin{itemize}
    \item \textbf{Ngôn ngữ lập trình và Framework}
    
 
    \item \textbf{Cơ sở dữ liệu} 
    
    
    \item \textbf{Giao diện người dùng (UI) và React}

    
    \item \textbf{Khả năng mở rộng và hiệu suất}
    
    
    \item \textbf{Bảo mật}

    
    \item \textbf{Hiệu suất API}
    
    
    \item \textbf{Tính dễ mở rộng}

\end{itemize}

\subsection{Đánh giá hiệu suất }
\subsubsection{Độ chính xác}
Độ chính xác là một yếu tố quan trọng trong nhiều hệ thống và ngữ cảnh khác nhau. Tùy thuộc vào loại hệ thống, độ chính xác có thể đóng vai trò quyết định đến hiệu suất và độ tin cậy của hệ thống đó. \\

Mức độ chính xác cao giúp đảm bảo rằng thông tin được xử lý đúng và đáng tin cậy.  Các quyết định dựa trên dữ liệu chính xác thường dẫn đến hiệu suất và hiệu quả cao hơn.\\

Tuy nhiên, cũng cần lưu ý rằng đôi khi có sự đánh đổi giữa độ chính xác và các yếu tố khác như tốc độ xử lý, chi phí, và nguồn lực. Trong một số trường hợp, việc chấp nhận một mức độ chính xác thấp hơn có thể làm tăng tính khả dụng và giảm chi phí. Do đó, quyết định về độ chính xác thường cần phải cân nhắc kỹ lưỡng dựa trên yêu cầu cụ thể của ứng dụng và ngữ cảnh sử dụng. \\

...
\subsubsection{Độ trễ}
Độ trễ trong hệ thống là một yếu tố quan trọng nhất là khi xây dựng ứng dụng, đặc biệt là những ứng dụng tương tác nhanh và yêu cầu phản hồi ngay lập tức từ người dùng. Độ trễ được định nghĩa là thời gian chờ đợi giữa hành động của người dùng và phản hồi của ứng dụng đối với hành động đó. Điều này bao gồm cả quá trình xử lý dữ liệu và thời gian tải của ứng dụng. \\

Ở mức độ cao, độ trễ có thể tạo ra trải nghiệm người dùng kém chất lượng, gây khó chịu và giảm hiệu suất sử dụng ứng dụng. Đặc biệt, trong các ứng dụng web và di động, độ trễ có thể ảnh hưởng đến thời gian tải của trang và ảnh hưởng đến sự liền mạch của trải nghiệm người dùng.\\

Do đó, đây là 1 trong những tiêu chí quan trọng nhất cần xem xét khi xây dựng một hệ thống. \\

...
