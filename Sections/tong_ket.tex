\section{TỔNG KẾT}

\subsection{Những gì đã thực hiện và đạt được}

Trong giai đoạn đầu tiên của đồ án – giai đoạn phân tích và thiết kế hệ thống – nhóm đã hoàn thiện đầy đủ các nội dung quan trọng, bao gồm:

\begin{itemize}
    \item \textbf{Phân tích yêu cầu hệ thống}: Xác định và đặc tả rõ các yêu cầu chức năng, phi chức năng; xây dựng các sơ đồ use-case, hành trình khách hàng và module chức năng cốt lõi.
    \item \textbf{Nghiên cứu hệ thống liên quan}: Phân tích nhiều hệ thống nhà hàng nổi bật để tham khảo thiết kế chức năng phù hợp cho hệ thống Menu+.
    \item \textbf{Thiết kế tổng thể hệ thống}: Bao gồm thiết kế kiến trúc (MVC), sơ đồ EERD, lược đồ quan hệ logic và vật lý cho cơ sở dữ liệu PostgreSQL.
    \item \textbf{Thiết kế giao diện và API}: Phác thảo sitemap, wireframe, screenflow và thiết kế OpenAPI cho các chức năng chính.
    \item \textbf{Xác định công nghệ}: Lựa chọn stack công nghệ sử dụng trong giai đoạn triển khai gồm ReactJS, Spring Boot, Redis, Docker, GitHub Actions,...
\end{itemize}

Các kết quả đạt được trong giai đoạn này là nền tảng quan trọng giúp đảm bảo sự mạch lạc và hiệu quả cho các bước phát triển tiếp theo của hệ thống.

\subsection{Định hướng và triển khai trong giai đoạn tiếp theo}

Giai đoạn tiếp theo sẽ tập trung vào hiện thực hóa hệ thống thông qua lập trình, kiểm thử và triển khai. Cụ thể, nhóm sẽ:

\begin{itemize}
    \item \textbf{Phát triển chức năng hệ thống}: Xây dựng API và giao diện người dùng cho các module: quản lý thực đơn, POS (tại chỗ, mang về, giao hàng), đặt bàn & món trước, quản lý lịch làm việc, tích hợp bếp (KDS), báo cáo vận hành và gọi xác nhận bằng bot.
    
    \item \textbf{Kiểm thử và tích hợp hệ thống}: Kiểm thử chức năng từng phần và toàn bộ quy trình nghiệp vụ (end-to-end); phân tích hiệu năng API, tốc độ phản hồi và tối ưu giao diện.

    \item \textbf{Triển khai và CI/CD}: Thiết lập quy trình triển khai bằng Docker và GitHub Actions để tự động hóa kiểm thử, build và deploy. Hệ thống sẽ được triển khai trên môi trường staging để đánh giá trước khi lên production. Toàn bộ tài liệu hướng dẫn triển khai, sử dụng và đào tạo cũng sẽ được chuẩn bị.
\end{itemize}

Giai đoạn này là bước kiểm chứng tính khả thi và hiệu quả của thiết kế, đồng thời đưa hệ thống tiến gần đến việc sẵn sàng vận hành thực tế.
