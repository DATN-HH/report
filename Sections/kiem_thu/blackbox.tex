\subsubsection{Kiểm thử hộp đen}
Kiểm thử hộp đen (Black box testing) là phương pháp kiểm thử phần mềm mà việc kiểm tra các chức năng của một ứng dụng không cần quan tâm vào cấu trúc nội bộ. Mục đích chính của kiểm thử hộp đen chỉ là để xem phần mềm có hoạt động như dự kiến và liệu nó có đáp ứng được sự mong đợi của người dùng hay không.\\

Kỹ thuật kiểm thử hộp đen thường sử dụng:

\begin{itemize}
    \item Kỹ thuật Phân vùng tương đương (Equivalence Class Partitioning Technique)

    Phân vùng tương đương là kỹ thuật chia đầu vào thành những nhóm tương đương nhau. Nếu một giá trị trong nhóm hoạt động đúng thì tất cả các giá trị trong nhóm đó cũng hoạt động đúng và ngược lại. 
    \item Kỹ thuật Phân tích giá trị biên (Boundary Value Analysis Technique)

    Phân tích giá trị biên là phương pháp kiểm thử các giá trị ở vùng biên của dữ liệu đầu vào. Thay vì phải kiểm thử toàn bộ dữ liệu vào và ra, ta có thể kiểm thử một số trường hợp mà vẫn đảm bảo hệ thống hoạt động tốt.
    \item Kỹ thuật Bảng quyết định (Decision Table Technique)

    Kỹ thuật kiểm thử giúp đánh giá dữ liệu đầu ra khi kết hợp các dữ liệu đầu vào với nhau.
    \item Kỹ thuật Kiểm thử trường hợp sử dụng (Use-case Testing Technique)

    Kỹ thuật kiểm thử dựa vào use-case. Use case mô tả sự tương tác giữa phần mềm và tác nhân khác như người dùng, hệ thống khác,… 
\end{itemize}

Kế hoạch kiểm thử: Nhóm chỉ tiến hành kiểm thử các tính năng quan trọng của hệ thống như sau:

\begin{itemize}
    \
\end{itemize}

Đối với trường dữ liệu đơn như tính năng comment, nhóm lựa chọn kiểm thử với kỹ thuật phân tích giá trị biên. Đối với kiểm thử hành vi của hệ thống với nhiều trường dữ liệu như tính năng đăng CV, nhận CV, đăng bài viết, nhóm lựa chọn kỹ thuật kiểm thử bảng quyết định hoặc kiểm thử dựa vào use-case.\\

Kết quả kiểm thử: 


Ngoài ra, nhóm có kiểm thử các non-requirements và thành công như kỳ vọng:

\begin{itemize}
    \item 
    \item 

\end{itemize}