\subsection{Module MD-03: Quản lý Đặt chỗ \& Đặt món trước}

\subsubsection{Use Case UC-MD03-01: Tìm kiếm Khung giờ/Bàn trống Online}

\begin{longtable}{|m{4cm}|p{11cm}|}
\caption{Đặc tả Use Case UC-MD03-01: Tìm kiếm Khung giờ/Bàn trống Online} \label{tab:uc_md03_01_revised_v3} \\
\hline
\multicolumn{2}{|c|}{\textbf{2.1. Tóm tắt (Summary)}} \\
\hline
\textbf{Mục} & \textbf{Nội dung} \\
\hline
\endhead % Header cho các trang tiếp theo
\hline
\endfoot % Footer cho bảng
\hline
\endlastfoot % Footer cho trang cuối cùng
Use Case Name & Tìm kiếm Khung giờ/Bàn trống Online \\
\hline
Use Case ID & UC-MD03-01 \\
\hline
Use Case Description & Cho phép Khách hàng (US-08) truy cập trang đặt chỗ của nhà hàng, nhập các tiêu chí như ngày, giờ mong muốn và số lượng người để hệ thống tìm kiếm và hiển thị các lựa chọn đặt bàn (khung giờ/bàn) còn trống. \\
\hline
Actor & US-08 (Khách hàng) \\
\hline
Priority & Must Have \\
\hline
Trigger & Khách hàng muốn đặt bàn tại nhà hàng qua kênh trực tuyến và cần tìm các lựa chọn phù hợp. \\
\hline
Pre-Condition & - Trang web/ứng dụng của nhà hàng có chức năng đặt chỗ online đang hoạt động. \newline - Các tham số cấu hình đặt chỗ (giờ hoạt động, giới hạn khách...) đã được thiết lập (FR-MD03-15). \\
\hline
Post-Condition & - Hệ thống hiển thị cho khách hàng danh sách các khung giờ còn trống phù hợp với tiêu chí tìm kiếm. \newline - HOẶC nếu được cấu hình cho phép chọn bàn, hệ thống hiển thị sơ đồ tầng với các bàn trống phù hợp. \newline - HOẶC hệ thống thông báo không tìm thấy lựa chọn phù hợp. \newline - Khách hàng sẵn sàng để chọn một khung giờ/bàn cụ thể (nếu có) hoặc điều chỉnh tiêu chí tìm kiếm. \\
\hline
\multicolumn{2}{|c|}{\textbf{2.2. Luồng thực thi (Flow)}} \\
\hline
\textbf{Mục} & \textbf{Nội dung} \\
\hline
Basic Flow & 1. Khách hàng (US-08) truy cập trang/màn hình đặt chỗ của nhà hàng. \newline 2. US-08 nhập (hoặc chọn từ lịch) Ngày muốn đặt bàn. \newline 3. US-08 nhập (hoặc chọn từ danh sách) Số lượng người. \newline 4. US-08 nhập (hoặc chọn từ danh sách) Giờ mong muốn đặt bàn (có thể là một khoảng giờ ưu tiên). \newline 5. US-08 nhấn nút "Tìm kiếm" / "Kiểm tra Tính khả dụng" / "Find a Table". \newline 6. Hệ thống (System) dựa trên thông tin đầu vào và dữ liệu đặt chỗ hiện tại, tìm kiếm các khung giờ/bàn còn trống phù hợp (tuân thủ BR-UC3.1-1, BR-UC3.1-2, BR-UC3.1-3). \newline 7. Hệ thống hiển thị kết quả cho US-08: \newline    a. Danh sách các khung giờ còn trống xung quanh giờ khách chọn. \newline    b. HOẶC (nếu cấu hình cho phép chọn bàn và có bàn phù hợp) chuyển sang giao diện chọn bàn cụ thể (UC-MD03-02). \\
\hline
Alternative Flow & \textbf{4a. Khách hàng không chọn giờ cụ thể:} \newline    1. US-08 chỉ chọn Ngày và Số lượng người, sau đó nhấn "Tìm kiếm". \newline    2. Hệ thống hiển thị tất cả các khung giờ còn trống trong ngày đó phù hợp với số lượng người. \newline \textbf{7c. Không tìm thấy lựa chọn phù hợp:} \newline    1. Hệ thống không tìm thấy khung giờ/bàn nào còn trống khớp với yêu cầu. \newline    2. Hệ thống hiển thị thông báo "Rất tiếc, không có bàn trống phù hợp. Vui lòng thử chọn ngày/giờ khác hoặc số lượng người khác." hoặc đề xuất các khung giờ/ngày gần nhất còn trống. \\
\hline
Exception Flow & \textbf{2a/3a/4a. Nhập liệu không hợp lệ:} \newline    1. US-08 nhập ngày quá khứ, số lượng người không hợp lệ (ví dụ: chữ, số âm, quá lớn/nhỏ so với cấu hình). \newline    2. Hệ thống hiển thị thông báo lỗi yêu cầu nhập lại. Use Case quay lại bước tương ứng. \newline \textbf{6a. Lỗi hệ thống khi tìm kiếm:} \newline    1. Hệ thống gặp lỗi kỹ thuật khi truy vấn dữ liệu hoặc xử lý logic tìm kiếm. \newline    2. Hệ thống hiển thị thông báo lỗi chung. \\
\hline
\multicolumn{2}{|c|}{\textbf{2.3. Thông tin bổ sung (Additional Information)}} \\
\hline
\textbf{Mục} & \textbf{Nội dung} \\
\hline
Business Rule & - \textbf{BR-UC3.1-1 (System):} Hệ thống chỉ hiển thị các ngày trong tương lai và trong khoảng thời gian cho phép đặt trước (cấu hình ở FR-MD03-15). \newline - \textbf{BR-UC3.1-2 (System):} Hệ thống chỉ xem xét các khung giờ nằm trong giờ hoạt động của nhà hàng và còn đủ chỗ cho số lượng người khách yêu cầu (cấu hình ở FR-MD03-15). \newline - \textbf{BR-UC3.1-3 (System):} Hệ thống phải kiểm tra tình trạng bàn thực tế (từ các đặt chỗ đã xác nhận) để xác định tính khả dụng. \\
\hline
Non-Functional Requirement & - \textbf{NFR-UC3.1-1 (Usability):} Giao diện tìm kiếm phải trực quan, dễ dàng chọn ngày, giờ, số người. \newline - \textbf{NFR-UC3.1-2 (Performance):} Thời gian hệ thống trả về kết quả tìm kiếm phải nhanh (dưới 2-3 giây). \newline - \textbf{NFR-UC3.1-3 (Accuracy):} Kết quả tìm kiếm (khung giờ/bàn trống) phải chính xác. \\
\hline
\end{longtable}

\subsubsection{Use Case UC-MD03-02: (Tùy chọn) Chọn Bàn cụ thể từ Sơ đồ tầng Online}
% (Giữ nguyên nội dung chi tiết như UC-MD03-03 version cũ, chỉ đổi ID)
\begin{longtable}{|m{4cm}|p{11cm}|}
\caption{Đặc tả Use Case UC-MD03-02: (Tùy chọn) Chọn Bàn cụ thể từ Sơ đồ tầng Online} \label{tab:uc_md03_02_revised_v3} \\
\hline
\multicolumn{2}{|c|}{\textbf{2.1. Tóm tắt (Summary)}} \\
\hline
\textbf{Mục} & \textbf{Nội dung} \\
\hline
\endhead % Header cho các trang tiếp theo
\hline
\endfoot % Footer cho bảng
\hline
\endlastfoot % Footer cho trang cuối cùng
Use Case Name & (Tùy chọn) Chọn Bàn cụ thể từ Sơ đồ tầng Online \\
\hline
Use Case ID & UC-MD03-02 \\
\hline
Use Case Description & Nếu được Quản lý nhà hàng cấu hình và sau khi khách hàng đã tìm được khung giờ phù hợp (UC-MD03-01), cho phép Khách hàng (US-08) xem sơ đồ mặt bằng (floor plan) của nhà hàng và chọn một bàn trống cụ thể cho lượt đặt chỗ của mình. \\
\hline
Actor & US-08 (Khách hàng) \\
\hline
Priority & Low / Nice to Have \\
\hline
Trigger & - Hệ thống chuyển hướng đến giao diện chọn bàn sau khi khách hàng hoàn thành UC-MD03-01 và hệ thống được cấu hình cho phép chọn bàn. \\
\hline
Pre-Condition & - Khách hàng đã hoàn thành UC-MD03-01 và hệ thống xác định có khung giờ/bàn trống. \newline - Quản lý nhà hàng (US-01) đã kích hoạt chức năng cho phép khách chọn bàn và đã thiết lập sơ đồ tầng. \\
\hline
Post-Condition & - Hệ thống ghi nhận bàn cụ thể mà khách hàng đã chọn. \newline - Bàn đó được tạm giữ cho khách hàng trong một khoảng thời gian để hoàn tất đặt chỗ. \\
\hline
\multicolumn{2}{|c|}{\textbf{2.2. Luồng thực thi (Flow)}} \\
\hline
\textbf{Mục} & \textbf{Nội dung} \\
\hline
Basic Flow & 1. Tiếp nối từ UC-MD03-01, hệ thống hiển thị sơ đồ mặt bằng (Floor Plan) cho khách hàng. \newline 2. Các bàn phù hợp với số lượng người và còn trống vào ngày giờ đã chọn được làm nổi bật. \newline 3. US-08 xem xét sơ đồ và nhấp vào một bàn trống cụ thể muốn chọn. \newline 4. Hệ thống xác nhận lựa chọn bàn. \newline 5. Hệ thống có thể hiển thị thông tin tóm tắt bao gồm bàn đã chọn và chuyển sang bước tiếp theo (chọn món hoặc xác nhận đặt chỗ). \\
\hline
Alternative Flow & \textbf{1a. Không chọn bàn cụ thể / Bỏ qua:} \newline    1. Nếu khách hàng không muốn chọn bàn cụ thể hoặc giao diện cho phép bỏ qua. \newline    2. US-08 chọn nút "Bỏ qua chọn bàn" / "Để nhà hàng xếp bàn". \newline    3. Hệ thống sẽ tự động gán một bàn phù hợp sau này. Use Case kết thúc. \\
\hline
Exception Flow & \textbf{3a. Bàn vừa chọn đã bị người khác đặt:} \newline    1. Trong lúc khách hàng xem, bàn đó bị người khác đặt. \newline    2. Hệ thống báo lỗi "Bàn này vừa có người khác đặt. Vui lòng chọn bàn khác." \newline    3. Use Case quay lại bước 2. \\
\hline
\multicolumn{2}{|c|}{\textbf{2.3. Thông tin bổ sung (Additional Information)}} \\
\hline
\textbf{Mục} & \textbf{Nội dung} \\
\hline
Business Rule & - \textbf{BR-UC3.2-1 (V3):} Chức năng này phải được kích hoạt trong cấu hình (FR-MD03-15). \newline - \textbf{BR-UC3.2-2 (V3):} Sơ đồ tầng phải chính xác. \newline - \textbf{BR-UC3.2-3 (V3):} Chỉ hiển thị bàn trống, phù hợp sức chứa. \newline - \textbf{BR-UC3.2-4 (V3):} Hệ thống nên tạm giữ bàn đã chọn. \\
\hline
Non-Functional Requirement & - \textbf{NFR-UC3.2-1 (V3) (Usability):} Sơ đồ tầng rõ ràng, trạng thái bàn dễ phân biệt. \newline - \textbf{NFR-UC3.2-2 (V3) (Performance):} Tải sơ đồ nhanh. \newline - \textbf{NFR-UC3.2-3 (V3) (Accuracy):} Trạng thái bàn chính xác. \\
\hline
\end{longtable}

\subsubsection{Use Case UC-MD03-03: Chọn Món ăn Đặt trước từ Thực đơn Online}
% (Gộp UC-MD03-04 cũ và UC-MD03-05 cũ)
\begin{longtable}{|m{4cm}|p{11cm}|}
\caption{Đặc tả Use Case UC-MD03-03: Chọn Món ăn Đặt trước từ Thực đơn Online} \label{tab:uc_md03_03_revised_v3} \\
\hline
\multicolumn{2}{|c|}{\textbf{2.1. Tóm tắt (Summary)}} \\
\hline
\textbf{Mục} & \textbf{Nội dung} \\
\hline
\endhead % Header cho các trang tiếp theo
\hline
\endfoot % Footer cho bảng
\hline
\endlastfoot % Footer cho trang cuối cùng
Use Case Name & Chọn Món ăn Đặt trước từ Thực đơn Online \\
\hline
Use Case ID & UC-MD03-03 \\
\hline
Use Case Description & Cho phép Khách hàng (US-08) duyệt thực đơn trực tuyến của nhà hàng, lựa chọn các món ăn/đồ uống mong muốn cùng số lượng và các tùy chọn biến thể (nếu có), sau đó thêm chúng vào giỏ hàng đặt trước cho lượt đặt bàn của mình. \\
\hline
Actor & US-08 (Khách hàng) \\
\hline
Priority & Must Have (nếu có chức năng đặt món trước) \\
\hline
Trigger & Khách hàng đã chọn xong thông tin đặt bàn (và có thể đã chọn bàn) và muốn đặt trước một số món ăn để tiết kiệm thời gian khi đến nhà hàng. \\
\hline
Pre-Condition & - Khách hàng đang trong luồng đặt chỗ online (đã qua UC-MD03-01, có thể cả UC-MD03-02). \newline - Chức năng đặt món trước được kích hoạt. \newline - Thực đơn online đã được Quản lý nhà hàng cấu hình (sản phẩm, danh mục, giá, ảnh...). \\
\hline
Post-Condition & - Một "giỏ hàng" hoặc "danh sách đặt trước" tạm thời được tạo, chứa các món ăn/đồ uống, số lượng và biến thể đã chọn. \newline - Khách hàng có thể xem tổng giá trị tạm tính của các món trong giỏ hàng này. \newline - Khách hàng sẵn sàng để chuyển sang bước xác nhận và thanh toán đặt cọc. \\
\hline
\multicolumn{2}{|c|}{\textbf{2.2. Luồng thực thi (Flow)}} \\
\hline
\textbf{Mục} & \textbf{Nội dung} \\
\hline
Basic Flow & 1. Sau khi hoàn thành việc chọn thông tin đặt bàn (UC-MD03-01/UC-MD03-02), hệ thống hiển thị tùy chọn "Đặt món trước" / "Chọn món từ thực đơn". \newline 2. US-08 chọn tùy chọn này. \newline 3. Hệ thống hiển thị giao diện thực đơn online, được sắp xếp theo danh mục (ví dụ: Khai vị, Món chính...). Mỗi món hiển thị tên, giá, hình ảnh (nếu có), mô tả (nếu có). \newline 4. US-08 duyệt qua các danh mục và món ăn. \newline 5. Khi tìm thấy món muốn đặt, US-08 nhấp vào nút "Thêm" / "+" hoặc tương tự. \newline 6. \textbf{Nếu món không có biến thể:} Hệ thống thêm món vào giỏ hàng đặt trước với số lượng mặc định là 1. \newline 7. \textbf{Nếu món có biến thể (ví dụ: size, độ cay):} \newline    a. Hệ thống hiển thị popup/dialog yêu cầu US-08 chọn các giá trị thuộc tính (ví dụ: Size L, Ít cay). \newline    b. US-08 chọn các giá trị và xác nhận. \newline    c. Hệ thống thêm biến thể cụ thể đó vào giỏ hàng với số lượng 1. \newline 8. US-08 có thể điều chỉnh số lượng của từng món trong giỏ hàng (tăng/giảm/nhập số). \newline 9. US-08 lặp lại bước 4-8 để chọn thêm các món khác. \newline 10. Giao diện hiển thị giỏ hàng đặt trước được cập nhật liên tục (số lượng món, tổng tiền món ăn). \newline 11. Sau khi chọn xong, US-08 nhấn nút "Tiếp tục" / "Xong" để chuyển sang bước tiếp theo. \\
\hline
Alternative Flow & \textbf{4a. Tìm kiếm/Lọc món ăn trên thực đơn:} Giao diện cho phép tìm kiếm món theo tên hoặc lọc theo các tiêu chí khác (nếu có). \newline \textbf{10a. Xóa món khỏi giỏ hàng:} US-08 có thể xóa bất kỳ món nào đã thêm vào giỏ. \\
\hline
Exception Flow & \textbf{3a. Lỗi tải thực đơn:} Hệ thống không thể hiển thị thực đơn. \newline \textbf{7d. Lỗi thêm món/biến thể vào giỏ:} Hệ thống báo lỗi, món không được thêm. \newline \textbf{7e. Chưa chọn đủ biến thể bắt buộc:} Hệ thống yêu cầu chọn đầy đủ. \\
\hline
\multicolumn{2}{|c|}{\textbf{2.3. Thông tin bổ sung (Additional Information)}} \\
\hline
\textbf{Mục} & \textbf{Nội dung} \\
\hline
Business Rule & - \textbf{BR-UC3.3-1 (V3):} Chỉ sản phẩm được cấu hình hiển thị online mới xuất hiện. \newline - \textbf{BR-UC3.3-2 (V3):} Phải chọn biến thể bắt buộc (nếu có). \newline - \textbf{BR-UC3.3-3 (V3):} Giá món/biến thể phải chính xác. \newline - \textbf{BR-UC3.3-4 (V3):} Giỏ hàng là tạm thời cho đến khi hoàn tất đặt chỗ. \\
\hline
Non-Functional Requirement & - \textbf{NFR-UC3.3-1 (V3) (Usability):} Thực đơn online dễ duyệt, chọn món và biến thể thuận tiện. \newline - \textbf{NFR-UC3.3-2 (V3) (Performance):} Tải thực đơn và thêm món vào giỏ phải nhanh. \newline - \textbf{NFR-UC3.3-3 (V3) (Accuracy):} Thông tin món và giỏ hàng phải chính xác. \\
\hline
\end{longtable}

\subsubsection{Use Case UC-MD03-04: Xác nhận Lượt Đặt chỗ và Thanh toán Đặt cọc Online}
% (Gộp UC-MD03-06, UC-MD03-07, UC-MD03-09, và một phần UC-MD03-10 cũ)
\begin{longtable}{|m{4cm}|p{11cm}|}
\caption{Đặc tả Use Case UC-MD03-04: Xác nhận Lượt Đặt chỗ và Thanh toán Đặt cọc Online} \label{tab:uc_md03_04_revised_v3} \\
\hline
\multicolumn{2}{|c|}{\textbf{2.1. Tóm tắt (Summary)}} \\
\hline
\textbf{Mục} & \textbf{Nội dung} \\
\hline
\endhead % Header cho các trang tiếp theo
\hline
\endfoot % Footer cho bảng
\hline
\endlastfoot % Footer cho trang cuối cùng
Use Case Name & Xác nhận Lượt Đặt chỗ và Thanh toán Đặt cọc Online \\
\hline
Use Case ID & UC-MD03-04 \\
\hline
Use Case Description & Cho phép Khách hàng (US-08) xem lại toàn bộ thông tin lượt đặt chỗ đã thiết lập (bàn, món ăn), cung cấp thông tin liên hệ cá nhân, và thực hiện thanh toán số tiền đặt cọc bắt buộc (nếu có) qua cổng thanh toán trực tuyến để hoàn tất và xác nhận đặt chỗ. \\
\hline
Actor & US-08 (Khách hàng), System (Tính cọc, Xác nhận, Gửi thông báo) \\
\hline
Priority & Must Have \\
\hline
Trigger & Khách hàng đã hoàn thành việc lựa chọn thông tin đặt bàn và các món ăn đặt trước (nếu có) và muốn tiến hành hoàn tất đặt chỗ. \\
\hline
Pre-Condition & - Khách hàng đang trong luồng đặt chỗ online. \newline - Các lựa chọn về bàn (UC-MD03-01/02) và món ăn (UC-MD03-03) đã được ghi nhận tạm thời. \newline - Quy tắc tính đặt cọc đã được cấu hình (FR-MD03-15). \newline - Cổng thanh toán đã được tích hợp (FR-MD10-05). \\
\hline
Post-Condition & - \textbf{Thành công:} Lượt đặt chỗ được xác nhận. Tiền đặt cọc (nếu có) được thanh toán thành công. Khách hàng nhận được xác nhận (mã đặt chỗ, email/SMS). Bàn được giữ. \newline - \textbf{Thất bại:} Đặt chỗ không được xác nhận (ví dụ: thanh toán lỗi). \\
\hline
\multicolumn{2}{|c|}{\textbf{2.2. Luồng thực thi (Flow)}} \\
\hline
\textbf{Mục} & \textbf{Nội dung} \\
\hline
Basic Flow & 1. Sau khi US-08 hoàn tất chọn món (UC-MD03-03) hoặc chọn thông tin bàn (nếu không đặt món trước) và nhấn "Tiếp tục"/"Xác nhận". \newline 2. Hệ thống hiển thị trang tóm tắt lượt đặt chỗ, bao gồm: thông tin bàn đã chọn (ngày, giờ, số người, bàn cụ thể nếu có), danh sách chi tiết món ăn đặt trước (tên, SL, giá, biến thể), tổng giá trị món ăn. \newline 3. Hệ thống tự động tính toán và hiển thị số tiền đặt cọc cần thanh toán (dựa trên BR-UC3.8-1, BR-UC3.8-2, BR-UC3.8-3, BR-UC3.8-4). \newline 4. Hệ thống yêu cầu US-08 nhập thông tin cá nhân: Họ và Tên, Số Điện Thoại, Địa chỉ Email (Bắt buộc - BR-UC3.7-1, BR-UC3.7-2, BR-UC3.7-3). \newline 5. (Tùy chọn) US-08 nhập Ghi chú cho nhà hàng. \newline 6. US-08 kiểm tra lại toàn bộ thông tin và nhấn "Xác nhận và Thanh toán Đặt cọc" (hoặc "Xác nhận Đặt chỗ" nếu không có cọc). \newline 7. Hệ thống kiểm tra tính hợp lệ của thông tin cá nhân. \newline 8. \textbf{Nếu có yêu cầu đặt cọc:} \newline    a. Hệ thống chuyển hướng khách hàng đến cổng thanh toán đã chọn hoặc hiển thị form thanh toán nhúng (Tương tự luồng của UC-MD03-09 cũ). \newline    b. US-08 thực hiện thanh toán. \newline    c. Nếu thanh toán thành công: Hệ thống nhận xác nhận. \newline    d. Nếu thanh toán thất bại: Hệ thống báo lỗi, cho phép thử lại hoặc quay lại. (Exception Flow 8d1) \newline 9. \textbf{Sau khi thanh toán thành công (hoặc nếu không có cọc và nhấn xác nhận ở bước 6):} \newline    a. Hệ thống (System) tạo bản ghi đặt chỗ chính thức với trạng thái "Đã xác nhận". \newline    b. Hệ thống (System) tạo mã đặt chỗ duy nhất. \newline    c. Hệ thống (System) cập nhật trạng thái bàn đã chọn là "Đã đặt". \newline    d. Hệ thống (System) gửi email/SMS xác nhận cho khách hàng với đầy đủ chi tiết và mã đặt chỗ. \newline    e. Hệ thống hiển thị trang "Đặt chỗ thành công" cho US-08, bao gồm mã đặt chỗ và thông tin tóm tắt. \\
\hline
Alternative Flow & \textbf{4a. Khách hàng đã đăng nhập:} Hệ thống tự động điền thông tin cá nhân. \newline \textbf{8e. Đặt chỗ không yêu cầu đặt cọc:} Nếu cấu hình không yêu cầu đặt cọc, bước 8 (thanh toán) được bỏ qua. Sau bước 7, hệ thống trực tiếp thực hiện các hành động ở bước 9. \\
\hline
Exception Flow & \textbf{7a. Lỗi xác thực thông tin cá nhân:} Hệ thống báo lỗi, yêu cầu nhập lại. \newline \textbf{8d1. Thanh toán đặt cọc thất bại:} Khách hàng được thông báo, có thể chọn thử lại phương thức khác hoặc hủy bỏ đặt chỗ. Đặt chỗ không được xác nhận. \newline \textbf{9f. Lỗi hệ thống khi tạo đặt chỗ/gửi xác nhận:} Hệ thống báo lỗi nghiêm trọng. Cần thông báo cho quản trị viên. Nếu tiền cọc đã trừ, cần có quy trình hoàn tiền hoặc xử lý thủ công. \\
\hline
\multicolumn{2}{|c|}{\textbf{2.3. Thông tin bổ sung (Additional Information)}} \\
\hline
\textbf{Mục} & \textbf{Nội dung} \\
\hline
Business Rule & (Kết hợp các BR liên quan từ UC-MD03-06, 07, 08, 09, 10 cũ) \newline - \textbf{BR-UC3.4-1 (V3):} Phải hiển thị tóm tắt chính xác. \newline - \textbf{BR-UC3.4-2 (V3):} Thông tin liên hệ (Tên, SĐT, Email) là bắt buộc. \newline - \textbf{BR-UC3.4-3 (V3):} Đặt cọc (nếu có) phải được thanh toán thành công để xác nhận. \newline - \textbf{BR-UC3.4-4 (V3):} Xác nhận đặt chỗ (mã, email/SMS) phải được gửi cho khách. \\
\hline
Non-Functional Requirement & (Kết hợp các NFR liên quan) \newline - \textbf{NFR-UC3.4-1 (V3) (Usability):} Toàn bộ quy trình từ xem tóm tắt, nhập thông tin đến thanh toán phải liền mạch, dễ hiểu. \newline - \textbf{NFR-UC3.4-2 (V3) (Security):} Thanh toán và xử lý thông tin cá nhân phải bảo mật. \newline - \textbf{NFR-UC3.4-3 (V3) (Performance):} Các bước phải nhanh chóng. \newline - \textbf{NFR-UC3.4-4 (V3) (Reliability):} Quy trình phải đáng tin cậy, đặc biệt là việc ghi nhận thanh toán và xác nhận đặt chỗ. \\
\hline
\end{longtable}

\subsubsection{Use Case UC-MD03-05: Xem Lịch sử Đặt chỗ Cá nhân Online}
% (Giữ nguyên nội dung chi tiết như UC-MD03-17 version cũ, chỉ đổi ID)
\begin{longtable}{|m{4cm}|p{11cm}|}
\caption{Đặc tả Use Case UC-MD03-05: Xem Lịch sử Đặt chỗ Cá nhân Online} \label{tab:uc_md03_05_revised_v3} \\
\hline
\multicolumn{2}{|c|}{\textbf{2.1. Tóm tắt (Summary)}} \\
\hline
\textbf{Mục} & \textbf{Nội dung} \\
\hline
\endhead % Header cho các trang tiếp theo
\hline
\endfoot % Footer cho bảng
\hline
\endlastfoot % Footer cho trang cuối cùng
Use Case Name & Xem Lịch sử Đặt chỗ Cá nhân Online \\
\hline
Use Case ID & UC-MD03-05 \\
\hline
Use Case Description & Cho phép Khách hàng (US-08) đã đăng nhập vào tài khoản trên website/app của nhà hàng xem lại danh sách các lượt đặt chỗ mà họ đã thực hiện trước đây và xem thông tin chi tiết của từng lượt đặt chỗ. \\
\hline
Actor & US-08 (Khách hàng) \\
\hline
Priority & Should Have \\
\hline
Trigger & Khách hàng muốn kiểm tra lại thông tin một lượt đặt chỗ sắp tới hoặc xem lại lịch sử các lần đặt chỗ trước đây. \\
\hline
Pre-Condition & - Khách hàng (US-08) có tài khoản trên website/app và đã đăng nhập thành công. \newline - Khách hàng đã thực hiện ít nhất một lượt đặt chỗ online thông qua tài khoản này trước đó. \\
\hline
Post-Condition & - Khách hàng xem được danh sách các lượt đặt chỗ của mình. \newline - Khách hàng xem được thông tin chi tiết của một lượt đặt chỗ cụ thể. \\
\hline
\multicolumn{2}{|c|}{\textbf{2.2. Luồng thực thi (Flow)}} \\
\hline
\textbf{Mục} & \textbf{Nội dung} \\
\hline
Basic Flow & 1. Khách hàng (US-08) đã đăng nhập vào tài khoản. \newline 2. US-08 điều hướng đến khu vực quản lý tài khoản cá nhân và chọn mục "Lịch sử Đặt chỗ", "Đặt chỗ của tôi" hoặc tương tự. \newline 3. Hệ thống truy vấn cơ sở dữ liệu để lấy danh sách các lượt đặt chỗ được liên kết với tài khoản của US-08. \newline 4. Hệ thống hiển thị danh sách các lượt đặt chỗ, bao gồm các thông tin cơ bản như: \newline    - Mã đặt chỗ. \newline    - Ngày giờ đặt. \newline    - Số lượng người. \newline    - Trạng thái (Đã xác nhận, Đã hủy, Đã hoàn thành...). \newline 5. US-08 nhấp vào một lượt đặt chỗ cụ thể trong danh sách để xem chi tiết. \newline 6. Hệ thống truy xuất và hiển thị thông tin chi tiết của lượt đặt chỗ đó. \newline 7. US-08 xem xét thông tin. \\
\hline
Alternative Flow & \textbf{4a. Lọc/Sắp xếp lịch sử.} \newline \textbf{6a. Hủy đặt chỗ từ màn hình chi tiết (nếu được phép - xem UC-MD03-06).} \\
\hline
Exception Flow & \textbf{3a. Lỗi tải lịch sử đặt chỗ.} \newline \textbf{3b. Không có lịch sử đặt chỗ.} \\
\hline
\multicolumn{2}{|c|}{\textbf{2.3. Thông tin bổ sung (Additional Information)}} \\
\hline
\textbf{Mục} & \textbf{Nội dung} \\
\hline
Business Rule & - \textbf{BR-UC3.5-1 (V3):} Khách hàng chỉ xem được lịch sử của chính mình. \newline - \textbf{BR-UC3.5-2 (V3):} Thông tin hiển thị phải chính xác. \\
\hline
Non-Functional Requirement & - \textbf{NFR-UC3.5-1 (V3) (Usability):} Dễ truy cập, dễ hiểu. \newline - \textbf{NFR-UC3.5-2 (V3) (Performance):} Tải nhanh. \newline - \textbf{NFR-UC3.5-3 (V3) (Security):} Đảm bảo bảo mật. \\
\hline
\end{longtable}

\subsubsection{Use Case UC-MD03-06: (Tùy chọn) Hủy Lượt Đặt chỗ Online}

\begin{longtable}{|m{4cm}|p{11cm}|}
\caption{Đặc tả Use Case UC-MD03-06: (Tùy chọn) Hủy Lượt Đặt chỗ Online} \label{tab:uc_md03_06_revised_v3} \\
\hline
\multicolumn{2}{|c|}{\textbf{2.1. Tóm tắt (Summary)}} \\
\hline
\textbf{Mục} & \textbf{Nội dung} \\
\hline
\endhead % Header cho các trang tiếp theo
\hline
\endfoot % Footer cho bảng
\hline
\endlastfoot % Footer cho trang cuối cùng
Use Case Name & (Tùy chọn) Hủy Lượt Đặt chỗ Online \\
\hline
Use Case ID & UC-MD03-06 \\
\hline
Use Case Description & Cho phép Khách hàng (US-08) đã đăng nhập, tự hủy một lượt đặt chỗ đã được xác nhận của mình thông qua giao diện website/app, nếu việc hủy đó tuân thủ chính sách và thời hạn cho phép của nhà hàng. \\
\hline
Actor & US-08 (Khách hàng) \\
\hline
Priority & Should Have \\
\hline
Trigger & Khách hàng có thay đổi kế hoạch và muốn hủy lượt đặt chỗ đã được xác nhận. \\
\hline
Pre-Condition & - Khách hàng đã đăng nhập và đang xem chi tiết một lượt đặt chỗ của mình (UC-MD03-05). \newline - Lượt đặt chỗ đang ở trạng thái "Đã xác nhận" (Confirmed). \newline - Việc hủy vẫn còn trong thời hạn cho phép theo chính sách của nhà hàng (ví dụ: hủy trước 24 giờ so với giờ đặt). \newline - Chức năng cho phép khách tự hủy online được kích hoạt. \\
\hline
Post-Condition & - Trạng thái của lượt đặt chỗ được cập nhật thành "Đã hủy bởi khách" (Cancelled by Customer). \newline - Bàn đã giữ cho đặt chỗ này được giải phóng. \newline - Tiền đặt cọc có thể được xử lý theo chính sách (hoàn lại một phần/toàn bộ hoặc không hoàn - quy trình hoàn tiền có thể là một UC riêng hoặc xử lý thủ công). \newline - Khách hàng nhận được thông báo xác nhận hủy. \\
\hline
\multicolumn{2}{|c|}{\textbf{2.2. Luồng thực thi (Flow)}} \\
\hline
\textbf{Mục} & \textbf{Nội dung} \\
\hline
Basic Flow & 1. US-08 đang xem chi tiết lượt đặt chỗ muốn hủy (từ UC-MD03-05). \newline 2. Giao diện hiển thị nút "Hủy Đặt chỗ" (Cancel Booking). \newline 3. US-08 nhấp vào nút "Hủy Đặt chỗ". \newline 4. Hệ thống hiển thị hộp thoại yêu cầu xác nhận việc hủy, có thể kèm thông tin về chính sách hoàn/mất cọc. \newline 5. US-08 xác nhận muốn hủy. \newline 6. Hệ thống (System) cập nhật trạng thái đặt chỗ thành "Cancelled by Customer". \newline 7. Hệ thống (System) giải phóng bàn đã giữ (nếu có). \newline 8. Hệ thống (System) xử lý tiền đặt cọc theo chính sách (ví dụ: tự động hoàn một phần nếu đủ điều kiện, hoặc ghi nhận mất cọc). \newline 9. Hệ thống (System) gửi email/SMS thông báo hủy đặt chỗ thành công cho khách hàng. \newline 10. Hệ thống hiển thị thông báo "Đặt chỗ của bạn đã được hủy thành công." \\
\hline
Alternative Flow & \textbf{8a. Hoàn cọc cần xử lý thủ công:} \newline    1. Nếu chính sách hoàn cọc phức tạp hoặc cần quản lý duyệt, hệ thống chỉ ghi nhận yêu cầu hoàn cọc. Việc hoàn tiền thực tế do nhân viên thực hiện sau. \\
\hline
Exception Flow & \textbf{2a. Không thể hủy online:} \newline    1. Đặt chỗ không đủ điều kiện để hủy online (ví dụ: quá sát giờ đặt, đã qua giờ đặt, loại đặt chỗ không cho hủy online). \newline    2. Nút "Hủy Đặt chỗ" bị vô hiệu hóa hoặc không hiển thị. Hệ thống có thể hiển thị thông báo yêu cầu liên hệ nhà hàng để hủy. \newline \textbf{6a. Lỗi hệ thống khi hủy:} \newline    1. Hệ thống gặp lỗi khi cập nhật trạng thái hoặc giải phóng bàn. \newline    2. Hệ thống báo lỗi. Việc hủy có thể chưa hoàn tất. \\
\hline
\multicolumn{2}{|c|}{\textbf{2.3. Thông tin bổ sung (Additional Information)}} \\
\hline
\textbf{Mục} & \textbf{Nội dung} \\
\hline
Business Rule & - \textbf{BR-UC3.6-1 (V3):} Chính sách hủy đặt chỗ (thời hạn cho phép hủy, điều kiện hoàn cọc) phải được cấu hình rõ ràng (FR-MD03-15) và thông báo cho khách hàng. \newline - \textbf{BR-UC3.6-2 (V3):} Việc giải phóng bàn là bắt buộc khi hủy thành công. \\
\hline
Non-Functional Requirement & - \textbf{NFR-UC3.6-1 (V3) (Usability):} Nút hủy dễ tìm, thông tin chính sách hủy rõ ràng. \newline - \textbf{NFR-UC3.6-2 (V3) (Reliability):} Quy trình hủy và xử lý cọc phải đáng tin cậy. \\
\hline
\end{longtable}

% Các Use Case cho Nhân viên (MD03-07 đến MD03-16) sẽ được giữ nguyên cấu trúc như đã làm
% với việc tách Tạo/Sửa và các hành động Quản lý Trạng thái.
% Do giới hạn độ dài, tôi sẽ tiếp tục với các UC này trong phản hồi tiếp theo nếu Giáo sư yêu cầu.

\subsubsection{Use Case UC-MD03-07: Xem Danh sách Tổng hợp các Lượt Đặt chỗ}
% Nội dung tương tự UC-MD03-12 (cũ)
\begin{longtable}{|m{4cm}|p{11cm}|}
\caption{Đặc tả Use Case UC-MD03-07: Xem Danh sách Tổng hợp các Lượt Đặt chỗ} \label{tab:uc_md03_07_revised_v3} \\
\hline
\multicolumn{2}{|c|}{\textbf{2.1. Tóm tắt (Summary)}} \\
\hline
\textbf{Mục} & \textbf{Nội dung} \\
\hline
\endhead % Header cho các trang tiếp theo
\hline
\endfoot % Footer cho bảng
\hline
\endlastfoot % Footer cho trang cuối cùng
Use Case Name & Xem Danh sách Tổng hợp các Lượt Đặt chỗ \\
\hline
Use Case ID & UC-MD03-07 \\
\hline
Use Case Description & Cho phép Nhân viên được phân quyền (Quản lý, Lễ tân) xem danh sách tổng hợp các lượt đặt chỗ đã được tạo trong hệ thống (bao gồm cả đặt online và nhập thủ công), với khả năng lọc và tìm kiếm theo các tiêu chí khác nhau. \\
\hline
Actor & US-01 (Quản lý nhà hàng), US-03 (Nhân viên lễ tân) \\
\hline
Priority & Must Have \\
\hline
Trigger & Nhân viên cần kiểm tra các lượt đặt chỗ sắp tới, xem tình hình đặt bàn chung, hoặc tìm kiếm một lượt đặt chỗ cụ thể. \\
\hline
Pre-Condition & - Người dùng (US-01 hoặc US-03) đã đăng nhập vào hệ thống với quyền xem đặt chỗ. \newline - Đã có ít nhất một lượt đặt chỗ được tạo trong hệ thống. \\
\hline
Post-Condition & - Danh sách các lượt đặt chỗ phù hợp với tiêu chí lọc/tìm kiếm được hiển thị cho người dùng. \newline - Người dùng có cái nhìn tổng quan về tình trạng đặt chỗ. \\
\hline
\multicolumn{2}{|c|}{\textbf{2.2. Luồng thực thi (Flow)}} \\
\hline
\textbf{Mục} & \textbf{Nội dung} \\
\hline
Basic Flow & 1. Người dùng (US-01/US-03) truy cập vào module quản lý Đặt chỗ (Reservations). \newline 2. Hệ thống mặc định hiển thị danh sách các lượt đặt chỗ, thường là các lượt đặt cho ngày hiện tại hoặc tương lai gần, ở dạng danh sách (List View) hoặc dạng lịch (Calendar View). \newline 3. Danh sách hiển thị các thông tin cơ bản của mỗi lượt đặt chỗ, ví dụ: \newline    - Mã đặt chỗ. \newline    - Tên khách hàng. \newline    - Ngày giờ đặt. \newline    - Số lượng người. \newline    - Bàn được gán (nếu có). \newline    - Trạng thái đặt chỗ (ví dụ: Đã xác nhận, Chờ xác nhận, Đã hủy, Đã đến). \newline    - Trạng thái thanh toán cọc (nếu có). \newline 4. Người dùng xem xét danh sách. \\
\hline
Alternative Flow & \textbf{4a. Lọc danh sách:} \newline    1. Người dùng sử dụng các bộ lọc có sẵn (Filters) để thu hẹp danh sách, ví dụ: lọc theo Ngày, theo Trạng thái, theo Bàn, theo Khách hàng. \newline    2. Hệ thống áp dụng bộ lọc và hiển thị lại danh sách kết quả. \newline    3. Use Case quay lại bước 4. \newline \textbf{4b. Tìm kiếm:} \newline    1. Người dùng nhập từ khóa (ví dụ: tên khách, SĐT, mã đặt chỗ) vào ô tìm kiếm. \newline    2. Hệ thống thực hiện tìm kiếm và hiển thị các lượt đặt chỗ khớp với từ khóa. \newline    3. Use Case quay lại bước 4. \newline \textbf{4c. Sắp xếp danh sách:} \newline    1. Người dùng nhấp vào tiêu đề cột (ví dụ: Ngày giờ đặt, Tên khách hàng) để sắp xếp danh sách tăng dần hoặc giảm dần theo cột đó. \newline    2. Hệ thống sắp xếp lại và hiển thị danh sách. \newline    3. Use Case quay lại bước 4. \newline \textbf{4d. Chuyển đổi dạng xem:} \newline    1. Người dùng chọn chuyển đổi sang dạng xem khác (ví dụ: từ List View sang Calendar View hoặc Kanban View nếu có). \newline    2. Hệ thống hiển thị dữ liệu đặt chỗ theo dạng xem mới. \newline    3. Use Case quay lại bước 4. \\
\hline
Exception Flow & \textbf{2a. Lỗi tải danh sách:} \newline    1. Hệ thống gặp lỗi khi truy vấn hoặc hiển thị danh sách đặt chỗ. \newline    2. Hệ thống hiển thị thông báo lỗi. \newline    3. Use Case kết thúc không thành công. \newline \textbf{2b. Không có đặt chỗ nào:} \newline    1. Nếu không có lượt đặt chỗ nào phù hợp với bộ lọc mặc định. \newline    2. Hệ thống hiển thị danh sách trống hoặc thông báo "Không có đặt chỗ nào". \\
\hline
\multicolumn{2}{|c|}{\textbf{2.3. Thông tin bổ sung (Additional Information)}} \\
\hline
\textbf{Mục} & \textbf{Nội dung} \\
\hline
Business Rule & - \textbf{BR-UC3.7-1 (V3):} Danh sách phải hiển thị đủ thông tin cơ bản để nhân viên có thể nhận diện nhanh lượt đặt chỗ. \newline - \textbf{BR-UC3.7-2 (V3):} Các bộ lọc và chức năng tìm kiếm phải hoạt động chính xác, giúp người dùng dễ dàng tìm thấy thông tin cần thiết. \newline - \textbf{BR-UC3.7-3 (V3):} Trạng thái đặt chỗ hiển thị phải là trạng thái mới nhất của lượt đặt chỗ đó. \\
\hline
Non-Functional Requirement & - \textbf{NFR-UC3.7-1 (V3) (Usability):} Giao diện danh sách phải rõ ràng, dễ đọc. Các chức năng lọc, tìm kiếm, sắp xếp phải dễ sử dụng. \newline - \textbf{NFR-UC3.7-2 (V3) (Performance):} Thời gian tải danh sách đặt chỗ (ví dụ: cho một ngày) phải nhanh chóng (dưới 3 giây). Việc lọc/tìm kiếm cũng cần phản hồi nhanh. \newline - \textbf{NFR-UC3.7-3 (V3) (Security):} Chỉ những người dùng có quyền hạn phù hợp mới được phép xem danh sách đặt chỗ. \newline - \textbf{NFR-UC3.7-4 (V3) (Accuracy):} Dữ liệu hiển thị trong danh sách phải chính xác và đồng bộ với dữ liệu gốc trong cơ sở dữ liệu. \\
\hline
\end{longtable}

\subsubsection{Use Case UC-MD03-08: Xem Thông tin Chi tiết một Lượt Đặt chỗ}
% Nội dung tương tự UC-MD03-13 (cũ)
\begin{longtable}{|m{4cm}|p{11cm}|}
\caption{Đặc tả Use Case UC-MD03-08: Xem Thông tin Chi tiết một Lượt Đặt chỗ} \label{tab:uc_md03_08_revised_v3} \\
\hline
\multicolumn{2}{|c|}{\textbf{2.1. Tóm tắt (Summary)}} \\
\hline
\textbf{Mục} & \textbf{Nội dung} \\
\hline
\endhead % Header cho các trang tiếp theo
\hline
\endfoot % Footer cho bảng
\hline
\endlastfoot % Footer cho trang cuối cùng
Use Case Name & Xem Thông tin Chi tiết một Lượt Đặt chỗ \\
\hline
Use Case ID & UC-MD03-08 \\
\hline
Use Case Description & Cho phép Nhân viên được phân quyền (Quản lý, Lễ tân) xem thông tin chi tiết đầy đủ của một lượt đặt chỗ cụ thể đã được chọn từ danh sách. \\
\hline
Actor & US-01 (Quản lý nhà hàng), US-03 (Nhân viên lễ tân) \\
\hline
Priority & Must Have \\
\hline
Trigger & Nhân viên nhấp vào một lượt đặt chỗ cụ thể từ danh sách đặt chỗ (UC-MD03-07) để xem thông tin chi tiết hơn. \\
\hline
Pre-Condition & - Người dùng đang xem danh sách đặt chỗ (UC-MD03-07 thành công). \newline - Người dùng có quyền xem chi tiết đặt chỗ. \\
\hline
Post-Condition & - Form/màn hình hiển thị chi tiết đầy đủ của lượt đặt chỗ đã chọn được hiển thị cho người dùng. \newline - Người dùng nắm được mọi thông tin liên quan đến lượt đặt chỗ đó. \\
\hline
\multicolumn{2}{|c|}{\textbf{2.2. Luồng thực thi (Flow)}} \\
\hline
\textbf{Mục} & \textbf{Nội dung} \\
\hline
Basic Flow & 1. Người dùng (US-01/US-03) đang xem danh sách đặt chỗ (UC-MD03-07). \newline 2. Người dùng nhấp vào mã đặt chỗ, tên khách hàng hoặc một vùng có thể nhấp được của một dòng đặt chỗ cụ thể. \newline 3. Hệ thống truy xuất toàn bộ thông tin chi tiết của lượt đặt chỗ đã chọn từ cơ sở dữ liệu. \newline 4. Hệ thống hiển thị Form/màn hình chi tiết đặt chỗ, bao gồm các thông tin: \newline    - Mã đặt chỗ. \newline    - Thông tin khách hàng (Tên, SĐT, Email). \newline    - Ngày giờ đặt. \newline    - Thời lượng đặt (ước tính). \newline    - Số lượng người. \newline    - Bàn được chỉ định (nếu có). \newline    - Trạng thái đặt chỗ hiện tại. \newline    - Thông tin thanh toán đặt cọc (Số tiền cọc, trạng thái thanh toán, phương thức thanh toán). \newline    - Danh sách chi tiết các món ăn đặt trước (tên món, biến thể, số lượng, đơn giá, thành tiền). \newline    - Tổng giá trị món ăn đặt trước. \newline    - Ghi chú của khách hàng (nếu có). \newline    - Ghi chú nội bộ (nếu có). \newline    - Lịch sử thay đổi trạng thái hoặc các thông tin quan trọng (nếu có). \newline 5. Người dùng xem xét các thông tin chi tiết. \\
\hline
Alternative Flow & \textbf{5a. Thực hiện hành động từ màn hình chi tiết:} \newline    1. Từ màn hình chi tiết, người dùng có thể truy cập các hành động khác như "Sửa" (UC-MD03-10), "Hủy" (UC-MD03-12), "Đánh dấu đã đến" (UC-MD03-13), "In thông tin"... tùy thuộc vào quyền hạn và trạng thái đặt chỗ. \\
\hline
Exception Flow & \textbf{3a. Lỗi truy xuất dữ liệu chi tiết:} \newline    1. Hệ thống gặp lỗi khi cố gắng lấy thông tin chi tiết của lượt đặt chỗ từ cơ sở dữ liệu. \newline    2. Hệ thống hiển thị thông báo lỗi. \newline    3. Use Case kết thúc không thành công. Người dùng có thể quay lại danh sách. \newline \textbf{3b. Đặt chỗ không tồn tại/không có quyền truy cập:} \newline    1. Do lỗi đồng bộ hoặc vấn đề phân quyền, người dùng nhấp vào một đặt chỗ mà họ không có quyền xem hoặc đã bị xóa. \newline    2. Hệ thống hiển thị thông báo lỗi "Không tìm thấy đặt chỗ" hoặc "Bạn không có quyền truy cập". \newline    3. Use Case kết thúc không thành công. \\
\hline
\multicolumn{2}{|c|}{\textbf{2.3. Thông tin bổ sung (Additional Information)}} \\
\hline
\textbf{Mục} & \textbf{Nội dung} \\
\hline
Business Rule & - \textbf{BR-UC3.8-1 (V3):} Màn hình chi tiết phải hiển thị tất cả các thông tin liên quan đến lượt đặt chỗ một cách đầy đủ và chính xác. \newline - \textbf{BR-UC3.8-2 (V3):} Các thông tin nhạy cảm (nếu có) cần được kiểm soát quyền truy cập. \\
\hline
Non-Functional Requirement & - \textbf{NFR-UC3.8-1 (V3) (Usability):} Thông tin chi tiết cần được trình bày một cách logic, dễ đọc. Các phần thông tin khác nhau (thông tin khách, chi tiết đặt bàn, món ăn, thanh toán) nên được phân tách rõ ràng. \newline - \textbf{NFR-UC3.8-2 (V3) (Performance):} Thời gian tải và hiển thị đầy đủ chi tiết của một lượt đặt chỗ phải nhanh chóng (dưới 2 giây). \newline - \textbf{NFR-UC3.8-3 (V3) (Accuracy):} Mọi thông tin hiển thị phải là dữ liệu mới nhất và chính xác nhất của lượt đặt chỗ đó. \\
\hline
\end{longtable}

\subsubsection{Use Case UC-MD03-09: Tạo mới Lượt Đặt chỗ Thủ công (Backend/POS)}
% (Tương ứng FR-MD03-14A)
\begin{longtable}{|m{4cm}|p{11cm}|}
\caption{Đặc tả Use Case UC-MD03-09: Tạo mới Lượt Đặt chỗ Thủ công (Backend/POS)} \label{tab:uc_md03_09_revised_v3} \\
\hline
\multicolumn{2}{|c|}{\textbf{2.1. Tóm tắt (Summary)}} \\
\hline
\textbf{Mục} & \textbf{Nội dung} \\
\hline
\endhead % Header cho các trang tiếp theo
\hline
\endfoot % Footer cho bảng
\hline
\endlastfoot % Footer cho trang cuối cùng
Use Case Name & Tạo mới Lượt Đặt chỗ Thủ công (Backend/POS) \\
\hline
Use Case ID & UC-MD03-09 \\
\hline
Use Case Description & Cho phép Nhân viên được phân quyền (Quản lý, Lễ tân) tạo một lượt đặt chỗ mới trực tiếp trong hệ thống (qua giao diện backend hoặc một giao diện POS được thiết kế cho quản lý đặt chỗ), thường dành cho các trường hợp khách đặt qua điện thoại, email, hoặc khách vãng lai muốn đặt trước. \\
\hline
Actor & US-01 (Quản lý nhà hàng), US-03 (Nhân viên lễ tân) \\
\hline
Priority & Must Have \\
\hline
Trigger & - Khách hàng liên hệ đặt bàn qua các kênh không trực tuyến (điện thoại, email). \newline - Nhân viên cần nhập một lượt đặt chỗ đặc biệt (ví dụ: cho khách VIP, cho sự kiện nội bộ). \\
\hline
Pre-Condition & - Người dùng (US-01 hoặc US-03) đã đăng nhập vào hệ thống với quyền tạo đặt chỗ. \newline - Module quản lý đặt chỗ đang hoạt động. \\
\hline
Post-Condition & - Một bản ghi đặt chỗ mới được tạo trong hệ thống với các thông tin do nhân viên nhập. \newline - Trạng thái ban đầu của đặt chỗ được thiết lập (thường là "Chờ xác nhận" hoặc "Đã xác nhận" tùy quy trình). \newline - Nếu bàn được chọn, trạng thái bàn có thể được cập nhật. \newline - Hệ thống ghi nhận hoạt động. \\
\hline
\multicolumn{2}{|c|}{\textbf{2.2. Luồng thực thi (Flow)}} \\
\hline
\textbf{Mục} & \textbf{Nội dung} \\
\hline
Basic Flow & 1. Người dùng (US-01/US-03) truy cập module quản lý Đặt chỗ và chọn hành động "Tạo mới". \newline 2. Hệ thống hiển thị form đặt chỗ trống. \newline 3. Người dùng tìm kiếm và chọn Khách hàng (nếu đã có) hoặc nhập thông tin khách hàng mới (Tên, SĐT, Email - BR-UC3.9-1). \newline 4. Người dùng chọn Ngày, Giờ đặt, và Số lượng người. \newline 5. Hệ thống (có thể) kiểm tra và hiển thị danh sách các Bàn còn trống phù hợp. Người dùng chọn một bàn (nếu muốn gán ngay). \newline 6. (Tùy chọn) Người dùng thêm các Món ăn đặt trước vào đặt chỗ (tương tự UC-MD03-03 nhưng trong giao diện backend/POS). \newline 7. (Tùy chọn) Người dùng ghi nhận thông tin về tiền Đặt cọc nếu khách đã thanh toán hoặc sẽ thanh toán qua kênh khác (ví dụ: tiền mặt, chuyển khoản sau). \newline 8. Người dùng chọn Trạng thái ban đầu cho đặt chỗ (ví dụ: "Chờ xác nhận", "Đã xác nhận"). \newline 9. Người dùng nhập Ghi chú nội bộ hoặc ghi chú của khách (nếu có). \newline 10. Người dùng chọn hành động "Lưu". \newline 11. Hệ thống kiểm tra tính hợp lệ của dữ liệu (thông tin khách, ngày giờ, bàn trống...). \newline 12. Hệ thống lưu bản ghi đặt chỗ mới. \newline 13. Hệ thống hiển thị thông báo tạo thành công. \newline 14. (Tùy chọn) Hệ thống có thể kích hoạt gửi email thông báo cho khách hàng (nếu email được cung cấp và cấu hình cho phép). \\
\hline
Alternative Flow & \textbf{5a. Không gán bàn ngay:} \newline    1. Người dùng có thể không chọn bàn cụ thể ngay lúc tạo, để bàn được gán sau hoặc khi khách đến. \\
\hline
Exception Flow & \textbf{11a. Lỗi Xác thực Dữ liệu:} \newline    1. Thiếu thông tin bắt buộc (khách, ngày giờ, số người) hoặc dữ liệu không hợp lệ (chọn bàn đã đặt). \newline    2. Hệ thống báo lỗi. Không lưu. Use Case quay lại bước nhập liệu. \newline \textbf{12a. Lỗi Hệ thống khi Lưu:} \newline    1. Hệ thống gặp lỗi kỹ thuật. \newline    2. Hệ thống báo lỗi chung. \\
\hline
\multicolumn{2}{|c|}{\textbf{2.3. Thông tin bổ sung (Additional Information)}} \\
\hline
\textbf{Mục} & \textbf{Nội dung} \\
\hline
Business Rule & - \textbf{BR-UC3.9-1 (V3):} Thông tin Khách hàng (Tên, SĐT) là bắt buộc. \newline - \textbf{BR-UC3.9-2 (V3):} Phải kiểm tra tính khả dụng của bàn/khung giờ khi tạo. \newline - \textbf{BR-UC3.9-3 (V3):} Quy trình xử lý đặt cọc cho đặt chỗ thủ công cần được định nghĩa. \\
\hline
Non-Functional Requirement & - \textbf{NFR-UC3.9-1 (V3) (Usability):} Form tạo đặt chỗ thủ công phải dễ sử dụng, cho phép nhập nhanh thông tin. \newline - \textbf{NFR-UC3.9-2 (V3) (Performance):} Kiểm tra bàn trống và lưu đặt chỗ phải nhanh. \\
\hline
\end{longtable}

\subsubsection{Use Case UC-MD03-10: Sửa Thông tin Lượt Đặt chỗ (Backend/POS)}
% (Tương ứng FR-MD03-14B)
\begin{longtable}{|m{4cm}|p{11cm}|}
\caption{Đặc tả Use Case UC-MD03-10: Sửa Thông tin Lượt Đặt chỗ (Backend/POS)} \label{tab:uc_md03_10_revised_v3} \\
\hline
\multicolumn{2}{|c|}{\textbf{2.1. Tóm tắt (Summary)}} \\
\hline
\textbf{Mục} & \textbf{Nội dung} \\
\hline
\endhead % Header cho các trang tiếp theo
\hline
\endfoot % Footer cho bảng
\hline
\endlastfoot % Footer cho trang cuối cùng
Use Case Name & Sửa Thông tin Lượt Đặt chỗ (Backend/POS) \\
\hline
Use Case ID & UC-MD03-10 \\
\hline
Use Case Description & Cho phép Nhân viên được phân quyền (Quản lý, Lễ tân) chỉnh sửa các thông tin của một lượt đặt chỗ đã tồn tại trong hệ thống, ví dụ: thay đổi ngày giờ, số lượng người, bàn được gán, danh sách món ăn đặt trước, hoặc thông tin liên hệ của khách hàng. \\
\hline
Actor & US-01 (Quản lý nhà hàng), US-03 (Nhân viên lễ tân) \\
\hline
Priority & Must Have \\
\hline
Trigger & - Khách hàng yêu cầu thay đổi thông tin đặt chỗ đã thực hiện trước đó. \newline - Nhân viên phát hiện có sai sót trong thông tin đặt chỗ cần được sửa lại. \\
\hline
Pre-Condition & - Người dùng (US-01 hoặc US-03) đã đăng nhập vào hệ thống với quyền sửa đặt chỗ. \newline - Lượt đặt chỗ cần sửa đã tồn tại trong hệ thống và đang ở trạng thái cho phép sửa đổi (ví dụ: chưa đến giờ, chưa hủy hoàn toàn). \\
\hline
Post-Condition & - Thông tin của lượt đặt chỗ được chọn đã được cập nhật trong cơ sở dữ liệu theo những thay đổi đã thực hiện. \newline - Nếu các thay đổi ảnh hưởng đến tính khả dụng (bàn, thời gian), hệ thống sẽ phản ánh điều này. \newline - (Tùy chọn) Thông báo về sự thay đổi có thể được gửi cho khách hàng. \newline - Hệ thống ghi nhận hoạt động. \\
\hline
\multicolumn{2}{|c|}{\textbf{2.2. Luồng thực thi (Flow)}} \\
\hline
\textbf{Mục} & \textbf{Nội dung} \\
\hline
Basic Flow & 1. Người dùng (US-01/US-03) tìm và mở chi tiết lượt đặt chỗ cần sửa (UC-MD03-08). \newline 2. Người dùng chọn hành động "Sửa" (Edit). \newline 3. Hệ thống cho phép chỉnh sửa các trường thông tin trên form đặt chỗ: \newline    - Thông tin khách hàng (Tên, SĐT, Email). \newline    - Ngày, Giờ đặt, Số lượng người. \newline    - Bàn được gán. \newline    - Danh sách Món ăn đặt trước (thêm/sửa/xóa món, thay đổi số lượng/biến thể). \newline    - Thông tin đặt cọc (nếu cần điều chỉnh thủ công). \newline    - Ghi chú. \newline 4. Người dùng thực hiện các thay đổi cần thiết. \newline 5. Người dùng chọn hành động "Lưu" (Save). \newline 6. Hệ thống kiểm tra tính hợp lệ của các dữ liệu đã thay đổi (ví dụ: nếu đổi giờ/bàn, kiểm tra xem còn trống không; nếu thay đổi món, tính lại tiền cọc món ăn nếu có). \newline 7. Hệ thống lưu lại các thay đổi vào bản ghi đặt chỗ. \newline 8. Hệ thống hiển thị thông báo cập nhật thành công. \newline 9. (Tùy chọn) Hệ thống có thể đề xuất hoặc tự động gửi thông báo cập nhật cho khách hàng về những thay đổi này. \\
\hline
Alternative Flow & Không có luồng thay thế đáng kể. \\
\hline
Exception Flow & \textbf{6a. Lỗi Xác thực Dữ liệu / Xung đột:} \newline    1. Hệ thống phát hiện dữ liệu sửa không hợp lệ (ví dụ: chọn bàn đã bị đặt vào giờ mới, số người vượt quá sức chứa bàn mới). \newline    2. Hệ thống báo lỗi cụ thể. \newline    3. Hệ thống không cho phép lưu. Use Case quay lại bước 4. \newline \textbf{7a. Lỗi Hệ thống khi Lưu:} \newline    1. Hệ thống gặp sự cố kỹ thuật khi lưu thay đổi. \newline    2. Hệ thống hiển thị thông báo lỗi chung. \\
\hline
\multicolumn{2}{|c|}{\textbf{2.3. Thông tin bổ sung (Additional Information)}} \\
\hline
\textbf{Mục} & \textbf{Nội dung} \\
\hline
Business Rule & - \textbf{BR-UC3.10-1 (V3):} Việc sửa đổi thông tin đặt chỗ phải tuân thủ các quy tắc về tính khả dụng của bàn/khung giờ. \newline - \textbf{BR-UC3.10-2 (V3):} Nếu thay đổi số lượng người hoặc món ăn đặt trước, tiền đặt cọc có thể cần được tính toán lại (hệ thống tự động hoặc nhân viên điều chỉnh). \newline - \textbf{BR-UC3.10-3 (V3):} Nên có chính sách rõ ràng về việc cho phép sửa đổi đặt chỗ gần giờ G và việc thông báo cho khách hàng. \\
\hline
Non-Functional Requirement & - \textbf{NFR-UC3.10-1 (V3) (Usability):} Form sửa đặt chỗ phải dễ dàng cho nhân viên cập nhật các loại thông tin khác nhau. \newline - \textbf{NFR-UC3.10-2 (V3) (Data Integrity):} Mọi thay đổi phải được lưu chính xác và đảm bảo tính nhất quán của dữ liệu đặt chỗ. \\
\hline
\end{longtable}

\subsubsection{Use Case UC-MD03-11: Xác nhận Thủ công Lượt Đặt chỗ}
% (Tương ứng FR-MD03-15A)
\begin{longtable}{|m{4cm}|p{11cm}|}
\caption{Đặc tả Use Case UC-MD03-11: Xác nhận Thủ công Lượt Đặt chỗ} \label{tab:uc_md03_11_revised_v3} \\
\hline
\multicolumn{2}{|c|}{\textbf{2.1. Tóm tắt (Summary)}} \\
\hline
\textbf{Mục} & \textbf{Nội dung} \\
\hline
\endhead % Header cho các trang tiếp theo
\hline
\endfoot % Footer cho bảng
\hline
\endlastfoot % Footer cho trang cuối cùng
Use Case Name & Xác nhận Thủ công Lượt Đặt chỗ \\
\hline
Use Case ID & UC-MD03-11 \\
\hline
Use Case Description & Cho phép Nhân viên được phân quyền (Quản lý, Lễ tân) thay đổi trạng thái của một lượt đặt chỗ từ "Chờ xác nhận" (Pending) hoặc một trạng thái tương tự sang "Đã xác nhận" (Confirmed), thường áp dụng cho các đặt chỗ được tạo thủ công không qua thanh toán cọc online hoặc các trường hợp đặc biệt. \\
\hline
Actor & US-01 (Quản lý nhà hàng), US-03 (Nhân viên lễ tân) \\
\hline
Priority & Must Have \\
\hline
Trigger & - Một lượt đặt chỗ được tạo thủ công (UC-MD03-09) đang ở trạng thái "Chờ xác nhận" và nhân viên muốn chính thức hóa nó. \newline - Một đặt chỗ online vì lý do nào đó chưa được tự động xác nhận và cần nhân viên can thiệp. \\
\hline
Pre-Condition & - Người dùng (US-01 hoặc US-03) đã đăng nhập với quyền quản lý trạng thái đặt chỗ. \newline - Lượt đặt chỗ cần xác nhận đang ở trạng thái cho phép chuyển sang "Đã xác nhận". \\
\hline
Post-Condition & - Trạng thái của lượt đặt chỗ được cập nhật thành "Đã xác nhận" (Confirmed). \newline - Bàn liên kết (nếu có) được chính thức giữ cho đặt chỗ này. \newline - (Tùy chọn) Email/SMS xác nhận được gửi cho khách hàng. \\
\hline
\multicolumn{2}{|c|}{\textbf{2.2. Luồng thực thi (Flow)}} \\
\hline
\textbf{Mục} & \textbf{Nội dung} \\
\hline
Basic Flow & 1. Người dùng (US-01/US-03) tìm và mở chi tiết lượt đặt chỗ cần xác nhận (UC-MD03-08). \newline 2. Người dùng xem xét thông tin đặt chỗ và đảm bảo mọi thứ hợp lệ. \newline 3. Người dùng tìm và nhấp vào nút/hành động "Xác nhận" (Confirm) trên form đặt chỗ. \newline 4. Hệ thống cập nhật trạng thái của bản ghi đặt chỗ thành "Confirmed". \newline 5. Nếu bàn chưa được giữ chính thức, hệ thống cập nhật trạng thái bàn tương ứng. \newline 6. (Tùy chọn) Hệ thống kích hoạt gửi email/SMS xác nhận cho khách hàng (nếu chưa gửi trước đó hoặc cần gửi lại). \newline 7. Hệ thống hiển thị thông báo "Đặt chỗ đã được xác nhận thành công." \\
\hline
Alternative Flow & \textbf{3a. Xác nhận từ List View:} \newline    1. Nhân viên có thể chọn một hoặc nhiều đặt chỗ "Chờ xác nhận" từ danh sách (UC-MD03-07) và chọn hành động "Xác nhận" hàng loạt. \\
\hline
Exception Flow & \textbf{4a. Lỗi cập nhật trạng thái/bàn:} \newline    1. Hệ thống gặp lỗi kỹ thuật khi lưu trạng thái mới hoặc cập nhật thông tin bàn. \newline    2. Hệ thống báo lỗi. Trạng thái có thể chưa được cập nhật. \\
\hline
\multicolumn{2}{|c|}{\textbf{2.3. Thông tin bổ sung (Additional Information)}} \\
\hline
\textbf{Mục} & \textbf{Nội dung} \\
\hline
Business Rule & - \textbf{BR-UC3.11-1 (V3):} Chỉ những đặt chỗ ở trạng thái phù hợp (ví dụ: "Pending", "Draft") mới có thể được xác nhận thủ công. \newline - \textbf{BR-UC3.11-2 (V3):} Việc xác nhận thủ công đồng nghĩa với việc nhà hàng cam kết giữ chỗ cho khách. \\
\hline
Non-Functional Requirement & - \textbf{NFR-UC3.11-1 (V3) (Usability):} Nút xác nhận phải rõ ràng. \newline - \textbf{NFR-UC3.11-2 (V3) (Performance):} Cập nhật trạng thái nhanh chóng. \\
\hline
\end{longtable}

\subsubsection{Use Case UC-MD03-12: Hủy bỏ Lượt Đặt chỗ (Backend/POS)}
% (Tương ứng FR-MD03-15B)
\begin{longtable}{|m{4cm}|p{11cm}|}
\caption{Đặc tả Use Case UC-MD03-12: Hủy bỏ Lượt Đặt chỗ (Backend/POS)} \label{tab:uc_md03_12_revised_v3} \\
\hline
\multicolumn{2}{|c|}{\textbf{2.1. Tóm tắt (Summary)}} \\
\hline
\textbf{Mục} & \textbf{Nội dung} \\
\hline
\endhead % Header cho các trang tiếp theo
\hline
\endfoot % Footer cho bảng
\hline
\endlastfoot % Footer cho trang cuối cùng
Use Case Name & Hủy bỏ Lượt Đặt chỗ (Backend/POS) \\
\hline
Use Case ID & UC-MD03-12 \\
\hline
Use Case Description & Cho phép Nhân viên được phân quyền (Quản lý, Lễ tân) hủy bỏ một lượt đặt chỗ đã tồn tại trong hệ thống, ví dụ khi khách hàng yêu cầu hủy qua điện thoại hoặc nhà hàng không thể đáp ứng đặt chỗ đó. \\
\hline
Actor & US-01 (Quản lý nhà hàng), US-03 (Nhân viên lễ tân) \\
\hline
Priority & Must Have \\
\hline
Trigger & - Khách hàng liên hệ yêu cầu hủy đặt chỗ. \newline - Nhà hàng cần hủy đặt chỗ do lý do bất khả kháng hoặc thay đổi kế hoạch. \\
\hline
Pre-Condition & - Người dùng (US-01 hoặc US-03) đã đăng nhập với quyền quản lý/hủy đặt chỗ. \newline - Lượt đặt chỗ cần hủy đang ở trạng thái cho phép hủy (ví dụ: "Confirmed", "Pending"). \\
\hline
Post-Condition & - Trạng thái của lượt đặt chỗ được cập nhật thành "Đã hủy" (Cancelled). \newline - Bàn đã được giữ cho đặt chỗ này (nếu có) được giải phóng. \newline - Tiền đặt cọc (nếu có) được xử lý theo chính sách (hoàn lại hoặc không). \newline - (Tùy chọn) Thông báo hủy được gửi cho khách hàng. \\
\hline
\multicolumn{2}{|c|}{\textbf{2.2. Luồng thực thi (Flow)}} \\
\hline
\textbf{Mục} & \textbf{Nội dung} \\
\hline
Basic Flow & 1. Người dùng (US-01/US-03) tìm và mở chi tiết lượt đặt chỗ cần hủy (UC-MD03-08). \newline 2. Người dùng chọn hành động "Hủy Đặt chỗ" (Cancel Booking). \newline 3. Hệ thống yêu cầu xác nhận việc hủy. \newline 4. (Tùy chọn) Hệ thống có thể yêu cầu nhập Lý do hủy. Người dùng nhập lý do. \newline 5. Người dùng xác nhận muốn hủy. \newline 6. Hệ thống cập nhật trạng thái đặt chỗ thành "Cancelled". \newline 7. Hệ thống giải phóng bàn đã được liên kết với đặt chỗ này. \newline 8. Hệ thống xử lý Tiền đặt cọc theo chính sách đã cấu hình (BR-UC3.12-1): \newline    a. Nếu đủ điều kiện hoàn cọc: Hệ thống có thể tạo một yêu cầu hoàn tiền hoặc nhân viên cần thực hiện quy trình hoàn tiền thủ công. \newline    b. Nếu không hoàn cọc: Tiền cọc được ghi nhận là doanh thu hoặc theo quy định kế toán. \newline 9. (Tùy chọn) Hệ thống gửi email/SMS thông báo hủy đặt chỗ cho khách hàng, có thể kèm lý do (nếu nhà hàng hủy) và thông tin về việc xử lý cọc. \newline 10. Hệ thống hiển thị thông báo "Đặt chỗ đã được hủy thành công." \\
\hline
Alternative Flow & \textbf{2a. Hủy từ List View:} Nhân viên có thể chọn hủy từ danh sách đặt chỗ. \\
\hline
Exception Flow & \textbf{5a. Không thể hủy do trạng thái không phù hợp:} \newline    1. Đặt chỗ đã ở trạng thái "Arrived" hoặc "Completed". \newline    2. Hệ thống báo lỗi "Không thể hủy đặt chỗ ở trạng thái này." \newline \textbf{6a. Lỗi cập nhật trạng thái/giải phóng bàn/xử lý cọc:} \newline    1. Hệ thống gặp lỗi kỹ thuật. \newline    2. Hệ thống báo lỗi. Việc hủy có thể chưa hoàn tất đúng cách. \\
\hline
\multicolumn{2}{|c|}{\textbf{2.3. Thông tin bổ sung (Additional Information)}} \\
\hline
\textbf{Mục} & \textbf{Nội dung} \\
\hline
Business Rule & - \textbf{BR-UC3.12-1 (V3):} Chính sách hoàn/mất tiền đặt cọc khi hủy phải được áp dụng đúng. \newline - \textbf{BR-UC3.12-2 (V3):} Việc giải phóng bàn là bắt buộc. \newline - \textbf{BR-UC3.12-3 (V3):} Nên ghi nhận lý do hủy để phục vụ báo cáo và cải thiện dịch vụ. \\
\hline
Non-Functional Requirement & - \textbf{NFR-UC3.12-1 (V3) (Usability):} Thao tác hủy phải rõ ràng. \newline - \textbf{NFR-UC3.12-2 (V3) (Reliability):} Xử lý hủy và cọc phải đáng tin cậy. \\
\hline
\end{longtable}

\subsubsection{Use Case UC-MD03-13: Đánh dấu Khách đã đến (Check-in) cho Lượt Đặt chỗ}
% (Tương ứng FR-MD03-15C)
\begin{longtable}{|m{4cm}|p{11cm}|}
\caption{Đặc tả Use Case UC-MD03-13: Đánh dấu Khách đã đến (Check-in) cho Lượt Đặt chỗ} \label{tab:uc_md03_13_revised_v3} \\
\hline
\multicolumn{2}{|c|}{\textbf{2.1. Tóm tắt (Summary)}} \\
\hline
\textbf{Mục} & \textbf{Nội dung} \\
\hline
\endhead % Header cho các trang tiếp theo
\hline
\endfoot % Footer cho bảng
\hline
\endlastfoot % Footer cho trang cuối cùng
Use Case Name & Đánh dấu Khách đã đến (Check-in) cho Lượt Đặt chỗ \\
\hline
Use Case ID & UC-MD03-13 \\
\hline
Use Case Description & Cho phép Nhân viên (Lễ tân, Phục vụ) đánh dấu trong hệ thống rằng khách hàng có đặt chỗ đã đến nhà hàng và nhận bàn. \\
\hline
Actor & US-03 (Nhân viên lễ tân), US-01 (Quản lý nhà hàng), US-02 (Nhân viên phục vụ - nếu có quyền) \\
\hline
Priority & Must Have \\
\hline
Trigger & Khách hàng có đặt chỗ đến nhà hàng vào đúng hoặc gần giờ đã đặt. \\
\hline
Pre-Condition & - Người dùng đã đăng nhập vào hệ thống. \newline - Lượt đặt chỗ của khách đang ở trạng thái "Đã xác nhận" (Confirmed). \newline - Nhân viên đã xác minh đúng là khách hàng của lượt đặt chỗ đó. \\
\hline
Post-Condition & - Trạng thái của lượt đặt chỗ được cập nhật thành "Đã đến" (Arrived / Seated). \newline - Bàn được gán cho đặt chỗ đó trên POS được chính thức chuyển sang trạng thái "Đang có khách" (Occupied) nếu chưa phải. \newline - Đơn hàng POS có thể được tự động mở cho bàn đó (liên kết với UC-MD05-03). \\
\hline
\multicolumn{2}{|c|}{\textbf{2.2. Luồng thực thi (Flow)}} \\
\hline
\textbf{Mục} & \textbf{Nội dung} \\
\hline
Basic Flow & 1. Khách hàng đến, thông báo có đặt chỗ. \newline 2. Nhân viên (US-03/US-01/US-02) tìm lượt đặt chỗ của khách trong danh sách (UC-MD03-07) hoặc trên Sơ đồ tầng POS (nếu bàn đã được gán và đánh dấu Reserved). \newline 3. Nhân viên mở chi tiết lượt đặt chỗ (UC-MD03-08). \newline 4. Nhân viên chọn hành động "Đánh dấu Đã đến" / "Check-in" / "Seat Customer". \newline 5. Hệ thống cập nhật trạng thái đặt chỗ thành "Arrived". \newline 6. Nếu đặt chỗ đã được gán bàn, hệ thống đảm bảo trạng thái bàn đó trên POS là "Occupied". Nếu chưa gán bàn, nhân viên có thể cần thực hiện gán bàn ngay lúc này (có thể là một phần của hành động check-in). \newline 7. Hệ thống hiển thị thông báo thành công. \newline 8. Nếu thao tác từ POS, hệ thống có thể tự động mở màn hình đơn hàng cho bàn đó (UC-MD05-03). \\
\hline
Alternative Flow & \textbf{4a. Check-in từ Sơ đồ tầng POS:} \newline    1. Nếu bàn đã được gán cho đặt chỗ và hiển thị là "Reserved" trên POS. \newline    2. Nhân viên nhấp vào bàn đó trên POS. \newline    3. Hệ thống có thể hiển thị thông tin đặt chỗ và có nút "Check-in" / "Seat". \newline    4. Nhân viên nhấn nút đó. Use Case tiếp tục từ bước 5. \\
\hline
Exception Flow & \textbf{6a. Lỗi cập nhật trạng thái đặt chỗ/bàn:} \newline    1. Hệ thống gặp lỗi kỹ thuật. \newline    2. Hệ thống báo lỗi. \\
\hline
\multicolumn{2}{|c|}{\textbf{2.3. Thông tin bổ sung (Additional Information)}} \\
\hline
\textbf{Mục} & \textbf{Nội dung} \\
\hline
Business Rule & - \textbf{BR-UC3.13-1 (V3):} Chỉ những đặt chỗ "Confirmed" mới có thể được check-in. \newline - \textbf{BR-UC3.13-2 (V3):} Việc check-in là cơ sở để theo dõi tình trạng no-show và quản lý bàn hiệu quả. \\
\hline
Non-Functional Requirement & - \textbf{NFR-UC3.13-1 (V3) (Usability):} Thao tác check-in phải nhanh chóng, dễ thực hiện. \newline - \textbf{NFR-UC3.13-2 (V3) (Integration):} Trạng thái check-in phải được đồng bộ giữa module Đặt chỗ và POS (nếu là module riêng). \\
\hline
\end{longtable}

\subsubsection{Use Case UC-MD03-14: Xem Báo cáo Tổng hợp Món ăn Cần chuẩn bị (Đặt trước)}
% (Giữ nguyên nội dung chi tiết như UC-MD03-16 version cũ, chỉ đổi ID)
\begin{longtable}{|m{4cm}|p{11cm}|}
\caption{Đặc tả Use Case UC-MD03-14: Xem Báo cáo Tổng hợp Món ăn Cần chuẩn bị (Đặt trước)} \label{tab:uc_md03_14_revised_v3} \\
\hline
\multicolumn{2}{|c|}{\textbf{2.1. Tóm tắt (Summary)}} \\
\hline
\textbf{Mục} & \textbf{Nội dung} \\
\hline
\endhead % Header cho các trang tiếp theo
\hline
\endfoot % Footer cho bảng
\hline
\endlastfoot % Footer cho trang cuối cùng
Use Case Name & Xem Báo cáo Tổng hợp Món ăn Cần chuẩn bị (Đặt trước) \\
\hline
Use Case ID & UC-MD03-14 \\
\hline
Use Case Description & Cung cấp cho bộ phận Bếp hoặc Quản lý một giao diện/báo cáo tổng hợp danh sách các món ăn và đồ uống đã được khách hàng đặt trước cho các lượt đặt chỗ sắp tới, giúp chuẩn bị nguyên liệu và lên kế hoạch chế biến hiệu quả. \\
\hline
Actor & US-04 (Nhân viên bếp), US-01 (Quản lý nhà hàng) \\
\hline
Priority & Should Have (Quan trọng nếu đặt món trước là phổ biến) \\
\hline
Trigger & Bộ phận bếp/quản lý cần biết trước các món ăn cần chuẩn bị cho các khách hàng đã đặt chỗ và đặt món trước. \\
\hline
Pre-Condition & - Người dùng (US-04 hoặc US-01) đã đăng nhập vào hệ thống với quyền truy cập báo cáo/danh sách món đặt trước. \newline - Có ít nhất một lượt đặt chỗ đã xác nhận và có chứa thông tin món ăn đặt trước. \\
\hline
Post-Condition & - Danh sách tổng hợp các món ăn cần chuẩn bị (tên món, biến thể, số lượng) cho một khoảng thời gian hoặc một ca làm việc cụ thể được hiển thị. \newline - Bộ phận bếp/quản lý có thông tin để chuẩn bị. \\
\hline
\multicolumn{2}{|c|}{\textbf{2.2. Luồng thực thi (Flow)}} \\
\hline
\textbf{Mục} & \textbf{Nội dung} \\
\hline
Basic Flow & 1. Người dùng (US-04/US-01) truy cập vào chức năng/báo cáo "Món ăn Đặt trước" (Pre-ordered Items Report/List). \newline 2. Hệ thống mặc định hiển thị danh sách các món ăn đã được đặt trước cho ngày hiện tại hoặc ca làm việc hiện tại. \newline 3. Danh sách này thường được tổng hợp theo từng món ăn/biến thể, hiển thị: \newline    - Tên món ăn / Biến thể. \newline    - Tổng số lượng cần chuẩn bị. \newline    - (Tùy chọn) Danh sách các lượt đặt chỗ liên quan đến món đó (Mã đặt chỗ, Giờ đến, Bàn). \newline    - (Tùy chọn) Ghi chú đặc biệt liên quan đến món ăn từ các lượt đặt chỗ. \newline 4. Người dùng xem xét danh sách để lên kế hoạch chuẩn bị. \\
\hline
Alternative Flow & \textbf{2a. Lọc theo khoảng thời gian:} \newline    1. Người dùng chọn một khoảng thời gian khác (ví dụ: ngày mai, tuần tới) hoặc một ca làm việc cụ thể. \newline    2. Hệ thống lọc và hiển thị lại danh sách món đặt trước cho khoảng thời gian đã chọn. \newline    3. Use Case quay lại bước 4. \newline \textbf{2b. Lọc theo trạng thái đặt chỗ:} \newline    1. Người dùng chỉ muốn xem món của các đặt chỗ "Đã xác nhận". \newline    2. Hệ thống áp dụng bộ lọc trạng thái. \newline    3. Use Case quay lại bước 4. \newline \textbf{3a. Xem chi tiết theo từng đặt chỗ:} \newline    1. Thay vì tổng hợp, giao diện hiển thị danh sách các lượt đặt chỗ sắp tới, và người dùng có thể nhấp vào từng lượt để xem danh sách món đặt trước của riêng lượt đó. \\
\hline
Exception Flow & \textbf{2a. Lỗi tải dữ liệu:} \newline    1. Hệ thống gặp lỗi khi truy vấn và tổng hợp dữ liệu món ăn đặt trước. \newline    2. Hệ thống hiển thị thông báo lỗi. \newline    3. Use Case kết thúc không thành công. \newline \textbf{2b. Không có món nào đặt trước:} \newline    1. Không có lượt đặt chỗ nào trong khoảng thời gian/bộ lọc có món đặt trước. \newline    2. Hệ thống hiển thị danh sách trống hoặc thông báo "Không có món ăn nào được đặt trước". \\
\hline
\multicolumn{2}{|c|}{\textbf{2.3. Thông tin bổ sung (Additional Information)}} \\
\hline
\textbf{Mục} & \textbf{Nội dung} \\
\hline
Business Rule & - \textbf{BR-UC3.14-1 (V3):} Danh sách chỉ nên bao gồm các món từ những lượt đặt chỗ đã được xác nhận (Confirmed) và chưa bị hủy (Not Cancelled). \newline - \textbf{BR-UC3.14-2 (V3):} Số lượng hiển thị phải là tổng số lượng của món ăn/biến thể đó từ tất cả các lượt đặt chỗ hợp lệ trong khoảng thời gian/bộ lọc được chọn. \newline - \textbf{BR-UC3.14-3 (V3):} Giao diện/báo cáo này nên dễ dàng truy cập đối với bộ phận bếp. Có thể cần in ra được. \\
\hline
Non-Functional Requirement & - \textbf{NFR-UC3.14-1 (V3) (Usability):} Giao diện/báo cáo phải rõ ràng, dễ đọc, dễ hiểu cho nhân viên bếp. Việc lọc theo thời gian/ca làm việc phải đơn giản. \newline - \textbf{NFR-UC3.14-2 (V3) (Performance):} Thời gian tải và tổng hợp danh sách món đặt trước cho một ngày phải nhanh chóng (dưới 5 giây). \newline - \textbf{NFR-UC3.14-3 (V3) (Accuracy):} Dữ liệu về tên món, biến thể, số lượng phải chính xác 100\%. \newline - \textbf{NFR-UC3.14-4 (V3) (Accessibility):} Nếu cần hiển thị trên màn hình trong bếp, giao diện cần có font chữ lớn, độ tương phản cao. \\
\hline
\end{longtable}

\subsubsection{Use Case UC-MD03-15: Cấu hình Quy tắc và Tham số Đặt chỗ}
% (Giữ nguyên nội dung chi tiết như UC-MD03-11 version cũ, chỉ đổi ID)
\begin{longtable}{|m{4cm}|p{11cm}|}
\caption{Đặc tả Use Case UC-MD03-15: Cấu hình Quy tắc và Tham số Đặt chỗ} \label{tab:uc_md03_15_revised_v3} \\
\hline
\multicolumn{2}{|c|}{\textbf{2.1. Tóm tắt (Summary)}} \\
\hline
\textbf{Mục} & \textbf{Nội dung} \\
\hline
\endhead % Header cho các trang tiếp theo
\hline
\endfoot % Footer cho bảng
\hline
\endlastfoot % Footer cho trang cuối cùng
Use Case Name & Cấu hình Quy tắc và Tham số Đặt chỗ \\
\hline
Use Case ID & UC-MD03-15 \\
\hline
Use Case Description & Cho phép Quản lý nhà hàng hoặc Quản trị viên hệ thống thiết lập các quy tắc và tham số vận hành cho chức năng đặt chỗ online và quản lý đặt chỗ nói chung. Bao gồm giờ hoạt động, khoảng thời gian đặt, giới hạn số khách, quy tắc đặt cọc, giá trị bàn, và các tùy chọn khác. \\
\hline
Actor & US-01 (Quản lý nhà hàng), US-10 (Quản trị viên Hệ thống) \\
\hline
Priority & Must Have \\
\hline
Trigger & Cần thiết lập ban đầu cho chức năng đặt chỗ hoặc cần thay đổi các quy tắc vận hành hiện tại. \\
\hline
Pre-Condition & - Người dùng (US-01 hoặc US-10) đã đăng nhập với quyền quản trị cấu hình module Đặt chỗ (Booking/Reservation) hoặc cấu hình Website/POS liên quan. \newline - Module Đặt chỗ (hoặc tương đương) đã được cài đặt. \\
\hline
Post-Condition & - Các quy tắc và tham số đặt chỗ được cập nhật trong cấu hình hệ thống. \newline - Chức năng đặt chỗ online (cho khách hàng) và quản lý đặt chỗ (cho nhân viên) sẽ hoạt động theo các quy tắc mới được thiết lập. \\
\hline
\multicolumn{2}{|c|}{\textbf{2.2. Luồng thực thi (Flow)}} \\
\hline
\textbf{Mục} & \textbf{Nội dung} \\
\hline
Basic Flow & 1. Người dùng (US-01/US-10) truy cập vào khu vực cấu hình của module Đặt chỗ (ví dụ: Reservations > Configuration > Settings). \newline 2. Hệ thống hiển thị giao diện cấu hình với nhiều tùy chọn được nhóm lại. \newline 3. Người dùng tìm đến các mục cấu hình cần thiết và thay đổi giá trị: \newline    - \textbf{Giờ hoạt động \& Khung giờ đặt chỗ:} Thiết lập giờ mở cửa, giờ đóng cửa cho phép đặt bàn, khoảng cách giữa các slot đặt (ví dụ: 15 phút), thời lượng mặc định của một lượt đặt. \newline    - \textbf{Giới hạn đặt chỗ:} Số ngày tối thiểu/tối đa cho phép đặt trước, số lượng khách tối thiểu/tối đa cho mỗi lượt đặt online. \newline    - \textbf{Quy tắc Đặt cọc:} Kích hoạt/Tắt yêu cầu đặt cọc, nhập Tỷ lệ phần trăm đặt cọc cho bàn, Tỷ lệ phần trăm đặt cọc cho món ăn. \newline    - \textbf{Giá trị Bàn:} Truy cập một khu vực riêng (ví dụ: quản lý tài nguyên bàn) để nhập giá trị tham chiếu cho từng bàn hoặc loại bàn (dùng để tính cọc bàn). \newline    - \textbf{Cho phép chọn bàn:} Kích hoạt/Tắt tùy chọn cho phép khách hàng tự chọn bàn cụ thể trên sơ đồ tầng khi đặt online. \newline    - \textbf{Thông báo \& Email Template:} Cấu hình nội dung các email/SMS xác nhận, nhắc nhở, hủy bỏ. \newline    - \textbf{Tích hợp Thanh toán:} Chọn và cấu hình cổng thanh toán sẽ sử dụng cho việc đặt cọc. \newline 4. Sau khi thực hiện các thay đổi mong muốn, Người dùng chọn hành động "Lưu" (Save). \newline 5. Hệ thống kiểm tra tính hợp lệ của các giá trị nhập vào (ví dụ: tỷ lệ phần trăm hợp lệ, giờ hợp lệ). \newline 6. Hệ thống lưu lại các cấu hình mới. \newline 7. Hệ thống hiển thị thông báo lưu thành công. \\
\hline
Alternative Flow & \textbf{3a. Cấu hình theo từng Điểm bán hàng (POS):} \newline    1. Nếu hệ thống hỗ trợ nhiều điểm bán hàng/nhà hàng, một số cấu hình (ví dụ: giờ hoạt động, giá bàn) có thể cần được thiết lập riêng cho từng điểm. Người dùng cần chọn đúng điểm bán hàng trước khi cấu hình. \\
\hline
Exception Flow & \textbf{5a. Lỗi Xác thực Dữ liệu:} \newline    1. Người dùng nhập giá trị không hợp lệ (ví dụ: tỷ lệ phần trăm > 100, giờ kết thúc trước giờ bắt đầu). \newline    2. Hệ thống báo lỗi, chỉ rõ trường bị sai. \newline    3. Hệ thống không lưu cấu hình. Use Case quay lại bước 3. \newline \textbf{6a. Lỗi Hệ thống khi Lưu:} \newline    1. Hệ thống gặp sự cố kỹ thuật khi cố gắng lưu cấu hình. \newline    2. Hệ thống hiển thị thông báo lỗi chung. \newline    3. Use Case kết thúc không thành công. \\
\hline
\multicolumn{2}{|c|}{\textbf{2.3. Thông tin bổ sung (Additional Information)}} \\
\hline
\textbf{Mục} & \textbf{Nội dung} \\
\hline
Business Rule & - \textbf{BR-UC3.15-1 (V3):} Các tham số cấu hình này ảnh hưởng trực tiếp đến luồng đặt chỗ của khách hàng và cách hệ thống quản lý đặt chỗ. \newline - \textbf{BR-UC3.15-2 (V3):} Giá trị bàn (Table Price) là giá tham chiếu để tính tiền cọc bàn, không nhất thiết là giá thuê bàn thực tế. Cần có cơ chế nhập giá trị này cho từng bàn hoặc loại bàn. \newline - \textbf{BR-UC3.15-3 (V3):} Việc thay đổi các cấu hình này (ví dụ: giờ hoạt động, tỷ lệ cọc) sẽ có hiệu lực cho các lượt đặt chỗ mới sau khi lưu. Các lượt đặt chỗ cũ không bị ảnh hưởng (trừ khi có cơ chế cập nhật lại). \\
\hline
Non-Functional Requirement & - \textbf{NFR-UC3.15-1 (V3) (Usability):} Giao diện cấu hình phải được tổ chức logic, dễ tìm các tùy chọn. Các thuật ngữ sử dụng phải rõ ràng. Nên có giải thích ngắn (tooltip) cho các tùy chọn phức tạp. \newline - \textbf{NFR-UC3.15-2 (V3) (Flexibility):} Hệ thống nên cung cấp đủ các tham số cấu hình cần thiết để đáp ứng các quy tắc kinh doanh phổ biến của nhà hàng về đặt chỗ. \newline - \textbf{NFR-UC3.15-3 (V3) (Security):} Chỉ những người dùng có quyền hạn cao (Quản lý, Admin) mới được phép thay đổi các cấu hình quan trọng này. \\
\hline
\end{longtable}


