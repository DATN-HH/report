\documentclass[12pt, a4paper]{article}
\usepackage{mathptmx}[ptm]
\usepackage{vntex}
\usepackage{titlesec}
\usepackage{titletoc}
\usepackage{array}
\usepackage{multirow}
\usepackage{array}
\usepackage{svg}
\newcolumntype{P}[1]{>{\centering\arraybackslash}p{#1}}
\usepackage{longtable}
\usepackage{pdflscape}  % Để xoay trang thành landscape

\titleformat{\section}{\normalfont\Large\bfseries}{Chương \thesection}{1em}{}
\titlecontents{section}
  [0pt]{\vspace{1ex}}{\bfseries Chương \thecontentslabel \quad}{}
  {\hfill\bfseries\contentspage}

\usepackage[vietnamese]{babel}
\usepackage[utf8]{inputenc}
\usepackage{placeins}
\usepackage{tocbibind}
\usepackage{setspace} %Font, size
\onehalfspacing
\fontsize{13}{1}\selectfont

\usepackage{a4wide,amssymb,epsfig,latexsym,array,hhline,fancyhdr}
\usepackage[normalem]{ulem}
\usepackage{float}
\usepackage[makeroom]{cancel}
\usepackage{amsmath}
\usepackage{amsthm}
\usepackage{multicol}
\usepackage{array}
\usepackage{multicol,longtable,amscd}
\usepackage{diagbox}%Make diagonal lines in tables
\usepackage{booktabs}
\usepackage{alltt}
\usepackage[framemethod=tikz]{mdframed}% For highlighting paragraph backgrounds
\usepackage{caption,subcaption}
\usepackage{color}
\usepackage{indentfirst}
\usetikzlibrary{calc} 

\usepackage{hyperref}

\usepackage{lastpage}
\usepackage[lined,boxed,commentsnumbered]{algorithm2e}
\usepackage{enumerate}
\usepackage{color}
\usepackage{graphicx}							% Standard graphics package
\usepackage{array}
\usepackage{tabularx, caption}
\usepackage{multirow}
\usepackage{multicol}
\usepackage{rotating}
\usepackage{graphics}
\usepackage{geometry}
\usepackage{epsfig}
\usepackage{tikz}
\usepackage{listings}

\usepackage[perpage, hang, flushmargin]{footmisc}
\renewcommand\footnoterule{
    \vspace{1em}
    \hrule width \textwidth height 0.4pt
    \vspace{1em}
}
    
\usepackage{color} % tô màu cho code
\definecolor{dkgreen}{rgb}{0,0.6,0}
\definecolor{gray}{rgb}{0.5,0.5,0.5}
\definecolor{mauve}{rgb}{0.58,0,0.82}

\definecolor{codegreen}{rgb}{0,0.6,0}
\definecolor{codegray}{rgb}{0.5,0.5,0.5}
\definecolor{codepurple}{rgb}{0.58,0,0.82}
\definecolor{backcolour}{rgb}{0.95,0.95,0.92}
\usepackage{framed,color,verbatim}
\definecolor{shadecolor}{rgb}{.98, .98, .98}
\newenvironment{code}%
   {\snugshade\verbatim}%
   {\endverbatim\endsnugshade}
\lstset{frame=tb,
  language= C++,
  aboveskip=3mm,
  belowskip=3mm,
  showstringspaces=false,
  columns=flexible,
  basicstyle={\small\ttfamily},
  numbers=none,
  numberstyle=\tiny\color{gray},
  keywordstyle=\color{blue},
  commentstyle=\color{dkgreen},
  stringstyle=\color{mauve},
  breaklines=true,
  breakatwhitespace=true,
  tabsize=3
}

\usetikzlibrary{arrows,snakes,backgrounds}
\usepackage{hyperref}
\hypersetup{urlcolor=blue,linkcolor=black,citecolor=blue,colorlinks=true} 
\setlength\parindent{15pt}

\def\thesislayout{	% A4: 210 × 297
	\geometry{
		a4paper,
		total={160mm,240mm},  % fix over page
		left=30mm,
        right=20mm,
		top=20mm,
        bottom = 20mm
	}
}
\thesislayout

\pagestyle{fancy}
\fancyhead{} % clear all header fields
\fancyhead[L]{
 \begin{tabular}{rl}
    \begin{picture}(25,15)(0,0)
    \put(0,-8){\includegraphics[width=8mm, height=8mm]{Images/hcmut.png}}
    %\put(0,-8){\epsfig{width=10mm,figure=hcmut.eps}}
   \end{picture}&

	\begin{tabular}{l}
		\textbf{\bf \ttfamily Trường Đại Học Bách Khoa Tp.Hồ Chí Minh}\\
		\textbf{\bf \ttfamily Khoa Khoa Học và Kỹ Thuật Máy Tính}
	\end{tabular} 	
 \end{tabular}
}
\fancyhead[R]{
	\begin{tabular}{l}
		\tiny \bf \\
		\tiny \bf 
	\end{tabular}  }
\fancyfoot{} % clear all footer fields

\fancyfoot[R]{\scriptsize \ttfamily Trang {\thepage}/\pageref{LastPage}}
\renewcommand{\headrulewidth}{0.3pt}

\setcounter{secnumdepth}{4}
\setcounter{tocdepth}{3}
\makeatletter
\newcounter {subsubsubsection}[subsubsection]
\renewcommand\thesubsubsubsection{\thesubsubsection.\arabic{subsubsubsection}}
\newcommand\subsubsubsection{\@startsection{subsubsubsection}{4}{\z@}%
                                     {-3.25ex\@plus -1ex \@minus -.2ex}%
                                     {1.5ex \@plus .2ex}%
                                     {\normalfont\normalsize\bfseries}}
\newcommand*\l@subsubsubsection{\@dottedtocline{3}{10.0em}{4.1em}}
\newcommand*{\subsubsubsectionmark}[1]{}
\makeatother
\everymath{\color{black}}%make in-line maths symbols blue to read/check easily
\sloppy
\captionsetup[figure]{labelfont={small,bf},textfont={small,it},belowskip=-1pt,aboveskip=-9pt}
\captionsetup[table]{labelfont={small,bf},textfont={small,it},belowskip=-1pt,aboveskip=7pt}
\setlength{\floatsep}{5pt plus 2pt minus 2pt}
\setlength{\textfloatsep}{5pt plus 2pt minus 2pt}
\setlength{\intextsep}{10pt plus 2pt minus 2pt}
\thesislayout
\begin{document}

\begin{titlepage}
	\begin{tikzpicture}[remember picture, overlay]
		\draw[line width = 4pt] ($(current page.north west) + (3cm,-1.5cm)$) rectangle ($(current page.south east) + (-0.5cm,1.5cm)$);
		\draw[line width=1.5pt]
		($ (current page.north west) + (3.05cm,-1.55cm) $)
		rectangle
		($ (current page.south east) + (-0.55cm,1.55cm) $);
	\end{tikzpicture}
	\begin{center}
		\hspace{1.5cm}\textbf{ \fontfamily{ptm}\fontsize{15}{1}\selectfont{ĐẠI HỌC QUỐC GIA THÀNH PHỐ HỒ CHÍ MINH}} \\
		\hspace{1.5cm}\textbf{ \fontfamily{ptm}\fontsize{15}{1}\selectfont{TRƯỜNG ĐẠI HỌC BÁCH KHOA}} \\
		\hspace{1.5cm}\textbf{ \fontfamily{ptm}\fontsize{15}{1}\selectfont{KHOA KHOA HỌC VÀ KỸ THUẬT MÁY TÍNH}}\\
	\end{center}

	\vspace{1cm}
	\begin{figure}[h!]
		\begin{center}
			\hspace{1.5cm}\includegraphics[width=5cm]{Images/hcmut.png}
		\end{center}
	\end{figure}

	\begin{center}
		\hspace{1.5cm} \textbf{\fontfamily{ptm}\fontsize{15}{1}\selectfont{BÁO CÁO}} \\

		\hspace{1.5cm} \textbf{\fontfamily{ptm}\fontsize{15}{1}\selectfont{ĐỒ ÁN CHUYÊN NGÀNH}}\\

	\end{center}

	\begin{center}
		\begin{tabular}{c}

			\hspace{1cm}\textbf{{ \fontfamily{ptm}\fontsize{25}{1}\selectfont{{PHÁT TRIỂN HỆ THỐNG}}}} \\\\
			\hspace{1cm}\textbf{{ \fontfamily{ptm}\fontsize{25}{1}\selectfont{{QUẢN LÝ ĐẶT MÓN}}}}     \\\\
			\hspace{1cm}\textbf{{ \fontfamily{ptm}\fontsize{25}{1}\selectfont{{Ở NHÀ HÀNG ẨM THỰC}}}}  \\\\
		\end{tabular}
		\\ \vspace{0.5cm}
		{\hspace{1.5cm}\fontfamily{ptm}\fontsize{15}{1}\selectfont{ Ngành: KHOA HỌC MÁY TÍNH}}
	\end{center}
	\begin{table}[h]
		\begin{tabular}{rll}

			                &                           & \textbf{\fontfamily{ptm}\fontsize{15}{1}\selectfont{HỘI ĐỒNG: 4 KHOA HỌC MÁY TÍNH}}
			\vspace{2pt}
			\\

			                &                           & \textbf{\fontfamily{ptm}\fontsize{15}{1}\selectfont{GVHD: Assoc. Prof Võ Thị Ngọc Châu}}
			\vspace{2pt}
			\\
			\vspace{2pt}
			\\

			\\

			\hspace{4.5 cm} & $\;$ $\;$$\;$$\;$$\;$$\;$ & \Large \hspace{3 cm}     \textbf{\fontfamily{ptm}\fontsize{15}{1}\selectfont{---o0o---}}
			\vspace{6pt}
			\\

			\hspace{4.5 cm} &                           & \textbf{\fontfamily{ptm}\fontsize{15}{1}\selectfont{SVTH1: Nguyễn Trịnh Ngọc Huân - 2211144}}
			\vspace{2pt}
			\\
			                &                           & \textbf{\fontsize{15}{1}\fontfamily{ptm}\selectfont{SVTH2: Hồ Nguyễn Phi Hùng - 2211327 }}
			\vspace{1pt}
			\\
		\end{tabular}
	\end{table}

	\vspace{0.5 cm}
	\begin{center}
		\hspace{1.5cm}\fontfamily{ptm}\fontsize{15}{1}\selectfont{ Tp. Hồ Chí Minh, Tháng 05/2025}
	\end{center}
\end{titlepage}
\newpage

\pagestyle{plain}
\pagenumbering{roman}
\phantomsection
% \begin{table}[h!]
% \centering
% \begin{tabular}{|l|p{10cm}|}
% \hline
% \textbf{Chương} & \textbf{Nội dung bổ sung} \\ \hline
% Chương 2 & Thêm trang web so sánh, bổ sung kết luận. \\ \hline
% Chương 3 & Thêm các UseCase Schema, so sánh công nghệ, trình bày lý do thực tế sử dụng công nghệ trong dự án, bổ sung kiến trúc hệ thống. \\ \hline
% Chương 4 & Thêm sitemap. \\ \hline

% Chương 1 & Thêm Nhà hàng ăn uống cao cấp và Nhà hàng phục vụ đồ ăn nhanh \\ \hline
% Chương 3 & Thêm bối cảnh kinh doanh và chính sách vận hành \\ \hline
% \end{tabular}
% \caption{Bảng thể hiện sự thay đổi nội dung các chương}
% \label{tab:chapter_changes}
% \end{table}


\null\vfill
\section*{LỜI CAM ĐOAN}
\addcontentsline{toc}{section}{\textbf{Lời cam đoan}}
\hrule
\vspace{1 cm}


Chúng tôi xin cam đoan rằng hệ thống mà chúng tôi đã phát triển là kết quả của công sức và nỗ lực đội ngũ nghiên cứu của chúng tôi, được thực hiện dưới sự hướng dẫn và giám sát của Assoc. Prof Võ Thị Ngọc Châu. Chúng tôi muốn khẳng định rằng mọi thông tin, kiến thức và thiết kế liên quan đến đề tài này đều được nghiên cứu và phát triển một cách trung thực và độc lập. Chúng tôi không sao chép hoặc sử dụng bất kỳ nội dung nghiên cứu hay kết quả nào từ bất kỳ đề tài khác có tính tương đồng. \\

Chúng tôi cam kết duy trì tính minh bạch và trung thực trong quá trình thực hiện đồ án nghiên cứu này. Trách nhiệm đối với nội dung và kết quả của đồ án này hoàn toàn thuộc về nhóm chúng tôi. Nếu có bất kỳ sự phát hiện nào về việc gian lận hay không trung thực trong quá trình nghiên cứu và triển khai đồ án, chúng tôi sẽ chịu trách nhiệm hoàn toàn và sẵn sàng chấp nhận mọi hình thức kỷ luật theo quy định của Ban chủ nhiệm Khoa Học và Kỹ thuật Máy tính, cũng như Ban Giám hiệu Trường Đại học Bách Khoa, Đại học Quốc gia Thành phố Hồ Chí Minh.\\

Chúng tôi hi vọng rằng công trình nghiên cứu này sẽ đóng góp tích cực và mang lại giá trị cho cộng đồng nghiên cứu và xã hội nói chung. Đồng thời, chúng tôi mong muốn rằng sự cam kết của chúng tôi về trung thực và chất lượng sẽ được đánh giá và đồng thuận từ cộng đồng học thuật và nhà quản lý đào tạo.\\


\begin{flushright}
    NHÓM SINH VIÊN THỰC HIỆN ĐỀ TÀI
\end{flushright}
\vfill\null

\null\vfill
\vfill\null


\phantomsection
\null\vfill
\section*{LỜI CẢM ƠN}
\addcontentsline{toc}{section}{\textbf{Lời cảm ơn}}
\hrule
\vspace{1 cm}
Nhóm chúng tôi xin bày tỏ lòng biết ơn sâu sắc đến Assoc. Prof. Võ Thị Ngọc Châu - người đã tận tâm hướng dẫn và chỉ bảo trong quá trình thực hiện Đồ án "Phát triển hệ thống quản lý đặt món ở nhà hàng ẩm thực".

Chúng tôi xin chân thành cảm ơn sự hỗ trợ và định hướng từ cô Võ Thị Ngọc Châu, người đã giúp nhóm xây dựng nền tảng kiến thức vững chắc, đồng thời cung cấp những phản hồi và khích lệ quý báu trong suốt quá trình nghiên cứu.

Cảm ơn cô đã dành thời gian và công sức để hướng dẫn chúng tôi. Sự am hiểu, kiến thức sâu rộng và tâm huyết của cô đã tạo động lực lớn, giúp nhóm vượt qua khó khăn và phát triển kỹ năng nghiên cứu.

Chúng tôi xin gửi lời tri ân đến quý thầy cô thuộc Khoa Khoa học và Kỹ thuật Máy tính, Trường Đại học Bách Khoa - Đại học Quốc gia Thành phố Hồ Chí Minh, những người đã trang bị cho chúng tôi kiến thức nền tảng cần thiết để hoàn thành đề tài.

Nhóm chúng tôi trân trọng những kiến thức và kỹ năng học được từ cô Châu và quý thầy cô, đây sẽ là hành trang quý giá cho công việc và sự nghiệp tương lai.

Chân thành cảm ơn!

\begin{flushright}
    NHÓM SINH VIÊN THỰC HIỆN ĐỀ TÀI
\end{flushright}
\newpage

\addtocontents{toc}{\protect\setcounter{tocdepth}{-1}}
\tableofcontents
\addtocontents{toc}{\protect\setcounter{tocdepth}{3}} % Đặt lại mức độ hiển thị của mục lục

\newpage
\phantomsection
\renewcommand{\listfigurename}{\textbf{Danh sách hình vẽ}}
\listoffigures
\newpage
\phantomsection
\renewcommand{\listtablename}{\textbf{Danh sách bảng}}
\listoftables
\newpage
\clearpage
\pagestyle{fancy}
\pagenumbering{arabic}



% Bỏ comment phần này ra
% \section{GIỚI THIỆU}
% \subsection{Giới thiệu đề tài}
% \subsubsection{Tổng quan về ngành công nghiệp nhà hàng}
% Tính đến năm 2023, ngành dịch vụ thực phẩm (bao gồm khách sạn, nhà hàng và dịch vụ thể chế) đã đạt tổng giá trị 26,9 tỷ USD, với mức tăng trưởng 14,7\%. Mặc dù phải đối mặt với nhiều thách thức như lạm phát, chi phí vận hành tăng cao và sự cạnh tranh khốc liệt, ngành này đã phục hồi mạnh mẽ và gần như đạt doanh thu tương đương với thời kỳ trước đại dịch. Trong nửa đầu năm 2024, doanh thu từ dịch vụ lưu trú, thực phẩm và đồ uống đạt 24,1 tỷ USD, tăng 12,5\% so với cùng kỳ năm trước \cite{USDA}. 
% \\
% % cái ở trên ref từ https://apps.fas.usda.gov/newgainapi/api/Report/DownloadReportByFileName?fileName=Food%20Service%20-%20Hotel%20Restaurant%20Institutional%20Annual_Ho%20Chi%20Minh%20City_Vietnam_VM2024-0038.pdf

% \begin{figure}[H]
%     \centering
%     \includegraphics[width=15cm]{Images/resindus.png}
%     \caption{Doanh thu từ Dịch vụ Lưu trú, Ăn uống và Số lượng Du khách Quốc tế từ năm 2013 đến năm 2024. Nguồn: Post calculations; Vietnam’s General Statistics Office.}
%     \label{fig:enter-label}
% \end{figure}

% Trong bối cảnh công nghệ số ngày càng phát triển mạnh mẽ, việc áp dụng các giải pháp công nghệ vào quản lý và vận hành doanh nghiệp không chỉ là xu hướng mà còn là yếu tố quan trọng giúp nâng cao hiệu quả hoạt động và cải thiện trải nghiệm khách hàng. Đặc biệt, đối với ngành dịch vụ ăn uống, đặc biệt là các nhà hàng, việc quản lý quy trình đặt món và phục vụ khách hàng có thể gặp nhiều khó khăn và thách thức do tính chất phức tạp của công việc, cũng như yêu cầu phải đáp ứng nhu cầu đa dạng của khách hàng. \\

% Phương pháp quản lý truyền thống, dựa chủ yếu vào ghi chép thủ công và giao tiếp trực tiếp, đã bộc lộ nhiều hạn chế, chẳng hạn như dễ mắc sai sót, khó theo dõi lịch sử đặt món và tốn thời gian trong quá trình vận hành. Những vấn đề này không chỉ làm giảm hiệu quả công việc mà còn ảnh hưởng xấu đến trải nghiệm của khách hàng, đặc biệt là trong môi trường cạnh tranh ngày càng gay gắt trong ngành. \\

% Dự án này được triển khai nhằm giải quyết những vấn đề đó thông qua việc xây dựng một hệ thống quản lý đặt món hiện đại. Hệ thống này không chỉ tự động hóa các quy trình từ việc đặt món, quản lý đơn hàng đến thanh toán, mà còn giúp nhà hàng dễ dàng mở rộng các tính năng như quản lý kho, tối ưu hóa lịch làm việc của nhân viên và phân tích doanh thu. \\

% Việc triển khai hệ thống quản lý đặt món hiện đại sẽ mang lại nhiều lợi ích cho cả khách hàng và nhà hàng. Khách hàng có thể dễ dàng tiếp cận thực đơn, đặt món nhanh chóng và thanh toán tiện lợi. Còn đối với nhà hàng, việc này giúp nâng cao hiệu quả vận hành, giảm thiểu chi phí và tối ưu hóa nguồn lực. Đây không chỉ là một bước tiến quan trọng trong việc cải thiện quy trình kinh doanh mà còn là nền tảng giúp các nhà hàng sẵn sàng hội nhập và phát triển bền vững trong kỷ nguyên số. \\



% \subsubsection{Chuyển đổi số trong ngành công nghiệp nhà hàng ở Việt Nam}
% Sự tích hợp công nghệ đã và đang đóng vai trò then chốt trong việc thay đổi sâu sắc ngành công nghiệp nhà hàng, mang đến những bước đột phá trong hoạt động và trải nghiệm khách hàng. Các công nghệ hiện đại không chỉ giúp cải thiện hiệu quả vận hành mà còn tối ưu hóa các quy trình và nâng cao chất lượng dịch vụ. Dưới đây là những công nghệ nổi bật đang làm thay đổi ngành nhà hàng:
% \begin{enumerate}
%     \item \textbf{Hệ Thống POS (Điểm Bán Hàng)}: Các hệ thống POS hiện nay không chỉ đơn giản là máy tính tiền mà đã phát triển thành những giải pháp toàn diện, hỗ trợ các nhà hàng trong việc quản lý đơn hàng, theo dõi tồn kho và phân tích xu hướng bán hàng. Với sự hỗ trợ của POS, các chủ nhà hàng có thể đưa ra các quyết định sáng suốt, tối ưu hóa các hoạt động quản lý và tăng cường lợi thế cạnh tranh, giúp duy trì sự phát triển bền vững trong một ngành công nghiệp luôn thay đổi.
%     % \item \textbf{Trí Tuệ Nhân Tạo (AI)}: AI đang cách mạng hóa ngành nhà hàng bằng cách tự động hóa nhiều quy trình, từ giao đồ ăn cho đến việc thanh toán hóa đơn. Các công cụ AI giúp nhà hàng tối ưu hóa thực đơn, điều chỉnh giá cả và giảm thiểu lãng phí thực phẩm thông qua việc quản lý lịch trình sản xuất. Hơn thế nữa, AI còn góp phần nâng cao trải nghiệm khách hàng khi có thể hỗ trợ quản lý đặt chỗ, trả lời các câu hỏi của khách hàng một cách nhanh chóng và chính xác, từ đó giúp tăng mức độ hài lòng của khách hàng.
%     \item \textbf{Đặt Món Trực Tuyến (Online)}: Thói quen tiêu dùng tại Việt Nam đang thay đổi mạnh mẽ với sự ưa chuộng các giao dịch số và thanh toán trực tuyến. Dịch vụ giao đồ ăn trực tuyến đã trở thành một phần không thể thiếu trong thói quen của nhiều người, đặc biệt là ở các thành phố lớn. Theo thống kê, trong năm 2023, chi tiêu của người tiêu dùng Việt Nam cho dịch vụ giao đồ ăn tăng 30\%, đạt mức 1,4 tỷ USD, mức tăng trưởng cao nhất trong khu vực Đông Nam Á \cite{USDA}. Các nhà hàng đang tận dụng mạnh mẽ kênh giao đồ ăn trực tuyến để thúc đẩy doanh thu và thu hút khách hàng mới. Ví dụ, KFC đã mở thêm các cửa hàng ảo trên các nền tảng thương mại điện tử như Shopee Food và Grab Food, đồng thời sử dụng TikTok để livestream và cung cấp các ưu đãi hấp dẫn, gia tăng sự tương tác với khách hàng và đảm bảo giao hàng nhanh chóng trong vòng một giờ. 

% \end{enumerate}

% \begin{figure}[H]
%     \centering
%     \includegraphics[width=15cm]{Images/kiosViet.png}
%     \caption{Máy POS KiotViet D10. Nguồn: \href{https://www.kiotviet.vn/top-4-may-tinh-tien-cho-quan-an-quan-cafe-tot-nhat-hien-nay/}{kiotviet.vn}}
% \end{figure}

% Mặc dù công nghệ đã mang lại những bước tiến vượt bậc, ngành nhà hàng vẫn phải đối mặt với một số thách thức lớn. Lạm phát và chi phí thực phẩm tăng cao đang tạo ra áp lực nặng nề, khiến hơn 30.000 cơ sở nhà hàng phải đóng cửa trong nửa đầu năm 2024 \cite{USDA}. Thêm vào đó, thiếu hụt lao động, đặc biệt là nhân viên có tay nghề cao, và các vấn đề liên quan đến chuỗi cung ứng cũng khiến các nhà hàng gặp khó khăn trong việc duy trì hoạt động hiệu quả. Gián đoạn toàn cầu ảnh hưởng đến nguồn cung nguyên liệu, làm gia tăng chi phí và giảm tính linh hoạt trong quản lý. Tuy nhiên, những thách thức này cũng tạo cơ hội lớn để đổi mới và cải tiến. Bên cạnh đó, sự phát triển mạnh mẽ của các nền tảng giao đồ ăn trực tuyến và thương mại điện tử đã mở ra những kênh doanh thu mới, giúp ngành nhà hàng không chỉ duy trì mà còn gia tăng sự phát triển. \\

% Nhờ vào sự cải tiến liên tục trong công nghệ, ngành nhà hàng tại Việt Nam đã và đang chứng kiến những bước tiến vượt bậc trong việc nâng cao chất lượng dịch vụ và cải thiện trải nghiệm khách hàng. Các giải pháp như hệ thống POS hiện đại, ứng dụng di động để đặt món và theo dõi giao hàng, cũng như sự tích hợp trí tuệ nhân tạo đã giúp các nhà hàng hoạt động hiệu quả hơn và đáp ứng nhu cầu ngày càng cao của khách hàng. Sự gia tăng dịch vụ giao đồ ăn trực tuyến càng làm cho việc ăn uống trở nên tiện lợi hơn, đặc biệt là ở các khu vực đô thị, khi mà khách hàng có thể dễ dàng đặt món và thanh toán qua các ví điện tử. Các nhà hàng không chỉ đơn giản là cung cấp thực phẩm, mà còn đang mang lại trải nghiệm mua sắm thuận tiện, nhanh chóng và an toàn cho khách hàng. \\

% Ngành công nghiệp nhà hàng tại Việt Nam hiện nay đang ở một ngã rẽ quan trọng, khi công nghệ đóng vai trò trung tâm trong sự phát triển của ngành. Mặc dù vẫn còn những thách thức đáng kể, việc áp dụng công nghệ chiến lược mang lại cơ hội lớn cho các doanh nghiệp nhà hàng. Tích hợp các hệ thống POS tiên tiến, ứng dụng di động giúp tối ưu hóa các hoạt động, tạo ra những cải tiến đáng kể trong trải nghiệm khách hàng. Trong khi thị trường tiếp tục phát triển, việc ứng dụng công nghệ sẽ là yếu tố then chốt giúp các nhà hàng duy trì sự cạnh tranh và đáp ứng nhu cầu thay đổi của khách hàng trong kỷ nguyên số.








\subsection{Giới thiệu đề tài}

Tính đến năm 2023, ngành dịch vụ thực phẩm (bao gồm khách sạn, nhà hàng và dịch vụ thể chế) đã đạt tổng giá trị 26,9 tỷ USD, với mức tăng trưởng 14,7\%. Mặc dù phải đối mặt với nhiều thách thức như lạm phát, chi phí vận hành tăng cao và sự cạnh tranh khốc liệt, ngành này đã phục hồi mạnh mẽ và gần như đạt doanh thu tương đương với thời kỳ trước đại dịch. Trong nửa đầu năm 2024, doanh thu từ dịch vụ lưu trú, thực phẩm và đồ uống đạt 24,1 tỷ USD, tăng 12,5\% so với cùng kỳ năm trước \cite{USDA}. 

\begin{figure}[H]
    \centering
    \includegraphics[width=15cm]{Images/resindus.png}
    \caption{Doanh thu từ Dịch vụ Lưu trú, Ăn uống và Số lượng Du khách Quốc tế từ năm 2013 đến năm 2024. Nguồn: Post calculations; Vietnam’s General Statistics Office.}
    \label{fig:enter-label}
\end{figure}

Trong bối cảnh công nghệ số ngày càng phát triển mạnh mẽ, việc áp dụng các giải pháp công nghệ vào quản lý và vận hành doanh nghiệp không chỉ là xu hướng mà còn là yếu tố quan trọng giúp nâng cao hiệu quả hoạt động và cải thiện trải nghiệm khách hàng. Đặc biệt, đối với ngành dịch vụ ăn uống, đặc biệt là các nhà hàng, việc quản lý quy trình đặt món và phục vụ khách hàng có thể gặp nhiều khó khăn và thách thức do tính chất phức tạp của công việc, cũng như yêu cầu phải đáp ứng nhu cầu đa dạng của khách hàng.

Phương pháp quản lý truyền thống, dựa chủ yếu vào ghi chép thủ công và giao tiếp trực tiếp, đã bộc lộ nhiều hạn chế, chẳng hạn như dễ mắc sai sót, khó theo dõi lịch sử đặt món và tốn thời gian trong quá trình vận hành. Những vấn đề này không chỉ làm giảm hiệu quả công việc mà còn ảnh hưởng xấu đến trải nghiệm của khách hàng, đặc biệt là trong môi trường cạnh tranh ngày càng gay gắt trong ngành.

Sự tích hợp công nghệ đã và đang đóng vai trò then chốt trong việc thay đổi sâu sắc ngành công nghiệp nhà hàng, mang đến những bước đột phá trong hoạt động và trải nghiệm khách hàng. Các công nghệ hiện đại không chỉ giúp cải thiện hiệu quả vận hành mà còn tối ưu hóa các quy trình và nâng cao chất lượng dịch vụ. Dưới đây là những công nghệ nổi bật đang làm thay đổi ngành nhà hàng:
\begin{enumerate}
    \item \textbf{Hệ Thống POS (Điểm Bán Hàng)}: Các hệ thống POS hiện nay không chỉ đơn giản là máy tính tiền mà đã phát triển thành những giải pháp toàn diện, hỗ trợ các nhà hàng trong việc quản lý đơn hàng, theo dõi tồn kho và phân tích xu hướng bán hàng. Với sự hỗ trợ của POS, các chủ nhà hàng có thể đưa ra các quyết định sáng suốt, tối ưu hóa các hoạt động quản lý và tăng cường lợi thế cạnh tranh, giúp duy trì sự phát triển bền vững trong một ngành công nghiệp luôn thay đổi.
    \item \textbf{Đặt Món Trực Tuyến (Online)}: Thói quen tiêu dùng tại Việt Nam đang thay đổi mạnh mẽ với sự ưa chuộng các giao dịch số và thanh toán trực tuyến. Dịch vụ giao đồ ăn trực tuyến đã trở thành một phần không thể thiếu trong thói quen của nhiều người, đặc biệt là ở các thành phố lớn. Theo thống kê, trong năm 2023, chi tiêu của người tiêu dùng Việt Nam cho dịch vụ giao đồ ăn tăng 30\%, đạt mức 1,4 tỷ USD, mức tăng trưởng cao nhất trong khu vực Đông Nam Á \cite{USDA}. Các nhà hàng đang tận dụng mạnh mẽ kênh giao đồ ăn trực tuyến để thúc đẩy doanh thu và thu hút khách hàng mới. Ví dụ, KFC đã mở thêm các cửa hàng ảo trên các nền tảng thương mại điện tử như Shopee Food và Grab Food, đồng thời sử dụng TikTok để livestream và cung cấp các ưu đãi hấp dẫn, gia tăng sự tương tác với khách hàng và đảm bảo giao hàng nhanh chóng trong vòng một giờ.
\end{enumerate}

\begin{figure}[H]
    \centering
    \includegraphics[width=10cm]{Images/kiosViet.png}
    \caption{Máy POS KiotViet D10. Nguồn: \href{https://www.kiotviet.vn/top-4-may-tinh-tien-cho-quan-an-quan-cafe-tot-nhat-hien-nay/}{kiotviet.vn}}
\end{figure}

Mặc dù công nghệ đã mang lại những bước tiến vượt bậc, ngành nhà hàng vẫn phải đối mặt với một số thách thức lớn. Lạm phát và chi phí thực phẩm tăng cao đang tạo ra áp lực nặng nề, khiến hơn 30.000 cơ sở nhà hàng phải đóng cửa trong nửa đầu năm 2024 \cite{USDA}. Thêm vào đó, thiếu hụt lao động, đặc biệt là nhân viên có tay nghề cao, và các vấn đề liên quan đến chuỗi cung ứng cũng khiến các nhà hàng gặp khó khăn trong việc duy trì hoạt động hiệu quả. Gián đoạn toàn cầu ảnh hưởng đến nguồn cung nguyên liệu, làm gia tăng chi phí và giảm tính linh hoạt trong quản lý. Tuy nhiên, những thách thức này cũng tạo cơ hội lớn để đổi mới và cải tiến. Bên cạnh đó, sự phát triển mạnh mẽ của các nền tảng giao đồ ăn trực tuyến và thương mại điện tử đã mở ra những kênh doanh thu mới, giúp ngành nhà hàng không chỉ duy trì mà còn gia tăng sự phát triển.

Dự án này được triển khai nhằm giải quyết những vấn đề đó thông qua việc xây dựng một hệ thống quản lý đặt món hiện đại. Hệ thống này không chỉ tự động hóa các quy trình từ việc đặt món, quản lý đơn hàng đến thanh toán, mà còn giúp nhà hàng dễ dàng mở rộng các tính năng như tối ưu hóa lịch làm việc của nhân viên và phân tích doanh thu.

Việc triển khai hệ thống quản lý đặt món hiện đại sẽ mang lại nhiều lợi ích cho cả khách hàng và nhà hàng. Khách hàng có thể dễ dàng tiếp cận thực đơn, đặt món nhanh chóng và thanh toán tiện lợi. Còn đối với nhà hàng, việc này giúp nâng cao hiệu quả vận hành, giảm thiểu chi phí và tối ưu hóa nguồn lực. Đây không chỉ là một bước tiến quan trọng trong việc cải thiện quy trình kinh doanh mà còn là nền tảng giúp các nhà hàng sẵn sàng hội nhập và phát triển bền vững trong kỷ nguyên số.

Nhờ vào sự cải tiến liên tục trong công nghệ, ngành nhà hàng tại Việt Nam đã và đang chứng kiến những bước tiến vượt bậc trong việc nâng cao chất lượng dịch vụ và cải thiện trải nghiệm khách hàng. Các giải pháp như hệ thống POS hiện đại, ứng dụng di động để đặt món và theo dõi giao hàng đã giúp các nhà hàng hoạt động hiệu quả hơn và đáp ứng nhu cầu ngày càng cao của khách hàng. Sự gia tăng dịch vụ giao đồ ăn trực tuyến càng làm cho việc ăn uống trở nên tiện lợi hơn, đặc biệt là ở các khu vực đô thị, khi mà khách hàng có thể dễ dàng đặt món và thanh toán qua các ví điện tử. Các nhà hàng không chỉ đơn giản là cung cấp thực phẩm, mà còn đang mang lại trải nghiệm mua sắm thuận tiện, nhanh chóng và an toàn cho khách hàng.

Ngành công nghiệp nhà hàng tại Việt Nam hiện nay đang ở một ngã rẽ quan trọng, khi công nghệ đóng vai trò trung tâm trong sự phát triển của ngành. Mặc dù vẫn còn những thách thức đáng kể, việc áp dụng công nghệ chiến lược mang lại cơ hội lớn cho các doanh nghiệp nhà hàng. Tích hợp các hệ thống POS tiên tiến, ứng dụng di động giúp tối ưu hóa các hoạt động, tạo ra những cải tiến đáng kể trong trải nghiệm khách hàng. Trong khi thị trường tiếp tục phát triển, việc ứng dụng công nghệ sẽ là yếu tố then chốt giúp các nhà hàng duy trì sự cạnh tranh và đáp ứng nhu cầu thay đổi của khách hàng trong kỷ nguyên số.
\input{Sections/gioi_thieu/tim_hieu_ve_restaurant_industry}
% % \subsection{Động lực}

% % Việc phát triển hệ thống quản lý đặt món trong nhà hàng xuất phát từ nhu cầu thực tế và những lợi ích mà nó mang lại:

% % \begin{enumerate}
% %     \item Xu hướng chuyển đổi số trong ngành dịch vụ ăn uống: Ngày nay, việc ứng dụng công nghệ vào các lĩnh vực kinh doanh không chỉ là một lựa chọn mà đã trở thành một yêu cầu bắt buộc. Các nhà hàng cần áp dụng các giải pháp hiện đại để nâng cao hiệu quả hoạt động và mang lại trải nghiệm tốt hơn cho khách hàng.

% %     \item Tăng trải nghiệm khách hàng: Một hệ thống quản lý đặt món tự động giúp khách hàng dễ dàng tiếp cận thực đơn, tùy chỉnh các lựa chọn, đặt món và thanh toán nhanh chóng. Điều này góp phần tạo ấn tượng tích cực và giữ chân khách hàng lâu dài.

% %     \item Giảm áp lực cho nhân viên: Nhân viên nhà hàng thường gặp khó khăn khi phải xử lý nhiều đơn hàng cùng lúc, đặc biệt vào giờ cao điểm. Hệ thống tự động hóa giúp giảm thiểu sai sót và tăng năng suất làm việc của nhân viên.

% %     \item Tăng tính cạnh tranh trên thị trường: Với sự phát triển của các mô hình nhà hàng thông minh, việc áp dụng công nghệ không chỉ giúp tối ưu hóa hoạt động mà còn tạo lợi thế cạnh tranh với các đối thủ.

% %     \item Hỗ trợ quản lý hiệu quả: Một hệ thống hiện đại không chỉ dừng lại ở việc quản lý đặt món mà còn cung cấp các công cụ phân tích, báo cáo giúp nhà quản lý đưa ra các quyết định kinh doanh chiến lược dựa trên dữ liệu thực tế.
% % \end{enumerate}

% % Động lực từ thực tiễn và xu hướng công nghệ chính là yếu tố thúc đẩy việc thực hiện đề tài này, nhằm giải quyết các vấn đề hiện tại và hướng đến sự phát triển bền vững.

% \subsection{Động lực}

% Việc phát triển hệ thống quản lý đặt món trong nhà hàng xuất phát từ nhu cầu thực tế và những lợi ích mà nó mang lại, đặc biệt trong bối cảnh ngành công nghiệp nhà hàng tại Việt Nam đang chuyển đổi số mạnh mẽ như đã đề cập:

% \begin{enumerate}
%     \item \textbf{Xu hướng chuyển đổi số trong ngành dịch vụ ăn uống}: Như đã trình bày, sự tích hợp công nghệ, chẳng hạn như hệ thống POS và dịch vụ đặt món trực tuyến, đã trở thành yếu tố then chốt giúp nhà hàng nâng cao hiệu quả và đáp ứng nhu cầu khách hàng. Việc áp dụng một hệ thống quản lý đặt món hiện đại là một bước đi cần thiết để theo kịp xu hướng này.
    
%     \item \textbf{Tăng trải nghiệm khách hàng}: Hệ thống tự động hóa giúp khách hàng dễ dàng tiếp cận thực đơn, tùy chỉnh món ăn, đặt hàng nhanh chóng và thanh toán tiện lợi qua các nền tảng số, như đã thấy trong sự phát triển của dịch vụ giao đồ ăn trực tuyến. Điều này không chỉ cải thiện trải nghiệm mà còn xây dựng lòng trung thành của khách hàng.
    
%     \item \textbf{Giảm áp lực cho nhân viên}: Trong môi trường cạnh tranh và áp lực cao của ngành nhà hàng, nhân viên thường phải xử lý nhiều đơn hàng cùng lúc. Hệ thống quản lý đặt món giúp giảm thiểu sai sót, tối ưu hóa quy trình phục vụ, từ đó nâng cao năng suất làm việc, như đã đề cập trong các hạn chế của phương pháp quản lý truyền thống.
    
%     \item \textbf{Tăng tính cạnh tranh trên thị trường}: Với sự gia tăng của các mô hình nhà hàng thông minh và kênh giao đồ ăn trực tuyến như Shopee Food hay Grab Food, việc áp dụng công nghệ giúp nhà hàng tạo lợi thế cạnh tranh, thu hút khách hàng mới và duy trì vị thế trên thị trường.
    
%     \item \textbf{Hỗ trợ quản lý hiệu quả}: Hệ thống không chỉ tự động hóa quy trình đặt món mà còn cung cấp các công cụ phân tích doanh thu, quản lý kho và tối ưu hóa lịch làm việc, giúp nhà quản lý đưa ra quyết định chiến lược dựa trên dữ liệu thực tế, như vai trò của hệ thống POS đã được đề cập.
% \end{enumerate}

% Động lực từ nhu cầu thực tiễn, xu hướng công nghệ và những thách thức trong vận hành nhà hàng chính là yếu tố thúc đẩy việc thực hiện đề tài này, nhằm mang lại giải pháp toàn diện và bền vững.



\subsection{Động lực}

Việc nghiên cứu và phát triển một hệ thống quản lý đặt món hiện đại cho nhà hàng không chỉ là một đề xuất cải tiến mà còn xuất phát từ những động lực cấp thiết và cơ hội chiến lược, bám sát vào thực trạng và xu hướng của ngành dịch vụ ẩm thực tại Việt Nam như đã được phân tích chi tiết trong phần giới thiệu:

\begin{enumerate}
    \item \textbf{Yêu cầu cấp bách từ làn sóng Chuyển đổi số trong ngành Food\&Beverage:} Như đã nêu, ngành dịch vụ thực phẩm Việt Nam (đạt 26,9 tỷ USD năm 2023) đang chứng kiến sự chuyển mình mạnh mẽ nhờ công nghệ. Sự phổ biến của các hệ thống POS tiên tiến và sự bùng nổ của thị trường đặt món trực tuyến (đạt 1,4 tỷ USD, tăng 30\% năm 2023 \cite{USDA}) không chỉ là xu hướng mà đã trở thành một phần quan trọng trong mô hình kinh doanh hiện đại. Việc không tích hợp công nghệ, cụ thể là một hệ thống quản lý đặt món hiệu quả, đồng nghĩa với việc nhà hàng tự đặt mình vào thế bất lợi, có nguy cơ bị tụt hậu so với các đối thủ đang nhanh chóng thích nghi và khai thác lợi ích từ công nghệ số. Do đó, phát triển hệ thống này là một bước đi tất yếu để hòa nhập và cạnh tranh.

    \item \textbf{Khắc phục những hạn chế cố hữu của phương pháp quản lý truyền thống:} Quy trình vận hành dựa trên ghi chép thủ công và giao tiếp trực tiếp, như đã đề cập, tiềm ẩn nhiều rủi ro: sai sót trong ghi nhận đơn hàng, nhầm lẫn trong quá trình chuyển thông tin đến bếp, khó khăn trong việc theo dõi lịch sử và sở thích khách hàng, và lãng phí thời gian của cả nhân viên lẫn khách hàng. Những sai sót này không chỉ gây thất thoát doanh thu mà còn làm tổn hại đến uy tín thương hiệu và sự hài lòng của khách hàng. Một hệ thống quản lý đặt món tự động hóa sẽ giải quyết các nút thắt này, đảm bảo tính chính xác, minh bạch và hiệu quả cho toàn bộ quy trình từ lúc khách chọn món đến khi hoàn tất thanh toán.

    \item \textbf{Nâng tầm trải nghiệm khách hàng trong kỷ nguyên số và tiện ích:} Khách hàng hiện đại, đặc biệt tại các đô thị, ngày càng đề cao sự tiện lợi, tốc độ và khả năng kiểm soát trong trải nghiệm dịch vụ. Họ đã quen với các giao dịch số và mong muốn quy trình đặt món tại nhà hàng cũng mượt mà như vậy. Hệ thống quản lý đặt món hiện đại đáp ứng kỳ vọng này bằng cách cho phép khách hàng dễ dàng xem thực đơn (có thể kèm hình ảnh, mô tả chi tiết), tùy chỉnh món ăn theo sở thích cá nhân, đặt hàng nhanh chóng (có thể ngay tại bàn qua thiết bị di động hoặc kiosk) và thanh toán thuận tiện qua nhiều hình thức điện tử. Việc giảm thiểu thời gian chờ đợi, đảm bảo đơn hàng chính xác và mang lại sự chủ động cho khách hàng chính là yếu tố then chốt để tạo ấn tượng tích cực, xây dựng lòng trung thành và khuyến khích họ quay trở lại.

    \item \textbf{Tối ưu hóa hiệu quả vận hành và tăng cường năng lực cạnh tranh trong bối cảnh thách thức:} Ngành nhà hàng đang đối mặt với áp lực không nhỏ từ lạm phát, chi phí nguyên liệu và vận hành tăng cao (thể hiện qua con số hơn 30.000 cơ sở đóng cửa \cite{USDA}), cùng với tình trạng thiếu hụt lao động có kỹ năng. Trong bối cảnh đó, việc tối ưu hóa nguồn lực và quy trình là yếu tố sống còn. Hệ thống quản lý đặt món giúp giảm tải đáng kể công việc thủ công cho nhân viên phục vụ và thu ngân, giảm thiểu sai sót do con người, cho phép phân bổ nhân sự hiệu quả hơn. Quan trọng hơn, hệ thống cung cấp dữ liệu vận hành chi tiết (tương tự vai trò phân tích của POS), giúp nhà quản lý nắm bắt xu hướng tiêu dùng, quản lý tốt hơn, đưa ra các quyết định về giá cả, thực đơn và chiến lược kinh doanh dựa trên bằng chứng cụ thể, từ đó nâng cao hiệu quả và tạo lợi thế cạnh tranh sắc bén.
\end{enumerate}

Tóm lại, động lực cốt lõi thúc đẩy việc thực hiện đề tài này là sự cộng hưởng mạnh mẽ giữa nhu cầu giải quyết các bài toán thực tiễn trong quản lý nhà hàng, yêu cầu phải bắt kịp và tận dụng xu thế chuyển đổi số không thể đảo ngược, cùng với khát vọng nâng cao chất lượng dịch vụ, tối ưu hóa hiệu quả hoạt động và xây dựng lợi thế cạnh tranh bền vững cho các doanh nghiệp trong ngành ẩm thực Việt Nam.
% % \subsection{Thách thức}
% % Trong quá trình phát triển và triển khai hệ thống quản lý đặt món cho nhà hàng, có thể đối mặt với nhiều thách thức, bao gồm:

% % \begin{enumerate}
% %     \item Đáp ứng đa dạng nhu cầu người dùng: Khách hàng, nhân viên phục vụ, quản lý chi nhánh, và quản lý tổng đều có các yêu cầu sử dụng khác nhau. Thiết kế hệ thống phải đảm bảo dễ sử dụng cho khách hàng, đồng thời cung cấp đủ công cụ và dữ liệu cho quản lý và nhân viên.

% %     \item Tích hợp với cơ sở hạ tầng hiện có: Nhiều nhà hàng đã có hệ thống quản lý cơ bản hoặc sử dụng các phần mềm khác. Việc tích hợp hệ thống mới với các công cụ hiện tại, như máy in hóa đơn, phần mềm kế toán hoặc thiết bị quét QR, là một thách thức cần giải quyết.

% %     \item Quản lý dữ liệu lớn: Khi hệ thống hoạt động tại nhiều chi nhánh với số lượng khách hàng lớn, việc lưu trữ, xử lý và bảo mật dữ liệu trở thành một bài toán quan trọng, đặc biệt đối với thông tin cá nhân và thanh toán.

% %     \item Bảo mật thông tin: Với việc tích hợp thanh toán qua QR hoặc các hình thức điện tử, việc đảm bảo an toàn dữ liệu giao dịch và thông tin khách hàng là một ưu tiên hàng đầu, đồng thời là một thách thức lớn trước các nguy cơ tấn công mạng.

% %     \item Đào tạo và thay đổi thói quen người dùng: Nhân viên và khách hàng có thể chưa quen với việc sử dụng công nghệ mới. Điều này đòi hỏi một quá trình đào tạo bài bản và hỗ trợ liên tục để đảm bảo mọi người đều sử dụng hệ thống hiệu quả.

% %     \item Quản lý lỗi và vận hành liên tục: Hệ thống phải hoạt động ổn định và có cơ chế xử lý lỗi nhanh chóng, đặc biệt trong các giờ cao điểm. Bất kỳ sự cố nào cũng có thể ảnh hưởng đến trải nghiệm khách hàng và hoạt động của nhà hàng.
% % \end{enumerate}

% % Những thách thức này đòi hỏi nhóm phải có kế hoạch chi tiết, giải pháp linh hoạt, ... để đảm bảo hệ thống hoạt động hiệu quả và mang lại giá trị cao nhất.


% \subsection{Thách thức}

% Trong quá trình phát triển và triển khai hệ thống quản lý đặt món, nhóm phải đối mặt với nhiều thách thức, đặc biệt khi liên kết với các vấn đề và giải pháp đã nêu trong phần giới thiệu:

% \begin{enumerate}
%     \item \textbf{Đáp ứng đa dạng nhu cầu người dùng}: Như đã đề cập, khách hàng yêu cầu trải nghiệm tiện lợi, trong khi nhà quản lý cần dữ liệu phân tích chi tiết và nhân viên cần giao diện dễ sử dụng. Thiết kế hệ thống phải cân bằng giữa tính đơn giản cho khách hàng và tính năng chuyên sâu cho quản lý, đảm bảo đáp ứng các nhu cầu đa dạng.
    
%     \item \textbf{Tích hợp với cơ sở hạ tầng hiện có}: Nhiều nhà hàng tại Việt Nam đã sử dụng các hệ thống POS hoặc phần mềm quản lý cơ bản, như KiotViet D10. Việc tích hợp hệ thống mới với các thiết bị hiện tại, chẳng hạn như máy in hóa đơn hoặc phần mềm kế toán, đòi hỏi sự tương thích cao để tránh gián đoạn vận hành.
    
%     \item \textbf{Quản lý dữ liệu lớn}: Với sự gia tăng giao dịch qua các nền tảng số, như dịch vụ giao đồ ăn trực tuyến, hệ thống cần xử lý và lưu trữ khối lượng dữ liệu lớn từ nhiều chi nhánh, đồng thời đảm bảo hiệu suất và bảo mật thông tin khách hàng.
    
%     \item \textbf{Bảo mật thông tin}: Trong bối cảnh thanh toán điện tử và giao dịch qua QR đang phổ biến, việc bảo vệ dữ liệu giao dịch và thông tin cá nhân trước các nguy cơ tấn công mạng là một thách thức lớn, đặc biệt khi ngành nhà hàng ngày càng phụ thuộc vào công nghệ số.
    
%     \item \textbf{Đào tạo và thay đổi thói quen người dùng}: Như đã nêu, phương pháp quản lý truyền thống vẫn phổ biến ở nhiều nhà hàng. Việc chuyển đổi sang hệ thống mới đòi hỏi đào tạo nhân viên và hướng dẫn khách hàng, đặc biệt với những người chưa quen sử dụng công nghệ, để đảm bảo hiệu quả sử dụng.
    
%     \item \textbf{Quản lý lỗi và vận hành liên tục}: Hệ thống phải hoạt động ổn định, đặc biệt trong giờ cao điểm, để tránh ảnh hưởng đến trải nghiệm khách hàng. Cơ chế xử lý lỗi nhanh chóng là cần thiết để duy trì hiệu quả vận hành, như yêu cầu về tính linh hoạt trong quản lý đã được đề cập.
% \end{enumerate}

% Những thách thức này đòi hỏi nhóm phát triển phải có kế hoạch chi tiết, giải pháp linh hoạt và sự phối hợp chặt chẽ để đảm bảo hệ thống không chỉ khắc phục các hạn chế hiện tại mà còn mang lại giá trị tối ưu cho ngành nhà hàng.

\subsection{Thách thức}

Mặc dù việc phát triển một hệ thống quản lý đặt món hiện đại mang lại nhiều lợi ích và phù hợp với xu hướng chuyển đổi số, quá trình xây dựng và triển khai hệ thống này cũng đối mặt với không ít thách thức đáng kể, đòi hỏi sự chuẩn bị kỹ lưỡng và giải pháp phù hợp:

\begin{enumerate}
    \item \textbf{Độ phức tạp về mặt kỹ thuật và tích hợp hệ thống:}
    Việc xây dựng một hệ thống toàn diện đòi hỏi xử lý nhiều nghiệp vụ phức tạp: quản lý thực đơn đa dạng (món lẻ, combo, tùy chọn gia giảm, topping), quản lý bàn và khu vực phục vụ, xử lý nhiều loại đơn hàng (tại chỗ, mang về, giao hàng), đồng bộ trạng thái đơn hàng giữa các bộ phận (phục vụ, bếp, thu ngân), và tích hợp với các hệ thống thanh toán khác nhau (tiền mặt, thẻ, ví điện tử).

    \item \textbf{Chi phí đầu tư ban đầu và chi phí duy trì:}
    Việc triển khai một hệ thống mới thường đi kèm với các chi phí duy trì định kỳ như nâng cấp hệ thống, phí dịch vụ lưu trữ (cloud) và hỗ trợ kỹ thuật.

    \item \textbf{Thiết kế trải nghiệm người dùng (UX/UI) tối ưu:}
    Hệ thống cần có giao diện thân thiện, trực quan và dễ sử dụng cho cả nhân viên và khách hàng (nếu có giao diện đặt món tự phục vụ). Một thiết kế UX/UI phức tạp, khó hiểu sẽ làm giảm hiệu quả công việc của nhân viên và gây khó chịu cho khách hàng, thậm chí dẫn đến việc từ bỏ sử dụng hệ thống. Việc cân bằng giữa tính năng đa dạng và sự đơn giản trong thiết kế, đồng thời đảm bảo tính nhất quán trên các nền tảng khác nhau (web, app, kiosk) là một thách thức không nhỏ.

    \item \textbf{Bảo mật dữ liệu và quyền riêng tư:}
    Hệ thống sẽ lưu trữ và xử lý nhiều dữ liệu nhạy cảm, bao gồm thông tin đơn hàng, dữ liệu bán hàng, thông tin cá nhân và lịch sử giao dịch của khách hàng, thông tin thanh toán. Việc đảm bảo an ninh, an toàn cho các dữ liệu này, chống lại các nguy cơ tấn công mạng, truy cập trái phép, mã hóa dữ liệu và tuân thủ các quy định pháp luật về bảo vệ dữ liệu cá nhân (như Nghị định 13/2023/NĐ-CP của Việt Nam) là một yêu cầu bắt buộc và là thách thức liên tục trong suốt vòng đời của hệ thống.

    % \item \textbf{Giới hạn về thời gian hoàn thành dự án:}
    % Việc phát triển một hệ thống phức tạp đòi hỏi thời gian nghiên cứu, thiết kế, lập trình, kiểm thử và triển khai kỹ lưỡng. Việc cân bằng giữa tốc độ phát triển và chất lượng sản phẩm, đồng thời quản lý hiệu quả các mốc thời gian và đối phó với những thay đổi yêu cầu phát sinh là một thách thức lớn đối với đội ngũ phát triển.

    % \item \textbf{Nguồn nhân lực phát triển và quản lý dự án:}
    % Việc xây dựng và duy trì kinh nghiệm về các công nghệ cần thiết (backend, frontend, database, mobile, DevOps, security) và hiểu biết về nghiệp vụ nhà hàng là một thách thức. Ngoài ra, việc quản lý dự án hiệu quả, điều phối công việc giữa các thành viên, và giao tiếp thông suốt cũng đóng vai trò quan trọng nhưng không hề dễ dàng.
\end{enumerate}

Việc nhận diện và có kế hoạch đối phó với những thách thức này ngay từ giai đoạn đầu của dự án là yếu tố then chốt để đảm bảo hệ thống quản lý đặt món được phát triển thành công, triển khai hiệu quả và mang lại giá trị thực sự cho nhà hàng.

% \subsection{Mục tiêu đề tài}

% Nhằm vượt qua những thách thức cố hữu và khai thác tối đa tiềm năng trong từng giai đoạn của hành trình khách hàng, nhóm nghiên cứu đã phát triển một ứng dụng tích hợp, được thiết kế chuyên biệt để nâng tầm trải nghiệm cho khách hàng, nhân viên và quản lý nhà hàng. Ứng dụng này tập trung vào việc cải thiện các điểm chạm then chốt trong hành trình khách hàng, tạo ra một quy trình liền mạch và hiệu quả hơn:
% \begin{enumerate}
%     \item Giai đoạn Cân nhắc (Consideration):     Ứng dụng cho phép khách hàng dễ dàng truy cập thực đơn trực tuyến chi tiết, hình ảnh chất lượng cao về món ăn và không gian nhà hàng, cũng như các đánh giá từ những khách hàng khác. Chức năng so sánh giá cả và tùy chọn tùy chỉnh món ăn giúp khách hàng đưa ra quyết định sáng suốt..
%     \item Giai đoạn Trải nghiệm (Experience): Ứng dụng cung cấp các tính năng hỗ trợ trong quá trình trải nghiệm tại nhà hàng, bao gồm một vài tính năng cơ bản như:
%         \begin{itemize}
%             \item \textit{Gọi món qua ứng dụng:} Khách hàng có thể dễ dàng xem thực đơn và gọi món trên điện thoại thông minh của mình.
%             \item \textit{Gửi yêu cầu trực tiếp đến nhân viên phục vụ:} Giúp tăng cường tương tác và giải quyết các vấn đề nhanh chóng.
%             \item \textit{Thu thập phản hồi thời gian thực:} Cho phép nhà hàng đánh giá và cải thiện chất lượng dịch vụ ngay lập tức.
%         \end{itemize}
%     \item Giai đoạn Thanh toán (Payment): Ứng dụng tích hợp nhiều phương thức thanh toán điện tử (thẻ tín dụng, ví điện tử), cho phép khách hàng thanh toán nhanh chóng và an toàn. Hóa đơn điện tử được tạo tự động và gửi trực tiếp đến khách hàng.
%     \item Giai đoạn Hậu trải nghiệm (Post-Experience): Ứng dụng hỗ trợ gửi email cảm ơn tự động, khuyến khích khách hàng để lại đánh giá và tham gia chương trình khách hàng thân thiết. Dữ liệu về lịch sử mua hàng và sở thích của khách hàng được sử dụng để cá nhân hóa các ưu đãi và khuyến mãi, tăng khả năng quay lại của khách hàng.

% \end{enumerate}

% \begin{figure}[H]
%     \centering
%     \includegraphics[width=15cm]{Images/restaurant-customer-journey.png}
%     \vspace{0.5cm}
%     \caption{Giới thiệu về Restaunrant Customer Journey}
%     \label{fig:my_label}
% \end{figure}

% % \subsubsection{Lí do chọn mục tiêu "Tích hợp và tối ưu hóa hành trình khách hàng trong dịch vụ nhà hàng"}

% Trong bối cảnh ngành dịch vụ nhà hàng ngày càng cạnh tranh khốc liệt, việc tối ưu hóa hành trình khách hàng không chỉ dừng lại ở việc nâng cao trải nghiệm mà còn trở thành một chiến lược then chốt để cải thiện hiệu quả vận hành và gia tăng lợi thế cạnh tranh. Hệ thống này tập trung vào việc tối ưu hóa các giai đoạn quan trọng trong hành trình khách hàng, bao gồm Cân nhắc, Quyết định, Trải nghiệm, Thanh toán và Hậu trải nghiệm, nhằm mang đến một quy trình đặt món, phục vụ và thanh toán liền mạch, nhanh chóng và thuận tiện. Bên cạnh đó, nhóm cũng hướng đến những mục tiêu song song quan trọng khác, bao gồm:

% \textbf{Đối với Khách hàng:}
% \begin{itemize}
%     \item \textit{Tối ưu trải nghiệm người dùng:}  Cung cấp một giao diện trực quan, thân thiện và dễ sử dụng, cho phép khách hàng dễ dàng khám phá thực đơn, thực hiện đặt món và thanh toán một cách nhanh chóng.
%     \item \textit{Tối ưu hóa trải nghiệm trong các mốc hành tình người dùng:}  Tích hợp các tính năng tiện lợi như quét mã QR để truy cập thực đơn số, đặt món theo thời gian thực và lựa chọn thanh toán linh hoạt, mang đến trải nghiệm nhất quán và không gián đoạn.
%     \item \textit{Minh bạch và tin cậy:}  Đảm bảo tính minh bạch tuyệt đối và rõ ràng trong mọi thông tin, bao gồm giá cả, các chương trình ưu đãi đặc biệt và lịch sử đặt món chi tiết.
% \end{itemize}

% \textbf{Đối với Nhân viên Nhà hàng:}
% \begin{itemize}
%     \item \textit{Nâng cao hiệu suất và độ chính xác:}  Hỗ trợ nhân viên trong việc theo dõi và xử lý các đơn hàng một cách nhanh chóng, chính xác, giảm thiểu sai sót và tăng năng suất.
%     \item \textit{Giảm tải công việc thủ công:}  Tích hợp các công cụ quản lý thông minh để giảm bớt gánh nặng công việc thủ công, từ đó giải phóng thời gian và nguồn lực cho các hoạt động quan trọng khác.
% \end{itemize}

% \textbf{Đối với Nhà Quản lý:}
% \begin{itemize}
%     \item \textit{Cung cấp công cụ hỗ trợ báo cáo và phân tích chi tiết:}  Giúp quản lý dễ dàng theo dõi doanh thu, phân tích xu hướng bán hàng và đánh giá hiệu quả hoạt động tổng thể của nhà hàng.
%     \item \textit{Quản lý hiệu quả nguồn lực:}  Tích hợp chức năng quản lý kho nguyên liệu và nhân sự, hỗ trợ việc ra quyết định nhanh chóng và hiệu quả, tối ưu hóa chi phí và đảm bảo nguồn cung ổn định.
%     \item \textit{Tính linh hoạt cao:}  Đảm bảo khả năng mở rộng và tích hợp dễ dàng các tính năng mới trong tương lai, phù hợp với sự phát triển không ngừng của nhà hàng và sự thay đổi của nhu cầu thị trường.
% \end{itemize}

% \textbf{Về mặt kỹ thuật:}

% \begin{itemize}
%     \item \textit{Hiệu suất và bảo mật tối ưu:}  Thiết kế một hệ thống có hiệu suất cao, khả năng xử lý đồng thời nhiều tác vụ và đảm bảo bảo mật tuyệt đối cho dữ liệu người dùng.
%     \item \textit{Kiến trúc linh hoạt và dễ bảo trì:}  Xây dựng một kiến trúc hệ thống linh hoạt, dễ dàng bảo trì, nâng cấp và mở rộng để đáp ứng nhu cầu phát triển trong tương lai.
% \end{itemize}


% \subsection{Mục tiêu đề tài}

% Mục tiêu của đề tài là xây dựng một hệ thống quản lý đặt món hiện đại, tích hợp các giải pháp công nghệ để tối ưu hóa quy trình vận hành nhà hàng và nâng cao trải nghiệm khách hàng, đồng thời khắc phục các hạn chế của phương pháp quản lý truyền thống đã nêu. Hệ thống này tập trung vào việc tự động hóa các quy trình từ đặt món, quản lý đơn hàng đến thanh toán, đồng thời cung cấp các công cụ phân tích và quản lý hiệu quả cho nhà hàng. Dự án hướng đến việc mang lại lợi ích thiết thực cho khách hàng, nhân viên và quản lý, góp phần thúc đẩy sự phát triển bền vững của ngành công nghiệp nhà hàng trong kỷ nguyên số.

% \begin{enumerate}
%     \item \textbf{Đối với Khách hàng:}
%         \begin{itemize}
%             \item \textit{Tối ưu trải nghiệm người dùng:} Cung cấp giao diện thân thiện, trực quan, cho phép khách hàng dễ dàng truy cập thực đơn, đặt món nhanh chóng và thanh toán linh hoạt qua các phương thức điện tử, như ví điện tử hoặc mã QR, nhằm mang lại trải nghiệm liền mạch.
%             \item \textit{Minh bạch và tin cậy:} Đảm bảo thông tin về giá cả, chương trình ưu đãi và lịch sử đặt món được trình bày rõ ràng, minh bạch, xây dựng niềm tin và sự hài lòng của khách hàng.
%             \item \textit{Cá nhân hóa trải nghiệm:} Sử dụng dữ liệu lịch sử đặt món để cung cấp các ưu đãi và khuyến mãi phù hợp, khuyến khích khách hàng quay lại và tham gia các chương trình khách hàng thân thiết.
%         \end{itemize}

%     \item \textbf{Đối với Nhân viên Nhà hàng:}
%         \begin{itemize}
%             \item \textit{Nâng cao hiệu suất và độ chính xác:} Hỗ trợ nhân viên xử lý đơn hàng nhanh chóng, giảm thiểu sai sót trong quá trình phục vụ, đặc biệt vào giờ cao điểm, thông qua các tính năng tự động hóa.
%             \item \textit{Giảm tải công việc thủ công:} Tích hợp công cụ quản lý thông minh để giảm bớt các tác vụ ghi chép thủ công, giúp nhân viên tập trung vào việc nâng cao chất lượng dịch vụ.
%         \end{itemize}

%     \item \textbf{Đối với Nhà Quản lý:}
%         \begin{itemize}
%             \item \textit{Cung cấp công cụ phân tích và báo cáo:} Hỗ trợ quản lý theo dõi doanh thu, phân tích xu hướng bán hàng và đánh giá hiệu quả hoạt động thông qua các báo cáo chi tiết, từ đó đưa ra quyết định kinh doanh chiến lược.
%             \item \textit{Quản lý hiệu quả nguồn lực:} Tích hợp chức năng quản lý kho, tối ưu hóa lịch làm việc của nhân viên và đảm bảo nguồn cung ổn định, giúp giảm chi phí và nâng cao hiệu quả vận hành.
%         \end{itemize}
% \end{enumerate}

\subsection{Mục tiêu đề tài}

Đề tài hướng đến xây dựng một \textbf{Hệ thống Quản lý Đặt món Hiện đại, Tích hợp và Hiệu quả}, đáp ứng nhu cầu vận hành của nhà hàng trong bối cảnh chuyển đổi số mạnh mẽ của ngành dịch vụ ăn uống. Hệ thống được thiết kế để vượt qua các hạn chế của phương pháp quản lý thủ công, như sai sót trong xử lý đơn hàng, giao tiếp không hiệu quả giữa các bộ phận và trải nghiệm khách hàng chưa tối ưu. Thông qua việc cung cấp giao diện thân thiện, hệ thống cho phép khách hàng dễ dàng xem thực đơn, đặt món, tùy chỉnh yêu cầu, theo dõi trạng thái đơn hàng và thanh toán nhanh chóng bằng nhiều phương thức, từ đó nâng cao sự tiện lợi và hài lòng. Đồng thời, hệ thống tối ưu hóa quy trình làm việc của nhân viên phục vụ, bếp/bar và thu ngân bằng các công cụ quản lý thông minh, giúp giảm thiểu sai sót, tăng tốc độ phục vụ và cung cấp báo cáo doanh thu chi tiết để hỗ trợ quản lý. Qua đó, hệ thống không chỉ hiện đại hóa quy trình vận hành mà còn góp phần giảm chi phí dài hạn, nâng cao hiệu quả quản lý và tạo lợi thế cạnh tranh bền vững cho nhà hàng.

% Trên cơ sở phân tích bối cảnh, động lực và những thách thức đã nêu, đề tài này đặt ra mục tiêu xây dựng một \textbf{Hệ thống Quản lý Đặt món Hiện đại, Tích hợp và Hiệu quả} cho nhà hàng. Hệ thống không chỉ nhằm giải quyết các hạn chế của phương pháp quản lý truyền thống mà còn được thiết kế để \textbf{trực tiếp đối phó và vượt qua các thách thức} đã được xác định, đáp ứng các kỳ vọng ngày càng cao của khách hàng và yêu cầu vận hành trong kỷ nguyên số.

% Các mục tiêu cụ thể của hệ thống, tập trung vào các tính năng chính và cách chúng giải quyết thách thức, bao gồm:

% \begin{enumerate}
%     \item \textbf{Đối với Khách hàng:} Nâng cao trải nghiệm đặt món và thanh toán, mang lại sự tiện lợi và chủ động.
%         \begin{itemize}
%             \item \textit{Xem thực đơn điện tử:} Truy cập thực đơn trực quan (hình ảnh, mô tả, giá cả) dễ dàng qua thiết bị di động hoặc kiosk tại bàn.
%             \item \textit{Đặt món và tùy chỉnh:} Chọn món, điều chỉnh số lượng, và dễ dàng thêm các yêu cầu đặc biệt (ví dụ: ít cay, không hành, thêm phô mai).
%             \item \textit{Theo dõi trạng thái đơn hàng:} Có thể xem được trạng thái đơn hàng của mình (đang chờ xác nhận, đang chuẩn bị, đã sẵn sàng - tùy mức độ triển khai).
%             \item \textit{Yêu cầu thanh toán và thanh toán tiện lợi:} Gọi thanh toán và thực hiện thanh toán nhanh chóng qua các phương thức đa dạng như tiền mặt, thẻ, và đặc biệt là quét mã QR ngay tại bàn (\textit{giải quyết phần nào thách thức về tích hợp thanh toán điện tử}).
%             \item \textit{Gọi phục vụ:} Có chức năng gọi nhân viên hỗ trợ khi cần thiết thông qua hệ thống.
%         \end{itemize}

%     \item \textbf{Đối với Nhân viên Phục vụ:} Tối ưu hóa quy trình nhận và chuyển đơn hàng, giảm sai sót và tăng tốc độ phục vụ.
%         \begin{itemize}
%             \item \textit{Ghi nhận đơn hàng di động:} Sử dụng máy tính bảng hoặc thiết bị POS di động để nhận đơn ngay tại bàn, giảm thiểu việc ghi chép thủ công.
%             \item \textit{Gửi đơn hàng tự động:} Đơn hàng được gửi tức thì và chính xác đến bộ phận bếp và/hoặc bar thông qua màn hình hiển thị.
%             \item \textit{Quản lý trạng thái bàn:} Dễ dàng theo dõi trạng thái các bàn (trống, đang có khách, đã đặt trước, cần dọn dẹp).
%             \item \textit{Xử lý yêu cầu đặc biệt:} Ghi chú và chuyển các yêu cầu đặc biệt của khách hàng một cách rõ ràng.
%             \item \textit{Hỗ trợ thanh toán tại bàn:} Thực hiện quy trình thanh toán (in hóa đơn tạm, xác nhận thanh toán) hiệu quả hơn.
%         \end{itemize}

%     \item \textbf{Đối với Nhân viên Bếp/Bar:} Nhận thông tin đơn hàng chính xác và kịp thời, tối ưu hóa quy trình chế biến.
%         \begin{itemize}
%             \item \textit{Hiển thị đơn hàng điện tử (Kitchen Display System - KDS):} Nhận đơn hàng rõ ràng trên màn hình, bao gồm chi tiết món và các tùy chỉnh, thay vì phiếu giấy dễ thất lạc hoặc nhòe mực.
%             \item \textit{Quản lý thứ tự chế biến:} Hệ thống có thể hỗ trợ sắp xếp thứ tự ưu tiên của các đơn hàng hoặc món ăn.
%             \item \textit{Đánh dấu trạng thái món ăn:} Thông báo cho bộ phận phục vụ khi món ăn đã sẵn sàng để phục vụ.
%         \end{itemize}

%     \item \textbf{Đối với Nhân viên Thu ngân và Quản lý:} Cung cấp công cụ quản lý hiệu quả, theo dõi doanh thu và hiệu suất hoạt động.
%         \begin{itemize}
%             \item \textit{Quản lý hóa đơn và thanh toán tập trung:} Dễ dàng xem lại, in ấn, và xử lý thanh toán cho các hóa đơn.
%             \item \textit{Áp dụng khuyến mãi, giảm giá:} Hệ thống hỗ trợ việc thiết lập và áp dụng các chương trình khuyến mãi một cách linh hoạt.
%             \item \textit{Báo cáo doanh thu chi tiết:} Cung cấp các báo cáo doanh thu theo thời gian (ngày, tuần, tháng), theo món ăn, theo nhân viên, giúp quản lý nắm bắt tình hình kinh doanh.
%             \item \textit{Quản lý thực đơn và giá bán:} Cho phép quản lý dễ dàng cập nhật thực đơn, giá cả và mô tả món ăn.
%             \item \textit{Quản lý tài khoản người dùng:} Phân quyền truy cập và sử dụng hệ thống cho từng vai trò nhân viên.
%         \end{itemize}

%     \item \textbf{Mục tiêu Hệ thống Tổng thể - Trực tiếp giải quyết các thách thức cốt lõi:}
%         \begin{itemize}
%             \item \textit{Tính ổn định và hiệu năng:} Đảm bảo hệ thống hoạt động mượt mà, ổn định, đặc biệt trong các giờ cao điểm, giảm thiểu tối đa thời gian chết (\textit{giải quyết thách thức về quản lý lỗi và vận hành liên tục}).
%             \item \textit{Tính bảo mật cao:} Xây dựng các cơ chế bảo mật mạnh mẽ để bảo vệ dữ liệu giao dịch, thông tin khách hàng và dữ liệu kinh doanh, tuân thủ các quy định pháp luật (\textit{giải quyết thách thức về bảo mật dữ liệu và quyền riêng tư}).
%             \item \textit{Kiến trúc linh hoạt và khả năng tích hợp:} Thiết kế hệ thống theo kiến trúc module, sử dụng các API rõ ràng để dễ dàng tích hợp với các thiết bị phần cứng phổ biến (máy in, máy POS) và có thể mở rộng để tích hợp với các hệ thống khác (kế toán, kho) trong tương lai (\textit{giải quyết thách thức về độ phức tạp kỹ thuật và tích hợp hệ thống}).
%             \item \textit{Giao diện thân thiện và trực quan (UX/UI):} Ưu tiên thiết kế giao diện người dùng đơn giản, dễ học, dễ sử dụng cho tất cả các đối tượng, từ khách hàng đến nhân viên và quản lý, nhằm giảm thời gian đào tạo và tăng tỷ lệ chấp nhận (\textit{giải quyết thách thức về thiết kế trải nghiệm người dùng và đào tạo/thay đổi thói quen}).
%             \item \textit{Quản lý dữ liệu hiệu quả:} Thiết kế cơ sở dữ liệu tối ưu để xử lý khối lượng lớn giao dịch và thông tin, đảm bảo khả năng truy vấn và báo cáo nhanh chóng (\textit{giải quyết thách thức về quản lý dữ liệu lớn}).
%             % Lưu ý: Thách thức về chi phí, thời gian và nhân lực được giải quyết thông qua việc lập kế hoạch dự án, lựa chọn công nghệ phù hợp và quản lý hiệu quả, hơn là một tính năng trực tiếp của hệ thống. Tuy nhiên, một hệ thống hiệu quả, dễ sử dụng và bảo trì sẽ góp phần giảm chi phí vận hành và đào tạo dài hạn.
%         \end{itemize}
% \end{enumerate}

% Việc đạt được các mục tiêu này không chỉ mang lại các tính năng mong đợi mà còn trực tiếp tháo gỡ những rào cản và khó khăn đã được nhận diện. Qua đó, hệ thống sẽ góp phần hiện đại hóa quy trình vận hành, nâng cao sự hài lòng của khách hàng, tăng cường hiệu quả quản lý và tạo lợi thế cạnh tranh bền vững cho nhà hàng trong bối cảnh ngành F\&B đang chuyển đổi số mạnh mẽ.

\subsection{Phạm vi đề tài}

Để đảm bảo dự án khả thi và tập trung vào việc giải quyết các mục tiêu cốt lõi trong khung thời gian và nguồn lực hạn chế, phạm vi của đề tài được xác định rõ ràng như sau:

\subsubsection{Phạm vi Chức năng}

Hệ thống sẽ tập trung vào các chức năng thiết yếu cho quy trình đặt món và quản lý cơ bản tại một nhà hàng điển hình. Các chức năng chính bao gồm:
        \begin{itemize}
            \item Quản lý Thực đơn: Thêm, sửa, xóa, ẩn/hiện món ăn, danh mục món ăn, quản lý giá bán và mô tả cơ bản.
            \item Quản lý Bàn/Khu vực: Thiết lập sơ đồ bàn, quản lý trạng thái bàn (trống, đang phục vụ, đã đặt, cần dọn).
            \item Quy trình Đặt món tại bàn: Nhân viên phục vụ nhận đơn qua thiết bị di động (máy tính bảng/điện thoại), gửi đơn tự động đến bếp/bar. Bao gồm khả năng chọn món, tùy chỉnh số lượng, ghi chú đơn giản.
            \item Hiển thị Đơn hàng tại Bếp (Kitchen Display System - KDS): Hiển thị danh sách các món cần chế biến, trạng thái món, ... lên màn hình.
            \item Quản lý Đơn hàng cơ bản: Xem danh sách đơn hàng, trạng thái đơn hàng.
            \item Quy trình Thanh toán: Hỗ trợ tạo hóa đơn tạm, áp dụng giảm giá đơn giản (theo phần trăm hoặc số tiền cố định), ghi nhận thanh toán bằng tiền mặt và tích hợp thanh toán cơ bản qua ví điện tử.
            \item Báo cáo cơ bản: Báo cáo doanh thu theo ngày, báo cáo các món bán chạy trong ngày.
            \item Quản lý Tài khoản người dùng: Tạo và quản lý tài khoản cho các vai trò (Quản lý, Phục vụ, Bếp, Thu ngân, Quản lý) với phân quyền truy cập chức năng tương ứng.
        \end{itemize}

\subsubsection{Phạm vi Kỹ thuật}

Các giới hạn và lựa chọn về mặt công nghệ được xác định như sau:

\begin{itemize}
    \item \textbf{Kiến trúc hệ thống:} Dự kiến xây dựng theo kiến trúc ứng dụng kiến trúc Client-Server cơ bản để đơn giản hóa việc phát triển và triển khai trong giai đoạn đầu.
    \item \textbf{Công nghệ phát triển (Dự kiến):}
        \begin{itemize}
            \item \textit{Frontend (Giao diện người dùng):} Sử dụng một framework JavaScript hiện đại như React cùng với ShadcnUI và TailwindCSS để xây dựng giao diện tương tác và responsive.
            \item \textit{Backend (Xử lý logic):} Sử dụng một nền tảng phía máy chủ phổ biến Java (với Spring Boot) dựa trên kinh nghiệm của thành viên trong nhóm và các đặc điểm của nền tảng.
            \item \textit{Cơ sở dữ liệu:} Sử dụng hệ quản trị cơ sở dữ liệu quan hệ (SQL) PostgreSQL để đảm bảo tính nhất quán dữ liệu.
            \item \textit{Giao tiếp Real-time (nếu cần cho KDS/cập nhật trạng thái):} Có thể sử dụng WebSockets hoặc các kỹ thuật tương tự.
        \end{itemize}
    \item \textbf{Nền tảng triển khai:} Hệ thống chủ yếu hoạt động trên trình duyệt Web trên các thiết bị như máy tính, máy POS, máy tính bảng. Chưa bao gồm việc phát triển ứng dụng di động gốc (Native Mobile App) cho iOS hay Android trong phạm vi này.
    \item \textbf{Bảo mật:} Áp dụng các biện pháp bảo mật cơ bản như mã hóa mật khẩu, bảo vệ chống lại các lỗ hổng web phổ biến, phân quyền dựa trên vai trò. Không bao gồm các biện pháp kiểm thử xâm nhập (penetration testing) chuyên sâu.
\end{itemize}
\subsection{Cấu trúc báo cáo đồ án}
Cấu trúc bài báo cáo bao gồm 7 chương:
\begin{itemize}
    \item \textbf{Chương 1 - Giới thiệu}: Trình bày bối cảnh, động lực, thách thức, mục tiêu và phạm vi của đề tài.
    \item \textbf{Chương 2 - Cơ sở lý thuyết}: Cung cấp nền tảng lý thuyết về lĩnh vực nhà hàng và công nghệ.
    \item \textbf{Chương 3 - Các hệ thống liên quan}: So sánh các hệ thống quản lý hiện có và rút ra bài học.
    \item \textbf{Chương 4 - Tổng quan hệ thống}: Mô tả bối cảnh, yêu cầu và quy trình tổng quát của hệ thống.
    \item \textbf{Chương 5 - Thiết kế}: Thiết kế Sitemap, giao diện, cơ sở dữ liệu, API và kiến trúc triển khai.
    \item \textbf{Chương 6 - Tổng kết}: Tóm tắt những gì đã thực hiện trong giai đoạn thiết kế, đồng thời định hướng phát triển và triển khai trong giai đoạn tiếp theo.
    \item \textbf{Chương 7 - Kế hoạch phát triển}: Liệt kê chi tiết các đầu việc, tính năng và kiểm thử dự kiến sẽ thực hiện trong giai đoạn phát triển hệ thống.
\end{itemize}


% \newpage
% \section{CƠ SỞ LÝ THUYẾT}
\subsection{Tìm hiểu về Restaurant Industry}

\subsubsection{Nhà hàng ăn uống cao cấp (Fine dining restaurant)}
\subsubsubsection{Tổng quan}
Fine dining được định nghĩa là trải nghiệm nhà hàng tinh tế, thường đắt đỏ và đặc biệt hơn so với nhà hàng thông thường. Đặc điểm bao gồm sử dụng khăn trải bàn trắng, dịch vụ bàn bởi nhân viên được đào tạo, và không gian sang trọng với vật liệu cao cấp. Lịch sử cho thấy fine dining bắt đầu vào những năm 1780 tại Paris, với các nhà hàng như Trois Frères và La Grande Taverne de Londres, và sau đó lan sang Hoa Kỳ với Delmonico's ở New York, nổi tiếng với hầm rượu 1.000 chai. Trong thời kìh hiện đại, fine dining tập trung vào sự sáng tạo và dịch vụ hoàn hảo.\\


\begin{figure}[H]
    \centering
    \includegraphics[width=15cm]{Images/fine-dining.jpg}
    \vspace{0.5cm}
    \caption{Nhà hàng fine dining. Nguồn: \href{https://chefin.com/blog/the-past-and-future-of-fine-dining/}{chefin.com}}
    \label{fig:my_label}
\end{figure}

Nhà hàng fine dining hoạt động dựa trên mô hình kinh doanh cân bằng giữa chi phí vận hành cao và giá bán cao cấp. Chi phí bao gồm nguyên liệu cao cấp, lao động có kỹ năng, và sự chú ý chi tiết trong cả chuẩn bị món ăn và dịch vụ. Doanh thu chủ yếu đến từ việc bán bữa ăn, với giá cao hơn nhiều so với nhà hàng ăn uống thông thường, phản ánh chất lượng và sự độc quyền của trải nghiệm. Ngoài ra, các chương trình rượu vang và đồ uống cũng đóng góp đáng kể vào doanh thu, cùng với các dịch vụ như tiệc riêng, catering, và bán hàng hóa. \\

Tuy nhiên, lợi nhuận trong fine dining là một thách thức do chi phí cố định và biến đổi cao. Chi phí cố định bao gồm tiền thuê ở vị trí đắc địa, trang trí cao cấp, và thiết bị, trong khi chi phí biến đổi bao gồm nguyên liệu đắt tiền và lao động cho đội ngũ được đào tạo bài bản. Thành công phụ thuộc vào việc duy trì tỷ lệ khách cao, khách hàng trung thành, và danh tiếng mạnh, có thể dẫn đến các giải thưởng như sao Michelin, từ đó thúc đẩy nhu cầu. \\

Fine dining hiện đại đã phát triển, nhấn mạnh vào sự sáng tạo và dịch vụ hoàn hảo. Sự kết hợp giữa chỗ ở sang trọng và ẩm thực cao cấp, như được tiên phong bởi César Ritz và Auguste Escoffier tại Grand Hotel of Monte Carlo, đã nâng tầm vị thế của fine dining.


\subsubsubsection{Sơ đồ tổ chức (Organization chart)}
Trong ngành công nghiệp nhà hàng, đặc biệt là nhà hàng ẩm thực cao cấp, sơ đồ tổ chức đóng vai trò quan trọng trong việc định hình cấu trúc, đảm bảo bộ máy hoạt động hiệu quả. Biểu đồ tổ chức giúp làm rõ mối quan hệ giữa các vị trí công việc, xác định ai báo cáo cho ai, và ai chịu trách nhiệm cho những nhiệm vụ cụ thể nào. Điều này đặc biệt quan trọng trong nhà hàng ẩm thực cao cấp, nơi mà chất lượng dịch vụ và sự hài lòng của khách hàng là yếu tố hàng đầu. \\
Nhà hàng ẩm thực cao cấp thường có cấu trúc tổ chức phức tạp và đa tầng, phản ánh nhu cầu cung cấp dịch vụ cá nhân hóa và sang trọng:

\begin{itemize}
    \item Quản trị viên: Vai trò này đảm bảo rằng nhà hàng tuân thủ các quy định pháp lý và hoạt động hành chính suôn sẻ. Họ xử lý thuế, làm việc với nhà cung cấp, và quản lý các vấn đề tài chính, tạo nền tảng cho hoạt động của nhà hàng.
    \item Quản lý: Là người giám sát tổng thể, đảm bảo rằng mọi bộ phận hoạt động hài hòa. Họ làm cầu nối giữa quản trị viên và các trưởng bộ phận, đảm bảo dịch vụ đạt tiêu chuẩn cao cấp.
    \item Quản lý nhà bếp: Vai trò này tập trung vào việc duy trì nguồn cung nguyên liệu chất lượng, làm việc chặt chẽ với đầu bếp trưởng để đảm bảo nhà bếp hoạt động hiệu quả.
    \item Đầu bếp trưởng: Là linh hồn của nhà hàng, chịu trách nhiệm cho tầm nhìn ẩm thực. Họ tạo ra thực đơn sáng tạo, phản ánh phong cách của nhà hàng, và giám sát toàn bộ hoạt động nhà bếp để đảm bảo chất lượng và sự nhất quán.
    \item Phó đầu bếp: Hỗ trợ đầu bếp trưởng, đảm bảo rằng mọi món ăn được chuẩn bị và trình bày đúng tiêu chuẩn, đặc biệt trong giờ cao điểm.
    \item Đầu bếp chuyên khoa: Mỗi station chef chịu trách nhiệm cho một lĩnh vực cụ thể, như món khai vị hoặc món chính, đảm bảo chuyên môn hóa và chất lượng cao.
    \item Đầu bếp, trợ lý: Là lực lượng hỗ trợ chính, thực hiện các nhiệm vụ hàng ngày như chuẩn bị nguyên liệu và nấu nướng, đảm bảo nhà bếp hoạt động trơn tru.
    \item Người rửa chén, nhân viên dọn dẹp hậu trường: Vai trò này rất quan trọng để duy trì vệ sinh, đảm bảo an toàn thực phẩm, và giữ cho nhà bếp luôn sẵn sàng.
    \item Trưởng phục vụ (Maitre d'): Là gương mặt đại diện của nhà hàng, đảm bảo khách hàng có trải nghiệm đầu tiên ấn tượng. Họ quản lý danh sách đặt chỗ và hướng dẫn khách, tạo nên bầu không khí chuyên nghiệp.
    \item Nhân viên phục vụ: Đảm bảo khách hàng được phục vụ chu đáo, từ việc lấy order đến phục vụ món ăn, góp phần vào trải nghiệm tổng thể.
    \item Chuyên viên rượu: Là chuyên gia về rượu, họ tư vấn cho khách hàng về các lựa chọn rượu phù hợp, nâng cao trải nghiệm ẩm thực, đặc biệt trong môi trường cao cấp.
    \item Nhân viên dọn dẹp tiền sảnh: Giữ khu vực ăn uống sạch sẽ và ngăn nắp, đảm bảo không gian sang trọng và thoải mái cho khách hàng.
    \item Trưởng bảo vệ: Quản lý an ninh, đảm bảo an toàn cho khách và tài sản, một vai trò quan trọng trong môi trường cao cấp.
    \item Bảo vệ, người đỗ xe: Cung cấp dịch vụ an ninh và đỗ xe, tạo ấn tượng chuyên nghiệp và sang trọng, đặc biệt cho khách hàng sử dụng dịch vụ đỗ xe.
\end{itemize}

\begin{table}[ht]
\centering
\resizebox{\textwidth}{!}{
    \begin{tabular}{| p{4cm} | p{11cm} | p{3cm} |}
    \hline
    \textbf{Vị trí} & \textbf{Chức năng} & \textbf{Báo cáo cho} \\
    \hline
    Quản trị viên (Administrator) & Xử lý thuế, nhà cung cấp, và các nhu cầu hành chính khác. Đảm bảo tuân thủ pháp lý và hoạt động hành chính suôn sẻ. & Không có (cấp cao nhất) \\
    \hline
    Quản lý (Manager) & Giám sát toàn bộ hoạt động của nhà hàng, bao gồm cả phần front-of-house và back-of-house. Đảm bảo dịch vụ chuyên nghiệp và khách hàng hài lòng. & Quản trị viên \\
    \hline
    Quản lý nhà bếp (Kitchen Manager) & Quản lý hàng tồn kho và mua sắm cho nhà bếp, đảm bảo nguyên liệu chất lượng cao. Làm việc chặt chẽ với đầu bếp trưởng. & Quản trị viên \\
    \hline
    Đầu bếp trưởng (Executive Chef) & Là người lãnh đạo culinaire, tạo thực đơn sáng tạo, giám sát hoạt động nhà bếp, đảm bảo chất lượng và sự nhất quán. & Quản lý nhà bếp \\
    \hline
    Phó đầu bếp (Sous-Chef) & Hỗ trợ đầu bếp trưởng, giám sát việc chuẩn bị và trình bày món ăn, đảm bảo tiêu chuẩn. & Đầu bếp trưởng \\
    \hline
    Đầu bếp chuyên khoa (Station Chef) & Chuyên trách một lĩnh vực cụ thể (khai vị, món chính, tráng miệng), xử lý chuẩn bị phức tạp, giám sát đầu bếp và trợ lý. & Phó đầu bếp \\
    \hline
    Đầu bếp, trợ lý (Cooks, Assistants) & Thực hiện nhiệm vụ được giao, bao gồm chuẩn bị nguyên liệu, nấu nướng, và trình bày món ăn. & Đầu bếp chuyên khoa \\
    \hline
    Người rửa chén, nhân viên dọn dẹp hậu trường (Dishwasher, Back of House Cleaners) & Giữ nhà bếp và thiết bị sạch sẽ, đảm bảo vệ sinh và an toàn thực phẩm. & Quản lý nhà bếp \\
    \hline
    Trưởng phục vụ (Maitre d') & Là gương mặt đầu tiên, hướng dẫn khách, quản lý danh sách đặt chỗ, đảm bảo dịch vụ xuất sắc. & Quản lý \\
    \hline
    Nhân viên phục vụ (Waiters) & Phục vụ khách, lấy order, phục vụ thức ăn và đồ uống, đảm bảo khách hài lòng. & Trưởng phục vụ \\
    \hline
    Chuyên viên rượu (Sommelier) & Là chuyên gia về rượu, tư vấn lựa chọn rượu phù hợp, nâng cao trải nghiệm ẩm thực. & Trưởng phục vụ \\
    \hline
    Nhân viên dọn dẹp tiền sảnh (Front-of-House Cleaning Staff) & Giữ khu vực ăn uống sạch sẽ và ngăn nắp, đảm bảo không gian thoải mái và sang trọng. & Trưởng phục vụ \\
    \hline
    Trưởng bảo vệ (Head of Security) & Quản lý an ninh, giám sát camera, bảo vệ, và người đỗ xe, đảm bảo an toàn cho khách và tài sản. & Quản lý \\
    \hline
    Bảo vệ, người đỗ xe (Guards, Valets) & Cung cấp dịch vụ an ninh và đỗ xe, tạo ấn tượng chuyên nghiệp và sang trọng. & Trưởng bảo vệ \\
    \hline
    \end{tabular}
}
\caption{Vị trí, chức năng và người quản lý trong một nhà hàng ăn uống cao cấp}
\end{table}

\clearpage

\begin{figure}[H]
    \centering
    \includegraphics[width=15cm]{Images/finedining_orig_chart.jpg}
    \vspace{0.5cm}
    \caption{Sơ đồ tổ chức cho nhà hàng ăn uống cao cấp. Nguồn: waiterio.com}
\end{figure}

\subsubsection{Nhà hàng ăn uống phục vụ đồ ăn nhanh (Fast-food restaurant)}
\subsubsubsection{Tổng quan}
Nhà hàng thức ăn nhanh, còn được gọi là nhà hàng dịch vụ nhanh (QSR), là cơ sở cung cấp thức ăn được chuẩn bị và phục vụ nhanh chóng, thường trong vòng vài phút, với thực đơn hạn chế như bánh burger, khoai tây chiên, pizza và gà rán, phục vụ trong đồ dùng một lần. Khách hàng thường đặt hàng tại quầy hoặc qua cửa sổ giao hàng, và thức ăn được dự định để ăn tại chỗ hoặc mang đi. \\


\begin{figure}[H]
    \centering
    \includegraphics[width=15cm]{Images/fastfood.jpg}
    \vspace{0.5cm}
    \caption{Nhà hàng phụ vụ thức ăn nhanh. Nguồn: \href{https://foodtank.com/news/2015/08/world-health-organization-study-proves-need-for-regulation-of-fast-food/}{foodtank.com}}
\end{figure}

Mô hình kinh doanh của nhà hàng thức ăn nhanh tập trung vào khối lượng lớn và biên lợi nhuận thấp, với thực đơn chuẩn hóa và hệ thống franchise để đảm bảo sự đồng nhất và khả năng mở rộng. Họ nhắm đến thị trường bận rộn, sử dụng tiếp thị mạnh mẽ và nền tảng kỹ thuật số để tiếp cận khách hàng.

Nhà hàng thức ăn nhanh vận hành với các quy trình tinh giản sử dụng bếp dây chuyền để đảm bảo hiệu quả, với chuỗi cung ứng tập trung và công nghệ như hệ thống POS để tối ưu hóa hoạt động. Đào tạo nhân viên tập trung vào tốc độ và chính xác, với mô hình tự phục vụ giảm chi phí lao động.

\subsubsubsection{Sơ đồ tổ chức (Organization chart)}
Nhà hàng thức ăn nhanh thường có cấu trúc tổ chức đơn giản và phẳng, phản ánh nhu cầu cung cấp dịch vụ nhanh chóng và hiệu quả:
\begin{itemize}
    \item Quản lý điều hành: Giám sát toàn bộ hoạt động, bao gồm tuyển dụng và ngân sách.
    \item Quản lý ca: Giám sát nhân viên trong ca, xử lý khiếu nại khách hàng.
    \item Nhân viên thu ngân: Xử lý thanh toán và đơn hàng.
    \item Đầu bếp: Chuẩn bị thức ăn, đảm bảo chất lượng.
    \item Nhân viên vệ sinh: Giữ nhà hàng sạch sẽ, đặc biệt quan trọng trong môi trường cao áp lực.
\end{itemize}
\begin{table}[ht]
\centering
\resizebox{\textwidth}{!}{
    \begin{tabular}{| p{4cm} | p{11cm} | p{3cm} |}
    \hline
    \textbf{Vị trí} & \textbf{Chức năng} & \textbf{Báo cáo cho} \\
    \hline
    Quản lý điều hành (Executive Manager) & Giám sát toàn bộ hoạt động của nhà hàng, bao gồm tuyển dụng, sa thải, ngân sách, lương, lịch trình, hàng tồn kho, và mua sắm nguyên liệu. & Không có (cấp cao nhất) \\
    \hline
    Quản lý ca (Shift Manager) & Giám sát nhân viên trong ca làm việc, xử lý khiếu nại khách hàng, lập lịch, kiểm tra tiền thu ngân. & Quản lý điều hành \\
    \hline
    Nhân viên thu ngân (Cashier) & Xử lý đơn hàng và thanh toán của khách hàng. & Quản lý ca \\
    \hline
    Nhân viên ghi order (Order Taker) & Ghi đơn hàng, đặc biệt ở đường lái xe. & Quản lý ca \\
    \hline
    Nhân viên dịch vụ khách hàng (Customer Service Representative) & Hỗ trợ giải quyết các câu hỏi hoặc vấn đề của khách hàng. & Quản lý ca \\
    \hline
    Đầu bếp và nhân viên bếp (Cooks and Kitchen Staff) & Chuẩn bị thức ăn, bổ sung nguyên liệu, bảo trì thiết bị. & Quản lý ca \\
    \hline
    Nhân viên rửa chén (Dishwasher) & Duy trì vệ sinh nhà bếp. & Quản lý ca \\
    \hline
    Nhân viên bảo trì (Maintenance Staff) & Xử lý sửa chữa thiết bị. & Quản lý ca \\
    \hline
    Nhân viên vệ sinh (Cleaning Crew) & Giữ cho nhà hàng sạch sẽ. & Quản lý ca \\
    \hline
    \end{tabular}
}
\caption{Vị trí, chức năng và người quản lý trong một nhà hàng phục vụ đồ ăn nhanh}
\end{table}

\begin{figure}[H]
    \centering
    \includegraphics[width=15cm]{Images/fast-food-org-chart.jpg}
    \vspace{0.5cm}
    \caption{Sơ đồ tổ chức cho nhà hàng phục vụ đồ ăn nhanh. Nguồn: waiterio.com}
\end{figure}


\subsubsection{Sự khác nhau giữa nhà hàng ăn uống cao cấp và nhà hàng phục vụ đồ ăn nhanh}

% \begin{figure}[H]
%     \centering
%     \includegraphics[width=15cm]{Images/process_map_1.jpg}
%     \includegraphics[width=15cm]{Images/process_map_2.jpg}
%     \vspace{0.5cm}
%     \caption{Process map về Restaurant Industry}
%     \label{fig:my_label}
% \end{figure}



\subsection*{Bảng so sánh tổng quan}

% *** THAY ĐỔI: Bắt đầu môi trường longtable ***
% Bỏ \begin{table}[H], \centering, và \resizebox
% Sử dụng p{} hoặc m{} cho các cột. p{} căn lề trên, m{} căn giữa dọc.
% Có thể dùng \RaggedRight trong các cột p/m để tránh giãn chữ không đẹp.
\begin{longtable}{| m{3.5cm} | >{\RaggedRight}m{6.5cm} | >{\RaggedRight}m{6.5cm} |} % Điều chỉnh lại chiều rộng nếu cần

% *** THAY ĐỔI: Định nghĩa Header cho bảng ***
\hline
\textbf{Tiêu chí} & \textbf{Nhà hàng Fine Dining} & \textbf{Nhà hàng Fast Food (QSR)} \\
\hline
\endfirsthead % Header chỉ xuất hiện ở đầu bảng (trang đầu tiên)

\multicolumn{3}{c}%
{{\tablename\ \thetable{} Bảng so sánh giữa nhà hàng phục vụ đồ ăn cao cấp và nhà hàng phục vụ đồ ăn nhanh}} \\ % Chú thích bảng tiếp tục
\hline
\textbf{Tiêu chí} & \textbf{Nhà hàng Fine Dining} & \textbf{Nhà hàng Fast Food (QSR)} \\
\hline
\endhead % Header lặp lại ở các trang tiếp theo

\hline \multicolumn{3}{r}{{Tiếp tục ở trang dưới}} \\ % Chú thích còn tiếp ở cuối trang (trừ trang cuối)
\endfoot

\hline % Đường kẻ cuối cùng của bảng
\endlastfoot

% *** Nội dung bảng giữ nguyên ***
\textbf{Định nghĩa \& Khái niệm} & Trải nghiệm nhà hàng tinh tế, thường đắt đỏ, đặc biệt hơn nhà hàng thông thường. Tập trung vào chất lượng cao và dịch vụ hoàn hảo. & Cơ sở cung cấp thức ăn được chuẩn bị và phục vụ nhanh chóng, thực đơn hạn chế, thường dùng đồ một lần. Tập trung vào tốc độ và sự tiện lợi. \\
\hline
\textbf{Trải nghiệm Khách hàng} & \begin{itemize} \item Sang trọng, độc quyền \item Dịch vụ cá nhân hóa, chu đáo tại bàn \item Không gian tinh tế, yên tĩnh \item Chú trọng từng chi tiết nhỏ \end{itemize} & \begin{itemize} \item Nhanh chóng, tiện lợi \item Tự phục vụ hoặc đặt tại quầy/drive-thru \item Không gian chức năng, đôi khi ồn ào \item Ít chú trọng chi tiết trải nghiệm không gian \end{itemize} \\
\hline
\textbf{Thực đơn} & \begin{itemize} \item Sáng tạo, phức tạp \item Nguyên liệu cao cấp, tươi ngon \item Thường có thực đơn cố định (set menu) hoặc gọi món (à la carte) phong phú \item Chịu ảnh hưởng lớn từ Bếp trưởng \end{itemize} & \begin{itemize} \item Hạn chế, chuẩn hóa \item Nguyên liệu tập trung vào chi phí và tốc độ chế biến \item Thường là các món phổ biến (burger, gà rán, pizza, khoai tây chiên) \item Đồng nhất giữa các chi nhánh (nếu là chuỗi) \end{itemize} \\
\hline
\textbf{Giá cả} & Cao cấp, phản ánh chất lượng nguyên liệu, kỹ năng chế biến, dịch vụ và không gian. & Thấp, phù hợp với đại chúng, tập trung vào việc bán số lượng lớn. \\
\hline
\textbf{Dịch vụ} & \begin{itemize} \item Dịch vụ tại bàn bởi nhân viên được đào tạo bài bản (waiters) \item Có Maitre d' chào đón và xếp chỗ \item Thường có Chuyên viên rượu (Sommelier) \item Tỷ lệ nhân viên/khách hàng cao \end{itemize} & \begin{itemize} \item Tự phục vụ hoặc dịch vụ tại quầy/drive-thru \item Nhân viên thực hiện các tác vụ nhanh gọn (nhận order, thu tiền, giao đồ ăn) \item Ít tương tác cá nhân hóa \item Tỷ lệ nhân viên/khách hàng thấp \end{itemize} \\
\hline
\textbf{Không gian \& Thiết kế} & Sang trọng, đầu tư vào nội thất, ánh sáng, âm nhạc, bộ đồ ăn. Thường dùng khăn trải bàn trắng. & Chức năng, hiệu quả, dễ lau dọn. Thiết kế thường theo chuẩn thương hiệu (nếu là chuỗi). Ít chú trọng yếu tố sang trọng. \\
\hline
\textbf{Mô hình Kinh doanh} & \begin{itemize} \item Chi phí vận hành rất cao (nguyên liệu, nhân sự tay nghề cao, mặt bằng đắc địa) \item Biên lợi nhuận có thể cao trên từng món nhưng tổng thể lợi nhuận là thách thức \item Phụ thuộc vào danh tiếng, đánh giá (sao Michelin), khách hàng trung thành \end{itemize} & \begin{itemize} \item Chi phí vận hành thấp hơn trên từng đơn vị sản phẩm \item Biên lợi nhuận thấp trên từng món, dựa vào bán số lượng cực lớn \item Thường hoạt động theo mô hình chuỗi, nhượng quyền (franchise) \item Marketing và nhận diện thương hiệu mạnh \end{itemize} \\
\hline
\textbf{Hoạt động Vận hành} & \begin{itemize} \item Quy trình chuẩn bị món ăn phức tạp, đòi hỏi kỹ năng cao \item Chú trọng kiểm soát chất lượng từng đĩa ăn \item Quản lý tồn kho nguyên liệu cao cấp phức tạp \end{itemize} & \begin{itemize} \item Quy trình tinh giản, dây chuyền hóa \item Sử dụng công nghệ (POS, KDS) để tối ưu tốc độ \item Chuỗi cung ứng tập trung, hiệu quả \item Đào tạo nhân viên tập trung vào tốc độ, chính xác \end{itemize} \\
\hline
\textbf{Nhân sự \& Cấu trúc Tổ chức} & \begin{itemize} \item Đội ngũ lớn, nhiều vị trí chuyên môn hóa cao (Bếp trưởng, Bếp phó, Đầu bếp chuyên khoa, Maitre d', Sommelier, Quản lý) \item Cấu trúc tổ chức phức tạp, đa tầng \item Yêu cầu kỹ năng và kinh nghiệm cao \end{itemize} & \begin{itemize} \item Đội ngũ tinh gọn hơn, các vai trò ít chuyên môn hóa hơn (Quản lý ca, Thu ngân, Nhân viên bếp, Nhân viên vệ sinh) \item Cấu trúc tổ chức phẳng, đơn giản \item Đào tạo tập trung vào quy trình chuẩn, dễ thay thế \end{itemize} \\
\hline
\textbf{Mục tiêu chính} & Cung cấp trải nghiệm ẩm thực đỉnh cao, độc đáo, xây dựng danh tiếng và thương hiệu đẳng cấp. & Tối đa hóa tốc độ phục vụ, sự tiện lợi, khối lượng bán và lợi nhuận thông qua hiệu quả vận hành. \\
% *** THAY ĐỔI: Bỏ \hline ở cuối vì \endlastfoot đã có ***
\hline % Bỏ dòng này

% *** THAY ĐỔI: Kết thúc môi trường longtable ***
% *** THAY ĐỔI: Caption và label đặt BÊN TRONG longtable ***
\caption{Bảng so sánh giữa nhà hàng phục vụ đồ ăn cao cấp và nhà hàng phục vụ đồ ăn nhanh}
\label{tab:comparison_detail} \\
\end{longtable}


Sự tương phản giữa nhà hàng fine dining và fast food không chỉ dừng lại ở các yếu tố bề mặt như giá cả hay tốc độ. Nó bắt nguồn từ triết lý kinh doanh, đối tượng khách hàng mục tiêu, và mô hình vận hành hoàn toàn khác biệt, dẫn đến những hệ quả sâu sắc:

\begin{itemize}
    \item \textbf{Giá trị cốt lõi:} Fine dining bán một \textit{trải nghiệm toàn diện}, nơi ẩm thực chỉ là một phần (dù là phần quan trọng nhất). Các yếu tố như không gian, dịch vụ cá nhân hóa, sự độc quyền, và cảm giác được chăm sóc đặc biệt đóng góp lớn vào giá trị mà khách hàng nhận được và sẵn sàng chi trả cao. Ngược lại, fast food bán \textit{sự tiện lợi và tính nhất quán}. Khách hàng tìm đến vì tốc độ, giá cả phải chăng và biết chính xác họ sẽ nhận được gì ở bất kỳ chi nhánh nào của thương hiệu.

    \item \textbf{Rủi ro và Lợi nhuận:} Mô hình fine dining có rủi ro cao hơn do chi phí đầu tư và vận hành lớn, cùng sự phụ thuộc vào danh tiếng và lượng khách hàng giới hạn. Một đánh giá tiêu cực hoặc thay đổi xu hướng ẩm thực có thể ảnh hưởng nghiêm trọng. Tuy nhiên, khi thành công, biên lợi nhuận trên mỗi khách hàng có thể rất cao. Fast food có rủi ro thấp hơn trên từng giao dịch nhờ mô hình chuẩn hóa và khối lượng lớn. Lợi nhuận đến từ việc tối ưu hóa chi phí và quy mô, nhưng lại nhạy cảm với cạnh tranh về giá và chi phí nguyên liệu đầu vào.

    \item \textbf{Vai trò của Con người và Công nghệ:} Trong fine dining, yếu tố con người là không thể thay thế. Kỹ năng của đầu bếp, sự tinh tế của nhân viên phục vụ, và kiến thức của sommelier tạo nên sự khác biệt. Công nghệ chủ yếu hỗ trợ quản lý (đặt bàn, POS) chứ không thay thế vai trò trung tâm của con người trong việc tạo ra trải nghiệm. Ngược lại, fast food tận dụng tối đa công nghệ để \textit{tăng hiệu suất và giảm sự phụ thuộc vào kỹ năng cá nhân}. Từ hệ thống đặt hàng tự động, bếp dây chuyền, đến phân tích dữ liệu bán hàng, công nghệ là chìa khóa để duy trì tốc độ và sự đồng nhất.

    \item \textbf{Cấu trúc tổ chức phản ánh sự phức tạp:} Sơ đồ tổ chức đa tầng của fine dining cho thấy sự chuyên môn hóa cao độ cần thiết để duy trì chất lượng ở mọi khâu, từ bếp đến tiền sảnh. Mỗi vị trí có vai trò và trách nhiệm rõ ràng, đòi hỏi kỹ năng chuyên biệt. Cấu trúc phẳng của fast food phản ánh quy trình làm việc đơn giản, lặp lại và tập trung vào hiệu quả hoạt động theo ca dưới sự giám sát của quản lý ca.

    \item \textbf{Thích ứng và Xu hướng:} Cả hai mô hình đều đang phải thích ứng. Fine dining ngày càng chú trọng hơn đến tính bền vững, nguồn gốc nguyên liệu và trải nghiệm độc đáo (ví dụ: bếp mở, tương tác với đầu bếp). Một số nhà hàng fine dining cũng tìm cách tiếp cận phân khúc rộng hơn thông qua các mô hình "casual dining" hoặc "bistro" cao cấp. Fast food đang đối mặt với áp lực về thực phẩm lành mạnh hơn, nguồn gốc rõ ràng và trải nghiệm khách hàng tốt hơn (không gian sạch đẹp, dịch vụ thân thiện hơn). Sự trỗi dậy của "fast-casual" (phân khúc giữa fast food và casual dining) là minh chứng cho sự thay đổi này, kết hợp tốc độ của fast food với chất lượng nguyên liệu và không gian tốt hơn.

\end{itemize}

\subsection{Hành trình khách hàng (Customer Journey) trong ngành dịch vụ nhà hàng}

Hành trình khách hàng trong ngành dịch vụ nhà hàng là một chuỗi các tương tác và trải nghiệm, từ giai đoạn nhận thức ban đầu đến hành động quay lại sử dụng dịch vụ. Hiểu rõ và tối ưu hóa từng giai đoạn trong hành trình này là yếu tố then chốt để nâng cao sự hài lòng của khách hàng, xây dựng lòng trung thành và tạo lợi thế cạnh tranh bền vững cho nhà hàng. Theo Orderable \cite{Orderable}, hành trình khách hàng trong ngành dịch vụ nhà hàng có thể chia thành các cột mốc như sau:

\begin{figure}[H]
    \centering
    \includegraphics[width=15cm]{Images/restaurant-customer-journey.png}
    \caption{Giới thiệu về Hành trình Khách hàng trong Nhà hàng.}
    \label{fig:my_label}
\end{figure}

\begin{enumerate}
    \item Giai đoạn Nhận thức (Awareness): Giai đoạn này đánh dấu điểm tiếp xúc đầu tiên của khách hàng tiềm năng với nhà hàng. Các kênh truyền thông đa dạng, từ quảng cáo trực tuyến (ví dụ: Google Ads, quảng cáo trên mạng xã hội), đánh giá trực tuyến (ví dụ: Google Reviews, TripAdvisor), đến các phương pháp truyền thống như quảng cáo in ấn, giới thiệu từ người thân, đóng vai trò quan trọng trong việc tạo dựng nhận diện thương hiệu.
    \item Giai đoạn Cân nhắc (Consideration): Sau khi nhận biết về nhà hàng, khách hàng tiềm năng sẽ tiến hành thu thập thông tin chi tiết hơn, so sánh với các lựa chọn khác. Họ sẽ xem xét thực đơn, bảng giá, hình ảnh không gian, đọc các đánh giá chi tiết và có thể tham khảo ý kiến từ bạn bè, người thân.
    \item Giai đoạn Quyết định (Decision): Giai đoạn này là thời điểm khách hàng đưa ra quyết định lựa chọn nhà hàng. Các yếu tố ảnh hưởng đến quyết định bao gồm sự thuận tiện trong việc đặt bàn, vị trí địa lý, đánh giá tổng quan về nhà hàng, và các chương trình khuyến mãi, ưu đãi đặc biệt.
    \item Giai đoạn Trải nghiệm (Experience): Đây là giai đoạn quan trọng nhất, khi khách hàng trực tiếp trải nghiệm dịch vụ và sản phẩm tại nhà hàng. Chất lượng món ăn, thái độ phục vụ, không gian nhà hàng, sự sạch sẽ và tiện nghi đều đóng vai trò then chốt. Việc ứng dụng công nghệ, ví dụ như hệ thống gọi món, thanh toán không tiền mặt, có thể nâng cao trải nghiệm khách hàng.
    \item Giai đoạn Thanh toán (Payment): Quá trình thanh toán cần được thực hiện nhanh chóng, chính xác và thuận tiện. Cung cấp đa dạng các phương thức thanh toán (ví dụ: tiền mặt, thẻ tín dụng, ví điện tử) để đáp ứng nhu cầu của khách hàng.
    \item Giai đoạn Hậu trải nghiệm (Post-Experience): Sau khi khách hàng rời nhà hàng, việc duy trì kết nối và khuyến khích quay lại là rất quan trọng. Gửi email cảm ơn, mời tham gia khảo sát đánh giá, cung cấp các chương trình khách hàng thân thiết, và gửi các ưu đãi đặc biệt là những phương pháp hiệu quả.
\end{enumerate}

\subsection{Mô hình MVC}
MVC là viết tắt của khái niệm "Model-View-Controller", một trong những mô hình thiết kế phần mềm phổ biến nhất.\\

MVC tách biệt dữ liệu, giao diện người dùng và logic xử lý thành ba thành phần riêng biệt nhưng vẫn được kết nối chặt chẽ với nhau.

\begin{itemize}
    \item Model (M): Đại diện cho dữ liệu và logic xử lý các nghiệp vụ của ứng dụng.
    \item View (V): Quản lý giao diện và hiển thị dữ liệu ra cho người dùng.
    \item Controller (C): Làm điều phối và điều hướng tương tác giữa Model và View. Nó nhận yêu cầu từ View, thực hiện xử lý trên Model và trả kết quả về cho View hiển thị.
\end{itemize}

MVC giúp tách biệt các thành phần của ứng dụng, tăng tính bảo trì và khả năng mở rộng trong tương lai \cite{MVC}.

\begin{figure}[H]
    \centering
    \includegraphics[width=10cm]{Images/ezgif-4-c53032b3fd.png}
    \vspace{0.5cm}
    \caption{MVC là gì?}
    \label{fig:my_label}
\end{figure}

Mô hình MVC (MVC pattern) thường được dùng để phát triển giao diện người dùng. Nó cung cấp các thành phần cơ bản để thiết kế một chương trình cho máy tính hoặc điện thoại di động, cũng như là các ứng dụng web.

\subsubsection{Các thành phần của MVC}
Mô hình MVC gồm 3 loại chính là thành phần bên trong không thể thiếu khi áp dụng mô hình này:
\begin{figure}[H]
    \centering
    \includegraphics[width=10cm]{Images/cacthanhphanmvc.png}
    \vspace{0.5cm}
    \caption{Thành phần của MVC}
    \label{fig:my_label}
\end{figure}

\begin{itemize}
    \item Model: Là bộ phận có chức năng lưu trữ toàn bộ dữ liệu của ứng dụng và là cầu nối giữa 2 thành phần bên dưới là View và Controller. Một model là dữ liệu được sử dụng bởi chương trình. Đây có thể là cơ sở dữ liệu, hoặc file XML bình thường hay một đối tượng đơn giản. Chẳng hạn như biểu tượng hay là một nhân vật trong game.
    \item View: Đây là phần giao diện (theme) dành cho người sử dụng. View là phương tiện hiển thị các đối tượng trong một ứng dụng. Chẳng hạn như hiển thị một cửa sổ, nút hay văn bản trong một cửa sổ khác. Nó bao gồm bất cứ thứ gì mà người dùng có thể nhìn thấy được.
    \item Controller: Là bộ phận có nhiệm vụ xử lý các yêu cầu người dùng đưa đến thông qua View. Một controller bao gồm cả Model lẫn View. Nó nhận input và thực hiện các update tương ứng.
\end{itemize}

\subsubsection{Luồng xử lý trong MVC}
Luồng xử lý trong của mô hình MVC, bạn có thể hình dung cụ thể và chi tiết qua từng bước dưới đây:
\begin{itemize}
    \item Khi một yêu cầu của từ máy khách (Client) gửi đến Server. Thì bị Controller trong MVC chặn lại để xem đó là URL request hay sự kiện.
    \item Sau đó, Controller xử lý input của user rồi giao tiếp với Model trong MVC.
    \item Model chuẩn bị data và gửi lại cho Controller.
    \item Cuối cùng, khi xử lý xong yêu cầu thì Controller gửi dữ liệu trở lại View và hiển thị cho người dùng trên trình duyệt.
\end{itemize}
\begin{figure}[H]
    \centering
    \includegraphics[width=10cm]{Images/luongmvc.png}
    \vspace{0.5cm}
    \caption{View và Model sẽ được xử lý bởi Controller}
    \label{fig:my_label}
\end{figure}

Ở đây, View không giao tiếp trực tiếp với Model. Sự tương tác giữa View và Model sẽ chỉ được xử lý bởi Controller.
% \newpage
% \section{CÁC HỆ THỐNG LIÊN QUAN}
\subsection{Ristorante Cracco}

\begin{figure}[H]
    \centering
    \includegraphics[width=15cm]{Images/ristorante_cracco.png}
    \vspace{0.5cm}
    \caption{Giao diện website Ristorante Cracco}
    \label{fig:my_label}
\end{figure}

Ristorante Cracco là một nhà hàng cao cấp tọa lạc tại trung tâm Milano, Ý, trong khu mua sắm lịch sử Galleria Vittorio Emanuele II. Được điều hành bởi đầu bếp danh tiếng Carlo Cracco, nhà hàng mang đến trải nghiệm ẩm thực tinh tế, kết hợp giữa truyền thống và sự sáng tạo hiện đại. Không gian của Ristorante Cracco trải rộng trên nhiều tầng, bao gồm nhà hàng chính, quán cà phê, hầm rượu và khu vực tổ chức sự kiện riêng. Thực đơn đa dạng với các món ăn độc đáo như súp cá bọc vỏ bánh và lòng đỏ trứng ngâm với măng tây và nấm cục đen, cùng với các món truyền thống như risotto nghệ với tủy xương nướng và ragù gan. Hầm rượu của nhà hàng chứa hơn 2.000 nhãn hiệu và hơn 10.000 chai rượu, chủ yếu từ Ý và Pháp. Trang web chính thức của Ristorante Cracco \href{https://www.ristorantecracco.it}{"https://www.ristorantecracco.it"} cung cấp thông tin chi tiết về thực đơn, đặt bàn trực tuyến và các sự kiện đặc biệt, giúp khách hàng dễ dàng tiếp cận và trải nghiệm dịch vụ đẳng cấp của nhà hàng.

\subsection{KFC Việt Nam}

\begin{figure}[H]
    \centering
    \includegraphics[width=15cm]{Images/kfc.png}
    \vspace{0.5cm}
    \caption{Giao diện website KFC Việt Nam}
    \label{fig:my_label}
\end{figure}

KFC Việt Nam là thành viên của tập đoàn Yum! Brands Inc. (Hoa Kỳ), chuyên cung cấp các sản phẩm gà rán và nướng, cùng với các món ăn kèm và sandwich chế biến từ thịt gà tươi. Kể từ khi khai trương nhà hàng đầu tiên tại TP. Hồ Chí Minh vào năm 1997, KFC đã mở rộng mạng lưới lên hơn 140 nhà hàng trên 21 tỉnh/thành phố, tạo việc làm cho hơn 3.000 lao động. Trang web chính thức của KFC Việt Nam \href{https://kfcvietnam.com.vn}{"https://kfcvietnam.com.vn"} cung cấp thông tin chi tiết về thực đơn đa dạng, bao gồm các món gà rán truyền thống và những món ăn được điều chỉnh phù hợp với khẩu vị Việt như Gà Big'n Juicy, Gà Giòn Không Xương, Cơm Gà KFC và Bắp Cải Trộn. Ngoài ra, trang web còn cập nhật các chương trình khuyến mãi, thông tin về nhà hàng và dịch vụ giao hàng trực tuyến, giúp khách hàng dễ dàng đặt món và tận hưởng hương vị KFC mọi lúc, mọi nơi.

\subsection{Haidilao}

\begin{figure}[H]
    \centering
    \includegraphics[width=15cm]{Images/haidilao.png}
    \vspace{0.5cm}
    \caption{Giao diện website Haidilao}
    \label{fig:my_label}
\end{figure}

Haidilao là chuỗi nhà hàng lẩu nổi tiếng, thành lập năm 1994 tại Giản Dương, Tứ Xuyên, Trung Quốc. Với dịch vụ khách hàng xuất sắc và hương vị lẩu đặc trưng, Haidilao đã mở rộng ra toàn cầu với hơn 1.300 nhà hàng tại nhiều quốc gia, bao gồm Việt Nam. Trang web chính thức của Haidilao \href{https://www.haidilao.com/}{"https://www.haidilao.com"} cung cấp thông tin về thực đơn, địa điểm các chi nhánh và dịch vụ khách hàng, giúp thực khách dễ dàng tiếp cận và trải nghiệm ẩm thực độc đáo của Haidilao.

\subsection{Yoshinoya}

\begin{figure}[H]
    \centering
    \includegraphics[width=15cm]{Images/yoshinoya.png}
    \vspace{0.5cm}
    \caption{Giao diện website Yoshinoya}
    \label{fig:my_label}
\end{figure}

Yoshinoya là một chuỗi nhà hàng nổi tiếng của Nhật Bản, ra đời từ năm 1899 tại Tokyo, chuyên phục vụ các món ăn truyền thống như gyudon (cơm bò hầm), cơm gà teriyaki và súp miso. Với hơn 120 năm lịch sử, Yoshinoya tự hào mang đến hương vị đậm đà, nguyên liệu tươi ngon và dịch vụ nhanh chóng, tiện lợi. Hiện nay, chuỗi này có mặt tại nhiều quốc gia, trở thành biểu tượng của ẩm thực Nhật Bản đơn giản nhưng tinh tế. Trang web chính thức của Yoshinoya \href{www.yoshinoya.com}{"www.yoshinoya.com"}  giúp khách hàng có thể thưởng thức món ăn yêu thích mọi lúc, mọi nơi thông qua các dịch vụ mà họ cung cấp.

\subsection{Cơm Niêu Sài Gòn}

\begin{figure}[H]
    \centering
    \includegraphics[width=15cm]{Images/comnieusaigon.png}
    \vspace{0.5cm}
    \caption{Giao diện website Cơm niêu Sài Gòn}
    \label{fig:my_label}
\end{figure}

Cơm Niêu Sài Gòn là một địa chỉ ẩm thực truyền thống nổi tiếng qua nhiều thế hệ. Nằm tại quận 3, TP.HCM, nhà hàng thu hút đông đảo khách hàng, đặc biệt là các gia đình và du khách, đến thưởng thức những món ăn đặc sản của miền Nam Việt Nam. Với không gian ấm cúng và thực đơn phong phú, nhà hàng mang đến những bữa ăn đậm đà hương vị quê hương, làm hài lòng cả những thực khách khó tính nhất. Bạn có thể khám phá thêm về thực đơn và dịch vụ của nhà hàng qua trang web chính thức của Cơm Niêu Sài Gòn tại \href{https://comnieusaigon.com}{"https://comnieusaigon.com"}.

\subsection{Thanh's Deli}

\begin{figure}[H]
    \centering
    \includegraphics[width=15cm]{Images/thanhsdeli.png}
    \vspace{0.5cm}
    \caption{Giao diện website Thanh's Deli}
    \label{fig:my_label}
\end{figure}

Thanh's Deli là một hệ thống quản lý đặt món và đặt bàn trực tuyến được phát triển bởi Nguyễn Duy Thành trong khuôn khổ đồ án tốt nghiệp. Hệ thống được thiết kế nhằm nâng cao trải nghiệm người dùng tại nhà hàng, giúp khách hàng dễ dàng thực hiện các thao tác đặt món và đặt bàn qua nền tảng trực tuyến.
% % % % % % % 
\subsection{So sánh với hệ thống của nhóm}

Chúng tôi sẽ tiến hành so sánh hệ thống của mình với các hệ thống khác trong ngành, tập trung vào các chức năng mà họ cung cấp cho khách hàng. Cụ thể, chúng tôi sẽ đánh giá các nội dung trên trang web của họ, cách thức trình bày các thông tin đó và mức độ tương tác giữa người dùng và hệ thống.

Chúng tôi sẽ kiểm tra xem các hệ thống có cung cấp 10 nội dung sau đây không: Giới thiệu, Thực đơn, Đặt món trực tuyến, Đặt bàn, Đánh giá và bình luận, Khuyến mãi, Thông tin liên hệ, Đăng nhập/Tài khoản cá nhân, Tương thích di động, và Hỗ trợ khách hàng. Chúng tôi sẽ đánh giá cách thức trình bày và mức độ hoàn thiện của từng nội dung theo các cấp độ phân loại chi tiết dưới đây:

\begin{enumerate}
	\item Giới thiệu
	      \begin{itemize}
		      \item Cấp độ 1: Cung cấp thông tin cơ bản về nhà hàng (tên, địa chỉ).
		      \item Cấp độ 2: Bao gồm thông tin cơ bản cùng với lịch sử, sứ mệnh và tầm nhìn của nhà hàng.
		      \item Cấp độ 3: Ngoài các thông tin trên, còn có câu chuyện thương hiệu, giới thiệu đội ngũ quản lý và đầu bếp, cùng các giải thưởng đã đạt được.
	      \end{itemize}
	\item Thực đơn
	      \begin{itemize}
		      \item Cấp độ 1: Danh sách các món ăn và đồ uống cơ bản.
		      \item Cấp độ 2: Thực đơn kèm theo mô tả chi tiết và giá cả cho từng món.
		      \item Cấp độ 3: Thực đơn bao gồm hình ảnh chất lượng cao cho mỗi món, thông tin về nguyên liệu và giá trị dinh dưỡng.
	      \end{itemize}
	\item Đặt món trực tuyến
	      \begin{itemize}
		      \item Cấp độ 1: Thực đơn sẽ có phần riêng để khách hàng thêm món ăn vào đơn hàng của mình.
		      \item Cấp độ 2: Có hệ thống giỏ hàng cho phép khách hàng có thể coi lại những món ăn trước khi tạo đơn.
		      \item Cấp độ 3: Cho phép thanh toán trực tuyến thông qua các hình thức thanh toán online.
	      \end{itemize}
	\item Đặt bàn
	      \begin{itemize}
		      \item Cấp độ 1: Cho phép đặt bàn thông qua form cơ bản.
		      \item Cấp độ 2: Hiển thị trạng thái bàn và cập nhật trạng thái theo thời gian thực.
		      \item Cấp độ 3: Cung cấp sơ đồ chi tiết của nhà hàng, cho phép người dùng xem tổng quan các vị trí bàn còn trống.
	      \end{itemize}
	\item Đánh giá và bình luận
	      \begin{itemize}
		      \item Cấp độ 1: Cho phép nhìn thấy những bình luận do nhà hàng thu thập được, nhưng không cho phép bình luận trực tiếp trên hệ thống.
		      \item Cấp độ 2: Cho phép khách hàng đánh giá, bình luận trực tiếp và nhận phản hồi từ quản lý nhà hàng.
	      \end{itemize}
	\item Khuyến mãi
	      \begin{itemize}
		      \item Cấp độ 1: Giảm giá cho một món ăn cụ thể với mức giảm cố định.
		      \item Cấp độ 2: Cho phép sử dụng mã giảm giá hoặc voucher.
		      \item Cấp độ 3: Cung cấp chương trình tích điểm dành cho khách hàng thân thiết.
	      \end{itemize}
	\item Thông tin liên hệ
	      \begin{itemize}
		      \item Cấp độ 1: Chỉ cung cấp địa chỉ và số điện thoại.
		      \item Cấp độ 2: Thêm email liên hệ và bản đồ chỉ đường.
		      \item Cấp độ 3: Có form liên hệ trực tuyến, liên kết với mạng xã hội và hỗ trợ chat trực tiếp.
	      \end{itemize}
	\item Đăng nhập/Tài khoản cá nhân
	      \begin{itemize}
		      \item Cấp độ 1: Cho phép tạo tài khoản, không dùng mạng xã hôi.
		      \item Cấp độ 2: Cho phép dùng mạng xã hôi để đăng nhập.
		      \item Cấp độ 3: Tài khoản cá nhân cho phép xem lịch sử đặt hàng, nhận ưu đãi dành riêng và quản lý thông tin cá nhân.
	      \end{itemize}
	\item Tương thích di động
	      \begin{itemize}
		      \item Cấp độ 1: Hiển thị trên di động nhưng chưa tối ưu giao diện và chức năng.
		      \item Cấp độ 2: Tương thích hoàn toàn với thiết bị di động, giao diện và chức năng được tối ưu hóa.
	      \end{itemize}
	\item Hỗ trợ khách hàng
	      \begin{itemize}
		      \item Cấp độ 1: Cung cấp số điện thoại hỗ trợ chỉ trong giờ hành chính (ví dụ: 8h-17h). Khách hàng có thể gọi khi cần hỗ trợ về các vấn đề cơ bản như đặt món, thắc mắc về dịch vụ.
		      \item Cấp độ 2: Cung cấp email hỗ trợ, khách hàng có thể gửi yêu cầu qua email về các vấn đề cần giải đáp, và nhận phản hồi trong vòng 24 giờ. Hỗ trợ các vấn đề liên quan đến đơn hàng, thanh toán hoặc yêu cầu thông tin thêm về dịch vụ.
		      \item Cấp độ 3: Hỗ trợ khách hàng qua nhiều kênh (chat trực tuyến, email, điện thoại, form liên hệ) với phản hồi nhanh chóng 24/7. Khách hàng có thể liên hệ bất cứ lúc nào để giải quyết các vấn đề khẩn cấp như yêu cầu thay đổi đơn hàng, khiếu nại, hay cần trợ giúp về dịch vụ trong suốt quá trình sử dụng.
	      \end{itemize}
\end{enumerate}

\begin{longtable}{|p{2cm}|p{1.5cm}|p{1.5cm}|p{1.5cm}|p{1.5cm}|p{1.5cm}|p{1.5cm}|p{1.5cm}|}
	\caption{Bảng so sánh các nhà hàng}                                                                                                                                                     \\
	\hline
	\textbf{Chức năng}          & \textbf{Menu+ (Our System)} & \textbf{Cracco} & \textbf{KFC} & \textbf{Haidilao} & \textbf{Yoshinoya} & \textbf{Cơm Niêu Sài Gòn} & \textbf{Thanh's Deli} \\
	\hline
	\endfirsthead
	\hline
	\textbf{Chức năng}          & \textbf{Menu+ (Our System)} & \textbf{Cracco} & \textbf{KFC} & \textbf{Haidilao} & \textbf{Yoshinoya} & \textbf{Cơm Niêu Sài Gòn} & \textbf{Thanh's Deli} \\
	\endhead
	\hline
	\multicolumn{8}{|r|}{\small\slshape Còn tiếp}                                                                                                                                           \\ \hline
	\endfoot
	\hline
	\endlastfoot
	Giới thiệu                  & 3                           & 2               & 3            & 3                 & 3                  & 3                         & 3                     \\
	\hline
	Thực đơn                    & 3                           & 2               & 3            & 2                 & 3                  & 2                         & 2                     \\
	\hline
	Đặt món trực tuyến          & 3                           & 0               & 3            & 3                 & 3                  & 3                         & 3                     \\
	\hline
	Đặt bàn                     & 3                           & 1               & 1            & 1                 & 2                  & 1                         & 3                     \\
	\hline
	Đánh giá và bình luận       & 2                           & 0               & 0            & 0                 & 2                  & 0                         & 2                     \\
	\hline
	Khuyến mãi                  & 2                           & 2               & 3            & 3                 & 2                  & 2                         & 2                     \\
	\hline
	Thông tin liên hệ           & 3                           & 1               & 3            & 2                 & 3                  & 3                         & 2                     \\
	\hline
	Đăng nhập/Tài khoản cá nhân & 3                           & 3               & 3            & 3                 & 1                  & 3                         & 1                     \\
	\hline
	Tương thích di động         & 2                           & 2               & 2            & 2                 & 2                  & 2                         & 2                     \\
	\hline
	Hỗ trợ khách hàng           & 2                           & 2               & 3            & 2                 & 1                  & 2                         & 2                     \\
	\hline
\end{longtable}

\subsection{Kết luận}
Việc so sánh hệ thống của nhóm với các hệ thống hiện có trong ngành nhà hàng đã giúp làm rõ những ưu điểm và nhược điểm của sản phẩm mà nhóm phát triển. Hệ thống của nhóm nổi bật với một số điểm mạnh quan trọng khi so sánh với các hệ thống liên quan.

Đầu tiên, hệ thống của chúng tôi được cải tiến với khả năng tích hợp các tính năng như tự gọi món qua ứng dụng và thanh toán trực tuyến. Mặc dù các hệ thống như Cơm Niêu Sài Gòn, KFC và Haidilao cũng đã triển khai các tính năng đặt món trực tuyến và thanh toán, nhưng hệ thống của nhóm mang đến một trải nghiệm người dùng dễ dàng và nhanh chóng hơn nhờ vào giao diện thân thiện và tiện lợi. Hơn nữa, hệ thống của nhóm cung cấp phản hồi thời gian thực và khả năng theo dõi đơn hàng một cách thuận tiện, điều mà một số hệ thống khác còn thiếu sót.

Mặc dù hệ thống của chúng tôi có những ưu điểm về việc tối ưu trải nghiệm người dùng và quy trình, song so với các hệ thống như Yoshinoya hay KFC, hệ thống của nhóm vẫn còn hạn chế về tính linh hoạt trong việc mở rộng các tính năng phức tạp hơn, như tích hợp chương trình khách hàng thân thiết hay quản lý các chương trình khuyến mãi phức tạp. Các tính năng như phân tích hành vi mua sắm và khuyến mãi dựa trên tần suất vẫn là những lĩnh vực cần cải tiến.

Tóm lại, hệ thống của chúng tôi thể hiện sự vượt trội trong việc tối ưu hóa quy trình đặt món, thanh toán, giúp nâng cao hiệu quả công việc và cải thiện trải nghiệm khách hàng. Tuy nhiên, để có thể phát triển mạnh mẽ hơn, hệ thống cần học hỏi và cải tiến thêm ở những lĩnh vực mở rộng tính năng và quản lý khách hàng. Việc so sánh này đã giúp nhóm nhận diện rõ các điểm mạnh hiện tại cũng như những khu vực cần cải thiện, làm cơ sở để phát triển hệ thống trong tương lai.







% \newpage
% \section{TỔNG QUAN VỀ HỆ THỐNG}

\subsection{Bối cảnh kinh doanh (Business context)}
Trong bối cảnh ngành công nghiệp thực phẩm và dịch vụ ngày càng phát triển và cạnh tranh khốc liệt, việc ứng dụng công nghệ vào quản lý nhà hàng đã trở thành một xu hướng tất yếu. Hệ thống quản lý chuỗi nhà hàng là một giải pháp kỹ thuật số được thiết kế để hỗ trợ và tối ưu hóa các hoạt động vận hành hàng ngày của nhà hàng (đặc biệt là nhà hàng fine dining), từ đặt bàn, xử lý đơn hàng, thanh toán cho đến quản lý nhân sự. Một hệ thống quản lý chuỗi nhà hàng hiệu quả không chỉ giúp nâng cao trải nghiệm của khách hàng mà còn cải thiện hiệu suất làm việc của nhân viên và tối ưu hóa quản lý doanh thu. 

% \subsubsection{Mục tiêu và phạm vi hệ thống}
% Hệ thống quản lý nhà hàng \textbf{\textit{Menu+}} được phát triển nhằm mục đích đơn giản hóa và nâng cao hiệu quả các hoạt động vận hành hàng ngày trong một nhà hàng fine dining. Các chức năng chính của \textbf{\textit{Menu+}} bao gồm:
% \begin{itemize}
%     \item \textbf{Đặt bàn trực tuyến và quản lý bàn}: Giúp khách hàng dễ dàng đặt chỗ và hỗ trợ nhân viên theo dõi trạng thái bàn một cách chính xác.
%      \item \textbf{Quản lý bàn}: Hỗ trợ nhân viên theo dõi trạng thái bàn (trống, đã đặt, đang phục vụ) và thực hiện các thao tác như chuyển bàn khi cần thiết.
%     \item \textbf{Xử lý đơn hàng}: Cho phép nhân viên nhập đơn hàng, tùy chỉnh theo yêu cầu của khách và gửi trực tiếp đến hệ thống hiển thị bếp (Kitchen Display System - KDS).
%     \item \textbf{Thanh toán và quản lý doanh thu}: Tích hợp các phương thức thanh toán đa dạng, tạo điều kiện thuận lợi cho cả khách hàng và nhà hàng.
%     \item \textbf{Quản lý quan hệ khách hàng (CRM) và nhân sự}: Lưu trữ thông tin khách hàng, quản lý lịch làm việc của nhân viên và cung cấp các báo cáo chi tiết về hoạt động kinh doanh.
% \end{itemize}
% \textbf{\textit{Menu+}} linh hoạt và thân thiện với người dùng đóng vai trò quan trọng trong việc giúp các nhà hàng, đặc biệt là những nhà hàng mới khởi nghiệp, mở rộng quy mô hoạt động một cách hiệu quả.

% \subsubsection{Ràng buộc kinh doanh (Business Constraint)}

% exclude

\subsubsection{Tổng quan hệ thống quản lý nhà hàng}

Dưới đây là mô tả tổng quan về các thành phần cấu thành hệ thống quản lý chuỗi nhà hàng, với các chức năng thiết yếu nhằm hỗ trợ việc vận hành và điều phối các hoạt động trong một chuỗi nhà hàng hiện đại. Những hệ thống con này đóng vai trò quan trọng trong việc tối ưu hóa hiệu quả công việc, nâng cao chất lượng dịch vụ và đảm bảo sự hoạt động trơn tru của toàn bộ chuỗi.

\begin{figure}[H]
    \centering
    \includegraphics[width=15cm]{Images/so-do-he-thong.png}
    \vspace{0.5cm}
    \caption{Sơ đồ tổng quan về một phần mềm Quản lý Nhà hàng}
    \label{fig:my_label}
\end{figure}

\begin{itemize}
    \item \textbf{Hệ thống Quản lý Đơn hàng (Order Management System - OMS)}: Đóng vai trò quan trọng trong việc tiếp nhận và xử lý các đơn đặt món từ khách hàng. Khi khách hàng đặt món, thông tin đơn hàng sẽ được hệ thống OMS ghi nhận và phân phối đến các bộ phận liên quan trong nhà hàng, như bếp để chế biến món ăn, thu ngân để xử lý thanh toán, và đội ngũ giao hàng nếu có. Hệ thống OMS không chỉ giúp theo dõi tình trạng của từng đơn hàng mà còn quản lý các thông tin liên quan đến khách hàng và việc giao nhận món ăn. Đặc biệt, OMS sẽ giúp nhà quản lý theo dõi hiệu suất của đơn hàng và cập nhật kịp thời trạng thái để khách hàng có thể nhận được món ăn một cách nhanh chóng. Mọi thay đổi về đơn hàng sẽ được cập nhật liên tục để các bộ phận liên quan có thể xử lý ngay khi có sự thay đổi, ví dụ như khi khách hàng hủy đơn hoặc có yêu cầu thay đổi món.

    \item \textbf{Hệ thống Quản lý Bếp (Kitchen Management System - KMS)}: Tập trung vào việc quản lý quá trình chế biến món ăn từ khi nhận được đơn hàng từ OMS cho đến khi món ăn hoàn thành và sẵn sàng phục vụ khách. Thông qua KMS, nhà bếp có thể lên kế hoạch chế biến cho từng món ăn và theo dõi hiệu quả làm việc của từng đầu bếp. Mỗi đơn hàng sẽ được chia thành các công đoạn nhỏ để các nhân viên bếp dễ dàng theo dõi và thực hiện. Hệ thống này cũng đóng vai trò trong việc quản lý chất lượng món ăn, đảm bảo rằng món ăn được chế biến đúng quy trình và đạt tiêu chuẩn chất lượng. KMS cũng kết nối với các hệ thống khác như Inventory Management System (IMS) để tự động kiểm tra và yêu cầu bổ sung nguyên liệu khi kho hàng thiếu hụt.

    \item \textbf{Hệ thống Quản lý Kho (Inventory Management System - IMS)}: Quản lý và giám sát tình trạng nguyên liệu trong kho của các chi nhánh nhà hàng. IMS giúp nhà quản lý kiểm soát số lượng nguyên liệu, hạn sử dụng và các mặt hàng còn lại trong kho để đảm bảo rằng luôn có đủ nguyên liệu cho việc chế biến món ăn. Hệ thống này sẽ tự động thông báo khi lượng nguyên liệu sắp hết hoặc sắp hết hạn sử dụng, giúp bộ phận kho có thể lên kế hoạch nhập hàng bổ sung từ các nhà cung cấp. IMS không chỉ giúp đảm bảo việc cung cấp nguyên liệu kịp thời cho KMS, mà còn giúp giảm thiểu tình trạng thiếu hụt nguyên liệu, qua đó đảm bảo việc phục vụ khách hàng không bị gián đoạn. Hệ thống này cũng giúp tối ưu hóa việc sử dụng nguyên liệu, giảm lãng phí và giúp duy trì chi phí vận hành hợp lý.

    \item \textbf{Hệ thống Quản lý Thanh toán (Payment Management System - PMS)}: Chịu trách nhiệm xử lý tất cả các giao dịch thanh toán của khách hàng sau khi món ăn được giao đến bàn hoặc giao tận nơi. Khi món ăn hoàn thành và khách hàng chuẩn bị thanh toán, hệ thống PMS sẽ tự động tính toán giá trị đơn hàng, bao gồm thuế, chiết khấu, và các khoản phí khác nếu có. Hệ thống này hỗ trợ nhiều phương thức thanh toán khác nhau, từ tiền mặt, thẻ tín dụng, thẻ ghi nợ, cho đến các phương thức thanh toán trực tuyến. Sau khi thanh toán hoàn tất, PMS sẽ in hóa đơn cho khách và cập nhật dữ liệu doanh thu vào hệ thống tài chính. Hệ thống này cũng đồng bộ hóa dữ liệu với các hệ thống báo cáo để nhà quản lý có cái nhìn tổng quan về tình hình tài chính của từng chi nhánh. PMS còn có khả năng theo dõi và phân tích các xu hướng chi tiêu của khách hàng để đưa ra các chiến lược giá hợp lý và tăng trưởng doanh thu.

    \item \textbf{Hệ thống Quản lý Nhân sự (Human Resource Management System - HRMS)}: Quản lý thông tin nhân viên, lịch làm việc và hiệu suất công việc của các nhân viên trong chuỗi nhà hàng. Hệ thống này theo dõi số giờ làm việc, ca làm việc của nhân viên, và các yếu tố liên quan đến chấm công, bảo hiểm, tiền lương. Ngoài ra, HRMS còn hỗ trợ việc phân công công việc cho các nhân viên bếp, thu ngân, và các bộ phận khác, giúp việc tổ chức công việc trở nên khoa học và hợp lý. Các tính năng như theo dõi kỳ nghỉ, đào tạo và phát triển nhân viên cũng được HRMS đảm bảo. Một trong những chức năng quan trọng của HRMS là giúp tối ưu hóa quy trình tuyển dụng, giúp tuyển chọn nhân viên phù hợp với yêu cầu công việc. Ngoài ra, HRMS còn hỗ trợ các bộ phận quản lý nhân sự của từng chi nhánh trong việc đánh giá hiệu suất làm việc và cải thiện chất lượng nhân sự.

    \item \textbf{Hệ thống Báo cáo \& Phân tích (Reporting \& Analytics System - R\&A)}: Quản lý có cái nhìn tổng quan về hiệu suất của toàn chuỗi nhà hàng. Hệ thống này thu thập và tổng hợp dữ liệu từ các hệ thống khác nhau như OMS, KMS, IMS, PMS và HRMS để tạo ra các báo cáo chi tiết. Những báo cáo này không chỉ về tình hình doanh thu, mà còn cung cấp các thông tin liên quan đến hiệu quả công việc của nhân viên, tình trạng kho nguyên liệu, mức độ hài lòng của khách hàng, và nhiều yếu tố khác. Thông qua phân tích dữ liệu, nhà quản lý có thể đưa ra những quyết định chiến lược giúp cải thiện quy trình hoạt động, tối ưu hóa chi phí và nâng cao chất lượng dịch vụ. Hệ thống cũng hỗ trợ tạo ra các báo cáo tài chính, giúp các chi nhánh có thể theo dõi doanh thu và chi phí một cách chi tiết.

    \item \textbf{Hệ thống Quản lý Chuỗi Cung ứng (Supply Chain Management System - SCM)}: Quản lý và điều phối mối quan hệ với các nhà cung cấp, đảm bảo nguồn nguyên liệu luôn được cung cấp đầy đủ và đúng chất lượng. SCM theo dõi tình trạng đơn hàng từ khi nguyên liệu được đặt hàng cho đến khi nhận được hàng và đưa vào kho. Hệ thống này không chỉ giúp kiểm soát chất lượng nguồn cung mà còn tối ưu hóa quá trình giao nhận nguyên liệu, giảm thiểu tình trạng thiếu hụt hoặc tồn đọng hàng hóa trong kho. Bằng cách kết hợp dữ liệu từ IMS và KMS, SCM có thể dự báo nhu cầu nguyên liệu cho các món ăn, từ đó có kế hoạch đặt hàng hiệu quả hơn.
    
    \item \textbf{Hệ thống Quản lý Khách hàng (Customer Relationship Management - CRM)}: Lưu trữ và quản lý thông tin của khách hàng, bao gồm các thông tin cá nhân, lịch sử đơn hàng và các thói quen tiêu dùng. Hệ thống CRM giúp nhà hàng không chỉ lưu giữ thông tin khách hàng mà còn tạo dựng mối quan hệ lâu dài với khách hàng thông qua các chương trình khách hàng thân thiết và các ưu đãi cá nhân hóa. Bằng cách phân tích dữ liệu từ CRM, nhà hàng có thể hiểu rõ hơn về nhu cầu và sở thích của khách hàng, từ đó đề xuất các món ăn phù hợp, tạo ra những trải nghiệm cá nhân hóa. Hệ thống này còn hỗ trợ việc gửi các thông báo về chương trình khuyến mãi, sự kiện đặc biệt, hoặc thông tin về các sản phẩm mới đến khách hàng, giúp duy trì và phát triển mối quan hệ với khách hàng cũ và thu hút khách hàng mới.

    \item \textbf{Hệ thống Quản lý Marketing \& Khuyến mãi (Marketing \& Promotion Management System)}: Lên kế hoạch, triển khai và theo dõi các chiến dịch marketing và chương trình khuyến mãi. Hệ thống này hỗ trợ việc tạo ra các chiến dịch quảng bá các món ăn, khuyến mãi theo mùa hoặc các chương trình giảm giá đặc biệt cho khách hàng. Marketing \& Promotion Management System cung cấp các công cụ để tạo mã giảm giá, quản lý các chương trình khuyến mãi và theo dõi hiệu quả của từng chiến dịch. Hệ thống này còn giúp phân tích dữ liệu khách hàng từ CRM, từ đó xác định đối tượng mục tiêu cho các chiến dịch marketing, giúp tăng tỷ lệ chuyển đổi và tối ưu hóa chi phí marketing. Ngoài ra, các chiến dịch và khuyến mãi cũng có thể được tích hợp với hệ thống thanh toán để khách hàng có thể dễ dàng sử dụng các ưu đãi khi thanh toán.

    \item \textbf{Hệ thống Quản lý Phản hồi \& Khiếu nại (Feedback \& Complaint Management System)}: Nhận và xử lý các phản hồi, khiếu nại từ khách hàng. Việc lắng nghe và giải quyết nhanh chóng các vấn đề của khách hàng giúp nhà hàng cải thiện chất lượng dịch vụ và tạo dựng lòng tin của khách hàng. Hệ thống này giúp theo dõi các khiếu nại về món ăn, thái độ phục vụ, không gian nhà hàng hoặc các vấn đề khác. Sau khi nhận được phản hồi hoặc khiếu nại từ khách hàng, hệ thống sẽ tự động phân loại và chuyển đến các bộ phận liên quan để xử lý, từ đó giúp khách hàng cảm thấy hài lòng hơn với dịch vụ của nhà hàng. Hệ thống này cũng có chức năng theo dõi các phản hồi tích cực để có thể ghi nhận và thưởng cho những nhân viên hoặc bộ phận có đóng góp xuất sắc.

    \item \textbf{Hệ thống Quản lý Chuỗi Nhà hàng (Chain Management System - CMS)}: Trung tâm quản lý tổng thể của chuỗi các chi nhánh nhà hàng. Hệ thống này giúp giám sát các hoạt động của tất cả các chi nhánh từ một hệ thống tập trung, bao gồm việc theo dõi doanh thu, tồn kho, nhân sự, cũng như các hoạt động vận hành khác của từng chi nhánh. CMS không chỉ giúp đồng bộ hóa các quy trình giữa các chi nhánh mà còn cung cấp các báo cáo tài chính và hoạt động chi tiết để hỗ trợ các quyết định quản lý chiến lược. Hệ thống này tích hợp dữ liệu từ các hệ thống khác như PMS, KMS, IMS, giúp nhà quản lý chuỗi có thể theo dõi tình trạng của từng chi nhánh và đưa ra các quyết định kịp thời để tối ưu hóa hoạt động của toàn chuỗi.

    \item \textbf{Hệ thống Quản lý Giao hàng (Delivery Management System)}: Quản lý các đơn hàng giao tận nơi, bao gồm việc điều phối các nhân viên giao hàng, theo dõi tình trạng đơn hàng và tối ưu hóa thời gian giao hàng. Hệ thống này giúp các nhân viên giao hàng nhận được thông tin đơn hàng một cách nhanh chóng, biết rõ địa chỉ giao hàng và các yêu cầu đặc biệt của khách hàng (nếu có). Hệ thống còn giúp theo dõi trạng thái của đơn hàng từ khi rời khỏi nhà hàng cho đến khi giao đến tay khách hàng, đồng thời cung cấp các công cụ để quản lý các tuyến đường giao hàng sao cho hiệu quả và tiết kiệm thời gian. Việc tích hợp hệ thống này với OMS giúp cập nhật trạng thái của đơn hàng cho khách hàng trong thời gian thực và đảm bảo dịch vụ giao hàng nhanh chóng, chính xác.
\end{itemize}

\subsubsection{Chính sách vận hành (Policy)}
Các nhà hàng hiện đại ngày nay thường áp dụng nhiều chính sách linh hoạt nhằm nâng cao chất lượng phục vụ và giảm thiểu các rủi ro trong kinh doanh. Các chính sách phổ biến được áp dụng có thể kể tới như sau:

\begin{itemize}
    \item \textbf{Đa dạng hóa phương thức thanh toán}: Chính sách này tập trung vào việc cung cấp đa dạng các hình thức thanh toán, như tiền mặt, thẻ tín dụng, ví điện tử và mã QR. Bên cạnh đó, các nhà hàng cũng triển khai hệ thống quản lý thanh toán tự động để rút ngắn thời gian xử lý, giảm thiểu sai sót và tăng trải nghiệm hài lòng của khách hàng. Chính sách này đã được các nhà hàng như \textit{Ngưu Phồn} và \textit{Cơm niêu Sài Gòn} áp dụng rất hiệu quả.
    
    \item \textbf{Yêu cầu đặt cọc khi đặt bàn}: Để giảm thiểu tổn thất khi khách hàng hủy đặt bàn vào phút chót hoặc không đến nhà hàng mà không báo trước, nhiều nhà hàng áp dụng chính sách yêu cầu khách hàng đặt cọc trước một khoản tiền nhất định, thường dao động từ 10\% đến 20\% tổng giá trị bàn tiệc. Khoản đặt cọc này sẽ được khấu trừ vào hóa đơn thanh toán hoặc bị giữ lại nếu khách hủy đặt bàn trễ hơn thời gian quy định. Một số nhà hàng áp dụng thành công chính sách này là \textit{Nhà Hàng Phúc Thành} và \textit{Vân Nghĩa Palace}.
    
    \item \textbf{Xác nhận đặt bàn trước ngày hẹn}: Chính sách này yêu cầu nhân viên nhà hàng chủ động liên hệ với khách hàng trước ngày đặt bàn để xác nhận lại thông tin và nhắc nhở khách đến đúng giờ. Điều này không chỉ giúp giảm tình trạng khách quên hay thay đổi kế hoạch mà không thông báo, mà còn giúp nhà hàng quản lý hiệu quả hơn trong việc chuẩn bị dịch vụ. Nhà hàng áp dụng tiêu biểu chính sách này là \textit{Nhà Hàng Phúc Thành}.
    
    \item \textbf{Chính sách hủy đặt bàn và hoàn tiền rõ ràng}: Các nhà hàng thường đưa ra quy định chi tiết về thời hạn hủy đặt bàn và mức phí áp dụng khi hủy. Điều này giúp khách hàng hiểu rõ trách nhiệm và quyền lợi của mình khi sử dụng dịch vụ, đồng thời tạo ra sự minh bạch và uy tín trong hoạt động kinh doanh. Chính sách này được áp dụng rõ ràng tại \textit{Nhà Hàng Ocean Bay Vũng Tàu}.
    
    \item \textbf{Điều kiện hủy đơn hàng}: Chính sách này quy định rõ ràng các điều kiện và thời điểm cụ thể khách hàng được phép hủy đơn hàng, thường là trước khi nhà hàng xác nhận và bắt đầu chuẩn bị món ăn. Điều này giúp nhà hàng giảm thiểu tổn thất về nguyên vật liệu và công sức chế biến không cần thiết. Một ví dụ cụ thể về chính sách này là tại nhà hàng \textit{Patyko}.
    
    \item \textbf{Phí hủy đơn hàng}: Để bảo vệ quyền lợi và giảm thiểu thiệt hại khi khách hàng hủy đơn sau khi nhà hàng đã bắt đầu chế biến, nhiều nhà hàng đặt ra quy định thu phí hủy hoặc không hoàn lại tiền đặt cọc trong một số trường hợp nhất định. Chính sách này nhằm bù đắp một phần chi phí nguyên vật liệu và nhân công đã bỏ ra. Điển hình trong việc áp dụng chính sách này là \textit{Nhà Hàng Khoái}.
    
    \item \textbf{Chính sách đổi trả sản phẩm}: Để đảm bảo quyền lợi khách hàng và duy trì chất lượng dịch vụ, các nhà hàng thường áp dụng chính sách đổi trả rõ ràng. Khách hàng được quyền đổi hoặc trả lại món ăn trong các trường hợp món bị lỗi, hỏng, không thể sử dụng hoặc không đảm bảo vệ sinh an toàn thực phẩm. Chính sách này giúp tạo dựng lòng tin và gia tăng uy tín của nhà hàng. Một số nhà hàng nổi bật áp dụng thành công chính sách này là \textit{Sườn Mười} và \textit{Nhà Hàng Khoái}.
    
\end{itemize}



\begin{table}[H]
\centering
\begin{tabular}{|p{4cm}|p{8cm}|p{4cm}|}
\hline
\textbf{Tên chính sách} & \textbf{Mô tả} & \textbf{Nhà hàng áp dụng} \\
\hline
Đa dạng hóa phương thức thanh toán & Cung cấp nhiều phương thức thanh toán như tiền mặt, thẻ tín dụng, ví điện tử và mã QR, đồng thời áp dụng hệ thống quản lý thanh toán tự động để giảm thiểu sai sót và tăng tốc độ phục vụ. & Ngưu Phồn, Cơm niêu Sài Gòn \\
\hline
Yêu cầu đặt cọc khi đặt bàn & Để giảm thiểu tình trạng khách hàng hủy bàn vào phút chót hoặc không đến mà không báo trước, nhiều nhà hàng yêu cầu khách đặt cọc trước một khoản tiền, thường khoảng 10-20\% tổng giá trị bàn tiệc. & Nhà Hàng Phúc Thành, Vân Nghĩa Palace \\
\hline
Xác nhận đặt bàn trước ngày hẹn & Trước ngày đặt bàn, nhân viên nhà hàng nên gọi điện hoặc nhắn tin xác nhận lại với khách hàng để nhắc nhở và đảm bảo họ sẽ đến. & Nhà Hàng Phúc Thành \\
\hline
Chính sách hủy và hoàn tiền rõ ràng & Nhà hàng cần quy định rõ ràng về thời hạn hủy đặt bàn và mức phí hủy, giúp khách hàng nắm rõ quyền lợi và trách nhiệm của mình. & Nhà Hàng Ocean Bay Vũng Tàu \\
\hline
Điều kiện hủy đơn hàng & Quy định rõ ràng về thời điểm và điều kiện khách hàng có thể hủy đơn hàng, đại khái hơn là trước khi nhà hàng xác nhận và lên đơn hàng. & Patyko \\
\hline
Phí hủy đơn hàng & Áp dụng phí hủy hoặc không hoàn tiền đặt cọc nếu khách hủy sau một thời điểm nhất định, giúp bù đắp chi phí nguyên liệu và công sức chuẩn bị. & Nhà Hàng Khoái \\
\hline
Chính sách đổi trả sản phẩm & Chấp nhận đổi, trả các sản phẩm bị lỗi, hỏng, không thể sử dụng hoặc không đảm bảo vệ sinh an toàn thực phẩm. & Sườn Mười, Nhà Hàng Khoái \\
\hline
\end{tabular}
\caption{Tổng hợp các chính sách, mô tả và tham khảo từ các nhà hàng hoặc hệ thống liên quan}
\end{table}

\subsection{Người dùng và mục đích hệ thống Menu+}

Mục đích của hệ thống quản lý nhà hàng \textbf{\textit{Menu+}} là cung cấp một giải pháp toàn diện nhằm tối ưu hóa và nâng cao hiệu quả các hoạt động vận hành hàng ngày trong một nhà hàng fine dining. Hệ thống được phát triển với mục tiêu hỗ trợ các nhóm người dùng chính trong nhà hàng, bao gồm khách hàng, thu ngân, quản lý, quản trị viên, bếp trưởng, nhân viên phục vụ và nhân viên vệ sinh.

\begin{figure}[H]
    \centering
    \includegraphics[width=15cm]{Images/OMS-Page-2.png}
    \vspace{0.5cm}
    \caption{Các người dùng trong hệ thống}
    \label{fig:my_label}
\end{figure}

% \begin{table}[h!]
% \centering
% \begin{tabular}{|p{3cm}|p{3cm}|p{9cm}|}
% \hline
% \textbf{Mã đối tượng} & \textbf{Tên} & \textbf{Mô tả} \\ \hline
% US-01 & Customer (Khách hàng) & Xem menu trực tuyến, đặt bàn hoặc order trực tiếp tại nhà hàng, thanh toán và sử dụng mã khuyến mãi. \\ \hline
% US-02 & Cashier (Thu ngân) & Thực hiện xác nhận thanh toán, gộp hoặc tách bill, hỗ trợ khách hàng sử dụng mã khuyến mãi, nhập số tiền nhận và tính tiền thối lại. \\ \hline
% US-03 & Manager (Quản lý) & Quản lý order, doanh thu, nhân sự, chia ca, thêm nhân sự mới, và điều hành chung hệ thống nhà hàng. \\ \hline
% US-04 & Administrator (Quản trị viên) & Quản lý và duy trì hệ thống, cấu hình menu, điều chỉnh các thông tin liên quan đến hệ thống chung. \\ \hline
% US-05 & Chef (Bếp trưởng) & Xem Kitchen Display System (KDS), xác nhận món ăn từ trạng thái "not ready", chuyển sang "cook" và cuối cùng là "ready to serve". \\ \hline
% US-06 & Waiter (Nhân viên phục vụ) & Giúp khách hàng đặt bàn, order món, lấy món ăn từ bếp, phục vụ món ăn, chuyển bàn, thực hiện self-order cho khách, cập nhật trạng thái món ăn là đã phục vụ. \\ \hline
% US-07 & Cleaning Staff (Nhân viên vệ sinh) & Dọn dẹp bàn ăn sau khi khách rời đi, cập nhật tình trạng bàn trên hệ thống để sẵn sàng phục vụ khách tiếp theo. \\ \hline
% \end{tabular}
% \caption{Bảng tổng hợp đối tượng và chức năng trong hệ thống quản lý nhà hàng}
% \label{tab:restaurant_objects}
% \end{table}

\textbf{Các chức năng chính của Menu+}

Hệ thống Menu+ bao gồm các chức năng chính giúp các nhóm người dùng trên thực hiện các công việc hàng ngày một cách hiệu quả và nhanh chóng:

\begin{itemize}


    \item Đặt bàn trực tuyến và quản lý bàn: Giúp khách hàng dễ dàng đặt chỗ và hỗ trợ nhân viên theo dõi trạng thái bàn (trống, đã đặt, đang phục vụ).

    \item Quản lý bàn: Cung cấp công cụ cho nhân viên theo dõi và quản lý bàn trong nhà hàng.

    \item Xử lý đơn hàng: Cho phép nhân viên nhập và gửi đơn hàng đến hệ thống Kitchen Display System (KDS) cho bếp.

    \item Thanh toán và quản lý doanh thu: Tích hợp các phương thức thanh toán đa dạng, tạo thuận tiện cho cả khách hàng và nhà hàng.

    \item Quản lý quan hệ khách hàng (CRM) và nhân sự: Lưu trữ thông tin khách hàng, quản lý lịch làm việc của nhân viên và cung cấp các báo cáo chi tiết về hoạt động kinh doanh.
\end{itemize}

\subsection{Đặc tả chức năng}
\begin{longtable}{|m{1.5cm}|m{3.5cm}|m{4.5cm}|m{5cm}|}
\hline
\textbf{Mã} & \textbf{Tên Người Dùng} & \textbf{Vai Trò Thực Tế} & \textbf{Mô Tả Ngắn} \\
\hline
\endhead % Header cho các trang tiếp theo

\hline
\endfoot % Footer cho bảng

\hline
\endlastfoot % Footer cho trang cuối cùng

US-01 & Quản lý nhà hàng & Chủ nhà hàng hoặc người quản lý cấp cao & Quản lý tổng thể hoạt động, cấu hình hệ thống (giá bàn, \% đặt cọc, thời gian gọi bot), xem báo cáo toàn diện, quản lý nhân viên và thực đơn. \\
\hline
US-02 & Nhân viên phục vụ & Waiter/Waitress & Sử dụng POS để nhận đơn hàng tại bàn, quản lý trạng thái bàn, xử lý thanh toán (bao gồm cả việc trừ tiền đặt cọc), tương tác với hệ thống bếp. \\
\hline
US-03 & Nhân viên lễ tân & Host/Hostess (Có thể là Nhân viên phục vụ đảm nhiệm) & Quản lý sơ đồ tầng, trạng thái bàn, nhận và quản lý đặt chỗ trực tiếp hoặc qua điện thoại (ít tương tác hệ thống hơn khách hàng tự đặt). \\
\hline
US-04 & Nhân viên bếp & Chef, Cook, Kitchen Assistant & Tương tác chính với Màn hình hiển thị Bếp (KDS) hoặc máy in phiếu bếp để xem chi tiết đơn hàng (bao gồm món đặt trước) và cập nhật trạng thái chuẩn bị. \\
\hline
US-05 & Nhân viên thu ngân & Cashier (Có thể là Nhân viên phục vụ hoặc Quản lý) & Chịu trách nhiệm đóng/mở phiên POS, đối soát tiền mặt cuối ngày, xử lý các giao dịch thanh toán. \\
\hline
US-06 & Kế toán & Accountant/Bookkeeper & Truy cập dữ liệu bán hàng đã được tổng hợp, báo cáo tài chính, quản lý công nợ (nếu có), đối soát doanh thu và tiền đặt cọc. \\
\hline
US-07 & Nhân viên (Chung) & Bất kỳ nhân viên nào cần xem lịch làm việc & Xem lịch làm việc cá nhân được phân công qua Employee Portal, có thể có quyền yêu cầu đổi ca hoặc báo nghỉ. \\
\hline
US-08 & Khách hàng & Customer/Guest & Tương tác qua giao diện web/app để đặt bàn, đặt món ăn trước, thanh toán đặt cọc, đặt hàng mang về hoặc giao hàng. \\
\hline
US-09 & Nhân viên hỗ trợ/ Vận hành & Support/Operations Staff & Tiếp nhận và xử lý các yêu cầu hỗ trợ từ khách hàng. \\
\hline
US-10 & Quản trị viên Hệ thống & System Administrator (Có thể là Quản lý nhà hàng hoặc IT) & Thực hiện các cấu hình kỹ thuật sâu, quản lý tích hợp (Shipday, Bot), quản lý tài khoản người dùng và phân quyền chi tiết. \\
\hline
\caption{Danh sách Người dùng Hệ thống} \label{tab:users} \\

\end{longtable}

\newpage % Ngắt trang



\begin{longtable}{|m{2.5cm}|m{2.5cm}|m{5cm}|m{5cm}|}
\hline
\textbf{Người dùng} & \textbf{Mã chức năng} & \textbf{Tên chức năng} & \textbf{Mô tả ngắn} \\
\hline
\endhead % Header cho các trang tiếp theo

\midrule
\endfoot % Footer cho bảng

\bottomrule
\endlastfoot % Footer cho trang cuối cùng

% === US-01: Quản lý nhà hàng ===
\multicolumn{4}{|l|}{\textbf{US-01: Quản lý nhà hàng}} \\ \hline
\multirow{15}{=}[2pt]{US-01: Quản lý nhà hàng} & FR-MD01-01 & Tạo ca làm việc & Cho phép định nghĩa các ca làm việc mới (thời gian, vai trò, số lượng). \\
& FR-MD01-02 & Gán nhân viên vào ca & Chỉ định nhân viên cụ thể cho các vị trí trống trong ca làm việc. \\
& FR-MD01-03 & Xem lịch biểu Gantt & Hiển thị lịch làm việc dạng Gantt, lọc theo vai trò, xem theo ngày/tuần/tháng. \\
& FR-MD01-04 & (Kích hoạt) Phát hiện trùng lịch & Kích hoạt hệ thống kiểm tra trùng lịch khi gán nhân viên. \\
& FR-MD01-05 & Xuất bản và Thông báo lịch & Công khai lịch làm việc và kích hoạt gửi thông báo đến nhân viên. \\
& FR-MD01-07 & Sao chép lịch tuần & Sao chép nhanh lịch làm việc của một tuần sang tuần khác. \\
& FR-MD01-08 & Quản lý vai trò công việc & Định nghĩa các vai trò công việc trong nhà hàng. \\
& FR-MD01-10 & Xem lịch theo vai trò & Lọc và xem lịch làm việc của các nhân viên theo vai trò cụ thể. \\ \cline{2-4}
& FR-MD02-01 & Tạo Sản phẩm Mới (Món ăn/Đồ uống) & Thêm món ăn, đồ uống mới vào hệ thống (tên, giá, loại...). \\
& FR-MD02-02 & Chỉnh sửa Thông tin Sản phẩm & Cập nhật chi tiết sản phẩm đã có (giá, mô tả, ảnh...). \\
& FR-MD02-03 & Lưu trữ/Hủy kích hoạt Sản phẩm & Ẩn/Hiện sản phẩm khỏi các giao dịch mà không xóa hẳn. \\
& FR-MD02-04 & Quản lý Danh mục Sản phẩm POS & Tạo, sửa, xóa, sắp xếp các danh mục hiển thị trên POS. \\
& FR-MD02-05 & Định nghĩa Thuộc tính \& Giá trị (cho Biến thể) & Định nghĩa các đặc tính (size, độ cay) và lựa chọn (S, M, L). (Có thể là US-10) \\
& FR-MD02-06 & Cấu hình Biến thể Sản phẩm & Áp dụng thuộc tính/giá trị vào sản phẩm gốc để tạo biến thể, cấu hình giá riêng. \\
& FR-MD02-07 & Thiết lập Loại Sản phẩm & Xác định loại sản phẩm (Consumable, Stockable, Service). \\
& FR-MD02-08 & Cấu hình Hiển thị trên POS & Đánh dấu sản phẩm bán trên POS và gán vào danh mục POS. \\
& FR-MD02-09 & Quản lý Hình ảnh Sản phẩm & Tải lên, thay thế, xóa hình ảnh đại diện cho sản phẩm. \\
& FR-MD02-10 & Cấu hình In Bếp/Hiển thị KDS & Chỉ định danh mục sản phẩm nào gửi đến máy in/KDS nào. \\
& FR-MD02-11 & Định nghĩa Sản phẩm Tùy chọn/Phụ thu & Tạo các sản phẩm nhỏ (Extra cheese) để dùng làm modifier trên POS. \\ \cline{2-4}
& FR-MD03-11 & Cấu hình Tham số Đặt chỗ & Thiết lập giờ hoạt động, giới hạn khách, quy tắc đặt cọc, giá bàn... \\
& FR-MD03-12 & Xem Danh sách Đặt chỗ & Xem danh sách tổng hợp các lượt đặt chỗ và trạng thái. \\
& FR-MD03-13 & Xem Chi tiết Đặt chỗ & Xem thông tin chi tiết của một lượt đặt chỗ cụ thể. \\
& FR-MD03-14 & Tạo/Sửa Đặt chỗ Thủ công & Tạo hoặc sửa đặt chỗ cho khách qua kênh offline (điện thoại...). \\
& FR-MD03-15 & Quản lý Trạng thái Đặt chỗ & Thay đổi trạng thái đặt chỗ (Xác nhận, Đã đến, Hủy...). \\
& FR-MD03-16 & Xem Danh sách Món đặt trước & Xem các món cần chuẩn bị cho các đặt chỗ sắp tới. \\ \cline{2-4}
& FR-MD04-05 & Cấu hình Dịch vụ Bot Call & Cấu hình tham số tích hợp Bot Call (số ngày gọi, kịch bản, số hỗ trợ...). (Có thể là US-10) \\ \cline{2-4}
& FR-MD05-01 & Mở phiên làm việc POS & Bắt đầu phiên làm việc POS, nhập tiền mặt đầu ca. (Thường là US-05) \\
& FR-MD05-13 & Đóng Phiên làm việc POS & Kết thúc phiên làm việc, tổng kết, đối chiếu tiền mặt. (Thường là US-05) \\
& FR-MD05-14 & Chuyển bàn/Ghép bàn & Di chuyển hoặc gộp đơn hàng giữa các bàn. \\
& FR-MD05-15 & Hủy món/Hủy đơn (Void) & Hủy bỏ món hoặc đơn hàng (có thể cần quyền quản lý). \\ \cline{2-4}
& FR-MD09-03 & Xem Báo cáo Doanh thu Phiên POS & Xem báo cáo tổng kết chi tiết của các phiên POS đã đóng. \\
& FR-MD09-04 & Xem Báo cáo Bán hàng theo Sản phẩm/Danh mục & Xem thống kê số lượng, doanh thu theo từng món ăn/danh mục. \\
& FR-MD09-05 & Xem Báo cáo Hiệu suất Nhân viên (POS) & Xem doanh thu, số đơn hàng theo từng nhân viên POS. \\
& FR-MD09-06 & Xem Báo cáo Tiền đặt cọc & Xem báo cáo tổng hợp tình hình thu, sử dụng, mất cọc. \\
& FR-MD09-07 & Xem Báo cáo Doanh thu theo Loại hình & Phân tích doanh thu theo Eat-in, Takeout, Delivery. \\
& FR-MD09-08 & Xuất dữ liệu Báo cáo & Xuất dữ liệu báo cáo ra file Excel/CSV. \\ \cline{2-4}
& FR-MD10-04 & Cấu hình Chung của Hệ thống & Cấu hình thông tin công ty, logo, tiền tệ, email... (Có thể là US-10) \\
& FR-MD10-05 & Cấu hình Tích hợp Bên thứ ba & Quản lý API keys cho Cổng thanh toán, Bot Call, Shipday. (Có thể là US-10) \\
& FR-MD10-06 & Cấu hình Tham số Nghiệp vụ Đặc thù & Cấu hình tỷ lệ cọc, giá bàn, số ngày gọi bot... (Có thể là US-10) \\
\hline

% === US-02: Nhân viên phục vụ ===
\multicolumn{4}{|l|}{\textbf{US-02: Nhân viên phục vụ}} \\ \hline
\multirow{12}{=}[2pt]{US-02: Nhân viên phục vụ} & FR-MD05-02 & Truy cập Sơ đồ tầng \& Chọn bàn & Xem trạng thái bàn và chọn bàn để phục vụ. \\
& FR-MD05-03 & Bắt đầu/Mở đơn hàng tại bàn & Mở giao diện đơn hàng cho bàn đã chọn. \\
& FR-MD05-04 & Tải và Xác nhận Món ăn Đặt trước & Xem và xác nhận các món khách đã đặt trước online. \\
& FR-MD05-05 & Thêm món ăn/đồ uống vào đơn hàng & Nhận order tại bàn và thêm món vào POS. \\
& FR-MD05-06 & Xử lý Yêu cầu đặc biệt/Ghi chú bếp & Thêm ghi chú (ít cay, dị ứng...) vào món ăn/đơn hàng. \\
& FR-MD05-07 & Gửi đơn hàng xuống Bếp/Bar & Gửi thông tin món cần chuẩn bị đến bếp/bar. \\
& FR-MD05-08 & Yêu cầu/In Hóa đơn Tạm tính & In bill tạm tính cho khách kiểm tra. \\
& FR-MD05-09 & (Kích hoạt) Áp dụng Tiền Đặt cọc vào Hóa đơn & Kích hoạt việc trừ cọc khi vào màn hình thanh toán. \\
& FR-MD05-10 & Tách hóa đơn (Split Bill) & Chia hóa đơn cho khách thanh toán riêng. \\
& FR-MD05-11 & Xử lý Thanh toán & Nhận thanh toán từ khách (tiền mặt, thẻ...), xử lý tiền boa. \\
& FR-MD05-12 & Đóng Đơn hàng và Bàn & Đóng đơn hàng và giải phóng bàn sau khi khách thanh toán. \\
& FR-MD05-14 & Chuyển bàn/Ghép bàn & Di chuyển hoặc gộp đơn hàng giữa các bàn. \\
& FR-MD05-15 & Hủy món/Hủy đơn (Void) & Hủy món/đơn (có thể cần quyền hoặc xác nhận của quản lý). \\ \cline{2-4}
& FR-MD06-01 & Chọn Chế độ Bán Mang về & Chuyển sang giao diện bán mang về trên POS. \\
& FR-MD06-02 & Tạo Đơn hàng Mang về & Khởi tạo đơn hàng mới cho khách mang về. \\
& FR-MD06-03 & (Tùy chọn) Liên kết Khách hàng & Gán đơn hàng mang về với khách hàng (nếu cần). \\
& FR-MD06-04 & Thêm món vào Đơn hàng Mang về & Thêm món vào đơn hàng mang về. \\
& FR-MD06-05 & Xử lý Ghi chú cho Đơn Mang về & Thêm ghi chú cho đơn hàng mang về. \\
& FR-MD06-06 & Gửi đơn Mang về xuống Bếp/Bar & Gửi món của đơn mang về xuống bếp/bar. \\
& FR-MD06-07 & (Kích hoạt) Áp dụng Đặt cọc (Nếu Đặt trước Online) & Kích hoạt trừ cọc cho đơn takeout đặt online. \\
& FR-MD06-08 & Thanh toán Đơn hàng Mang về & Nhận thanh toán cho đơn hàng mang về tại quầy. \\
& FR-MD06-09 & (Nhận) In Hóa đơn/Phiếu thu Mang về & Nhận hóa đơn in ra để đưa khách. \\
& FR-MD06-10 & Đóng Đơn hàng Mang về & Đóng đơn hàng mang về sau khi thanh toán/giao hàng. \\ \cline{2-4}
& FR-MD07-01 & Chọn Chế độ Giao hàng & Chuyển sang giao diện xử lý đơn giao hàng. \\
& FR-MD07-02 & Tạo/Mở Đơn hàng Giao hàng & Tạo/mở đơn hàng giao đi. \\
& FR-MD07-03 & Liên kết/Nhập Thông tin Khách hàng Giao hàng & Nhập/chọn thông tin khách và địa chỉ giao hàng. \\
& FR-MD07-04 & Thêm món vào Đơn hàng Giao hàng & Thêm món vào đơn hàng giao đi. \\
& FR-MD07-05 & Xử lý Ghi chú cho Đơn Giao hàng & Thêm ghi chú cho món hoặc cho tài xế. \\
& FR-MD07-06 & Gửi đơn Giao hàng xuống Bếp/Bar & Gửi món của đơn giao hàng xuống bếp/bar. \\
& FR-MD07-07 & (Kích hoạt) Áp dụng Đặt cọc/Thanh toán Trước & Kích hoạt trừ cọc/thanh toán trước cho đơn giao hàng. \\
& FR-MD07-08 & Xác nhận và Gửi Đơn hàng sang Shipday & Gửi thông tin đơn hàng qua Shipday để điều phối giao hàng. \\
& FR-MD07-10 & Xử lý Thanh toán Đơn hàng Giao hàng (Nếu COD) & Ghi nhận tiền COD tài xế nộp lại. \\
& FR-MD07-11 & In Hóa đơn/Phiếu Giao hàng & In phiếu giao hàng/hóa đơn cho tài xế và khách. \\
& FR-MD07-12 & Đóng Đơn hàng Giao hàng & Đóng đơn hàng giao đi sau khi hoàn tất. \\
\hline

% === US-03: Nhân viên lễ tân ===
\multicolumn{4}{|l|}{\textbf{US-03: Nhân viên lễ tân}} \\ \hline
\multirow{5}{=}[2pt]{US-03: Nhân viên lễ tân} & FR-MD03-12 & Xem Danh sách Đặt chỗ & Xem danh sách tổng hợp các lượt đặt chỗ và trạng thái. \\
& FR-MD03-13 & Xem Chi tiết Đặt chỗ & Xem thông tin chi tiết của một lượt đặt chỗ cụ thể. \\
& FR-MD03-14 & Tạo/Sửa Đặt chỗ Thủ công & Tạo hoặc sửa đặt chỗ cho khách qua kênh offline (điện thoại...). \\
& FR-MD03-15 & Quản lý Trạng thái Đặt chỗ & Thay đổi trạng thái đặt chỗ (Xác nhận, Đã đến, Hủy...). \\ \cline{2-4}
& FR-MD05-02 & Truy cập Sơ đồ tầng \& Chọn bàn & Xem trạng thái bàn và xếp khách vào bàn (có thể dùng POS). \\
\hline

% === US-04: Nhân viên bếp ===
\multicolumn{4}{|l|}{\textbf{US-04: Nhân viên bếp}} \\ \hline
\multirow{5}{=}[2pt]{US-04: Nhân viên bếp} & FR-MD03-16 & Xem Danh sách Món đặt trước & Xem các món cần chuẩn bị cho các đặt chỗ sắp tới. \\ \cline{2-4}
& FR-MD08-02 & Xem Đơn hàng trên KDS & Xem các đơn hàng/phiếu chờ xử lý trên màn hình KDS. \\
& FR-MD08-03 & Thay đổi Trạng thái Món ăn/Đơn hàng trên KDS & Đánh dấu món/đơn hàng đang làm, đã xong trên KDS. \\
& FR-MD08-04 & Xem Chi tiết Món ăn trên KDS & Xem chi tiết món ăn, biến thể, ghi chú trên KDS. \\
& FR-MD08-05 & (Tùy chọn) Sắp xếp/Ưu tiên Đơn hàng trên KDS & Sắp xếp hoặc đánh dấu ưu tiên các đơn hàng trên KDS. \\
& FR-MD08-07 & Nhận và Xử lý Phiếu in Bếp & Nhận phiếu in từ máy in và thực hiện chế biến (thủ công). \\
\hline

% === US-05: Nhân viên thu ngân ===
\multicolumn{4}{|l|}{\textbf{US-05: Nhân viên thu ngân}} \\ \hline
\multirow{11}{=}[2pt]{US-05: Nhân viên thu ngân} & FR-MD05-01 & Mở phiên làm việc POS & Bắt đầu phiên làm việc POS, nhập tiền mặt đầu ca. \\
& FR-MD05-11 & Xử lý Thanh toán & Nhận thanh toán từ khách (tiền mặt, thẻ...), xử lý tiền boa. (Có thể là US-02) \\
& FR-MD05-13 & Đóng Phiên làm việc POS & Kết thúc phiên làm việc, tổng kết, đối chiếu tiền mặt. \\ \cline{2-4}
& FR-MD06-01 & Chọn Chế độ Bán Mang về & Chuyển sang giao diện bán mang về trên POS. \\
& FR-MD06-02 & Tạo Đơn hàng Mang về & Khởi tạo đơn hàng mới cho khách mang về. \\
& FR-MD06-03 & (Tùy chọn) Liên kết Khách hàng & Gán đơn hàng mang về với khách hàng (nếu cần). \\
& FR-MD06-04 & Thêm món vào Đơn hàng Mang về & Thêm món vào đơn hàng mang về. \\
& FR-MD06-05 & Xử lý Ghi chú cho Đơn Mang về & Thêm ghi chú cho đơn hàng mang về. \\
& FR-MD06-06 & Gửi đơn Mang về xuống Bếp/Bar & Gửi món của đơn mang về xuống bếp/bar. \\
& FR-MD06-08 & Thanh toán Đơn hàng Mang về & Nhận thanh toán cho đơn hàng mang về tại quầy. \\
& FR-MD06-09 & (Nhận) In Hóa đơn/Phiếu thu Mang về & Nhận hóa đơn in ra để đưa khách. \\
& FR-MD06-10 & Đóng Đơn hàng Mang về & Đóng đơn hàng mang về sau khi thanh toán/giao hàng. \\ \cline{2-4}
& FR-MD07-01 & Chọn Chế độ Giao hàng & Chuyển sang giao diện xử lý đơn giao hàng. \\
& FR-MD07-02 & Tạo/Mở Đơn hàng Giao hàng & Tạo/mở đơn hàng giao đi. \\
& FR-MD07-03 & Liên kết/Nhập Thông tin Khách hàng Giao hàng & Nhập/chọn thông tin khách và địa chỉ giao hàng. \\
& FR-MD07-04 & Thêm món vào Đơn hàng Giao hàng & Thêm món vào đơn hàng giao đi. \\
& FR-MD07-05 & Xử lý Ghi chú cho Đơn Giao hàng & Thêm ghi chú cho món hoặc cho tài xế. \\
& FR-MD07-06 & Gửi đơn Giao hàng xuống Bếp/Bar & Gửi món của đơn giao hàng xuống bếp/bar. \\
& FR-MD07-08 & Xác nhận và Gửi Đơn hàng sang Shipday & Gửi thông tin đơn hàng qua Shipday để điều phối giao hàng. \\
& FR-MD07-10 & Xử lý Thanh toán Đơn hàng Giao hàng (Nếu COD) & Ghi nhận tiền COD tài xế nộp lại. \\
& FR-MD07-11 & In Hóa đơn/Phiếu Giao hàng & In phiếu giao hàng/hóa đơn cho tài xế và khách. \\
& FR-MD07-12 & Đóng Đơn hàng Giao hàng & Đóng đơn hàng giao đi sau khi hoàn tất. \\
\hline

% === US-06: Kế toán ===
\multicolumn{4}{|l|}{\textbf{US-06: Kế toán}} \\ \hline
\multirow{5}{=}[2pt]{US-06: Kế toán} & FR-MD09-03 & Xem Báo cáo Doanh thu Phiên POS & Xem báo cáo tổng kết chi tiết của các phiên POS đã đóng. \\
& FR-MD09-04 & Xem Báo cáo Bán hàng theo Sản phẩm/Danh mục & Xem thống kê số lượng, doanh thu theo từng món ăn/danh mục. \\
& FR-MD09-06 & Xem Báo cáo Tiền đặt cọc & Xem báo cáo tổng hợp tình hình thu, sử dụng, mất cọc. \\
& FR-MD09-07 & Xem Báo cáo Doanh thu theo Loại hình & Phân tích doanh thu theo Eat-in, Takeout, Delivery. \\
& FR-MD09-08 & Xuất dữ liệu Báo cáo & Xuất dữ liệu báo cáo ra file Excel/CSV. \\
\hline

% === US-07: Nhân viên (Chung) ===
\multicolumn{4}{|l|}{\textbf{US-07: Nhân viên (Chung)}} \\ \hline
\multirow{2}{=}[2pt]{US-07: Nhân viên (Chung)} & FR-MD01-06 & Xem lịch cá nhân & Xem lịch làm việc đã được xuất bản của bản thân. \\
& FR-MD01-09 & Đánh dấu không sẵn sàng & Thông báo cho quản lý về thời gian không thể làm việc. \\
\hline

% === US-08: Khách hàng ===
\multicolumn{4}{|l|}{\textbf{US-08: Khách hàng}} \\ \hline
\multirow{9}{=}[2pt]{US-08: Khách hàng} & FR-MD03-01 & Xem Giao diện Đặt chỗ & Truy cập và xem giao diện đặt chỗ online. \\
& FR-MD03-02 & Chọn Thông tin Đặt bàn & Chọn ngày, giờ, số lượng người đặt bàn. \\
& FR-MD03-03 & (Tùy chọn) Chọn Bàn cụ thể & Xem sơ đồ tầng và chọn bàn trống (nếu được phép). \\
& FR-MD03-04 & Xem Thực đơn & Xem thực đơn online để chọn món đặt trước. \\
& FR-MD03-05 & Chọn Món ăn Đặt trước & Thêm món ăn/đồ uống vào giỏ hàng đặt trước. \\
& FR-MD03-06 & Xem Tóm tắt Đặt chỗ & Xem lại thông tin đặt bàn, món ăn, tiền cọc dự kiến. \\
& FR-MD03-07 & Nhập Thông tin Khách hàng & Cung cấp Tên, SĐT, Email. \\
& FR-MD03-09 & Thanh toán Đặt cọc & Thực hiện thanh toán tiền đặt cọc qua cổng thanh toán. \\
& FR-MD03-10 & (Nhận) Xác nhận Đặt chỗ & Nhận email/SMS xác nhận sau khi thanh toán thành công. \\
& FR-MD03-17 & Xem Lịch sử/Chi tiết Đặt chỗ Cá nhân & Xem lại các đặt chỗ đã thực hiện qua tài khoản. \\ \cline{2-4}
& FR-MD04-02 & (Tương tác) Thực hiện Cuộc gọi và Tương tác Khách hàng & Nghe máy và bấm phím 1, 0, 2 khi nhận cuộc gọi từ Bot. \\
\hline

% === US-09: Nhân viên hỗ trợ/ Vận hành ===
\multicolumn{4}{|l|}{\textbf{US-09: Nhân viên hỗ trợ/ Vận hành}} \\ \hline
\multirow{1}{=}[2pt]{US-09: Nhân viên hỗ trợ/ Vận hành} & FR-MD04-03 & (Tiếp nhận) Xử lý Phản hồi Khách hàng từ Bot Call & Nhận cuộc gọi chuyển tiếp từ Bot khi khách bấm phím 2 để hỗ trợ. \\
&&&
\tabularnewline\hline
% === US-10: Quản trị viên Hệ thống ===
\multicolumn{4}{|l|}{\textbf{US-10: Quản trị viên Hệ thống}} \\ \hline
\multirow{7}{=}[2pt]{US-10: Quản trị viên Hệ thống} & FR-MD02-05 & Định nghĩa Thuộc tính \& Giá trị (cho Biến thể) & Định nghĩa các đặc tính (size, độ cay) và lựa chọn (S, M, L). (Có thể là US-01) \\ \cline{2-4}
& FR-MD04-05 & Cấu hình Dịch vụ Bot Call & Cấu hình tham số tích hợp Bot Call. (Có thể là US-01) \\ \cline{2-4}
& FR-MD07-13 & Cấu hình Tích hợp Shipday & Cấu hình tham số kết nối API Shipday. (Có thể là US-01) \\ \cline{2-4}
& FR-MD10-01 & Quản lý Người dùng (Nhân viên) & Tạo, sửa, vô hiệu hóa tài khoản người dùng nhân viên. \\
& FR-MD10-02 & Quản lý Nhóm Quyền & Xem, tạo, sửa, xóa các nhóm quyền truy cập. \\
& FR-MD10-03 & Phân quyền Truy cập cho Người dùng & Gán người dùng vào các nhóm quyền phù hợp. \\
& FR-MD10-04 & Cấu hình Chung của Hệ thống & Cấu hình thông tin công ty, logo, tiền tệ, email... (Có thể là US-01) \\
& FR-MD10-05 & Cấu hình Tích hợp Bên thứ ba & Quản lý API keys cho Cổng thanh toán, Bot Call, Shipday. (Có thể là US-01) \\
& FR-MD10-06 & Cấu hình Tham số Nghiệp vụ Đặc thù & Cấu hình tỷ lệ cọc, giá bàn, số ngày gọi bot... (Có thể là US-01) \\
& FR-MD10-07 & Xem Nhật ký Hệ thống (Logs) & Xem log hệ thống để theo dõi và khắc phục sự cố. \\
\hline
\caption{Phân công Chức năng theo Người dùng} \label{tab:user_function_map} \\

\end{longtable}

\newpage


\begin{longtable}{|m{1.5cm}|m{4.5cm}|m{9cm}|}
\hline
\textbf{Mã} & \textbf{Tên Module} & \textbf{Mô Tả} \\
\hline
\endhead % Header cho các trang tiếp theo

\hline
\endfoot % Footer cho bảng

\hline
\endlastfoot % Footer cho trang cuối cùng

MD-01 & Quản lý Lịch làm việc (Scheduling) & Bao gồm các chức năng: tạo ca làm việc, gán nhân viên theo vai trò, hiển thị dạng Gantt, kiểm tra trùng lịch, xuất bản và thông báo lịch trình cho nhân viên. \\
\hline
MD-02 & Quản lý Thực đơn \& Sản phẩm (Menu \& Product) & Quản lý danh sách món ăn, đồ uống dưới dạng sản phẩm. Bao gồm tạo mới, chỉnh sửa giá, mô tả, hình ảnh, phân loại theo danh mục POS, quản lý biến thể (ví dụ: size S/M/L), quản lý tồn kho cho các mặt hàng cụ thể (vd: rượu chai). \\
\hline
MD-03 & Quản lý Đặt chỗ \& Đặt món trước (Booking \& Pre-order) & Cung cấp giao diện cho khách hàng (web/app) để chọn thời gian, số lượng người, chọn bàn (nếu có), và tùy chọn đặt trước các món ăn từ thực đơn. Tính toán và xử lý thanh toán tiền đặt cọc (15\% giá trị bàn + 15\% giá trị món ăn đặt trước). Quản lý trạng thái các lượt đặt chỗ. \\
\hline
MD-04 & Xác nhận Tự động qua Bot (Automated Bot Confirmation) & Tự động kích hoạt cuộc gọi thoại tới khách hàng trước N ngày (cấu hình được) so với ngày đặt chỗ. Phát thông báo và xử lý lựa chọn của khách (1: Xác nhận, 0: Hủy - cập nhật trạng thái đặt chỗ và ghi nhận mất cọc, 2: Chuyển hướng cuộc gọi đến Nhân viên hỗ trợ). \\
\hline
MD-05 & Quản lý Bán hàng Tại chỗ (POS - Eat-in) & Giao diện POS cho nhân viên phục vụ, quản lý sơ đồ tầng, trạng thái bàn, nhận đơn hàng tại bàn (bao gồm cả món khách đã đặt trước), xử lý yêu cầu đặc biệt (ghi chú bếp), tách/gộp hóa đơn, áp dụng tiền đặt cọc đã thanh toán vào hóa đơn cuối cùng, xử lý thanh toán đa phương thức, quản lý tiền boa. \\
\hline
MD-06 & Quản lý Bán mang về (POS - Takeout) & Cung cấp một giao diện/luồng riêng biệt trên POS (ví dụ: nút "Takeout") để tạo đơn hàng mang về. Không yêu cầu chọn bàn. Có thể liên kết với khách hàng nếu họ đã đăng nhập hoặc cung cấp thông tin. Áp dụng tiền đặt cọc nếu đơn hàng được đặt trước qua kênh online. Xử lý thanh toán. \\
\hline
MD-07 & Quản lý Giao hàng (POS - Delivery) & Cung cấp một giao diện/luồng riêng biệt trên POS (ví dụ: nút "Delivery"). Yêu cầu thông tin khách hàng (đã đăng nhập từ web/app đặt hàng). Tích hợp với Shipday: gửi thông tin đơn hàng (địa chỉ, chi tiết món, thông tin khách) sang Shipday để điều phối tài xế, nhận lại cập nhật trạng thái giao hàng từ Shipday. Áp dụng tiền đặt cọc đã thanh toán. Xử lý thanh toán (có thể là online trước hoặc COD tùy cấu hình). \\
\hline
MD-08 & Tích hợp Bếp (Kitchen Integration) & Truyền thông tin đơn hàng (từ Eat-in, Takeout, Delivery - bao gồm món đặt trước và ghi chú) tới khu vực bếp thông qua Màn hình hiển thị Bếp (KDS) hoặc máy in phiếu bếp. Cho phép nhân viên bếp cập nhật trạng thái món ăn (đang làm, đã xong). \\
\hline
MD-09 & Quản lý Phiên \& Báo cáo (Session \& Reporting) & Quản lý việc mở và đóng phiên làm việc trên POS, đối soát tiền mặt. Cung cấp các báo cáo về doanh thu (theo loại hình: Eat-in, Takeout, Delivery), sản phẩm bán chạy, hiệu suất nhân viên, tiền đặt cọc, trạng thái đơn hàng giao, tích hợp dữ liệu. \\
\hline
MD-10 & Quản lý Hệ thống \& Người dùng (System \& User) & Quản lý tài khoản người dùng (tạo, sửa, xóa), phân quyền truy cập chi tiết cho từng vai trò vào các module và chức năng. Quản lý các cấu hình chung như tỷ lệ đặt cọc, giá trị bàn, thời gian gọi bot, cấu hình tích hợp bên thứ ba (API keys,...). \\
\hline
\caption{Phân chia Module Hệ thống} \label{tab:modules} \\

\end{longtable}

\subsubsection{Module MD-01: Quản lý Lịch làm việc (Scheduling)}


\begin{longtable}{|m{2cm}|m{2.5cm}|m{2cm}|m{4cm}|m{4.5cm}|}
\caption{Danh sách Yêu cầu Chức năng cho Module MD-01: Quản lý Lịch làm việc} \label{tab:fr_md01} \\
\hline
\textbf{Mã Module} & \textbf{Mã Yêu cầu CN} & \textbf{Mã Người dùng} & \textbf{Tên Chức năng} & \textbf{Mô tả Ngắn} \\
\hline
\endhead % Header cho các trang tiếp theo

\hline
\endfoot % Footer cho bảng

\hline
\endlastfoot % Footer cho trang cuối cùng

MD-01 & FR-MD01-01 & US-01 & Tạo ca làm việc & Cho phép Quản lý nhà hàng định nghĩa các ca làm việc mới (thời gian bắt đầu, kết thúc, ngày, vai trò cần thiết, số lượng nhân viên cần cho vai trò đó). \\
\hline
MD-01 & FR-MD01-02 & US-01 & Gán nhân viên vào ca & Cho phép Quản lý nhà hàng chỉ định nhân viên cụ thể cho các vị trí trống trong ca làm việc đã tạo, dựa trên vai trò phù hợp của nhân viên. \\
\hline
MD-01 & FR-MD01-03 & US-01 & Xem lịch biểu Gantt & Hiển thị lịch làm việc của tất cả nhân viên hoặc lọc theo vai trò dưới dạng biểu đồ Gantt trực quan theo dòng thời gian (ngày, tuần, tháng). \\
\hline
MD-01 & FR-MD01-04 & US-01 (Trigger), System (Action) & Phát hiện trùng lịch & Hệ thống tự động kiểm tra và đưa ra cảnh báo trực quan (ví dụ: đổi màu ca bị trùng) nếu một nhân viên được gán vào hai ca làm việc có thời gian trùng nhau. \\
\hline
MD-01 & FR-MD01-05 & US-01 & Xuất bản và Thông báo lịch & Cho phép Quản lý nhà hàng công khai (publish) lịch làm việc đã xếp và kích hoạt hệ thống tự động gửi thông báo (ví dụ: email, thông báo trong ứng dụng) đến từng nhân viên về lịch trình cá nhân của họ. \\
\hline
MD-01 & FR-MD01-06 & US-07 & Xem lịch cá nhân & Cho phép Nhân viên xem lịch làm việc đã được xuất bản của riêng mình thông qua cổng thông tin nhân viên hoặc ứng dụng di động. \\
\hline
MD-01 & FR-MD01-07 & US-01 & Sao chép lịch tuần & Cho phép Quản lý nhà hàng sao chép nhanh lịch làm việc của một tuần (hoặc khoảng thời gian tùy chọn) sang tuần kế tiếp để tiết kiệm thời gian lập lịch. \\
\hline
MD-01 & FR-MD01-08 & US-01 & Quản lý vai trò công việc & Cho phép Quản lý nhà hàng định nghĩa các vai trò công việc trong nhà hàng (ví dụ: Bếp trưởng, Phục vụ, Pha chế, Lễ tân) để sử dụng khi tạo ca và gán nhân viên. \\
\hline
MD-01 & FR-MD01-09 & US-07 & Đánh dấu không sẵn sàng & Cho phép Nhân viên (nếu được cấu hình) đánh dấu các khoảng thời gian không sẵn sàng làm việc (ví dụ: nghỉ phép, bận việc riêng) để Quản lý nhà hàng xem xét khi xếp lịch. \\ % Thêm mới dựa trên case study
\hline
MD-01 & FR-MD01-10 & US-01 & Xem lịch theo vai trò & Cho phép Quản lý nhà hàng lọc và xem lịch làm việc chỉ của các nhân viên thuộc một vai trò cụ thể (ví dụ: xem tất cả ca của Bếp trưởng). \\ % Làm rõ FR-MD01-03
\hline

\end{longtable}


\begin{figure}[H]
	\centering
	\includegraphics[width=15cm]{Sections/tong_quan/functional_spec/img/ucd01.png}

     \vspace{0.5cm}
    \caption{Use case diagram cho Module MD-01: Quản lý Lịch làm việc (Scheduling)}
\end{figure}

\subsubsubsection{FR-MD01-01: Tạo ca làm việc mới}

\begin{figure}[H]
	\centering
	\includegraphics[width=15cm]{Sections/tong_quan/functional_spec/img/Screenshot 2025-04-30 at 20.11.53.png}

     \vspace{0.5cm}
    \caption{Quy trình Tạo ca làm việc mới (UC-MD-1-01)}
\end{figure}

\subsubsubsection{FR-MD01-02: Gán nhân viên vào ca làm việc}
\begin{figure}[H]
	\centering
	\includegraphics[width=15cm]{Sections/tong_quan/functional_spec/img/1.2.png}

     \vspace{0.5cm}
    \caption{Quy trình Gán nhân viên vào ca làm việc (UC-MD-1-02)}
\end{figure}


\subsubsubsection{FR-MD01-03: Xem lịch biểu Gantt}

\begin{figure}[H]
	\centering
	\includegraphics[width=15cm]{Sections/tong_quan/functional_spec/img/1.3.png}

     \vspace{0.5cm}
    \caption{Quy trình Xem lịch biểu Gantt (UC-MD-1-03)}
\end{figure}

\subsubsubsection{FR-MD01-05: Phát hiện và Cảnh báo Trùng lịch}

\begin{figure}[H]
	\centering
	\includegraphics[width=15cm]{Sections/tong_quan/functional_spec/img/1.4.png}

     \vspace{0.5cm}
    \caption{Quy trình  Phát hiện và Cảnh báo Trùng lịch (UC-MD-1-04)}
\end{figure}

\subsubsubsection{FR-MD01-05: Xuất bản và Thông báo Lịch làm việc}


\subsubsubsection{FR-MD01-06: Xem lịch làm việc cá nhân}

\begin{figure}[H]
	\centering
	\includegraphics[width=15cm]{Sections/tong_quan/functional_spec/img/1.6.png}

     \vspace{0.5cm}
    \caption{Quy trình Xem lịch làm việc cá nhân (UC-MD-1-06)}
\end{figure}

\subsubsubsection{FR-MD01-07: Sao chép lịch tuần}

\begin{figure}[H]
	\centering
	\includegraphics[width=15cm]{Sections/tong_quan/functional_spec/img/1.7.png}

     \vspace{0.5cm}
    \caption{Quy trình Sao chép lịch tuần (UC-MD-1-07)}
\end{figure}

\subsubsubsection{FR-MD01-08: Quản lý vai trò công việc}

\begin{figure}[H]
	\centering
	\includegraphics[width=15cm]{Sections/tong_quan/functional_spec/img/1.8.1.png}

     \vspace{0.5cm}
    \caption{Quy trình Tạo mới Vai trò công việc (UC-MD-1-08 - Create)}
\end{figure}
\begin{figure}[H]
	\centering
	\includegraphics[width=15cm]{Sections/tong_quan/functional_spec/img/1.8.2.png}

     \vspace{0.5cm}
    \caption{Quy trình Xóa Vai trò công việc (UC-MD-1-08 - Delete)}
\end{figure}
\begin{figure}[H]
	\centering
	\includegraphics[width=15cm]{Sections/tong_quan/functional_spec/img/1.8.3.png}

     \vspace{0.5cm}
    \caption{Quy trình Sửa Vai trò công việc (UC-MD-1-08 - Edit)}
\end{figure}

\subsubsubsection{FR-MD01-09: Đánh dấu không sẵn sàng làm việc}

\begin{figure}[H]
	\centering
	\includegraphics[width=15cm]{Sections/tong_quan/functional_spec/img/1.9.png}

     \vspace{0.5cm}
    \caption{Quy trình Đánh dấu không sẵn sàng làm việc (UC-MD-1-09)}
\end{figure}

\subsubsubsection{Use Case UC-MD01-10: Xem lịch theo vai trò}

\subsubsubsection{MVP (Minimum viable product) và Screen Flow}

\begin{figure}[H]
	\centering
	\includegraphics[width=15cm]{Sections/tong_quan/functional_spec/img/proto1.1.png}

     \vspace{0.5cm}
    \caption{Trang Dashboard (1)}
\end{figure}
\begin{figure}[H]
	\centering
	\includegraphics[width=15cm]{Sections/tong_quan/functional_spec/img/proto1.2.png}

     \vspace{0.5cm}
    \caption{Trang Dashboard (2)}
\end{figure}
\begin{figure}[H]
	\centering
	\includegraphics[width=15cm]{Sections/tong_quan/functional_spec/img/proto1.3.png}

     \vspace{0.5cm}
    \caption{Trang Dashboard (3)}
\end{figure}
\begin{figure}[H]
	\centering
	\includegraphics[width=15cm]{Sections/tong_quan/functional_spec/img/proto1.4.png}

     \vspace{0.5cm}
    \caption{Trang Dashboard (4)}
\end{figure}
\begin{figure}[H]
	\centering
	\includegraphics[width=15cm]{Sections/tong_quan/functional_spec/img/proto1.5.png}

     \vspace{0.5cm}
    \caption{Trang Dashboard (5)}
\end{figure}
\begin{figure}[H]
	\centering
	\includegraphics[width=15cm]{Sections/tong_quan/functional_spec/img/proto1.6.png}

     \vspace{0.5cm}
    \caption{Trang Staff Management}
\end{figure}
\begin{figure}[H]
	\centering
	\includegraphics[width=15cm]{Sections/tong_quan/functional_spec/img/proto1.7.png}

     \vspace{0.5cm}
    \caption{Trang Shift Management}
\end{figure}
\begin{figure}[H]
	\centering
	\includegraphics[width=15cm]{Sections/tong_quan/functional_spec/img/proto1.8.png}

     \vspace{0.5cm}
    \caption{Trang Settings (1)}
\end{figure}
\begin{figure}[H]
	\centering
	\includegraphics[width=15cm]{Sections/tong_quan/functional_spec/img/proto1.9.png}

     \vspace{0.5cm}
    \caption{Trang Settings (2)}
\end{figure}
\begin{figure}[H]
	\centering
	\includegraphics[width=15cm]{Sections/tong_quan/functional_spec/img/proto1.10.png}

     \vspace{0.5cm}
    \caption{Trang Settings (3)}
\end{figure}

\textbf{User Flow:}

\begin{figure}[H]
	\centering
	\includegraphics[width=15cm]{Sections/tong_quan/functional_spec/img/dashboard1.png}

     \vspace{0.5cm}
    \caption{User flow cho trang Dashboard}
\end{figure}
\begin{figure}[H]
	\centering
	\includegraphics[width=15cm]{Sections/tong_quan/functional_spec/img/staffmanage1.png}

     \vspace{0.5cm}
    \caption{User flow cho trang Staff Management}
\end{figure}
\begin{figure}[H]
	\centering
	\includegraphics[width=15cm]{Sections/tong_quan/functional_spec/img/shiftmanage1.png}

     \vspace{0.5cm}
    \caption{User flow cho trang Staff Management}
\end{figure}
\begin{figure}[H]
	\centering
	\includegraphics[width=15cm]{Sections/tong_quan/functional_spec/img/setting1.png}

     \vspace{0.5cm}
    \caption{User flow cho trang Settings}
\end{figure}
\subsubsection{Module MD-02: Quản lý Thực đơn \& Sản phẩm}
\begin{longtable}{|m{2cm}|m{2.5cm}|m{2cm}|m{4cm}|m{4.5cm}|}
\caption{Danh sách Yêu cầu Chức năng cho Module MD-02: Quản lý Thực đơn \& Sản phẩm} \label{tab:fr_md02} \\
\hline
\textbf{Mã Module} & \textbf{Mã Yêu cầu CN} & \textbf{Mã Người dùng} & \textbf{Tên Chức năng} & \textbf{Mô tả Ngắn} \\
\hline
\endhead % Header cho các trang tiếp theo

\hline
\endfoot % Footer cho bảng

\hline
\endlastfoot % Footer cho trang cuối cùng

MD-02 & FR-MD02-01 & US-01 & Tạo Sản phẩm Mới (Món ăn/Đồ uống) & Cho phép Quản lý nhà hàng thêm một món ăn, đồ uống, hoặc dịch vụ mới vào hệ thống với các thông tin cơ bản (tên, giá, loại). \\
\hline
MD-02 & FR-MD02-02 & US-01 & Chỉnh sửa Thông tin Sản phẩm & Cho phép Quản lý nhà hàng cập nhật các chi tiết của một sản phẩm đã tồn tại (ví dụ: thay đổi giá bán, mô tả, cập nhật hình ảnh, gán lại danh mục). \\
\hline
MD-02 & FR-MD02-03 & US-01 & Lưu trữ/Hủy kích hoạt Sản phẩm & Cho phép Quản lý nhà hàng ẩn một sản phẩm khỏi các giao dịch (ví dụ: POS, đặt hàng online) mà không xóa hẳn dữ liệu lịch sử. \\
\hline
MD-02 & FR-MD02-04 & US-01 & Quản lý Danh mục Sản phẩm POS & Cho phép Quản lý nhà hàng tạo, sửa, xóa và sắp xếp thứ tự các danh mục được sử dụng để nhóm sản phẩm trên giao diện POS (ví dụ: Khai vị, Món chính, Tráng miệng, Đồ uống). \\
\hline
MD-02 & FR-MD02-05 & US-01 / US-10 & Định nghĩa Thuộc tính \& Giá trị (cho Biến thể) & Cho phép người dùng định nghĩa các thuộc tính (ví dụ: Kích cỡ, Độ cay, Loại topping) và các giá trị tương ứng cho từng thuộc tính (ví dụ: S, M, L; Ít cay, Cay vừa, Cay nhiều; Phô mai, Thịt nguội). \\
\hline
MD-02 & FR-MD02-06 & US-01 & Cấu hình Biến thể Sản phẩm & Cho phép Quản lý nhà hàng áp dụng các thuộc tính đã định nghĩa vào một sản phẩm gốc để tạo ra các biến thể (ví dụ: Cà phê size S, Cà phê size L), quản lý giá và mã SKU riêng cho từng biến thể nếu cần. \\
\hline
MD-02 & FR-MD02-07 & US-01 & Thiết lập Loại Sản phẩm & Cho phép Quản lý nhà hàng xác định loại sản phẩm (Consumable, Stockable, Service) để quyết định cách hệ thống quản lý tồn kho (nếu là Stockable) hoặc không theo dõi tồn kho. \\
\hline
MD-02 & FR-MD02-08 & US-01 & Cấu hình Hiển thị trên POS & Cho phép Quản lý nhà hàng đánh dấu sản phẩm có sẵn sàng để bán trên POS hay không và gán sản phẩm vào (các) danh mục POS phù hợp. \\
\hline
MD-02 & FR-MD02-09 & US-01 & Quản lý Hình ảnh Sản phẩm & Cho phép Quản lý nhà hàng tải lên và quản lý hình ảnh đại diện cho sản phẩm, hiển thị trên POS hoặc các kênh bán hàng khác. \\
\hline
MD-02 & FR-MD02-10 & US-01 & Cấu hình In Bếp/Hiển thị KDS & Cho phép Quản lý nhà hàng chỉ định danh mục sản phẩm nào sẽ được gửi đến máy in bếp hoặc màn hình KDS cụ thể khi đặt hàng qua POS. (Có thể liên quan cấu hình POS/IoT). \\
\hline
MD-02 & FR-MD02-11 & US-01 & Định nghĩa Sản phẩm Tùy chọn/Phụ thu & Cho phép Quản lý nhà hàng tạo các sản phẩm nhỏ (ví dụ: Extra cheese, Thêm sốt) để dùng làm tùy chọn có tính phí khi khách hàng yêu cầu thêm vào món chính (sử dụng trong cấu hình POS modifiers). \\
\hline

\end{longtable}

\subsubsubsection{FR-MD01-01: Tạo ca làm việc mới}

\begin{figure}[H]
	\centering
	\includegraphics[width=15cm]{Sections/tong_quan/functional_spec/img/2.1.png}

     \vspace{0.5cm}
    \caption{Quy trình Tạo ca làm việc mới (UC-MD-1-01)}
\end{figure}

\subsubsubsection{FR-MD01-01: Tạo ca làm việc mới}

\begin{figure}[H]
	\centering
	\includegraphics[width=15cm]{Sections/tong_quan/functional_spec/img/2.2.png}

     \vspace{0.5cm}
    \caption{Quy trình Tạo ca làm việc mới (UC-MD-1-01)}
\end{figure}
\subsubsubsection{FR-MD01-01: Tạo ca làm việc mới}

\begin{figure}[H]
	\centering
	\includegraphics[width=15cm]{Sections/tong_quan/functional_spec/img/2.3.1.png}

     \vspace{0.5cm}
    \caption{Quy trình Tạo ca làm việc mới (UC-MD-1-01)}
\end{figure}

\begin{figure}[H]
	\centering
	\includegraphics[width=15cm]{Sections/tong_quan/functional_spec/img/2.3.2.png}

     \vspace{0.5cm}
    \caption{Quy trình Tạo ca làm việc mới (UC-MD-1-01)}
\end{figure}
\subsubsubsection{FR-MD01-01: Tạo ca làm việc mới}

\begin{figure}[H]
	\centering
	\includegraphics[width=15cm]{Sections/tong_quan/functional_spec/img/2.4.1.png}

     \vspace{0.5cm}
    \caption{Quy trình Tạo ca làm việc mới (UC-MD-1-01)}
\end{figure}

\begin{figure}[H]
	\centering
	\includegraphics[width=15cm]{Sections/tong_quan/functional_spec/img/2.4.png}

     \vspace{0.5cm}
    \caption{Quy trình Tạo ca làm việc mới (UC-MD-1-01)}
\end{figure}
\subsubsubsection{FR-MD01-01: Tạo ca làm việc mới}

\begin{figure}[H]
	\centering
	\includegraphics[width=15cm]{Sections/tong_quan/functional_spec/img/2.5.png}

     \vspace{0.5cm}
    \caption{Quy trình Tạo ca làm việc mới (UC-MD-1-01)}
\end{figure}
\begin{figure}[H]
	\centering
	\includegraphics[width=15cm]{Sections/tong_quan/functional_spec/img/2.5.1.png}

     \vspace{0.5cm}
    \caption{Quy trình Tạo ca làm việc mới (UC-MD-1-01)}
\end{figure}
\begin{figure}[H]
	\centering
	\includegraphics[width=15cm]{Sections/tong_quan/functional_spec/img/2.5.2.png}

     \vspace{0.5cm}
    \caption{Quy trình Tạo ca làm việc mới (UC-MD-1-01)}
\end{figure}
\subsubsubsection{FR-MD01-01: Tạo ca làm việc mới}

\begin{figure}[H]
	\centering
	\includegraphics[width=15cm]{Sections/tong_quan/functional_spec/img/2.6.png}

     \vspace{0.5cm}
    \caption{Quy trình Tạo ca làm việc mới (UC-MD-1-01)}
\end{figure}
\subsubsubsection{FR-MD01-01: Tạo ca làm việc mới}


\subsubsubsection{FR-MD01-01: Tạo ca làm việc mới}

\begin{figure}[H]
	\centering
	\includegraphics[width=15cm]{Sections/tong_quan/functional_spec/img/2.8.png}

     \vspace{0.5cm}
    \caption{Quy trình Tạo ca làm việc mới (UC-MD-1-01)}
\end{figure}
\subsubsubsection{FR-MD01-01: Tạo ca làm việc mới}

\begin{figure}[H]
	\centering
	\includegraphics[width=15cm]{Sections/tong_quan/functional_spec/img/2.9.png}

     \vspace{0.5cm}
    \caption{Quy trình Tạo ca làm việc mới (UC-MD-1-01)}
\end{figure}
\subsubsubsection{FR-MD01-01: Tạo ca làm việc mới}

\begin{figure}[H]
	\centering
	\includegraphics[width=15cm]{Sections/tong_quan/functional_spec/img/2.10.png}

     \vspace{0.5cm}
    \caption{Quy trình Tạo ca làm việc mới (UC-MD-1-01)}
\end{figure}

\subsubsubsection{FR-MD01-01: Tạo ca làm việc mới}



\subsubsection{Module MD-03: Quản lý Đặt chỗ \& Đặt món trước}

\begin{longtable}{|m{2cm}|m{2.5cm}|m{2cm}|m{4.5cm}|m{4cm}|}
\caption{Danh sách Yêu cầu Chức năng cho Module MD-03: Quản lý Đặt chỗ \& Đặt món trước} \label{tab:fr_md03} \\
\hline
\textbf{Mã Module} & \textbf{Mã Yêu cầu CN} & \textbf{Mã Người dùng} & \textbf{Tên Chức năng} & \textbf{Mô tả Ngắn} \\
\hline
\endhead % Header cho các trang tiếp theo

\hline
\endfoot % Footer cho bảng

\hline
\endlastfoot % Footer cho trang cuối cùng

MD-03 & FR-MD03-01 & US-08 & Xem Giao diện Đặt chỗ & Hiển thị giao diện cho phép khách hàng xem các khung giờ/bàn còn trống và bắt đầu quá trình đặt chỗ. \\
\hline
MD-03 & FR-MD03-02 & US-08 & Chọn Thông tin Đặt bàn & Cho phép khách hàng chọn ngày, giờ, số lượng người cho lượt đặt bàn mong muốn. \\
\hline
MD-03 & FR-MD03-03 & US-08 & Chọn Bàn cụ thể & Nếu được cấu hình, cho phép khách hàng xem sơ đồ bàn và chọn một bàn cụ thể còn trống phù hợp với số lượng người. \\
\hline
MD-03 & FR-MD03-04 & US-08 & Xem Thực đơn & Hiển thị thực đơn (các sản phẩm từ MD-02) để khách hàng lựa chọn món ăn muốn đặt trước. \\
\hline
MD-03 & FR-MD03-05 & US-08 & Chọn Món ăn Đặt trước & Cho phép khách hàng thêm các món ăn/đồ uống từ thực đơn vào giỏ hàng đặt trước của lượt đặt chỗ. Bao gồm chọn biến thể (nếu có). \\
\hline
MD-03 & FR-MD03-06 & US-08 & Xem Tóm tắt Đặt chỗ & Hiển thị thông tin tổng hợp về lượt đặt chỗ: ngày giờ, số người, bàn (nếu chọn), danh sách món đặt trước, và số tiền đặt cọc dự kiến. \\
\hline
MD-03 & FR-MD03-07 & US-08 & Nhập Thông tin Khách hàng & Thu thập thông tin bắt buộc của khách hàng (Tên, Số điện thoại, Email) để hoàn tất đặt chỗ. \\
\hline
MD-03 & FR-MD03-08 & System / US-08 & Tính toán Tiền Đặt cọc & Hệ thống tự động tính toán số tiền đặt cọc dựa trên cấu hình: 15\% giá bàn + 15\% tổng giá trị món ăn đặt trước. \\
\hline
MD-03 & FR-MD03-09 & US-08 & Thanh toán Đặt cọc & Tích hợp với cổng thanh toán để khách hàng thực hiện thanh toán số tiền đặt cọc đã được tính toán. \\
\hline
MD-03 & FR-MD03-10 & System / US-08 & Xác nhận Đặt chỗ & Sau khi thanh toán thành công, hệ thống tạo bản ghi đặt chỗ, gửi email/SMS xác nhận cho khách hàng và cập nhật trạng thái bàn/lịch. \\
\hline
MD-03 & FR-MD03-11 & US-01 / US-10 & Cấu hình Tham số Đặt chỗ & Cho phép quản lý cấu hình các tùy chọn đặt chỗ: giờ hoạt động, thời gian mỗi lượt đặt, số lượng khách tối thiểu/tối đa, đặt trước bao lâu, giá trị từng bàn (cho việc tính cọc). \\
\hline
MD-03 & FR-MD03-12 & US-01 / US-03 & Xem Danh sách Đặt chỗ & Hiển thị danh sách tất cả các lượt đặt chỗ (online và offline) với các thông tin cơ bản và trạng thái (Chờ xác nhận, Đã xác nhận, Đã hủy...). \\
\hline
MD-03 & FR-MD03-13 & US-01 / US-03 & Xem Chi tiết Đặt chỗ & Cho phép xem thông tin chi tiết của một lượt đặt chỗ cụ thể: thông tin khách, bàn, món đặt trước, trạng thái thanh toán cọc, lịch sử thay đổi. \\
\hline
MD-03 & FR-MD03-14 & US-01 / US-03 & Tạo/Sửa Đặt chỗ Thủ công & Cho phép nhân viên (Lễ tân, Quản lý) tạo hoặc chỉnh sửa một lượt đặt chỗ trong hệ thống (ví dụ: cho khách gọi điện thoại), bao gồm cả việc nhập món đặt trước và ghi nhận đặt cọc (nếu có). \\
\hline
MD-03 & FR-MD03-15 & US-01 / US-03 & Quản lý Trạng thái Đặt chỗ & Cho phép nhân viên thay đổi trạng thái của một lượt đặt chỗ (ví dụ: Xác nhận thủ công, Đánh dấu đã đến, Hủy bỏ). \\
\hline
MD-03 & FR-MD03-16 & US-04 / US-01 & Xem Danh sách Món đặt trước & Cung cấp giao diện (có thể là báo cáo hoặc màn hình riêng) cho bộ phận bếp/quản lý xem trước các món ăn cần chuẩn bị cho các lượt đặt chỗ sắp tới. \\
\hline
MD-03 & FR-MD03-17 & US-08 & Xem Lịch sử/Chi tiết Đặt chỗ Cá nhân & Cho phép khách hàng đã đăng nhập xem lại các lượt đặt chỗ đã thực hiện và trạng thái của chúng. \\
\hline


\end{longtable}


\subsubsection{Module MD-04: Xác nhận Tự động qua Bot}

\begin{longtable}{|m{2cm}|m{2.5cm}|m{2.5cm}|m{4.5cm}|m{3.5cm}|}
\caption{Danh sách Yêu cầu Chức năng cho Module MD-04: Xác nhận Tự động qua Bot} \label{tab:fr_md04} \\
\hline
\textbf{Mã Module} & \textbf{Mã Yêu cầu CN} & \textbf{Mã Người dùng} & \textbf{Tên Chức năng} & \textbf{Mô tả Ngắn} \\
\hline
\endhead % Header cho các trang tiếp theo

\hline
\endfoot % Footer cho bảng

\hline
\endlastfoot % Footer cho trang cuối cùng

MD-04 & FR-MD04-01 & System & Lên lịch và Kích hoạt Cuộc gọi Xác nhận & Tự động xác định các đặt chỗ 'Đã xác nhận' sắp diễn ra và lên lịch kích hoạt cuộc gọi xác nhận N ngày trước ngày đặt (N cấu hình được). \\
\hline
MD-04 & FR-MD04-02 & System (Bot Service), US-08 (Tương tác) & Thực hiện Cuộc gọi và Tương tác Khách hàng & Tích hợp với dịch vụ Bot Call bên ngoài để thực hiện cuộc gọi đến SĐT khách hàng, phát thông điệp và nhận lựa chọn (phím 1, 0, 2). \\
\hline
MD-04 & FR-MD04-03 & System, US-09 (Tiếp nhận cuộc gọi hỗ trợ) & Xử lý Phản hồi Khách hàng từ Bot Call & Cập nhật trạng thái đặt chỗ và thực hiện hành động tương ứng (xác nhận lại, hủy bỏ & giải phóng bàn, chuyển cuộc gọi hỗ trợ) dựa trên phím khách hàng đã bấm. \\
\hline
MD-04 & FR-MD04-04 & System & Ghi nhận Kết quả Cuộc gọi & Lưu trữ lại kết quả của mỗi cuộc gọi Bot Call (thành công, thất bại, không liên lạc được, lựa chọn của khách) vào thông tin đặt chỗ hoặc nhật ký hệ thống. \\
\hline
MD-04 & FR-MD04-05 & US-01 / US-10 & Cấu hình Dịch vụ Bot Call & Cho phép cấu hình các tham số tích hợp Bot Call: số ngày N gọi trước, nội dung kịch bản thoại, số điện thoại chuyển tiếp hỗ trợ, API key/credentials của dịch vụ Bot Call. \\
\hline

\end{longtable}


\subsubsection{Module MD-05: Quản lý Bán hàng Tại chỗ (POS - Eat-in)}


\begin{longtable}{|m{2cm}|m{2.5cm}|m{2.5cm}|m{4.5cm}|m{4cm}|}
\caption{Danh sách Yêu cầu Chức năng cho Module MD-05: Quản lý Bán hàng Tại chỗ (POS - Eat-in)} \label{tab:fr_md05} \\
\hline
\textbf{Mã Module} & \textbf{Mã Yêu cầu CN} & \textbf{Mã Người dùng} & \textbf{Tên Chức năng} & \textbf{Mô tả Ngắn} \\
\hline
\endhead % Header cho các trang tiếp theo

\hline
\endfoot % Footer cho bảng

\hline
\endlastfoot % Footer cho trang cuối cùng

MD-05 & FR-MD05-01 & US-05, US-01 & Mở phiên làm việc POS & Cho phép nhân viên thu ngân/quản lý bắt đầu một phiên làm việc mới trên POS, nhập số tiền mặt ban đầu (nếu có kiểm soát tiền mặt). \\
\hline
MD-05 & FR-MD05-02 & US-02, US-03 & Truy cập Sơ đồ tầng \& Chọn bàn & Hiển thị sơ đồ tầng trực quan, cho phép nhân viên xem trạng thái bàn (trống, đang có khách, đã đặt trước) và chọn một bàn cụ thể. \\
\hline
MD-05 & FR-MD05-03 & US-02 & Bắt đầu/Mở đơn hàng tại bàn & Khởi tạo một đơn hàng mới hoặc mở lại đơn hàng đang hoạt động khi nhân viên chọn một bàn đang có khách hoặc xếp khách mới vào bàn trống. \\
\hline
MD-05 & FR-MD05-04 & US-02, System & Tải và Xác nhận Món ăn Đặt trước & Nếu bàn có liên kết với một đặt chỗ (từ MD-03) có món đặt trước, hệ thống tự động tải các món này vào đơn hàng POS để nhân viên xác nhận với khách và gửi bếp. \\
\hline
MD-05 & FR-MD05-05 & US-02 & Thêm món ăn/đồ uống vào đơn hàng & Cho phép nhân viên chọn các món ăn/đồ uống từ giao diện menu POS (theo danh mục - FR-MD02-04) và thêm vào đơn hàng hiện tại của bàn. \\
\hline
MD-05 & FR-MD05-06 & US-02 & Xử lý Yêu cầu đặc biệt/Ghi chú bếp & Cho phép nhân viên đính kèm ghi chú tùy chỉnh hoặc chọn các tùy chọn/ghi chú được định sẵn (liên quan FR-MD02-11) cho từng món ăn hoặc toàn bộ đơn hàng để gửi xuống bếp. \\
\hline
MD-05 & FR-MD05-07 & US-02 & Gửi đơn hàng xuống Bếp/Bar & Gửi thông tin các món ăn mới thêm (hoặc các món đặt trước đã xác nhận) đến máy in bếp/bar hoặc màn hình KDS tương ứng (theo cấu hình FR-MD02-10). \\
\hline
MD-05 & FR-MD05-08 & US-02 & Yêu cầu/In Hóa đơn Tạm tính & Cho phép nhân viên tạo và in ra hóa đơn tạm tính (pro-forma invoice / bill) cho khách hàng kiểm tra trước khi thanh toán. \\
\hline
MD-05 & FR-MD05-09 & US-02, System & Áp dụng Tiền Đặt cọc vào Hóa đơn & Trước khi tính tiền thanh toán cuối cùng, hệ thống tự động xác định và áp dụng (trừ đi) số tiền đặt cọc mà khách hàng đã thanh toán trước đó (từ MD-03) vào tổng hóa đơn. \\
\hline
MD-05 & FR-MD05-10 & US-02 & Tách hóa đơn (Split Bill) & Cung cấp chức năng tách hóa đơn của một bàn thành nhiều hóa đơn nhỏ hơn (theo người hoặc theo món ăn), có xem xét việc phân bổ tiền đặt cọc đã áp dụng. \\
\hline
MD-05 & FR-MD05-11 & US-02, US-05 & Xử lý Thanh toán & Cho phép nhận thanh toán từ khách hàng bằng nhiều phương thức (tiền mặt, thẻ, ví điện tử...), xử lý tiền boa, và ghi nhận giao dịch vào hệ thống sau khi đã trừ tiền đặt cọc. \\
\hline
MD-05 & FR-MD05-12 & US-02 & Đóng Đơn hàng và Bàn & Hoàn tất đơn hàng sau khi thanh toán thành công và cập nhật trạng thái bàn thành trống (sẵn sàng cho khách tiếp theo). \\
\hline
MD-05 & FR-MD05-13 & US-05, US-01 & Đóng Phiên làm việc POS & Kết thúc phiên làm việc POS, hệ thống tổng kết doanh thu theo từng phương thức thanh toán, đối chiếu tiền mặt và chuẩn bị dữ liệu cho bộ phận kế toán. \\
\hline
MD-05 & FR-MD05-14 & US-01, US-02 & Chuyển bàn/Ghép bàn & Cho phép nhân viên chuyển đơn hàng của khách từ bàn này sang bàn khác hoặc ghép nhiều bàn/đơn hàng lại với nhau. \\
\hline
MD-05 & FR-MD05-15 & US-01, US-02 & Hủy món/Hủy đơn (Void) & Cho phép nhân viên (có thể cần quyền quản lý) hủy bỏ một món đã gọi hoặc toàn bộ đơn hàng với lý do cụ thể, có ghi nhận vào hệ thống. \\
\hline

\end{longtable}

\subsubsection{Module MD-06: Quản lý Bán mang về (POS - Takeout)}

Module Quản lý Bán mang về (MD-06) là một phần mở rộng hoặc một chế độ hoạt động chuyên biệt của hệ thống Point of Sale (POS), được thiết kế để phục vụ nhu cầu của khách hàng mua đồ ăn, thức uống để mang đi (Takeout/Takeaway). Module này tập trung vào việc xử lý nhanh chóng các đơn hàng không yêu cầu quản lý bàn, từ việc tạo đơn, chọn món, cho đến thanh toán và hoàn tất giao dịch.

\begin{figure}[H]
    \centering
    \includegraphics[width=15cm]{Sections/tong_quan/functional_spec/img/uc6.png}
    \vspace{0.5cm}
    \caption{Use case diagram cho Module MD-06}
    \label{fig:my_label}
\end{figure}

\begin{longtable}{|m{2cm}|m{2.5cm}|m{2.5cm}|m{4.5cm}|m{4cm}|}
\caption{Danh sách Yêu cầu Chức năng cho Module MD-06: Quản lý Bán hàng Mang về (POS - Takeout)} \label{tab:fr_md06_revised_v3} \\
\hline
\textbf{Mã Module} & \textbf{Mã Yêu cầu CN} & \textbf{Mã Người dùng} & \textbf{Tên Chức năng} & \textbf{Mô tả Ngắn} \\
\hline
\endhead % Header cho các trang tiếp theo
\hline
\endfoot % Footer cho bảng
\hline
\endlastfoot % Footer cho trang cuối cùng

MD-06 & FR-MD06-01 & US-02, US-05 & Chọn Chế độ Bán Mang về & Nhân viên chọn chế độ/giao diện riêng trên POS cho đơn mang về. \\
\hline
MD-06 & FR-MD06-02 & US-02, US-05 & Tạo Đơn hàng Mang về Mới & Nhân viên khởi tạo một đơn hàng mới trong chế độ mang về. \\
\hline
MD-06 & FR-MD06-03 & US-02, US-05 & Gán Khách hàng vào Đơn Mang về & Nhân viên tìm và liên kết đơn mang về với khách hàng có sẵn hoặc tạo mới. \\
\hline
MD-06 & FR-MD06-04 & US-02, US-05 & Thêm Món vào Đơn hàng Mang về & Nhân viên thêm món ăn/đồ uống vào đơn hàng mang về. (Hành động tương tự FR-MD05-05, FR-MD05-06). \\
\hline
MD-06 & FR-MD06-05 & US-02, US-05 & Thêm Ghi chú cho Đơn Mang về & Nhân viên thêm ghi chú đặc biệt cho món hoặc cả đơn mang về. (Hành động tương tự FR-MD05-07). \\
\hline
MD-06 & FR-MD06-06 & US-02, US-05 & Gửi Yêu cầu Chuẩn bị Đơn Mang về (Bếp/Bar) & Nhân viên gửi thông tin món cần chuẩn bị đến bếp/bar, có đánh dấu "Takeout". (Hành động tương tự FR-MD05-08). \\
\hline
MD-06 & FR-MD06-07 & US-02, US-05 & Xác nhận và Tiến hành Thanh toán Đơn Mang về (có xem xét Cọc/Trả trước) & Nhân viên vào màn hình thanh toán, nơi hệ thống đã tự động áp dụng cọc/trả trước (nếu đơn hàng được đặt online và có trả trước). \\
\hline
MD-06 & FR-MD06-08 & US-02, US-05 & Thực hiện Thanh toán Tiền mặt cho Đơn Mang về & Nhân viên nhận tiền mặt và ghi nhận thanh toán. (Hành động tương tự FR-MD05-12). \\
\hline
MD-06 & FR-MD06-09 & US-02, US-05 & Ghi nhận Thanh toán bằng Phương thức Khác (Không Thẻ) cho Đơn Mang về & Nhân viên ghi nhận thanh toán bằng các phương thức khác được hỗ trợ (ví dụ: ví điện tử nếu có). \\
\hline
MD-06 & FR-MD06-10 & US-02, US-05 & Thực hiện Thanh toán Đơn Mang về bằng Nhiều Phương thức (Không Thẻ) & Nhân viên nhận thanh toán bằng cách kết hợp nhiều phương thức được hỗ trợ. (Hành động tương tự FR-MD05-13). \\
\hline
MD-06 & FR-MD06-11 & US-02, US-05 & In Hóa đơn/Biên lai cho Đơn Mang về & Nhân viên kích hoạt in hóa đơn/biên lai sau khi thanh toán. (Hành động tương tự FR-MD05-15). \\
\hline
MD-06 & FR-MD06-12 & US-02, US-05 & Hoàn tất Đơn hàng Mang về & Nhân viên đóng đơn hàng mang về sau khi khách đã thanh toán và nhận hàng. (Hành động tương tự FR-MD05-16). \\
\hline

\end{longtable}


\subsubsubsection{Mục tiêu và Phạm vi}
\label{sssec:md06_objectives_scope}
Mục tiêu chính của module MD-06 là:
\begin{itemize}
    \item \textbf{Xử lý nhanh đơn hàng mang về:} Cung cấp một quy trình tinh gọn cho nhân viên để tiếp nhận và xử lý các đơn hàng mang đi một cách hiệu quả, giảm thời gian chờ đợi cho khách hàng.
    \item \textbf{Quản lý đơn hàng không cần bàn:} Cho phép tạo và quản lý các đơn hàng mà không cần liên kết với một bàn cụ thể trong nhà hàng.
    \item \textbf{Tích hợp với quy trình chuẩn bị:} Đảm bảo thông tin đơn hàng mang về được gửi chính xác xuống bếp/bar, có phân biệt rõ ràng với đơn ăn tại bàn để bộ phận chuẩn bị có quy trình đóng gói phù hợp.
    \item \textbf{Linh hoạt trong việc gán khách hàng (tùy chọn):} Cho phép liên kết đơn hàng với thông tin khách hàng để tiện theo dõi hoặc áp dụng các chương trình khuyến mãi.
    \item \textbf{Hỗ trợ thanh toán đa dạng:} Cho phép khách hàng thanh toán bằng nhiều phương thức khác nhau.
    \item \textbf{Xử lý các khoản trả trước/đặt cọc:} Tự động áp dụng các khoản tiền khách hàng có thể đã thanh toán trước khi đặt hàng mang về qua các kênh khác (ví dụ: website, ứng dụng).
\end{itemize}
Phạm vi của module bao gồm từ việc nhân viên chọn chế độ bán mang về, tạo đơn hàng, thêm món, gửi yêu cầu chuẩn bị, cho đến khi xử lý thanh toán và hoàn tất đơn hàng. Module này không bao gồm các chức năng quản lý bàn hoặc đặt chỗ phức tạp như ở module ăn tại bàn (MD-05) hay đặt chỗ (MD-03).

\subsubsubsection{Đối tượng Sử dụng Chính}
\label{sssec:md06_primary_users}
Các đối tượng người dùng chính tương tác với module này bao gồm:
\begin{itemize}
    \item \textbf{US-02 (Nhân viên phục vụ):} Có thể trực tiếp nhận đơn hàng mang về từ khách tại quầy.
    \item \textbf{US-05 (Nhân viên thu ngân):} Thường là người chính xử lý các đơn hàng mang về, bao gồm việc tạo đơn, nhận thanh toán và hoàn tất giao dịch.
    \item \textbf{US-01 (Quản lý nhà hàng):} Có thể sử dụng các chức năng này và giám sát hoạt động bán mang về.
\end{itemize}
Khách hàng (US-08) là người yêu cầu dịch vụ mang về và cung cấp thông tin đơn hàng.

\subsubsubsection{Các Chức năng Chính}
\label{sssec:md06_key_functionalities}
Module MD-06 cung cấp các chức năng cần thiết để quản lý hiệu quả quy trình bán mang về, được mô tả chi tiết qua các Use Case sau:

\begin{itemize}
    \item \textbf{Khởi tạo và Quản lý Đơn hàng Mang về (UC-MD06-01 đến UC-MD06-03):}
    \begin{itemize}
        \item Cho phép nhân viên chủ động chọn hoặc chuyển sang chế độ hoạt động dành riêng cho bán mang về trên giao diện POS (UC-MD06-01).
        \item Hệ thống tự động hoặc nhân viên khởi tạo một đơn hàng POS mới, được đánh dấu là loại hình "Mang về" (UC-MD06-02).
        \item (Tùy chọn) Cho phép nhân viên tìm kiếm khách hàng đã có hoặc tạo nhanh thông tin khách hàng mới để liên kết với đơn hàng mang về (UC-MD06-03).
    \end{itemize}

    \item \textbf{Thao tác trên Đơn hàng Mang về (UC-MD06-04 đến UC-MD06-06):}
    \begin{itemize}
        \item Thêm các món ăn/đồ uống vào đơn hàng mang về, bao gồm việc chọn biến thể và điều chỉnh số lượng (UC-MD06-04, tương tự UC-MD05-05 và UC-MD05-06).
        \item Thêm các ghi chú hoặc yêu cầu đặc biệt của khách (ví dụ: về đóng gói, khẩu vị) cho từng món hoặc toàn bộ đơn hàng mang về (UC-MD06-05, tương tự UC-MD05-07).
        \item Gửi thông tin các món đã chọn của đơn hàng mang về xuống các máy in bếp/bar hoặc màn hình KDS, có chỉ dẫn rõ đây là đơn mang về (UC-MD06-06, tương tự UC-MD05-08 nhưng có thêm thông tin "Takeout").
    \end{itemize}

    \item \textbf{Xử lý Thanh toán cho Đơn hàng Mang về (UC-MD06-07 đến UC-MD06-12):}
    \begin{itemize}
        \item Trước khi vào màn hình thanh toán, hệ thống tự động kiểm tra và áp dụng (trừ đi) các khoản tiền đặt cọc hoặc thanh toán trước mà khách hàng có thể đã thực hiện cho đơn hàng mang về đó (UC-MD06-07).
        \item Thực hiện thanh toán bằng tiền mặt cho đơn hàng mang về (UC-MD06-08, tương tự UC-MD05-12).
        \item Ghi nhận thanh toán bằng các phương thức khác được hỗ trợ (không bao gồm thẻ ngân hàng qua terminal tích hợp) cho đơn hàng mang về (UC-MD06-09, tương tự UC-MD05-13).
        \item Thực hiện thanh toán cho đơn hàng mang về bằng cách kết hợp nhiều phương thức khác nhau (không bao gồm thẻ) (UC-MD06-10, tương tự UC-MD05-14).
        \item Sau khi nhận đủ thanh toán, cho phép nhân viên kích hoạt (hoặc hệ thống tự động) in hóa đơn/biên lai cuối cùng cho đơn hàng mang về (UC-MD06-11, tương tự UC-MD05-16).
        \item Hoàn tất và đóng đơn hàng mang về trong hệ thống sau khi khách đã thanh toán và nhận hàng (UC-MD06-12, tương tự UC-MD05-17).
    \end{itemize}
\end{itemize}

\subsubsubsection{Tóm tắt Luồng Hoạt động Tổng thể}
\label{sssec:md06_overall_workflow}
Luồng hoạt động điển hình trong module Bán mang về (POS - Takeout) diễn ra như sau:
\begin{enumerate}
    \item \textbf{Chuyển sang chế độ mang về:} Nhân viên Chọn Chế độ Bán Mang về (UC-MD06-01) trên giao diện POS.
    \item \textbf{Tạo đơn hàng:} Hệ thống tự động hoặc nhân viên Tạo Đơn hàng Mang về Mới (UC-MD06-02).
    \item \textbf{(Tùy chọn) Gán khách hàng:} Nhân viên Gán Khách hàng vào Đơn Mang về (UC-MD06-03).
    \item \textbf{Nhập món ăn:}
        \begin{itemize}
            \item Nhân viên Thêm Món vào Đơn hàng Mang về (UC-MD06-04).
            \item Thêm Ghi chú cho Đơn Mang về (UC-MD06-05) nếu khách có yêu cầu đặc biệt.
        \end{itemize}
    \item \textbf{Gửi yêu cầu chuẩn bị:} Nhân viên Gửi Yêu cầu Chuẩn bị Đơn Mang về (Bếp/Bar) (UC-MD06-06), phiếu gửi đi có ghi rõ là "Takeout".
    \item \textbf{Tiến hành thanh toán:}
        \begin{itemize}
            \item Nhân viên chọn thanh toán, hệ thống Xác nhận và Tiến hành Thanh toán Đơn Mang về, có xem xét và tự động áp dụng Cọc/Trả trước nếu có (UC-MD06-07).
            \item Nhân viên nhận thanh toán bằng một hoặc nhiều phương thức: Tiền mặt (UC-MD06-08), Phương thức Khác (Không Thẻ) (UC-MD06-09), hoặc kết hợp Nhiều Phương thức (Không Thẻ) (UC-MD06-10).
        \end{itemize}
    \item \textbf{Hoàn tất giao dịch:}
        \begin{itemize}
            \item Sau khi thanh toán đủ, nhân viên In Hóa đơn/Biên lai cho Đơn Mang về (UC-MD06-11).
            \item Cuối cùng, nhân viên Hoàn tất Đơn hàng Mang về (UC-MD06-12) trong hệ thống.
        \end{itemize}
\end{enumerate}
Module MD-06 đảm bảo rằng các đơn hàng mang về được xử lý một cách nhanh chóng, chính xác và hiệu quả, đáp ứng nhu cầu của cả khách hàng và nhà hàng.


\subsubsection{Module MD-07: Quản lý Giao hàng (POS - Delivery)}

\begin{longtable}{|m{2cm}|m{2.5cm}|m{2.5cm}|m{4.5cm}|m{4cm}|}
\caption{Danh sách Yêu cầu Chức năng cho Module MD-07: Quản lý Giao hàng (POS - Delivery)} \label{tab:fr_md07} \\
\hline
\textbf{Mã Module} & \textbf{Mã Yêu cầu CN} & \textbf{Mã Người dùng} & \textbf{Tên Chức năng} & \textbf{Mô tả Ngắn} \\
\hline
\endhead % Header cho các trang tiếp theo

\hline
\endfoot % Footer cho bảng

\hline
\endlastfoot % Footer cho trang cuối cùng

MD-07 & FR-MD07-01 & US-02, US-05 & Chọn Chế độ Giao hàng & Cho phép nhân viên chọn một chế độ/giao diện riêng biệt trên POS dành cho việc xử lý các đơn hàng giao đi (Delivery). \\
\hline
MD-07 & FR-MD07-02 & US-02, US-05 & Tạo/Mở Đơn hàng Giao hàng & Khởi tạo đơn hàng mới hoặc mở lại đơn hàng giao hàng đang chờ xử lý. Yêu cầu liên kết với thông tin khách hàng (địa chỉ giao, SĐT). \\
\hline
MD-07 & FR-MD07-03 & US-02, US-05 & Liên kết/Nhập Thông tin Khách hàng Giao hàng & Yêu cầu bắt buộc tìm kiếm/chọn khách hàng đã có (với địa chỉ) hoặc nhập thông tin khách hàng mới bao gồm Tên, SĐT và Địa chỉ giao hàng chi tiết. \\
\hline
MD-07 & FR-MD07-04 & US-02, US-05 & Thêm món vào Đơn hàng Giao hàng & Cho phép nhân viên thêm các món ăn/đồ uống vào đơn hàng giao đi. Tương tự UC-MD05-05 / UC-MD06-04. \\
\hline
MD-07 & FR-MD07-05 & US-02, US-05 & Xử lý Ghi chú cho Đơn Giao hàng & Cho phép thêm ghi chú đặc biệt (ví dụ: "Giao vào giờ nghỉ trưa", "Gọi trước khi đến"). Tương tự UC-MD05-06 / UC-MD06-05. \\
\hline
MD-07 & FR-MD07-06 & US-02, US-05 & Gửi đơn Giao hàng xuống Bếp/Bar & Gửi thông tin món cần chuẩn bị xuống bếp/bar, đánh dấu là đơn "Delivery". Tương tự UC-MD05-07 / UC-MD06-06. \\
\hline
MD-07 & FR-MD07-07 & US-02, US-05, System & Áp dụng Đặt cọc/Thanh toán Trước (Nếu có) & Nếu đơn hàng giao đi được đặt online và đã thanh toán trước (toàn bộ hoặc đặt cọc), hệ thống cần áp dụng khoản đã thanh toán này. Logic tương tự UC-MD05-09 / UC-MD06-07. \\
\hline
MD-07 & FR-MD07-08 & US-02, US-05 & Xác nhận và Gửi Đơn hàng sang Shipday & Sau khi đơn hàng sẵn sàng, nhân viên xác nhận và kích hoạt việc gửi thông tin đơn hàng (chi tiết món, thông tin khách hàng, địa chỉ giao) sang hệ thống Shipday qua API. \\
\hline
MD-07 & FR-MD07-09 & System (Odoo/Shipday) & Nhận và Hiển thị Trạng thái Giao hàng từ Shipday & Hệ thống Odoo nhận cập nhật trạng thái giao hàng (ví dụ: Đã gán tài xế, Đang giao, Đã giao thành công, Giao thất bại) từ Shipday (qua webhook) và hiển thị trên chi tiết đơn hàng POS/Backend. \\
\hline
MD-07 & FR-MD07-10 & US-02, US-05 & Xử lý Thanh toán Đơn hàng Giao hàng (Nếu COD) & Nếu đơn hàng thanh toán khi nhận hàng (COD), cho phép nhân viên ghi nhận thanh toán sau khi nhận được tiền từ tài xế giao hàng. \\
\hline
MD-07 & FR-MD07-11 & US-02, US-05 & In Hóa đơn/Phiếu Giao hàng & In hóa đơn/phiếu giao hàng chứa thông tin chi tiết đơn hàng, thông tin khách hàng, địa chỉ giao để tài xế sử dụng và giao cho khách. \\
\hline
MD-07 & FR-MD07-12 & US-02, US-05, System & Đóng Đơn hàng Giao hàng & Hoàn tất và đóng đơn hàng giao đi sau khi đã giao thành công và (nếu COD) đã nhận đủ thanh toán. \\
\hline
MD-07 & FR-MD07-13 & US-01 / US-10 & Cấu hình Tích hợp Shipday & Cho phép cấu hình các tham số kết nối API giữa Odoo và Shipday (API Key, Endpoint...), và các quy tắc đồng bộ dữ liệu. \\
\hline

\end{longtable}


\subsubsection{Module MD-08: Tích hợp Bếp (Kitchen Integration)}
Module Quản lý Giao hàng (MD-07) là một thành phần quan trọng của hệ thống Point of Sale (POS), được thiết kế đặc biệt để hỗ trợ nhà hàng quản lý các đơn hàng mà khách yêu cầu giao đến một địa chỉ cụ thể. Module này tập trung vào việc thu thập thông tin khách hàng và địa chỉ giao hàng, xử lý đơn hàng, và đặc biệt là tích hợp với dịch vụ quản lý giao hàng của bên thứ ba (trong trường hợp này là Shipday) để tự động hóa việc gửi yêu cầu giao hàng và theo dõi trạng thái.

Module Tích hợp Bếp (MD-08) đóng vai trò cầu nối quan trọng giữa bộ phận phục vụ (thông qua hệ thống POS) và bộ phận bếp/bar. Mục tiêu chính của module này là đảm bảo thông tin đơn hàng được truyền tải một cách chính xác, kịp thời và hiệu quả đến các nhân viên bếp, giúp họ chuẩn bị món ăn đúng theo yêu cầu và tối ưu hóa quy trình làm việc trong bếp. Module này có thể được triển khai dưới dạng Màn hình Hiển thị Bếp (Kitchen Display System - KDS) hoặc thông qua việc sử dụng máy in bếp truyền thống.


\begin{figure}[H]
    \centering
    \includegraphics[width=15cm]{Sections/tong_quan/functional_spec/img/uc8.png}
    \vspace{0.5cm}
    \caption{Use case diagram cho Module MD-08}
    \label{fig:my_label}
\end{figure}

\begin{longtable}{|m{2cm}|m{2.5cm}|m{2.5cm}|m{4.5cm}|m{4cm}|}
\caption{Danh sách Yêu cầu Chức năng cho Module MD-08: Tích hợp Bếp (Kitchen Integration)} \label{tab:fr_md08_revised_v2} \\
\hline
\textbf{Mã Module} & \textbf{Mã Yêu cầu CN} & \textbf{Mã Người dùng} & \textbf{Tên Chức năng} & \textbf{Mô tả Ngắn} \\
\hline
\endhead % Header cho các trang tiếp theo
\hline
\endfoot % Footer cho bảng
\hline
\endlastfoot % Footer cho trang cuối cùng

MD-08 & FR-MD08-01 & US-04 & Xem Đơn hàng/Món ăn Mới trên KDS/Máy in Bếp & Nhân viên bếp xem các đơn hàng/món ăn mới được gửi đến KDS hoặc nhận phiếu in từ máy in bếp. (Việc gửi đi là kết quả của FR-MD05-08, FR-MD06-06, FR-MD07-06). \\
\hline
MD-08 & FR-MD08-02 & US-04 & Xem Chi tiết Yêu cầu Món ăn trên KDS/Phiếu in & Nhân viên bếp đọc thông tin chi tiết của từng món cần chuẩn bị: tên, số lượng, biến thể, ghi chú đặc biệt. \\
\hline
MD-08 & FR-MD08-03 & US-04 & Cập nhật Trạng thái Chế biến Món ăn trên KDS & Nhân viên bếp tương tác với KDS để đánh dấu trạng thái chế biến của món ăn (ví dụ: Bắt đầu làm, Đã xong). \\
\hline
MD-08 & FR-MD08-04 & US-04 & Đánh dấu Hoàn thành Toàn bộ Đơn hàng/Phiếu trên KDS & Nhân viên bếp đánh dấu toàn bộ các món trong một đơn hàng/phiếu đã được chuẩn bị xong trên KDS. \\
\hline
MD-08 & FR-MD08-05 & US-04 & Sắp xếp/Đánh dấu Ưu tiên Đơn hàng trên KDS & Nhân viên bếp thay đổi thứ tự hoặc đánh dấu ưu tiên cho các đơn hàng/phiếu trên KDS. \\
\hline
MD-08 & FR-MD08-06 & US-02/US-05 & Xem Cập nhật Trạng thái Món ăn từ Bếp trên POS & Nhân viên phục vụ/thu ngân xem được thông tin món nào đã sẵn sàng từ bếp (nếu KDS có gửi cập nhật về POS). \\
\hline

\end{longtable}


\subsubsubsection{Mục tiêu và Phạm vi}
\label{sssec:md08_objectives_scope}
Mục tiêu chính của module MD-08 là:
\begin{itemize}
    \item \textbf{Truyền tải chính xác yêu cầu món ăn:} Đảm bảo mọi chi tiết của đơn hàng (tên món, số lượng, biến thể, ghi chú đặc biệt) được gửi từ POS đến bếp một cách đầy đủ và không sai sót.
    \item \textbf{Tối ưu hóa quy trình làm việc trong bếp:} Giúp nhân viên bếp dễ dàng tiếp nhận, xem, quản lý và theo dõi tiến độ chuẩn bị các món ăn.
    \item \textbf{Giảm thiểu sai sót và nhầm lẫn:} Hạn chế việc trao đổi thông tin bằng miệng hoặc giấy tờ dễ thất lạc, từ đó giảm lỗi trong quá trình chế biến.
    \item \textbf{Cải thiện thời gian phục vụ:} Giúp bếp nhận yêu cầu nhanh hơn và quản lý thứ tự ưu tiên hiệu quả hơn (đặc biệt với KDS).
    \item \textbf{(Nếu dùng KDS) Cung cấp khả năng theo dõi và cập nhật trạng thái:} Cho phép nhân viên bếp đánh dấu trạng thái chế biến (đang làm, đã xong) và (tùy chọn) đồng bộ thông tin này ngược lại cho nhân viên phục vụ.
    \item \textbf{Hỗ trợ định tuyến thông minh:} Đảm bảo các món ăn được gửi đến đúng trạm chuẩn bị (ví dụ: món chính gửi bếp chính, đồ uống gửi quầy bar) nếu nhà hàng có nhiều khu vực bếp/bar.
\end{itemize}
Phạm vi của module bao gồm việc tiếp nhận yêu cầu món ăn từ hệ thống POS (MD-05, MD-06, MD-07), hiển thị thông tin chi tiết cho nhân viên bếp, và (nếu sử dụng KDS) cho phép nhân viên bếp tương tác để cập nhật trạng thái chế biến. Nó không bao gồm việc quản lý công thức, định lượng nguyên vật liệu, hay các chức năng quản lý kho chi tiết (thuộc các module khác).

\subsubsubsection{Đối tượng Sử dụng Chính}
\label{sssec:md08_primary_users}
Đối tượng người dùng chính của module này là:
\begin{itemize}
    \item \textbf{US-04 (Nhân viên bếp):} Là người trực tiếp sử dụng KDS hoặc nhận phiếu in từ máy in bếp để xem yêu cầu, chuẩn bị món ăn, và (nếu có KDS) cập nhật trạng thái chế biến.
\end{itemize}
Các đối tượng khác tương tác gián tiếp:
\begin{itemize}
    \item \textbf{US-02 (Nhân viên phục vụ) / US-05 (Nhân viên thu ngân):} Là người gửi yêu cầu chuẩn bị món từ POS. Họ cũng có thể (tùy chọn) nhận được cập nhật trạng thái món ăn từ KDS (UC-MD08-06).
    \item \textbf{US-01 (Quản lý nhà hàng) / US-10 (Quản trị viên Hệ thống):} Chịu trách nhiệm cấu hình máy in bếp, KDS, và các quy tắc định tuyến.
\end{itemize}

\subsubsubsection{Các Chức năng Chính}
\label{sssec:md08_key_functionalities}
Module MD-08 cung cấp các chức năng thiết yếu cho việc vận hành bếp, được mô tả chi tiết qua các Use Case sau:

\begin{itemize}
    \item \textbf{Tiếp nhận và Hiển thị Yêu cầu (UC-MD08-01, UC-MD08-02):}
    \begin{itemize}
        \item Nhân viên bếp xem các đơn hàng/món ăn mới xuất hiện trên Màn hình Hiển thị Bếp (KDS) hoặc nhận phiếu yêu cầu được in ra từ máy in bếp (UC-MD08-01).
        \item Nhân viên bếp xem thông tin chi tiết của từng yêu cầu món ăn, bao gồm tên món, số lượng, các tùy chọn biến thể và ghi chú đặc biệt (UC-MD08-02).
    \end{itemize}

    \item \textbf{Quản lý Trạng thái Chế biến trên KDS (UC-MD08-03, UC-MD08-04):} (Áp dụng nếu sử dụng KDS)
    \begin{itemize}
        \item Nhân viên bếp cập nhật trạng thái chế biến của từng món ăn cụ thể trên KDS (ví dụ: "Đang làm", "Đã xong") (UC-MD08-03).
        \item Nhân viên bếp đánh dấu hoàn thành toàn bộ một đơn hàng/phiếu trên KDS khi tất cả các món trong đó đã được chuẩn bị xong (UC-MD08-04).
    \end{itemize}

    \item \textbf{Tối ưu hóa và Đồng bộ hóa (UC-MD08-05, UC-MD08-06):} (Chủ yếu áp dụng cho KDS)
    \begin{itemize}
        \item (Tùy chọn) Nhân viên bếp có thể sắp xếp lại thứ tự hoặc đánh dấu ưu tiên cho các đơn hàng/phiếu trên KDS để quản lý công việc hiệu quả hơn (UC-MD08-05).
        \item (Tùy chọn) Hệ thống cho phép nhân viên phục vụ trên POS xem được thông tin cập nhật về trạng thái món ăn ("Đã xong") từ KDS (UC-MD08-06).
    \end{itemize}
\end{itemize}

\subsubsubsection{Tóm tắt Luồng Hoạt động Tổng thể}
\label{sssec:md08_overall_workflow}
Luồng hoạt động chính trong module Tích hợp Bếp thường diễn ra như sau:
\begin{enumerate}
    \item \textbf{Nhận yêu cầu từ POS:}
        \begin{itemize}
            \item Khi nhân viên phục vụ gửi yêu cầu chuẩn bị món từ POS (UC-MD05-08, UC-MD06-06, UC-MD07-06), thông tin được chuyển đến bếp.
            \item Nhân viên bếp Xem Đơn hàng/Món ăn Mới trên KDS hoặc Nhận Phiếu in Bếp (UC-MD08-01).
        \end{itemize}
    \item \textbf{Xem chi tiết và chuẩn bị:}
        \begin{itemize}
            \item Nhân viên bếp Xem Chi tiết Yêu cầu Món ăn trên KDS/Phiếu in (UC-MD08-02) để nắm rõ các yêu cầu về món, biến thể, và ghi chú.
            \item Nhân viên bếp tiến hành chuẩn bị món ăn.
        \end{itemize}
    \item \textbf{Cập nhật trạng thái (Nếu dùng KDS):}
        \begin{itemize}
            \item Trong quá trình chuẩn bị, nhân viên bếp Cập nhật Trạng thái Chế biến Món ăn trên KDS (UC-MD08-03), ví dụ: chuyển từ "Chờ" sang "Đang làm".
            \item (Tùy chọn) Nhân viên bếp có thể Sắp xếp/Đánh dấu Ưu tiên Đơn hàng trên KDS (UC-MD08-05) nếu cần.
            \item Khi tất cả các món trong một đơn/phiếu đã xong, nhân viên bếp Đánh dấu Hoàn thành Toàn bộ Đơn hàng/Phiếu trên KDS (UC-MD08-04).
        \end{itemize}
    \item \textbf{(Tùy chọn) Đồng bộ về POS (Nếu dùng KDS và có cấu hình):}
        \begin{itemize}
            \item Hệ thống cho phép nhân viên phục vụ Xem Cập nhật Trạng thái Món ăn từ Bếp trên POS (UC-MD08-06), giúp họ biết món nào đã sẵn sàng để phục vụ.
        \end{itemize}
\end{enumerate}
Module MD-08 giúp số hóa và tối ưu hóa giao tiếp giữa bộ phận phục vụ và bếp, góp phần nâng cao hiệu suất và chất lượng dịch vụ của nhà hàng.




\subsubsection{Module MD-09: Quản lý Phiên \& Báo cáo}

Module Quản lý Phiên \& Báo cáo (MD-09) là một thành phần thiết yếu trong hệ thống quản lý nhà hàng, tập trung vào việc cung cấp các công cụ để theo dõi, tổng kết và phân tích hoạt động bán hàng từ các phiên làm việc Point of Sale (POS). Module này cho phép Quản lý nhà hàng và Kế toán truy cập vào dữ liệu lịch sử, xem xét hiệu suất kinh doanh, đối soát tài chính, và đưa ra các quyết định dựa trên số liệu thực tế.




\begin{figure}[H]
    \centering
    \includegraphics[width=15cm]{Sections/tong_quan/functional_spec/img/uc9.png}
    \vspace{0.5cm}
    \caption{Use case diagram cho Module MD-09}
    \label{fig:my_label}
\end{figure}

\begin{longtable}{|m{2cm}|m{2.5cm}|m{2cm}|m{4.5cm}|m{4cm}|}
\caption{Danh sách Yêu cầu Chức năng cho Module MD-09: Quản lý Phiên \& Báo cáo} \label{tab:fr_md09_revised_v2} \\
\hline
\textbf{Mã Module} & \textbf{Mã Yêu cầu CN} & \textbf{Mã Người dùng} & \textbf{Tên Chức năng} & \textbf{Mô tả Ngắn} \\
\hline
\endhead % Header cho các trang tiếp theo
\hline
\endfoot % Footer cho bảng
\hline
\endlastfoot % Footer cho trang cuối cùng

MD-09 & FR-MD09-01 & US-01/US-06 & Xem Danh sách các Phiên POS đã Đóng & Cho phép Quản lý/Kế toán xem danh sách các phiên POS đã được đóng. \\
\hline
MD-09 & FR-MD09-02 & US-01/US-06 & Xem Chi tiết một Phiên POS đã Đóng (Báo cáo Doanh thu Phiên) & Cung cấp báo cáo chi tiết về doanh thu, thanh toán, tiền mặt... của một phiên POS cụ thể đã đóng. \\
\hline
MD-09 & FR-MD09-03 & US-01/US-06 & Xem Báo cáo Tổng hợp Bán hàng theo Sản phẩm/Danh mục POS & Thống kê số lượng và doanh thu của từng sản phẩm/danh mục POS trong một khoảng thời gian. \\
\hline
MD-09 & FR-MD09-04 & US-01 & Xem Báo cáo Hiệu suất Bán hàng của Nhân viên POS & Thống kê doanh thu hoặc số lượng đơn hàng do từng nhân viên xử lý trên POS. \\
\hline
MD-09 & FR-MD09-05 & US-01/US-06 & Xem Báo cáo Quản lý Tiền Đặt cọc & Báo cáo tổng hợp tình hình thu, sử dụng, và mất cọc từ các lượt đặt chỗ. \\
\hline
MD-09 & FR-MD09-06 & US-01/US-06 & Xem Báo cáo Doanh thu theo Loại hình Đơn hàng & Phân tích doanh thu dựa trên các loại hình: Ăn tại chỗ, Mang về, Giao hàng. \\
\hline
MD-09 & FR-MD09-07 & US-01/US-06 & Xuất Dữ liệu từ các Báo cáo & Cho phép xuất dữ liệu báo cáo ra định dạng Excel/CSV. \\
\hline
MD-09 & FR-MD09-08 & US-01/US-06 & Xem Báo cáo Thanh toán theo Phương thức & Thống kê tổng số tiền thu được theo từng phương thức thanh toán (Tiền mặt, Ví điện tử...) trong một khoảng thời gian. \\
\hline
MD-09 & FR-MD09-09 & US-01 & Xem Báo cáo các Khoản Giảm giá/Khuyến mãi đã Áp dụng & Thống kê tổng giá trị giảm giá đã được áp dụng trên các đơn hàng trong một khoảng thời gian. \\
\hline
MD-09 & FR-MD09-10 & US-01 & Xem Báo cáo các Đơn hàng/Món ăn đã Hủy (Void) & Thống kê số lượng và giá trị các đơn hàng hoặc món ăn đã bị hủy trên POS, có thể kèm lý do. \\
\hline
MD-09 & FR-MD09-11 & US-01/US-06 & (Tùy chọn) Thiết lập và Lên lịch Gửi Báo cáo Tự động & Cho phép cấu hình để hệ thống tự động gửi một số báo cáo nhất định (ví dụ: báo cáo doanh thu ngày) đến email của người quản lý theo lịch. \\
\hline

\end{longtable}

\subsubsubsection{Mục tiêu và Phạm vi}
\label{sssec:md08_objectives_scope}
Mục tiêu chính của module MD-08 là:
\begin{itemize}
    \item \textbf{Truyền tải chính xác yêu cầu món ăn:} Đảm bảo mọi chi tiết của đơn hàng (tên món, số lượng, biến thể, ghi chú đặc biệt) được gửi từ POS đến bếp một cách đầy đủ và không sai sót.
    \item \textbf{Tối ưu hóa quy trình làm việc trong bếp:} Giúp nhân viên bếp dễ dàng tiếp nhận, xem, quản lý và theo dõi tiến độ chuẩn bị các món ăn.
    \item \textbf{Giảm thiểu sai sót và nhầm lẫn:} Hạn chế việc trao đổi thông tin bằng miệng hoặc giấy tờ dễ thất lạc, từ đó giảm lỗi trong quá trình chế biến.
    \item \textbf{Cải thiện thời gian phục vụ:} Giúp bếp nhận yêu cầu nhanh hơn và quản lý thứ tự ưu tiên hiệu quả hơn (đặc biệt với KDS).
    \item \textbf{(Nếu dùng KDS) Cung cấp khả năng theo dõi và cập nhật trạng thái:} Cho phép nhân viên bếp đánh dấu trạng thái chế biến (đang làm, đã xong) và (tùy chọn) đồng bộ thông tin này ngược lại cho nhân viên phục vụ.
    \item \textbf{Hỗ trợ định tuyến thông minh:} Đảm bảo các món ăn được gửi đến đúng trạm chuẩn bị (ví dụ: món chính gửi bếp chính, đồ uống gửi quầy bar) nếu nhà hàng có nhiều khu vực bếp/bar.
\end{itemize}

\subsubsubsection{Mục tiêu và Phạm vi}
\label{sssec:md09_objectives_scope}
Mục tiêu chính của module MD-09 là:
\begin{itemize}
    \item \textbf{Cung cấp thông tin tổng kết phiên POS:} Cho phép xem lại chi tiết tài chính và hoạt động của từng phiên POS đã đóng.
    \item \textbf{Phân tích hiệu quả bán hàng:} Cung cấp các báo cáo đa dạng về doanh thu theo sản phẩm, danh mục, nhân viên, và loại hình đơn hàng.
    \item \textbf{Hỗ trợ đối soát tài chính:} Cung cấp báo cáo về các phương thức thanh toán và quản lý tiền đặt cọc, giúp kế toán dễ dàng đối soát.
    \item \textbf{Kiểm soát thất thoát và hoạt động bất thường:} Thông qua báo cáo về các đơn hàng/món ăn đã hủy.
    \item \textbf{Tăng cường khả năng ra quyết định dựa trên dữ liệu:} Cung cấp thông tin đầu vào quan trọng cho việc lập kế hoạch kinh doanh, marketing, và quản lý nhân sự.
    \item \textbf{(Tùy chọn) Tự động hóa việc gửi báo cáo:} Giúp các nhà quản lý nhận được thông tin quan trọng một cách định kỳ mà không cần thao tác thủ công.
\end{itemize}
Phạm vi của module bao gồm việc hiển thị danh sách các phiên POS đã đóng, cung cấp báo cáo chi tiết cho từng phiên, tạo các báo cáo tổng hợp và phân tích về doanh thu, sản phẩm, nhân viên, thanh toán, giảm giá, hủy đơn, và (tùy chọn) thiết lập cơ chế gửi báo cáo tự động. Dữ liệu đầu vào cho module này chủ yếu đến từ các giao dịch được ghi nhận trong các module POS (MD-05, MD-06, MD-07) và Đặt chỗ (MD-03, liên quan đến đặt cọc).

\subsubsubsection{Đối tượng Sử dụng Chính}
\label{sssec:md09_primary_users}
Các đối tượng người dùng chính tương tác với module này bao gồm:
\begin{itemize}
    \item \textbf{US-01 (Quản lý nhà hàng):} Là người dùng thường xuyên nhất, sử dụng các báo cáo để theo dõi tình hình kinh doanh, đánh giá hiệu suất, kiểm soát hoạt động và đưa ra các quyết định chiến lược.
    \item \textbf{US-06 (Kế toán):} Sử dụng các báo cáo (đặc biệt là báo cáo chi tiết phiên, báo cáo thanh toán theo phương thức, báo cáo tiền đặt cọc) để thực hiện công việc đối soát tài chính, hạch toán kế toán.
\end{itemize}

\subsubsubsection{Các Chức năng Chính}
\label{sssec:md09_key_functionalities}
Module MD-09 cung cấp một loạt các chức năng báo cáo và quản lý dữ liệu lịch sử, được mô tả chi tiết qua các Use Case sau:

\begin{itemize}
    \item \textbf{Quản lý và Xem lại Phiên POS (UC-MD09-01, UC-MD09-02):}
    \begin{itemize}
        \item Cho phép xem danh sách các phiên làm việc POS đã được đóng, với các thông tin tóm tắt và khả năng lọc/tìm kiếm (UC-MD09-01).
        \item Xem báo cáo chi tiết tổng kết của một phiên POS cụ thể đã đóng, bao gồm doanh thu, chi tiết thanh toán, và đối chiếu tiền mặt (nếu có) (UC-MD09-02).
    \end{itemize}

    \item \textbf{Báo cáo Phân tích Bán hàng (UC-MD09-03, UC-MD09-04, UC-MD09-06, UC-MD09-08, UC-MD09-09, UC-MD09-10):}
    \begin{itemize}
        \item Xem báo cáo thống kê về số lượng bán ra và doanh thu của từng sản phẩm hoặc theo từng danh mục sản phẩm POS (UC-MD09-03).
        \item Xem báo cáo về hiệu suất bán hàng của từng nhân viên POS, bao gồm tổng doanh thu và số lượng đơn hàng (UC-MD09-04).
        \item Xem báo cáo phân tích tổng doanh thu theo từng loại hình đơn hàng (Ăn tại chỗ, Mang về, Giao hàng) (UC-MD09-06).
        \item Xem báo cáo thống kê tổng số tiền đã thu được qua từng phương thức thanh toán khác nhau (UC-MD09-08).
        \item Xem báo cáo về tổng giá trị các khoản giảm giá hoặc chương trình khuyến mãi đã được áp dụng (UC-MD09-09).
        \item Xem báo cáo thống kê về các món ăn hoặc toàn bộ đơn hàng đã bị hủy (voided) trên POS (UC-MD09-10).
    \end{itemize}

    \item \textbf{Báo cáo Quản lý Đặc thù (UC-MD09-05):}
    \begin{itemize}
        \item Xem báo cáo tổng hợp về tình hình thu và sử dụng tiền đặt cọc từ các lượt đặt chỗ (UC-MD09-05).
    \end{itemize}

    \item \textbf{Tiện ích Báo cáo (UC-MD09-07, UC-MD09-11):}
    \begin{itemize}
        \item Cho phép xuất dữ liệu từ các giao diện báo cáo ra các định dạng tệp phổ biến như Excel hoặc CSV (UC-MD09-07).
        \item (Tùy chọn) Cho phép thiết lập và lên lịch để hệ thống tự động tạo và gửi một số loại báo cáo nhất định đến email theo định kỳ (UC-MD09-11).
    \end{itemize}
\end{itemize}

\subsubsubsection{Tóm tắt Luồng Hoạt động Tổng thể}
\label{sssec:md09_overall_workflow}
Luồng hoạt động chính trong module Quản lý Phiên & Báo cáo thường diễn ra như sau:
\begin{enumerate}
    \item \textbf{Truy cập thông tin phiên đã đóng:}
        \begin{itemize}
            \item Người dùng (Quản lý/Kế toán) Xem Danh sách các Phiên POS đã Đóng (UC-MD09-01).
            \item Từ danh sách, người dùng chọn một phiên cụ thể để Xem Chi tiết một Phiên POS đã Đóng (Báo cáo Doanh thu Phiên) (UC-MD09-02).
        \end{itemize}
    \item \textbf{Xem các báo cáo phân tích và tổng hợp:}
        \begin{itemize}
            \item Người dùng lựa chọn và xem các loại báo cáo khác nhau tùy theo nhu cầu phân tích:
                \begin{itemize}
                    \item Báo cáo Tổng hợp Bán hàng theo Sản phẩm/Danh mục POS (UC-MD09-03).
                    \item Báo cáo Hiệu suất Bán hàng của Nhân viên POS (UC-MD09-04).
                    \item Báo cáo Quản lý Tiền Đặt cọc (UC-MD09-05).
                    \item Báo cáo Doanh thu theo Loại hình Đơn hàng (UC-MD09-06).
                    \item Báo cáo Thanh toán theo Phương thức (UC-MD09-08).
                    \item Báo cáo các Khoản Giảm giá/Khuyến mãi đã Áp dụng (UC-MD09-09).
                    \item Báo cáo các Đơn hàng/Món ăn đã Hủy (Void) (UC-MD09-10).
                \end{itemize}
            \item Trong quá trình xem các báo cáo này, người dùng có thể cần Xuất Dữ liệu từ các Báo cáo (UC-MD09-07) ra file để lưu trữ hoặc phân tích thêm.
        \end{itemize}
    \item \textbf{(Tùy chọn) Thiết lập gửi báo cáo tự động:}
        \begin{itemize}
            \item Nếu có nhu cầu, người dùng có thể Thiết lập và Lên lịch Gửi Báo cáo Tự động (UC-MD09-11) cho các báo cáo quan trọng.
        \end{itemize}
\end{enumerate}
Module MD-09 cung cấp cái nhìn sâu sắc về hoạt động kinh doanh của nhà hàng, hỗ trợ việc đưa ra các quyết định quản lý dựa trên dữ liệu chính xác và kịp thời, đồng thời đảm bảo tính minh bạch và kiểm soát tài chính.



\subsubsection{Module MD-10: Quản lý Hệ thống \& Người 
dùng}
Module Quản lý Hệ thống \& Người dùng (MD-10) đóng vai trò là hạt nhân quản trị và cấu hình của toàn bộ hệ thống nhà hàng. Module này cung cấp các công cụ thiết yếu cho Quản trị viên hệ thống và Quản lý nhà hàng để quản lý tài khoản người dùng (nhân viên và khách hàng), phân quyền truy cập, thiết lập các thông số hoạt động chung của nhà hàng, cấu hình các tích hợp với dịch vụ của bên thứ ba (như cổng thanh toán, dịch vụ bot call, dịch vụ giao hàng), và theo dõi hoạt động hệ thống. Sự ổn định, bảo mật và khả năng tùy biến của module này là nền tảng cho hoạt động hiệu quả của tất cả các module nghiệp vụ khác.


\begin{figure}[H]
    \centering
    \includegraphics[width=15cm]{Sections/tong_quan/functional_spec/img/uc10.png}
    \vspace{0.5cm}
    \caption{Use case diagram cho Module MD-10}
    \label{fig:my_label}
\end{figure}

\begin{longtable}{|m{2cm}|m{2.5cm}|m{2.5cm}|m{4.5cm}|m{4cm}|}
\caption{Danh sách Yêu cầu Chức năng cho Module MD-10: Quản lý Hệ thống, Người dùng \& Xác thực} 
\hline
\textbf{Mã Module} & \textbf{Mã Yêu cầu CN} & \textbf{Mã Người dùng} & \textbf{Tên Chức năng} & \textbf{Mô tả Ngắn} \
\hline
\endhead % Header cho các trang tiếp theo
\midrule
\endfoot % Footer cho bảng
\bottomrule
\endlastfoot % Footer cho trang cuối cùng
MD-10 & FR-MD10-01 & US-10 & Tạo mới Tài khoản Người dùng (Nhân viên) & Quản trị viên tạo tài khoản đăng nhập mới cho nhân viên, bao gồm thông tin cơ bản. \
\hline
MD-10 & FR-MD10-02 & US-10 & Xem Danh sách Người dùng & Quản trị viên xem danh sách các tài khoản người dùng hiện có trong hệ thống. \
\hline
MD-10 & FR-MD10-03 & US-10 & Sửa Thông tin Tài khoản Người dùng & Quản trị viên cập nhật các thông tin cá nhân, liên hệ của một tài khoản người dùng. \
\hline
MD-10 & FR-MD10-04 & US-10 & Gán/Gỡ bỏ Nhóm Quyền cho Người dùng & Quản trị viên thay đổi các nhóm quyền truy cập mà một người dùng thuộc về. \
\hline
MD-10 & FR-MD10-05 & US-10 & Vô hiệu hóa Tài khoản Người dùng & Quản trị viên tạm khóa (archive) khả năng đăng nhập của một tài khoản người dùng. \
\hline
MD-10 & FR-MD10-06 & US-10 & Kích hoạt lại Tài khoản Người dùng đã Vô hiệu hóa & Quản trị viên mở lại (unarchive) khả năng đăng nhập cho một tài khoản đã bị khóa. \
\hline
MD-10 & FR-MD10-07 & US-10 & Đặt lại Mật khẩu cho Người dùng (bởi Admin) & Quản trị viên hỗ trợ đặt lại mật khẩu cho người dùng (gửi link hoặc đặt trực tiếp). \
\hline
MD-10 & FR-MD10-08 & US-10 & Xem Chi tiết một Nhóm Quyền Truy cập & Quản trị viên xem chi tiết cấu hình và các quyền hạn của một nhóm quyền cụ thể. \
\hline
MD-10 & FR-MD10-09 & US-10, US-01 & Cấu hình Thông tin Chung của Nhà hàng/Công ty & Quản trị viên/Quản lý thiết lập tên, địa chỉ, logo, tiền tệ mặc định cho nhà hàng/công ty. \
\hline
MD-10 & FR-MD10-10 & US-10, US-01 & Cấu hình Máy chủ Gửi Email (Outgoing Email Server) & Quản trị viên/Quản lý thiết lập thông tin máy chủ SMTP để hệ thống có thể gửi email đi. \
\hline
MD-10 & FR-MD10-11 & US-10, US-01 & Cấu hình Tích hợp Cổng Thanh toán & Quản trị viên/Quản lý nhập API keys và các tham số cho các cổng thanh toán trực tuyến. \
\hline
MD-10 & FR-MD10-12 & US-10, US-01 & Cấu hình Tích hợp Dịch vụ Bot Call & Quản trị viên/Quản lý nhập API keys và các tham số vận hành cho dịch vụ Bot Call. \
\hline
MD-10 & FR-MD10-13 & US-10, US-01 & Cấu hình Tích hợp Dịch vụ Giao hàng (Shipday) & Quản trị viên/Quản lý nhập API keys và các tham số vận hành cho dịch vụ Shipday. \
\hline
MD-10 & FR-MD10-14 & US-10, US-01 & Cấu hình Tham số Nghiệp vụ Đặc thù cho Đặt chỗ & Quản trị viên/Quản lý thiết lập các quy tắc kinh doanh như tỷ lệ đặt cọc, giá trị bàn, số ngày gọi bot... \
\hline
MD-10 & FR-MD10-15 & US-10 & Xem Nhật ký Hoạt động Hệ thống (Logs) & Quản trị viên xem các bản ghi log hệ thống để theo dõi, chẩn đoán lỗi và hoạt động. \
\hline
MD-10 & FR-MD10-16 & US-01, US-02, US-03, US-04, US-05, US-06, US-07, US-09, US-10 & Đăng nhập vào Hệ thống & Người dùng cung cấp thông tin xác thực (email/mật khẩu) để truy cập hệ thống. \
\hline
MD-10 & FR-MD10-17 & US-01, US-02, US-03, US-04, US-05, US-06, US-07, US-09, US-10 & Đăng xuất khỏi Hệ thống & Người dùng kết thúc phiên làm việc hiện tại và thoát khỏi hệ thống an toàn. \
\hline
MD-10 & FR-MD10-18 & US-01, US-02, US-03, US-04, US-05, US-06, US-07, US-09, US-10 & Người dùng Yêu cầu Đặt lại Mật khẩu (Quên Mật khẩu) & Người dùng tự yêu cầu hệ thống gửi hướng dẫn đặt lại mật khẩu qua email khi họ quên mật khẩu. \
\hline
MD-10 & FR-MD10-19 & US-08 & Khách hàng Tự Đăng ký Tài khoản & Khách hàng tự tạo tài khoản người dùng trên giao diện web/app để đặt chỗ, quản lý thông tin. \
\hline
\end{longtable}
    

% Bỏ comment và sửa đường dẫn nếu bạn có file ảnh uc10.png
% \begin{figure}[H]
%     \centering
%     % \includegraphics[width=15cm]{Sections/tong_quan/functional_spec/img/uc10.png.png} % Thay thế bằng đường dẫn thực tế đến file ảnh của bạn
%     \caption{Sơ đồ Use Case cho Module MD-10 (Ví dụ)} % Sửa caption nếu cần
%     \label{fig:uc_diagram_md10_full} % Sửa label nếu cần
% \end{figure}

\subsubsubsection{Mục tiêu và Phạm vi}
\label{sssec:md10_objectives_scope_full}
Mục tiêu chính của module MD-10 là:
\begin{itemize}
    \item \textbf{Quản lý tài khoản người dùng toàn diện:} Cho phép tạo, sửa đổi, vô hiệu hóa, kích hoạt lại và quản lý mật khẩu cho tất cả các loại tài khoản người dùng (nhân viên và khách hàng).
    \item \textbf{Kiểm soát truy cập chặt chẽ:} Thông qua việc quản lý các nhóm quyền và gán người dùng vào các nhóm phù hợp, đảm bảo nguyên tắc phân quyền tối thiểu và bảo mật dữ liệu.
    \item \textbf{Cấu hình thông tin và hoạt động chung của nhà hàng:} Thiết lập các thông tin định danh của nhà hàng (tên, địa chỉ, logo), đơn vị tiền tệ, và các tham số nghiệp vụ đặc thù ảnh hưởng đến hoạt động của các module khác (ví dụ: quy tắc đặt cọc cho module đặt chỗ).
    \item \textbf{Quản lý tích hợp với các dịch vụ bên thứ ba:} Cho phép cấu hình và quản lý kết nối API với các dịch vụ thiết yếu như máy chủ gửi email, cổng thanh toán trực tuyến, dịch vụ bot call, và dịch vụ quản lý giao hàng.
    \item \textbf{Cung cấp cơ chế xác thực an toàn:} Đảm bảo quy trình đăng nhập, đăng xuất và đặt lại mật khẩu được thực hiện một cách an toàn và hiệu quả.
    \item \textbf{Theo dõi và chẩn đoán hệ thống:} Cung cấp khả năng xem nhật ký hoạt động của hệ thống để hỗ trợ việc giám sát, chẩn đoán và khắc phục sự cố.
\end{itemize}
Phạm vi của module bao gồm toàn bộ vòng đời quản lý người dùng, quản lý phân quyền, cấu hình các thông số hệ thống cơ bản và các tích hợp quan trọng, cũng như các chức năng xác thực người dùng và theo dõi log hệ thống.

\subsubsubsection{Đối tượng Sử dụng Chính}
\label{sssec:md10_primary_users_full}
Module này phục vụ nhiều nhóm đối tượng người dùng với các vai trò khác nhau:
\begin{itemize}
    \item \textbf{US-10 (Quản trị viên Hệ thống):} Là người dùng có quyền hạn cao nhất, chịu trách nhiệm chính trong việc tạo và quản lý tài khoản người dùng (nhân viên), phân quyền, cấu hình các tích hợp kỹ thuật (máy chủ email, API), và xem log hệ thống.
    \item \textbf{US-01 (Quản lý nhà hàng):} Có thể được cấp quyền để cấu hình một số thông tin chung của nhà hàng, các tham số nghiệp vụ, và một số tích hợp dịch vụ (cổng thanh toán, bot call, giao hàng). Cũng có thể có quyền quản lý một số khía cạnh của tài khoản nhân viên.
    \item \textbf{US-01, US-02, US-03, US-04, US-05, US-06, US-07, US-09 (Tất cả Nhân viên có tài khoản):} Là người dùng của các chức năng xác thực cơ bản như Đăng nhập, Đăng xuất, và Yêu cầu Đặt lại Mật khẩu khi quên.
    \item \textbf{US-08 (Khách hàng):} Là người dùng của chức năng Tự Đăng ký Tài khoản trên giao diện web/app của nhà hàng.
\end{itemize}

\subsubsubsection{Các Chức năng Chính}
\label{sssec:md10_key_functionalities_full}
Module MD-10 cung cấp một bộ các chức năng quản trị và cấu hình hệ thống toàn diện, được mô tả chi tiết qua các Use Case sau:

\begin{itemize}
    \item \textbf{Quản lý Tài khoản Người dùng (Nhân viên) (UC-MD10-01 đến UC-MD10-07):}
    \begin{itemize}
        \item Tạo mới tài khoản đăng nhập cho nhân viên (UC-MD10-01).
        \item Xem danh sách tất cả các tài khoản người dùng trong hệ thống (UC-MD10-02).
        \item Sửa đổi thông tin cá nhân và liên hệ của một tài khoản người dùng (UC-MD10-03).
        \item Gán hoặc gỡ bỏ các Nhóm Quyền truy cập cho một người dùng để xác định phạm vi hoạt động (UC-MD10-04).
        \item Vô hiệu hóa (tạm khóa) khả năng đăng nhập của một tài khoản người dùng (UC-MD10-05).
        \item Kích hoạt lại một tài khoản người dùng đã bị vô hiệu hóa trước đó (UC-MD10-06).
        \item Quản trị viên hỗ trợ đặt lại mật khẩu cho một người dùng (UC-MD10-07).
    \end{itemize}

    \item \textbf{Quản lý Phân quyền (UC-MD10-08):}
    \begin{itemize}
        \item Xem chi tiết cấu hình và các quyền hạn cụ thể của một Nhóm Quyền truy cập đã tồn tại (UC-MD10-08).
    \end{itemize}

    \item \textbf{Cấu hình Hệ thống và Tích hợp (UC-MD10-09 đến UC-MD10-14):}
    \begin{itemize}
        \item Thiết lập các thông tin chung của nhà hàng/công ty như tên, địa chỉ, logo, tiền tệ (UC-MD10-09).
        \item Cấu hình thông tin máy chủ SMTP để hệ thống có thể gửi email (UC-MD10-10).
        \item Nhập API keys và các tham số để tích hợp với các cổng thanh toán trực tuyến (UC-MD10-11).
        \item Nhập API keys và các tham số vận hành để tích hợp với dịch vụ Bot Call (UC-MD10-12, liên quan FR-MD04-05).
        \item Nhập API keys và các tham số vận hành để tích hợp với dịch vụ quản lý giao hàng Shipday (UC-MD10-13, liên quan FR-MD07-15).
        \item Thiết lập các quy tắc và tham số nghiệp vụ đặc thù cho chức năng Đặt chỗ (ví dụ: tỷ lệ đặt cọc, giá trị bàn, số ngày gọi bot) (UC-MD10-14, liên quan FR-MD03-15 và FR-MD04-05).
    \end{itemize}

    \item \textbf{Theo dõi và Bảo trì Hệ thống (UC-MD10-15):}
    \begin{itemize}
        \item Cho phép Quản trị viên xem các bản ghi nhật ký hoạt động của hệ thống để theo dõi và chẩn đoán lỗi (UC-MD10-15).
    \end{itemize}

    \item \textbf{Xác thực Người dùng (UC-MD10-16 đến UC-MD10-19):}
    \begin{itemize}
        \item Cho phép tất cả người dùng có tài khoản (nhân viên, khách hàng) đăng nhập vào hệ thống bằng tên đăng nhập/email và mật khẩu (UC-MD10-16).
        \item Cho phép người dùng đã đăng nhập kết thúc phiên làm việc và thoát khỏi hệ thống một cách an toàn (UC-MD10-17).
        \item Cho phép người dùng tự yêu cầu hệ thống gửi hướng dẫn đặt lại mật khẩu qua email khi họ quên mật khẩu (UC-MD10-18).
        \item Cho phép khách hàng mới tự tạo tài khoản người dùng trên giao diện web/app của nhà hàng (UC-MD10-19).
    \end{itemize}
\end{itemize}

\subsubsubsection{Tóm tắt Luồng Hoạt động Tổng thể}
\label{sssec:md10_overall_workflow_full}
Luồng hoạt động trong module MD-10 rất đa dạng, phụ thuộc vào vai trò của người dùng và nhu cầu cụ thể. Một số luồng chính bao gồm:
\begin{enumerate}
    \item \textbf{Thiết lập hệ thống ban đầu hoặc khi có thay đổi lớn:}
        \begin{itemize}
            \item Quản trị viên/Quản lý Cấu hình Thông tin Chung của Nhà hàng (UC-MD10-09).
            \item Cấu hình Máy chủ Gửi Email (UC-MD10-10).
            \item Cấu hình Tích hợp Cổng Thanh toán (UC-MD10-11).
            \item Cấu hình Tích hợp Dịch vụ Bot Call (UC-MD10-12).
            \item Cấu hình Tích hợp Dịch vụ Giao hàng (Shipday) (UC-MD10-13).
            \item Cấu hình Tham số Nghiệp vụ Đặc thù cho Đặt chỗ (UC-MD10-14).
        \end{itemize}
    \item \textbf{Quản lý tài khoản nhân viên:}
        \begin{itemize}
            \item Quản trị viên Tạo mới Tài khoản Người dùng (Nhân viên) (UC-MD10-01).
            \item Gán/Gỡ bỏ Nhóm Quyền cho Người dùng (UC-MD10-04) (có thể tham khảo UC-MD10-08 để hiểu rõ nhóm quyền).
            \item Khi cần, Sửa Thông tin Tài khoản Người dùng (UC-MD10-03).
            \item Hỗ trợ Đặt lại Mật khẩu cho Người dùng (bởi Admin) (UC-MD10-07) nếu nhân viên yêu cầu.
            \item Vô hiệu hóa Tài khoản Người dùng (UC-MD10-05) khi nhân viên nghỉ việc, hoặc Kích hoạt lại Tài khoản Người dùng đã Vô hiệu hóa (UC-MD10-06) khi cần.
            \item Quản trị viên thường xuyên Xem Danh sách Người dùng (UC-MD10-02) để kiểm tra.
        \end{itemize}
    \item \textbf{Quy trình xác thực của tất cả người dùng:}
        \begin{itemize}
            \item Người dùng (nhân viên hoặc khách hàng đã đăng ký) Đăng nhập vào Hệ thống (UC-MD10-16).
            \item Nếu quên mật khẩu, người dùng Yêu cầu Đặt lại Mật khẩu (Quên Mật khẩu) (UC-MD10-18).
            \item Sau khi làm việc xong, người dùng Đăng xuất khỏi Hệ thống (UC-MD10-17).
        \end{itemize}
    \item \textbf{Khách hàng tự đăng ký (nếu có kênh online):}
        \begin{itemize}
            \item Khách hàng mới Tự Đăng ký Tài khoản (UC-MD10-19) trên website/app.
        \end{itemize}
    \item \textbf{Theo dõi và bảo trì hệ thống:}
        \begin{itemize}
            \item Quản trị viên Xem Nhật ký Hoạt động Hệ thống (Logs) (UC-MD10-15) để chẩn đoán sự cố hoặc theo dõi hoạt động.
        \end{itemize}
\end{enumerate}
Module MD-10 là xương sống đảm bảo cho toàn bộ hệ thống nhà hàng có thể vận hành một cách trơn tru, an toàn và tuân thủ các quy định, chính sách đã đặt ra.



\subsubsection{Module MD-11: Quản lý Quan hệ Khách hàng (CRM)}
Module Quản lý Quan hệ Khách hàng (MD-11) là một công cụ chiến lược giúp nhà hàng xây dựng và duy trì mối quan hệ bền chặt với khách hàng. Module này tập trung vào việc thu thập, lưu trữ, quản lý thông tin khách hàng, theo dõi lịch sử tương tác, phân loại khách hàng, quản lý các chương trình khuyến mãi/voucher, và thu thập phản hồi/đánh giá từ khách hàng. Mục tiêu cuối cùng là nâng cao sự hài lòng của khách hàng, tăng cường lòng trung thành và thúc đẩy doanh thu.



\begin{figure}[H]
    \centering
    \includegraphics[width=15cm]{Sections/tong_quan/functional_spec/img/uc11.png}
    \vspace{0.5cm}
    \caption{Use case diagram cho Module MD-11}
    \label{fig:my_label}
\end{figure}

\begin{longtable}{|m{2cm}|m{2.5cm}|m{2.5cm}|m{4.5cm}|m{4cm}|}
\caption{Danh sách Yêu cầu Chức năng cho Module MD-11: Quản lý Quan hệ Khách hàng (CRM)} \label{tab:fr_md11_crm_marketing_revised_in_codeblock} \\
\hline
\textbf{Mã Module} & \textbf{Mã Yêu cầu CN} & \textbf{Mã Người dùng} & \textbf{Tên Chức năng} & \textbf{Mô tả Ngắn} \\
\hline
\endhead % Header cho các trang tiếp theo
\midrule
\endfoot % Footer cho bảng
\bottomrule
\endlastfoot % Footer cho trang cuối cùng

% === Quản lý Hồ sơ Khách hàng (CRM View) - Tách nhỏ ===
MD-11 & FR-MD11-01 & US-01, US-03, US-10 & Tạo mới Hồ sơ Khách hàng (CRM) & Nhân viên/Quản lý tạo hồ sơ mới cho khách hàng với các thông tin chi tiết. \\
\hline
MD-11 & FR-MD11-02 & US-01, US-03, US-10 & Xem Danh sách Hồ sơ Khách hàng (CRM) & Xem danh sách tất cả khách hàng trong hệ thống CRM, có thể tìm kiếm, lọc. \\
\hline
MD-11 & FR-MD11-03 & US-01, US-03, US-10 & Xem Chi tiết Hồ sơ Khách hàng (CRM) & Xem toàn bộ thông tin chi tiết của một khách hàng cụ thể (thông tin cá nhân, lịch sử, sở thích...). \\
\hline
MD-11 & FR-MD11-04 & US-01, US-03, US-10 & Sửa Thông tin Hồ sơ Khách hàng (CRM) & Cập nhật, chỉnh sửa các thông tin trong hồ sơ của một khách hàng đã có. \\
\hline
MD-11 & FR-MD11-05 & US-01, US-10 & Xóa/Lưu trữ Hồ sơ Khách hàng (CRM) & Xóa (nếu chưa có giao dịch) hoặc lưu trữ (ẩn đi) hồ sơ khách hàng không còn hoạt động. \\
\hline
MD-11 & FR-MD11-06 & US-01, US-06, US-10 & Phân loại/Gắn thẻ Khách hàng & Phân loại khách hàng (VIP, thường xuyên...) hoặc gắn thẻ (tag) cho khách hàng để phục vụ marketing, chăm sóc. \\
\hline
MD-11 & FR-MD11-07 & US-01, US-06, US-10 & Xem Lịch sử Tương tác/Đặt chỗ của Khách hàng & Truy cập toàn bộ lịch sử đặt chỗ, hóa đơn, phản hồi, ghi chú tương tác của một khách hàng cụ thể. \\
\hline

% === Quản lý Voucher/Khuyến mãi ===
MD-11 & FR-MD11-08 & US-01, US-10 & Tạo mới Chương trình Khuyến mãi/Voucher & Định nghĩa các chương trình khuyến mãi (giảm giá \%, giảm tiền cố định, mua X tặng Y) hoặc tạo mã voucher. \\
\hline
MD-11 & FR-MD11-09 & US-01, US-10 & Thiết lập Điều kiện Áp dụng Khuyến mãi/Voucher & Cấu hình điều kiện: thời gian, giá trị đơn tối thiểu, sản phẩm/danh mục áp dụng, số lần sử dụng... \\
\hline
MD-11 & FR-MD11-10 & US-01, US-10 & Quản lý Danh sách Mã Voucher & Xem danh sách mã voucher đã tạo, trạng thái (đã dùng, còn hạn), có thể xuất mã hàng loạt, vô hiệu hóa voucher. \\
\hline
MD-11 & FR-MD11-11 & US-02, US-05 & Áp dụng Khuyến mãi/Voucher vào Đơn hàng POS & Nhân viên POS nhập mã voucher hoặc chọn chương trình khuyến mãi đủ điều kiện để áp dụng cho đơn hàng. \\
\hline
MD-11 & FR-MD11-12 & US-08 & Khách hàng Sử dụng Voucher khi Đặt chỗ Online & Khách hàng nhập mã voucher hợp lệ vào đơn đặt chỗ/đặt món online để được giảm giá. \\
\hline
MD-11 & FR-MD11-13 & US-01 & Xem Báo cáo Hiệu quả Khuyến mãi/Voucher & Thống kê số lần sử dụng, tổng giá trị giảm giá của từng chương trình/voucher. (Mở rộng từ UC-MD09-09) \\
\hline

% === Thu thập & Quản lý Đánh giá/Phản hồi ===
MD-11 & FR-MD11-14 & US-08 & Khách hàng Gửi Đánh giá/Review sau Khi sử dụng Dịch vụ & Khách hàng (có thể qua email mời hoặc link trên hóa đơn) gửi đánh giá về chất lượng món ăn, dịch vụ. \\
\hline
MD-11 & FR-MD11-15 & US-01, US-09 & Xem và Quản lý Đánh giá/Phản hồi của Khách hàng & Quản lý xem các đánh giá, có thể phản hồi, phân loại (tích cực, tiêu cực), hoặc đánh dấu đã xử lý. \\
\hline
MD-11 & FR-MD11-16 & US-01 & (Tùy chọn) Hiển thị Đánh giá Tích cực trên Website & Chọn lọc và hiển thị các đánh giá tốt của khách hàng trên trang web nhà hàng (nếu muốn). \\
\hline
% Các chức năng khác như Loyalty, Email Marketing sẽ được thêm vào sau nếu cần
\end{longtable}

\subsubsubsection{Mục tiêu và Phạm vi}
\label{sssec:md11_objectives_scope}
Mục tiêu chính của module MD-11 là:
\begin{itemize}
    \item \textbf{Xây dựng cơ sở dữ liệu khách hàng toàn diện:} Lưu trữ thông tin liên hệ, lịch sử giao dịch, sở thích và các ghi chú quan trọng về từng khách hàng.
    \item \textbf{Phân loại và phân khúc khách hàng:} Cho phép gắn thẻ, phân loại khách hàng (ví dụ: VIP, khách thường xuyên) để phục vụ các chiến dịch chăm sóc và marketing cá nhân hóa.
    \item \textbf{Quản lý chương trình khuyến mãi và voucher hiệu quả:} Tạo, cấu hình điều kiện áp dụng, theo dõi việc sử dụng và đánh giá hiệu quả của các chương trình khuyến mãi và mã voucher.
    \item \textbf{Thu thập và quản lý phản hồi của khách hàng:} Cung cấp kênh cho khách hàng gửi đánh giá và cho phép nhà hàng xem xét, phân tích các phản hồi này.
    \item \textbf{Tăng cường tương tác và chăm sóc khách hàng:} Cung cấp thông tin để nhân viên có thể phục vụ khách hàng tốt hơn và xây dựng mối quan hệ lâu dài.
    \item \textbf{Hỗ trợ các hoạt động marketing:} Cung cấp dữ liệu khách hàng cho việc tạo các chiến dịch email marketing hoặc các hoạt động quảng bá khác (có thể tích hợp với module marketing riêng).
\end{itemize}
Phạm vi của module bao gồm việc quản lý hồ sơ khách hàng từ khi tạo mới, cập nhật thông tin, theo dõi lịch sử, cho đến việc thiết kế và quản lý các chương trình khuyến mãi/voucher, cũng như quy trình thu thập và xem xét đánh giá từ khách hàng.

\subsubsubsection{Đối tượng Sử dụng Chính}
\label{sssec:md11_primary_users}
Các đối tượng người dùng chính tương tác với module này bao gồm:
\begin{itemize}
    \item \textbf{US-01 (Quản lý nhà hàng):} Là người dùng chính, chịu trách nhiệm tạo và quản lý các chương trình khuyến mãi, xem báo cáo hiệu quả, quản lý hồ sơ khách hàng VIP, và xem xét các đánh giá của khách hàng.
    \item \textbf{US-03 (Nhân viên lễ tân):} Thường xuyên tạo mới và cập nhật hồ sơ khách hàng, có thể xem lịch sử đặt chỗ của khách để hỗ trợ tốt hơn.
    \item \textbf{US-06 (Kế toán):} Có thể cần xem thông tin khách hàng liên quan đến hóa đơn hoặc các chương trình khách hàng thân thiết.
    \item \textbf{US-10 (Quản trị viên Hệ thống):} Có thể tham gia vào việc cấu hình ban đầu của module CRM, tạo các trường tùy chỉnh hoặc thiết lập các quy tắc tự động (nếu có).
    \item \textbf{US-08 (Khách hàng):} Là người cung cấp thông tin cá nhân, sử dụng voucher khi đặt chỗ online, và gửi đánh giá/review về dịch vụ.
    \item \textbf{US-02 (Nhân viên phục vụ) / US-05 (Nhân viên thu ngân):} Áp dụng các chương trình khuyến mãi hoặc mã voucher cho đơn hàng tại POS.
\end{itemize}

\subsubsubsection{Các Chức năng Chính}
\label{sssec:md11_key_functionalities}
Module MD-11 cung cấp một bộ các chức năng tập trung vào việc quản lý và tăng cường mối quan hệ với khách hàng:

\begin{itemize}
    \item \textbf{Quản lý Hồ sơ Khách hàng (UC-MD11-01 đến UC-MD11-07):}
    \begin{itemize}
        \item Tạo mới một hồ sơ khách hàng trong hệ thống CRM với các thông tin liên hệ cơ bản (UC-MD11-01).
        \item Xem danh sách tất cả các hồ sơ khách hàng với khả năng tìm kiếm và lọc (UC-MD11-02).
        \item Xem thông tin chi tiết đầy đủ của một hồ sơ khách hàng, bao gồm lịch sử giao dịch và tương tác (UC-MD11-03).
        \item Sửa đổi và cập nhật các thông tin trong một hồ sơ khách hàng đã tồn tại (UC-MD11-04).
        \item Xóa vĩnh viễn (nếu không có giao dịch) hoặc lưu trữ (ẩn đi) các hồ sơ khách hàng không còn phù hợp (UC-MD11-05).
        \item Gán các thẻ (tags) hoặc phân loại khách hàng (ví dụ: VIP, Thường xuyên) để phục vụ việc nhóm và phân tích (UC-MD11-06).
        \item Truy cập và xem lại toàn bộ lịch sử các hoạt động và giao dịch liên quan đến một khách hàng cụ thể (UC-MD11-07).
    \end{itemize}

    \item \textbf{Quản lý Chương trình Khuyến mãi và Voucher (UC-MD11-08 đến UC-MD11-13):}
    \begin{itemize}
        \item Định nghĩa một chương trình khuyến mãi mới hoặc một lô mã voucher mới, bao gồm loại hình và giá trị khuyến mãi (UC-MD11-08).
        \item Thiết lập các điều kiện và quy tắc chi tiết để chương trình/voucher có thể được áp dụng (ví dụ: thời gian hiệu lực, giá trị đơn hàng tối thiểu, sản phẩm áp dụng) (UC-MD11-09).
        \item Xem danh sách các mã voucher đã tạo, kiểm tra trạng thái, và có thể thực hiện xuất file hoặc vô hiệu hóa mã (UC-MD11-10).
        \item Nhân viên tại POS áp dụng một chương trình khuyến mãi hoặc nhập mã voucher hợp lệ vào đơn hàng của khách (UC-MD11-11).
        \item Khách hàng tự nhập và sử dụng mã voucher hợp lệ khi thực hiện đặt chỗ hoặc đặt món trước trực tuyến (UC-MD11-12).
        \item Xem báo cáo thống kê chi tiết về tình hình sử dụng và hiệu quả của các chương trình khuyến mãi hoặc voucher (UC-MD11-13).
    \end{itemize}

    \item \textbf{Quản lý Đánh giá và Phản hồi từ Khách hàng (UC-MD11-14, và sẽ có các UC tiếp theo như UC-MD11-15, UC-MD11-16):}
    \begin{itemize}
        \item Cho phép khách hàng gửi các ý kiến đánh giá, nhận xét, hoặc xếp hạng về dịch vụ của nhà hàng thông qua các kênh trực tuyến (UC-MD11-14).
        \item (Dự kiến) Cho phép Quản lý nhà hàng xem danh sách các đánh giá đã nhận được.
        \item (Dự kiến) Cho phép Quản lý nhà hàng xem chi tiết một đánh giá cụ thể và có thể thực hiện các hành động phản hồi.
    \end{itemize}
\end{itemize}

\subsubsubsection{Tóm tắt Luồng Hoạt động Tổng thể}
\label{sssec:md11_overall_workflow}
Luồng hoạt động trong module Quản lý Quan hệ Khách hàng (CRM) thường bao gồm các giai đoạn và quy trình sau:
\begin{enumerate}
    \item \textbf{Thu thập và Quản lý Thông tin Khách hàng:}
        \begin{itemize}
            \item Nhân viên Tạo mới Hồ sơ Khách hàng (CRM) (UC-MD11-01) khi có khách mới hoặc nhập dữ liệu.
            \item Nhân viên thường xuyên Xem Danh sách Hồ sơ Khách hàng (UC-MD11-02), Xem Chi tiết Hồ sơ Khách hàng (UC-MD11-03) để tra cứu, và Sửa Thông tin Hồ sơ Khách hàng (UC-MD11-04) khi cần cập nhật.
            \item Thực hiện Phân loại/Gắn thẻ Khách hàng (UC-MD11-06) để phục vụ các mục đích khác nhau.
            \item Khi cần, Xóa/Lưu trữ Hồ sơ Khách hàng (CRM) (UC-MD11-05).
            \item Xem Lịch sử Tương tác/Đặt chỗ của Khách hàng (UC-MD11-07) để hiểu rõ hơn về khách.
        \end{itemize}
    \item \textbf{Triển khai và Quản lý Chương trình Khuyến mãi/Voucher:}
        \begin{itemize}
            \item Quản lý Tạo mới Chương trình Khuyến mãi/Voucher (UC-MD11-08).
            \item Thiết lập Điều kiện Áp dụng Khuyến mãi/Voucher (UC-MD11-09) chi tiết cho từng chương trình.
            \item (Nếu là voucher) Quản lý Danh sách Mã Voucher (UC-MD11-10), có thể xuất file hoặc vô hiệu hóa mã.
            \item Nhân viên tại POS Áp dụng Khuyến mãi/Voucher vào Đơn hàng POS (UC-MD11-11) cho khách.
            \item Khách hàng Sử dụng Voucher khi Đặt chỗ Online (UC-MD11-12) trên website/app.
            \item Quản lý định kỳ Xem Báo cáo Hiệu quả Khuyến mãi/Voucher (UC-MD11-13) để đánh giá.
        \end{itemize}
    \item \textbf{Thu thập và Xử lý Phản hồi Khách hàng:}
        \begin{itemize}
            \item Khách hàng Gửi Đánh giá/Review sau Khi sử dụng Dịch vụ (UC-MD11-14).
            \item (Dự kiến) Quản lý xem xét các đánh giá này và có thể phản hồi hoặc thực hiện các hành động cải thiện dịch vụ.
        \end{itemize}
\end{enumerate}
Module MD-11 giúp nhà hàng không chỉ quản lý giao dịch mà còn xây dựng mối quan hệ ý nghĩa với khách hàng, từ đó tạo lợi thế cạnh tranh và phát triển bền vững.




% \subsection{Yêu cầu chất lương}

% % \subsubsection{Yêu cầu chức năng}

% Trong phạm vi đồ án này, chúng em sẽ tập trung vào các đối tượng chính sử dụng hệ thống, bao gồm Khách hàng, Nhân viên phục vụ, Nhân viên thu ngân, Nhân viên chăm sóc khách hàng, Nhân viên bếp, Quản lý chi nhánh và Quản lý tổng, nhằm đảm bảo hệ thống đáp ứng tốt nhu cầu vận hành và trải nghiệm của từng vai trò.

% \begin{figure}[H]
%     \centering
%     \includegraphics[width=15cm]{Images/nguoi-dung-he-thong.png}
%     \vspace{0.5cm}
%     \caption{Các đối tượng người dùng của hệ thống}
%     \label{fig:my_label}
% \end{figure}

% \textbf{Đối với Khách hàng}
% \begin{itemize}
%     \item Đăng ký, đăng nhập
%     \item Gửi yêu cầu đặt bàn trực tuyến, xem được tổng quan vị trí, trạng thái bàn thông qua các sơ đồ
%     \item Hủy yêu cầu đặt bàn cho đến khi trước thời gian hẹn 2 tiếng
%     \item Quét QR để truy cập vào menu và đặt món, không cần gọi nhân viên
%     \item Xem danh sách món ăn và đồ uống, kèm theo hình ảnh, mô tả và giá cả
%     \item Tìm kiếm món ăn theo tên hoặc danh mục, giá cả
%     \item Chọn món, số lượng và tùy chọn (size, topping, gia vị)
%     \item Gợi ý món ăn phù hợp dựa trên lịch sử đơn hàng và số liệu phân tích
%     \item Thêm món vào giỏ hàng
%     \item Xem lại giỏ hàng trước khi xác nhận đặt món
%     \item Đặt món nhiều lần trong cùng một lượt sử dụng tại nhà hàng, để có thể thay đổi món ăn trong suốt quá trình ăn
%     \item Gửi yêu cầu hủy đơn hoặc hủy/thay đổi các món cụ thể
%     \item Xem lại các đơn hàng đã đặt trước đó
%     \item Xem chi tiết các món đã đặt, tổng tiền và các khuyến mãi (nếu có)
%     \item Gửi yêu cầu thanh toán bằng cách quét mã QR, hoặc thao tác trên ứng dụng
%     \item Nhận thông báo về các khuyến mãi và ưu đãi
%     \item Gửi phản hồi về món ăn hoặc dịch vụ
%     \item Gửi khiếu nại nếu có sự cố
%     \item Trò chuyện trực tiếp với nhân viên chăm sóc khách hàng
% \end{itemize}

% \textbf{Đối với Phục vụ}
% \begin{itemize}
%     \item Xem danh sách trạng thái bàn
%     \item Hỗ trợ khách hàng đặt món, thanh toán
%     \item Xem danh sách đơn hàng của khách
%     \item Nhận thông báo khi món ăn đã sẵn sàng
%     \item Xử lý yêu cầu hủy đơn hoặc hủy/thay đổi món của khách
%     \item Chuyển vị trí đơn hàng sang bàn khác.
%     \item Đặt lại trạng thái bàn
% \end{itemize}

% \textbf{Đối với Thu ngân}
% \begin{itemize}
%     \item Nhận thông báo khi có yêu cầu thanh toán
%     \item Xem chi tiết đơn hàng và tổng hóa đơn
%     \item Áp dụng khuyến mãi và giảm giá khi thanh toán
%     \item Quản lý các phương thức thanh toán
%     \item Nhận thông báo về thanh toán thành công với phương thức quét QR, quẹt thẻ
%     \item Cập nhật trạng thái thanh toán với phương thức thanh toán tiền mặt
%     \item In hóa đơn cho khách hàng
%     \item Xử lý yêu cầu hoàn tiền (vấn đề phát sinh)
%     \item Tạo báo cáo doanh thu hàng ngày
% \end{itemize}

% \textbf{Đối với Nhân viên chăm sóc khách hàng}
% \begin{itemize}
%     \item Tiếp nhận và xử lý yêu cầu từ khách hàng
%     \item Theo dõi và quản lý khiếu nại \& phản hồi từ khách hàng
%     \item Trả lời tin nhắn trực tuyến với khách hàng
%     \item Cập nhật thông tin khách hàng
%     \item Theo dõi và nhắc nhở khách hàng về các chương trình ưu đãi qua tài khoản và email
% \end{itemize}

% \textbf{Đối với Nhân viên bếp}
% \begin{itemize}
%     \item Nhận đơn hàng từ hệ thống quản lý đơn hàng
%     \item Xem tất cả danh sách món ăn được sắp xếp theo thứ tự ưu tiên
%     \item Xem được các yêu cầu đặc biệt của từng món ăn cần làm
%     \item Cập nhật trạng thái chế biến của từng món ăn (chưa làm, đang chế biến, hoàn thành...)
% \end{itemize}

% \textbf{Đối với Quản lý chi nhánh}
% \begin{itemize}
%     \item Thêm/Chỉnh sửa sơ đồ nhà hàng
%     \item Xử lý, xác nhận các yêu cầu đặt bàn của khách hàng tại chi nhánh
%     \item Xem báo cáo kinh doanh và doanh thu của chi nhánh
%     \item Điều chỉnh và thiết lập các chương trình khuyến mãi, marketing chi nhánh
%     \item Theo dõi các phản hồi của khách hàng
%     \item Xem danh sách nhân viên
%     \item Phân chia công việc cho các tài khoản nhân viên
% \end{itemize}

% \textbf{Đối với Quản lý tổng}
% \begin{itemize}
%     \item Quản lý thông tin chi nhánh
%     \item Thêm/Xóa chi nhánh
%     \item Quản lý các tài khoản nhân viên và khách hàng
%     \item Xem báo cáo tổng quan tất cả các chi nhánh
%     \item Xem báo cáo chi tiết của một chi nhánh tổng
% \end{itemize}

% % User Story là một kỹ thuật phát triển phần mềm tập trung vào nhu cầu của người dùng trong quá trình sử dụng sản phẩm. Mục đích của User Story là giúp các nhà phát triển phần mềm hiểu rõ những tính năng cốt lõi của sản phẩm và xác định được các chức năng cần thiết để hiện thực đầu tiên của ứng dụng.\\

% % Để đưa ra danh sách User Story của ứng dụng, nhóm chúng em đã tiến hành thảo luận, nghiên cứu và phân tích yêu cầu của người dùng. Chúng tôi bắt đầu xây dựng hệ thống tuyển dụng bằng cách xác định các tính năng cơ bản cần phải có trong ứng dụng để đáp ứng nhu cầu và mong muốn của người dùng.\\

% % \textbf{Một số user story cơ bản như sau:}
% % \begin{itemize}
% %     \item Đối với Khách hàng
% %     \begin{enumerate}
% %         \item Là Khách Hàng, tôi muốn quét mã QR trên bàn để truy cập vào menu và đặt món.
% %         \item Là Khách Hàng, tôi muốn xem danh sách các món ăn và đồ uống, kèm theo hình ảnh, mô tả và giá cả.
% %         \item Là Khách Hàng, tôi muốn tìm kiếm món ăn theo tên hoặc danh mục.
% %         \item Là Khách Hàng, tôi muốn chọn món, số lượng và các tùy chọn (ví dụ: size, topping, gia vị).
% %         \item Là Khách Hàng, tôi muốn thêm các món đã chọn vào giỏ hàng.
% %         \item Là Khách Hàng, tôi muốn xem lại giỏ hàng trước khi xác nhận đặt món.
% %         \item Là Khách Hàng, tôi muốn đặt món nhiều lần trong cùng một lượt sử dụng tại nhà hàng.
% %         \item Là Khách Hàng, tôi muốn xem lại các order đã đặt trước đó (nếu đã đăng nhập).
% %         \item Là Khách Hàng, tôi muốn xem chi tiết các món đã đặt, tổng tiền và các khuyến mãi (nếu có).
% %         \item Là Khách Hàng, tôi muốn thanh toán bằng cách quét mã QR hoặc thanh toán tiền mặt.
% %         \item Là Khách Hàng, tôi muốn xem lại hóa đơn sau khi đã thanh toán.
% %         \item Là Khách Hàng, tôi muốn đăng ký tài khoản thành viên để tham gia chương trình khách hàng thân thiết.
% %         \item Là Khách Hàng, tôi muốn đăng nhập để xem lịch sử đặt món, nhận khuyến mãi và các ưu đãi khác.
% %     \end{enumerate}
% %     \item Đối với Nhân Viên Thu Ngân/Phục Vụ
% %     \begin{enumerate}
% %         \item Là Nhân Viên Thu Ngân/Phục Vụ, tôi muốn xem danh sách các order mới và đang chờ xử lý.
% %         \item Là Nhân Viên Thu Ngân/Phục Vụ, tôi muốn xem chi tiết các order của khách hàng.
% %         \item Là Nhân Viên Thu Ngân/Phục Vụ, tôi muốn gộp các order của một bàn thành một hóa đơn duy nhất.
% %         \item Là Nhân Viên Thu Ngân/Phục Vụ, tôi muốn xóa bỏ các order nếu cần.
% %         \item Là Nhân Viên Thu Ngân/Phục Vụ, tôi muốn xác nhận thanh toán bằng QR.
% %         \item Là Nhân Viên Thu Ngân/Phục Vụ, tôi muốn xác nhận thanh toán bằng tiền mặt (tự thao tác).
% %         \item Là Nhân Viên Thu Ngân/Phục Vụ, tôi muốn in hóa đơn cho khách hàng (có mã QR để thanh toán).
% %         \item Là Nhân Viên Thu Ngân/Phục Vụ, tôi muốn check-in (chấm công) đầu ca và check-out (chấm công) cuối ca.
% %         \item Là Nhân Viên Thu Ngân/Phục Vụ, tôi muốn chuyển order từ bàn này sang bàn khác (nếu khách hàng muốn đổi bàn).
% %     \end{enumerate}
% %     \item Đối với Quản Lý Chi Nhánh
% %     \begin{enumerate}
% %         \item Là Quản Lý Chi Nhánh, tôi muốn nhập kho, xem danh sách các nguyên liệu trong kho.
% %         \item Là Quản Lý Chi Nhánh, tôi muốn theo dõi số lượng còn lại của từng nguyên liệu.
% %         \item Là Quản Lý Chi Nhánh, tôi muốn xem báo cáo nhập/xuất kho hàng ngày.
% %         \item Là Quản Lý Chi Nhánh, tôi muốn xem danh sách nhân viên.
% %         \item Là Quản Lý Chi Nhánh, tôi muốn phân công ca làm cho nhân viên.
% %         \item Là Quản Lý Chi Nhánh, tôi muốn xem lịch sử check-in/check-out của nhân viên.
% %         \item Là Quản Lý Chi Nhánh, tôi muốn xem các báo cáo về doanh thu và số lượng món ăn bán được trong chi nhánh.
% %         \item Là Quản Lý Chi Nhánh, tôi muốn xem các báo cáo về kho nguyên liệu.
% %         \item Là Quản Lý Chi Nhánh, tôi muốn xem danh sách bàn đang có.
% %         \item Là Quản Lý Chi Nhánh, tôi muốn sắp xếp bàn cho khách hàng và quản lý bàn trống.
% %     \end{enumerate}
% %     \item Đối với Quản Lý Tổng
% %     \begin{enumerate}
% %         \item Là Quản Lý Tổng, tôi muốn xem tổng doanh thu của tất cả các chi nhánh.
% %         \item Là Quản Lý Tổng, tôi muốn xem chi tiết doanh thu của từng chi nhánh, theo ngày, tuần, tháng, năm.
% %         \item Là Quản Lý Tổng, tôi muốn xem các báo cáo tổng quan về hoạt động của nhà hàng.
% %         \item Là Quản Lý Tổng, tôi muốn xuất báo cáo tổng quan về doanh thu, chi phí và lợi nhuận.
% %         \item Là Quản Lý Tổng, tôi muốn xem danh sách tất cả các chi nhánh.
% %         \item Là Quản Lý Tổng, tôi muốn thêm, sửa hoặc xóa thông tin chi nhánh.
% %         \item Là Quản Lý Tổng, tôi muốn xem các báo cáo tổng quan từ các chi nhánh.
% %         \item Là Quản Lý Tổng, tôi muốn thêm, chỉnh sửa hoặc xóa các món ăn và đồ uống trong thực đơn.
% %         \item Là Quản Lý Tổng, tôi muốn thay đổi giá cả, mô tả và hình ảnh của món ăn.
% %     \end{enumerate}
% %     \item Các Tính Năng Mở Rộng (Optional)
% %     \begin{enumerate}
% %         \item Là Khách Hàng, tôi muốn hệ thống gợi ý món ăn dựa trên lịch sử đặt món của mình.
% %         \item Là Quản Lý (Tổng/Chi Nhánh), tôi muốn tạo và quản lý các chương trình khuyến mãi, giảm giá.
% %         \item Là Khách Hàng, tôi muốn đặt bàn trước thông qua ứng dụng.
% %         \item Là Khách Hàng, tôi muốn đánh giá chất lượng món ăn và dịch vụ của nhà hàng.
% %         \item Là Quản Lý (Tổng/Chi Nhánh), tôi muốn hệ thống tự động xuất dữ liệu sang hệ thống kế toán.
% %     \end{enumerate}
% % \end{itemize}

% % \textbf{Một số usecase diagram cơ bản như sau:}

% % \begin{figure}[H]
% %     \centering
% %     \includegraphics[width=15cm]{Images/us-dat-mon.png}
% %     \vspace{0.5cm}
% %     \caption{Các use case liên quan đến đặt món}
% %     \label{fig:my_label}
% % \end{figure}

% % \begin{figure}[H]
% %     \centering
% %     \includegraphics[width=15cm]{Images/us-tai-khoan.png}
% %     \vspace{0.5cm}
% %     \caption{Các use case liên quan đến quản lý tài khoản}
% %     \label{fig:my_label}
% % \end{figure}

% % \begin{figure}[H]
% %     \centering
% %     \includegraphics[width=15cm]{Images/us-quan-ly.png}
% %     \vspace{0.5cm}
% %     \caption{Các use case liên quan đến quản lý}
% %     \label{fig:my_label}
% % \end{figure}

% % \begin{enumerate}[(a)]
% %     \item Khách hàng
% %     \begin{itemize}
% %         \item Quét mã QR
% %         \begin{itemize}
% %             \item Khách hàng quét mã QR trên bàn để truy cập vào menu và đặt món.
% %         \end{itemize}
% %         \item Xem menu
% %         \begin{itemize}
% %             \item Xem danh sách các món ăn và đồ uống, kèm theo hình ảnh, mô tả, giá cả.
% %             \item Tìm kiếm món ăn theo tên hoặc danh mục.
% %         \end{itemize}
% %         \item Đặt món
% %         \begin{itemize}
% %             \item Chọn món, số lượng, tùy chọn (ví dụ: size, topping, gia vị)
% %             \item Thêm món vào giỏ hàng.
% %             \item Xem lại giỏ hàng trước khi xác nhận đặt món.
% %             \item Đặt món nhiều lần trong cùng một lượt sử dụng.
% %         \end{itemize}
% %         \item Xem lịch sử đặt món
% %         \begin{itemize}
% %             \item Xem lại các order đã đặt trước đó (nếu đã đăng nhập).
% %         \end{itemize}
% %         \item Xem hóa đơn
% %         \begin{itemize}
% %             \item Xem chi tiết các món đã đặt, tổng tiền, các khuyến mãi (nếu có).
% %         \end{itemize}
% %         \item Thanh toán
% %         \begin{itemize}
% %             \item Thanh toán bằng cách quét mã QR hoặc thanh toán tiền mặt.
% %             \item Xem lại hóa đơn sau khi đã thanh toán.
% %         \end{itemize}
% %         \item Đăng ký/Đăng nhập (Tùy chọn)
% %         \begin{itemize}
% %             \item Đăng ký tài khoản thành viên để tham gia chương trình khách hàng thân thiết.
% %             \item Đăng nhập để xem lịch sử đặt món, nhận khuyến mãi và các ưu đãi khác.
% %         \end{itemize}
% %     \end{itemize}
% %     \item Nhân viên thu ngân/phục vụ
% %     \begin{itemize}
% %         \item Quản lý order
% %         \begin{itemize}
% %             \item Xem danh sách các order mới và đang chờ xử lý.
% %             \item Xem chi tiết các order của khách hàng.
% %         \end{itemize}
% %         \item Gộp hóa đơn
% %         \begin{itemize}
% %             \item Gộp các order của một bàn thành một hóa đơn duy nhất.
% %             \item Xóa bỏ các order (nếu cần)
% %         \end{itemize}
% %         \item Xác nhận thanh toán
% %         \begin{itemize}
% %             \item Xác nhận thanh toán bằng QR.
% %             \item Xác nhận thanh toán bằng tiền mặt (tự thao tác).
% %         \end{itemize}
% %         \item In hóa đơn
% %         \begin{itemize}
% %             \item In hóa đơn cho khách hàng (có mã QR để thanh toán).
% %         \end{itemize}
% %         \item Check-in/Check-out
% %         \begin{itemize}
% %             \item Chấm công đầu ca và cuối ca.
% %         \end{itemize}
% %         \item Chuyển Order
% %         \begin{itemize}
% %             \item Có thể chuyển order từ bàn này sang bàn khác (nếu khách hàng muốn đổi bàn)
% %         \end{itemize}
% %     \end{itemize}
% %     \item Chức năng cho quản lý chi nhánh
% %     \begin{itemize}
% %         \item Quản lý kho nguyên liệu
% %         \begin{itemize}
% %             \item Nhập kho, xem danh sách các nguyên liệu trong kho.
% %             \item Theo dõi số lượng còn lại của từng nguyên liệu.
% %             \item Báo cáo nhập/xuất kho hàng ngày.
% %         \end{itemize}
% %         \item Quản lý nhân viên
% %         \begin{itemize}
% %             \item Xem danh sách nhân viên.
% %             \item Phân công ca làm.
% %             \item Xem lịch sử check-in/check-out của nhân viên.
% %         \end{itemize}
% %         \item Xem báo cáo
% %         \begin{itemize}
% %             \item Xem các báo cáo về doanh thu, số lượng món ăn bán được trong chi nhánh.
% %             \item Xem các báo cáo về kho nguyên liệu.
% %         \end{itemize}
% %         \item Quản lý bàn
% %         \begin{itemize}
% %             \item Xem danh sách bàn đang có.
% %             \item Có thể sắp xếp bàn cho khách hàng, quản lý bàn trống.
% %         \end{itemize}
% %     \end{itemize}
% %     \item Chức năng cho quản lý tổng
% %     \begin{itemize}
% %         \item Quản lý doanh thu
% %         \begin{itemize}
% %             \item Xem tổng doanh thu của tất cả các chi nhánh.
% %             \item Xem chi tiết doanh thu của từng chi nhánh, theo ngày, tuần, tháng, năm.
% %         \end{itemize}
% %         \item Xem báo cáo
% %         \begin{itemize}
% %             \item Xem các báo cáo tổng quan về hoạt động của nhà hàng.
% %             \item Xuất báo cáo tổng quan về doanh thu, chi phí, lợi nhuận.
% %         \end{itemize}
% %         \item Quản lý chi nhánh
% %         \begin{itemize}
% %             \item Xem danh sách tất cả các chi nhánh.
% %             \item Thêm/sửa/xóa thông tin chi nhánh.
% %             \item Xem các báo cáo tổng quan từ các chi nhánh.
% %         \end{itemize}
% %         \item Quản lý thực đơn
% %         \begin{itemize}
% %             \item Thêm, chỉnh sửa, xóa các món ăn và đồ uống trong thực đơn.
% %             \item Thay đổi giá cả, mô tả, hình ảnh của món ăn.
% %         \end{itemize}
% %     \end{itemize}
% %     \item Mở rộng (có thể làm nếu kịp thời gian)
% %     \begin{itemize}
% %         \item Hệ thống gợi ý món ăn: Dựa trên lịch sử đặt món của khách hàng, hệ thống có thể gợi ý các món ăn phù hợp.
% %         \item Hệ thống quản lý khuyến mãi: Cho phép tạo và quản lý các chương trình khuyến mãi, giảm giá.
% %         \item Hệ thống đặt bàn: Cho phép khách hàng đặt bàn trước thông qua ứng dụng.
% %         \item Hệ thống quản lý đánh giá: Cho phép khách hàng đánh giá chất lượng món ăn và dịch vụ của nhà hàng.
% %         \item Tích hợp với hệ thống kế toán: Tự động xuất dữ liệu sang hệ thống kế toán.
% %     \end{itemize}
% % \end{enumerate}

% \subsubsection{Yêu cầu phi chức năng}
% \begin{itemize}
%     \item Bảo mật:
%     \begin{itemize}
%         \item Dữ liệu người dùng và dữ liệu giao dịch phải được bảo vệ.
%         \item Hệ thống sử dụng HTTPS để đảm bảo an toàn cho quá trình truyền dữ liệu.
%     \end{itemize}

%     \item Hiệu năng:
%     \begin{itemize}
%         \item Hệ thống phải có tốc độ xử lý nhanh và ổn định.
%         \item Khả năng đáp ứng nhanh chóng khi nhiều người dùng truy cập cùng một lúc.    
%     \end{itemize}

%     \item Tính khả dụng:
%     \begin{itemize}
%         \item Hệ thống hoạt động ổn định và có thời gian uptime cao.
%         \item Giao diện thân thiện và dễ sử dụng trên cả web và mobile.
%     \end{itemize}

%     \item Khả năng mở rộng:
%     \begin{itemize}
%         \item Hệ thống có thể mở rộng để hỗ trợ thêm nhiều chi nhánh và người dùng.
%         \item Dễ dàng thêm các tính năng mới khi cần thiết.
%     \end{itemize}

% \end{itemize}












% \newpage
\section{THIẾT KẾ}
\subsection{Công nghệ sử dụng}
\subsubsection{Công nghệ Front-end}

\subsubsubsection{ReactJS}
    \begin{enumerate}[(a)]
        \item \textit{Giới thiệu}
        
            ReactJS là một thư viện JavaScript mã nguồn mở, được phát triển bởi Facebook, nhằm hỗ trợ xây dựng giao diện người dùng (UI) cho các ứng dụng web. Ra mắt vào năm 2013, ReactJS nhanh chóng trở thành một trong những công cụ phổ biến nhất cho việc phát triển UI tương tác và hiệu quả.
            
            % Các đặc điểm nổi bật của ReactJS:
    
            % \begin{itemize}
            %     \item \textbf{Kiến trúc dựa trên thành phần (Component-Based Architecture)}: React cho phép chia UI thành các thành phần nhỏ, độc lập và có thể tái sử dụng. Mỗi thành phần quản lý trạng thái và logic riêng, giúp việc phát triển và bảo trì ứng dụng trở nên dễ dàng hơn.
            %     \item \textbf{Virtual DOM}: React sử dụng một bản sao ảo của DOM thật, gọi là Virtual DOM. Khi có sự thay đổi trong trạng thái hoặc dữ liệu, React cập nhật Virtual DOM trước, sau đó so sánh với DOM thật và chỉ thực hiện những thay đổi cần thiết. Điều này cải thiện hiệu suất và tăng tốc độ render của ứng dụng.
            %     \item \textbf{JSX (JavaScript XML)}: React sử dụng JSX, một cú pháp mở rộng cho phép viết mã HTML trong JavaScript. Điều này giúp code trở nên dễ đọc và dễ hiểu hơn, đồng thời tối ưu hóa quá trình phát triển UI.
            %     \item \textbf{One-way Data Binding}: React áp dụng cơ chế liên kết dữ liệu một chiều, giúp luồng dữ liệu trở nên rõ ràng và dễ kiểm soát. Dữ liệu được truyền từ component cha đến component con thông qua props, giúp việc quản lý và debug ứng dụng trở nên hiệu quả hơn. 
            % \end{itemize}

        
        \item \textit{Ưu điểm} \cite{React}.

        \begin{itemize}
            \item  \textbf{Khai báo giao diện (Declarative)}: React giúp tạo ra các giao diện người dùng tương tác một cách dễ dàng. Bằng cách thiết kế các khung nhìn cho từng trạng thái trong ứng dụng, React sẽ tự động cập nhật và render các thành phần phù hợp khi dữ liệu thay đổi. Việc khai báo giao diện một cách tường minh giúp mã nguồn dễ hiểu và dễ dàng gỡ lỗi hơn.
            \item \textbf{Component-Based}: React cho phép xây dựng các thành phần độc lập và quản lý trạng thái riêng biệt của chúng. Những thành phần này có thể được kết hợp để tạo ra các giao diện phức tạp. Việc viết logic của thành phần bằng JavaScript thay vì template giúp dễ dàng truyền dữ liệu và tránh thao tác trực tiếp với DOM.
            \item \textbf{Học một lần, viết ở mọi nơi (Learn Once, Write Anywhere)}: React không phụ thuộc vào kỹ năng công nghệ cụ thể, cho phép phát triển các tính năng mới mà không cần viết lại mã hiện có. React có thể render trên máy chủ bằng Node.js và xây dựng ứng dụng di động thông qua React Native 
        \end{itemize}

        \item \textit{So sánh React và các công nghệ khác} 
        
            Khi so sánh giữa các công nghệ phổ biến hiện nay như ReactJS, Angular, Vue.js và Svelte, chúng ta có thể thấy những đặc điểm riêng biệt của từng công nghệ, giúp lựa chọn được công nghệ phù hợp với nhu cầu và yêu cầu của dự án. Dựa theo tài liệu \cite{FrontCompare1} và \cite{FrontCompare2}, dưới đây là bảng so sánh chi tiết giữa ReactJS và các công nghệ khác:

            % \begin{longtable}{|p{3.5cm}|p{5.5cm}|p{5.5cm}|}
            % \hline
            % \textbf{Tiêu Chí} & \textbf{ReactJS} & \textbf{Angular} \\ 
            % \hline
            % \endfirsthead
            % \hline
            % \textbf{Tiêu Chí} & \textbf{ReactJS} & \textbf{Angular} \\ 
            % \endhead
            % \hline
            % % \multicolumn{3}{|r|}{\small\slshape Còn tiếp} \\ \hline
            % \endfoot
            % \hline
            % \endlastfoot
            % Ngôn ngữ & JavaScript (hỗ trợ TypeScript)     & TypeScript (bắt buộc)\\ 
            % \hline
            % Định nghĩa & Thư viện JavaScript để xây dựng giao diện người dùng & Framework toàn diện để phát triển ứng dụng web\\ 
            % \hline
            % Kiến trúc & Dựa trên thành phần, Virtual DOM, luồng dữ liệu một chiều & Dựa trên thành phần, MVC/MVVM, cơ chế phát hiện thay đổi\\ 
            % \hline
            % Thời gian phát triển & Nhanh cho dự án nhỏ, chậm hơn nếu cần tích hợp nhiều thư viện & Chậm hơn do cấu hình ban đầu phức tạp, nhanh cho dự án lớn nhờ tích hợp sẵn\\ 
            % \hline
            % Độ phức tạp & Trung bình, phụ thuộc vào cách quản lý trạng thái & Cao, nhiều khái niệm (Dependency Injection, RxJS, Modules)\\ 
            % \hline
            % Tính linh hoạt & Cao, tự do chọn thư viện và công cụ & Thấp hơn, bị ràng buộc bởi cấu trúc framework\\ 
            % \hline
            % Bảo trì & Dễ bảo trì với các thành phần nhỏ & Dễ bảo trì cho dự án lớn, khó hơn với dự án nhỏ do cấu trúc phức tạp\\ 
            % \hline
            % Dễ sử dụng/Dễ học & Dễ học với người biết JavaScript & Khó học, cần hiểu TypeScript và các khái niệm framework\\ 
            % \hline
            % Hỗ trợ cộng đồng & Rất lớn, nhiều tài liệu, thư viện và diễn đàn & Lớn, hỗ trợ chính thức từ Google, tài liệu chi tiết\\ 
            % \hline
            % Kiểm thử & Jest, React Testing Library, dễ thiết lập & Jasmine, Karma, tích hợp sẵn nhưng phức tạp hơn\\ 
            % \hline
            % \caption{Bảng so sánh ReactJS và Angular}\\
            % \end{longtable}

            % \begin{longtable}{|p{3.5cm}|p{5.5cm}|p{5.5cm}|}
            % \hline
            % \textbf{Tiêu Chí} & \textbf{Svelt} & \textbf{Vue.js} \\ 
            % \hline
            % \endfirsthead
            % \hline
            % \textbf{Tiêu Chí} & \textbf{Svelt} & \textbf{Vue.js} \\ 
            % \endhead
            % \hline
            % % \multicolumn{3}{|r|}{\small\slshape Còn tiếp} \\ \hline
            % \endfoot
            % \hline
            % \endlastfoot
            % Ngôn ngữ & JavaScript (hỗ trợ TypeScript)     & JavaScript (hỗ trợ TypeScript)\\ 
            % \hline
            % Định nghĩa & Framework biên dịch, chuyển mã thành Vanilla JS & Framework tiến bộ để xây dựng giao diện người dùng\\ 
            % \hline
            % Kiến trúc & Dựa trên thành phần, hệ thống phản ứng tích hợp sẵn, không dùng Virtual DOM & Dựa trên thành phần, hệ thống phản ứng, Virtual DOM\\ 
            % \hline
            % Thời gian phát triển & Nhanh, nhờ cú pháp đơn giản và không cần runtime & Nhanh, nhờ cú pháp dễ hiểu và tích hợp tốt với các công cụ\\ 
            % \hline
            % Độ phức tạp & Thấp, ít khái niệm phức tạp & Trung bình, phức tạp hơn khi dùng Vuex hoặc các tính năng nâng cao\\ 
            % \hline
            % Tính linh hoạt & Trung bình, ít tùy chỉnh hơn do biên dịch & Cao, dễ tích hợp với nhiều thư viện và công cụ\\ 
            % \hline
            % Bảo trì & Dễ bảo trì nhờ mã biên dịch sạch, ít lỗi runtime & Dễ bảo trì với dự án nhỏ, phức tạp hơn với dự án lớn nếu không tổ chức tốt\\ 
            % \hline
            % Dễ sử dụng/Dễ học & Rất dễ học, gần với HTML, CSS, JS truyền thống & Dễ học, cú pháp trực quan, phù hợp với cả người mới và chuyên gia\\ 
            % \hline
            % Hỗ trợ cộng đồng & Nhỏ hơn, đang phát triển, ít tài liệu hơn & Lớn, năng động, nhiều tài liệu và tài nguyên\\ 
            % \hline
            % Kiểm thử & Vitest, Playwright, đơn giản nhưng ít tài liệu & Jest, Vue Test Utils, dễ thiết lập và có cộng đồng hỗ trợ\\ 
            % \hline
            % \caption{Bảng so sánh Svelt và Vue.js}\\
            % \end{longtable}

            \newpage

\begin{landscape}  % Bắt đầu phần landscape
\begin{longtable}{|p{3.5cm}|p{5cm}|p{5cm}|p{5cm}|p{5cm}|}
\hline
\textbf{Tiêu Chí} & \textbf{ReactJS} & \textbf{Angular} & \textbf{Svelt} & \textbf{Vue.js} \\
\hline
\endfirsthead
\hline
\textbf{Tiêu Chí} & \textbf{ReactJS} & \textbf{Angular} & \textbf{Svelt} & \textbf{Vue.js} \\
\hline
\endhead
\hline
Ngôn ngữ & JavaScript (hỗ trợ TypeScript) & TypeScript (bắt buộc) & JavaScript (hỗ trợ TypeScript) & JavaScript (hỗ trợ TypeScript) \\
\hline
Định nghĩa & Thư viện JavaScript để xây dựng giao diện người dùng & Framework toàn diện để phát triển ứng dụng web & Framework biên dịch, chuyển mã thành Vanilla JS & Framework tiến bộ để xây dựng giao diện người dùng \\
\hline
Kiến trúc & Dựa trên thành phần, Virtual DOM, luồng dữ liệu một chiều & Dựa trên thành phần, MVC/MVVM, cơ chế phát hiện thay đổi & Dựa trên thành phần, hệ thống phản ứng tích hợp sẵn, không dùng Virtual DOM & Dựa trên thành phần, hệ thống phản ứng, Virtual DOM \\
\hline
Thời gian phát triển & Nhanh cho dự án nhỏ, chậm hơn nếu cần tích hợp nhiều thư viện & Chậm hơn do cấu hình ban đầu phức tạp, nhanh cho dự án lớn nhờ tích hợp sẵn & Nhanh, nhờ cú pháp đơn giản và không cần runtime & Nhanh, nhờ cú pháp dễ hiểu và tích hợp tốt với các công cụ \\
\hline
Độ phức tạp & Trung bình, phụ thuộc vào cách quản lý trạng thái & Cao, nhiều khái niệm (Dependency Injection, RxJS, Modules) & Thấp, ít khái niệm phức tạp & Trung bình, phức tạp hơn khi dùng Vuex hoặc các tính năng nâng cao \\
\hline
Tính linh hoạt & Cao, tự do chọn thư viện và công cụ & Thấp hơn, bị ràng buộc bởi cấu trúc framework & Trung bình, ít tùy chỉnh hơn do biên dịch & Cao, dễ tích hợp với nhiều thư viện và công cụ \\
\hline
Bảo trì & Dễ bảo trì với các thành phần nhỏ & Dễ bảo trì cho dự án lớn, khó hơn với dự án nhỏ do cấu trúc phức tạp & Dễ bảo trì nhờ mã biên dịch sạch, ít lỗi runtime & Dễ bảo trì với dự án nhỏ, phức tạp hơn với dự án lớn nếu không tổ chức tốt \\
\hline
Dễ sử dụng/Dễ học & Dễ học với người biết JavaScript & Khó học, cần hiểu TypeScript và các khái niệm framework & Rất dễ học, gần với HTML, CSS, JS truyền thống & Dễ học, cú pháp trực quan, phù hợp với cả người mới và chuyên gia \\
\hline
Kiểm thử & Jest, React Testing Library, dễ thiết lập & Jasmine, Karma, tích hợp sẵn nhưng phức tạp hơn & Vitest, Playwright, đơn giản nhưng ít tài liệu & Jest, Vue Test Utils, dễ thiết lập và có cộng đồng hỗ trợ \\
\hline
\caption{Bảng so sánh các công nghệ Front-end phổ biến}
\end{longtable}
\end{landscape}
            
            Ngoài ra, theo các dữ liệu về xu hướng công nghệ từ Stack Overflow \cite{FrontendFrameworks}, với việc theo dõi mức độ sử dụng tag từ năm 2008 - khi nền tảng này được ra mắt - ReactJS vẫn luôn duy trì vị trí hàng đầu cho đến nay.

            \begin{figure}[H]
                \centering
                \includegraphics[width=10cm]{Images/react-stack-overflow.png}
                \vspace{0.5cm}
                \caption{Đồ thị phần trăm việc sử dụng tag tên trên Stackoverflow. Truy cập ngày: 13/02/2025}
                \label{fig:my_label}
            \end{figure}

            \textbf{Kết luận: } Dựa trên các bảng so sánh, nhóm quyết định chọn ReactJS là công nghệ chính để phát triển giao diện người dùng vì các lý do sau:

            \begin{itemize}
                \item Phù hợp với quy mô dự án: ReactJS không yêu cầu cấu trúc quá phức tạp như Angular, rất phù hợp với các dự án vừa và nhỏ. Điều này cho phép nhóm tập trung vào tính năng chính, như giao diện đặt món, mà không bị phân tâm vào các chi tiết kỹ thuật phức tạp.
                \item Sự quen thuộc: Việc đã có kinh nghiệm với ReactJS giúp nhóm phát triển hệ thống nhanh chóng hơn, đồng thời có thể tận dụng cộng đồng hỗ trợ lớn từ ReactJS.
            \end{itemize}
            % Dựa trên số liệu từ tháng 1 năm 2020 đến tháng 7 năm 2022, React đã thể hiện sự thống trị vượt trội trong cộng đồng phát triển frontend. Sự phổ biến của React được thể hiện qua số lượng repository phụ thuộc vào nó trên GitHub, vượt xa so với các framework khác như Vue và Angular 2+. Cụ thể, React đã đạt hơn 200.000 lượt sao trên GitHub, trong khi Vue và Angular còn chưa tới 100.000 lượt sao trên GitHub.

            % \begin{figure}[H]
            %     \centering
            %     \includegraphics[width=10cm]{Images/react-repository.png}
            %     \vspace{0.5cm}
            %     \caption{Biểu đồ đường số lượng repository sử dụng framework}
            %     \label{fig:my_label}
            % \end{figure}

            % \begin{figure}[H]
            %     \centering
            %     \includegraphics[width=10cm]{Images/react-star-history.png}
            %     \vspace{0.5cm}
            %     \caption{Biểu đồ thống kê GitHub Stars các framework phổ biến}
            %     \label{fig:my_label}
            % \end{figure}
            
            % Và còn nhiều kết quả thống kê khác đều chứng minh React luôn đi đầu trong số các Frontend framework khác như Vue, Angular hay Svelte \cite{FrontendFrameworks}. 
        
        % \item \textit{Nhược điểm}
        % \begin{itemize}
        %     \item \textbf{Vấn đề hiệu suất trên thiết bị cũ}: Ứng dụng React có thể gặp khó khăn trong việc duy trì hiệu suất trên các thiết bị cũ hoặc cấu hình thấp, do việc sử dụng Virtual DOM và các tính năng phức tạp có thể tiêu tốn tài nguyên hệ thống. 
        %     \item \textbf{Cập nhật thường xuyên}: React phát triển nhanh chóng, với các bản cập nhật và tính năng mới liên tục. Điều này có thể tạo ra gánh nặng cho các nhà phát triển trong việc duy trì và cập nhật mã nguồn để tương thích với các phiên bản mới.
        %     \item \textbf{JSX Learning Curve}: JSX, cú pháp kết hợp giữa JavaScript và HTML, có thể gây khó khăn cho những người mới bắt đầu, đặc biệt là khi làm quen với các khái niệm như state, props và lifecycle methods.
        % \end{itemize}
    \end{enumerate}

\subsubsubsection{ShadCN}
    \begin{enumerate}[(a)]
        \item \textit{Giới thiệu}

            ShadCN là một bộ sưu tập các thành phần giao diện người dùng (UI components) có thể tái sử dụng, được thiết kế đẹp mắt và dễ dàng tích hợp vào ứng dụng.

            Được xây dựng dựa trên thư viện Radix và framework Tailwind CSS, ShadCN cung cấp các thành phần như Button, Modal, Dropdown và nhiều hơn nữa, với khả năng tùy chỉnh linh hoạt để phù hợp với nhu cầu của từng dự án.

        \item \textit{Ưu điểm}

        \begin{itemize}
            \item \textbf{Thiết kế hướng đến doanh nghiệp}: ShadCN tập trung vào giao diện người dùng sạch sẽ và chuyên nghiệp, phù hợp cho các công cụ nội bộ và các ứng dụng nghiêm túc.
            \item \textbf{Tích hợp dễ dàng với React}: ShadCN UI được xây dựng đặc biệt cho React, giúp việc tích hợp và sử dụng trong các dự án React trở nên đơn giản và trực quan hơn. Điều này giúp giảm thiểu thời gian học tập và triển khai cho các nhà phát triển React.
            \item \textbf{Hiệu suất tối ưu}: Thư viện được thiết kế để tối ưu hóa hiệu suất, giúp giảm thiểu kích thước gói và tăng tốc độ tải trang. Điều này đặc biệt quan trọng trong các ứng dụng web hiện đại, nơi hiệu suất là yếu tố quan trọng.
        \end{itemize}

        \item \textit{Vì sao chọn ShadCN} 

            Nhóm quyết định chọn ShadCN để phát triển hệ thống Menu+ vì những lý do chính sau:

            \begin{itemize}
                \item Giao diện đồng nhất: ShadCN giúp hệ thống có giao diện đẹp, chuyên nghiệp và nhất quán, mang lại trải nghiệm mượt mà cho người dùng.
                \item Tăng tốc phát triển front-end: Với các thành phần UI đã được thiết kế sẵn, ShadCN giúp giảm thời gian phát triển, cho phép tập trung vào chức năng chính của hệ thống. Bên cạnh đó, nó còn hỗ trợ tính năng responsive, đảm bảo trải nghiệm người dùng trên mọi nền tảng.
                \item Tùy chỉnh dễ dàng: ShadCN cho phép tùy biến giao diện linh hoạt, giúp hệ thống có thể thay đổi theo nhu cầu và yêu cầu cụ thể.
            \end{itemize}

            Về mặt sử dụng, ShadCN hiện đang được sử dụng bởi khoảng 11.395 trang web, trong đó có 14 trang web nằm trong top 10 nghìn.

            \begin{figure}[H]
                \centering
                \includegraphics[width=10cm]{Images/shadcn-stats.png}
                \vspace{0.5cm}
                \caption{Thống kê sử dụng shadcn ui \cite{ShadcnUI}}
                \label{fig:my_label}
            \end{figure}
            
        % \item \textit{Nhược điểm}

        % \begin{itemize}
        %     \item \textbf{Thiếu cộng đồng lớn}: So với các thư viện UI phổ biến như MUI, ShadCN UI có cộng đồng người dùng và nhà phát triển nhỏ hơn. Điều này có thể dẫn đến việc thiếu tài liệu hỗ trợ và ít giải pháp cho các vấn đề thường gặp.
        %     \item \textbf{Hạn chế về thành phần giao diện}: Mặc dù ShadCN UI cung cấp nhiều thành phần cơ bản, nhưng so với MUI, thư viện này có thể thiếu một số thành phần phức tạp hoặc đặc biệt, điều này có thể yêu cầu bạn tự phát triển hoặc tích hợp thêm các thư viện khác.
        % \end{itemize}
    \end{enumerate}

\subsubsubsection{TanStack}
    \begin{enumerate}[(a)]
        \item \textit{Giới thiệu}

            TanStack là một tập hợp các thư viện mã nguồn mở chất lượng cao, được thiết kế để hỗ trợ các nhà phát triển web trong việc xây dựng ứng dụng hiệu quả và mạnh mẽ. Các thư viện của TanStack tập trung vào việc cung cấp các tiện ích "headless" (không có giao diện mặc định), an toàn về kiểu dữ liệu và mạnh mẽ cho các tác vụ như quản lý trạng thái, định tuyến, hiển thị dữ liệu và hơn thế nữa.

            TanStack cam kết cung cấp các công cụ mạnh mẽ và linh hoạt, giúp các nhà phát triển tạo ra các ứng dụng web hiệu quả và dễ bảo trì. Với triết lý "headless", các thư viện của TanStack cho phép tùy chỉnh giao diện hoàn toàn, dễ dàng tích hợp vào bất kỳ dự án nào mà không bị ràng buộc bởi các thiết kế mặc định \cite{TanStack}.
        
        \item \textit{Ưu điểm}

        \begin{itemize}
            \item \textbf{Quản lý trạng thái máy chủ hiệu quả}: TanStack Query (trước đây là React Query) giúp quản lý trạng thái bất đồng bộ trong React, đặc biệt trong việc fetching, caching và đồng bộ hóa dữ liệu từ server. Điều này giúp cải thiện hiệu suất và trải nghiệm người dùng.
            \item \textbf{Tối ưu hóa hiệu suất}: Với cơ chế caching tự động, TanStack giảm tải cho server và tăng tốc độ phản hồi của ứng dụng. Dữ liệu được lưu trữ trong bộ nhớ đệm và cập nhật nền, đảm bảo người dùng luôn có thông tin mới nhất mà không cần thực hiện các yêu cầu API không cần thiết. 
            \item \textbf{Xử lý lỗi và quản lý truy vấn dễ dàng}: TanStack cung cấp cách tiếp cận rõ ràng và nhất quán để xử lý lỗi, bao gồm khả năng tự động thử lại các truy vấn HTTP thất bại. Ngoài ra, việc quản lý truy vấn trở nên dễ dàng hơn với các tính năng như nhóm truy vấn, vô hiệu hóa và tìm nạp lại khi cần thiết.
        \end{itemize}

        \item \textit{Vì sao chọn TanStack} 

            TanStack được chọn để phát triển hệ thống Menu+ bởi các lí do chính sau:
            \begin{itemize}
                \item Quản lý trạng thái hiệu quả: TanStack giúp đồng bộ dữ liệu giữa các trang như thực đơn, đơn hàng, và thông tin khách hàng, đảm bảo trải nghiệm người dùng mượt mà. 
                \item Tối ưu hiệu suất: Tính năng caching và prefetching giúp giảm số lần gọi API, tăng tốc độ xử lý dữ liệu như thông tin món ăn, đơn hàng, và thanh toán.
                \item Xử lý truy vấn phức tạp: TanStack hỗ trợ việc tìm kiếm món ăn, lọc theo danh mục và theo dõi đơn hàng theo thời gian thực một cách hiệu quả và nhanh chóng.
            \end{itemize}

            Sự phổ biến của TanStack được minh chứng qua những con số ấn tượng sau:

            \begin{itemize}
                \item Hơn 1,7 tỷ lượt tải trên NPM, thể hiện sự tin dùng rộng rãi trong cộng đồng phát triển.
                \item 95,271 sao trên GitHub, chứng tỏ sự đánh giá cao từ lập trình viên trên toàn cầu.
                \item 2,142 người đóng góp trên GitHub, cho thấy cộng đồng phát triển mạnh mẽ và liên tục cải tiến.
                \item 837,561 dự án phụ thuộc trên GitHub, khẳng định mức độ tích hợp cao và ứng dụng thực tế rộng rãi.
            \end{itemize}
            
            \begin{figure}[H]
                \centering
                \includegraphics[width=10cm]{Images/tanstack-stats.png}
                \vspace{0.5cm}
                \caption{Các số liệu thống kê về Tanstack \cite{TanStack}}
                \label{fig:my_label}
            \end{figure}
 
        % \item \textit{Nhược điểm}

        % \begin{itemize}
        %     \item \textbf{Quản lý bộ nhớ đệm phức tạp}: Việc quản lý tầng dữ liệu cache yêu cầu sự cẩn trọng; nếu không, có thể dẫn đến các lỗi không mong muốn trong ứng dụng. 
        %     \item \textbf{Phức tạp khi kết hợp với các thư viện khác}: Sử dụng TanStack song song với các thư viện quản lý trạng thái khác như Redux có thể gây ra sự phức tạp trong luồng mã và quản lý trạng thái.
        % \end{itemize}
        
    \end{enumerate}
\subsubsection{Công nghệ Back-end}
\subsubsubsection{SpringBoot}
\begin{enumerate}[(a)]
	\item \textit{Giới thiệu}

	      Spring Boot là một framework mã nguồn mở dựa trên nền tảng Java, được thiết kế để đơn giản hóa việc phát triển các ứng dụng Spring độc lập, sẵn sàng cho môi trường sản xuất. Nó cung cấp các cấu hình mặc định thông minh và tự động, giúp giảm thiểu cấu hình thủ công và tối ưu hóa quá trình phát triển ứng dụng Java. Spring Boot tích hợp tốt với nhiều công nghệ và thư viện khác trong hệ sinh thái Spring Framework, cho phép hệ thống dễ dàng tích hợp các module và dịch vụ khác nhau mà không cần phải lo lắng về cấu hình phức tạp. Ngoài ra, Spring Boot đi kèm với các máy chủ nhúng như Tomcat, Jetty hoặc Undertow, giúp triển khai ứng dụng một cách đơn giản mà không cần cấu hình thêm bất kỳ máy chủ nào khác

	\item \textit{Ưu điểm} \cite{SpringBootBenefits}

	      \begin{itemize}
		      \item \textbf{Phát triển nhanh chóng}: Tuân theo nguyên tắc "quy ước hơn cấu hình" (convention over configuration), Spring Boot cung cấp các thiết lập mặc định hợp lý cho các trường hợp sử dụng phổ biến. Cách tiếp cận này giúp giảm thiểu mã lặp (boilerplate code), cho phép lập trình viên tập trung vào logic nghiệp vụ thay vì phải cấu hình phức tạp.
		      \item \textbf{Tự động cấu hình}: Cơ chế tự động cấu hình (auto-configuration) của Spring Boot giúp thiết lập các thành phần ứng dụng dựa trên các thư viện và đặc điểm của dự án, giảm bớt công việc cấu hình thủ công và tối ưu hóa quy trình phát triển.
		      \item \textbf{Tích hợp liền mạch với hệ sinh thái Spring}: Spring Boot kết nối dễ dàng với các dự án khác trong hệ sinh thái Spring như Spring Data, Spring Security và Spring Cloud, cung cấp một nền tảng toàn diện để phát triển ứng dụng.
		      \item \textbf{Tính sẵn sàng cho môi trường sản xuất}: Spring Boot tích hợp sẵn các tính năng hỗ trợ triển khai trong môi trường thực tế như kiểm tra sức khỏe (health checks), thu thập số liệu (metrics), và cấu hình bên ngoài (externalized configuration), giúp giám sát và quản lý ứng dụng hiệu quả hơn.
	      \end{itemize}

	\item \textit{So sánh Spring Boot và các công nghệ khác}

	      Khi so sánh giữa các công nghệ liên quan, đặc biệt là Spring Boot với Express, Flask và Django, chúng ta có thể nhận thấy mỗi công nghệ có những ưu điểm và hạn chế riêng. Dưới đây là bảng so sánh chi tiết giữa Spring Boot và các công nghệ khác, dựa trên các tài liệu \cite{BackCompare1}, \cite{BackCompare2}, \cite{BackCompare3}:

	      \begin{landscape}  % Bắt đầu phần landscape
		      \begin{longtable}{|p{3.5cm}|p{5cm}|p{5cm}|p{5cm}|p{5cm}|}
			      \caption{Bảng so sánh các công nghệ Back-end phổ biến}
			      \hline
			      \textbf{Tiêu Chí}    & \textbf{Spring Boot}                                                         & \textbf{Express}                                                         & \textbf{Flask}                                           & \textbf{Django}                                           \\
			      \hline
			      \endfirsthead
			      \hline
			      \textbf{Tiêu Chí}    & \textbf{Spring Boot}                                                         & \textbf{Express}                                                         & \textbf{Flask}                                           & \textbf{Django}                                           \\
			      \hline
			      \endhead
			      \hline
			      Ngôn ngữ             & Java (hỗ trợ TypeScript qua tích hợp)                                        & JavaScript (hỗ trợ TypeScript)                                           & Python                                                   & Python                                                    \\
			      \hline
			      Định nghĩa           & Framework Java để xây dựng ứng dụng web và microservices                     & Môi trường runtime JavaScript với Express là framework nhẹ cho backend   & Framework Python nhẹ cho phát triển web                  & Framework Python toàn diện cho phát triển web             \\
			      \hline
			      Kiến trúc            & Dựa trên thành phần, MVC, tự động cấu hình, luồng dữ liệu một chiều          & Dựa trên sự kiện, không chặn (non-blocking), linh hoạt                   & Không áp đặt, linh hoạt, microframework                  & MVC (MTV - Model-Template-View), tích hợp sẵn             \\
			      \hline
			      Thời gian phát triển & Nhanh cho dự án lớn nhờ tự động cấu hình, chậm hơn cho dự án nhỏ             & Rất nhanh, đặc biệt cho ứng dụng nhỏ và thời gian đưa ra thị trường ngắn & Rất nhanh cho dự án nhỏ, cần thêm cấu hình cho dự án lớn & Nhanh, tích hợp sẵn nhiều tính năng                       \\
			      \hline
			      Độ phức tạp          & Trung bình đến cao, cần hiểu Spring ecosystem                                & Thấp, dễ bắt đầu nhưng phức tạp hơn khi mở rộng quy mô                   & Thấp, đơn giản nhưng phức tạp khi mở rộng                & Trung bình, dễ dùng nhưng phức tạp với tùy chỉnh nâng cao \\
			      \hline
			      Tính linh hoạt       & Trung bình, bị ràng buộc bởi cấu trúc Spring nhưng dễ tích hợp thư viện Java & Cao, tự do chọn công cụ và thư viện                                      & Cao, tự do chọn công cụ và cấu hình                      & Thấp hơn, bị ràng buộc bởi cấu trúc Django                \\
			      \hline
			      Bảo trì              & Dễ bảo trì cho ứng dụng lớn, khó hơn nếu không tổ chức tốt                   & Dễ bảo trì cho ứng dụng nhỏ, khó hơn khi codebase lớn                    & Dễ cho ứng dụng nhỏ, khó hơn khi codebase lớn            & Dễ bảo trì nhờ ORM và cấu trúc rõ ràng                    \\
			      \hline
			      Dễ sử dụng/Dễ học    & Khó hơn, cần kiến thức Java và Spring                                        & Dễ học, đặc biệt với người biết JavaScript                               & Rất dễ học, tối giản, phù hợp người mới                  & Dễ học, tài liệu tốt, phù hợp người mới                   \\
			      \hline
			      Kiểm thử             & JUnit, Mockito, tích hợp sẵn nhưng cần cấu hình                              & Jest, Mocha, dễ thiết lập và linh hoạt                                   & unittest, pytest, dễ thiết lập                           & unittest, pytest, tích hợp tốt                            \\
			      \hline
		      \end{longtable}
	      \end{landscape}

	      \textbf{Kết luận: } Dựa trên các bảng so sánh, nhóm quyết định chọn Spring Boot là công nghệ chính để phát triển hệ thống vì các lý do sau:

	      \begin{itemize}
		      \item Phù hợp với kiến trúc MVC: Spring Boot hỗ trợ tốt kiến trúc MVC (Model-View-Controller), giúp phân chia rõ ràng các phần của hệ thống và tạo cấu trúc dễ hiểu, phù hợp với yêu cầu đồ án. Điều này sẽ giúp nhóm triển khai hệ thống có tổ chức và dễ dàng phân chia công việc.
		      \item Linh hoạt và dễ mở rộng: Spring Boot mang lại khả năng tích hợp microservices và hỗ trợ các thư viện Java phong phú. Điều này cho phép nhóm dễ dàng mở rộng hệ thống trong tương lai mà không cần thay đổi quá nhiều mã nguồn.
		      \item Dễ bảo trì: Với cấu trúc dự án rõ ràng và các công cụ như Spring Boot Actuator, Spring Boot giúp nhóm dễ dàng theo dõi và bảo trì hệ thống, rất thuận lợi khi cần sửa lỗi nhanh chóng hoặc thực hiện các thay đổi trong quá trình phát triển.
	      \end{itemize}

\end{enumerate}

\subsubsubsection{Hibernate}
\begin{enumerate}[(a)]
	\item \textit{Giới thiệu}

	      Hibernate là một framework ORM (Object-Relational Mapping) mạnh mẽ dành cho Java, giúp lập trình viên làm việc với cơ sở dữ liệu một cách hiệu quả hơn. Nó cung cấp một lớp trừu tượng giữa ứng dụng và cơ sở dữ liệu, cho phép thao tác dữ liệu bằng các đối tượng Java thay vì truy vấn SQL thuần túy. Hibernate hỗ trợ nhiều hệ quản trị cơ sở dữ liệu khác nhau và có khả năng tự động ánh xạ các bảng trong cơ sở dữ liệu thành các lớp Java thông qua tập hợp các quy tắc ánh xạ linh hoạt. Ngoài ra, Hibernate còn đi kèm với các tính năng như quản lý phiên làm việc (session management), bộ nhớ đệm (caching), và hỗ trợ giao dịch (transaction management), giúp tối ưu hóa hiệu suất và đơn giản hóa quá trình phát triển ứng dụng. Với khả năng tích hợp dễ dàng cùng các framework khác như Spring và Java EE, Hibernate đã trở thành một trong những lựa chọn phổ biến nhất trong các ứng dụng doanh nghiệp sử dụng Java.

	\item \textit{Ưu điểm}

	      \begin{itemize}
		      \item \textbf{Ánh xạ đối tượng-quan hệ tự động}: Hibernate cho phép ánh xạ tự động giữa các lớp Java và các bảng trong cơ sở dữ liệu thông qua các tệp cấu hình XML hoặc annotation, giúp giảm thiểu mã nguồn và đơn giản hóa việc quản lý dữ liệu.
		      \item \textbf{Độc lập với cơ sở dữ liệu}: Mã lệnh của Hibernate có thể hoạt động với nhiều hệ quản trị cơ sở dữ liệu khác nhau như MySQL, Oracle mà không cần thay đổi mã HQL. Người dùng chỉ cần cập nhật thông tin cấu hình khi chuyển đổi hệ quản trị, giúp tiết kiệm thời gian và công sức.
		      \item \textbf{Quản lý phiên và giao dịch hiệu quả}: Hibernate cung cấp cơ chế quản lý phiên (Session) và giao dịch (Transaction) mạnh mẽ, đảm bảo tính toàn vẹn dữ liệu và hỗ trợ các thao tác như lưu trữ, cập nhật và xóa dữ liệu một cách an toàn.
		      \item \textbf{Hỗ trợ tải chậm (Lazy Loading)}: Hibernate hỗ trợ cơ chế tải chậm, chỉ tải dữ liệu khi cần thiết, giúp tiết kiệm tài nguyên hệ thống và cải thiện hiệu suất ứng dụng.

		      \item \textit{Vì sao chọn Hibernate}

		            Hibernate được chọn để phát triển hệ thống Menu+ vì những lý do sau:

		            \begin{itemize}
			            \item Quản lý dữ liệu đơn giản và hiệu quả: Menu+ cần quản lý lượng lớn dữ liệu như thực đơn, đơn hàng và thông tin khách hàng. Hibernate giúp ánh xạ tự động giữa các đối tượng trong Java và cơ sở dữ liệu, giúp giảm thiểu công việc thủ công trong việc viết mã SQL và làm cho việc quản lý dữ liệu trở nên dễ dàng và linh hoạt.
			            \item Hỗ trợ mở rộng: Menu+ có thể phát triển và mở rộng trong tương lai, ví dụ như thêm nhiều tính năng mới hoặc thay đổi cơ sở dữ liệu. Hibernate hỗ trợ nhiều loại cơ sở dữ liệu khác nhau, cho phép hệ thống dễ dàng thay đổi hoặc nâng cấp mà không gặp khó khăn lớn.
		            \end{itemize}
	      \end{itemize}

	      % \item \textit{Nhược điểm}

	      % \begin{itemize}
	      %     \item \textbf{Không hỗ trợ tốt cho các truy vấn phức tạp}: Hibernate có thể gặp khó khăn khi xử lý các truy vấn SQL phức tạp, đặc biệt là những truy vấn yêu cầu tối ưu hóa cao hoặc sử dụng các tính năng đặc thù của hệ quản trị cơ sở dữ liệu. Trong những trường hợp này, lập trình viên có thể phải sử dụng truy vấn SQL gốc (native SQL), làm giảm tính trừu tượng và lợi ích của ORM.
	      %     \item \textbf{Tăng độ trễ khởi tạo}: Việc sử dụng Hibernate có thể dẫn đến thời gian khởi tạo đối tượng và kết nối cơ sở dữ liệu lâu hơn so với cách viết SQL trực tiếp, do quá trình ánh xạ và cấu hình phức tạp.
	      %     \item \textbf{Tiêu thụ tài nguyên hệ thống}: Hibernate có thể tiêu thụ nhiều tài nguyên hệ thống hơn so với việc sử dụng JDBC thuần túy, đặc biệt khi không được cấu hình và tối ưu hóa đúng cách.
	      % \end{itemize}
\end{enumerate}

\subsubsubsection{Redis}
\begin{enumerate}[(a)]
	\item \textit{Giới thiệu}

	      Redis, viết tắt của "Remote Dictionary Server", là một hệ thống lưu trữ dữ liệu mã nguồn mở, hoạt động trong bộ nhớ (in-memory), thuộc loại NoSQL và sử dụng mô hình key-value. Được phát triển bởi Salvatore Sanfilippo vào năm 2009, Redis ban đầu được thiết kế để cải thiện khả năng mở rộng cho một dự án phân tích nhật ký web theo thời gian thực. Kể từ đó, nó đã trở thành một trong những cơ sở dữ liệu NoSQL phổ biến nhất, được sử dụng rộng rãi trong các ứng dụng yêu cầu hiệu suất cao và độ trễ thấp.

	      Redis lưu trữ toàn bộ dữ liệu trong bộ nhớ, cho phép truy xuất và ghi dữ liệu với tốc độ rất nhanh. Ngoài việc hỗ trợ các kiểu dữ liệu đơn giản như chuỗi (strings), Redis còn hỗ trợ các cấu trúc dữ liệu phức tạp như danh sách (lists), tập hợp (sets), tập hợp có thứ tự (sorted sets), băm (hashes), và nhiều cấu trúc khác. Điều này giúp Redis linh hoạt trong việc giải quyết nhiều bài toán khác nhau, từ lưu trữ phiên làm việc (session storage), hàng đợi tin nhắn (message queues), đến bộ đệm (caching) và phân tích thời gian thực.

	      Hiện nay, Redis được sử dụng bởi nhiều công ty lớn như Twitter, Airbnb, Tinder, Yahoo, Adobe, Hulu, Amazon và OpenAI, nhờ vào khả năng cung cấp hiệu suất cao và linh hoạt trong nhiều trường hợp sử dụng khác nhau.

	      \begin{figure}[H]
		      \centering
		      \includegraphics[width=15cm]{Images/redis.png}
		      \vspace{0.5cm}
		      \caption{Logo của Redis}
		      \label{fig:my_label}
	      \end{figure}

	\item \textit{Ưu điểm} \cite{Redis}

	      \begin{itemize}
		      \item \textbf{Hiệu suất cao}: Là một hệ thống lưu trữ dữ liệu trong bộ nhớ (in-memory data store), Redis cung cấp các thao tác đọc và ghi cực kỳ nhanh chóng, có thể xử lý hàng triệu yêu cầu mỗi giây. Điều này làm cho Redis trở nên lý tưởng cho các ứng dụng yêu cầu truy cập dữ liệu với độ trễ thấp.
		      \item \textbf{Hỗ trợ đa dạng cấu trúc dữ liệu}: Redis hỗ trợ nhiều kiểu dữ liệu khác nhau, bao gồm chuỗi (strings), danh sách (lists), băm (hashes), tập hợp (sets) và tập hợp có thứ tự (sorted sets), cho phép các nhà phát triển triển khai hiệu quả nhiều chức năng khác nhau.
		      \item \textbf{Hệ thống nhắn tin xuất bản/đăng ký (Publish/Subscribe)}: Redis bao gồm một hệ thống nhắn tin xuất bản/đăng ký (Pub/Sub), cho phép giao tiếp theo thời gian thực giữa các ứng dụng, hữu ích cho việc xây dựng các hệ thống trò chuyện, thông báo và các tính năng thời gian thực khác.
	      \end{itemize}

	\item \textit{Vì sao chọn Redis}

	      Redis được chọn để phát triển hệ thống Menu+ vì những lý do sau:

	      \begin{itemize}
		      \item Tăng tốc độ truy xuất dữ liệu: Redis là một cơ sở dữ liệu lưu trữ theo kiểu key-value trong bộ nhớ (in-memory), giúp truy xuất dữ liệu cực kỳ nhanh chóng. Điều này rất quan trọng trong hệ thống Menu+ khi cần xử lý nhanh các tác vụ như lưu trữ thông tin đơn hàng, phiên làm việc của người dùng hay các trạng thái tạm thời.
		      \item Quản lý phiên làm việc (Session Management): Trong hệ thống quản lý đặt món, Redis rất hữu ích để lưu trữ và quản lý phiên làm việc của người dùng. Điều này giúp theo dõi trạng thái đăng nhập và các hoạt động của người dùng một cách hiệu quả, đồng thời giảm thiểu việc truy xuất cơ sở dữ liệu truyền thống.
		      \item Dễ dàng tích hợp: Redis có thể dễ dàng tích hợp vào hệ thống hiện tại của Menu+ mà không yêu cầu thay đổi lớn về cấu trúc. Nó cũng tương thích tốt với các hệ thống khác như Hibernate và TanStack, giúp cải thiện hiệu quả hoạt động của toàn bộ hệ thống.
	      \end{itemize}

	      % \item \textit{Nhược điểm}

	      % \begin{itemize}
	      %     \item \textbf{Giới hạn về dung lượng bộ nhớ}: Redis lưu trữ toàn bộ dữ liệu trong bộ nhớ RAM, do đó, khả năng lưu trữ bị giới hạn bởi dung lượng bộ nhớ vật lý của hệ thống. Đối với các ứng dụng yêu cầu lưu trữ lượng dữ liệu lớn, việc sử dụng Redis có thể dẫn đến chi phí cao và không khả thi. 
	      %     \item \textbf{Thiếu tính nhất quán mạnh mẽ}: Redis sử dụng mô hình sao chép bất đồng bộ, điều này có thể dẫn đến tình trạng dữ liệu không nhất quán giữa các nút chủ và nút phụ, đặc biệt trong các tình huống chuyển đổi dự phòng hoặc phân tách mạng.
	      %     \item \textbf{Hạn chế trong truy vấn phức tạp}: Redis không hỗ trợ các truy vấn phức tạp như join hoặc các phép tổng hợp, điều này làm giảm tính linh hoạt khi cần thao tác với dữ liệu phức tạp.
	      %     \item \textbf{Phức tạp trong việc thiết lập phân cụm}: Việc cấu hình và quản lý Redis Cluster có thể phức tạp, đòi hỏi kiến thức chuyên sâu và kinh nghiệm để đảm bảo hệ thống hoạt động ổn định và hiệu quả.
	      % \end{itemize}
\end{enumerate}

% \subsubsubsection{Kafka}
%     \begin{enumerate}[(a)]
%         \item \textit{Giới thiệu}

%         Apache Kafka là một nền tảng phân tán mã nguồn mở được thiết kế để xử lý các luồng dữ liệu theo thời gian thực. Ban đầu được phát triển bởi LinkedIn và sau đó trở thành dự án của Apache Software Foundation, Kafka cho phép các ứng dụng xuất bản, lưu trữ và tiêu thụ các luồng bản ghi (record streams) một cách hiệu quả. Hệ thống này hoạt động dựa trên mô hình xuất bản-đăng ký (publish-subscribe), trong đó các nhà sản xuất (producers) gửi thông điệp đến các chủ đề (topics), và các người tiêu thụ (consumers) đăng ký để nhận các thông điệp này. Kafka được sử dụng rộng rãi trong việc xây dựng các hệ thống xử lý dữ liệu thời gian thực, như theo dõi hoạt động người dùng, giám sát hệ thống, và phân tích dữ liệu trực tuyến. 

%         \begin{figure}[H]
%             \centering
%             \includegraphics[width=15cm]{Images/kafka.png}
%             \vspace{0.5cm}
%             \caption{Vị trí của Kafka trong dự án}
%             \label{fig:my_label}
%         \end{figure}
%         \item \textit{Ưu điểm}

%         \begin{itemize}
%             \item \textbf{Hiệu suất cao}: Kafka có khả năng xử lý lượng lớn thông tin với thông lượng cao và độ trễ thấp, cho phép xử lý hàng triệu thông điệp mỗi giây.
%             \item \textbf{Khả năng mở rộng}: Với kiến trúc phân tán, Kafka dễ dàng mở rộng bằng cách thêm các broker vào cụm, tăng khả năng xử lý dữ liệu mà không cần thay đổi cấu trúc hệ thống. 
%             \item \textbf{Độ tin cậy và bền vững}: Kafka lưu trữ các sự kiện theo định dạng nhật ký đơn giản, đảm bảo dữ liệu bền vững và chính sách lưu giữ dễ triển khai. 
%             \item \textbf{Xử lý dữ liệu thời gian thực}: Kafka được thiết kế để xử lý dữ liệu thời gian thực và streaming, cho phép bạn đáp ứng nhanh chóng đối với sự kiện mới xảy ra. 
%         \end{itemize}

%         % \item \textit{Nhược điểm}

%         % \begin{itemize}
%         %     \item \textbf{Phức tạp trong cài đặt và quản lý}: Việc triển khai và cấu hình Kafka ban đầu có thể phức tạp, đặc biệt đối với những người mới bắt đầu. Để quản lý hiệu quả, cần có kiến thức sâu về hệ thống phân tán và mạng.
%         %     \item \textbf{Yêu cầu tài nguyên hệ thống cao}: Kafka đòi hỏi một lượng tài nguyên phần cứng đáng kể, bao gồm bộ nhớ, dung lượng lưu trữ và khả năng xử lý, để hoạt động hiệu quả. 
%         %     \item \textbf{Thiếu công cụ giám sát hoàn chỉnh}: Kafka không cung cấp một bộ công cụ giám sát và quản lý tích hợp đầy đủ. Người dùng thường phải dựa vào các công cụ của bên thứ ba để theo dõi và quản lý hệ thống.
%         % \end{itemize}
%     \end{enumerate}

% % \subsubsubsection{RESTful API}
% %     \begin{enumerate}[(a)]
% %         \item \textit{Giới thiệu}

% %         \item \textit{Vì sao chọn RESTful API} 

% %         \item \textit{Ưu điểm}

% %         \item \textit{Nhược điểm}
% %     \end{enumerate}
\subsubsection{Cơ sở dữ liệu - PostgreSQL}
    \begin{enumerate}[(a)]
        \item \textit{Giới thiệu}
        
            PostgreSQL là một hệ thống quản lý cơ sở dữ liệu quan hệ đối tượng (ORDBMS) mã nguồn mở, được phát triển từ năm 1986 tại Đại học California, Berkeley, dưới sự dẫn dắt của giáo sư Michael Stonebraker. Ban đầu, nó được gọi là "Postgres" và đã trải qua nhiều giai đoạn phát triển để trở thành một trong những hệ quản trị cơ sở dữ liệu phổ biến nhất hiện nay. 

            PostgreSQL hỗ trợ cả truy vấn SQL truyền thống và JSON, cho phép xử lý linh hoạt cả dữ liệu quan hệ và phi quan hệ. Nó cung cấp các tính năng nâng cao như hỗ trợ các kiểu dữ liệu đa dạng, khả năng mở rộng cao và tuân thủ các tiêu chuẩn SQL mới nhất.

            Ngày nay, PostgreSQL được sử dụng rộng rãi trong nhiều lĩnh vực, từ các ứng dụng web động đến các hệ thống thông tin địa lý, nhờ vào khả năng hỗ trợ các đối tượng địa lý và tính năng ghi nhật ký trước, giúp đảm bảo tính toàn vẹn và khả năng khôi phục dữ liệu.
                    
        \item \textit{Ưu điểm}

        \begin{itemize}
            \item \textbf{Xử lý khối lượng dữ liệu lớn}: PostgreSQL có khả năng quản lý các hệ thống lớn với hàng triệu bản ghi, đáp ứng nhu cầu của các tập đoàn và tổ chức yêu cầu hệ thống cơ sở dữ liệu mạnh mẽ và đáng tin cậy. 
            \item \textbf{Kiểm soát đồng thời nhiều phiên bản (MVCC)}: Tính năng MVCC cho phép PostgreSQL xử lý các thao tác ghi thường xuyên và truy vấn phức tạp một cách hiệu quả, phù hợp cho các ứng dụng cấp doanh nghiệp.
            \item \textbf{Độ tin cậy và phục hồi}: Sử dụng ghi nhật ký ghi trước (WAL), PostgreSQL đảm bảo độ tin cậy và hỗ trợ khôi phục điểm-theo-thời-gian (PITR) và active standbys, giúp hệ thống hoạt động ổn định và phục hồi nhanh chóng khi gặp sự cố.
        \end{itemize}

        \item \textit{So sánh PostgreSQL và các hệ quản trị cơ sở dữ liệu}
        
Khi so sánh PostgreSQL với các hệ quản trị cơ sở dữ liệu khác như MySQL, SQL Server và MongoDB, mỗi hệ thống đều sở hữu những đặc điểm nổi bật và phù hợp với các nhu cầu khác nhau. Dưới đây là bảng so sánh chi tiết giữa PostgreSQL và các cơ sở dữ liệu này, giúp làm rõ ưu điểm và hạn chế của từng loại.

\begin{landscape} 
\begin{longtable}{|p{3.5cm}|p{4cm}|p{4cm}|p{4cm}|p{4cm}|}
\hline
\textbf{Tiêu chí} & \textbf{PostgreSQL} & \textbf{MySQL} & \textbf{SQL Server} & \textbf{MongoDB} \\
\hline
\endfirsthead
\hline
\textbf{Tiêu chí} & \textbf{PostgreSQL} & \textbf{MySQL} & \textbf{SQL Server} & \textbf{MongoDB} \\
\hline
\endhead
\hline
\textbf{Loại cơ sở dữ liệu} & Quan hệ - Đối tượng (ORDBMS), mã nguồn mở & Quan hệ (RDBMS), mã nguồn mở & Quan hệ (RDBMS), độc quyền & Phi quan hệ (NoSQL), mã nguồn mở \\
\hline
\textbf{Hiệu suất} & Tốt cho truy vấn phức tạp, chậm hơn MySQL trong truy vấn đơn giản & Cao cho truy vấn đơn giản, ứng dụng web & Cao, tối ưu cho doanh nghiệp lớn & Cao cho dữ liệu phi cấu trúc, đọc/ghi nhanh \\
\hline
\textbf{Khả năng mở rộng} & Cao, hỗ trợ terabyte/petabyte dữ liệu & Tốt, hỗ trợ cluster và replication & Cao, tích hợp đám mây Azure & Rất cao, mở rộng ngang tốt \\
\hline
\textbf{Kiểu dữ liệu hỗ trợ} & Phong phú: JSON, XML, PostGIS, tùy chỉnh & Cơ bản, hỗ trợ JSON từ phiên bản 5.7 & Cơ bản, hỗ trợ XML, JSON & Linh hoạt: JSON, BSON \\
\hline
\textbf{Khả năng tìm kiếm văn bản} & Tìm kiếm văn bản đầy đủ, hỗ trợ ký tự quốc tế & Tìm kiếm văn bản cơ bản & Tìm kiếm văn bản tốt, tích hợp Full Text Search & Tìm kiếm văn bản tốt, tích hợp Atlas Search \\
\hline
\textbf{Hỗ trợ nền tảng} & Linux, Windows, macOS, Solaris & Linux, Windows, macOS & Chủ yếu Windows, hỗ trợ Linux & Linux, Windows, macOS \\
\hline
\textbf{Chi phí} & Miễn phí, mã nguồn mở & Miễn phí, có phiên bản thương mại & Độc quyền, chi phí cao & Miễn phí, có dịch vụ đám mây trả phí \\
\hline
\textbf{Ứng dụng phù hợp} & GIS, tài chính, ứng dụng phức tạp & Ứng dụng web, WordPress, Joomla & Doanh nghiệp lớn, tích hợp Microsoft & Ứng dụng thời gian thực, dữ liệu lớn \\
\hline
\caption{Bảng so sánh PostgreSQL và các hệ quản trị cơ sở dữ liệu}
\end{longtable}
\end{landscape} 

    \end{enumerate}
\subsubsection{Công nghệ triển khai}
\subsubsubsection{GitHub Actions}
    \begin{enumerate}[(a)]
        \item \textit{Giới thiệu}

        GitHub Actions là một nền tảng tích hợp sẵn trong GitHub, được giới thiệu vào năm 2018, cho phép các nhà phát triển tự động hóa các quy trình trong vòng đời phát triển phần mềm trực tiếp trên kho lưu trữ của họ. Nó hỗ trợ việc xây dựng, kiểm thử và triển khai ứng dụng thông qua việc tạo ra các workflow (quy trình làm việc) được định nghĩa bằng tệp YAML \cite{Viblo}.

        GitHub Actions cung cấp một hệ sinh thái phong phú với nhiều actions (hành động) có sẵn từ cộng đồng, cho phép người dùng dễ dàng tích hợp các công cụ và dịch vụ bên ngoài vào quy trình làm việc của họ. Ngoài ra, nó hỗ trợ chạy trên nhiều hệ điều hành như Windows, macOS và Ubuntu, giúp kiểm thử ứng dụng trên các môi trường khác nhau \cite{Viblo}.

        Với GitHub Actions, các nhà phát triển có thể thiết lập các quy trình CI/CD (Continuous Integration/Continuous Deployment) một cách linh hoạt và hiệu quả, tự động hóa các tác vụ như xây dựng, kiểm thử và triển khai ứng dụng, từ đó nâng cao hiệu suất làm việc và chất lượng sản phẩm.

        \item \textit{Ưu điểm}

        \begin{itemize}
            \item \textbf{Tích hợp liền mạch với GitHub}: Là một tính năng gốc của nền tảng GitHub, GitHub Actions cho phép các nhà phát triển tự động hóa workflow trực tiếp trong kho lưu trữ của họ. Sự tích hợp chặt chẽ này giúp đơn giản hóa các tác vụ như build, test và deploy code mà không cần sử dụng các công cụ bên ngoài.
            \item \textbf{Marketplace phong phú}: GitHub Marketplace cung cấp hơn 10.000 actions được xây dựng sẵn, giúp các nhà phát triển dễ dàng tích hợp nhiều công cụ và dịch vụ vào workflow của họ. Thư viện rộng lớn này giúp đơn giản hóa quá trình thiết lập các pipeline phức tạp bằng cách cung cấp các thành phần có thể tái sử dụng \cite{GitHubActionsIntro}.
            \item \textbf{Tự động hóa linh hoạt}: GitHub Actions hỗ trợ workflow dựa trên sự kiện (event-driven workflows), cho phép tự động hóa theo phản hồi của các sự kiện trong kho lưu trữ, chẳng hạn như push, pull request hoặc issue creation. Tính linh hoạt này giúp các nhà phát triển tùy chỉnh workflow để phù hợp với yêu cầu cụ thể của dự án.
            \item \textbf{Hỗ trợ đa nền tảng}: GitHub Actions cung cấp runners cho môi trường Linux, Windows và macOS, cho phép các nhà phát triển test và deploy ứng dụng trên nhiều hệ điều hành khác nhau. Khả năng hỗ trợ đa nền tảng này giúp đảm bảo ứng dụng hoạt động ổn định trong các môi trường khác nhau \cite{GitHubActionsDocs}.
        \end{itemize}
        
        % \item \textit{Nhược điểm} \cite{GitHubActionsMerits}

        % \begin{itemize}
        %     \item \textbf{Độ phức tạp trong workflow phức tạp}: Thiết kế workflow với nhiều bước và điều kiện có thể trở nên rối rắm, gây khó khăn trong việc bảo trì, đặc biệt đối với những người mới làm quen với các khái niệm về Continuous Integration/Continuous Deployment (CI/CD).
        %     \item \textbf{Giới hạn tài nguyên}: GitHub Actions áp đặt các giới hạn về mức sử dụng tài nguyên, bao gồm thời gian thực thi tối đa và dung lượng ổ đĩa có sẵn. Những hạn chế này có thể cản trở các workflow yêu cầu nhiều tài nguyên tính toán.
        %     \item \textbf{Phụ thuộc vào hệ sinh thái GitHub}: Do GitHub Actions được tích hợp chặt chẽ với nền tảng GitHub, bất kỳ sự cố gián đoạn hoặc downtime nào trên GitHub đều có thể ảnh hưởng trực tiếp đến các workflow CI/CD. Sự phụ thuộc này làm dấy lên lo ngại về độ tin cậy và khả năng sẵn sàng của hệ thống.
        %     \item \textbf{Thách thức trong quá trình debugging}: Việc xác định và khắc phục sự cố trong GitHub Actions có thể gặp khó khăn do khả năng quan sát (observability) và cơ chế phản hồi (feedback) còn hạn chế. Các nhà phát triển thường cần chèn nhiều log statement và chờ workflow thực thi để chẩn đoán vấn đề, dẫn đến chu kỳ phát triển kéo dài hơn.
        % \end{itemize}
    \end{enumerate}

\subsubsubsection{Docker}
    \begin{enumerate}[(a)]
        \item \textit{Giới thiệu}

        Docker là một nền tảng mã nguồn mở giúp các nhà phát triển tự động hóa việc triển khai, mở rộng và quản lý các ứng dụng trong các container nhẹ và di động. Các container này đóng gói ứng dụng và các phụ thuộc của nó, đảm bảo ứng dụng hoạt động nhất quán trên nhiều môi trường khác nhau.
        
        \item \textit{Ưu điểm} \cite{DockerAdvantages}

        \begin{itemize}
            \item \textbf{Tính di động và nhất quán}: Các container Docker đóng gói ứng dụng và các phụ thuộc của chúng, đảm bảo chúng chạy nhất quán trên nhiều môi trường khác nhau, từ phát triển đến sản xuất. Điều này loại bỏ vấn đề "it works on my machine" thường gặp. 
            \item \textbf{Hiệu quả tài nguyên}: Khác với các máy ảo truyền thống, các container Docker chia sẻ nhân hệ thống của máy chủ, giúp chúng nhẹ và hiệu quả hơn. Điều này cho phép khởi động nhanh hơn và chạy nhiều container hơn trên cùng một phần cứng, dẫn đến việc sử dụng tài nguyên tốt hơn. 
            \item \textbf{Triển khai nhanh chóng và khả năng mở rộng}: Docker cho phép triển khai và mở rộng ứng dụng một cách nhanh chóng. Các container có thể được khởi động hoặc dừng gần như ngay lập tức, tạo điều kiện cho việc mở rộng nhanh chóng dựa trên nhu cầu. Sự linh hoạt này rất quan trọng đối với các ứng dụng yêu cầu khả năng mở rộng động.
            \item \textbf{Tách biệt và bảo mật}: Docker cung cấp mức độ tách biệt cao giữa các ứng dụng, giảm thiểu rủi ro xung đột và tăng cường bảo mật. Mỗi container hoạt động độc lập, giảm thiểu tác động của các lỗ hổng và đảm bảo hiệu suất ứng dụng ổn định.
            \item \textbf{Quản lý phiên bản và hợp tác đơn giản}: Các image Docker có thể được quản lý phiên bản, cho phép các nhà phát triển theo dõi thay đổi và quay lại các trạng thái trước đó nếu cần. Quản lý phiên bản này đảm bảo tính nhất quán trên các giai đoạn khác nhau của vòng đời phát triển và tăng cường hợp tác giữa các nhóm phát triển.
        \end{itemize}
        
        % \item \textit{Nhược điểm}

        % \begin{itemize}
        %     \item \textbf{Hạn chế với ứng dụng GUI}: Docker chủ yếu hỗ trợ các ứng dụng dựa trên dòng lệnh và không phù hợp với các ứng dụng yêu cầu giao diện người dùng đồ họa phong phú. Việc chạy ứng dụng GUI trong Docker có thể gặp khó khăn và không được hỗ trợ tốt.
        %     \item \textbf{Phụ thuộc vào hệ điều hành Linux}: Docker hoạt động tốt nhất trên hệ điều hành Linux. Trên Windows và macOS, Docker cần sử dụng máy ảo Linux để chạy các container, điều này có thể ảnh hưởng đến hiệu suất và khả năng tương thích.
        %     \item \textbf{Quản lý tài nguyên phức tạp}: Việc chia sẻ tài nguyên giữa các container có thể dẫn đến xung đột hoặc giảm hiệu suất nếu không được quản lý đúng cách. Nếu một container chiếm dụng quá nhiều tài nguyên, nó có thể ảnh hưởng đến các container khác trên cùng hệ thống.
        % \end{itemize}
    \end{enumerate}
\subsection{Kiến trúc hệ thống}

\begin{figure}[H]
    \centering
    \includegraphics[width=10cm]{Images/kien-truc-he-thong.drawio.png}
    \vspace{0.5cm}
    \caption{Kiến trúc hệ thống}
    \label{fig:my_label}
\end{figure}

% Describing the importance of system architecture
Kiến trúc hệ thống đóng vai trò quan trọng trong dự án hệ thống quản lý nhà hàng. Nó xác định các thành phần chính của hệ thống và cách chúng tương tác với nhau để đáp ứng yêu cầu của người dùng. Kiến trúc này đảm bảo sự phân tách rõ ràng và sắp xếp logic giữa các thành phần, giúp dễ dàng quản lý, nâng cấp, mở rộng, tích hợp và tương tác giữa các thành phần khác nhau. Đồng thời, nó cung cấp khả năng mở rộng linh hoạt và thay đổi trong tương lai.

% Describing the architecture flow of the "Menu+" system
Dòng luồng kiến trúc của hệ thống "Menu+" được mô tả như sau:
\begin{itemize}
    \item Người dùng truy cập vào URL của hệ thống "Menu+" thông qua Internet.
    \item Khi truy cập, giao diện frontend của hệ thống được tải lên, bao gồm:
    \begin{itemize}
        \item Thư viện giao diện: ReactJS.
        \item Quản lý dữ liệu: TanStack.
        \item Thiết kế/Bố cục: ShadCN.
    \end{itemize}
    \item Nếu có hành động liên quan đến truy cập cơ sở dữ liệu hoặc dịch vụ bên thứ ba, hệ thống sử dụng TanStack để gửi yêu cầu đến server backend.
    \item Tại backend, Spring Boot Controller xử lý các yêu cầu API và ánh xạ chúng vào các URL tương ứng.
    \item Tiếp theo, tầng Service trong Spring Boot xác định hành động cần thực hiện dựa trên yêu cầu API.
    \item Nếu yêu cầu API cần truy cập cơ sở dữ liệu, Hibernate (ORM) sử dụng các entity để tạo truy vấn đến cơ sở dữ liệu PostgreSQL.
    \item Hibernate thực hiện truy vấn đến PostgreSQL, nơi lưu trữ dữ liệu của hệ thống.
    \item Để tối ưu hiệu suất, Service kiểm tra dữ liệu trong Redis (Caching/In-memory Data Store) trước; nếu không có, hệ thống truy vấn PostgreSQL và lưu kết quả vào Redis.
    \item Cuối cùng, Controller trả về kết quả dưới dạng JSON thông qua giao thức REST về frontend, và frontend hiển thị kết quả cho người dùng thông qua các UI Component.
\end{itemize}

% Describing how the system applies the MVC model with REST
Hệ thống "Menu+" áp dụng mô hình MVC (Model-View-Controller) kết hợp với REST API như sau:
\begin{itemize}
    \item \textbf{Model}: Bao gồm các thành phần liên quan đến dữ liệu và logic nghiệp vụ:
    \begin{itemize}
        \item \textit{Entity}: Định nghĩa cấu trúc dữ liệu (ví dụ: bảng Menu, Order) và ánh xạ với cơ sở dữ liệu PostgreSQL thông qua Hibernate.
        \item \textit{Service}: Chứa logic nghiệp vụ, xử lý các quy tắc kinh doanh (ví dụ: tính tổng hóa đơn, kiểm tra trạng thái món ăn).
        \item \textit{Caching}: Sử dụng Redis để lưu trữ dữ liệu truy cập thường xuyên, tối ưu hóa hiệu suất.
    \end{itemize}
    \item \textbf{Controller}: Được triển khai bởi Spring Boot Controller, chịu trách nhiệm xử lý các yêu cầu HTTP (GET, POST, PUT, DELETE) từ frontend, gọi đến Service để xử lý logic, và trả về phản hồi dưới dạng JSON qua REST API.
    \item \textbf{View}: Trong REST, View không phải là giao diện trực tiếp mà là dữ liệu JSON được trả về từ Controller. Frontend (ReactJS, TanStack, ShadCN) nhận dữ liệu này và hiển thị giao diện người dùng thông qua các UI Component.
\end{itemize}

% Concluding the architecture description
Kiến trúc này không chỉ đảm bảo sự phân tách rõ ràng giữa các tầng mà còn tận dụng REST API để giao tiếp hiệu quả giữa backend và frontend, đồng thời áp dụng MVC để tổ chức mã nguồn một cách logic và dễ bảo trì. \\

\begin{figure}[H]
    \centering
    \includegraphics[width=10cm]{Images/kthtm.png}
    \vspace{0.5cm}
    \caption{Sơ đồ luồng kiến trúc hệ thống theo module chức năng}
    \label{fig:my_label}
\end{figure}

Sơ đồ luồng kiến trúc hệ thống được thiết kế để minh họa mối quan hệ và luồng dữ liệu giữa các module chức năng trong hệ thống quản lý nhà hàng. Các module được chia thành ba nhóm chính: Quản lý hệ thống (System Management), Quản lý thực thực đơn và đặt chỗ (Menu & Booking), và Bán hàng & Vận hành (Sales & Operations). Mỗi module, từ quản lý lịch làm việc (MD-01) đến báo cáo doanh thu (MD-09), được kết nối với nhau thông qua các luồng dữ liệu rõ ràng, hỗ trợ các chức năng như đặt chỗ, bán hàng tại chỗ, giao hàng và tích hợp với hệ thống bên ngoài như Shipday. Sơ đồ sử dụng bố cục lưới với các đường dẫn cong để đảm bảo tính trực quan và dễ theo dõi.

% Hệ thống bao gồm các thành phần như sau:
% \begin{enumerate}
%     \item \textbf{Frontend}
%     \begin{itemize}
%         \item Được phát triển bằng ReactJS.
%         \item Giao tiếp với backend thông qua RESTful API.
%         \item Cung cấp giao diện người dùng trực quan, tối ưu trải nghiệm (UX/UI), hỗ trợ responsive trên nhiều thiết bị.
%         \item Xử lý logic hiển thị, xác thực người dùng và tương tác với API.
%     \end{itemize}
%     \item \textbf{Backend}
%     \begin{itemize}
%         \item Được xây dựng bằng Spring Boot, theo mô hình Modular Monolithic và tổ chức theo nguyên tắc Domain-Driven Design (DDD).
%         \item Cung cấp API RESTful để frontend và các hệ thống khác có thể truy cập dữ liệu.
%         \item Xử lý các nghiệp vụ cốt lõi của hệ thống, bao gồm quản lý đặt bàn, thực đơn, đơn hàng, thanh toán, báo cáo, v.v.
%         \item Sử dụng Spring Security để xác thực và phân quyền người dùng.
%         \item Hỗ trợ caching bằng Redis để tăng hiệu suất truy vấn dữ liệu.
%     \end{itemize}
%     \item \textbf{Cơ sở dữ liệu}
%     \begin{itemize}
%         \item Sử dụng PostgreSQL làm hệ quản trị cơ sở dữ liệu chính.
%         \item Được thiết kế theo nguyên tắc CQRS (Command Query Responsibility Segregation) nhằm tối ưu hóa hiệu suất đọc/ghi.
%     \end{itemize}
%     \item \textbf{Third-party Integrations}
%     \begin{itemize}
%         \item Hỗ trợ kết nối với các cổng thanh toán như VNPay, Momo, Stripe để xử lý giao dịch.
%         \item Tích hợp với các dịch vụ bên ngoài như SMS, Email (SendGrid, Twilio), CRM, POS để mở rộng chức năng của hệ thống.
%     \end{itemize}


% \subsection{Kiến trúc phần mềm}
% % \subsubsection{Mô hình MVC \cite{MVC}}
% MVC là viết tắt của khái niệm "Model-View-Controller", một trong những mô hình thiết kế phần mềm phổ biến nhất.\\

% MVC tách biệt dữ liệu, giao diện người dùng và logic xử lý thành ba thành phần riêng biệt nhưng vẫn được kết nối chặt chẽ với nhau.

% \begin{itemize}
%     \item Model (M): Đại diện cho dữ liệu và logic xử lý các nghiệp vụ của ứng dụng.
%     \item View (V): Quản lý giao diện và hiển thị dữ liệu ra cho người dùng.
%     \item Controller (C): Làm điều phối và điều hướng tương tác giữa Model và View. Nó nhận yêu cầu từ View, thực hiện xử lý trên Model và trả kết quả về cho View hiển thị.
% \end{itemize}

% MVC giúp tách biệt các thành phần của ứng dụng, tăng tính bảo trì và khả năng mở rộng trong tương lai.\\

% \begin{figure}[H]
%     \centering
%     \includegraphics[width=10cm]{Images/ezgif-4-c53032b3fd.png}
%     \vspace{0.5cm}
%     \caption{MVC là gì?}
%     \label{fig:my_label}
% \end{figure}

% Mô hình MVC (MVC pattern) thường được dùng để phát triển giao diện người dùng. Nó cung cấp các thành phần cơ bản để thiết kế một chương trình cho máy tính hoặc điện thoại di động, cũng như là các ứng dụng web.

% \subsubsection{Các thành phần của MVC}
% Mô hình MVC gồm 3 loại chính là thành phần bên trong không thể thiếu khi áp dụng mô hình này:
% \begin{figure}[H]
%     \centering
%     \includegraphics[width=10cm]{Images/cacthanhphanmvc.png}
%     \vspace{0.5cm}
%     \caption{Thành phần của MVC}
%     \label{fig:my_label}
% \end{figure}

% \begin{itemize}
%     \item Model: Là bộ phận có chức năng lưu trữ toàn bộ dữ liệu của ứng dụng và là cầu nối giữa 2 thành phần bên dưới là View và Controller. Một model là dữ liệu được sử dụng bởi chương trình. Đây có thể là cơ sở dữ liệu, hoặc file XML bình thường hay một đối tượng đơn giản. Chẳng hạn như biểu tượng hay là một nhân vật trong game.
%     \item View: Đây là phần giao diện (theme) dành cho người sử dụng. View là phương tiện hiển thị các đối tượng trong một ứng dụng. Chẳng hạn như hiển thị một cửa sổ, nút hay văn bản trong một cửa sổ khác. Nó bao gồm bất cứ thứ gì mà người dùng có thể nhìn thấy được.
%     \item Controller: Là bộ phận có nhiệm vụ xử lý các yêu cầu người dùng đưa đến thông qua View. Một controller bao gồm cả Model lẫn View. Nó nhận input và thực hiện các update tương ứng.
% \end{itemize}

% \subsubsection{Luồng xử lý trong MVC}
% Luồng xử lý trong của mô hình MVC, bạn có thể hình dung cụ thể và chi tiết qua từng bước dưới đây:
% \begin{itemize}
%     \item Khi một yêu cầu của từ máy khách (Client) gửi đến Server. Thì bị Controller trong MVC chặn lại để xem đó là URL request hay sự kiện.
%     \item Sau đó, Controller xử lý input của user rồi giao tiếp với Model trong MVC.
%     \item Model chuẩn bị data và gửi lại cho Controller.
%     \item Cuối cùng, khi xử lý xong yêu cầu thì Controller gửi dữ liệu trở lại View và hiển thị cho người dùng trên trình duyệt.
% \end{itemize}
% \begin{figure}[H]
%     \centering
%     \includegraphics[width=10cm]{Images/luongmvc.png}
%     \vspace{0.5cm}
%     \caption{View và Model sẽ được xử lý bởi Controller}
%     \label{fig:my_label}
% \end{figure}

% Ở đây, View không giao tiếp trực tiếp với Model. Sự tương tác giữa View và Model sẽ chỉ được xử lý bởi Controller.
\subsubsection{Ưu và nhược điểm của MVC}
\textbf{Ưu điểm mô hình MVC}
\begin{itemize}
    \item Đầu tiên, nhắc tới ưu điểm mô hình MVC thì đó là băng thông (Bandwidth) nhẹ vì không sử dụng viewstate nên khá tiết kiệm băng thông. Việc giảm băng thông giúp website hoạt động ổn định hơn.
    \item Kiểm tra đơn giản và dễ dàng, kiểm tra lỗi phần mềm trước khi bàn giao lại cho người dùng.
    \item Một lợi thế chính của MVC là nó tách biệt các phần Model, Controller và View với nhau.
    \item Sử dụng mô hình MVC chức năng Controller có vai trò quan trọng và tối ưu trên các nền tảng ngôn ngữ khác nhau
    \item Ta có thể dễ dàng duy trì ứng dụng vì chúng được tách biệt với nhau.
    \item Có thể chia nhiều developer làm việc cùng một lúc. Công việc của các developer sẽ không ảnh hưởng đến nhau.
    \item Hỗ trợ TTD (test-driven development). Chúng ta có thể tạo một ứng dụng với unit test và viết các won test case.
    \item Phiên bản mới nhất của MVC hỗ trợ trợ thiết kế responsive website mặc định và các mẫu cho mobile. Chúng ta có thể tạo công cụ View của riêng mình với cú pháp đơn giản hơn nhiều so với công cụ truyền thống.
\end{itemize}
\textbf{Nhược điểm mô hình MVC}\\

MVC đa phần phù hợp với công ty chuyên về website hoặc các dự án lớn thì mô hình này phù hợp hơn so với với các dự án nhỏ, lẻ vì khá là cồng kềnh và mất thời gian.

\begin{itemize}
    \item Không thể Preview các trang như ASP.NET.
    \item Khó triển khai.
\end{itemize}
\subsubsection{Lý do sử dụng MVC}
\textbf{Quy trình phát triển nhanh hơn}\\

MVC hỗ trợ phát việc phát triển nhanh chóng và song song. Nếu một mô hình MVC được dùng để phát triển bất kỳ ứng dụng web cụ thể nào, một lập trình viên có thể làm việc trên View và một developer khác có thể làm việc với Controller để tạo logic nghiệp vụ cho ứng dụng web đó.\\

Do đó, ứng dụng mô hình MVC có thể được hoàn thành nhanh hơn ba lần so với các ứng dụng mô hình khác.\\

\textbf{Khả năng cung cấp nhiều chế độ view}\\

Trong mô hình MVC, bạn có thể tạo nhiều View cho chỉ một mô hình. Ngày nay, nhu cầu có thêm nhiều cách mới để truy cập ứng dụng và đang ngày càng tăng. Do đó, việc sử dụng MVC để phát triển chắc chắn là một giải pháp tuyệt vời.\\

Hơn nữa, với phương pháp này, việc nhân bản code rất hạn chế. Vì nó tách biệt dữ liệu và logic nghiệp vụ khỏi màn hình.\\

\textbf{Các sửa đổi không ảnh hưởng đến toàn bộ mô hình}\\

Đối với bất kỳ ứng dụng web nào, người dùng có xu hướng thay đổi thường xuyên. Bạn có thể quan sát thông qua những thay đổi thường xuyên về màu sắc, font chữ, bố cục màn hình. Hay là thêm hỗ trợ thiết bị mới cho điện thoại hay máy tính bảng…\\

Việc thêm một kiểu view mới trong MVC rất đơn giản. Vì phần Model không phụ thuộc vào phần View. Do đó, bất kỳ thay đổi nào trong Model sẽ không ảnh hưởng đến toàn bộ kiến trúc.\\

\textbf{MVC Model trả về dữ liệu mà không cần định dạng}\\

MVC pattern có thể trả về dữ liệu mà không cần áp dụng bất kỳ định dạng nào. Do đó, các thành phần giống nhau có thể được sử dụng với bất kỳ giao diện nào.\\

Ví dụ: tất cả loại dữ liệu đều có thể được định dạng bằng HTML. Ngoài ra, nó cũng có thể được định dạng bằng Macromedia Flash hay Dream Viewer.\\


\subsection{Thiết kế Sitemap}

\begin{figure}[H]
    \centering
    \includegraphics[width=\linewidth]{Images/sitemap.png}
    \vspace{0.5cm}
    \caption{Sitemap của hệ thống}
    \label{fig:my_label}
\end{figure}

Sitemap này mô tả cấu trúc và các tính năng chính của Hệ thống quản lý đặt món cho nhà hàng, nêu chi tiết các chức năng và mô-đun cốt lõi của nó.

\begin{enumerate}
    \item \textbf{Tất cả người dùng}
        \begin{itemize}
            \item \textit{Homepage}: Trang chính của ứng dụng, hiển thị thông tin giới thiệu về nhà hàng, các danh mục chính (Menu, Đặt hàng, Liên hệ), và nút đăng nhập/đăng ký để chuyển đến các cổng phù hợp với vai trò người dùng.
            \item \textit{About Us (Giới thiệu)}: Cung cấp thông tin về nhà hàng, bao gồm lịch sử, giá trị cốt lõi, đội ngũ, và các chi nhánh.
            \item \textit{Menu}: Hiển thị danh sách các món ăn và khuyến mãi hiện có, phân loại theo loại (đồ ăn, đồ uống, món tráng miệng), cho phép người dùng xem chi tiết món ăn và thêm vào giỏ hàng.
            \item \textit{Food Details (Chi tiết món ăn)}: Food Details (Chi tiết món ăn): Hiển thị thông tin chi tiết về một món ăn cụ thể (tên, giá, mô tả, hình ảnh, thành phần, đánh giá), cho phép thêm vào giỏ hàng hoặc quay lại danh sách menu.
            \item \textit{Cart (Giỏ hàng)}: Hiển thị danh sách món ăn đã chọn, cho phép chỉnh sửa số lượng và ghi chú, xóa món, và chọn phương thức đặt hàng (Dine-In Order hoặc Delivery Order).
            \item \textit{Dine-In Order (Đặt món tại quán)}: Cho phép người dùng xác nhận đơn hàng để ăn tại quán, cho phép chuyển sang trang thanh toán khi hoàn thành bữa ăn.
            \item \textit{Payment (Thanh toán)}: Xử lý thanh toán cho đơn hàng tại quán, hỗ trợ các phương thức thanh toán (tiền mặt, thẻ, QR code), và hiển thị xác nhận sau khi thanh toán thành công.
            \item \textit{Contact Us (Liên hệ)}: Cung cấp thông tin liên hệ (số điện thoại, email, địa chỉ), và biểu mẫu để khách hàng gửi câu hỏi hoặc phản hồi.
        \end{itemize}
    \item \textbf{Khách hàng đã có tài khoản}
        \begin{itemize}
            \item \textit{Profile Management (Quản lý thông tin cá nhân)}: Cho phép khách hàng xem và chỉnh sửa thông tin cá nhân (tên, số điện thoại, email, địa chỉ), quản lý tài khoản và mật khẩu.
            \item \textit{Table Reservation (Đặt bàn)}: Cho phép khách hàng chọn thời gian, số lượng người, xem sơ đồ và đặt bàn tại nhà hàng, với xác nhận qua email hoặc tin nhắn. Nếu đang có yêu cầu đặt bàn thì cũng sẽ hiển thị và hủy ở đây.
            \item \textit{Delivery Order (Đặt món giao về)}: Cho phép người dùng chọn giao hàng, nhập thông tin giao hàng (địa chỉ, số điện thoại), và chuyển sang trang thanh toán.
            \item \textit{Order Summary (Tóm tắt đơn hàng)}: Hiển thị tất cả đơn hàng (món ăn, số lượng, tổng tiền, thông tin giao hàng), cho phép chuyển qua trang xem chi tiết.
            \item \textit{View Order Details (Xem chi tiết đơn hàng)}: Hiển thị trạng thái và thông tin chi tiết của một đơn hàng đã đặt (thời gian, địa chỉ, tình trạng giao hàng), cho phép khách hàng theo dõi. Nếu đơn hàng đã thành công, có thể chuyển đến trang gửi đánh giá.
            \item \textit{Submit Review (Gửi đánh giá)}: Cho phép người dùng viết đánh giá hoặc xếp hạng cho món ăn hoặc dịch vụ sau khi hoàn tất đơn hàng, gửi lên hệ thống.
        \end{itemize}
    \item \textbf{Nhân viên}
    
        \textbf{\textit{Tất cả nhân viên}}
        \begin{itemize}
            \item \textit{View word schedule (Xem lịch làm)}: Hiển thị lịch làm việc được xếp.
        \end{itemize}
        \textbf{\textit{Nhân viên Phục vụ}}
        \begin{itemize}
            \item \textit{Table List (Danh sách bàn)}: Hiển thị trạng thái các bàn (trống, đang sử dụng, đặt trước), cho phép nhân viên phục vụ cập nhật hoặc quản lý.
            \item \textit{Order List (Danh sách đơn hàng)}: Hiển thị danh sách các đơn hàng đang chờ xử lý, cho phép nhân viên chọn đơn để xem chi tiết hoặc chuyển sang bếp.
            \item \textit{Order Details (Chi tiết đơn hàng)}: Hiển thị thông tin chi tiết của một đơn hàng (món ăn, số lượng, ghi chú), cho phép nhân viên xác nhận hoặc cập nhật trạng thái.
        \end{itemize}
            
        \textit{\textbf{Nhân viên Thu ngân}}
        \begin{itemize}
            \item \textit{Process Payment (Xử lý thanh toán)}: Cho phép nhân viên thu ngân nhập thông tin thanh toán, xử lý các phương thức (tiền mặt, thẻ), và xuất hóa đơn.
            \item \textit{Cash In/Out}: Thêm thông tin vào dòng tiền bên ngoài.
            \item \textit{Daily Revenue Report (Báo cáo doanh thu hàng ngày)}: Hiển thị tổng doanh thu, số lượng đơn hàng, và các thống kê khác trong ngày, hỗ trợ nhân viên thu ngân theo dõi.
        \end{itemize}
            
        \textit{\textbf{Nhân viên Bếp}}
        \begin{itemize}
            \item \textit{Area Display (Các khu vực bếp): Hiển thị danh sách các khu vực bếp tương ứng với các món ăn được chỉ định sẵn.}
            \item \textit{Pending Dishes (Món ăn đang chờ)}: Hiển thị danh sách món ăn cần chuẩn bị, cho phép nhân viên bếp theo dõi và cập nhật trạng thái hoàn thành.
        \end{itemize}
            
        \textit{\textbf{Nhân viên Vận hành}}
        \begin{itemize}
            \item \textit{Manage Reservations (Quản lý đặt bàn)}: Cho phép nhân viên xem, chỉnh sửa, hoặc hủy đặt bàn (bao gồm hủy khi khách đến trễ quá), cập nhật trạng thái bàn.
            \item \textit{Contact Requests (Yêu cầu liên hệ)}: Hiển thị danh sách yêu cầu hỗ trợ từ khách hàng, cho phép nhân viên hỗ trợ trả lời hoặc chuyển tiếp.
            \item \textit{Promotions (Khuyến mãi)}: Hiển thị danh sách các chương trình khuyến mãi hiện tại, cho phép quản lý tạo mới hoặc chỉnh sửa.
            \item \textit{Promotion Details (Chi tiết khuyến mãi)}: Hiển thị thông tin chi tiết của từng chương trình khuyến mãi (thời gian, điều kiện, ưu đãi), cho phép chỉnh sửa hoặc xóa. Thống kê số liệu về chương trình khuyến mãi
        \end{itemize}
    \item \textbf{Quản lý}
        \begin{itemize}
            \item \textit{Staff Management (Quản lý nhân viên)}: Cho phép quản lý thêm, chỉnh sửa, hoặc xóa thông tin nhân viên (tên, vai trò, lịch làm việc).
            \item \textit{Staff Details (Chi tiết nhân viên)}: Hiển thị thông tin chi tiết của từng nhân viên (hồ sơ, lịch sử làm việc, hiệu suất), cho phép chỉnh sửa.
            \item \textit{Branch Details (Chi tiết chi nhánh)}: Hiển thị thông tin chi tiết của chi nhánh, cho phép chỉnh sửa sơ đồ nhà hàng.
            \item \textit{Branch Dashboard (Bảng điều khiển chi nhánh)}: Hiển thị thông tin cụ thể của từng chi nhánh (doanh thu, đơn hàng, nhân viên), hỗ trợ quản lý theo dõi.
            \item \textit{Distribute work schedule (Xếp lịch làm)}: Cho phép lập lịch trực qua, thêm ghi chú công việc cho nhân viên.
        \end{itemize}
        
        \textit{\textbf{Dành cho quản trị viên}}
        \begin{itemize}
            \item \textit{Dashboard (Bảng điều khiển)}: Hiển thị tổng quan về hoạt động nhà hàng (doanh thu, số đơn hàng, trạng thái nhân viên), với các biểu đồ và số liệu chính.
            \item \textit{Branch Management (Quản lý chi nhánh)}: Cho phép quản lý thêm, chỉnh sửa, hoặc xóa thông tin chi nhánh (địa chỉ, số điện thoại, giờ hoạt động).
            \item \textit{Menu Management (Quản lý menu)}: Cho phép quản lý thêm, chỉnh sửa, hoặc xóa món ăn trong menu (tên, giá, hình ảnh, mô tả). Thống kế các thông tin số liệu về món ăn.
            \item \textit{Food Details (Chi tiết món ăn)}: Hiển thị và chỉnh sửa thông tin chi tiết của từng món ăn trong menu (dành cho quản lý).
        \end{itemize}
    
\end{enumerate}

\subsection{Thiết kế Database}
\subsubsection{EERD Database}

Hệ thống cơ sở dữ liệu quản lý một chuỗi nhà hàng, tập trung vào việc quản lý người dùng, nhân viên, khách hàng, đặt bàn, gọi món, hóa đơn thanh toán, chương trình khuyến mãi, phản hồi và hỗ trợ khách hàng. Người dùng (USER) lưu trữ các thông tin như tên đăng nhập, mật khẩu đã mã hóa, họ tên đầy đủ và địa chỉ email. Một người dùng có thể là một nhân viên (STAFF), đảm nhiệm các vai trò như thu ngân (CASHIER), đầu bếp (CHEF), nhân viên phục vụ (WAITER), nhân viên vệ sinh (CLEANING STAFF), nhân viên vận hành (OPERATION STAFF). Người dùng cũng có thể là quản lý (MANAGER) hoặc khách hàng (CUSTOMER). Mỗi nhân viên bắt buộc phải làm việc tại một chi nhánh (BRANCH) cụ thể. Mỗi chi nhánh lưu trữ các thông tin về tên, địa chỉ và số điện thoại liên hệ, đồng thời phải có một bản cấu hình (CONFIGURATION) riêng biệt để thiết lập các chính sách như tiền đặt cọc bàn, tỷ lệ đặt cọc khi đặt món trước, thời gian tự động gọi xác nhận đơn hàng hoặc các thông số kỹ thuật khác. Mỗi chi nhánh có đúng một người quản lý (MANAGER) chịu trách nhiệm vận hành toàn bộ hoạt động tại chi nhánh đó.

Nhân viên làm việc theo ca (SHIFT), mỗi ca ghi nhận giờ bắt đầu, giờ kết thúc, ghi chú đặc biệt nếu có và trạng thái như nháp, xung đột, hoàn thành hoặc đã lên lịch. Nhân viên nhà bếp sẽ được phân công làm việc tại các khu vực bếp (KITCHEN STATION) thuộc từng chi nhánh riêng biệt, mỗi khu vực bếp có tên và mô tả cụ thể. Một chi nhánh phải có ít nhất một khu vực bếp được cấu hình trước khi có thể bắt đầu vận hành.

Nhà hàng bố trí nhiều bàn ăn (TABLE) với các thông tin về số lượng chỗ ngồi và vị trí cụ thể trên bản đồ nhà hàng. Bàn có các trạng thái vận hành như trống, đang sử dụng hoặc chờ làm sạch. Khách hàng có thể thực hiện việc đặt bàn trước (BOOKING TABLE), ghi lại số lượng khách, thời gian ăn, các ghi chú thêm và trạng thái đặt bàn như đã xác nhận, đã hủy hoặc thành công. Khách hàng cũng có thể đặt món trước (BOOKING DISH), ghi nhận các món ăn mong muốn, số lượng từng món, thời gian ăn dự kiến tới ăn và ghi chú riêng cho từng món nếu cần.

Món ăn (DISH) lưu trữ thông tin về tên món, mô tả, kích cỡ phần ăn, giá tiền, thời gian chế biến ước tính và trạng thái hoạt động. Mỗi món ăn được tạo thành từ nhiều nguyên liệu. Ngoài ra, nhà hàng còn xây dựng các combo (COMBO) kết hợp nhiều món ăn với giá ưu đãi để bán cho khách.

Khi khách hàng gọi món hoặc đặt món, hệ thống tạo ra đơn hàng (ORDER) với thông tin loại đơn (ăn tại chỗ, mang đi, giao hàng), trạng thái đơn hàng (thành công, đang thực hiện, đã hủy...), tiền cọc nếu có và tổng giá trị đơn hàng, được tính bằng cách lấy giá từng món ăn hoặc combo (áp dụng giá khuyến mãi nếu có) nhân với số lượng rồi cộng lại. Đối với đơn giao hàng sẽ có thêm thông tin vận chuyển. Mỗi đơn hàng có thể gắn với một hoặc nhiều hóa đơn (INVOICE) trong trường hợp chia nhỏ hóa đơn, ghi lại tổng số tiền thanh toán (sau khi áp dụng voucher nếu có, trừ đi tiền cọc), thuế VAT, thời gian lập hóa đơn và loại hóa đơn (hóa đơn gộp hoặc hóa đơn lẻ). Khách hàng thanh toán qua các phương thức thanh toán (PAYMENT) khác nhau như tiền mặt, ví điện tử hoặc ebanking. Các khoản đặt cọc cho đặt bàn hoặc đặt món cũng được ghi nhận dưới dạng PAYMENT riêng biệt, với số tiền tính theo phần trăm cấu hình tại CONFIGURATION.

Hệ thống quản lý các mã giảm giá (VOUCHER), mỗi voucher có điều kiện áp dụng cụ thể, giới hạn số lần sử dụng, thời gian hiệu lực, lưu lại số phần trăm giảm giá hoặc số tiền giảm giá. Khách hàng có thể lưu trữ các voucher được phát hành để sử dụng về sau nếu voucher còn hiệu lực. Ngoài voucher, nhà hàng còn có các chương trình khuyến mãi (PROMOTION) diễn ra trong các khoảng thời gian xác định, giảm giá trực tiếp trên món ăn.

Khách hàng có thể gửi yêu cầu hỗ trợ (SUPPORT TICKET) với tiêu đề, nội dung cụ thể và trạng thái xử lý như đã xử lý, đang xử lý hoặc đang chờ bổ sung thông tin. Mỗi yêu cầu sẽ được phân công cho nhân viên vận hành để tiếp nhận và xử lý. Khách hàng cũng có thể gửi phản hồi (FEEDBACK) với nội dung giới hạn trong 500 ký tự về dịch vụ hoặc món ăn, gắn liền với từng đơn hàng, và đánh giá chất lượng từ 1 đến 5 sao. Trong trường hợp có sự cố, khách hàng có thể yêu cầu hoàn tiền (REFUND), yêu cầu này gắn liền với hóa đơn (INVOICE) liên quan để nhân viên có thể theo dõi và xử lý.

\begin{landscape}
\begin{figure}[H]
    \centering
    \includegraphics[height=0.85\textheight]{Images/db.png}
    \vspace{0.5cm}
    \caption{EERD của cơ sở dữ liệu}
    \label{fig:my_label}
\end{figure}
\end{landscape}

\subsubsection{Các ràng buộc}
\paragraph{Ràng buộc miền trị (Domain Constraints – Attribute Domains):}

\begin{itemize}
  \item \textbf{Trạng thái (SHIFT - STAFF)}
    \begin{itemize}
        \item \texttt{DRAFT}: Ca làm việc đang trong quá trình tạo, chưa được công bố.
        \item \texttt{PUBLISHED}: Ca làm việc đã được công bố và có hiệu lực.
        \item \texttt{CONFLICTED}: Ca làm việc bị xung đột về thời gian hoặc nhân sự.
    \end{itemize}

  \item \textbf{Trạng thái (SUPPORT TICKET)}
    \begin{itemize}
        \item \texttt{PENDING}: Yêu cầu đang chờ xử lý.
        \item \texttt{RECEIVED}: Yêu cầu đã được tiếp nhận.
        \item \texttt{IN\_PROGRESS}: Yêu cầu đang trong quá trình xử lý.
        \item \texttt{RESOLVED}: Yêu cầu đã được xử lý xong.
    \end{itemize}

  \item \textbf{Đánh giá (Feedback)}
    \begin{itemize}
        \item Các giá trị từ \texttt{1} đến \texttt{5}, thể hiện mức độ hài lòng của khách hàng.
    \end{itemize}

  \item \textbf{Mô tả (Feedback)}
    \begin{itemize}
        \item Chuỗi văn bản tối đa \texttt{500} ký tự để khách hàng mô tả chi tiết phản hồi.
    \end{itemize}

  \item \textbf{Loại đối tượng phản hồi (Target\_Type)}
    \begin{itemize}
        \item \texttt{FEEDBACK\_BOOKING\_TABLE}: Phản hồi về việc đặt bàn.
        \item \texttt{FEEDBACK\_BOOKING\_PRODUCT}: Phản hồi về món ăn đặt trước.
        \item \texttt{FEEDBACK\_ORDER}: Phản hồi về đơn hàng đã gọi.
        \item \texttt{GENERAL\_FEEDBACK}: Phản hồi chung, không phân loại cụ thể.
    \end{itemize}

  \item \textbf{Trạng thái (TABLE)}
    \begin{itemize}
        \item \texttt{AVAILABLE}: Bàn sẵn sàng sử dụng.
        \item \texttt{OCCUPIED}: Bàn đang có khách.
        \item \texttt{NEEDS\_CLEANING}: Bàn cần được dọn dẹp.
    \end{itemize}

  \item \textbf{Loại hình (BOOKING PRODUCT và ORDER)}
    \begin{itemize}
        \item \texttt{DINE\_IN}: Dùng món tại nhà hàng.
        \item \texttt{TAKE\_AWAY}: Mua mang về.
        \item \texttt{DELIVERY}: Giao hàng tận nơi.
    \end{itemize}

  \item \textbf{Trạng thái (BOOKING TABLE và BOOKING PRODUCT)}
    \begin{itemize}
        \item \texttt{BOOKED}: Đã đặt.
        \item \texttt{DEPOSIT\_PAID}: Đã thanh toán tiền cọc.
        \item \texttt{CANCELLED}: Đã hủy.
        \item \texttt{COMPLETED}: Đã hoàn tất dịch vụ.
    \end{itemize}

  \item \textbf{Loại sản phẩm (PRODUCT)}
    \begin{itemize}
        \item \texttt{CONSUMABLE}: Món ăn đếm được, ví dụ: chai rượu, hộp bánh.
        \item \texttt{STOCKABLE}: Món ăn không đếm đơn vị, ví dụ: tô cơm, nồi canh.
        \item \texttt{SERVICE}: Các dịch vụ khác.
    \end{itemize}

  \item \textbf{Trạng thái (ORDER)}
    \begin{itemize}
        \item \texttt{PLACED}: Đơn hàng đã được tạo.
        \item \texttt{PREPARING}: Đơn hàng đang được chuẩn bị.
        \item \texttt{COMPLETED}: Đơn hàng đã hoàn tất.
        \item \texttt{CANCELLED}: Đơn hàng đã bị hủy.
    \end{itemize}

  \item \textbf{Trạng thái (PRODUCT-ORDER)}
    \begin{itemize}
        \item \texttt{PENDING}: Món chờ chuẩn bị.
        \item \texttt{PREPARING}: Đang được chuẩn bị.
        \item \texttt{COMPLETED}: Đã chuẩn bị xong.
        \item \texttt{SERVED}: Đã phục vụ cho khách.
        \item \texttt{CANCELLED}: Đã hủy.
    \end{itemize}

  \item \textbf{Loại hóa đơn (INVOICE)}
    \begin{itemize}
        \item \texttt{NORMAL}: Hóa đơn bình thường cho một đơn hàng.
        \item \texttt{MERGED}: Hóa đơn gộp từ nhiều đơn hàng.
        \item \texttt{SPLIT}: Hóa đơn chia nhỏ từ một đơn hàng.
    \end{itemize}

  \item \textbf{Mục đích thanh toán (Payment\_For)}
    \begin{itemize}
        \item \texttt{DEPOSIT\_BOOKING\_TABLE}: Tiền cọc cho việc đặt bàn.
        \item \texttt{DEPOSIT\_BOOKING\_PRODUCT}: Tiền cọc cho đặt món trước.
        \item \texttt{INVOICE\_PAYMENT}: Thanh toán cho hóa đơn.
    \end{itemize}

  \item \textbf{Phương thức thanh toán (Payment\_Method)}
    \begin{itemize}
        \item \texttt{CASH}: Thanh toán bằng tiền mặt.
        \item \texttt{CARD}: Thanh toán qua thẻ (ghi nợ hoặc tín dụng).
        \item \texttt{EBANKING}: Thanh toán qua ngân hàng điện tử.
        \item \texttt{E\_WALLET}: Thanh toán qua ví điện tử (Momo, ZaloPay, v.v.)
    \end{itemize}
\end{itemize}

\paragraph{Ràng buộc tham chiếu/thời gian (Referential/Temporal Constraints):}
\begin{itemize}
  \item \textbf{POS Session}: Thời gian phiên làm việc phải trùng với thời gian của một ca làm việc (shift) tại cùng chi nhánh (branch).
  \item \textbf{Thời gian bắt đầu/kết thúc}: Mọi thời điểm kết thúc (end\_time) phải luôn lớn hơn thời điểm bắt đầu (start\_time).
  \item \textbf{Ràng buộc khuyến mãi (Promotion)}: Trong cùng một khoảng thời gian, mỗi sản phẩm (product) chỉ được áp dụng tối đa một khuyến mãi duy nhất.
\end{itemize}

\paragraph{Thuộc tính dẫn xuất (Derived Attributes):}
\begin{itemize}
  \item \textbf{closing\_cash (POS Session - Thu ngân)}: Tổng doanh thu trong phiên, sau khi đã trừ các khoản khuyến mãi và voucher.

  \item \textbf{total\_amount (Order)}: Tính bằng tổng \textit{(đơn giá món ăn hoặc giá khuyến mãi nếu có) \texttimes số lượng} cộng với \textit{giá trị combo \texttimes số lượng}, sau đó trừ đi khoản đặt cọc (nếu có).

  \item \textbf{total\_amount (Invoice)}: Tính từ \textit{total\_amount} của đơn hàng, áp dụng voucher tương ứng. Nếu là hóa đơn gộp thì cộng tổng nhiều đơn hàng. Nếu là hóa đơn chia thì lấy phần giá trị được chia.

  \item \textbf{tax\_amount (Invoice)}: Bằng \textit{total\_amount \texttimes thuế suất} do chi nhánh cấu hình.

  \item \textbf{total\_used (Voucher)}: Tổng số lần voucher đã được sử dụng trong các hóa đơn.

  \item \textbf{amount (Payment)}:
    \begin{itemize}
        \item \texttt{DEPOSIT\_BOOKING\_TABLE}: Số bàn \texttimes giá trị đặt cọc của chi nhánh.
        \item \texttt{DEPOSIT\_BOOKING\_PRODUCT}: Tổng giá trị món ăn \texttimes phần trăm đặt cọc theo chi nhánh.
        \item \texttt{INVOICE\_PAYMENT}: Bằng tổng \textit{total\_amount + tax\_amount} của hóa đơn.
    \end{itemize}
\end{itemize}

\subsubsection{Relational Mapping}

\begin{landscape}

\begin{figure}[H]
    \centering
    \includegraphics[height=0.9\textheight]{Images/relation.png}
    \vspace{0.5cm}
    \caption{Relational Mapping}
    \label{fig:my_label}
\end{figure}

\end{landscape}


\subsubsection{Lược đồ quan hệ}

\begin{figure}[H]
    \centering
    \includegraphics[height=0.9\textheight]{Images/ldqh.png}
    \vspace{0.5cm}
    \caption{Lược đồ quan hệ}
    \label{fig:my_label}
\end{figure}

% \subsection{Thiết kế cơ sở dữ liệu}
ER diagram là công cụ hữu hiệu giúp kiểm tra tính nhất quán và hoàn thiện trong thiết kế cơ sở dữ liệu. Nếu có sai sót hoặc mâu thuẫn trong hiểu biết ban đầu về hệ thống, chúng có thể được phát hiện và sửa chữa kịp thời thông qua ER diagram. Điều này góp phần loại bỏ các lỗi thiết kế và tối ưu hóa cơ sở dữ liệu.

\subsubsection{Bảng mô tả các schema}
Dưới đây là các bảng mô tả thuộc tính của các schema có trong lược đồ cơ sở dữ liệu quan hệ:

\begin{table}[H]
\centering
\renewcommand{\arraystretch}{1.5}
\begin{tabular}{|l|p{7cm}|p{4cm}|}
\hline
\textbf{Thuộc tính mô tả} & \textbf{Mô tả} & \textbf{Kiểu dữ liệu} \\
\hline
user\_id & ID của người dùng & bigint \\
\hline
username & Username của người dùng & varchar(255) \\
\hline
email & Email của người dùng & varchar(255) \\
\hline
password & Mật khẩu của người dùng đã được hash & varchar(255) \\
\hline
avatar\_url & URL dẫn đến avatar của người dùng & varchar(255) \\
\hline
user\_type & Loại người dùng (recruiter, candidate) & enum \\
\hline
\end{tabular}
\caption{Users Schema}
\end{table}


% \subsection{Quản lí source code}
 % phần này ghi cách làm việc với source code của nhóm: quy trình làm việc nhóm chuẩn: tạo nhánh -> làm -> đẩy code -> tạo merge req -> review -> merge. 

Ở đồ án này, nhóm sử dụng GitHub để quản lý sourcce code và tương tác với các thành viên khác. Nhằm tạo sự thống nhất, tiện lợi và hiệu quả trong quá trình làm việc nhóm với source code, chúng ta sẽ sử dụng \textbf{GitHub flow branching strategy } được đề xuất bởi chính GitHub. \\
% Nhớ tạo ref cho cái này https://docs.github.com/en/get-started/quickstart/github-flow

GitHub flow branching strategy là một quy trình làm việc tương đối đơn giản cho phép các nhóm nhỏ hoặc các ứng dụng/sản phẩm web không yêu cầu hỗ trợ nhiều phiên bản để nhanh chóng hoàn thành công việc của họ. \\

Trong GitHub flow, nhánh chính chứa mã nguồn đã sẵn sàng cho sản xuất. Các nhánh khác, được gọi là nhánh tính năng, nên chứa công việc về các tính năng mới và sửa lỗi, và sẽ được hợp nhất trở lại nhánh chính khi công việc hoàn thành và được đánh giá đúng cách. 
\subsubsection{GitHub Flow Best Practices}
Khi làm việc với chiến lược GitHub flow branching, có sáu nguyên tắc cần nên tuân thủ để đảm bảo duy trì mã nguồn tốt:
\begin{itemize}
    \item Mọi mã nguồn trong nhánh main nên có thể triển khai.
    \item Tạo các nhánh mới có tên mô tả rõ ràng từ nhánh chính để thực hiện công việc mới, chẳng hạn như features/add-new-payment-types.
    \item Commit công việc mới vào các nhánh local và thường xuyên push công việc lên remote.
    \item Để yêu cầu phản hồi hoặc sự giúp đỡ, hoặc khi nghĩ rằng công việc của mình đã sẵn sàng để hợp nhất vào nhánh chính, mở một pull request.
    \item Sau khi công việc hoặc tính năng của bạn đã được xem xét và chấp nhận, nó có thể được hợp nhất vào nhánh main.
    \item Sau khi công việc đã được hợp nhất vào nhánh main, nó nên được triển khai ngay lập tức.
\end{itemize}

\subsubsection{GitHub Flow: Lợi ích và nhược điểm}
\begin{figure}[H]
    \centering
    \includegraphics[width=8cm]{Images/githubflow.png}
    \vspace{0.5cm}
    \caption{Mô phỏng các branch trong 1 GitHub flow}
\end{figure}
\begin{itemize}
    \item Lợi ích:
    \begin{itemize}
    \item Do tính đơn giản của quy trình làm việc, chiến lược này cho phép triển khai liên tục và tích hợp liên tục.
    \item Chiến lược này hoạt động tốt cho các nhóm nhỏ và ứng dụng web.
    \end{itemize}

    \item Nhược điểm: 
    \begin{itemize}
    \item Chiến lược này không thể hỗ trợ nhiều phiên bản mã nguồn cùng một lúc trên môi trường production.
    \item Sự thiếu hụt các nhánh phát triển riêng biệt khiến cho GitHub Flow trở nên dễ bị ảnh hưởng bởi lỗi trong môi trường production.
    \end{itemize}
\end{itemize}

\subsubsection{Các bước trong GitHub flow}
Khi cần thực hiện một tính năng mới trong hệ thống thì mỗi thành viên phải hoàn thiện hết các bước cần thiết như sau.
\begin{enumerate}
    \item \textbf{Tạo nhánh}
    \begin{itemize}
        \item Tạo một nhánh trong repository. Một tên nhánh ngắn gọn và mô tả sẽ giúp mọi gnười dễ dàng nhìn thấy công việc đang diễn ra và dựa theo cú pháp sau: features/{feature-description} 
        \item Bằng cách tạo một nhánh, bạn tạo ra một không gian để làm việc mà không ảnh hưởng đến nhánh mặc định. Ngoài ra, điều này cũng tạo cơ hội cho mọi người xem xét công việc của bạn.
    \end{itemize}

    \item \textbf{Thực hiện các thay đổi}
    \begin{itemize}
        \item Nhánh của bạn là một nơi an toàn để thực hiện các thay đổi. Nếu bạn mắc lỗi, bạn có thể hoàn nguyên các thay đổi hoặc đẩy các thay đổi bổ sung để sửa lỗi. Các thay đổi của bạn sẽ không xuất hiện trên nhánh mặc định cho đến khi bạn hợp nhất nhánh của mình.
        \item Commit và đẩy các thay đổi lên nhánh của bạn. Đặt một thông điệp mô tả cho mỗi commit để giúp bạn và các đóng góp viên trong tương lai hiểu rõ những thay đổi mà commit chứa. Ví dụ, "fix typo" hoặc "increase rate limit". 
        \item Lý tưởng là mỗi commit chứa một thay đổi độc lập, hoàn chỉnh. Điều này giúp dễ dàng hoàn nguyên các thay đổi nếu bạn quyết định thay đổi hướng tiếp cận. Ví dụ, nếu bạn muốn đổi tên một biến và thêm một số tests, hãy đặt thay đổi tên biến trong một commit và các tests trong một commit khác. Sau này, nếu bạn muốn giữ lại các tests nhưng hoàn nguyên thay đổi tên biến, bạn có thể hoàn nguyên commit cụ thể chứa thay đổi tên biến. Nếu bạn đặt thay đổi tên biến và tests trong cùng một commit hoặc phân tán thay đổi tên biến qua nhiều commit, bạn sẽ phải mất nhiều công sức hơn để hoàn nguyên các thay đổi của mình.
        \item Bằng cách commit và đẩy các thay đổi của bạn, bạn sao lưu công việc của mình lên lưu trữ từ xa. Điều này có nghĩa là bạn có thể truy cập công việc của mình từ bất kỳ thiết bị nào. Đồng thời, mọi người cũng có thể xem công việc của bạn, trả lời câu hỏi, và đưa ra đề xuất hoặc đóng góp.
        \item Tiếp tục thực hiện, commit, và đẩy các thay đổi lên nhánh của bạn cho đến khi bạn sẵn sàng để yêu cầu phản hồi.
    \end{itemize}
    

    \item \textbf{Tạo pull request}
    \begin{itemize}
        \item Tạo một pull request để yêu cầu đồng đội đưa ra phản hồi về các thay đổi của bạn. Phản hồi từ việc xem xét pull request là rất quan trọng, đến nỗi một số repository yêu cầu một xem xét chấp thuận trước khi có thể hợp nhất pull request. Nếu bạn muốn có phản hồi sớm hoặc tư vấn trước khi hoàn tất các thay đổi của bạn, bạn có thể đánh dấupull request của mình như một bản "draft".

        \item Khi bạn tạo một pull request, bao gồm một tóm tắt về các thay đổi và vấn đề nào mà chúng giải quyết. Bạn có thể bao gồm hình ảnh, liên kết và bảng để giúp truyền đạt thông tin này. Nếu pull request của bạn liên quan đến một vấn đề, hãy liên kết vấn đề để những người liên quan đến vấn đề biết về pull request và ngược lại. Nếu bạn liên kết với một từ khóa, vấn đề sẽ tự động đóng khi pull request được hợp nhất.
    
        \item Ngoài việc điền thông tin vào thân yêu cầu pull, bạn có thể thêm bình luận vào các dòng cụ thể của pull request để chỉ ra rõ điều gì đó đối với người xem xét. 
    
        \item Repository của bạn có thể được cấu hình tự động yêu cầu xem xét từ các nhóm hoặc người dùng cụ thể khi một pull request được tạo. Bạn cũng có thể thêm bằng cách \textbf{@đề cập} hoặc yêu cầu xem xét từ các người hoặc nhóm cụ thể. 
    
        \item Nếu repository của bạn đã cấu hình để chạy kiểm tra trạng thái trên các pull request, bạn sẽ thấy bất kỳ kiểm tra nào thất bại trên pull request của bạn. Điều này giúp bạn phát hiện lỗi trước khi hợp nhất nhánh của mình. 
    \end{itemize}

    \item \textbf{Giải quyết review comments}
    \begin{itemize}
        \item Người xem xét nên để lại câu hỏi, ý kiến và gợi ý. Người xem xét có thể bình luận về toàn bộ pull request hoặc thêm bình luận vào các dòng hoặc tệp tin cụ thể. Bạn và người xem xét có thể chèn hình ảnh hoặc đề xuất mã nguồn để làm rõ ý kiến.
        \item Bạn có thể tiếp tục commit và đẩy các thay đổi phản hồi. Pull request của bạn sẽ được cập nhật tự động.
    \end{itemize}

    \item \textbf{Merge pull request}
    \begin{itemize}
        \item Khi pull request của bạn được chấp thuận, bạn có thể merge pull request vào nhánh main. Điều này sẽ tự động hợp nhất nhánh của bạn để các thay đổi của bạn xuất hiện trên nhánh mặc định. GitHub giữ lại lịch sử của bình luận và commit trong pull request để giúp đội ngũ đóng góp viên trong tương lai hiểu rõ các thay đổi của bạn.
        \item GitHub sẽ thông báo nếu pull request của bạn có xung đột cần giải quyết trước khi hợp nhất. 
        \item Cài đặt bảo vệ nhánh có thể ngăn chặn quá trình hợp nhất nếu pull request của bạn không đáp ứng một số yêu cầu nhất định. Ví dụ, bạn cần một số lượng xác nhận đánh giá hoặc một đánh giá chấp thuận từ một nhóm cụ thể.
    \end{itemize}

    \item \textbf{Xóa nhánh đã được merge}
    \begin{itemize}
        \item Sau khi hợp nhất pull request của bạn, hãy xóa nhánh của bạn. Điều này chỉ ra rằng công việc trên nhánh đã hoàn tất và ngăn chặn bạn hoặc người khác sử dụng nhánh cũ một cách tình cờ.
        \item Đừng lo lắng về việc mất thông tin. Pull request và lịch sử commit của bạn sẽ không bị xóa. Bạn luôn có thể khôi phục lại nhánh bị xóa hoặc hoàn nguyên pull request nếu cần thiết.
    \end{itemize}    
\end{enumerate}

\subsection{Thiết kế giao diện người dùng}
Như đã biết, phần front-end web của ứng dụng demo của nhóm được phát triển bằng ReactJS, sử dụng TypeScript như ngôn ngữ lập trình. Cấu trúc giao diện web là một khía cạnh quan trọng trong quá trình phát triển phần front-end, ảnh hưởng trực tiếp tới việc phát triển, bảo trì và mở rộng ứng dụng.\\

Hiện nay trên thế giới có nhiều phương pháp để xây dựng cấu trúc giao diện web như: feature-based, component-based, module-based, Atomic Design...Mỗi phương pháp đều có ưu nhược điểm riêng phù hợp với từng loại dự án.\\

Phương pháp mà nhóm sẽ áp dụng để viết cấu trúc giao diện của ứng dụng demo là phương pháp feature-based. Theo đó, giao diện sẽ được chia thành các component dựa trên từng tính năng/chức năng của ứng dụng. Mỗi component sẽ quản lý render và xử lý logic riêng cho từng tính năng.
\subsubsection{Tại sao sử dụng cấu trúc Feature-Based}
Khi phát triển ứng dụng React, việc sử dụng cách tổ chức folder theo tính năng (feature-based) mang lại nhiều lợi ích. Dưới đây là một số lý do tại sao nên áp dụng cách tổ chức này:

\begin{itemize}
    \item Dễ quản lý và mở rộng: Tách biệt các tính năng hoặc thành phần (component) thành các folder riêng biệt giúp quản lý dễ dàng hơn khi ứng dụng phát triển lớn dần. Mỗi tính năng được xem như một module độc lập, có thể quản lý và phát triển riêng biệt mà không ảnh hưởng đến các tính năng khác. Điều này giúp đảm bảo tính tổ chức và cấu trúc của dự án, giúp nhóm phát triển dễ dàng tìm kiếm và chỉnh sửa mã nguồn.
    \item Dễ mở rộng và sửa đổi tính năng: Với cách tổ chức feature-based, việc thêm, xóa hoặc cập nhật tính năng trở nên dễ dàng hơn. Bằng cách tách riêng từng tính năng thành các folder, nhóm phát triển có thể làm việc trên một tính năng cụ thể mà không cần quan tâm đến các tính năng khác. Điều này giúp giảm thiểu xung đột và rủi ro gây lỗi khi thay đổi mã nguồn.
    \item Dễ tái sử dụng code: Khi mỗi tính năng được tách ra thành một package độc lập, code trong từng tính năng có thể dễ dàng tái sử dụng giữa các dự án khác nhau. Điều này tạo điều kiện thuận lợi cho việc chia sẻ và sử dụng lại code, giúp tiết kiệm thời gian và công sức phát triển.
    \item Giảm sự phức tạp khi tổ chức theo loại component: Thay vì tổ chức theo loại component như components, containers, reducers, việc sắp xếp theo tính năng giúp giảm sự phức tạp của dự án. Code được tổ chức theo tính năng, giúp tăng tính nhất quán và dễ đọc, hiểu và bảo trì hơn. Nhóm phát triển có thể tập trung vào từng tính năng cụ thể mà không phải lo lắng về cấu trúc tổ chức.
    \item Kiểm soát và giới hạn sự phụ thuộc: Cách tổ chức feature-based giúp áp dụng quy tắc về tính khả dụng của các component. Điều này giúp kiểm soát và giới hạn sự phụ thuộc lẫn nhau giữa các tính năng, tránh ảnh hưởng không mong muốn khi thay đổi. Đồng thời, giúp cải thiện khả năng kiểm thử và tái sử dụng code.
    \item Hỗ trợ phát triển ứng dụng lớn và phức tạp: Cách tổ chức feature-based thích hợp cho việc phát triển ứng dụng lớn, có nhiều tính năng phức tạp. Nó giúp tăng tính tổ chức, quản lý và tiếp cận dự án, giảm thiểu sự mất rối và mâu thuẫn khi làm việc với nhiều thành viên cùng một lúc.
\end{itemize}
Có một số lý do tại sao cách tổ chức feature-based có ưu điểm hơn cách tổ chức theo loại thành phần như cách tổ chức theo loại thành phần như actions, components, containers khi dự án phát triển lớn:

\begin{itemize}
    \item Về tính khả dụng: Khi tổ chức theo loại thành phần, các thành phần có thể sử dụng lẫn nhau một cách tùy tiện. Điều này dễ dẫn đến sự phụ thuộc chéo không cần thiết giữa các module. Còn cách feature-based giới hạn tính khả dụng của thành phần trong phạm vi tính năng/module riêng, tránh được vấn đề này.
   
    \item Về tái sử dụng: Khi tổ chức theo tính năng, mỗi tính năng trở thành một package/module độc lập có thể dễ dàng tái sử dụng trong các dự án khác. Còn cách theo loại thành phần thì khó tái sử dụng riêng lẻ từng thành phần.
   
    \item Về bảo trì: Khi dự án lớn, việc quản lý theo từng tính năng thay vì loại thành phần sẽ đơn giản và rõ ràng hơn. Dễ dàng tìm kiếm, thêm bớt các tính năng một cách độc lập.
    
    \item Về mở rộng: Cách feature-based cho phép mở rộng từng tính năng độc lập mà không ảnh hưởng đến các tính năng khác. Trong khi đó, cách theo loại thành phần dễ dẫn đến tình trạng phức tạp khi mở rộng.
\end{itemize}

Do đó, cách tổ chức feature-based thích hợp hơn khi quy mô dự án lớn về mặt tính khả dụng, tái sử dụng, bảo trì và khả năng mở rộng.\\

Do vậy nhóm chọn cách tổ chức theo feature-based cho trang web demo của nhóm.
\subsubsection{Cấu trúc phần UI}
Cách tổ chức này chia ứng dụng thành các thư mục chính, mỗi thư mục đại diện cho một tính năng cụ thể trong ứng dụng. Đây là một cách tiếp cận phổ biến trong việc tổ chức dự án, giúp tăng tính tổ chức và dễ quản lý.

\begin{itemize}
    \item \textbf{Thư mục "Components":}\\
    Thư mục này chứa các thành phần (components) của ứng dụng. Các thành phần có thể được định nghĩa dưới dạng các file độc lập hoặc nhóm lại thành các thư mục con. Điều này cho phép bạn xây dựng các thành phần con bên trong một thành phần cha, giúp quản lý và sử dụng lại mã nguồn một cách dễ dàng.
    \item \textbf{Thư mục "Scenes":}\\
    Thư mục này chứa các trang hoặc màn hình trong ứng dụng. Các scenes có thể bao gồm các thành phần, scenes và services khác. Điều này cho phép bạn tổ chức các thành phần trong một ngữ cảnh cụ thể và quản lý logic của từng màn hình một cách rõ ràng.
    \item \textbf{Thư mục "Services":}\\
    Thư mục này chứa các module phục vụ logic ứng dụng. Các module này có thể được sử dụng để xử lý các tác vụ như xử lý dữ liệu, gọi API hoặc thao tác với cơ sở dữ liệu. Tổ chức các module theo thư mục này giúp tách biệt và quản lý tốt hơn các chức năng trong ứng dụng.

\end{itemize}
    Mỗi tính năng hoặc chức năng của ứng dụng sẽ được đặt trong một thư mục riêng biệt, bao gồm tất cả những gì cần thiết để tính năng đó hoạt động. Điều này giúp tăng tính tổ chức và khả năng tìm kiếm mã nguồn liên quan đến từng tính năng cụ thể.\\

Nhóm dựa vào những tìm hiểu ở trên đây để tiến hành xây dựng cấu trúc của ứng dụng. Cấu trúc thư mục của dự án có thể được tổ chức như sau:

\begin{itemize}
    \item \textbf{apis:} Thư mục chứa các tệp tin liên quan đến việc gọi API.
    \item \textbf{assets:} Thư mục chứa các tài nguyên như hình ảnh, file đính kèm và các tài liệu khác sử dụng trong ứng dụng.
    \item \textbf{components:} Thư mục chứa các thành phần React, được sắp xếp theo cấu trúc phân cấp cho phép định nghĩa các component con bên trong component cha.
    \item \textbf{constants:} Thư mục chứa các hằng số và cấu hình cho dự án.
    \item \textbf{hooks:} Thư mục chứa các custom hooks, đây là các hàm tái sử dụng giúp quản lý trạng thái và logic trong ứng dụng.
    \item \textbf{layouts:} Thư mục chứa các layout hoặc template để sắp xếp giao diện của ứng dụng.
    \item \textbf{libs:} Thư mục chứa các thư viện và công cụ bên thứ ba được sử dụng trong dự án.
    \item \textbf{pages:} Thư mục chứa các trang hoặc màn hình trong ứng dụng, tương đương với scenes.
    \item \textbf{routes:} Thư mục chứa các tệp tin liên quan đến định tuyến trong ứng dụng.
    \item \textbf{styles:} Thư mục chứa các tệp tin CSS hoặc SCSS để định nghĩa kiểu dáng cho các component.
    \item \textbf{types:} Thư mục chứa các tệp tin liên quan đến kiểu dữ liệu và khai báo cho TypeScript.
\end{itemize}
Với cấu trúc thư mục này, dự án của bạn được tổ chức một cách rõ ràng và dễ quản lý. Mỗi thành phần có vị trí và chức năng riêng, giúp tăng tính tái sử dụng và dễ bảo trì trong quá trình phát triển.
\subsubsection{Cách các thành phần tương tác với nhau}
Trong cấu trúc feature-based, các thành phần chính sẽ tương tác với nhau một cách rõ ràng và có trình tự nhất định. Dưới đây là mô tả chi tiết về cách các thành phần này tương tác:

\begin{itemize}
    \item Components (Thành phần): Components được sử dụng trong Scenes và Pages để xây dựng giao diện người dùng. Chúng có thể bao gồm các thành phần con bên trong, tạo ra một cấu trúc phân cấp. Components có khả năng nhận thông tin từ các props và trả về các thành phần UI tương ứng.
    \item Scenes và Pages: Scenes và Pages là các trang hoặc màn hình trong ứng dụng, chứa các Components. Chúng là nơi tổ chức và sắp xếp các Components để hiển thị giao diện người dùng. Scenes và Pages có thể sử dụng Services để truy xuất dữ liệu và xử lý logic nghiệp vụ.
    \item Services (Dịch vụ): Services là các modules hoặc lớp được sử dụng để thực hiện các tác vụ như gọi API, xử lý logic phức tạp, hoặc truy xuất dữ liệu từ nguồn ngoài. Components, Scenes, và Pages có thể sử dụng Services để gửi yêu cầu và nhận kết quả từ các hoạt động này.
    \item Actions (Hành động): Actions là các sự kiện được phát ra từ Components, Scenes và Services khi có sự kiện xảy ra, ví dụ như người dùng click vào một nút hoặc hoàn thành một form. Actions mô tả hành động cần thực hiện và có thể chứa dữ liệu liên quan.
    \item Reducers (Bộ giảm nhẹ): Reducers là các hàm xử lý Actions để cập nhật trạng thái ứng dụng lưu trữ trong store. Khi một Action được gửi đi, Reducers sẽ xác định cách cập nhật trạng thái hiện tại và trả về một trạng thái mới.
    \item Store (Kho lưu trữ): Store là nơi lưu trữ trạng thái toàn cục của ứng dụng. Nó chứa các Reducers và cung cấp các phương thức để truy cập và cập nhật trạng thái. Store có thể được truy cập bởi tất cả các thành phần trong ứng dụng.
    \item Hooks (Các hàm tái sử dụng): Components, Scenes, và Services có thể sử dụng các Hooks custom để quản lý logic phức tạp hơn. Hooks cho phép tái sử dụng mã logic và trạng thái trong các Components và Scenes mà không cần sử dụng các lớp (class) React.
    \item Assets (Tài nguyên): Assets bao gồm các tệp tin như hình ảnh, các tài liệu đính kèm, các tệp tin CSS hoặc SCSS, v.v. Các thành phần như Components, Scenes và Services có thể sử dụng Assets để xây dựng giao diện và logic trong ứng dụng.
\end{itemize}

Tóm lại, cấu trúc feature-based cho phép các thành phần trong ứng dụng tương tác một cách rõ ràng và có trình tự nhất định. Thông qua việc sử dụng các phương thức như props, Actions, Reducers, và Hooks, dữ liệu và logic được truyền và xử lý một cách mạch lạc và dễ dàng quản lý.

\subsubsection{Kết quả thiết kế giao diện Website}
\subsubsubsection{Màn hình Home}

% \subsection{Triển khai Web API cho cơ sở dữ liệu}

\subsubsection{Triển khai Backend}
Nhóm đã áp dụng mô hình MVC (Model-View-Controller) trong việc xây dựng phần Backend cho ứng dụng của mình. Trong mô hình MVC, mỗi đối tượng sẽ có ba thành phần chính là Controller, Model và View (Router) để đảm bảo tính phân tách và dễ bảo trì của ứng dụng.
\begin{itemize}
    \item Controller: Đây là thành phần chịu trách nhiệm xử lý yêu cầu từ phía client và điều khiển việc xử lý của ứng dụng. Controller sẽ nhận các yêu cầu từ phía client, xử lý chúng và gửi kết quả trả về cho client. Controller cũng có thể gọi các phương thức của Model để thực hiện các thao tác truy vấn cơ sở dữ liệu.
    \item Model: Đây là thành phần đại diện cho dữ liệu và xử lý các thao tác truy vấn cơ sở dữ liệu. 
    \item View (Router): Đây là thành phần đại diện cho giao diện người dùng. View sẽ nhận các yêu cầu từ phía client và điều hướng chúng đến các Controller tương ứng. View cũng có thể chứa các template để hiển thị dữ liệu trả về từ Controller.
\end{itemize}

% \input{Sections/hien_thuc/deploy}

% \newpage
% \section{TỔNG KẾT}

\subsection{Những gì đã thực hiện và đạt được}

Trong giai đoạn đầu tiên của đồ án – giai đoạn phân tích và thiết kế hệ thống – nhóm đã hoàn thiện đầy đủ các nội dung quan trọng, bao gồm:

\begin{itemize}
    \item \textbf{Phân tích yêu cầu hệ thống}: Xác định và đặc tả rõ các yêu cầu chức năng, phi chức năng; xây dựng các sơ đồ use-case, hành trình khách hàng và module chức năng cốt lõi.
    \item \textbf{Nghiên cứu hệ thống liên quan}: Phân tích nhiều hệ thống nhà hàng nổi bật để tham khảo thiết kế chức năng phù hợp cho hệ thống Menu+.
    \item \textbf{Thiết kế tổng thể hệ thống}: Bao gồm thiết kế kiến trúc (MVC), sơ đồ EERD, lược đồ quan hệ logic và vật lý cho cơ sở dữ liệu PostgreSQL.
    \item \textbf{Thiết kế giao diện và API}: Phác thảo sitemap, wireframe, screenflow và thiết kế OpenAPI cho các chức năng chính.
    \item \textbf{Xác định công nghệ}: Lựa chọn stack công nghệ sử dụng trong giai đoạn triển khai gồm ReactJS, Spring Boot, Redis, Docker, GitHub Actions,...
\end{itemize}

Các kết quả đạt được trong giai đoạn này là nền tảng quan trọng giúp đảm bảo sự mạch lạc và hiệu quả cho các bước phát triển tiếp theo của hệ thống.

\subsection{Định hướng và triển khai trong giai đoạn tiếp theo}

Giai đoạn tiếp theo sẽ tập trung vào hiện thực hóa hệ thống thông qua lập trình, kiểm thử và triển khai. Cụ thể, nhóm sẽ:

\begin{itemize}
    \item \textbf{Phát triển chức năng hệ thống}: Xây dựng API và giao diện người dùng cho các module: quản lý thực đơn, POS (tại chỗ, mang về, giao hàng), đặt bàn & món trước, quản lý lịch làm việc, tích hợp bếp (KDS), báo cáo vận hành và gọi xác nhận bằng bot.
    
    \item \textbf{Kiểm thử và tích hợp hệ thống}: Kiểm thử chức năng từng phần và toàn bộ quy trình nghiệp vụ (end-to-end); phân tích hiệu năng API, tốc độ phản hồi và tối ưu giao diện.

    \item \textbf{Triển khai và CI/CD}: Thiết lập quy trình triển khai bằng Docker và GitHub Actions để tự động hóa kiểm thử, build và deploy. Hệ thống sẽ được triển khai trên môi trường staging để đánh giá trước khi lên production. Toàn bộ tài liệu hướng dẫn triển khai, sử dụng và đào tạo cũng sẽ được chuẩn bị.
\end{itemize}

Giai đoạn này là bước kiểm chứng tính khả thi và hiệu quả của thiết kế, đồng thời đưa hệ thống tiến gần đến việc sẵn sàng vận hành thực tế.

% \newpage
% \section{KẾ HOẠCH PHÁT TRIỂN}

Để đảm bảo tiến độ và chất lượng của phần mềm quản lý đặt món nhà hàng, kế hoạch phát triển được xây dựng chi tiết, phân chia thành các giai đoạn rõ ràng cho từng chức năng chính. Kế hoạch này được thể hiện qua các biểu đồ Gantt, mô tả thời gian thực hiện và các mốc quan trọng trong quá trình phát triển. Chi tiết kế hoạch có thể được xem tại: \url{https://byvn.net/fze2}.

\begin{figure}[H]
    \centering
    \includegraphics[width=\linewidth]{Images/gantt_1}
    \vspace{0.5cm}
    \caption{Biểu đồ Gantt cho các chức năng Quản lý thực đơn - sản phẩm và Hệ thống POS Ăn tại chỗ}
    \label{fig:gantt_menu_pos}
\end{figure}

\begin{figure}[H]
    \centering
    \includegraphics[width=\linewidth]{Images/gantt_2}
    \vspace{0.5cm}
    \caption{Biểu đồ Gantt cho các chức năng Hệ thống POS Mang về và Giao hàng}
    \label{fig:gantt_takeaway_delivery}
\end{figure}

\begin{figure}[H]
    \centering
    \includegraphics[width=\linewidth]{Images/gantt_3}
    \vspace{0.5cm}
    \caption{Biểu đồ Gantt cho các chức năng Tích hợp bếp và Đặt bàn/món trước}
    \label{fig:gantt_kitchen_booking}
\end{figure}

\begin{figure}[H]
    \centering
    \includegraphics[width=\linewidth]{Images/gantt_4}
    \vspace{0.5cm}
    \caption{Biểu đồ Gantt cho các chức năng Quản lý lịch, Báo cáo, Xác nhận bot và Kiểm thử}
    \label{fig:gantt_schedule_report_test}
\end{figure}
% \newpage



% \section{TRIỂN KHAI HỆ THỐNG}
\subsection{Triền khai Front-end}
Như đã được đề cập, nhóm triển khai hệ thống lên nền tảng Vercel nhằm tối ưu chi phí và tận dụng những lợi ích liên quan đến việc thống kê trang web. Việc triển khai lên Vercel có thể được thực hiện qua các bước sau:
\begin{itemize}
    \item Đăng ký và tạo tài khoản Vercel: Đầu tiên, nhóm sẽ đăng ký một tài khoản trên Vercel. Quá trình này thường đơn giản và chỉ đòi hỏi thông tin cơ bản như địa chỉ email và mật khẩu hoặc có thể sử dụng Github làm tài khoản.
    \item Tạo dự án trên Vercel: Sau khi có tài khoản, nhóm sẽ tạo một dự án mới trên Vercel. Điều này thường liên quan đến việc cung cấp tên dự án và liên kết đến kho lưu trữ mã nguồn, chẳng hạn như GitHub hoặc GitLab.
    \item Cấu hình dự án: Trong quá trình tạo dự án, nhóm sẽ phải cấu hình các thông số cho dự án trên Vercel. Điều này có thể bao gồm chỉ định các biến môi trường, xác định cách triển khai và xử lý các bước xây dựng. \\
    
    Các cấu hình của nhóm cho phần front-end của dự án như sau: 
    \begin{itemize}
        \item $Build Command$: next build
        \item $Output Directory$: Next.js default
        \item $Install Command$: yarn
        \item $Development Command$: next
        \item $Node.js Version$: 18.18
        \item $NEXT\_PUBLIC\_GOOGLE\_SITE\_VERIFICATION$: hbwzV-DeoMMsgwNxZaw6s9L74x53\_w8KjhK68izrsnE
        \item $NEXT\_PUBLIC\_GOOGLE\_MEASUREMENT\_ID$: G-MDBKYG0VYG
        \item $CLIENT\_URL$: https://share-cv.vercel.app
        \item $SERVER\_URL$: https://share-cv-ubv1.onrender.com
    \end{itemize}
    \item Triển khai ứng dụng: Khi các thiết lập cấu hình hoàn tất, nhóm có thể tiến hành triển khai ứng dụng lên Vercel. Quá trình này thường đơn giản và chỉ đòi hỏi một số thao tác cơ bản để khởi động quá trình triển khai.
    \item Kiểm tra và giám sát: Sau khi triển khai, nhóm nên kiểm tra kỹ lưỡng để đảm bảo rằng trang web hoạt động đúng như mong đợi. Ngoài ra, Vercel cũng cung cấp các công cụ giám sát để theo dõi hiệu suất và hoạt động của ứng dụng trên nền tảng.
\end{itemize}

Thêm hình 



\subsection{Vercel Web Analytics cho Front-end}
Sử dụng gói npm @vercel/analytics nhằm cài đặt Vercel Web Analytics cho Front-end:\\

\textbf{Lợi ích:}

\begin{itemize}
    \item Dễ dàng cài đặt: Đây là cách đơn giản nhất để tích hợp Web Analytics vào ứng dụng Vercel của bạn. Chỉ cần cài đặt một gói npm và thêm một thành phần vào mã của bạn.
    \item Dữ liệu phân tích toàn diện: Gói @vercel/analytics cung cấp nhiều thông tin chi tiết về lưu lượng truy cập trang web của bạn, bao gồm lượt xem trang, nguồn lưu lượng truy cập, thông tin người dùng (quốc gia, hệ điều hành, trình duyệt,...)
    \item Miễn phí: Web Analytics cho Vercel hoàn toàn miễn phí cho tất cả các dự án Vercel.
\end{itemize}

\textbf{Cách thực hiện:}

Cài đặt gói npm @vercel/analytics:
Thêm Analytics component vào ứng dụng. Mở tệp JavaScript chính của ứng dụng Vercel (index.js). Nhập gói @vercel/analytics và thêm Analytics component vào mã
\begin{lstlisting}
import { Analytics } from '@vercel/analytics';

function App() {
  return (
    <div>
      <Analytics />
      {/* App content */}
    </div>
  );
}
\end{lstlisting}

Sau khi thêm Analytics component vào mã, lưu tệp và triển khai ứng dụng đến Vercel và test thử chức năng của ứng dụng.

Thêm hình

\subsection{Triển khai Back-end}

\subsection{Triển khai Google Search Console}
Google Search Console (trước đây được gọi là Google Webmaster Tools) là một công cụ miễn phí của Google dành cho các chủ sở hữu trang web và nhà phát triển. Nó cung cấp các công cụ và tài nguyên để giúp họ hiểu và quản lý hiệu suất của trang web trên công cụ tìm kiếm của Google.\\

Google Search Console cho phép người dùng theo dõi và báo cáo về khái quát về cách trang web của họ được tìm thấy và hiển thị trong kết quả tìm kiếm của Google. Bằng cách đăng ký và xác minh trang web của mình, họ có thể thu thập thông tin quan trọng về lượng truy cập, từ khóa tìm kiếm, liên kết đến trang web và nhiều thông tin khác.\\

Với Google Search Console, người dùng có thể:
\begin{itemize}
    \item Xác minh và gửi bản đồ trang web của họ cho Google để đảm bảo rằng các trang web của họ được chỉ định và hiển thị đúng cách trong kết quả tìm kiếm.
    \item Kiểm tra xem Google có gặp vấn đề nào khi truy cập trang web của họ và cung cấp thông tin về các lỗi crawl hoặc lỗi chỉ mục.
    \item Theo dõi các chỉ số hiệu suất trang web, bao gồm lượng truy cập, tỷ lệ nhấp chuột, thứ hạng từ khóa và nhiều thông tin liên quan khác.
    \item Tìm hiểu về các từ khóa mà trang web của họ đang xếp hạng và hiển thị trong kết quả tìm kiếm.
    \item Nhận thông báo từ Google về các vấn đề quan trọng liên quan đến trang web của họ, như lỗi tìm kiếm, vi phạm chính sách và nhiều hơn nữa.
\end{itemize}
Google Search Console là một công cụ hữu ích để tối ưu hóa trang web và cải thiện hiệu suất tìm kiếm trên Google. Nó cung cấp thông tin chi tiết về cách Google xem và xếp hạng trang web, giúp người dùng điều chỉnh và tăng cường chiến lược SEO của mình.\\

Cài đặt bằng cách thêm thẻ này vào root layout của trang web:
\begin{lstlisting}
    <meta
        name="google-site-verification"
        content={CONFIG.googleSearchConsole.config.siteVerification}
    />
\end{lstlisting}

Thêm hình

\subsection{Triển khai Dashboard}
Appsmith là một nền tảng phát triển ứng dụng mã nguồn mở, cho phép bạn nhanh chóng xây dựng giao diện người dùng và kết nối với các nguồn dữ liệu khác nhau như cơ sở dữ liệu MongoDB. Dưới đây là sơ lược về quá trình triển khai ứng dụng lên Appsmith và cách quản lý thông tin trong cơ sở dữ liệu MongoDB cho các đối tượng như CV, bài viết (post) và người dùng (user) trong hệ thống.

\textbf{Kết nối Appsmith với MongoDB:}
\begin{itemize}
    \item Mở Appsmith trên trình duyệt web.
    \item Tạo kết nối đến MongoDB: Trong giao diện Appsmith, tạo một kết nối mới đến MongoDB bằng cách cung cấp thông tin đăng nhập và cấu hình của cơ sở dữ liệu MongoDB.
\end{itemize}
\textbf{Xây dựng giao diện ứng dụng:}
\begin{itemize}
    \item Tạo trang cho CV: Sử dụng trình tạo giao diện của Appsmith để tạo một trang hiển thị danh sách CV. Lấy dữ liệu từ bảng "cv" trong cơ sở dữ liệu và hiển thị nó trên trang.
    \item Tạo trang cho người dùng: Tương tự, tạo một trang hiển thị danh sách người dùng, lấy dữ liệu từ bảng "user" và hiển thị nó trên trang.
\end{itemize}
\textbf{Thao tác với dữ liệu:}
\begin{itemize}
    \item Thêm, sửa, xóa CV: Cho phép người dùng thêm, sửa đổi hoặc xóa CV thông qua giao diện ứng dụng. Khi người dùng thực hiện các thao tác này, gửi yêu cầu tương ứng đến Appsmith, và Appsmith sẽ thực hiện các thao tác tương ứng với cơ sở dữ liệu MongoDB.
    \item Thêm, sửa, xóa người dùng: Tương tự, cho phép người dùng thêm, sửa đổi hoặc xóa người dùng thông qua giao diện ứng dụng, và Appsmith sẽ thực hiện các thao tác tương ứng với cơ sở dữ liệu MongoDB.
\end{itemize}

Thêm hình XXXXXXX
% \newpage
% \section{KIỂM THỬ}
\subsection{Kiểm thử API}
Kiểm thử API là quá trình kiểm tra yêu cầu và phản hồi trong giao tiếp giữa client và server trong ứng dụng API. Quá trình này tập trung vào kiểm tra tính chính xác và tuân thủ các quy tắc và giao thức của API, với sự tập trung vào lớp business logic của phần mềm mà không liên quan đến giao diện người dùng. \\

Nhóm đã chọn công cụ sử dụng là Postman để thực hiện kiểm thử API. Postman cung cấp một môi trường để gửi yêu cầu API và kiểm tra phản hồi từ hệ thống. Bằng cách sử dụng Postman, nhóm có thể thực hiện kiểm thử API một cách hiệu quả và đáng tin cậy cho hệ thống. \\

Để đảm bảo tính đầy đủ và chính xác của API, nhóm tập trung chú ý vào việc:

\begin{itemize}
    \item Kiểm tra tính đúng đắn của dữ liệu đầu vào: Đảm bảo rằng API xử lý đúng các dạng dữ liệu đầu vào và kiểm tra xử lý hợp lệ của các trường dữ liệu. Bao gồm kiểm tra các trường hợp dữ liệu hợp lệ, không hợp lệ và ranh giới.

    \item Kiểm tra trạng thái và phản hồi của API: Xác minh rằng API trả về trạng thái và mã phản hồi chính xác cho các yêu cầu. Đảm bảo rằng các mã phản hồi như 200 OK, 400 Bad Request, 401 Unauthorized, 500 Internal Server Error được xử lý đúng và phù hợp.

    \item Kiểm tra các chức năng và hoạt động của API: Đảm bảo rằng các chức năng và hoạt động của API hoạt động chính xác và đáp ứng yêu cầu. 
\end{itemize}

Quá trình thực hiện kiểm thử API:

\begin{itemize}
    \item Sử dụng Postman tạo collection để tổ chức các API đã triển khai. Collection sẽ chứa các request tương ứng với từng API.
    \item Mô tả các API đã triển khai: Trong mỗi request trong collection, nhóm xác định thông tin về API đã triển khai, bao gồm URL của endpoint, phương thức HTTP (GET, POST, PUT, DELETE), các tham số yêu cầu, tiêu đề và dữ liệu yêu cầu.
    \item  Thực hiện việc thiết lập Environment trong Postman
    để dễ dàng sử dụng các biến môi trường cho các request. (URL, thông số, ...)
    \item Kiểm thử API: Thực hiện các request trong Postman bằng cách gửi yêu cầu tới endpoint của API đã triển khai. Postman hiển thị phản hồi từ server, bao gồm mã phản hồi, dữ liệu trả về và bất kỳ lỗi nào có thể xảy ra.
    \item Để minh họa cách sử dụng API một cách rõ ràng, nhóm có Example cho mỗi request. Example hiển thị một ví dụ cụ thể về dữ liệu yêu cầu và phản hồi mà nhóm đã thực hiện.

\end{itemize}

Kết quả kiểm thử: 


Đường dẫn truy cập API Document: 

\subsection{Kiểm thử chức năng}
Kiểm thử chức năng là một trong các quy trình đảm bảo chất lượng của lĩnh vực kiểm thử phần mềm.

\subsubsection{Kiểm thử hộp đen}
Kiểm thử hộp đen (Black box testing) là phương pháp kiểm thử phần mềm mà việc kiểm tra các chức năng của một ứng dụng không cần quan tâm vào cấu trúc nội bộ. Mục đích chính của kiểm thử hộp đen chỉ là để xem phần mềm có hoạt động như dự kiến và liệu nó có đáp ứng được sự mong đợi của người dùng hay không.\\

Kỹ thuật kiểm thử hộp đen thường sử dụng:

\begin{itemize}
    \item Kỹ thuật Phân vùng tương đương (Equivalence Class Partitioning Technique)

    Phân vùng tương đương là kỹ thuật chia đầu vào thành những nhóm tương đương nhau. Nếu một giá trị trong nhóm hoạt động đúng thì tất cả các giá trị trong nhóm đó cũng hoạt động đúng và ngược lại. 
    \item Kỹ thuật Phân tích giá trị biên (Boundary Value Analysis Technique)

    Phân tích giá trị biên là phương pháp kiểm thử các giá trị ở vùng biên của dữ liệu đầu vào. Thay vì phải kiểm thử toàn bộ dữ liệu vào và ra, ta có thể kiểm thử một số trường hợp mà vẫn đảm bảo hệ thống hoạt động tốt.
    \item Kỹ thuật Bảng quyết định (Decision Table Technique)

    Kỹ thuật kiểm thử giúp đánh giá dữ liệu đầu ra khi kết hợp các dữ liệu đầu vào với nhau.
    \item Kỹ thuật Kiểm thử trường hợp sử dụng (Use-case Testing Technique)

    Kỹ thuật kiểm thử dựa vào use-case. Use case mô tả sự tương tác giữa phần mềm và tác nhân khác như người dùng, hệ thống khác,… 
\end{itemize}

Kế hoạch kiểm thử: Nhóm chỉ tiến hành kiểm thử các tính năng quan trọng của hệ thống như sau:

\begin{itemize}
    \
\end{itemize}

Đối với trường dữ liệu đơn như tính năng comment, nhóm lựa chọn kiểm thử với kỹ thuật phân tích giá trị biên. Đối với kiểm thử hành vi của hệ thống với nhiều trường dữ liệu như tính năng đăng CV, nhận CV, đăng bài viết, nhóm lựa chọn kỹ thuật kiểm thử bảng quyết định hoặc kiểm thử dựa vào use-case.\\

Kết quả kiểm thử: 


Ngoài ra, nhóm có kiểm thử các non-requirements và thành công như kỳ vọng:

\begin{itemize}
    \item 
    \item 

\end{itemize}
\subsubsection{Kiểm thử tự động}

Kiểm thử tự động (Automation testing) là một kỹ thuật mà người kiểm thử viết các kịch bản một cách độc lập và sử dụng phần mềm phù hợp hoặc Công cụ Tự động hóa để kiểm thử phần mềm. Đó là một Quy trình Tự động hóa của một Quy trình Thủ công. Nó cho phép thực hiện các nhiệm vụ lặp đi lặp lại mà không cần sự can thiệp của một Người kiểm thử Thủ công.

\begin{itemize}
    \item Nó được sử dụng để tự động hóa các nhiệm vụ kiểm thử khó khăn để thực hiện thủ công.
    \item Các bài kiểm tra tự động có thể được thực hiện bất kỳ lúc nào trong ngày vì chúng sử dụng các chuỗi kịch bản đã được viết trước để kiểm tra phần mềm.
    \item Các bài kiểm tra tự động cũng có thể nhập dữ liệu kiểm tra, so sánh kết quả mong đợi với kết quả thực tế và tạo ra các báo cáo kiểm thử chi tiết.
    \item Mục tiêu của các bài kiểm tra tự động là giảm số lượng các trường hợp kiểm tra cần được thực hiện thủ công nhưng không phải loại bỏ kiểm thử thủ công.
    \item Có thể ghi lại bộ kiểm tra và phát lại khi cần thiết.
\end{itemize}

Để hiện thực kiểm thử tự động cho hệ thống, chúng tôi sử dụng Selenium IDE để kiểm tra trên toàn hệ thống.

Selenium IDE (Integrated Development Environment) là một công cụ tự động hóa kiểm thử dựa trên trình duyệt web được sử dụng để ghi và chạy các tác vụ kiểm thử trên ứng dụng web. Selenium IDE được phát triển dựa trên Selenium WebDriver, là một phần của dự án Selenium. Nó cung cấp một giao diện đồ họa dễ sử dụng cho người dùng để tạo và quản lý các kịch bản kiểm thử mà không cần viết mã lập trình.\\

Selenium IDE thích hợp cho người mới bắt đầu với kiểm thử tự động trên giao diện web, nhưng nó cũng có một số hạn chế và không phù hợp cho các kịch bản kiểm thử phức tạp. Đối với các dự án lớn hơn và phức tạp hơn, người dùng thường sử dụng Selenium WebDriver và các thư viện kiểm thử tự động khác để có kiểm soát và tùy chỉnh chi tiết hơn.

\subsubsubsection{Cách sử dụng Selenium IDE}
\textbf{Welcome Screen}
Khi mở IDE, bạn sẽ thấy một hộp thoại chào mừng. Điều này sẽ cung cấp cho bạn quyền truy cập nhanh đến các tùy chọn sau:
\begin{itemize}
    \item Ghi lại một bài kiểm tra mới trong một dự án mới
    \item Mở một dự án hiện tại
    \item Tạo một dự án mới
    \item Đóng IDE
\end{itemize}

\textbf{Ghi lại tests}
Sau khi tạo một dự án mới, bạn sẽ được yêu cầu đặt tên cho nó và sau đó được hỏi để cung cấp một địa chỉ URL cơ sở. Địa chỉ URL cơ sở là địa chỉ URL của ứng dụng bạn đang kiểm tra. Điều này là điều bạn thiết lập một lần và nó sẽ được sử dụng cho tất cả các bài kiểm tra trong dự án này. Bạn có thể thay đổi nó sau này nếu cần. \\

Sau khi hoàn thành các thiết lập này, một cửa sổ trình duyệt mới sẽ mở ra, tải địa chỉ URL cơ sở và bắt đầu ghi lại. \\

Tương tác với trang và mọi hành động của bạn sẽ được ghi lại trong IDE. Để dừng việc ghi lại, chuyển sang cửa sổ IDE và nhấp vào biểu tượng ghi lại. 


\textbf{Tổ chức tests}
\begin{itemize}
    \item \textbf{Tests}
    
    Bạn có thể thêm một bài kiểm tra mới bằng cách nhấp vào biểu tượng + ở đầu của thanh menu bên trái (bên phải của tiêu đề Bài kiểm tra), đặt tên cho nó và nhấp vào THÊM.
    
    Sau khi thêm, bạn có thể nhập các lệnh một cách thủ công hoặc nhấp vào biểu tượng ghi lại ở phía trên bên phải của IDE.

    \item \textbf{Suites}
    
    Các bài kiểm tra có thể được nhóm lại thành các bộ kiểm tra.
    Khi tạo dự án, một Bộ kiểm tra Mặc định được tạo và bài kiểm tra đầu tiên của bạn được thêm vào tự động.
    
    Để tạo và quản lý các bộ kiểm tra, hãy chuyển đến bảng điều khiển Bộ kiểm tra. Bạn có thể đến đó bằng cách nhấp vào danh sách thả xuống ở đầu thanh menu bên trái (ví dụ: nhấp vào từng chữ Tests) và chọn Bộ kiểm tra.

    \begin{itemize}
        \item \textbf{Add a suite}
        Để thêm một bộ kiểm tra, nhấp vào biểu tượng + ở đầu của thanh menu bên trái, bên phải của tiêu đề "Test Suites", cung cấp tên và nhấp vào "ADD".

        \item \textbf{Add a test}
        Để thêm một bài kiểm tra vào một bộ kiểm tra, di chuột qua tên bộ kiểm tra, sau đó thực hiện các bước sau:
        \begin{enumerate}
            \item Nhấp vào biểu tượng xuất hiện bên phải của tiêu đề "Test Suites"
            \item Nhấp vào "Add tests"
            \item Chọn các bài kiểm tra bạn muốn thêm từ menu
            \item Nhấp vào "Select"
        \end{enumerate}

        \item \textbf{Remove a test}
        Để xóa một bài kiểm tra, di chuột qua bài kiểm tra và nhấp vào biểu tượng "X" xuất hiện bên phải tên.

        
        \item \textbf{Remove or rename a suite}
        
        Để xóa một bộ kiểm tra, nhấp vào biểu tượng xuất hiện bên phải tên nó, nhấp vào "Delete", và nhấp vào "Delete" khi được nhắc.
        
        Để đổi tên một bộ kiểm tra, di chuột qua tên bộ kiểm tra, nhấp vào biểu tượng xuất hiện bên phải tên, nhấp vào "Rename", cập nhật tên và nhấp vào "RENAME".
        
    \end{itemize}
    
\end{itemize}

\textbf{Lưu lại project}
Để lưu tất cả những gì bạn vừa làm trong IDE, nhấp vào biểu tượng lưu ở góc phải trên cùng của IDE.

Nó sẽ yêu cầu bạn chọn vị trí và tên để lưu dự án. Kết quả cuối cùng là một tệp duy nhất có phần mở rộng .side.


\textbf{Chạy lại test}
\begin{itemize}
    \item \textbf{Trong trình duyệt}
    
    Bạn có thể chạy lại các bài kiểm tra trong IDE bằng cách chọn bài kiểm tra hoặc bộ kiểm tra bạn muốn chạy và nhấp vào nút chạy ở thanh menu phía trên trình soạn thảo bài kiểm tra.
    
    Các bài kiểm tra sẽ chạy lại trong trình duyệt. Nếu cửa sổ vẫn mở từ khi ghi lại, nó sẽ được sử dụng để chạy lại. Nếu không, một cửa sổ mới sẽ được mở và sử dụng.

    \item \textbf{Cross-browser}
    
    Nếu bạn muốn chạy các bài kiểm tra IDE của mình trên các trình duyệt khác nhau, hãy chắc chắn kiểm tra Runner dòng lệnh.
\end{itemize}

\subsubsubsection{Kết quả kiểm thử}





 xXXXXXXXXXX
% \newpage
% \section{ĐÁNH GIÁ}
\subsection{Đánh giá hệ thống }

Đánh giá hệ thống giúp đảm bảo rằng ứng dụng hoạt động mượt mà và hiệu quả, không gặp vấn đề về hiệu suất hay tốc độ phản hồi. Điều này làm tăng trải nghiệm người dùng và duy trì sự hài lòng. Bên cạnh đó, việc này sẽ giúp đảm bảo rằng các chức năng và tính năng của ứng dụng đáp ứng đúng nhu cầu của người dùng. Việc này giữ cho ứng dụng hữu ích và thú vị cho người sử dụng.	

\subsubsection{Về mặt kỹ thuật}

\begin{itemize}
    \item \textbf{Ngôn ngữ lập trình và Framework}
    
 
    \item \textbf{Cơ sở dữ liệu} 
    
    
    \item \textbf{Giao diện người dùng (UI) và React}

    
    \item \textbf{Khả năng mở rộng và hiệu suất}
    
    
    \item \textbf{Bảo mật}

    
    \item \textbf{Hiệu suất API}
    
    
    \item \textbf{Tính dễ mở rộng}

\end{itemize}

\subsection{Đánh giá hiệu suất }
\subsubsection{Độ chính xác}
Độ chính xác là một yếu tố quan trọng trong nhiều hệ thống và ngữ cảnh khác nhau. Tùy thuộc vào loại hệ thống, độ chính xác có thể đóng vai trò quyết định đến hiệu suất và độ tin cậy của hệ thống đó. \\

Mức độ chính xác cao giúp đảm bảo rằng thông tin được xử lý đúng và đáng tin cậy.  Các quyết định dựa trên dữ liệu chính xác thường dẫn đến hiệu suất và hiệu quả cao hơn.\\

Tuy nhiên, cũng cần lưu ý rằng đôi khi có sự đánh đổi giữa độ chính xác và các yếu tố khác như tốc độ xử lý, chi phí, và nguồn lực. Trong một số trường hợp, việc chấp nhận một mức độ chính xác thấp hơn có thể làm tăng tính khả dụng và giảm chi phí. Do đó, quyết định về độ chính xác thường cần phải cân nhắc kỹ lưỡng dựa trên yêu cầu cụ thể của ứng dụng và ngữ cảnh sử dụng. \\

...
\subsubsection{Độ trễ}
Độ trễ trong hệ thống là một yếu tố quan trọng nhất là khi xây dựng ứng dụng, đặc biệt là những ứng dụng tương tác nhanh và yêu cầu phản hồi ngay lập tức từ người dùng. Độ trễ được định nghĩa là thời gian chờ đợi giữa hành động của người dùng và phản hồi của ứng dụng đối với hành động đó. Điều này bao gồm cả quá trình xử lý dữ liệu và thời gian tải của ứng dụng. \\

Ở mức độ cao, độ trễ có thể tạo ra trải nghiệm người dùng kém chất lượng, gây khó chịu và giảm hiệu suất sử dụng ứng dụng. Đặc biệt, trong các ứng dụng web và di động, độ trễ có thể ảnh hưởng đến thời gian tải của trang và ảnh hưởng đến sự liền mạch của trải nghiệm người dùng.\\

Do đó, đây là 1 trong những tiêu chí quan trọng nhất cần xem xét khi xây dựng một hệ thống. \\

...
 xXXXXX



\newpage

\phantomsection
\renewcommand{\refname}{\textbf{TÀI LIỆU THAM KHẢO}}
\begin{thebibliography}{}
	\bibitem[1]{USDA} \textit{Food Service - Hotel Restaurant Institutional Annual
	} (Lần truy cập cuối: 22/04/2025), [Online]. Available: \url{https://apps.fas.usda.gov/newgainapi/api/Report/DownloadReportByFileName?fileName=Food\%20Service%20-\%20Hotel\%20Restaurant%20Institutional\%20Annual_Ho\%20Chi\%20Minh\%20City_Vietnam_VM2024-0038.pdf}
	\bibitem[2]{Orderable} \textit{How to Improve Your Restaurant Customer Journey
	} (Lần truy cập cuối: 13/02/2025), [Online]. Available: \url{https://orderable.com/blog/restaurant-customer-journey/}
	\bibitem[3]{paper1} \textit{An Innovative Approach for Online Food Order Management System} (Lần truy cập cuối: 22/04/2025), [Online]. Available: \url{https://globaljournals.org/GJCST_Volume18/1-An-Innovative-Approach-for-Online.pdf}
	\bibitem[4]{paper2} \textit{Examining Technology Adoption and Management Perception of Inventory Management Systems: The Case of Aruba Restaurants
	} (Lần truy cập cuối: 22/04/2025), [Online]. Available: \url{https://digitalcommons.fiu.edu/cgi/viewcontent.cgi?article=1482&context=hospitalityreview/}
	\bibitem[5]{paper3} \textit{Restaurant Inventory Management System
	} (Lần truy cập cuối: 22/04/2025), [Online]. Available: \url{https://iciset.in/Paper1217.pdf}
	\bibitem[6]{MVC} \textit{Model-View-Controller (MVC) Architecture} (Lần truy cập cuối: 11/12/2023), [Online]. Available: \url{https://www.academia.edu/30077059/Model_View_Controller_MVC_Architecture}
	\bibitem[7]{FrontendFrameworks} \textit{Front-end frameworks popularity (React, Vue, Angular and Svelte)} (Lần truy cập cuối: 13/02/2025), [Online]. Available: \url{https://gist.github.com/tkrotoff/b1caa4c3a185629299ec234d2314e190}

	\bibitem[8]{React} \textit{React – A JavaScript library for building user interfaces} (Lần truy cập cuối: 13/02/2025), [Online]. Available: \url{https://legacy.reactjs.org}
	% https://star-history.com/#shadcn-ui/ui&Date

	\bibitem[9]{ShadcnUIUX} \textit{GitHub Star History - shadcn-ui/ux} (Lần truy cập cuối: 13/02/2025), [Online]. Available: \url{https://star-history.com/#shadcn-ui/ui}

	\bibitem[10]{ShadcnUI} \textit{shadcn ui Usage Statistics} (Lần truy cập cuối: 13/02/2025), [Online]. Available: \url{https://trends.builtwith.com/framework/shadcn-ui}

	\bibitem[11]{TanStack} \textit{TanStack | High-quality open-source software for web developers} (Lần truy cập cuối: 13/02/2025), [Online]. Available: \url{https://tanstack.com}

	\bibitem[12]{SpringBootBenefits} \textit{Top Benefits of Spring Boot in Java that Every Developer Should Know} (Lần truy cập cuối: 13/02/2025), [Online]. Available: \url{https://www.syncfusion.com/web-stories/top-benefits-of-spring-boot-in-java-that-every-developer-should-know}

	\bibitem[13]{SpringBootProsCons} \textit{Pros and Cons of Using Spring Boot} (Lần truy cập cuối: 13/02/2025), [Online]. Available: \url{https://bambooagile.eu/insights/pros-and-cons-of-using-spring-boot}

	\bibitem[14]{Redis} \textit{What is Redis?} (Lần truy cập cuối: 13/02/2025), [Online]. Available: \url{https://www.ibm.com/think/topics/redis}


	\bibitem[15]{Viblo} \textit{CI/CD - GitHub Actions và các kiến thức cơ bản} (Lần truy cập cuối: 13/02/2025), [Online]. Available: \url{https://viblo.asia/p/cicd-github-actions-va-cac-kien-thuc-co-ban-EoW4oRMrVm/}


	\bibitem[16]{GitHubActionsIntro} \textit{What is GitHub Actions? How CI/CD \& automation work on GitHub} (Lần truy cập cuối: 13/02/2025), [Online]. Available: \url{https://resources.github.com/devops/tools/automation/actions}

	\bibitem[17]{GitHubActionsDocs} \textit{Understanding GitHub Actions} (Lần truy cập cuối: 13/02/2025), [Online]. Available: \url{https://docs.github.com/en/actions/about-github-actions/understanding-github-actions}

	\bibitem[18]{GitHubActionsMerits} \textit{Surveying the Merits and Drawbacks of GitHub Actions} (Lần truy cập cuối: 13/02/2025), [Online]. Available: \url{https://blog.mirrorfolio.com/surveying-the-merits-and-drawbacks-of-github-actions}

	\bibitem[19]{DockerWiki} \textit{Docker (software)} (Lần truy cập cuối: 13/02/2025), [Online]. Available: \url{https://en.wikipedia.org/wiki/Docker\_(software)}

	\bibitem[20]{DockerAdvantages} \textit{Advantages of Docker} (Lần truy cập cuối: 13/02/2025), [Online]. Available: \url{https://medium.com/@alrazak/advantages-of-docker-5d153d8e0c79}

	\bibitem[21]{FrontCompare1} \textit{React, Angular, Vue, and Svelte: A Comparison} (Lần truy cập cuối: 30/04/2025), [Online]. Available: \url{https://www.ascendientlearning.com/blog/comparing-angular-react-vue-svelte}

	\bibitem[22]{FrontCompare2} \textit{Comparing Angular, React, Vue, and Svelte: What you need to know} (Lần truy cập cuối: 30/04/2025), [Online]. Available: \url{https://www.infoworld.com/article/3962039/what-you-need-to-know-about-angular-react-vue-and-svelte-popular-javascript-frameworks-compared.html}

	\bibitem[23]{BackCompare1} \textit{ExpressJS vs Flask vs Spring Boot} (Lần truy cập cuối: 30/04/2025), [Online]. Available: \url{https://stackshare.io/stackups/expressjs-vs-flask-vs-spring-boot}

	\bibitem[24]{BackCompare2} \textit{Node.js vs Spring Boot vs Django: Which One Is Best for Beginners?} (Lần truy cập cuối: 30/04/2025), [Online]. Available: \url{https://sandydev.medium.com/node-js-vs-spring-boot-vs-django-which-one-is-best-for-beginners-8782d3be54}

	\bibitem[25]{BackCompare3} \textit{Spring Boot vs Django: Which Backend Framework Should You Master in 2025?} (Lần truy cập cuối: 30/04/2025), [Online]. Available: \url{https://medium.com/codex/spring-boot-vs-django-which-backend-framework-should-you-master-in-2025-4dafa4d6b2a6}
\end{thebibliography}
% \newpage
% \appendix
% \renewcommand{\thesection}{\arabic{section}} % Định dạng section là 1, 2, 3,...
% \titleformat{\section}{\normalfont\Large\bfseries}{\thesection}{1em}{} % Bỏ chữ "Chương"
% \titlecontents{section}
%   [0pt]{\vspace{1ex}}{\bfseries \thecontentslabel \quad}{}
%   {\hfill\bfseries\contentspage}

% \section*{\centering \Large Phụ lục}
% \addcontentsline{toc}{section}{\textbf{PHỤ LỤC}}
% \section{Đặc tả Usecase}

\subsection{Module MD-01: Quản lý Lịch làm việc (Scheduling)}

\subsubsection{Use Case UC-MD01-01: Tạo ca làm việc mới}


\begin{longtable}{|m{4cm}|p{11cm}|}
\caption{Đặc tả Use Case UC-MD01-01: Tạo ca làm việc mới} \label{tab:uc_md01_01} \\
\hline
\endhead % Header cho các trang tiếp theo

\hline
\endfoot % Footer cho bảng

\hline
\endlastfoot % Footer cho trang cuối cùng

\multicolumn{2}{|c|}{\textbf{2.1. Tóm tắt (Summary)}} \\
\hline
\textbf{Mục} & \textbf{Nội dung} \\
\hline

Use Case Name & Tạo ca làm việc mới \\
\hline
Use Case ID & UC-MD01-01 \\
\hline
Use Case Description & Cho phép Quản lý nhà hàng định nghĩa và lưu trữ thông tin chi tiết về một ca làm việc mới trong hệ thống lập lịch. \\
\hline
Actor & US-01 (Quản lý nhà hàng) \\
\hline
Priority & Must Have \\
\hline
Trigger & Quản lý nhà hàng cần tạo một khung thời gian làm việc cụ thể với các yêu cầu về vai trò và số lượng nhân sự cho một ngày trong tương lai. \\
\hline
Pre-Condition & - Người dùng US-01 đã đăng nhập vào hệ thống với quyền quản lý lịch trình. \newline - Danh sách các vai trò công việc (FR-MD01-08) đã được định nghĩa trong hệ thống. \\
\hline
Post-Condition & - Một bản ghi ca làm việc mới được tạo và lưu trong hệ thống với trạng thái "Nháp". \newline - Ca làm việc mới này hiển thị trên giao diện lịch trình (ví dụ: Gantt chart) dưới dạng chưa được gán nhân viên và chưa xuất bản. \newline - Hệ thống ghi nhận hoạt động tạo ca vào nhật ký hệ thống (activity log). \\
\hline
\multicolumn{2}{|c|}{\textbf{2.2. Luồng thực thi (Flow)}} \\
\hline
\textbf{Mục} & \textbf{Nội dung} \\
\hline
Basic Flow & 1. Quản lý nhà hàng (US-01) truy cập chức năng quản lý lịch làm việc. \newline 2. US-01 chọn hành động để tạo ca làm việc mới (ví dụ: nhấn nút "New" hoặc click vào một ô trống trên lịch). \newline 3. Hệ thống hiển thị form/dialog để nhập thông tin ca làm việc. \newline 4. US-01 nhập/chọn thông tin chi tiết cho ca làm việc: \newline    - Ngày diễn ra ca làm việc. \newline    - Giờ bắt đầu. \newline    - Giờ kết thúc. \newline    - Chọn (các) vai trò cần thiết cho ca làm việc từ danh sách vai trò đã có. \newline    - Nhập số lượng nhân viên cần cho mỗi vai trò đã chọn. \newline    - (Tùy chọn) Nhập ghi chú cho ca làm việc. \newline 5. US-01 chọn lệnh "Lưu" hoặc "Tạo". \newline 6. Hệ thống kiểm tra tính hợp lệ của dữ liệu nhập vào (theo BR-UC1.1-1, BR-UC1.1-2). \newline 7. Hệ thống lưu thông tin ca làm việc vào cơ sở dữ liệu với trạng thái "Nháp". \newline 8. Hệ thống cập nhật giao diện lịch trình để hiển thị ca làm việc mới (dạng nháp, chưa gán nhân viên). \newline 9. Hệ thống hiển thị thông báo tạo ca thành công. \newline 10. Hệ thống ghi nhận hoạt động vào Activity Log. \\
\hline
Alternative Flow & Không có luồng thay thế đáng kể cho chức năng cơ bản này. \\
\hline
Exception Flow & \textbf{6a. Dữ liệu không hợp lệ:} \newline    1. Hệ thống phát hiện dữ liệu nhập không hợp lệ (ví dụ: giờ kết thúc trước giờ bắt đầu, số lượng nhân viên là số âm hoặc không phải số). \newline    2. Hệ thống hiển thị thông báo lỗi cụ thể, chỉ rõ trường và lý do không hợp lệ. \newline    3. Hệ thống giữ nguyên các dữ liệu đã nhập và cho phép US-01 chỉnh sửa. Use Case quay lại bước 4 của Basic Flow. \newline \textbf{7a. Lỗi hệ thống khi lưu:} \newline    1. Hệ thống gặp lỗi trong quá trình lưu dữ liệu (ví dụ: lỗi kết nối cơ sở dữ liệu). \newline    2. Hệ thống hiển thị thông báo lỗi chung về việc không thể lưu ca làm việc. \newline    3. Use Case kết thúc trong trạng thái lỗi. \\
\hline
\multicolumn{2}{|c|}{\textbf{2.3. Thông tin bổ sung (Additional Information)}} \\
\hline
\textbf{Mục} & \textbf{Nội dung} \\
\hline
Business Rule & - \textbf{BR-UC1.1-1:} Giờ kết thúc của ca làm việc phải lớn hơn giờ bắt đầu. \newline - \textbf{BR-UC1.1-2:} Số lượng nhân viên cần cho mỗi vai trò phải là một số nguyên dương (>0). \newline - \textbf{BR-UC1.1-3:} Ca làm việc mới tạo mặc định có trạng thái là "Nháp" (Draft). \newline - \textbf{BR-UC1.1-4:} Chỉ có thể chọn các vai trò đã được định nghĩa trong hệ thống (liên kết FR-MD01-08). \\
\hline
Non-Functional Requirement & - \textbf{NFR-UC1.1-1 (Usability):} Giao diện nhập thông tin ca làm việc phải trực quan, dễ sử dụng, có gợi ý hoặc calendar picker cho ngày, dropdown/search cho vai trò. \newline - \textbf{NFR-UC1.1-2 (Performance):} Thời gian hệ thống kiểm tra và lưu ca làm việc mới phải dưới 2 giây trong điều kiện tải bình thường. \newline - \textbf{NFR-UC1.1-3 (Data Integrity):} Dữ liệu ca làm việc phải được lưu trữ chính xác theo thông tin người dùng đã nhập. \\
\hline

\end{longtable}

\subsubsection{Use Case UC-MD01-02: Gán nhân viên vào ca làm việc}
\begin{longtable}{|m{4cm}|p{11cm}|}
\caption{Đặc tả Use Case UC-MD01-02: Gán nhân viên vào ca làm việc} \label{tab:uc_md01_02} \\

\hline
\endhead % Header cho các trang tiếp theo

\hline
\endfoot % Footer cho bảng

\hline
\endlastfoot % Footer cho trang cuối cùng
\hline
\multicolumn{2}{|c|}{\textbf{2.1. Tóm tắt (Summary)}} \\
\hline
\textbf{Mục} & \textbf{Nội dung} \\
Use Case Name & Gán nhân viên vào ca làm việc \\
\hline
Use Case ID & UC-MD01-02 \\
\hline
Use Case Description & Cho phép Quản lý nhà hàng chỉ định một nhân viên cụ thể vào một vị trí/vai trò còn trống trong một ca làm việc đang ở trạng thái Nháp. \\
\hline
Actor & US-01 (Quản lý nhà hàng) \\
\hline
Priority & Must Have \\
\hline
Trigger & Quản lý nhà hàng cần phân công nhân sự cụ thể cho các ca làm việc đã được tạo khung. \\
\hline
Pre-Condition & - Người dùng US-01 đã đăng nhập vào hệ thống với quyền quản lý lịch trình. \newline - Tồn tại ít nhất một ca làm việc trong hệ thống với trạng thái "Nháp" (được tạo bởi FR-MD01-01). \newline - Ca làm việc được chọn có ít nhất một vị trí/vai trò chưa được gán nhân viên. \newline - Dữ liệu nhân viên (với vai trò được định nghĩa) đã tồn tại trong hệ thống. \\
\hline
Post-Condition & - Nhân viên được chọn được gán thành công vào vị trí/vai trò cụ thể trong bản ghi ca làm việc. \newline - Giao diện lịch trình (ví dụ: Gantt chart) được cập nhật để hiển thị tên nhân viên đã được gán cho vị trí đó. \newline - Số lượng vị trí đã được lấp đầy cho vai trò đó trong ca làm việc được cập nhật (nếu có theo dõi). \newline - Hệ thống ghi nhận hoạt động gán ca vào nhật ký hệ thống (activity log). \\
\hline
\multicolumn{2}{|c|}{\textbf{2.2. Luồng thực thi (Flow)}} \\
\hline
\textbf{Mục} & \textbf{Nội dung} \\
\hline
Basic Flow & 1. Quản lý nhà hàng (US-01) truy cập giao diện quản lý lịch làm việc (ví dụ: chế độ xem Gantt). \newline 2. US-01 chọn một ca làm việc cụ thể đang ở trạng thái "Nháp" và có vị trí cần gán nhân viên. \newline 3. US-01 chọn (ví dụ: click vào) vị trí/vai trò còn trống trong ca làm việc đó. \newline 4. Hệ thống hiển thị danh sách các nhân viên thỏa mãn các điều kiện sau: \newline    - Có vai trò phù hợp với yêu cầu của vị trí. \newline    - Khả dụng trong khung thời gian của ca làm việc (không bị đánh dấu không sẵn sàng - FR-MD01-09, và không bị trùng lịch với ca khác đã được gán - kiểm tra theo FR-MD01-04). \newline 5. US-01 chọn một nhân viên từ danh sách hiển thị. \newline 6. US-01 xác nhận việc gán (ví dụ: nhấn nút "Gán" hoặc kéo thả nhân viên vào vị trí trên Gantt). \newline 7. Hệ thống lưu thông tin gán nhân viên vào vị trí đã chọn của ca làm việc. \newline 8. Hệ thống cập nhật giao diện Gantt chart, hiển thị tên nhân viên vừa được gán vào vị trí đó. \newline 9. Hệ thống (tùy chọn) cập nhật số lượng vị trí còn trống cho vai trò/ca đó. \newline 10. Hệ thống ghi nhận hoạt động vào Activity Log. \\
\hline
Alternative Flow & \textbf{4a. Lọc/Tìm kiếm nhân viên:} \newline    1. Nếu danh sách nhân viên dài, US-01 sử dụng chức năng tìm kiếm (theo tên) hoặc bộ lọc (theo kỹ năng, nếu có) để thu hẹp danh sách. \newline    2. Use Case tiếp tục từ bước 5 của Basic Flow. \newline \textbf{4b. Xem chi tiết/lịch trình nhân viên:} \newline    1. Trước khi chọn, US-01 nhấp vào tên nhân viên trong danh sách để xem thông tin chi tiết hơn (ví dụ: tổng số giờ đã được xếp lịch trong tuần). \newline    2. US-01 đóng cửa sổ/panel chi tiết. \newline    3. Use Case tiếp tục từ bước 5 của Basic Flow. \\
\hline
Exception Flow & \textbf{4c. Không có nhân viên phù hợp:} \newline    1. Hệ thống xác định không có nhân viên nào thỏa mãn tất cả các điều kiện (vai trò, thời gian khả dụng). \newline    2. Hệ thống hiển thị thông báo: "Không tìm thấy nhân viên phù hợp và khả dụng cho vị trí này". \newline    3. Use Case kết thúc đối với nỗ lực gán vị trí này. \newline \textbf{7a. Xung đột lịch (kiểm tra lại):} \newline    1. Mặc dù đã có kiểm tra ở bước 4, hệ thống có thể phát hiện xung đột ngay tại thời điểm lưu (ví dụ: do thao tác đồng thời). \newline    2. Hệ thống từ chối việc gán. \newline    3. Hệ thống hiển thị thông báo lỗi về việc trùng lịch. \newline    4. Use Case có thể quay lại bước 4 hoặc 5 của Basic Flow. \newline \textbf{7b. Lỗi hệ thống khi gán:} \newline    1. Hệ thống gặp sự cố kỹ thuật khi cố gắng lưu thông tin gán. \newline    2. Hệ thống hiển thị thông báo lỗi chung (ví dụ: "Đã xảy ra lỗi. Không thể gán nhân viên."). \newline    3. Use Case kết thúc trong trạng thái lỗi. \\
\hline
\multicolumn{2}{|c|}{\textbf{2.3. Thông tin bổ sung (Additional Information)}} \\
\hline
\textbf{Mục} & \textbf{Nội dung} \\
\hline
Business Rule & - \textbf{BR-UC1.2-1:} Chỉ những nhân viên có vai trò được định nghĩa khớp với vai trò yêu cầu của vị trí trong ca mới được hiển thị để lựa chọn. \newline - \textbf{BR-UC1.2-2:} Nhân viên sẽ không xuất hiện trong danh sách lựa chọn nếu họ đã đánh dấu không sẵn sàng (FR-MD01-09) trong khoảng thời gian của ca làm việc. \newline - \textbf{BR-UC1.2-3:} Nhân viên sẽ không xuất hiện trong danh sách lựa chọn nếu họ đã được gán vào một ca làm việc khác có thời gian trùng lặp (liên quan đến FR-MD01-04). \newline - \textbf{BR-UC1.2-4:} Việc gán nhân viên trực tiếp như mô tả trong Use Case này chỉ áp dụng cho các ca làm việc có trạng thái "Nháp". Các ca đã "Xuất bản" có thể cần quy trình khác để thay đổi (ví dụ: hủy xuất bản trước). \\
\hline
Non-Functional Requirement & - \textbf{NFR-UC1.2-1 (Performance):} Danh sách nhân viên khả dụng phải được tải và hiển thị trong vòng dưới 3 giây (với giả định < 100 nhân viên). \newline - \textbf{NFR-UC1.2-2 (Usability):} Giao diện cần phân biệt rõ ràng nhân viên khả dụng và không khả dụng (nếu hiển thị cả hai). Nên hỗ trợ thao tác kéo thả nhân viên vào vị trí trên biểu đồ Gantt. \newline - \textbf{NFR-UC1.2-3 (Security):} Chỉ người dùng có vai trò US-01 (hoặc quyền tương đương được cấp) mới có thể thực hiện chức năng gán nhân viên vào ca. \\
\hline

\end{longtable}


\subsubsection{Use Case UC-MD01-03: Xem lịch biểu Gantt}


\begin{longtable}{|m{4cm}|p{11cm}|}
\caption{Đặc tả Use Case UC-MD01-03: Xem lịch biểu Gantt} \label{tab:uc_md01_03} \\
\hline

\endhead % Header cho các trang tiếp theo

\hline
\endfoot % Footer cho bảng

\hline
\endlastfoot % Footer cho trang cuối cùng
\multicolumn{2}{|c|}{\textbf{2.1. Tóm tắt (Summary)}} \\
\hline
\textbf{Mục} & \textbf{Nội dung} \\
\hline
Use Case Name & Xem lịch biểu Gantt \\
\hline
Use Case ID & UC-MD01-03 \\
\hline
Use Case Description & Cho phép Quản lý nhà hàng xem tổng quan lịch làm việc của nhân viên (bao gồm ca nháp và ca đã xuất bản) dưới dạng biểu đồ Gantt trực quan, với khả năng thay đổi khung thời gian hiển thị (ngày, tuần, tháng) và lọc theo vai trò công việc. \\
\hline
Actor & US-01 (Quản lý nhà hàng) \\
\hline
Priority & Must Have \\
\hline
Trigger & Quản lý nhà hàng muốn xem xét, đánh giá, hoặc kiểm tra lịch làm việc đã được phân công hoặc đang trong quá trình lập kế hoạch. \\
\hline
Pre-Condition & - Người dùng US-01 đã đăng nhập vào hệ thống với quyền truy cập module Lịch làm việc. \newline - Có dữ liệu lịch làm việc (các ca đã được tạo, có thể đã hoặc chưa gán nhân viên) tồn tại trong hệ thống cho khoảng thời gian được xem xét. \\
\hline
Post-Condition & - Biểu đồ Gantt hiển thị trực quan lịch làm việc theo các tiêu chí (thời gian, vai trò) do người dùng chọn hoặc theo mặc định. \newline - Quản lý nhà hàng có cái nhìn tổng quan về việc phân bổ nhân sự theo thời gian. \\
\hline
\multicolumn{2}{|c|}{\textbf{2.2. Luồng thực thi (Flow)}} \\
\hline
\textbf{Mục} & \textbf{Nội dung} \\
\hline
Basic Flow & 1. Quản lý nhà hàng (US-01) truy cập vào module/chức năng quản lý Lịch làm việc. \newline 2. Hệ thống tự động tải và hiển thị biểu đồ Gantt với chế độ xem mặc định (ví dụ: Tuần hiện tại, hiển thị theo Nhân viên - mỗi nhân viên một hàng). \newline 3. US-01 xem xét lịch trình hiển thị trên biểu đồ. \\
\hline
Alternative Flow & \textbf{3a. Thay đổi Thang Thời Gian:} \newline    1. US-01 chọn một thang thời gian khác (ví dụ: Ngày, Tháng) từ các nút điều khiển có sẵn. \newline    2. Hệ thống tải lại và hiển thị biểu đồ Gantt tương ứng với thang thời gian mới được chọn. \newline    3. Use Case quay lại bước 3 của Basic Flow. \newline \textbf{3b. Lọc theo Vai Trò:} \newline    1. US-01 chọn một hoặc nhiều Vai trò công việc từ danh sách lọc (ví dụ: chỉ xem lịch của 'Bếp trưởng', 'Phục vụ'). \newline    2. Hệ thống lọc và hiển thị lại biểu đồ Gantt chỉ bao gồm các ca làm việc liên quan đến vai trò đã chọn. \newline    3. Use Case quay lại bước 3 của Basic Flow. \newline \textbf{3c. Chuyển đổi Khung Thời Gian (Tuần/Tháng Tiếp theo/Trước):} \newline    1. US-01 sử dụng các nút điều hướng (ví dụ: mũi tên "<", ">") để chuyển sang xem tuần/tháng trước đó hoặc kế tiếp. \newline    2. Hệ thống tải lại và hiển thị biểu đồ Gantt cho khung thời gian mới. \newline    3. Use Case quay lại bước 3 của Basic Flow. \newline \textbf{3d. Xem chi tiết ca làm việc (Hover/Click):} \newline    1. US-01 di chuột qua một thanh biểu diễn ca làm việc trên biểu đồ. \newline    2. Hệ thống hiển thị một tooltip/popup nhỏ chứa thông tin tóm tắt về ca đó (Nhân viên, Vai trò, Giờ bắt đầu/kết thúc, Trạng thái). \newline    3. (Tùy chọn) US-01 nhấp vào thanh ca làm việc để mở chi tiết đầy đủ hoặc thực hiện hành động khác (như Sửa, Gán nhân viên - liên quan đến UC khác). \newline    4. Use Case quay lại bước 3 của Basic Flow. \\
\hline
Exception Flow & \textbf{2a. Không có dữ liệu lịch trình:} \newline    1. Hệ thống không tìm thấy bất kỳ ca làm việc nào trong khoảng thời gian/bộ lọc hiện tại. \newline    2. Hệ thống hiển thị một biểu đồ trống hoặc một thông báo rõ ràng (ví dụ: "Không có lịch trình nào được tìm thấy cho lựa chọn này."). \newline    3. Use Case kết thúc hoặc chờ người dùng thay đổi bộ lọc/thời gian. \newline \textbf{2b. Lỗi tải dữ liệu:} \newline    1. Hệ thống gặp sự cố khi truy vấn hoặc xử lý dữ liệu lịch trình. \newline    2. Hệ thống hiển thị một thông báo lỗi chung (ví dụ: "Không thể tải dữ liệu lịch trình. Vui lòng thử lại sau."). \newline    3. Use Case kết thúc trong trạng thái lỗi. \\
\hline
\multicolumn{2}{|c|}{\textbf{2.3. Thông tin bổ sung (Additional Information)}} \\
\hline
\textbf{Mục} & \textbf{Nội dung} \\
\hline
Business Rule & - \textbf{BR-UC1.3-1:} Chế độ xem mặc định của biểu đồ Gantt là theo tuần hiện tại, nhóm theo nhân viên (mỗi hàng một nhân viên). \newline - \textbf{BR-UC1.3-2:} Biểu đồ Gantt phải hiển thị cả các ca làm việc ở trạng thái "Nháp" và "Đã xuất bản". Cần có sự phân biệt trực quan rõ ràng giữa hai trạng thái này (ví dụ: ca nháp có sọc chéo, ca xuất bản có màu liền mạch). \newline - \textbf{BR-UC1.3-3:} Các thanh (bar) trên biểu đồ biểu diễn ca làm việc, chiều dài của thanh tương ứng với thời lượng của ca. \newline - \textbf{BR-UC1.3-4:} Màu sắc của các thanh ca làm việc có thể được sử dụng để biểu thị Vai trò hoặc trạng thái khác (nếu được cấu hình). \newline - \textbf{BR-UC1.3-5:} Khi di chuột qua một thanh ca làm việc, hệ thống phải hiển thị thông tin chi tiết cơ bản (tooltip). \newline - \textbf{BR-UC1.3-6:} Các tùy chọn lọc theo vai trò phải dựa trên danh sách vai trò đã được định nghĩa trong hệ thống (FR-MD01-08). \\
\hline
Non-Functional Requirement & - \textbf{NFR-UC1.3-1 (Performance):} Thời gian tải và hiển thị biểu đồ Gantt cho một tuần làm việc tiêu chuẩn (ví dụ: < 50 nhân viên, < 200 ca) phải dưới 5 giây. Việc lọc hoặc thay đổi thang thời gian phải cập nhật dưới 3 giây. \newline - \textbf{NFR-UC1.3-2 (Usability):} Biểu đồ phải dễ đọc, dễ điều hướng (cuộn ngang/dọc). Các nút điều khiển thang thời gian, bộ lọc, điều hướng tuần/tháng phải rõ ràng và dễ tiếp cận. Phân biệt trạng thái Nháp/Xuất bản phải rõ ràng. \newline - \textbf{NFR-UC1.3-3 (Accuracy):} Dữ liệu hiển thị trên Gantt chart phải chính xác và đồng bộ với dữ liệu ca làm việc được lưu trong cơ sở dữ liệu tại thời điểm tải. \newline - \textbf{NFR-UC1.3-4 (Scalability):} Hệ thống nên có khả năng xử lý và hiển thị hiệu quả khi số lượng nhân viên và ca làm việc tăng lên (ví dụ: > 100 nhân viên, > 500 ca/tuần), mặc dù hiệu suất có thể giảm nhẹ. \\
\hline

\end{longtable}


\subsubsection{Use Case UC-MD01-04: Phát hiện và Cảnh báo Trùng lịch}

\begin{longtable}{|m{4cm}|p{11cm}|}
\caption{Đặc tả Use Case UC-MD01-04: Phát hiện và Cảnh báo Trùng lịch} \label{tab:uc_md01_04} \\
\hline

\endhead % Header cho các trang tiếp theo

\hline
\endfoot % Footer cho bảng

\hline
\endlastfoot % Footer cho trang cuối cùng
\multicolumn{2}{|c|}{\textbf{2.1. Tóm tắt (Summary)}} \\
\hline
\textbf{Mục} & \textbf{Nội dung} \\
\hline
Use Case Name & Phát hiện và Cảnh báo Trùng lịch \\
\hline
Use Case ID & UC-MD01-04 \\
\hline
Use Case Description & Hệ thống tự động xác định và cung cấp cảnh báo trực quan cho Quản lý nhà hàng khi một nhân viên được phân công vào các ca làm việc có thời gian bị trùng lặp nhau. \\
\hline
Actor & System (Thực hiện chính), US-01 (Quản lý nhà hàng - Kích hoạt thông qua hành động) \\
\hline
Priority & Must Have \\
\hline
Trigger & - US-01 thực hiện hành động gán một nhân viên vào một ca làm việc (trong UC-MD01-02). \newline - US-01 lưu lại thay đổi trên lịch trình. \newline - Hệ thống tải/hiển thị giao diện lịch làm việc (ví dụ: Gantt chart - UC-MD01-03). \\
\hline
Pre-Condition & - Tồn tại ít nhất một bản ghi nhân viên. \newline - Tồn tại ít nhất hai bản ghi ca làm việc được gán cho cùng một nhân viên (hoặc một ca đang được gán). \\
\hline
Post-Condition & - Nếu phát hiện trùng lịch, các ca làm việc bị trùng của nhân viên đó được đánh dấu bằng một chỉ báo trực quan rõ ràng (ví dụ: màu đỏ) trên giao diện Gantt chart. \newline - (Tùy chọn, theo cấu hình) Việc gán ca gây ra xung đột có thể bị chặn hoặc được phép lưu nhưng vẫn giữ cảnh báo. \\
\hline
\multicolumn{2}{|c|}{\textbf{2.2. Luồng thực thi (Flow)}} \\
\hline
\textbf{Mục} & \textbf{Nội dung} \\
\hline
Basic Flow (Triggered by Assignment Attempt) & 1. US-01 thực hiện hành động gán Nhân viên A vào Ca làm việc X (Thời gian T1 đến T2). \newline 2. Hệ thống nhận yêu cầu gán. \newline 3. Hệ thống truy vấn tất cả các Ca làm việc khác (ví dụ: Ca Y, thời gian T3 đến T4) đã được gán cho Nhân viên A. \newline 4. Hệ thống kiểm tra xem có Ca Y nào có khoảng thời gian (T3-T4) trùng lặp với khoảng thời gian của Ca X (T1-T2) hay không (theo BR-UC1.4-1). \newline 5. Giả sử Hệ thống phát hiện Ca Y trùng lặp với Ca X. \newline 6. Hệ thống đánh dấu cả Ca X và Ca Y là có xung đột. \newline 7. Hệ thống hiển thị Ca X và Ca Y trên giao diện Gantt chart với cảnh báo trực quan (theo BR-UC1.4-2). \newline 8. Hệ thống (có thể) hiển thị một thông báo xác nhận cho US-01 về việc xung đột đã được phát hiện (theo BR-UC1.4-3). \newline 9. Việc gán Ca X có thể được hoàn tất (với cảnh báo) hoặc bị hủy bỏ tùy thuộc vào cấu hình (BR-UC1.4-3). \\
\hline
Alternative Flow & \textbf{4a. Không phát hiện trùng lặp:} \newline    1. Hệ thống không tìm thấy Ca Y nào có thời gian trùng lặp với Ca X. \newline    2. Hệ thống tiến hành gán Nhân viên A vào Ca X bình thường (tiếp tục luồng thành công của UC-MD01-02). \newline \textbf{Basic Flow (Triggered by Loading Schedule):} \newline    1. US-01 truy cập giao diện lịch làm việc (UC-MD01-03). \newline    2. Hệ thống tải dữ liệu các ca làm việc trong phạm vi hiển thị. \newline    3. Đối với mỗi nhân viên có ca trong phạm vi hiển thị, Hệ thống thực hiện kiểm tra trùng lặp giữa tất cả các ca của nhân viên đó (tương tự bước 3-4 của Basic Flow Assignment). \newline    4. Nếu phát hiện bất kỳ cặp ca nào trùng lặp, Hệ thống đánh dấu và hiển thị cảnh báo trực quan cho các ca đó trên Gantt chart (bước 6-7). \\
\hline
Exception Flow & \textbf{3a. Lỗi truy vấn dữ liệu:} \newline    1. Hệ thống gặp lỗi khi cố gắng truy vấn danh sách các ca làm việc của nhân viên. \newline    2. Hệ thống có thể bỏ qua việc kiểm tra xung đột hoặc hiển thị thông báo lỗi. \newline    3. Quá trình gán/hiển thị lịch có thể tiếp tục nhưng không đảm bảo không có xung đột. \\
\hline
\multicolumn{2}{|c|}{\textbf{2.3. Thông tin bổ sung (Additional Information)}} \\
\hline
\textbf{Mục} & \textbf{Nội dung} \\
\hline
Business Rule & - \textbf{BR-UC1.4-1 (Overlap Definition):} Hai ca làm việc được coi là trùng lặp (xung đột) nếu khoảng thời gian của chúng có bất kỳ phần nào chung, bao gồm cả trường hợp thời gian bắt đầu hoặc kết thúc trùng nhau. Ví dụ: Ca1(8h-12h) và Ca2(11h-15h) là trùng lặp; Ca1(8h-12h) và Ca2(12h-16h) không được coi là trùng lặp. Cần xác nhận định nghĩa chính xác. Mặc định: (StartA < EndB) và (StartB < EndA). \newline - \textbf{BR-UC1.4-2 (Visual Warning):} Các ca làm việc bị xung đột phải được hiển thị với màu nền khác biệt (ví dụ: màu đỏ) và/hoặc có biểu tượng cảnh báo rõ ràng trên biểu đồ Gantt. \newline - \textbf{BR-UC1.4-3 (Conflict Handling):} Hệ thống phải cảnh báo về xung đột lịch nhưng cho phép Quản lý nhà hàng lưu lại việc gán ca bị trùng. Người quản lý chịu trách nhiệm giải quyết xung đột sau đó. \newline - \textbf{BR-UC1.4-4 (Scope of Check):} Kiểm tra xung đột phải được thực hiện đối với tất cả các ca làm việc đã gán cho nhân viên, bất kể trạng thái của ca là "Nháp" hay "Đã xuất bản". \\
\hline
Non-Functional Requirement & - \textbf{NFR-UC1.4-1 (Performance):} Việc kiểm tra và hiển thị cảnh báo xung đột khi gán một ca đơn lẻ phải hoàn thành trong vòng dưới 1 giây. Khi tải lịch trình tuần, việc kiểm tra xung đột cho tất cả nhân viên hiển thị không được làm tăng thời gian tải quá 20\%. \newline - \textbf{NFR-UC1.4-2 (Usability):} Cảnh báo trực quan phải rõ ràng, dễ nhận biết và không gây nhầm lẫn với các trạng thái khác của ca làm việc. \newline - \textbf{NFR-UC1.4-3 (Accuracy):} Logic phát hiện trùng lặp phải chính xác 100\% dựa trên định nghĩa trùng lặp (BR-UC1.4-1) và dữ liệu thời gian thực tế của các ca. \\
\hline

\end{longtable}

\subsubsection{Use Case UC-MD01-05: Xuất bản và Thông báo Lịch làm việc}

\begin{longtable}{|m{4cm}|p{11cm}|}
\caption{Đặc tả Use Case UC-MD01-05: Xuất bản và Thông báo Lịch làm việc} \label{tab:uc_md01_05} \\
\hline

\endhead % Header cho các trang tiếp theo

\hline
\endfoot % Footer cho bảng

\hline
\endlastfoot % Footer cho trang cuối cùng
\multicolumn{2}{|c|}{\textbf{2.1. Tóm tắt (Summary)}} \\
\hline
\textbf{Mục} & \textbf{Nội dung} \\
\hline
Use Case Name & Xuất bản và Thông báo Lịch làm việc \\
\hline
Use Case ID & UC-MD01-05 \\
\hline
Use Case Description & Cho phép Quản lý nhà hàng chính thức hóa lịch làm việc đã xếp (chuyển từ Nháp sang Xuất bản) và tự động gửi thông báo lịch trình cá nhân đến từng nhân viên liên quan qua email hoặc kênh khác đã cấu hình. \\
\hline
Actor & US-01 (Quản lý nhà hàng), System (Thực hiện thay đổi trạng thái và gửi thông báo) \\
\hline
Priority & Must Have \\
\hline
Trigger & Quản lý nhà hàng đã hoàn tất việc xếp lịch cho một khoảng thời gian (ví dụ: tuần tới) và muốn công bố chính thức cho nhân viên. \\
\hline
Pre-Condition & - US-01 đã đăng nhập với quyền quản lý lịch trình. \newline - Tồn tại ít nhất một ca làm việc ở trạng thái "Nháp" trong phạm vi cần xuất bản. \newline - Thông tin liên hệ (ví dụ: địa chỉ email) của các nhân viên liên quan đã được cấu hình chính xác trong hệ thống. \newline - Hệ thống gửi thông báo (ví dụ: Email Server) đã được cấu hình và hoạt động. \\
\hline
Post-Condition & - Các ca làm việc được chọn trong phạm vi xuất bản chuyển trạng thái từ "Nháp" sang "Đã xuất bản". \newline - Các ca "Đã xuất bản" trở nên chính thức và có thể được xem bởi nhân viên liên quan (US-07 qua FR-MD01-06). \newline - Thông báo chứa tóm tắt lịch trình cá nhân được gửi (hoặc đưa vào hàng đợi gửi) thành công đến email (hoặc kênh khác) của từng nhân viên có ca được xuất bản. \newline - Giao diện Gantt chart cập nhật, thể hiện trạng thái "Đã xuất bản" của các ca (ví dụ: thay đổi màu sắc/mất sọc chéo). \newline - Hệ thống ghi nhận hoạt động xuất bản vào nhật ký. \\
\hline
\multicolumn{2}{|c|}{\textbf{2.2. Luồng thực thi (Flow)}} \\
\hline
\textbf{Mục} & \textbf{Nội dung} \\
\hline
Basic Flow & 1. US-01 truy cập giao diện quản lý lịch làm việc (ví dụ: Gantt view). \newline 2. US-01 chọn(các) ca làm việc ở trạng thái "Nháp" muốn xuất bản HOẶC chọn một hành động xuất bản theo phạm vi (ví dụ: nút "Publish" cho tuần hiện tại/kế tiếp). \newline 3. Hệ thống (có thể) hiển thị danh sách các ca sẽ được xuất bản và yêu cầu xác nhận từ US-01. \newline 4. US-01 xác nhận hành động xuất bản. \newline 5. Hệ thống xác định tất cả các ca làm việc "Nháp" thuộc phạm vi đã chọn. \newline 6. Hệ thống cập nhật trạng thái của các ca này thành "Đã xuất bản" trong cơ sở dữ liệu. \newline 7. Hệ thống xác định danh sách các nhân viên có ca làm việc vừa được xuất bản. \newline 8. Đối với mỗi nhân viên trong danh sách: \newline    a. Hệ thống tạo nội dung thông báo (ví dụ: email) liệt kê các ca làm việc (Ngày, Giờ bắt đầu, Giờ kết thúc, Vai trò) của nhân viên đó trong phạm vi vừa xuất bản. \newline    b. Hệ thống gửi thông báo này đến địa chỉ email (hoặc kênh khác) đã đăng ký của nhân viên (thường thông qua hàng đợi mail). \newline 9. Hệ thống cập nhật giao diện Gantt chart để phản ánh trạng thái "Đã xuất bản" mới của các ca. \newline 10. Hệ thống hiển thị thông báo thành công cho US-01 (ví dụ: "Lịch trình đã được xuất bản và thông báo đã được gửi"). \newline 11. Hệ thống ghi nhận hoạt động vào Activity Log. \\
\hline
Alternative Flow & \textbf{2a. Xuất bản và gửi sau:} \newline    1. US-01 chọn tùy chọn "Chỉ xuất bản" (nếu hệ thống hỗ trợ). \newline    2. Hệ thống thực hiện các bước 5, 6, 9, 10, 11 của Basic Flow nhưng bỏ qua bước 7, 8 (không gửi thông báo). \newline    3. Sau đó, US-01 có thể thực hiện một hành động riêng biệt "Gửi thông báo lịch trình" cho các ca đã xuất bản. \newline (Lưu ý: Cần kiểm tra khả năng cấu hình cho luồng này, luồng mặc định thường là xuất bản kèm thông báo). \newline \textbf{3a. Thông báo xem trước nhân viên bị ảnh hưởng:} \newline    1. Trước khi US-01 xác nhận (bước 4), hệ thống hiển thị danh sách các nhân viên sẽ nhận được thông báo. \newline    2. Use Case tiếp tục từ bước 4 của Basic Flow. \\
\hline
Exception Flow & \textbf{2b. Không có ca nháp nào được chọn/tìm thấy:} \newline    1. Hệ thống không tìm thấy ca "Nháp" nào trong phạm vi US-01 đã chọn. \newline    2. Hệ thống hiển thị thông báo "Không có ca làm việc nào ở trạng thái Nháp để xuất bản." \newline    3. Use Case kết thúc. \newline \textbf{6a. Lỗi cập nhật trạng thái ca:} \newline    1. Hệ thống gặp lỗi khi cố gắng cập nhật trạng thái ca trong cơ sở dữ liệu. \newline    2. Hệ thống hiển thị thông báo lỗi "Không thể cập nhật trạng thái ca làm việc." \newline    3. Use Case kết thúc trong trạng thái lỗi, các ca có thể vẫn ở trạng thái "Nháp". \newline \textbf{8c. Lỗi gửi thông báo:} \newline    1. Hệ thống gặp lỗi khi cố gắng tạo hoặc gửi thông báo cho một hoặc nhiều nhân viên (ví dụ: email không hợp lệ, lỗi máy chủ mail). \newline    2. Hệ thống (nên) ghi nhận lỗi chi tiết vào log hệ thống. \newline    3. Hệ thống (nên) hoàn thành việc xuất bản các ca (bước 6, 9). \newline    4. Hệ thống hiển thị thông báo cho US-01, ví dụ: "Lịch trình đã được xuất bản, nhưng đã xảy ra lỗi khi gửi thông báo cho một số nhân viên. Vui lòng kiểm tra nhật ký lỗi." \\
\hline
\multicolumn{2}{|c|}{\textbf{2.3. Thông tin bổ sung (Additional Information)}} \\
\hline
\textbf{Mục} & \textbf{Nội dung} \\
\hline
Business Rule & - \textbf{BR-UC1.5-1:} Chỉ các ca làm việc ở trạng thái "Nháp" mới có thể được chuyển sang trạng thái "Đã xuất bản" thông qua chức năng này. \newline - \textbf{BR-UC1.5-2:} Hành động "Xuất bản" được coi là hành động chính thức hóa lịch làm việc. \newline - \textbf{BR-UC1.5-3:} Thông báo phải được gửi đến địa chỉ email công ty (hoặc kênh liên lạc chính thức khác) được định nghĩa trong hồ sơ nhân viên trên hệ thống. \newline - \textbf{BR-UC1.5-4:} Nội dung thông báo tối thiểu phải bao gồm tên nhân viên, danh sách các ca làm việc được xuất bản (Ngày, Giờ bắt đầu, Giờ kết thúc, Vai trò). \newline - \textbf{BR-UC1.5-5:} Việc xuất bản có thể áp dụng cho toàn bộ lịch trình trong một khoảng thời gian (ví dụ: một tuần) hoặc cho các ca được chọn thủ công. \newline - \textbf{BR-UC1.5-6:} Sau khi xuất bản, việc thay đổi ca (ví dụ: hủy ca, đổi nhân viên) có thể yêu cầu quy trình khác (ví dụ: hủy xuất bản trước hoặc gửi thông báo cập nhật). \\
\hline
Non-Functional Requirement & - \textbf{NFR-UC1.5-1 (Performance):} Thời gian xử lý cho việc xuất bản lịch trình của một tuần (~200 ca) và đưa thông báo vào hàng đợi phải dưới 10 giây. \newline - \textbf{NFR-UC1.5-2 (Reliability):} Hệ thống gửi thông báo phải đảm bảo email được đưa vào hàng đợi thành công. Việc giám sát hàng đợi email là một phần vận hành hệ thống nói chung. \newline - \textbf{NFR-UC1.5-3 (Usability):} Phải có phản hồi rõ ràng cho người quản lý biết hành động xuất bản và gửi thông báo đã thành công hay gặp lỗi. Nút/hành động "Publish" phải dễ dàng nhận biết và sử dụng. \newline - \textbf{NFR-UC1.5-4 (Security):} Chỉ người dùng có quyền hạn quản lý lịch trình (thường là US-01) mới có thể thực hiện hành động xuất bản. \\
\hline

\end{longtable}


\subsubsection{Use Case UC-MD01-06: Xem lịch làm việc cá nhân}
\begin{longtable}{|m{4cm}|p{11cm}|}
\caption{Đặc tả Use Case UC-MD01-06: Xem lịch làm việc cá nhân} \label{tab:uc_md01_06} \\
\hline

\endhead % Header cho các trang tiếp theo

\hline
\endfoot % Footer cho bảng

\hline
\endlastfoot % Footer cho trang cuối cùng
\multicolumn{2}{|c|}{\textbf{2.1. Tóm tắt (Summary)}} \\
\hline
\textbf{Mục} & \textbf{Nội dung} \\
\hline
Use Case Name & Xem lịch làm việc cá nhân \\
\hline
Use Case ID & UC-MD01-06 \\
\hline
Use Case Description & Cho phép Nhân viên (US-07) xem các ca làm việc đã được Quản lý nhà hàng (US-01) xuất bản chính thức cho bản thân mình thông qua cổng thông tin nhân viên hoặc ứng dụng di động. \\
\hline
Actor & US-07 (Nhân viên) \\
\hline
Priority & Must Have \\
\hline
Trigger & Nhân viên muốn biết lịch trình làm việc sắp tới của mình. \\
\hline
Pre-Condition & - Nhân viên (US-07) có tài khoản và mật khẩu hợp lệ để đăng nhập vào cổng thông tin/ứng dụng nhân viên. \newline - Quản lý nhà hàng (US-01) đã tạo, gán và xuất bản (FR-MD01-05) ít nhất một ca làm việc cho nhân viên này. \newline - Cổng thông tin/ứng dụng nhân viên đang hoạt động. \\
\hline
Post-Condition & - Lịch làm việc đã được xuất bản của nhân viên được hiển thị rõ ràng cho nhân viên đó xem. \newline - Nhân viên nắm được thông tin về các ca làm việc sắp tới của mình (ngày, giờ, vai trò). \\
\hline
\multicolumn{2}{|c|}{\textbf{2.2. Luồng thực thi (Flow)}} \\
\hline
\textbf{Mục} & \textbf{Nội dung} \\
\hline
Basic Flow & 1. Nhân viên (US-07) mở ứng dụng di động hoặc truy cập trang web cổng thông tin nhân viên. \newline 2. US-07 nhập thông tin đăng nhập (tên người dùng/email và mật khẩu) và chọn Đăng nhập. \newline 3. Hệ thống xác thực thông tin đăng nhập thành công (Xử lý bởi UC Đăng nhập chung). \newline 4. US-07 chọn mục menu "Lịch làm việc của tôi" hoặc tương tự. \newline 5. Hệ thống truy xuất tất cả các ca làm việc có trạng thái "Đã xuất bản" được gán cho US-07 trong khoảng thời gian mặc định (ví dụ: tuần hiện tại). \newline 6. Hệ thống hiển thị danh sách các ca làm việc này, bao gồm thông tin: Ngày, Giờ bắt đầu, Giờ kết thúc, Vai trò. \newline 7. US-07 xem lịch trình của mình. \\
\hline
Alternative Flow & \textbf{6a. Thay đổi khoảng thời gian xem:} \newline    1. US-07 sử dụng các điều khiển (ví dụ: chọn tuần/tháng, lịch nhỏ) để chọn một khoảng thời gian khác muốn xem. \newline    2. Hệ thống thực hiện lại bước 5 và 6 cho khoảng thời gian mới. \newline \textbf{6b. Thay đổi định dạng hiển thị:} \newline    1. Nếu hệ thống hỗ trợ nhiều định dạng (ví dụ: danh sách, lịch tháng), US-07 chọn định dạng mong muốn. \newline    2. Hệ thống thực hiện lại bước 6 với định dạng mới. \\
\hline
Exception Flow & \textbf{3a. Đăng nhập không thành công:} \newline    1. Hệ thống xác thực thông tin đăng nhập thất bại. \newline    2. Hệ thống hiển thị thông báo lỗi đăng nhập. \newline    3. Use Case kết thúc (hoặc quay lại bước 2). \newline \textbf{5a. Không có lịch làm việc được xuất bản:} \newline    1. Hệ thống không tìm thấy ca làm việc nào có trạng thái "Đã xuất bản" cho US-07 trong khoảng thời gian được chọn. \newline    2. Hệ thống hiển thị thông báo "Bạn không có lịch làm việc nào được xếp trong khoảng thời gian này." hoặc hiển thị lịch trống. \newline    3. Use Case có thể kết thúc hoặc chờ US-07 thay đổi khoảng thời gian. \newline \textbf{5b. Lỗi hệ thống khi truy xuất dữ liệu:} \newline    1. Hệ thống gặp sự cố kỹ thuật khi cố gắng lấy dữ liệu lịch trình từ cơ sở dữ liệu. \newline    2. Hệ thống hiển thị thông báo lỗi chung (ví dụ: "Không thể tải lịch làm việc. Vui lòng thử lại sau."). \newline    3. Use Case kết thúc trong trạng thái lỗi. \\
\hline
\multicolumn{2}{|c|}{\textbf{2.3. Thông tin bổ sung (Additional Information)}} \\
\hline
\textbf{Mục} & \textbf{Nội dung} \\
\hline
Business Rule & - \textbf{BR-UC1.6-1:} Nhân viên chỉ có thể xem các ca làm việc của chính mình. Không được phép xem lịch của nhân viên khác (trừ khi có cấu hình đặc biệt cho "ca mở" hoặc vai trò quản lý). \newline - \textbf{BR-UC1.6-2:} Chỉ những ca làm việc có trạng thái "Đã xuất bản" mới được hiển thị cho nhân viên. Các ca "Nháp" không hiển thị. \newline - \textbf{BR-UC1.6-3:} Thông tin tối thiểu cần hiển thị cho mỗi ca: Ngày, Giờ bắt đầu, Giờ kết thúc, Vai trò được gán. \newline - \textbf{BR-UC1.6-4:} Định dạng hiển thị mặc định là danh sách theo tuần hiện tại. \\
\hline
Non-Functional Requirement & - \textbf{NFR-UC1.6-1 (Usability):} Giao diện xem lịch phải đơn giản, rõ ràng, dễ đọc trên cả máy tính và thiết bị di động (nếu có app). \newline - \textbf{NFR-UC1.6-2 (Performance):} Thời gian tải lịch làm việc của tuần hiện tại phải dưới 3 giây. \newline - \textbf{NFR-UC1.6-3 (Security):} Đảm bảo cơ chế xác thực và phân quyền chỉ cho phép nhân viên xem đúng lịch của mình. \newline - \textbf{NFR-UC1.6-4 (Availability):} Cổng thông tin/ứng dụng nhân viên nên có độ sẵn sàng cao, cho phép nhân viên kiểm tra lịch bất cứ lúc nào (tùy theo chính sách công ty). \\
\hline

\end{longtable}

\subsubsection{Use Case UC-MD01-07: Sao chép lịch tuần}
\begin{longtable}{|m{4cm}|p{11cm}|}
\caption{Đặc tả Use Case UC-MD01-07: Sao chép lịch tuần} \label{tab:uc_md01_07} \\
\hline

\endhead % Header cho các trang tiếp theo

\hline
\endfoot % Footer cho bảng

\hline
\endlastfoot % Footer cho trang cuối cùng
\multicolumn{2}{|c|}{\textbf{2.1. Tóm tắt (Summary)}} \\
\hline
\textbf{Mục} & \textbf{Nội dung} \\
\hline
Use Case Name & Sao chép lịch tuần \\
\hline
Use Case ID & UC-MD01-07 \\
\hline
Use Case Description & Cho phép Quản lý nhà hàng tạo nhanh lịch làm việc cho một tuần mới bằng cách sao chép toàn bộ cấu trúc ca (và tùy chọn cả phân công nhân viên) từ một tuần đã có lịch trước đó. \\
\hline
Actor & US-01 (Quản lý nhà hàng) \\
\hline
Priority & Should Have \\
\hline
Trigger & Quản lý nhà hàng muốn tiết kiệm thời gian khi lập lịch cho tuần mới, dựa trên một lịch trình tuần cũ tương tự. \\
\hline
Pre-Condition & - US-01 đã đăng nhập với quyền quản lý lịch trình. \newline - Tồn tại lịch làm việc (ít nhất một ca) trong tuần được chọn làm nguồn. \newline - Giao diện lịch làm việc đang hiển thị. \\
\hline
Post-Condition & - Các ca làm việc mới, giống hệt (về thời gian, vai trò, tùy chọn cả nhân viên) các ca của tuần nguồn, được tạo ra trong tuần đích với trạng thái "Nháp". \newline - Giao diện Gantt chart cập nhật, hiển thị các ca nháp mới trong tuần đích. \newline - Quản lý nhà hàng có thể chỉnh sửa các ca nháp mới này trước khi xuất bản. \newline - Hệ thống ghi nhận hoạt động sao chép vào nhật ký. \\
\hline
\multicolumn{2}{|c|}{\textbf{2.2. Luồng thực thi (Flow)}} \\
\hline
\textbf{Mục} & \textbf{Nội dung} \\
\hline
Basic Flow & 1. US-01 truy cập giao diện quản lý lịch làm việc (ví dụ: Gantt view). \newline 2. US-01 đảm bảo đang xem tuần mà mình muốn dùng làm nguồn (Tuần Nguồn). \newline 3. US-01 chọn hành động "Sao chép tuần" (Copy Week) (thường có trong menu hoặc nút lệnh). \newline 4. Hệ thống (có thể) yêu cầu xác nhận tuần nguồn và tuần đích (mặc định là tuần kế tiếp của tuần nguồn - Tuần Đích). US-01 xác nhận. \newline 5. Hệ thống truy vấn và lấy thông tin tất cả các ca làm việc (bao gồm thời gian, vai trò, nhân viên đã gán nếu có) thuộc Tuần Nguồn. \newline 6. Đối với mỗi ca làm việc lấy được từ Tuần Nguồn: \newline    a. Hệ thống tạo một bản ghi ca làm việc mới trong Tuần Đích với cùng thời gian trong tuần (ví dụ: Thứ Hai 9:00-17:00), cùng vai trò. \newline    b. (Theo cấu hình/mặc định) Hệ thống cũng sao chép luôn việc gán nhân viên từ ca nguồn sang ca đích mới. \newline    c. Hệ thống đặt trạng thái của ca làm việc mới này là "Nháp". \newline 7. Hệ thống cập nhật giao diện Gantt chart để hiển thị Tuần Đích với các ca làm việc mới vừa tạo. \newline 8. Hệ thống hiển thị thông báo sao chép thành công. \newline 9. Hệ thống ghi nhận hoạt động vào Activity Log. \\
\hline
Alternative Flow & \textbf{4a. Chọn tuần đích khác:} \newline    1. Trước khi xác nhận, US-01 chọn một tuần đích khác (không phải tuần kế tiếp). \newline    2. Use Case tiếp tục từ bước 5 của Basic Flow với tuần đích đã chọn. \newline \textbf{6b-alt. Sao chép không kèm phân công:} \newline    1. Trước khi kích hoạt sao chép (bước 3 hoặc 4), US-01 bỏ chọn tùy chọn "Sao chép phân công nhân viên". \newline    2. Trong bước 6b, Hệ thống chỉ sao chép thời gian và vai trò, không sao chép nhân viên đã gán. Ca mới tạo ra ở tuần đích sẽ ở trạng thái "Nháp" và "Chưa phân công". \\
\hline
Exception Flow & \textbf{3a. Không có quyền sao chép:} \newline    1. US-01 không có quyền thực hiện chức năng này. \newline    2. Hành động "Sao chép tuần" bị vô hiệu hóa hoặc hệ thống báo lỗi quyền hạn. \newline    3. Use Case kết thúc. \newline \textbf{5a. Tuần nguồn không có ca nào:} \newline    1. Hệ thống không tìm thấy ca làm việc nào trong Tuần Nguồn. \newline    2. Hệ thống hiển thị thông báo "Tuần nguồn không có lịch trình để sao chép." \newline    3. Use Case kết thúc. \newline \textbf{6d. Lỗi hệ thống khi tạo ca mới:} \newline    1. Hệ thống gặp sự cố kỹ thuật khi cố gắng tạo bản ghi ca mới trong cơ sở dữ liệu. \newline    2. Hệ thống có thể dừng quá trình sao chép hoặc bỏ qua ca bị lỗi và tiếp tục. \newline    3. Hệ thống hiển thị thông báo lỗi (ví dụ: "Đã xảy ra lỗi trong quá trình sao chép. Một số ca có thể chưa được sao chép."). \\
\hline
\multicolumn{2}{|c|}{\textbf{2.3. Thông tin bổ sung (Additional Information)}} \\
\hline
\textbf{Mục} & \textbf{Nội dung} \\
\hline
Business Rule & - \textbf{BR-UC1.7-1:} Tất cả các ca làm việc được tạo ra từ quá trình sao chép đều phải có trạng thái ban đầu là "Nháp". \newline - \textbf{BR-UC1.7-2:} Hành động sao chép tạo ra các bản ghi ca mới, không ghi đè lên các ca có thể đã tồn tại trong tuần đích. Nếu tuần đích đã có ca, kết quả là sẽ có các ca trùng lặp (ở trạng thái nháp) mà người quản lý cần xử lý sau. \newline - \textbf{BR-UC1.7-3:} Mặc định, việc gán nhân viên cũng được sao chép từ tuần nguồn sang tuần đích. \newline - \textbf{BR-UC1.7-4:} Sau khi sao chép, các ca mới ở tuần đích phải tuân theo tất cả các quy tắc khác (ví dụ: kiểm tra trùng lịch FR-MD01-04 sẽ được áp dụng khi xem/xuất bản tuần đích). \\
\hline
Non-Functional Requirement & - \textbf{NFR-UC1.7-1 (Performance):} Thời gian hoàn thành việc sao chép lịch của một tuần tiêu chuẩn (~200 ca) phải dưới 5 giây. \newline - \textbf{NFR-UC1.7-2 (Usability):} Chức năng "Sao chép tuần" phải dễ tìm và dễ sử dụng. Việc chọn tuần nguồn/đích phải trực quan. \newline - \textbf{NFR-UC1.7-3 (Accuracy):} Thông tin (thời gian, vai trò, nhân viên - nếu có) của các ca mới tạo phải sao chép chính xác từ các ca tương ứng của tuần nguồn. \\
\hline

\end{longtable}

\subsubsection{Use Case UC-MD01-08: Quản lý vai trò công việc}


\begin{longtable}{|m{4cm}|p{11cm}|}
\caption{Đặc tả Use Case UC-MD01-08: Quản lý vai trò công việc} \label{tab:uc_md01_08} \\
\hline

\endhead % Header cho các trang tiếp theo

\hline
\endfoot % Footer cho bảng

\hline
\endlastfoot % Footer cho trang cuối cùng
\multicolumn{2}{|c|}{\textbf{2.1. Tóm tắt (Summary)}} \\
\hline
\textbf{Mục} & \textbf{Nội dung} \\
\hline
Use Case Name & Quản lý vai trò công việc \\
\hline
Use Case ID & UC-MD01-08 \\
\hline
Use Case Description & Cho phép Quản lý nhà hàng tạo, xem, sửa đổi và xóa các vai trò công việc (ví dụ: Bếp trưởng, Phục vụ, Pha chế) được sử dụng trong module lập lịch (Planning) để định nghĩa yêu cầu nhân sự cho các ca làm việc và gán cho nhân viên. \\
\hline
Actor & US-01 (Quản lý nhà hàng) \\
\hline
Priority & Must Have \\
\hline
Trigger & Quản lý nhà hàng cần định nghĩa một vai trò công việc mới, cập nhật thông tin vai trò hiện có, hoặc loại bỏ một vai trò không còn sử dụng. \\
\hline
Pre-Condition & - US-01 đã đăng nhập vào hệ thống với quyền quản trị module Planning hoặc cấu hình nhân sự liên quan. \\
\hline
Post-Condition & - \textbf{Tạo mới:} Một bản ghi vai trò công việc mới được tạo và lưu trong hệ thống, sẵn sàng để sử dụng khi tạo ca hoặc gán cho nhân viên. \newline - \textbf{Sửa đổi:} Thông tin của vai trò công việc đã chọn được cập nhật trong hệ thống. \newline - \textbf{Xóa:} Vai trò công việc đã chọn bị xóa khỏi hệ thống (nếu không đang được sử dụng). \newline - Danh sách vai trò công việc trên giao diện được cập nhật tương ứng. \newline - Hệ thống ghi nhận hoạt động vào nhật ký. \\
\hline
\multicolumn{2}{|c|}{\textbf{2.2. Luồng thực thi (Flow)}} \\
\hline
\textbf{Mục} & \textbf{Nội dung} \\
\hline
Basic Flow (Xem và Tạo mới) & 1. US-01 truy cập vào khu vực quản lý/cấu hình Vai trò công việc (trong Settings của Planning hoặc HR). \newline 2. Hệ thống hiển thị danh sách các vai trò công việc hiện có. \newline 3. US-01 chọn hành động "Tạo mới". \newline 4. Hệ thống hiển thị form để nhập thông tin vai trò mới. \newline 5. US-01 nhập Tên vai trò (bắt buộc). \newline 6. (Tùy chọn) US-01 nhập Mô tả, chọn Màu sắc đại diện (dùng trên Gantt), hoặc các thuộc tính khác nếu có. \newline 7. US-01 chọn lệnh "Lưu". \newline 8. Hệ thống kiểm tra tính hợp lệ (ví dụ: tên không trùng - BR-UC1.8-1). \newline 9. Hệ thống lưu bản ghi vai trò mới vào cơ sở dữ liệu. \newline 10. Hệ thống cập nhật danh sách vai trò, hiển thị vai trò mới được tạo. \newline 11. Hệ thống hiển thị thông báo tạo thành công. \newline 12. Hệ thống ghi nhận hoạt động vào Activity Log. \\
\hline
Alternative Flow & \textbf{3a. Sửa vai trò:} \newline    1. Từ danh sách vai trò (bước 2), US-01 chọn một vai trò muốn sửa. \newline    2. Hệ thống hiển thị form với thông tin hiện tại của vai trò đó. \newline    3. US-01 chỉnh sửa các thông tin cần thiết (Tên, Mô tả, Màu sắc...). \newline    4. US-01 chọn lệnh "Lưu". \newline    5. Hệ thống kiểm tra tính hợp lệ (ví dụ: tên không trùng với vai trò khác). \newline    6. Hệ thống cập nhật thông tin cho bản ghi vai trò đã chọn. \newline    7. Hệ thống cập nhật danh sách vai trò (nếu tên thay đổi). \newline    8. Hệ thống hiển thị thông báo cập nhật thành công. \newline    9. Hệ thống ghi nhận hoạt động vào Activity Log. \newline \textbf{3b. Xóa vai trò:} \newline    1. Từ danh sách vai trò (bước 2), US-01 chọn một vai trò muốn xóa. \newline    2. US-01 chọn hành động "Xóa". \newline    3. Hệ thống hiển thị hộp thoại yêu cầu xác nhận xóa. \newline    4. US-01 xác nhận muốn xóa. \newline    5. Hệ thống kiểm tra xem vai trò này có đang được gán cho bất kỳ nhân viên nào hoặc sử dụng trong bất kỳ ca làm việc nào (kể cả nháp) hay không (BR-UC1.8-2). \newline    6. Nếu KHÔNG đang sử dụng: \newline       a. Hệ thống xóa bản ghi vai trò khỏi cơ sở dữ liệu. \newline       b. Hệ thống cập nhật danh sách vai trò. \newline       c. Hệ thống hiển thị thông báo xóa thành công. \newline       d. Hệ thống ghi nhận hoạt động vào Activity Log. \newline    7. Nếu CÓ đang sử dụng: \newline       a. Hệ thống hiển thị thông báo lỗi, nêu rõ không thể xóa vì vai trò đang được sử dụng. \newline       b. Use Case kết thúc (xóa thất bại). \newline \textbf{2a. Tìm kiếm/Lọc vai trò:} \newline    1. Nếu danh sách vai trò dài, US-01 sử dụng thanh tìm kiếm để tìm vai trò theo tên. \newline    2. Hệ thống lọc và hiển thị kết quả tìm kiếm. \newline    3. Use Case tiếp tục từ bước 3 (chọn hành động). \\
\hline
Exception Flow & \textbf{8a/5a. Dữ liệu không hợp lệ khi Lưu/Cập nhật:} \newline    1. Hệ thống phát hiện Tên vai trò bị bỏ trống hoặc trùng với tên vai trò khác đã tồn tại. \newline    2. Hệ thống hiển thị thông báo lỗi cụ thể (ví dụ: "Tên vai trò không được để trống", "Tên vai trò đã tồn tại"). \newline    3. Hệ thống giữ nguyên form và cho phép US-01 sửa lại. Use Case quay lại bước 5 (Tạo mới) hoặc bước 3 (Sửa). \newline \textbf{9a/6a/6a-delete. Lỗi hệ thống khi Lưu/Cập nhật/Xóa:} \newline    1. Hệ thống gặp sự cố kỹ thuật khi tương tác với cơ sở dữ liệu. \newline    2. Hệ thống hiển thị thông báo lỗi chung. \newline    3. Use Case kết thúc trong trạng thái lỗi. \\
\hline
\multicolumn{2}{|c|}{\textbf{2.3. Thông tin bổ sung (Additional Information)}} \\
\hline
\textbf{Mục} & \textbf{Nội dung} \\
\hline
Business Rule & - \textbf{BR-UC1.8-1:} Tên của mỗi vai trò công việc phải là duy nhất trong hệ thống. \newline - \textbf{BR-UC1.8-2:} Một vai trò công việc không thể bị xóa nếu nó đang được gán cho ít nhất một nhân viên hoặc đang được sử dụng trong ít nhất một ca làm việc (kể cả trạng thái "Nháp" hoặc "Đã xuất bản"). Người quản lý phải gỡ bỏ vai trò khỏi tất cả các nơi đang sử dụng trước khi xóa. \newline - \textbf{BR-UC1.8-3:} Việc sửa tên vai trò sẽ tự động cập nhật tên này ở tất cả những nơi nó đang được sử dụng (ví dụ: trong thông tin nhân viên, trong các ca làm việc đã tạo). \newline - \textbf{BR-UC1.8-4:} Phải có ít nhất một vai trò công việc được định nghĩa trong hệ thống để module Planning hoạt động đúng. Không thể xóa vai trò cuối cùng. \\
\hline
Non-Functional Requirement & - \textbf{NFR-UC1.8-1 (Usability):} Giao diện quản lý vai trò phải đơn giản, dễ dàng thực hiện các thao tác CRUD. Việc chọn màu sắc nên có công cụ chọn màu trực quan. \newline - \textbf{NFR-UC1.8-2 (Performance):} Thời gian hiển thị danh sách vai trò (< 100 vai trò) và lưu/cập nhật/xóa một vai trò phải dưới 2 giây. \newline - \textbf{NFR-UC1.8-3 (Security):} Chỉ người dùng có quyền quản trị cấu hình Planning/HR mới được phép thực hiện các thao tác CRUD đối với vai trò công việc. \\
\hline

\end{longtable}

\subsubsection{Use Case UC-MD01-09: Đánh dấu không sẵn sàng làm việc}
\begin{longtable}{|m{4cm}|p{11cm}|}
\caption{Đặc tả Use Case UC-MD01-09: Đánh dấu không sẵn sàng làm việc} \label{tab:uc_md01_09} \\
\hline

\endhead % Header cho các trang tiếp theo

\hline
\endfoot % Footer cho bảng

\hline
\endlastfoot % Footer cho trang cuối cùng
\multicolumn{2}{|c|}{\textbf{2.1. Tóm tắt (Summary)}} \\
\hline
\textbf{Mục} & \textbf{Nội dung} \\
\hline
Use Case Name & Đánh dấu không sẵn sàng làm việc \\
\hline
Use Case ID & UC-MD01-09 \\
\hline
Use Case Description & Cho phép Nhân viên (US-07) chủ động thông báo cho hệ thống (và Quản lý nhà hàng) về các khoảng thời gian cụ thể mà họ không thể làm việc (ví dụ: do nghỉ phép, lịch hẹn cá nhân), thông qua cổng thông tin nhân viên hoặc ứng dụng di động. \\
\hline
Actor & US-07 (Nhân viên) \\
\hline
Priority & Should Have \\
\hline
Trigger & Nhân viên biết trước mình sẽ không thể làm việc trong một khoảng thời gian và muốn thông báo cho quản lý để tránh bị xếp lịch vào thời gian đó. \\
\hline
Pre-Condition & - Nhân viên (US-07) có tài khoản và mật khẩu hợp lệ để đăng nhập vào cổng thông tin/ứng dụng nhân viên. \newline - Cổng thông tin/ứng dụng nhân viên có chức năng cho phép nhập thời gian không sẵn sàng. \newline - (Tùy chọn) Chức năng này đã được Quản lý nhà hàng (US-01) kích hoạt cho phép nhân viên sử dụng. \\
\hline
Post-Condition & - Một bản ghi về khoảng thời gian không sẵn sàng (ngày bắt đầu, giờ bắt đầu, ngày kết thúc, giờ kết thúc, tùy chọn lý do) được tạo và liên kết với nhân viên đó trong hệ thống. \newline - Thông tin này sẽ được hiển thị hoặc được hệ thống xem xét khi Quản lý nhà hàng (US-01) thực hiện gán ca làm việc (trong UC-MD01-02), giúp tránh xếp lịch cho nhân viên vào thời gian họ đã báo bận. \newline - Hệ thống ghi nhận hoạt động vào nhật ký. \\
\hline
\multicolumn{2}{|c|}{\textbf{2.2. Luồng thực thi (Flow)}} \\
\hline
\textbf{Mục} & \textbf{Nội dung} \\
\hline
Basic Flow & 1. Nhân viên (US-07) mở ứng dụng di động hoặc truy cập trang web cổng thông tin nhân viên. \newline 2. US-07 đăng nhập thành công vào hệ thống. \newline 3. US-07 điều hướng đến chức năng "Báo cáo không sẵn sàng" hoặc tương tự (có thể nằm trong mục "Lịch của tôi"). \newline 4. Hệ thống hiển thị giao diện để nhập thông tin. \newline 5. US-07 chọn/nhập Ngày bắt đầu và Giờ bắt đầu không sẵn sàng. \newline 6. US-07 chọn/nhập Ngày kết thúc và Giờ kết thúc không sẵn sàng. \newline 7. (Tùy chọn) US-07 nhập Lý do không sẵn sàng vào ô ghi chú. \newline 8. US-07 chọn lệnh "Lưu" hoặc "Gửi". \newline 9. Hệ thống kiểm tra tính hợp lệ của ngày giờ (ví dụ: thời gian kết thúc phải sau thời gian bắt đầu - BR-UC1.9-1). \newline 10. Hệ thống lưu bản ghi thông tin không sẵn sàng này vào cơ sở dữ liệu, gắn với tài khoản của US-07. \newline 11. Hệ thống hiển thị thông báo lưu thành công. \newline 12. Hệ thống ghi nhận hoạt động vào Activity Log. \\
\hline
Alternative Flow & \textbf{3a. Đánh dấu trực tiếp trên lịch:} \newline    1. Nếu giao diện hỗ trợ, US-07 có thể chọn trực tiếp một khoảng thời gian trên lịch cá nhân và chọn hành động "Đánh dấu không sẵn sàng". \newline    2. Hệ thống tự động điền ngày giờ vào form (bước 5, 6). \newline    3. Use Case tiếp tục từ bước 7. \newline \textbf{10a. Xem lại/Sửa/Xóa lịch không sẵn sàng đã báo:} \newline    1. Sau khi lưu, hoặc vào một thời điểm khác, US-07 truy cập lại chức năng này. \newline    2. Hệ thống hiển thị danh sách các khoảng thời gian không sẵn sàng mà US-07 đã báo trước đó. \newline    3. US-07 có thể chọn một bản ghi để xem chi tiết, sửa đổi (ví dụ: thay đổi ngày giờ, lý do) hoặc xóa bỏ nếu kế hoạch thay đổi. Quy trình sửa/xóa tương tự như các UC quản lý dữ liệu khác. \\
\hline
Exception Flow & \textbf{9a. Dữ liệu ngày giờ không hợp lệ:} \newline    1. Hệ thống phát hiện Ngày/Giờ kết thúc trước hoặc bằng Ngày/Giờ bắt đầu. \newline    2. Hệ thống hiển thị thông báo lỗi cụ thể. \newline    3. Hệ thống giữ nguyên thông tin đã nhập và cho phép US-07 sửa lại. Use Case quay lại bước 5. \newline \textbf{10a. Lỗi hệ thống khi lưu:} \newline    1. Hệ thống gặp sự cố kỹ thuật khi cố gắng lưu dữ liệu vào cơ sở dữ liệu. \newline    2. Hệ thống hiển thị thông báo lỗi chung. \newline    3. Use Case kết thúc trong trạng thái lỗi. \newline \textbf{3b. Chức năng bị vô hiệu hóa:} \newline    1. Quản lý nhà hàng chưa kích hoạt chức năng này cho nhân viên. \newline    2. Nhân viên không tìm thấy mục menu hoặc chức năng bị vô hiệu hóa. \newline    3. Use Case không thể tiếp tục. \\
\hline
\multicolumn{2}{|c|}{\textbf{2.3. Thông tin bổ sung (Additional Information)}} \\
\hline
\textbf{Mục} & \textbf{Nội dung} \\
\hline
Business Rule & - \textbf{BR-UC1.9-1:} Thời gian kết thúc không sẵn sàng phải sau thời gian bắt đầu không sẵn sàng. \newline - \textbf{BR-UC1.9-2:} Việc đánh dấu không sẵn sàng của nhân viên là một thông tin để Quản lý tham khảo khi xếp lịch. Quản lý vẫn có thể ghi đè và xếp lịch nếu cần thiết (tùy thuộc vào chính sách công ty và cấu hình hệ thống - cần xác nhận). Tuy nhiên, hệ thống nên ưu tiên không đề xuất nhân viên trong thời gian họ báo bận (như trong UC-MD01-02). \newline - \textbf{BR-UC1.9-3:} Nhân viên chỉ có thể đánh dấu không sẵn sàng cho chính mình. \newline - \textbf{BR-UC1.9-4:} (Tùy chọn cấu hình) Có thể có giới hạn về việc báo không sẵn sàng quá gần ngày làm việc (ví dụ: phải báo trước 24h). \\
\hline
Non-Functional Requirement & - \textbf{NFR-UC1.9-1 (Usability):} Giao diện nhập liệu phải dễ sử dụng, đặc biệt là việc chọn ngày giờ (nên dùng calendar/time picker). \newline - \textbf{NFR-UC1.9-2 (Performance):} Thời gian lưu thông tin không sẵn sàng phải dưới 2 giây. \newline - \textbf{NFR-UC1.9-3 (Security):} Đảm bảo chỉ nhân viên mới có thể đánh dấu không sẵn sàng cho tài khoản của họ. \newline - \textbf{NFR-UC1.9-4 (Integration):} Dữ liệu không sẵn sàng phải được module Planning (cụ thể là chức năng gán ca UC-MD01-02 và hiển thị lịch UC-MD01-03) sử dụng để đưa ra đề xuất/cảnh báo phù hợp. \\
\hline

\end{longtable}

\subsubsection{Use Case UC-MD01-10: Xem lịch theo vai trò}

Use case này về cơ bản đã được mô tả như một luồng thay thế (Alternative Flow 3b) trong \textbf{UC-MD01-03: Xem lịch biểu Gantt}.

\subsection{Module MD-02: Quản lý Thực đơn \& Sản phẩm}

\subsubsection{Use Case UC-MD02-01: Tạo mới Sản phẩm}

\begin{longtable}{|m{4cm}|p{11cm}|}
\caption{Đặc tả Use Case UC-MD02-01: Tạo mới Sản phẩm} \label{tab:uc_md02_01_revised} \\
\hline
\multicolumn{2}{|c|}{\textbf{2.1. Tóm tắt (Summary)}} \\
\hline
\textbf{Mục} & \textbf{Nội dung} \\
\hline
\endhead % Header cho các trang tiếp theo
\hline
\endfoot % Footer cho bảng
\hline
\endlastfoot % Footer cho trang cuối cùng
Use Case Name & Tạo mới Sản phẩm \\
\hline
Use Case ID & UC-MD02-01 \\
\hline
Use Case Description & Cho phép Quản lý nhà hàng (US-01) thêm một món ăn, đồ uống, hoặc dịch vụ mới vào hệ thống với các chi tiết cần thiết ban đầu như tên, giá bán, và loại sản phẩm. \\
\hline
Actor & US-01 (Quản lý nhà hàng) \\
\hline
Priority & Must Have \\
\hline
Trigger & Nhà hàng cần bổ sung một món mới vào thực đơn, hoặc quản lý một dịch vụ/mặt hàng mới. \\
\hline
Pre-Condition & - US-01 đã đăng nhập vào hệ thống với quyền quản lý sản phẩm. \\
\hline
Post-Condition & - Một bản ghi Sản phẩm Gốc (Product Template) mới được tạo và lưu thành công trong cơ sở dữ liệu. \newline - Sản phẩm này sẵn sàng cho các cấu hình chi tiết hơn (biến thể, hiển thị POS, hình ảnh...). \newline - Hệ thống ghi nhận hoạt động. \\
\hline
\multicolumn{2}{|c|}{\textbf{2.2. Luồng thực thi (Flow)}} \\
\hline
\textbf{Mục} & \textbf{Nội dung} \\
\hline
Basic Flow & 1. US-01 truy cập vào chức năng quản lý Sản phẩm (ví dụ: Inventory > Products > Products hoặc Sales > Products > Products). \newline 2. US-01 chọn hành động "Tạo mới" (Create). \newline 3. Hệ thống hiển thị một form sản phẩm trống. \newline 4. US-01 nhập Tên sản phẩm (Product Name) (Bắt buộc - BR-UC2.1-1). \newline 5. US-01 chọn Loại sản phẩm (Product Type) từ danh sách thả xuống (Consumable, Stockable, Service) (Bắt buộc - BR-UC2.1-3). Đối với hầu hết món ăn/đồ uống nhà hàng không cần theo dõi tồn kho chi tiết, chọn "Consumable". \newline 6. US-01 nhập Giá bán (Sales Price) (Bắt buộc - BR-UC2.1-2). \newline 7. (Tùy chọn) US-01 nhập Giá vốn (Cost). \newline 8. (Tùy chọn) US-01 chọn Danh mục sản phẩm nội bộ (Internal Product Category). \newline 9. (Tùy chọn) US-01 nhập Mã nội bộ (Internal Reference). \newline 10. US-01 đảm bảo các tùy chọn mặc định như "Có thể Bán" (Can be Sold) và "Có thể Mua" (Can be Purchased - nếu áp dụng) được chọn đúng. \newline 11. US-01 chọn hành động "Lưu" (Save). \newline 12. Hệ thống kiểm tra các trường bắt buộc đã được điền và dữ liệu hợp lệ. \newline 13. Hệ thống lưu bản ghi sản phẩm mới. \newline 14. Hệ thống hiển thị form sản phẩm ở chế độ xem với thông tin vừa lưu và thông báo "Sản phẩm đã được tạo". \\
\hline
Alternative Flow & \textbf{11a. Lưu và Tạo mới (Save \& New):} \newline    1. US-01 chọn "Lưu và Tạo mới". \newline    2. Hệ thống thực hiện bước 12, 13. \newline    3. Hệ thống hiển thị lại form sản phẩm trống để tiếp tục nhập. \\
\hline
Exception Flow & \textbf{12a. Lỗi Xác thực Dữ liệu:} \newline    1. Hệ thống phát hiện thiếu Tên sản phẩm, Giá bán, hoặc Loại SP; hoặc Giá bán không phải số. \newline    2. Hệ thống hiển thị thông báo lỗi, chỉ rõ trường lỗi. \newline    3. Hệ thống không lưu, giữ nguyên form để US-01 sửa. Use Case quay lại bước 4. \newline \textbf{13a. Lỗi Hệ thống khi Lưu:} \newline    1. Hệ thống gặp sự cố khi lưu. \newline    2. Hệ thống hiển thị thông báo lỗi chung. Use Case kết thúc không thành công. \\
\hline
\multicolumn{2}{|c|}{\textbf{2.3. Thông tin bổ sung (Additional Information)}} \\
\hline
\textbf{Mục} & \textbf{Nội dung} \\
\hline
Business Rule & - \textbf{BR-UC2.1-1:} Tên sản phẩm là bắt buộc. \newline - \textbf{BR-UC2.1-2:} Giá bán là bắt buộc và phải là số không âm. \newline - \textbf{BR-UC2.1-3:} Loại sản phẩm là bắt buộc. "Consumable" cho món ăn/đồ uống không quản lý tồn kho chi tiết. "Stockable" cho hàng hóa cần theo dõi tồn kho (rượu chai, bia lon). "Service" cho phí dịch vụ. \newline - \textbf{BR-UC2.1-4:} Sản phẩm mới tạo mặc định "Active" và "Can be Sold". \\
\hline
Non-Functional Requirement & - \textbf{NFR-UC2.1-1 (Usability):} Form tạo sản phẩm rõ ràng, trường bắt buộc dễ nhận biết. \newline - \textbf{NFR-UC2.1-2 (Performance):} Lưu sản phẩm mới dưới 3 giây. \newline - \textbf{NFR-UC2.1-3 (Data Integrity):} Dữ liệu nhập phải được lưu chính xác. \\
\hline
\end{longtable}

\subsubsection{Use Case UC-MD02-02: Xem Danh sách Sản phẩm}

\begin{longtable}{|m{4cm}|p{11cm}|}
\caption{Đặc tả Use Case UC-MD02-02: Xem Danh sách Sản phẩm} \label{tab:uc_md02_02_revised} \\
\hline
\multicolumn{2}{|c|}{\textbf{2.1. Tóm tắt (Summary)}} \\
\hline
\textbf{Mục} & \textbf{Nội dung} \\
\hline
\endhead % Header cho các trang tiếp theo
\hline
\endfoot % Footer cho bảng
\hline
\endlastfoot % Footer cho trang cuối cùng
Use Case Name & Xem Danh sách Sản phẩm \\
\hline
Use Case ID & UC-MD02-02 \\
\hline
Use Case Description & Cho phép Quản lý nhà hàng (US-01) xem danh sách tất cả các sản phẩm (món ăn, đồ uống, dịch vụ) đã được tạo trong hệ thống, với các thông tin tóm tắt. \\
\hline
Actor & US-01 (Quản lý nhà hàng) \\
\hline
Priority & Must Have \\
\hline
Trigger & Quản lý nhà hàng cần kiểm tra các sản phẩm hiện có, tìm kiếm một sản phẩm cụ thể, hoặc chuẩn bị cho các thao tác quản lý khác (sửa, xóa, lưu trữ). \\
\hline
Pre-Condition & - US-01 đã đăng nhập vào hệ thống với quyền xem sản phẩm. \\
\hline
Post-Condition & - Danh sách các sản phẩm (mặc định là các sản phẩm đang hoạt động - Active) được hiển thị. \newline - US-01 có thể xem các thông tin cơ bản của từng sản phẩm và thực hiện các hành động tiếp theo. \\
\hline
\multicolumn{2}{|c|}{\textbf{2.2. Luồng thực thi (Flow)}} \\
\hline
\textbf{Mục} & \textbf{Nội dung} \\
\hline
Basic Flow & 1. US-01 truy cập vào chức năng quản lý Sản phẩm (ví dụ: Inventory > Products > Products). \newline 2. Hệ thống truy vấn và hiển thị danh sách các sản phẩm (Product Templates) đang hoạt động (Active=True). \newline 3. Với mỗi sản phẩm trong danh sách (List View), hệ thống hiển thị các thông tin cơ bản như: Hình ảnh (nếu có), Tên sản phẩm, Mã nội bộ (nếu có), Giá bán, Giá vốn (nếu có), Số lượng tồn kho dự báo (nếu là Stockable). \newline 4. US-01 xem xét danh sách. \\
\hline
Alternative Flow & \textbf{4a. Tìm kiếm sản phẩm:} \newline    1. US-01 sử dụng ô tìm kiếm, nhập tên, mã nội bộ hoặc một phần thông tin sản phẩm. \newline    2. Hệ thống lọc và hiển thị các sản phẩm khớp với từ khóa. \newline \textbf{4b. Lọc sản phẩm:} \newline    1. US-01 sử dụng các bộ lọc có sẵn (Filters) như "Có thể Bán", "Có thể Mua", "Loại sản phẩm (Consumable, Stockable, Service)", "Danh mục sản phẩm nội bộ", "Đã lưu trữ (Archived)". \newline    2. Hệ thống áp dụng bộ lọc và hiển thị kết quả. \newline \textbf{4c. Nhóm sản phẩm (Group By):} \newline    1. US-01 sử dụng chức năng nhóm (ví dụ: nhóm theo Loại sản phẩm, theo Danh mục sản phẩm nội bộ). \newline    2. Hệ thống hiển thị danh sách được nhóm lại. \newline \textbf{4d. Chuyển đổi dạng xem:} \newline    1. US-01 có thể chuyển sang dạng xem Kanban (thẻ sản phẩm với hình ảnh) hoặc các dạng xem khác nếu có. \\
\hline
Exception Flow & \textbf{2a. Lỗi tải danh sách:} \newline    1. Hệ thống gặp lỗi khi truy vấn dữ liệu sản phẩm. \newline    2. Hệ thống hiển thị thông báo lỗi. \newline \textbf{2b. Không có sản phẩm nào:} \newline    1. Nếu chưa có sản phẩm nào được tạo (hoặc không có sản phẩm nào khớp bộ lọc). \newline    2. Hệ thống hiển thị danh sách trống hoặc thông báo "Chưa có sản phẩm nào." \\
\hline
\multicolumn{2}{|c|}{\textbf{2.3. Thông tin bổ sung (Additional Information)}} \\
\hline
\textbf{Mục} & \textbf{Nội dung} \\
\hline
Business Rule & - \textbf{BR-UC2.2-1:} Mặc định, danh sách chỉ hiển thị các sản phẩm đang hoạt động (Active=True). Người dùng cần bỏ bộ lọc "Archived=False" để xem các sản phẩm đã lưu trữ. \\
\hline
Non-Functional Requirement & - \textbf{NFR-UC2.2-1 (Usability):} Giao diện danh sách phải rõ ràng, dễ đọc. Chức năng tìm kiếm, lọc, nhóm phải hiệu quả. \newline - \textbf{NFR-UC2.2-2 (Performance):} Thời gian tải danh sách sản phẩm (ngay cả với hàng trăm sản phẩm) phải nhanh chóng. \\
\hline
\end{longtable}

\subsubsection{Use Case UC-MD02-03: Xem Chi tiết Sản phẩm}

\begin{longtable}{|m{4cm}|p{11cm}|}
\caption{Đặc tả Use Case UC-MD02-03: Xem Chi tiết Sản phẩm} \label{tab:uc_md02_03_revised} \\
\hline
\multicolumn{2}{|c|}{\textbf{2.1. Tóm tắt (Summary)}} \\
\hline
\textbf{Mục} & \textbf{Nội dung} \\
\hline
\endhead % Header cho các trang tiếp theo
\hline
\endfoot % Footer cho bảng
\hline
\endlastfoot % Footer cho trang cuối cùng
Use Case Name & Xem Chi tiết Sản phẩm \\
\hline
Use Case ID & UC-MD02-03 \\
\hline
Use Case Description & Cho phép Quản lý nhà hàng (US-01) xem thông tin chi tiết đầy đủ của một sản phẩm cụ thể đã được chọn từ danh sách, bao gồm tất cả các tab thông tin (Chung, Biến thể, Bán hàng, Mua hàng, Tồn kho, Kế toán, POS...). \\
\hline
Actor & US-01 (Quản lý nhà hàng) \\
\hline
Priority & Must Have \\
\hline
Trigger & Quản lý nhà hàng nhấp vào một sản phẩm từ danh sách (UC-MD02-02) để xem hoặc chuẩn bị sửa đổi thông tin. \\
\hline
Pre-Condition & - US-01 đang xem danh sách sản phẩm (UC-MD02-02 thành công). \newline - US-01 có quyền xem chi tiết sản phẩm. \\
\hline
Post-Condition & - Form chi tiết (Product Template Form View) của sản phẩm được chọn được hiển thị. \newline - US-01 nắm được mọi thông tin đã cấu hình cho sản phẩm đó. \\
\hline
\multicolumn{2}{|c|}{\textbf{2.2. Luồng thực thi (Flow)}} \\
\hline
\textbf{Mục} & \textbf{Nội dung} \\
\hline
Basic Flow & 1. US-01 đang xem danh sách sản phẩm (UC-MD02-02). \newline 2. US-01 nhấp vào tên hoặc một vùng có thể nhấp được của dòng sản phẩm muốn xem chi tiết. \newline 3. Hệ thống truy xuất toàn bộ thông tin của Sản phẩm Gốc (Product Template) và các Biến thể (Product Variants) liên quan (nếu có). \newline 4. Hệ thống hiển thị Form chi tiết sản phẩm, thường được tổ chức thành nhiều tab: \newline    - \textbf{General Information (Thông tin chung):} Tên, Loại sản phẩm, Danh mục nội bộ, Mã nội bộ, Giá bán, Giá vốn, Thuế, Đơn vị tính... \newline    - \textbf{Variants (Biến thể):} Danh sách các Thuộc tính và Giá trị đã gán, nút để xem/quản lý các biến thể cụ thể. \newline    - \textbf{Sales (Bán hàng):} Cấu hình cho đơn bán hàng, chính sách hóa đơn, mô tả bán hàng. \newline    - \textbf{Point of Sale (Điểm bán hàng):} Cấu hình hiển thị trên POS (Available in POS, POS Category). \newline    - \textbf{Purchase (Mua hàng):} Cấu hình cho đơn mua hàng (nếu sản phẩm có thể mua). \newline    - \textbf{Inventory (Tồn kho):} Cấu hình liên quan đến quản lý kho (Lộ trình, Quy tắc Tồn kho, Trọng lượng, Thể tích...). \newline    - \textbf{Accounting (Kế toán):} Cấu hình tài khoản doanh thu, chi phí. \newline 5. US-01 xem xét các thông tin chi tiết. \\
\hline
Alternative Flow & \textbf{5a. Nhấp vào các nút thông minh (Smart Buttons):} \newline    1. Trên form sản phẩm, có các nút thông minh hiển thị số lượng liên quan (ví dụ: "X Variants", "Y On Hand", "Z Sold"). \newline    2. US-01 nhấp vào một nút thông minh để xem danh sách các bản ghi liên quan (ví dụ: danh sách biến thể, bút toán kho, đơn hàng đã bán). \\
\hline
Exception Flow & \textbf{3a. Lỗi tải chi tiết sản phẩm:} \newline    1. Hệ thống gặp lỗi khi lấy thông tin chi tiết của sản phẩm. \newline    2. Hệ thống báo lỗi. \newline \textbf{3b. Sản phẩm không tồn tại/không có quyền xem:} \newline    1. Do lỗi đồng bộ hoặc vấn đề phân quyền. \newline    2. Hệ thống báo lỗi "Không tìm thấy sản phẩm" hoặc "Không có quyền truy cập". \\
\hline
\multicolumn{2}{|c|}{\textbf{2.3. Thông tin bổ sung (Additional Information)}} \\
\hline
\textbf{Mục} & \textbf{Nội dung} \\
\hline
Business Rule & - \textbf{BR-UC2.3-1:} Form chi tiết phải hiển thị tất cả các trường thông tin đã được cấu hình cho sản phẩm một cách đầy đủ và chính xác. \\
\hline
Non-Functional Requirement & - \textbf{NFR-UC2.3-1 (Usability):} Thông tin trên form phải được tổ chức logic theo các tab, dễ tìm kiếm. \newline - \textbf{NFR-UC2.3-2 (Performance):} Thời gian tải form chi tiết sản phẩm (kể cả sản phẩm có nhiều biến thể) phải nhanh. \\
\hline
\end{longtable}

\subsubsection{Use Case UC-MD02-04: Sửa Thông tin Cơ bản của Sản phẩm}

\begin{longtable}{|m{4cm}|p{11cm}|}
\caption{Đặc tả Use Case UC-MD02-04: Sửa Thông tin Cơ bản của Sản phẩm} \label{tab:uc_md02_04_revised} \\
\hline
\multicolumn{2}{|c|}{\textbf{2.1. Tóm tắt (Summary)}} \\
\hline
\textbf{Mục} & \textbf{Nội dung} \\
\hline
\endhead % Header cho các trang tiếp theo
\hline
\endfoot % Footer cho bảng
\hline
\endlastfoot % Footer cho trang cuối cùng
Use Case Name & Sửa Thông tin Cơ bản của Sản phẩm \\
\hline
Use Case ID & UC-MD02-04 \\
\hline
Use Case Description & Cho phép Quản lý nhà hàng (US-01) cập nhật các thông tin chung và cơ bản của một sản phẩm đã tồn tại, như Tên sản phẩm, Loại sản phẩm, Giá bán, Giá vốn, Mã nội bộ, Danh mục sản phẩm nội bộ, Đơn vị tính, và các tùy chọn "Có thể Bán"/"Có thể Mua". \\
\hline
Actor & US-01 (Quản lý nhà hàng) \\
\hline
Priority & Must Have \\
\hline
Trigger & Thông tin cơ bản của một sản phẩm cần được cập nhật (ví dụ: đổi tên, thay đổi giá, phân loại lại). \\
\hline
Pre-Condition & - US-01 đã đăng nhập với quyền quản lý sản phẩm. \newline - Sản phẩm cần sửa đã tồn tại và US-01 đang xem form chi tiết của sản phẩm đó (UC-MD02-03). \\
\hline
Post-Condition & - Các thông tin cơ bản của sản phẩm được cập nhật thành công trong cơ sở dữ liệu. \newline - Các thay đổi (ví dụ: giá bán, tên) sẽ được phản ánh trên các giao diện liên quan. \\
\hline
\multicolumn{2}{|c|}{\textbf{2.2. Luồng thực thi (Flow)}} \\
\hline
\textbf{Mục} & \textbf{Nội dung} \\
\hline
Basic Flow & 1. US-01 đang xem form chi tiết sản phẩm (UC-MD02-03). \newline 2. US-01 chọn hành động "Sửa" (Edit). \newline 3. Hệ thống cho phép chỉnh sửa các trường trên tab "General Information" (hoặc tương đương). \newline 4. US-01 thực hiện các thay đổi mong muốn: \newline    - Sửa Tên sản phẩm (Product Name). \newline    - Thay đổi Loại sản phẩm (Product Type) (Lưu ý BR-UC2.4-2). \newline    - Cập nhật Giá bán (Sales Price), Giá vốn (Cost). \newline    - Sửa Danh mục sản phẩm nội bộ (Internal Product Category). \newline    - Sửa Mã nội bộ (Internal Reference). \newline    - Thay đổi Đơn vị tính (Unit of Measure), Đơn vị tính mua hàng (Purchase UoM). \newline    - Tick/Bỏ tick các ô "Can be Sold", "Can be Purchased". \newline    - (Tùy chọn) Sửa Mô tả nội bộ (Internal Notes). \newline 5. US-01 chọn hành động "Lưu" (Save). \newline 6. Hệ thống kiểm tra tính hợp lệ của các dữ liệu đã thay đổi (Tên không trống, Giá là số...). \newline 7. Hệ thống lưu các thay đổi vào bản ghi sản phẩm. \newline 8. Hệ thống chuyển form về chế độ xem với thông tin đã cập nhật và báo thành công. \\
\hline
Alternative Flow & \textbf{4a. Sửa các thông tin ở tab khác:} \newline    1. US-01 có thể chuyển sang các tab khác (Sales, Purchase, Inventory, Accounting) để sửa các thông tin chuyên sâu hơn liên quan đến các module đó (ví dụ: chính sách hóa đơn, lộ trình tồn kho, tài khoản kế toán). Các Use Case này có thể được coi là mở rộng hoặc thuộc về các module chuyên biệt đó. \\
\hline
Exception Flow & \textbf{6a. Lỗi Xác thực Dữ liệu:} \newline    1. Hệ thống phát hiện lỗi (ví dụ: Giá bán không phải số, Tên trống). \newline    2. Hệ thống báo lỗi, không cho lưu, giữ nguyên form để sửa. \newline \textbf{7a. Lỗi Hệ thống khi Cập nhật:} \newline    1. Hệ thống gặp lỗi khi lưu. \newline    2. Hệ thống báo lỗi chung. \\
\hline
\multicolumn{2}{|c|}{\textbf{2.3. Thông tin bổ sung (Additional Information)}} \\
\hline
\textbf{Mục} & \textbf{Nội dung} \\
\hline
Business Rule & - \textbf{BR-UC2.4-1:} Tên sản phẩm và Giá bán sau khi sửa không được để trống và phải hợp lệ. \newline - \textbf{BR-UC2.4-2:} Việc thay đổi Loại sản phẩm (đặc biệt từ/sang Stockable) của sản phẩm đã có giao dịch tồn kho có thể bị hạn chế hoặc yêu cầu các bước xử lý bổ sung (tham khảo BR-UC2.7-4 của UC tạo mới). \\
\hline
Non-Functional Requirement & - \textbf{NFR-UC2.4-1 (Usability):} Form sửa sản phẩm phải dễ dàng định vị và thay đổi các trường thông tin. \newline - \textbf{NFR-UC2.4-2 (Performance):} Thời gian lưu thay đổi phải nhanh. \\
\hline
\end{longtable}

\subsubsection{Use Case UC-MD02-05: Lưu trữ Sản phẩm}

\begin{longtable}{|m{4cm}|p{11cm}|}
\caption{Đặc tả Use Case UC-MD02-05: Lưu trữ Sản phẩm} \label{tab:uc_md02_05_revised} \\
\hline
\multicolumn{2}{|c|}{\textbf{2.1. Tóm tắt (Summary)}} \\
\hline
\textbf{Mục} & \textbf{Nội dung} \\
\hline
\endhead % Header cho các trang tiếp theo
\hline
\endfoot % Footer cho bảng
\hline
\endlastfoot % Footer cho trang cuối cùng
Use Case Name & Lưu trữ Sản phẩm \\
\hline
Use Case ID & UC-MD02-05 \\
\hline
Use Case Description & Cho phép Quản lý nhà hàng (US-01) tạm thời hoặc vĩnh viễn ẩn một sản phẩm khỏi các giao diện hoạt động (POS, danh sách chọn sản phẩm bán hàng...) bằng cách đặt sản phẩm đó vào trạng thái "Lưu trữ" (Archived). Dữ liệu lịch sử của sản phẩm vẫn được giữ lại. \\
\hline
Actor & US-01 (Quản lý nhà hàng) \\
\hline
Priority & Should Have \\
\hline
Trigger & Một sản phẩm không còn được bán/sử dụng nữa (ví dụ: món theo mùa đã hết, ngừng kinh doanh mặt hàng). \\
\hline
Pre-Condition & - US-01 đã đăng nhập với quyền quản lý sản phẩm. \newline - Sản phẩm cần lưu trữ đang ở trạng thái hoạt động (Active=True). \\
\hline
Post-Condition & - Trường 'Active' của sản phẩm được đặt thành False. \newline - Sản phẩm không còn hiển thị trong danh sách sản phẩm mặc định và không thể chọn trong các giao dịch mới. \newline - Dữ liệu lịch sử của sản phẩm không bị ảnh hưởng. \\
\hline
\multicolumn{2}{|c|}{\textbf{2.2. Luồng thực thi (Flow)}} \\
\hline
\textbf{Mục} & \textbf{Nội dung} \\
\hline
Basic Flow (Từ Form View) & 1. US-01 đang xem form chi tiết của sản phẩm muốn lưu trữ (UC-MD02-03). \newline 2. US-01 chọn menu "Hành động" (Action). \newline 3. US-01 chọn tùy chọn "Lưu trữ" (Archive). \newline 4. Hệ thống (có thể) hiển thị hộp thoại xác nhận. US-01 xác nhận. \newline 5. Hệ thống cập nhật trường `active` của sản phẩm thành `False`. \newline 6. Hệ thống có thể tự động điều hướng người dùng quay lại danh sách sản phẩm (nơi sản phẩm vừa lưu trữ sẽ không còn hiển thị theo bộ lọc mặc định). \newline 7. Hệ thống hiển thị thông báo "Sản phẩm đã được lưu trữ." \\
\hline
Alternative Flow & \textbf{Basic Flow (Từ List View):} \newline    1. US-01 đang xem danh sách sản phẩm (UC-MD02-02). \newline    2. US-01 chọn (tick vào ô vuông) một hoặc nhiều sản phẩm muốn lưu trữ. \newline    3. US-01 chọn menu "Hành động" (Action) của danh sách. \newline    4. US-01 chọn tùy chọn "Lưu trữ" (Archive). \newline    5. Hệ thống (có thể) yêu cầu xác nhận. US-01 xác nhận. \newline    6. Hệ thống cập nhật `active = False` cho tất cả các sản phẩm đã chọn. \newline    7. Hệ thống làm mới danh sách, các sản phẩm vừa lưu trữ biến mất. \newline    8. Hệ thống báo thành công. \\
\hline
Exception Flow & \textbf{5a. Lỗi hệ thống khi cập nhật trạng thái:} \newline    1. Hệ thống gặp lỗi khi cố gắng cập nhật trường `active`. \newline    2. Hệ thống báo lỗi chung. Trạng thái sản phẩm có thể không thay đổi. \\
\hline
\multicolumn{2}{|c|}{\textbf{2.3. Thông tin bổ sung (Additional Information)}} \\
\hline
\textbf{Mục} & \textbf{Nội dung} \\
\hline
Business Rule & - \textbf{BR-UC2.5-1:} Sản phẩm bị Lưu trữ sẽ không xuất hiện trong các lựa chọn sản phẩm mặc định trên POS, đơn bán hàng, v.v. \newline - \textbf{BR-UC2.5-2:} Lưu trữ không xóa dữ liệu lịch sử. \\
\hline
Non-Functional Requirement & - \textbf{NFR-UC2.5-1 (Usability):} Hành động Lưu trữ phải dễ dàng truy cập. \newline - \textbf{NFR-UC2.5-2 (Performance):} Cập nhật trạng thái phải nhanh. \\
\hline
\end{longtable}

\subsubsection{Use Case UC-MD02-06: Hủy Lưu trữ Sản phẩm}

\begin{longtable}{|m{4cm}|p{11cm}|}
\caption{Đặc tả Use Case UC-MD02-06: Hủy Lưu trữ Sản phẩm} \label{tab:uc_md02_06_revised} \\
\hline
\multicolumn{2}{|c|}{\textbf{2.1. Tóm tắt (Summary)}} \\
\hline
\textbf{Mục} & \textbf{Nội dung} \\
\hline
\endhead % Header cho các trang tiếp theo
\hline
\endfoot % Footer cho bảng
\hline
\endlastfoot % Footer cho trang cuối cùng
Use Case Name & Hủy Lưu trữ Sản phẩm \\
\hline
Use Case ID & UC-MD02-06 \\
\hline
Use Case Description & Cho phép Quản lý nhà hàng (US-01) kích hoạt lại một sản phẩm đã bị đặt vào trạng thái "Lưu trữ" (Archived), làm cho sản phẩm đó hoạt động trở lại và có thể được sử dụng trong các giao dịch. \\
\hline
Actor & US-01 (Quản lý nhà hàng) \\
\hline
Priority & Should Have \\
\hline
Trigger & Cần bán lại hoặc sử dụng lại một sản phẩm đã từng bị ẩn đi. \\
\hline
Pre-Condition & - US-01 đã đăng nhập với quyền quản lý sản phẩm. \newline - Sản phẩm cần hủy lưu trữ đang ở trạng thái "Lưu trữ" (Active=False). \\
\hline
Post-Condition & - Trường 'Active' của sản phẩm được đặt thành True. \newline - Sản phẩm xuất hiện trở lại trong danh sách sản phẩm mặc định và có thể được sử dụng trong các giao dịch. \\
\hline
\multicolumn{2}{|c|}{\textbf{2.2. Luồng thực thi (Flow)}} \\
\hline
\textbf{Mục} & \textbf{Nội dung} \\
\hline
Basic Flow (Từ Form View) & 1. US-01 truy cập danh sách sản phẩm và bỏ bộ lọc "Archived=False" (hoặc áp dụng bộ lọc "Archived=True") để tìm sản phẩm đã lưu trữ. \newline 2. US-01 chọn sản phẩm muốn hủy lưu trữ để mở form chi tiết. \newline 3. US-01 chọn menu "Hành động" (Action). \newline 4. US-01 chọn tùy chọn "Hủy lưu trữ" (Unarchive). \newline 5. Hệ thống (có thể) yêu cầu xác nhận. US-01 xác nhận. \newline 6. Hệ thống cập nhật trường `active` của sản phẩm thành `True`. \newline 7. Hệ thống hiển thị thông báo "Sản phẩm đã được hủy lưu trữ." Sản phẩm giờ sẽ hiển thị trong danh sách mặc định. \\
\hline
Alternative Flow & \textbf{Basic Flow (Từ List View):} \newline    1. US-01 truy cập danh sách sản phẩm và lọc để hiển thị các sản phẩm đã lưu trữ. \newline    2. US-01 chọn (tick) một hoặc nhiều sản phẩm muốn hủy lưu trữ. \newline    3. US-01 chọn menu "Hành động" (Action) của danh sách. \newline    4. US-01 chọn tùy chọn "Hủy lưu trữ" (Unarchive). \newline    5. Hệ thống (có thể) yêu cầu xác nhận. US-01 xác nhận. \newline    6. Hệ thống cập nhật `active = True` cho các sản phẩm đã chọn. \newline    7. Hệ thống làm mới danh sách. Nếu đang lọc "Archived", các sản phẩm này biến mất. Nếu về bộ lọc mặc định, chúng sẽ xuất hiện. \newline    8. Hệ thống báo thành công. \\
\hline
Exception Flow & \textbf{6a. Lỗi hệ thống khi cập nhật trạng thái:} \newline    1. Hệ thống gặp lỗi khi cố gắng cập nhật trường `active`. \newline    2. Hệ thống báo lỗi chung. \\
\hline
\multicolumn{2}{|c|}{\textbf{2.3. Thông tin bổ sung (Additional Information)}} \\
\hline
\textbf{Mục} & \textbf{Nội dung} \\
\hline
Business Rule & - \textbf{BR-UC2.6-1:} Chỉ những sản phẩm đang ở trạng thái "Lưu trữ" mới có thể được Hủy lưu trữ. \\
\hline
Non-Functional Requirement & - \textbf{NFR-UC2.6-1 (Usability):} Việc tìm và hủy lưu trữ sản phẩm phải dễ dàng. \newline - \textbf{NFR-UC2.6-2 (Performance):} Cập nhật trạng thái phải nhanh. \\
\hline
\end{longtable}

\subsubsection{Use Case UC-MD02-07: Tạo mới Danh mục POS}

\begin{longtable}{|m{4cm}|p{11cm}|}
\caption{Đặc tả Use Case UC-MD02-07: Tạo mới Danh mục POS} \label{tab:uc_md02_07_revised} \\
\hline
\multicolumn{2}{|c|}{\textbf{2.1. Tóm tắt (Summary)}} \\
\hline
\textbf{Mục} & \textbf{Nội dung} \\
\hline
\endhead % Header cho các trang tiếp theo
\hline
\endfoot % Footer cho bảng
\hline
\endlastfoot % Footer cho trang cuối cùng
Use Case Name & Tạo mới Danh mục POS \\
\hline
Use Case ID & UC-MD02-07 \\
\hline
Use Case Description & Cho phép Quản lý nhà hàng (US-01) tạo một danh mục mới (ví dụ: "Khai vị", "Món chính", "Đồ uống đặc biệt") để sử dụng cho việc phân loại và hiển thị sản phẩm trên giao diện Point of Sale (POS). \\
\hline
Actor & US-01 (Quản lý nhà hàng) \\
\hline
Priority & Must Have \\
\hline
Trigger & Cần một nhóm mới để tổ chức các món ăn/đồ uống trên màn hình POS. \\
\hline
Pre-Condition & - US-01 đã đăng nhập với quyền quản trị cấu hình Point of Sale. \\
\hline
Post-Condition & - Một bản ghi Danh mục POS mới được tạo và lưu. \newline - Danh mục mới này sẵn sàng để được gán sản phẩm vào và hiển thị trên POS. \\
\hline
\multicolumn{2}{|c|}{\textbf{2.2. Luồng thực thi (Flow)}} \\
\hline
\textbf{Mục} & \textbf{Nội dung} \\
\hline
Basic Flow & 1. US-01 truy cập vào phần cấu hình của Point of Sale. \newline 2. US-01 chọn mục quản lý "Danh mục Sản phẩm POS" (POS Product Categories). \newline 3. Hệ thống hiển thị danh sách các danh mục POS hiện có (UC-MD02-08). \newline 4. US-01 chọn hành động "Tạo mới" (Create). \newline 5. Hệ thống hiển thị form để nhập thông tin danh mục mới. \newline 6. US-01 nhập Tên Danh mục (Category Name) (bắt buộc - BR-UC2.7-1). \newline 7. (Tùy chọn) US-01 chọn Danh mục Cha (Parent Category) nếu muốn tạo cấu trúc phân cấp. \newline 8. (Tùy chọn) US-01 nhập Số thứ tự (Sequence) để kiểm soát vị trí hiển thị. \newline 9. (Tùy chọn) US-01 chọn hình ảnh đại diện cho danh mục (nếu POS theme hỗ trợ). \newline 10. US-01 chọn lệnh "Lưu" (Save). \newline 11. Hệ thống kiểm tra Tên Danh mục không trống. \newline 12. Hệ thống lưu bản ghi danh mục POS mới. \newline 13. Hệ thống cập nhật danh sách, hiển thị danh mục mới. \newline 14. Hệ thống hiển thị thông báo tạo thành công. \\
\hline
Alternative Flow & Tương tự UC-MD01-01 (Lưu và Tạo mới). \\
\hline
Exception Flow & \textbf{11a. Lỗi Xác thực Dữ liệu:} Tên danh mục trống. \newline \textbf{12a. Lỗi Hệ thống khi Lưu:} Hệ thống gặp sự cố khi lưu. \\
\hline
\multicolumn{2}{|c|}{\textbf{2.3. Thông tin bổ sung (Additional Information)}} \\
\hline
\textbf{Mục} & \textbf{Nội dung} \\
\hline
Business Rule & - \textbf{BR-UC2.7-1:} Tên Danh mục POS là bắt buộc. \newline - \textbf{BR-UC2.7-2:} Hỗ trợ cấu trúc danh mục cha-con. \\
\hline
Non-Functional Requirement & - \textbf{NFR-UC2.7-1 (Usability):} Giao diện tạo danh mục POS đơn giản. \newline - \textbf{NFR-UC2.7-2 (Performance):} Lưu danh mục mới nhanh chóng. \\
\hline
\end{longtable}

\subsubsection{Use Case UC-MD02-08: Xem Danh sách Danh mục POS}

\begin{longtable}{|m{4cm}|p{11cm}|}
\caption{Đặc tả Use Case UC-MD02-08: Xem Danh sách Danh mục POS} \label{tab:uc_md02_08_revised} \\
\hline
\multicolumn{2}{|c|}{\textbf{2.1. Tóm tắt (Summary)}} \\
\hline
\textbf{Mục} & \textbf{Nội dung} \\
\hline
\endhead % Header cho các trang tiếp theo
\hline
\endfoot % Footer cho bảng
\hline
\endlastfoot % Footer cho trang cuối cùng
Use Case Name & Xem Danh sách Danh mục POS \\
\hline
Use Case ID & UC-MD02-08 \\
\hline
Use Case Description & Cho phép Quản lý nhà hàng (US-01) xem danh sách tất cả các Danh mục Sản phẩm POS đã được tạo trong hệ thống, bao gồm tên, thứ tự, và cấu trúc cha-con (nếu có). \\
\hline
Actor & US-01 (Quản lý nhà hàng) \\
\hline
Priority & Must Have \\
\hline
Trigger & Cần kiểm tra các danh mục POS hiện có, tìm kiếm hoặc chuẩn bị cho việc tạo/sửa/xóa/sắp xếp. \\
\hline
Pre-Condition & - US-01 đã đăng nhập với quyền quản trị cấu hình Point of Sale. \\
\hline
Post-Condition & - Danh sách các danh mục POS được hiển thị. \\
\hline
\multicolumn{2}{|c|}{\textbf{2.2. Luồng thực thi (Flow)}} \\
\hline
\textbf{Mục} & \textbf{Nội dung} \\
\hline
Basic Flow & 1. US-01 truy cập vào phần cấu hình của Point of Sale > "Danh mục Sản phẩm POS". \newline 2. Hệ thống hiển thị danh sách các danh mục POS đã tạo, thường theo thứ tự và cấu trúc phân cấp. \newline 3. Thông tin hiển thị: Tên Danh mục, Danh mục Cha (nếu có), Số thứ tự. \newline 4. US-01 xem xét danh sách. \\
\hline
Alternative Flow & \textbf{4a. Tìm kiếm/Lọc/Sắp xếp:} Tương tự như UC-MD01-02. \\
\hline
Exception Flow & \textbf{2a. Lỗi tải danh sách.} \newline \textbf{2b. Không có danh mục nào.} \\
\hline
\multicolumn{2}{|c|}{\textbf{2.3. Thông tin bổ sung (Additional Information)}} \\
\hline
\textbf{Mục} & \textbf{Nội dung} \\
\hline
Business Rule & - \textbf{BR-UC2.8-1:} Danh sách phải hiển thị chính xác các danh mục POS. \\
\hline
Non-Functional Requirement & - \textbf{NFR-UC2.8-1 (Usability):} Danh sách rõ ràng, cấu trúc phân cấp dễ nhìn. \newline - \textbf{NFR-UC2.8-2 (Performance):} Tải danh sách nhanh. \\
\hline
\end{longtable}

\subsubsection{Use Case UC-MD02-09: Sửa Danh mục POS}

\begin{longtable}{|m{4cm}|p{11cm}|}
\caption{Đặc tả Use Case UC-MD02-09: Sửa Danh mục POS} \label{tab:uc_md02_09_revised} \\
\hline
\multicolumn{2}{|c|}{\textbf{2.1. Tóm tắt (Summary)}} \\
\hline
\textbf{Mục} & \textbf{Nội dung} \\
\hline
\endhead % Header cho các trang tiếp theo
\hline
\endfoot % Footer cho bảng
\hline
\endlastfoot % Footer cho trang cuối cùng
Use Case Name & Sửa Danh mục POS \\
\hline
Use Case ID & UC-MD02-09 \\
\hline
Use Case Description & Cho phép Quản lý nhà hàng (US-01) chỉnh sửa thông tin của một Danh mục Sản phẩm POS đã tồn tại, như Tên, Danh mục Cha, Số thứ tự, hoặc Hình ảnh. \\
\hline
Actor & US-01 (Quản lý nhà hàng) \\
\hline
Priority & Must Have \\
\hline
Trigger & Cần thay đổi thông tin hoặc cấu trúc của một danh mục POS. \\
\hline
Pre-Condition & - US-01 đã đăng nhập với quyền quản trị cấu hình Point of Sale. \newline - Danh mục POS cần sửa đã tồn tại. \\
\hline
Post-Condition & - Thông tin của danh mục POS được cập nhật. \newline - Thay đổi sẽ ảnh hưởng đến cách hiển thị trên POS. \\
\hline
\multicolumn{2}{|c|}{\textbf{2.2. Luồng thực thi (Flow)}} \\
\hline
\textbf{Mục} & \textbf{Nội dung} \\
\hline
Basic Flow & 1. US-01 truy cập danh sách Danh mục POS (UC-MD02-08). \newline 2. US-01 chọn danh mục muốn sửa. \newline 3. Hệ thống hiển thị form chi tiết danh mục. \newline 4. US-01 chọn "Sửa". \newline 5. US-01 chỉnh sửa các thông tin (Tên, Cha, Thứ tự, Ảnh...). \newline 6. US-01 chọn "Lưu". \newline 7. Hệ thống kiểm tra hợp lệ và lưu thay đổi. \newline 8. Hệ thống báo thành công. \\
\hline
Alternative Flow & Không có. \\
\hline
Exception Flow & \textbf{7a. Lỗi Xác thực/Lưu:} Tương tự UC-MD01-03. \\
\hline
\multicolumn{2}{|c|}{\textbf{2.3. Thông tin bổ sung (Additional Information)}} \\
\hline
\textbf{Mục} & \textbf{Nội dung} \\
\hline
Business Rule & - \textbf{BR-UC2.9-1:} Tên Danh mục POS không được để trống. \newline - \textbf{BR-UC2.9-2:} Thay đổi Danh mục Cha có thể thay đổi cấu trúc phân cấp. \\
\hline
Non-Functional Requirement & - \textbf{NFR-UC2.9-1 (Usability):} Dễ sử dụng. \newline - \textbf{NFR-UC2.9-2 (Performance):} Lưu nhanh. \\
\hline
\end{longtable}

\subsubsection{Use Case UC-MD02-10: Xóa Danh mục POS}

\begin{longtable}{|m{4cm}|p{11cm}|}
\caption{Đặc tả Use Case UC-MD02-10: Xóa Danh mục POS} \label{tab:uc_md02_10_revised} \\
\hline
\multicolumn{2}{|c|}{\textbf{2.1. Tóm tắt (Summary)}} \\
\hline
\textbf{Mục} & \textbf{Nội dung} \\
\hline
\endhead % Header cho các trang tiếp theo
\hline
\endfoot % Footer cho bảng
\hline
\endlastfoot % Footer cho trang cuối cùng
Use Case Name & Xóa Danh mục POS \\
\hline
Use Case ID & UC-MD02-10 \\
\hline
Use Case Description & Cho phép Quản lý nhà hàng (US-01) xóa một Danh mục Sản phẩm POS không còn sử dụng, với điều kiện danh mục đó không chứa sản phẩm nào hoặc không có danh mục con. \\
\hline
Actor & US-01 (Quản lý nhà hàng) \\
\hline
Priority & Should Have \\
\hline
Trigger & Một danh mục POS không còn cần thiết. \\
\hline
Pre-Condition & - US-01 đã đăng nhập với quyền quản trị cấu hình Point of Sale. \newline - Danh mục POS cần xóa đã tồn tại. \\
\hline
Post-Condition & - Nếu xóa thành công, danh mục POS bị xóa. \newline - Nếu không, danh mục vẫn tồn tại và hệ thống báo lỗi. \\
\hline
\multicolumn{2}{|c|}{\textbf{2.2. Luồng thực thi (Flow)}} \\
\hline
\textbf{Mục} & \textbf{Nội dung} \\
\hline
Basic Flow & 1. US-01 truy cập danh sách Danh mục POS (UC-MD02-08). \newline 2. US-01 chọn danh mục muốn xóa. \newline 3. US-01 chọn hành động "Xóa". \newline 4. Hệ thống yêu cầu xác nhận. US-01 xác nhận. \newline 5. Hệ thống kiểm tra điều kiện xóa (BR-UC2.10-1). \newline 6. Nếu thỏa điều kiện, hệ thống xóa danh mục và báo thành công. \newline 7. Nếu không thỏa, hệ thống báo lỗi và giải thích. \\
\hline
Alternative Flow & Xóa từ danh sách (tương tự UC-MD01-04). \\
\hline
Exception Flow & Lỗi hệ thống khi kiểm tra/xóa. \\
\hline
\multicolumn{2}{|c|}{\textbf{2.3. Thông tin bổ sung (Additional Information)}} \\
\hline
\textbf{Mục} & \textbf{Nội dung} \\
\hline
Business Rule & - \textbf{BR-UC2.10-1:} Không thể xóa Danh mục POS nếu đang chứa sản phẩm hoặc có danh mục con. Cần di chuyển/xóa sản phẩm/danh mục con trước. \\
\hline
Non-Functional Requirement & - \textbf{NFR-UC2.10-1 (Data Integrity):} Ràng buộc không cho xóa quan trọng. \newline - \textbf{NFR-UC2.10-2 (Usability):} Thông báo lỗi rõ ràng. \\
\hline
\end{longtable}

\subsubsection{Use Case UC-MD02-11: Sắp xếp Thứ tự Danh mục POS}

\begin{longtable}{|m{4cm}|p{11cm}|}
\caption{Đặc tả Use Case UC-MD02-11: Sắp xếp Thứ tự Danh mục POS} \label{tab:uc_md02_11_revised} \\
\hline
\multicolumn{2}{|c|}{\textbf{2.1. Tóm tắt (Summary)}} \\
\hline
\textbf{Mục} & \textbf{Nội dung} \\
\hline
\endhead % Header cho các trang tiếp theo
\hline
\endfoot % Footer cho bảng
\hline
\endlastfoot % Footer cho trang cuối cùng
Use Case Name & Sắp xếp Thứ tự Danh mục POS \\
\hline
Use Case ID & UC-MD02-11 \\
\hline
Use Case Description & Cho phép Quản lý nhà hàng (US-01) thay đổi thứ tự xuất hiện của các Danh mục Sản phẩm POS trên giao diện chọn món của POS, thường bằng cách sửa giá trị "Sequence" hoặc kéo thả. \\
\hline
Actor & US-01 (Quản lý nhà hàng) \\
\hline
Priority & Should Have \\
\hline
Trigger & Cần thay đổi cách các danh mục được sắp xếp trên màn hình POS để tối ưu hóa cho nhân viên hoặc làm nổi bật danh mục nào đó. \\
\hline
Pre-Condition & - US-01 đã đăng nhập với quyền quản trị cấu hình Point of Sale. \newline - Có ít nhất hai danh mục POS để sắp xếp. \\
\hline
Post-Condition & - Thứ tự mới của các danh mục POS được lưu lại. \newline - Thứ tự này sẽ được phản ánh trên giao diện POS sau khi đồng bộ. \\
\hline
\multicolumn{2}{|c|}{\textbf{2.2. Luồng thực thi (Flow)}} \\
\hline
\textbf{Mục} & \textbf{Nội dung} \\
\hline
Basic Flow (Kéo thả) & 1. US-01 truy cập danh sách Danh mục POS (UC-MD02-08), thường ở dạng Tree View hoặc List View có hỗ trợ kéo thả. \newline 2. US-01 nhấn giữ vào một danh mục và kéo nó đến vị trí mong muốn trong danh sách (so với các danh mục cùng cấp). \newline 3. US-01 thả chuột. \newline 4. Hệ thống tự động cập nhật giá trị "Sequence" cho các danh mục bị ảnh hưởng và lưu lại thứ tự mới. \newline 5. Giao diện danh sách cập nhật theo thứ tự mới. \\
\hline
Alternative Flow & \textbf{1a. Sửa trực tiếp trường Sequence:} \newline    1. US-01 mở form chi tiết của từng danh mục (UC-MD02-09). \newline    2. US-01 sửa giá trị trong trường "Sequence" (số nhỏ hơn hiển thị trước). \newline    3. US-01 lưu lại. \\
\hline
Exception Flow & \textbf{4a. Lỗi hệ thống khi lưu thứ tự:} \newline    1. Hệ thống gặp lỗi khi cập nhật giá trị sequence. \newline    2. Hệ thống báo lỗi. Thứ tự có thể không được lưu đúng. \\
\hline
\multicolumn{2}{|c|}{\textbf{2.3. Thông tin bổ sung (Additional Information)}} \\
\hline
\textbf{Mục} & \textbf{Nội dung} \\
\hline
Business Rule & - \textbf{BR-UC2.11-1:} Thứ tự hiển thị dựa trên trường "Sequence". Các danh mục cùng cấp được sắp xếp theo giá trị này. \\
\hline
Non-Functional Requirement & - \textbf{NFR-UC2.11-1 (Usability):} Chức năng kéo thả (nếu có) rất tiện lợi. Việc sắp xếp phải trực quan. \newline - \textbf{NFR-UC2.11-2 (Performance):} Cập nhật thứ tự phải nhanh. \\
\hline
\end{longtable}

\subsubsection{Use Case UC-MD02-12: Tạo mới Thuộc tính Sản phẩm}

\begin{longtable}{|m{4cm}|p{11cm}|}
\caption{Đặc tả Use Case UC-MD02-12: Tạo mới Thuộc tính Sản phẩm} \label{tab:uc_md02_12_revised} \\
\hline
\multicolumn{2}{|c|}{\textbf{2.1. Tóm tắt (Summary)}} \\
\hline
\textbf{Mục} & \textbf{Nội dung} \\
\hline
\endhead % Header cho các trang tiếp theo
\hline
\endfoot % Footer cho bảng
\hline
\endlastfoot % Footer cho trang cuối cùng
Use Case Name & Tạo mới Thuộc tính Sản phẩm \\
\hline
Use Case ID & UC-MD02-12 \\
\hline
Use Case Description & Cho phép người dùng có quyền (US-01/US-10) định nghĩa một đặc tính chung mới cho sản phẩm mà có thể có nhiều lựa chọn khác nhau (ví dụ: "Kích cỡ", "Màu sắc", "Độ cay", "Loại đế bánh"). Đây là bước đầu để tạo biến thể sản phẩm. \\
\hline
Actor & US-01 (Quản lý nhà hàng), US-10 (Quản trị viên Hệ thống) \\
\hline
Priority & Must Have (Nếu cần biến thể) \\
\hline
Trigger & Cần quản lý các phiên bản khác nhau của sản phẩm dựa trên một đặc tính mới chưa có trong hệ thống. \\
\hline
Pre-Condition & - Người dùng đã đăng nhập với quyền quản trị cấu hình sản phẩm/inventory. \\
\hline
Post-Condition & - Một bản ghi Thuộc tính (Attribute) mới được tạo và lưu. \newline - Thuộc tính này sẵn sàng để được thêm các Giá trị (UC-MD02-13) và sau đó gán vào sản phẩm (UC-MD02-14). \\
\hline
\multicolumn{2}{|c|}{\textbf{2.2. Luồng thực thi (Flow)}} \\
\hline
\textbf{Mục} & \textbf{Nội dung} \\
\hline
Basic Flow & 1. Người dùng truy cập khu vực cấu hình Thuộc tính (ví dụ: Inventory > Configuration > Product Attributes). \newline 2. Hệ thống hiển thị danh sách Thuộc tính đã có. \newline 3. Người dùng chọn "Tạo mới". \newline 4. Hệ thống hiển thị form tạo Thuộc tính. \newline 5. Người dùng nhập Tên Thuộc tính (Attribute Name) (ví dụ: "Kích cỡ Pizza") (Bắt buộc, duy nhất - BR-UC2.12-1). \newline 6. Người dùng chọn Loại hiển thị (Display Type - Radio, Select, Color) để xác định cách thuộc tính hiển thị khi chọn. \newline 7. Người dùng chọn Chế độ tạo Biến thể (Variants Creation Mode - Instantly, Dynamically, Never). "Instantly" thường dùng. \newline 8. Người dùng chọn "Lưu". \newline 9. Hệ thống kiểm tra và lưu Thuộc tính mới. \newline 10. Hệ thống báo thành công. \\
\hline
Alternative Flow & Lưu và Tạo mới. \\
\hline
Exception Flow & \textbf{9a. Lỗi Xác thực/Lưu:} Tên trống/trùng. Lỗi hệ thống. \\
\hline
\multicolumn{2}{|c|}{\textbf{2.3. Thông tin bổ sung (Additional Information)}} \\
\hline
\textbf{Mục} & \textbf{Nội dung} \\
\hline
Business Rule & - \textbf{BR-UC2.12-1:} Tên Thuộc tính phải là duy nhất. \\
\hline
Non-Functional Requirement & - \textbf{NFR-UC2.12-1 (Usability):} Dễ sử dụng. \newline - \textbf{NFR-UC2.12-2 (Performance):} Lưu nhanh. \\
\hline
\end{longtable}

\subsubsection{Use Case UC-MD02-13: Tạo mới Giá trị cho Thuộc tính}

\begin{longtable}{|m{4cm}|p{11cm}|}
\caption{Đặc tả Use Case UC-MD02-13: Tạo mới Giá trị cho Thuộc tính} \label{tab:uc_md02_13_revised} \\
\hline
\multicolumn{2}{|c|}{\textbf{2.1. Tóm tắt (Summary)}} \\
\hline
\textbf{Mục} & \textbf{Nội dung} \\
\hline
\endhead % Header cho các trang tiếp theo
\hline
\endfoot % Footer cho bảng
\hline
\endlastfoot % Footer cho trang cuối cùng
Use Case Name & Tạo mới Giá trị cho Thuộc tính \\
\hline
Use Case ID & UC-MD02-13 \\
\hline
Use Case Description & Sau khi một Thuộc tính đã được tạo (UC-MD02-12), cho phép người dùng có quyền (US-01/US-10) định nghĩa các lựa chọn/giá trị cụ thể cho Thuộc tính đó (ví dụ: cho Thuộc tính "Kích cỡ Pizza", tạo các Giá trị "Nhỏ", "Vừa", "Lớn"). \\
\hline
Actor & US-01 (Quản lý nhà hàng), US-10 (Quản trị viên Hệ thống) \\
\hline
Priority & Must Have (Nếu cần biến thể) \\
\hline
Trigger & Một Thuộc tính đã được tạo và cần định nghĩa các lựa chọn khả dĩ cho nó. \\
\hline
Pre-Condition & - Người dùng đã đăng nhập với quyền quản trị cấu hình sản phẩm/inventory. \newline - Thuộc tính cha đã được tạo (UC-MD02-12). \\
\hline
Post-Condition & - Một hoặc nhiều bản ghi Giá trị Thuộc tính (Attribute Value) mới được tạo và liên kết với Thuộc tính cha. \newline - Các Giá trị này sẵn sàng để được chọn khi gán Thuộc tính vào sản phẩm (UC-MD02-14). \\
\hline
\multicolumn{2}{|c|}{\textbf{2.2. Luồng thực thi (Flow)}} \\
\hline
\textbf{Mục} & \textbf{Nội dung} \\
\hline
Basic Flow & 1. Người dùng truy cập form chi tiết của một Thuộc tính (đã tạo ở UC-MD02-12). \newline 2. Người dùng chọn tab/mục "Giá trị" (Attribute Values). \newline 3. Người dùng chọn "Thêm một dòng" (Add a line) hoặc "Tạo mới". \newline 4. Hệ thống hiển thị dòng/form để nhập Giá trị. \newline 5. Người dùng nhập Tên Giá trị (Value Name) (ví dụ: "Nhỏ") (Bắt buộc, duy nhất trong Thuộc tính - BR-UC2.13-1). \newline 6. (Tùy chọn) Nếu Thuộc tính có Loại hiển thị là Color, người dùng chọn màu tương ứng. \newline 7. Người dùng lặp lại bước 3-6 để thêm các Giá trị khác cho Thuộc tính này. \newline 8. Người dùng chọn "Lưu" trên form Thuộc tính để lưu tất cả các Giá trị. \newline 9. Hệ thống kiểm tra và lưu các Giá trị mới. \newline 10. Hệ thống báo thành công. \\
\hline
Alternative Flow & Không có. \\
\hline
Exception Flow & \textbf{9a. Lỗi Xác thực/Lưu:} Tên Giá trị trống/trùng. Lỗi hệ thống. \\
\hline
\multicolumn{2}{|c|}{\textbf{2.3. Thông tin bổ sung (Additional Information)}} \\
\hline
\textbf{Mục} & \textbf{Nội dung} \\
\hline
Business Rule & - \textbf{BR-UC2.13-1:} Tên Giá trị phải duy nhất trong phạm vi Thuộc tính cha của nó. \\
\hline
Non-Functional Requirement & - \textbf{NFR-UC2.13-1 (Usability):} Việc thêm giá trị trong ngữ cảnh thuộc tính phải thuận tiện. \newline - \textbf{NFR-UC2.13-2 (Performance):} Lưu nhanh. \\
\hline
\end{longtable}

\subsubsection{Use Case UC-MD02-14: Gán Thuộc tính và Giá trị vào Sản phẩm Gốc}

\begin{longtable}{|m{4cm}|p{11cm}|}
\caption{Đặc tả Use Case UC-MD02-14: Gán Thuộc tính và Giá trị vào Sản phẩm Gốc} \label{tab:uc_md02_14_revised} \\
\hline
\multicolumn{2}{|c|}{\textbf{2.1. Tóm tắt (Summary)}} \\
\hline
\textbf{Mục} & \textbf{Nội dung} \\
\hline
\endhead % Header cho các trang tiếp theo
\hline
\endfoot % Footer cho bảng
\hline
\endlastfoot % Footer cho trang cuối cùng
Use Case Name & Gán Thuộc tính và Giá trị vào Sản phẩm Gốc \\
\hline
Use Case ID & UC-MD02-14 \\
\hline
Use Case Description & Cho phép Quản lý nhà hàng (US-01) áp dụng các Thuộc tính (đã tạo ở UC-MD02-12) và chọn các Giá trị cụ thể (đã tạo ở UC-MD02-13) cho một Sản phẩm Gốc (Product Template). Hành động này sẽ làm cơ sở để hệ thống tự động tạo ra các Sản phẩm Biến thể. \\
\hline
Actor & US-01 (Quản lý nhà hàng) \\
\hline
Priority & Must Have (Nếu cần biến thể) \\
\hline
Trigger & Cần tạo các phiên bản khác nhau cho một sản phẩm dựa trên các đặc tính đã định nghĩa. \\
\hline
Pre-Condition & - US-01 đã đăng nhập với quyền quản lý sản phẩm. \newline - Sản phẩm Gốc đã được tạo (UC-MD02-01). \newline - Các Thuộc tính và Giá trị liên quan đã được tạo (UC-MD02-12, UC-MD02-13). \\
\hline
Post-Condition & - Sản phẩm Gốc được liên kết với các dòng Thuộc tính, mỗi dòng Thuộc tính chứa các Giá trị được chọn. \newline - Hệ thống tự động tạo ra các bản ghi Sản phẩm Biến thể (Product Variant) tương ứng với mọi tổ hợp của các Giá trị đã chọn. \\
\hline
\multicolumn{2}{|c|}{\textbf{2.2. Luồng thực thi (Flow)}} \\
\hline
\textbf{Mục} & \textbf{Nội dung} \\
\hline
Basic Flow & 1. US-01 mở Form chi tiết của Sản phẩm Gốc cần cấu hình biến thể, ở chế độ Sửa. \newline 2. US-01 chuyển đến tab "Variants" (Biến thể). \newline 3. Trong phần "Attributes", US-01 chọn "Add a line". \newline 4. US-01 chọn một Thuộc tính từ danh sách thả xuống (ví dụ: "Kích cỡ Pizza"). \newline 5. Trong cột "Values" tương ứng, US-01 chọn (tick) vào (các) Giá trị sẽ áp dụng cho sản phẩm này (ví dụ: "Nhỏ", "Vừa", "Lớn"). \newline 6. US-01 lặp lại bước 3-5 để thêm các Thuộc tính và Giá trị khác nếu cần. \newline 7. US-01 chọn "Lưu" (Save) sản phẩm gốc. \newline 8. Hệ thống tự động tạo ra các bản ghi Sản phẩm Biến thể dựa trên tổ hợp các Giá trị đã chọn. Số lượng biến thể được tạo sẽ hiển thị trên form (ví dụ: nút "X Variants"). \\
\hline
Alternative Flow & Không có. \\
\hline
Exception Flow & \textbf{4a/5a. Lỗi chọn Thuộc tính/Giá trị:} Thuộc tính/Giá trị không tồn tại (phải được tạo trước). \newline \textbf{7a. Lỗi khi lưu cấu hình thuộc tính.} \newline \textbf{8a. Lỗi khi tự động tạo biến thể.} \\
\hline
\multicolumn{2}{|c|}{\textbf{2.3. Thông tin bổ sung (Additional Information)}} \\
\hline
\textbf{Mục} & \textbf{Nội dung} \\
\hline
Business Rule & - \textbf{BR-UC2.14-1:} Số lượng Biến thể tạo ra bằng tích số lượng Giá trị được chọn cho mỗi Thuộc tính. \\
\hline
Non-Functional Requirement & - \textbf{NFR-UC2.14-1 (Usability):} Giao diện gán thuộc tính/giá trị phải trực quan. \newline - \textbf{NFR-UC2.14-2 (Performance):} Việc tự động tạo biến thể (< 50 biến thể) phải nhanh (< 5 giây). \\
\hline
\end{longtable}

\subsubsection{Use Case UC-MD02-15: Cấu hình Giá/Phụ thu cho Biến thể Sản phẩm}

\begin{longtable}{|m{4cm}|p{11cm}|}
\caption{Đặc tả Use Case UC-MD02-15: Cấu hình Giá/Phụ thu cho Biến thể Sản phẩm} \label{tab:uc_md02_15_revised} \\
\hline
\multicolumn{2}{|c|}{\textbf{2.1. Tóm tắt (Summary)}} \\
\hline
\textbf{Mục} & \textbf{Nội dung} \\
\hline
\endhead % Header cho các trang tiếp theo
\hline
\endfoot % Footer cho bảng
\hline
\endlastfoot % Footer cho trang cuối cùng
Use Case Name & Cấu hình Giá/Phụ thu cho Biến thể Sản phẩm \\
\hline
Use Case ID & UC-MD02-15 \\
\hline
Use Case Description & Sau khi các Sản phẩm Biến thể đã được tạo (từ UC-MD02-14), cho phép Quản lý nhà hàng (US-01) truy cập vào từng biến thể cụ thể để đặt một giá bán riêng cho nó, hoặc thiết lập một giá trị phụ thu (price extra) dựa trên các giá trị thuộc tính tạo nên biến thể đó. \\
\hline
Actor & US-01 (Quản lý nhà hàng) \\
\hline
Priority & Must Have (Nếu các biến thể có giá khác nhau) \\
\hline
Trigger & Các biến thể khác nhau của cùng một sản phẩm gốc có giá bán không giống nhau. \\
\hline
Pre-Condition & - Các Sản phẩm Biến thể đã được tạo (UC-MD02-14). \newline - US-01 đang xem form Sản phẩm Gốc hoặc danh sách các Biến thể của nó. \\
\hline
Post-Condition & - Giá bán hoặc phụ thu của (các) Biến thể được chọn đã được cập nhật. \newline - Khi biến thể đó được chọn trên POS, giá sẽ được tính đúng theo cấu hình này. \\
\hline
\multicolumn{2}{|c|}{\textbf{2.2. Luồng thực thi (Flow)}} \\
\hline
\textbf{Mục} & \textbf{Nội dung} \\
\hline
Basic Flow (Đặt giá riêng cho biến thể) & 1. US-01 đang xem form Sản phẩm Gốc, nhấp vào nút "X Variants" để mở danh sách các Biến thể. \newline 2. US-01 chọn một Biến thể cụ thể từ danh sách để mở form chi tiết của Biến thể đó. \newline 3. US-01 chọn "Sửa" (Edit). \newline 4. US-01 tìm đến trường "Giá bán" (Sales Price) của Biến thể và nhập giá bán mong muốn cho riêng biến thể này. \newline 5. US-01 chọn "Lưu" (Save). \newline 6. Hệ thống lưu giá bán mới cho Biến thể. \\
\hline
Alternative Flow & \textbf{Basic Flow (Đặt phụ thu cho giá trị thuộc tính):} \newline    1. US-01 đang xem form Sản phẩm Gốc, ở tab "Variants", trong chế độ Sửa. \newline    2. Bên cạnh mỗi Giá trị của Thuộc tính (ví dụ: bên cạnh "Lớn" của "Kích cỡ Pizza"), có một cột "Phụ thu giá trị" (Value Price Extra). \newline    3. US-01 nhập số tiền phụ thu cho Giá trị đó (ví dụ: +20,000 VNĐ cho size Lớn). \newline    4. US-01 chọn "Lưu" (Save) Sản phẩm Gốc. \newline    5. Hệ thống sẽ tự động tính giá của Biến thể bằng cách cộng giá Sản phẩm Gốc với tổng các phụ thu của các Giá trị tạo nên Biến thể đó. \\
\hline
Exception Flow & \textbf{6a/5a-alt. Lỗi lưu giá/phụ thu:} Hệ thống báo lỗi khi lưu. \\
\hline
\multicolumn{2}{|c|}{\textbf{2.3. Thông tin bổ sung (Additional Information)}} \\
\hline
\textbf{Mục} & \textbf{Nội dung} \\
\hline
Business Rule & - \textbf{BR-UC2.15-1:} Nếu không đặt giá riêng cho Biến thể, nó sẽ kế thừa giá của Sản phẩm Gốc cộng với các phụ thu giá trị thuộc tính. \newline - \textbf{BR-UC2.15-2:} Giá bán phải là số không âm. \\
\hline
Non-Functional Requirement & - \textbf{NFR-UC2.15-1 (Usability):} Việc cấu hình giá/phụ thu phải rõ ràng. \newline - \textbf{NFR-UC2.15-2 (Accuracy):} Giá cuối cùng của biến thể phải được tính toán chính xác. \\
\hline
\end{longtable}

\subsubsection{Use Case UC-MD02-16: Thiết lập Sản phẩm được Bán trên POS}

\begin{longtable}{|m{4cm}|p{11cm}|}
\caption{Đặc tả Use Case UC-MD02-16: Thiết lập Sản phẩm được Bán trên POS} \label{tab:uc_md02_16_revised} \\
\hline
\multicolumn{2}{|c|}{\textbf{2.1. Tóm tắt (Summary)}} \\
\hline
\textbf{Mục} & \textbf{Nội dung} \\
\hline
\endhead % Header cho các trang tiếp theo
\hline
\endfoot % Footer cho bảng
\hline
\endlastfoot % Footer cho trang cuối cùng
Use Case Name & Thiết lập Sản phẩm được Bán trên POS \\
\hline
Use Case ID & UC-MD02-16 \\
\hline
Use Case Description & Cho phép Quản lý nhà hàng (US-01) đánh dấu hoặc bỏ đánh dấu một sản phẩm là có sẵn để bán trên giao diện Point of Sale (POS), qua đó kiểm soát những mặt hàng nào sẽ xuất hiện trên menu POS. \\
\hline
Actor & US-01 (Quản lý nhà hàng) \\
\hline
Priority & Must Have \\
\hline
Trigger & - Cần đưa một sản phẩm mới lên bán trên POS. \newline - Cần tạm thời ẩn một sản phẩm khỏi POS mà không cần lưu trữ. \\
\hline
Pre-Condition & - US-01 đang xem form chi tiết sản phẩm ở chế độ Sửa. \\
\hline
Post-Condition & - Trạng thái "Available in POS" của sản phẩm được cập nhật. \newline - Thay đổi này ảnh hưởng đến việc sản phẩm có hiển thị trên POS hay không sau khi đồng bộ. \\
\hline
\multicolumn{2}{|c|}{\textbf{2.2. Luồng thực thi (Flow)}} \\
\hline
\textbf{Mục} & \textbf{Nội dung} \\
\hline
Basic Flow & 1. US-01 đang ở form chi tiết Sản phẩm, chế độ Sửa. \newline 2. US-01 chuyển đến tab "Point of Sale" (hoặc "Sales"). \newline 3. US-01 tìm ô kiểm "Available in POS" (Có sẵn trong POS). \newline 4. US-01 đánh dấu (tick) để cho phép bán trên POS, hoặc bỏ đánh dấu để ẩn khỏi POS. \newline 5. US-01 chọn "Lưu". \newline 6. Hệ thống lưu thay đổi. \\
\hline
Alternative Flow & Không có. \\
\hline
Exception Flow & \textbf{6a. Lỗi lưu:} Hệ thống báo lỗi khi lưu. \\
\hline
\multicolumn{2}{|c|}{\textbf{2.3. Thông tin bổ sung (Additional Information)}} \\
\hline
\textbf{Mục} & \textbf{Nội dung} \\
\hline
Business Rule & - \textbf{BR-UC2.16-1:} Chỉ sản phẩm được đánh dấu "Available in POS" mới hiển thị trên menu POS. \\
\hline
Non-Functional Requirement & - \textbf{NFR-UC2.16-1 (Usability):} Ô kiểm dễ tìm. \newline - \textbf{NFR-UC2.16-2 (Integration):} Cấu hình phải đồng bộ đúng xuống POS. \\
\hline
\end{longtable}

\subsubsection{Use Case UC-MD02-17: Gán Sản phẩm vào Danh mục POS}

\begin{longtable}{|m{4cm}|p{11cm}|}
\caption{Đặc tả Use Case UC-MD02-17: Gán Sản phẩm vào Danh mục POS} \label{tab:uc_md02_17_revised} \\
\hline
\multicolumn{2}{|c|}{\textbf{2.1. Tóm tắt (Summary)}} \\
\hline
\textbf{Mục} & \textbf{Nội dung} \\
\hline
\endhead % Header cho các trang tiếp theo
\hline
\endfoot % Footer cho bảng
\hline
\endlastfoot % Footer cho trang cuối cùng
Use Case Name & Gán Sản phẩm vào Danh mục POS \\
\hline
Use Case ID & UC-MD02-17 \\
\hline
Use Case Description & Cho phép Quản lý nhà hàng (US-01) chỉ định một sản phẩm (đã được đánh dấu "Available in POS") sẽ thuộc về (các) Danh mục POS nào, để sản phẩm đó được hiển thị đúng nhóm trên giao diện chọn món của POS. \\
\hline
Actor & US-01 (Quản lý nhà hàng) \\
\hline
Priority & Must Have \\
\hline
Trigger & Cần phân loại một sản phẩm vào đúng nhóm trên menu POS. \\
\hline
Pre-Condition & - US-01 đang xem form chi tiết sản phẩm ở chế độ Sửa. \newline - Sản phẩm đã được đánh dấu "Available in POS" (UC-MD02-16). \newline - Các Danh mục POS liên quan đã được tạo (UC-MD02-07). \\
\hline
Post-Condition & - Liên kết giữa sản phẩm và (các) Danh mục POS được cập nhật. \newline - Sản phẩm sẽ hiển thị trong (các) danh mục đó trên POS sau khi đồng bộ. \\
\hline
\multicolumn{2}{|c|}{\textbf{2.2. Luồng thực thi (Flow)}} \\
\hline
\textbf{Mục} & \textbf{Nội dung} \\
\hline
Basic Flow & 1. US-01 đang ở form chi tiết Sản phẩm, chế độ Sửa, tab "Point of Sale". \newline 2. US-01 tìm trường "POS Category" (Danh mục POS). \newline 3. US-01 nhấp vào trường này và chọn một Danh mục POS từ danh sách thả xuống. \newline 4. US-01 chọn "Lưu". \newline 5. Hệ thống lưu thay đổi. \\
\hline
Alternative Flow & \textbf{3a. Gán nhiều danh mục (nếu hệ thống hỗ trợ trường Many2many riêng):} \newline    1. US-01 thao tác trên trường cho phép chọn nhiều danh mục để gán sản phẩm. \\
\hline
Exception Flow & \textbf{3b. Danh mục không tồn tại:} Danh sách trống nếu chưa tạo danh mục. \newline \textbf{5a. Lỗi lưu.} \\
\hline
\multicolumn{2}{|c|}{\textbf{2.3. Thông tin bổ sung (Additional Information)}} \\
\hline
\textbf{Mục} & \textbf{Nội dung} \\
\hline
Business Rule & - \textbf{BR-UC2.17-1:} Sản phẩm phải được gán vào ít nhất một Danh mục POS để hiển thị có tổ chức trên POS (trừ khi có khu vực "Chưa phân loại"). \\
\hline
Non-Functional Requirement & - \textbf{NFR-UC2.17-1 (Usability):} Việc chọn danh mục phải dễ dàng. \\
\hline
\end{longtable}

\subsubsection{Use Case UC-MD02-18: Tải lên/Thay đổi Hình ảnh Sản phẩm}

\begin{longtable}{|m{4cm}|p{11cm}|}
\caption{Đặc tả Use Case UC-MD02-18: Tải lên/Thay đổi Hình ảnh Sản phẩm} \label{tab:uc_md02_18_revised} \\
\hline
\multicolumn{2}{|c|}{\textbf{2.1. Tóm tắt (Summary)}} \\
\hline
\textbf{Mục} & \textbf{Nội dung} \\
\hline
\endhead % Header cho các trang tiếp theo
\hline
\endfoot % Footer cho bảng
\hline
\endlastfoot % Footer cho trang cuối cùng
Use Case Name & Tải lên/Thay đổi Hình ảnh Sản phẩm \\
\hline
Use Case ID & UC-MD02-18 \\
\hline
Use Case Description & Cho phép Quản lý nhà hàng (US-01) tải lên một hình ảnh mới hoặc thay thế hình ảnh hiện có cho một sản phẩm. \\
\hline
Actor & US-01 (Quản lý nhà hàng) \\
\hline
Priority & Should Have \\
\hline
Trigger & Cần thêm/cập nhật ảnh minh họa cho sản phẩm. \\
\hline
Pre-Condition & - US-01 đang xem form chi tiết sản phẩm ở chế độ Sửa. \newline - Có sẵn tệp hình ảnh phù hợp trên máy tính. \\
\hline
Post-Condition & - Sản phẩm được cập nhật với hình ảnh mới. \newline - Ảnh sẽ hiển thị trên các giao diện liên quan. \\
\hline
\multicolumn{2}{|c|}{\textbf{2.2. Luồng thực thi (Flow)}} \\
\hline
\textbf{Mục} & \textbf{Nội dung} \\
\hline
Basic Flow & 1. US-01 đang ở form chi tiết Sản phẩm, chế độ Sửa. \newline 2. US-01 nhấp vào khu vực hình ảnh (thường có biểu tượng máy ảnh hoặc bút chì). \newline 3. Hệ thống mở hộp thoại chọn tệp. US-01 chọn tệp ảnh và nhấn "Open". \newline 4. Hệ thống kiểm tra tệp (định dạng, kích thước - BR-UC2.18-1, BR-UC2.18-2). \newline 5. Nếu hợp lệ, hệ thống tải lên và hiển thị ảnh xem trước. \newline 6. US-01 chọn "Lưu". \newline 7. Hệ thống lưu ảnh mới cho sản phẩm. \\
\hline
Alternative Flow & Không có. \\
\hline
Exception Flow & \textbf{4a. Tệp không hợp lệ (định dạng/kích thước):} Hệ thống báo lỗi, yêu cầu chọn lại. \newline \textbf{5a. Lỗi tải tệp lên server.} \newline \textbf{7a. Lỗi lưu sản phẩm.} \\
\hline
\multicolumn{2}{|c|}{\textbf{2.3. Thông tin bổ sung (Additional Information)}} \\
\hline
\textbf{Mục} & \textbf{Nội dung} \\
\hline
Business Rule & - \textbf{BR-UC2.18-1:} Hỗ trợ các định dạng ảnh phổ biến (JPG, PNG...). \newline - \textbf{BR-UC2.18-2:} Có giới hạn kích thước tệp tối đa. \newline - \textbf{BR-UC2.18-3:} Ảnh tải lên ở cấp Sản phẩm Gốc. Biến thể dùng chung ảnh này. \\
\hline
Non-Functional Requirement & - \textbf{NFR-UC2.18-1 (Usability):} Tải ảnh đơn giản. \newline - \textbf{NFR-UC2.18-2 (Performance):} Tải ảnh và lưu nhanh. \\
\hline
\end{longtable}

\subsubsection{Use Case UC-MD02-19: Xóa Hình ảnh Sản phẩm}

\begin{longtable}{|m{4cm}|p{11cm}|}
\caption{Đặc tả Use Case UC-MD02-19: Xóa Hình ảnh Sản phẩm} \label{tab:uc_md02_19_revised} \\
\hline
\multicolumn{2}{|c|}{\textbf{2.1. Tóm tắt (Summary)}} \\
\hline
\textbf{Mục} & \textbf{Nội dung} \\
\hline
\endhead % Header cho các trang tiếp theo
\hline
\endfoot % Footer cho bảng
\hline
\endlastfoot % Footer cho trang cuối cùng
Use Case Name & Xóa Hình ảnh Sản phẩm \\
\hline
Use Case ID & UC-MD02-19 \\
\hline
Use Case Description & Cho phép Quản lý nhà hàng (US-01) xóa bỏ hình ảnh hiện tại đang được liên kết với một sản phẩm. \\
\hline
Actor & US-01 (Quản lý nhà hàng) \\
\hline
Priority & Should Have \\
\hline
Trigger & Hình ảnh sản phẩm không còn phù hợp hoặc không muốn hiển thị ảnh nữa. \\
\hline
Pre-Condition & - US-01 đang xem form chi tiết sản phẩm ở chế độ Sửa. \newline - Sản phẩm đang có hình ảnh. \\
\hline
Post-Condition & - Liên kết hình ảnh bị xóa khỏi sản phẩm. \newline - Sản phẩm sẽ hiển thị ảnh mặc định (placeholder) hoặc không có ảnh. \\
\hline
\multicolumn{2}{|c|}{\textbf{2.2. Luồng thực thi (Flow)}} \\
\hline
\textbf{Mục} & \textbf{Nội dung} \\
\hline
Basic Flow & 1. US-01 đang ở form chi tiết Sản phẩm, chế độ Sửa. \newline 2. US-01 di chuột vào khu vực hình ảnh sản phẩm hiện tại. \newline 3. Biểu tượng xóa ảnh (ví dụ: thùng rác, dấu 'x') xuất hiện. US-01 nhấp vào đó. \newline 4. Hệ thống (có thể) yêu cầu xác nhận xóa ảnh. US-01 xác nhận. \newline 5. Hệ thống xóa liên kết ảnh, khung ảnh trở thành trống hoặc hiển thị ảnh placeholder. \newline 6. US-01 chọn "Lưu". \newline 7. Hệ thống lưu thay đổi (sản phẩm không còn ảnh). \\
\hline
Alternative Flow & Không có. \\
\hline
Exception Flow & \textbf{7a. Lỗi lưu sản phẩm.} \\
\hline
\multicolumn{2}{|c|}{\textbf{2.3. Thông tin bổ sung (Additional Information)}} \\
\hline
\textbf{Mục} & \textbf{Nội dung} \\
\hline
Business Rule & - \textbf{BR-UC2.19-1:} Việc xóa ảnh là không thể hoàn tác trực tiếp (trừ khi tải lại ảnh cũ). \\
\hline
Non-Functional Requirement & - \textbf{NFR-UC2.19-1 (Usability):} Thao tác xóa ảnh phải rõ ràng. \\
\hline
\end{longtable}

\subsubsection{Use Case UC-MD02-20: Gán Danh mục Sản phẩm vào Máy in Bếp/KDS}

\begin{longtable}{|m{4cm}|p{11cm}|}
\caption{Đặc tả Use Case UC-MD02-20: Gán Danh mục Sản phẩm vào Máy in Bếp/KDS} \label{tab:uc_md02_20_revised} \\
\hline
\multicolumn{2}{|c|}{\textbf{2.1. Tóm tắt (Summary)}} \\
\hline
\textbf{Mục} & \textbf{Nội dung} \\
\hline
\endhead % Header cho các trang tiếp theo
\hline
\endfoot % Footer cho bảng
\hline
\endlastfoot % Footer cho trang cuối cùng
Use Case Name & Gán Danh mục Sản phẩm vào Máy in Bếp/KDS \\
\hline
Use Case ID & UC-MD02-20 \\
\hline
Use Case Description & Cho phép Quản lý nhà hàng (US-01) cấu hình trong cài đặt Point of Sale để chỉ định những Danh mục Sản phẩm POS nào sẽ được gửi đến một Máy in Bếp hoặc Màn hình KDS cụ thể khi có đơn hàng. \\
\hline
Actor & US-01 (Quản lý nhà hàng) \\
\hline
Priority & Must Have (nếu có nhiều điểm chuẩn bị/in) \\
\hline
Trigger & Cần thiết lập hoặc thay đổi quy tắc định tuyến đơn hàng cho các trạm bếp/bar. \\
\hline
Pre-Condition & - US-01 đã đăng nhập với quyền quản trị cấu hình POS. \newline - Các thiết bị Máy in Bếp/KDS đã được khai báo trong cấu hình POS (thường qua IoT Box). \newline - Các Danh mục POS liên quan đã được tạo (UC-MD02-07). \\
\hline
Post-Condition & - Quy tắc định tuyến (Danh mục POS -> Thiết bị Bếp/KDS) được lưu. \newline - Đơn hàng từ POS sẽ được gửi đúng nơi dựa trên cấu hình này. \\
\hline
\multicolumn{2}{|c|}{\textbf{2.2. Luồng thực thi (Flow)}} \\
\hline
\textbf{Mục} & \textbf{Nội dung} \\
\hline
Basic Flow & 1. US-01 truy cập Cấu hình Point of Sale, chọn một cấu hình POS cụ thể. \newline 2. US-01 tìm đến mục "Order Printers" (Máy in Đơn hàng) hoặc "Kitchen Display" (KDS). \newline 3. US-01 chọn một Máy in/KDS đã khai báo để chỉnh sửa. \newline 4. Trong form cấu hình của Máy in/KDS đó, US-01 tìm trường "Printed Product Categories" (hoặc tương tự). \newline 5. US-01 nhấp vào trường này và chọn (tick) vào (các) Danh mục POS mà món ăn thuộc các danh mục đó cần được gửi đến thiết bị này. \newline 6. US-01 xác nhận lựa chọn. \newline 7. US-01 lưu cấu hình cho Máy in/KDS đó. \newline 8. US-01 lưu lại toàn bộ cấu hình POS. \newline 9. Hệ thống báo thành công. \\
\hline
Alternative Flow & \textbf{5a. Bỏ chọn danh mục:} Loại bỏ một danh mục khỏi định tuyến của thiết bị. \\
\hline
Exception Flow & \textbf{2a. Chưa khai báo Máy in/KDS.} \newline \textbf{5b. Chưa có Danh mục POS nào.} \newline \textbf{8a. Lỗi lưu cấu hình POS.} \\
\hline
\multicolumn{2}{|c|}{\textbf{2.3. Thông tin bổ sung (Additional Information)}} \\
\hline
\textbf{Mục} & \textbf{Nội dung} \\
\hline
Business Rule & - \textbf{BR-UC2.20-1:} Định tuyến dựa trên Danh mục POS của sản phẩm. \newline - \textbf{BR-UC2.20-2:} Thay đổi cấu hình này yêu cầu POS client đồng bộ lại. \\
\hline
Non-Functional Requirement & - \textbf{NFR-UC2.20-1 (Usability):} Giao diện cấu hình định tuyến phải rõ ràng. \newline - \textbf{NFR-UC2.20-2 (Integration):} Cấu hình phải được IoT Box/dịch vụ in diễn giải đúng. \\
\hline
\end{longtable}


\subsection{Module MD-03: Quản lý Đặt chỗ \& Đặt món trước}

\subsubsection{Use Case UC-MD03-01: Tìm kiếm Khung giờ/Bàn trống Online}

\begin{longtable}{|m{4cm}|p{11cm}|}
\caption{Đặc tả Use Case UC-MD03-01: Tìm kiếm Khung giờ/Bàn trống Online} \label{tab:uc_md03_01_revised_v3} \\
\hline
\multicolumn{2}{|c|}{\textbf{2.1. Tóm tắt (Summary)}} \\
\hline
\textbf{Mục} & \textbf{Nội dung} \\
\hline
\endhead % Header cho các trang tiếp theo
\hline
\endfoot % Footer cho bảng
\hline
\endlastfoot % Footer cho trang cuối cùng
Use Case Name & Tìm kiếm Khung giờ/Bàn trống Online \\
\hline
Use Case ID & UC-MD03-01 \\
\hline
Use Case Description & Cho phép Khách hàng (US-08) truy cập trang đặt chỗ của nhà hàng, nhập các tiêu chí như ngày, giờ mong muốn và số lượng người để hệ thống tìm kiếm và hiển thị các lựa chọn đặt bàn (khung giờ/bàn) còn trống. \\
\hline
Actor & US-08 (Khách hàng) \\
\hline
Priority & Must Have \\
\hline
Trigger & Khách hàng muốn đặt bàn tại nhà hàng qua kênh trực tuyến và cần tìm các lựa chọn phù hợp. \\
\hline
Pre-Condition & - Trang web/ứng dụng của nhà hàng có chức năng đặt chỗ online đang hoạt động. \newline - Các tham số cấu hình đặt chỗ (giờ hoạt động, giới hạn khách...) đã được thiết lập (FR-MD03-15). \\
\hline
Post-Condition & - Hệ thống hiển thị cho khách hàng danh sách các khung giờ còn trống phù hợp với tiêu chí tìm kiếm. \newline - HOẶC nếu được cấu hình cho phép chọn bàn, hệ thống hiển thị sơ đồ tầng với các bàn trống phù hợp. \newline - HOẶC hệ thống thông báo không tìm thấy lựa chọn phù hợp. \newline - Khách hàng sẵn sàng để chọn một khung giờ/bàn cụ thể (nếu có) hoặc điều chỉnh tiêu chí tìm kiếm. \\
\hline
\multicolumn{2}{|c|}{\textbf{2.2. Luồng thực thi (Flow)}} \\
\hline
\textbf{Mục} & \textbf{Nội dung} \\
\hline
Basic Flow & 1. Khách hàng (US-08) truy cập trang/màn hình đặt chỗ của nhà hàng. \newline 2. US-08 nhập (hoặc chọn từ lịch) Ngày muốn đặt bàn. \newline 3. US-08 nhập (hoặc chọn từ danh sách) Số lượng người. \newline 4. US-08 nhập (hoặc chọn từ danh sách) Giờ mong muốn đặt bàn (có thể là một khoảng giờ ưu tiên). \newline 5. US-08 nhấn nút "Tìm kiếm" / "Kiểm tra Tính khả dụng" / "Find a Table". \newline 6. Hệ thống (System) dựa trên thông tin đầu vào và dữ liệu đặt chỗ hiện tại, tìm kiếm các khung giờ/bàn còn trống phù hợp (tuân thủ BR-UC3.1-1, BR-UC3.1-2, BR-UC3.1-3). \newline 7. Hệ thống hiển thị kết quả cho US-08: \newline    a. Danh sách các khung giờ còn trống xung quanh giờ khách chọn. \newline    b. HOẶC (nếu cấu hình cho phép chọn bàn và có bàn phù hợp) chuyển sang giao diện chọn bàn cụ thể (UC-MD03-02). \\
\hline
Alternative Flow & \textbf{4a. Khách hàng không chọn giờ cụ thể:} \newline    1. US-08 chỉ chọn Ngày và Số lượng người, sau đó nhấn "Tìm kiếm". \newline    2. Hệ thống hiển thị tất cả các khung giờ còn trống trong ngày đó phù hợp với số lượng người. \newline \textbf{7c. Không tìm thấy lựa chọn phù hợp:} \newline    1. Hệ thống không tìm thấy khung giờ/bàn nào còn trống khớp với yêu cầu. \newline    2. Hệ thống hiển thị thông báo "Rất tiếc, không có bàn trống phù hợp. Vui lòng thử chọn ngày/giờ khác hoặc số lượng người khác." hoặc đề xuất các khung giờ/ngày gần nhất còn trống. \\
\hline
Exception Flow & \textbf{2a/3a/4a. Nhập liệu không hợp lệ:} \newline    1. US-08 nhập ngày quá khứ, số lượng người không hợp lệ (ví dụ: chữ, số âm, quá lớn/nhỏ so với cấu hình). \newline    2. Hệ thống hiển thị thông báo lỗi yêu cầu nhập lại. Use Case quay lại bước tương ứng. \newline \textbf{6a. Lỗi hệ thống khi tìm kiếm:} \newline    1. Hệ thống gặp lỗi kỹ thuật khi truy vấn dữ liệu hoặc xử lý logic tìm kiếm. \newline    2. Hệ thống hiển thị thông báo lỗi chung. \\
\hline
\multicolumn{2}{|c|}{\textbf{2.3. Thông tin bổ sung (Additional Information)}} \\
\hline
\textbf{Mục} & \textbf{Nội dung} \\
\hline
Business Rule & - \textbf{BR-UC3.1-1 (System):} Hệ thống chỉ hiển thị các ngày trong tương lai và trong khoảng thời gian cho phép đặt trước (cấu hình ở FR-MD03-15). \newline - \textbf{BR-UC3.1-2 (System):} Hệ thống chỉ xem xét các khung giờ nằm trong giờ hoạt động của nhà hàng và còn đủ chỗ cho số lượng người khách yêu cầu (cấu hình ở FR-MD03-15). \newline - \textbf{BR-UC3.1-3 (System):} Hệ thống phải kiểm tra tình trạng bàn thực tế (từ các đặt chỗ đã xác nhận) để xác định tính khả dụng. \\
\hline
Non-Functional Requirement & - \textbf{NFR-UC3.1-1 (Usability):} Giao diện tìm kiếm phải trực quan, dễ dàng chọn ngày, giờ, số người. \newline - \textbf{NFR-UC3.1-2 (Performance):} Thời gian hệ thống trả về kết quả tìm kiếm phải nhanh (dưới 2-3 giây). \newline - \textbf{NFR-UC3.1-3 (Accuracy):} Kết quả tìm kiếm (khung giờ/bàn trống) phải chính xác. \\
\hline
\end{longtable}

\subsubsection{Use Case UC-MD03-02: (Tùy chọn) Chọn Bàn cụ thể từ Sơ đồ tầng Online}
% (Giữ nguyên nội dung chi tiết như UC-MD03-03 version cũ, chỉ đổi ID)
\begin{longtable}{|m{4cm}|p{11cm}|}
\caption{Đặc tả Use Case UC-MD03-02: (Tùy chọn) Chọn Bàn cụ thể từ Sơ đồ tầng Online} \label{tab:uc_md03_02_revised_v3} \\
\hline
\multicolumn{2}{|c|}{\textbf{2.1. Tóm tắt (Summary)}} \\
\hline
\textbf{Mục} & \textbf{Nội dung} \\
\hline
\endhead % Header cho các trang tiếp theo
\hline
\endfoot % Footer cho bảng
\hline
\endlastfoot % Footer cho trang cuối cùng
Use Case Name & (Tùy chọn) Chọn Bàn cụ thể từ Sơ đồ tầng Online \\
\hline
Use Case ID & UC-MD03-02 \\
\hline
Use Case Description & Nếu được Quản lý nhà hàng cấu hình và sau khi khách hàng đã tìm được khung giờ phù hợp (UC-MD03-01), cho phép Khách hàng (US-08) xem sơ đồ mặt bằng (floor plan) của nhà hàng và chọn một bàn trống cụ thể cho lượt đặt chỗ của mình. \\
\hline
Actor & US-08 (Khách hàng) \\
\hline
Priority & Low / Nice to Have \\
\hline
Trigger & - Hệ thống chuyển hướng đến giao diện chọn bàn sau khi khách hàng hoàn thành UC-MD03-01 và hệ thống được cấu hình cho phép chọn bàn. \\
\hline
Pre-Condition & - Khách hàng đã hoàn thành UC-MD03-01 và hệ thống xác định có khung giờ/bàn trống. \newline - Quản lý nhà hàng (US-01) đã kích hoạt chức năng cho phép khách chọn bàn và đã thiết lập sơ đồ tầng. \\
\hline
Post-Condition & - Hệ thống ghi nhận bàn cụ thể mà khách hàng đã chọn. \newline - Bàn đó được tạm giữ cho khách hàng trong một khoảng thời gian để hoàn tất đặt chỗ. \\
\hline
\multicolumn{2}{|c|}{\textbf{2.2. Luồng thực thi (Flow)}} \\
\hline
\textbf{Mục} & \textbf{Nội dung} \\
\hline
Basic Flow & 1. Tiếp nối từ UC-MD03-01, hệ thống hiển thị sơ đồ mặt bằng (Floor Plan) cho khách hàng. \newline 2. Các bàn phù hợp với số lượng người và còn trống vào ngày giờ đã chọn được làm nổi bật. \newline 3. US-08 xem xét sơ đồ và nhấp vào một bàn trống cụ thể muốn chọn. \newline 4. Hệ thống xác nhận lựa chọn bàn. \newline 5. Hệ thống có thể hiển thị thông tin tóm tắt bao gồm bàn đã chọn và chuyển sang bước tiếp theo (chọn món hoặc xác nhận đặt chỗ). \\
\hline
Alternative Flow & \textbf{1a. Không chọn bàn cụ thể / Bỏ qua:} \newline    1. Nếu khách hàng không muốn chọn bàn cụ thể hoặc giao diện cho phép bỏ qua. \newline    2. US-08 chọn nút "Bỏ qua chọn bàn" / "Để nhà hàng xếp bàn". \newline    3. Hệ thống sẽ tự động gán một bàn phù hợp sau này. Use Case kết thúc. \\
\hline
Exception Flow & \textbf{3a. Bàn vừa chọn đã bị người khác đặt:} \newline    1. Trong lúc khách hàng xem, bàn đó bị người khác đặt. \newline    2. Hệ thống báo lỗi "Bàn này vừa có người khác đặt. Vui lòng chọn bàn khác." \newline    3. Use Case quay lại bước 2. \\
\hline
\multicolumn{2}{|c|}{\textbf{2.3. Thông tin bổ sung (Additional Information)}} \\
\hline
\textbf{Mục} & \textbf{Nội dung} \\
\hline
Business Rule & - \textbf{BR-UC3.2-1 (V3):} Chức năng này phải được kích hoạt trong cấu hình (FR-MD03-15). \newline - \textbf{BR-UC3.2-2 (V3):} Sơ đồ tầng phải chính xác. \newline - \textbf{BR-UC3.2-3 (V3):} Chỉ hiển thị bàn trống, phù hợp sức chứa. \newline - \textbf{BR-UC3.2-4 (V3):} Hệ thống nên tạm giữ bàn đã chọn. \\
\hline
Non-Functional Requirement & - \textbf{NFR-UC3.2-1 (V3) (Usability):} Sơ đồ tầng rõ ràng, trạng thái bàn dễ phân biệt. \newline - \textbf{NFR-UC3.2-2 (V3) (Performance):} Tải sơ đồ nhanh. \newline - \textbf{NFR-UC3.2-3 (V3) (Accuracy):} Trạng thái bàn chính xác. \\
\hline
\end{longtable}

\subsubsection{Use Case UC-MD03-03: Chọn Món ăn Đặt trước từ Thực đơn Online}
% (Gộp UC-MD03-04 cũ và UC-MD03-05 cũ)
\begin{longtable}{|m{4cm}|p{11cm}|}
\caption{Đặc tả Use Case UC-MD03-03: Chọn Món ăn Đặt trước từ Thực đơn Online} \label{tab:uc_md03_03_revised_v3} \\
\hline
\multicolumn{2}{|c|}{\textbf{2.1. Tóm tắt (Summary)}} \\
\hline
\textbf{Mục} & \textbf{Nội dung} \\
\hline
\endhead % Header cho các trang tiếp theo
\hline
\endfoot % Footer cho bảng
\hline
\endlastfoot % Footer cho trang cuối cùng
Use Case Name & Chọn Món ăn Đặt trước từ Thực đơn Online \\
\hline
Use Case ID & UC-MD03-03 \\
\hline
Use Case Description & Cho phép Khách hàng (US-08) duyệt thực đơn trực tuyến của nhà hàng, lựa chọn các món ăn/đồ uống mong muốn cùng số lượng và các tùy chọn biến thể (nếu có), sau đó thêm chúng vào giỏ hàng đặt trước cho lượt đặt bàn của mình. \\
\hline
Actor & US-08 (Khách hàng) \\
\hline
Priority & Must Have (nếu có chức năng đặt món trước) \\
\hline
Trigger & Khách hàng đã chọn xong thông tin đặt bàn (và có thể đã chọn bàn) và muốn đặt trước một số món ăn để tiết kiệm thời gian khi đến nhà hàng. \\
\hline
Pre-Condition & - Khách hàng đang trong luồng đặt chỗ online (đã qua UC-MD03-01, có thể cả UC-MD03-02). \newline - Chức năng đặt món trước được kích hoạt. \newline - Thực đơn online đã được Quản lý nhà hàng cấu hình (sản phẩm, danh mục, giá, ảnh...). \\
\hline
Post-Condition & - Một "giỏ hàng" hoặc "danh sách đặt trước" tạm thời được tạo, chứa các món ăn/đồ uống, số lượng và biến thể đã chọn. \newline - Khách hàng có thể xem tổng giá trị tạm tính của các món trong giỏ hàng này. \newline - Khách hàng sẵn sàng để chuyển sang bước xác nhận và thanh toán đặt cọc. \\
\hline
\multicolumn{2}{|c|}{\textbf{2.2. Luồng thực thi (Flow)}} \\
\hline
\textbf{Mục} & \textbf{Nội dung} \\
\hline
Basic Flow & 1. Sau khi hoàn thành việc chọn thông tin đặt bàn (UC-MD03-01/UC-MD03-02), hệ thống hiển thị tùy chọn "Đặt món trước" / "Chọn món từ thực đơn". \newline 2. US-08 chọn tùy chọn này. \newline 3. Hệ thống hiển thị giao diện thực đơn online, được sắp xếp theo danh mục (ví dụ: Khai vị, Món chính...). Mỗi món hiển thị tên, giá, hình ảnh (nếu có), mô tả (nếu có). \newline 4. US-08 duyệt qua các danh mục và món ăn. \newline 5. Khi tìm thấy món muốn đặt, US-08 nhấp vào nút "Thêm" / "+" hoặc tương tự. \newline 6. \textbf{Nếu món không có biến thể:} Hệ thống thêm món vào giỏ hàng đặt trước với số lượng mặc định là 1. \newline 7. \textbf{Nếu món có biến thể (ví dụ: size, độ cay):} \newline    a. Hệ thống hiển thị popup/dialog yêu cầu US-08 chọn các giá trị thuộc tính (ví dụ: Size L, Ít cay). \newline    b. US-08 chọn các giá trị và xác nhận. \newline    c. Hệ thống thêm biến thể cụ thể đó vào giỏ hàng với số lượng 1. \newline 8. US-08 có thể điều chỉnh số lượng của từng món trong giỏ hàng (tăng/giảm/nhập số). \newline 9. US-08 lặp lại bước 4-8 để chọn thêm các món khác. \newline 10. Giao diện hiển thị giỏ hàng đặt trước được cập nhật liên tục (số lượng món, tổng tiền món ăn). \newline 11. Sau khi chọn xong, US-08 nhấn nút "Tiếp tục" / "Xong" để chuyển sang bước tiếp theo. \\
\hline
Alternative Flow & \textbf{4a. Tìm kiếm/Lọc món ăn trên thực đơn:} Giao diện cho phép tìm kiếm món theo tên hoặc lọc theo các tiêu chí khác (nếu có). \newline \textbf{10a. Xóa món khỏi giỏ hàng:} US-08 có thể xóa bất kỳ món nào đã thêm vào giỏ. \\
\hline
Exception Flow & \textbf{3a. Lỗi tải thực đơn:} Hệ thống không thể hiển thị thực đơn. \newline \textbf{7d. Lỗi thêm món/biến thể vào giỏ:} Hệ thống báo lỗi, món không được thêm. \newline \textbf{7e. Chưa chọn đủ biến thể bắt buộc:} Hệ thống yêu cầu chọn đầy đủ. \\
\hline
\multicolumn{2}{|c|}{\textbf{2.3. Thông tin bổ sung (Additional Information)}} \\
\hline
\textbf{Mục} & \textbf{Nội dung} \\
\hline
Business Rule & - \textbf{BR-UC3.3-1 (V3):} Chỉ sản phẩm được cấu hình hiển thị online mới xuất hiện. \newline - \textbf{BR-UC3.3-2 (V3):} Phải chọn biến thể bắt buộc (nếu có). \newline - \textbf{BR-UC3.3-3 (V3):} Giá món/biến thể phải chính xác. \newline - \textbf{BR-UC3.3-4 (V3):} Giỏ hàng là tạm thời cho đến khi hoàn tất đặt chỗ. \\
\hline
Non-Functional Requirement & - \textbf{NFR-UC3.3-1 (V3) (Usability):} Thực đơn online dễ duyệt, chọn món và biến thể thuận tiện. \newline - \textbf{NFR-UC3.3-2 (V3) (Performance):} Tải thực đơn và thêm món vào giỏ phải nhanh. \newline - \textbf{NFR-UC3.3-3 (V3) (Accuracy):} Thông tin món và giỏ hàng phải chính xác. \\
\hline
\end{longtable}

\subsubsection{Use Case UC-MD03-04: Xác nhận Lượt Đặt chỗ và Thanh toán Đặt cọc Online}
% (Gộp UC-MD03-06, UC-MD03-07, UC-MD03-09, và một phần UC-MD03-10 cũ)
\begin{longtable}{|m{4cm}|p{11cm}|}
\caption{Đặc tả Use Case UC-MD03-04: Xác nhận Lượt Đặt chỗ và Thanh toán Đặt cọc Online} \label{tab:uc_md03_04_revised_v3} \\
\hline
\multicolumn{2}{|c|}{\textbf{2.1. Tóm tắt (Summary)}} \\
\hline
\textbf{Mục} & \textbf{Nội dung} \\
\hline
\endhead % Header cho các trang tiếp theo
\hline
\endfoot % Footer cho bảng
\hline
\endlastfoot % Footer cho trang cuối cùng
Use Case Name & Xác nhận Lượt Đặt chỗ và Thanh toán Đặt cọc Online \\
\hline
Use Case ID & UC-MD03-04 \\
\hline
Use Case Description & Cho phép Khách hàng (US-08) xem lại toàn bộ thông tin lượt đặt chỗ đã thiết lập (bàn, món ăn), cung cấp thông tin liên hệ cá nhân, và thực hiện thanh toán số tiền đặt cọc bắt buộc (nếu có) qua cổng thanh toán trực tuyến để hoàn tất và xác nhận đặt chỗ. \\
\hline
Actor & US-08 (Khách hàng), System (Tính cọc, Xác nhận, Gửi thông báo) \\
\hline
Priority & Must Have \\
\hline
Trigger & Khách hàng đã hoàn thành việc lựa chọn thông tin đặt bàn và các món ăn đặt trước (nếu có) và muốn tiến hành hoàn tất đặt chỗ. \\
\hline
Pre-Condition & - Khách hàng đang trong luồng đặt chỗ online. \newline - Các lựa chọn về bàn (UC-MD03-01/02) và món ăn (UC-MD03-03) đã được ghi nhận tạm thời. \newline - Quy tắc tính đặt cọc đã được cấu hình (FR-MD03-15). \newline - Cổng thanh toán đã được tích hợp (FR-MD10-05). \\
\hline
Post-Condition & - \textbf{Thành công:} Lượt đặt chỗ được xác nhận. Tiền đặt cọc (nếu có) được thanh toán thành công. Khách hàng nhận được xác nhận (mã đặt chỗ, email/SMS). Bàn được giữ. \newline - \textbf{Thất bại:} Đặt chỗ không được xác nhận (ví dụ: thanh toán lỗi). \\
\hline
\multicolumn{2}{|c|}{\textbf{2.2. Luồng thực thi (Flow)}} \\
\hline
\textbf{Mục} & \textbf{Nội dung} \\
\hline
Basic Flow & 1. Sau khi US-08 hoàn tất chọn món (UC-MD03-03) hoặc chọn thông tin bàn (nếu không đặt món trước) và nhấn "Tiếp tục"/"Xác nhận". \newline 2. Hệ thống hiển thị trang tóm tắt lượt đặt chỗ, bao gồm: thông tin bàn đã chọn (ngày, giờ, số người, bàn cụ thể nếu có), danh sách chi tiết món ăn đặt trước (tên, SL, giá, biến thể), tổng giá trị món ăn. \newline 3. Hệ thống tự động tính toán và hiển thị số tiền đặt cọc cần thanh toán (dựa trên BR-UC3.8-1, BR-UC3.8-2, BR-UC3.8-3, BR-UC3.8-4). \newline 4. Hệ thống yêu cầu US-08 nhập thông tin cá nhân: Họ và Tên, Số Điện Thoại, Địa chỉ Email (Bắt buộc - BR-UC3.7-1, BR-UC3.7-2, BR-UC3.7-3). \newline 5. (Tùy chọn) US-08 nhập Ghi chú cho nhà hàng. \newline 6. US-08 kiểm tra lại toàn bộ thông tin và nhấn "Xác nhận và Thanh toán Đặt cọc" (hoặc "Xác nhận Đặt chỗ" nếu không có cọc). \newline 7. Hệ thống kiểm tra tính hợp lệ của thông tin cá nhân. \newline 8. \textbf{Nếu có yêu cầu đặt cọc:} \newline    a. Hệ thống chuyển hướng khách hàng đến cổng thanh toán đã chọn hoặc hiển thị form thanh toán nhúng (Tương tự luồng của UC-MD03-09 cũ). \newline    b. US-08 thực hiện thanh toán. \newline    c. Nếu thanh toán thành công: Hệ thống nhận xác nhận. \newline    d. Nếu thanh toán thất bại: Hệ thống báo lỗi, cho phép thử lại hoặc quay lại. (Exception Flow 8d1) \newline 9. \textbf{Sau khi thanh toán thành công (hoặc nếu không có cọc và nhấn xác nhận ở bước 6):} \newline    a. Hệ thống (System) tạo bản ghi đặt chỗ chính thức với trạng thái "Đã xác nhận". \newline    b. Hệ thống (System) tạo mã đặt chỗ duy nhất. \newline    c. Hệ thống (System) cập nhật trạng thái bàn đã chọn là "Đã đặt". \newline    d. Hệ thống (System) gửi email/SMS xác nhận cho khách hàng với đầy đủ chi tiết và mã đặt chỗ. \newline    e. Hệ thống hiển thị trang "Đặt chỗ thành công" cho US-08, bao gồm mã đặt chỗ và thông tin tóm tắt. \\
\hline
Alternative Flow & \textbf{4a. Khách hàng đã đăng nhập:} Hệ thống tự động điền thông tin cá nhân. \newline \textbf{8e. Đặt chỗ không yêu cầu đặt cọc:} Nếu cấu hình không yêu cầu đặt cọc, bước 8 (thanh toán) được bỏ qua. Sau bước 7, hệ thống trực tiếp thực hiện các hành động ở bước 9. \\
\hline
Exception Flow & \textbf{7a. Lỗi xác thực thông tin cá nhân:} Hệ thống báo lỗi, yêu cầu nhập lại. \newline \textbf{8d1. Thanh toán đặt cọc thất bại:} Khách hàng được thông báo, có thể chọn thử lại phương thức khác hoặc hủy bỏ đặt chỗ. Đặt chỗ không được xác nhận. \newline \textbf{9f. Lỗi hệ thống khi tạo đặt chỗ/gửi xác nhận:} Hệ thống báo lỗi nghiêm trọng. Cần thông báo cho quản trị viên. Nếu tiền cọc đã trừ, cần có quy trình hoàn tiền hoặc xử lý thủ công. \\
\hline
\multicolumn{2}{|c|}{\textbf{2.3. Thông tin bổ sung (Additional Information)}} \\
\hline
\textbf{Mục} & \textbf{Nội dung} \\
\hline
Business Rule & (Kết hợp các BR liên quan từ UC-MD03-06, 07, 08, 09, 10 cũ) \newline - \textbf{BR-UC3.4-1 (V3):} Phải hiển thị tóm tắt chính xác. \newline - \textbf{BR-UC3.4-2 (V3):} Thông tin liên hệ (Tên, SĐT, Email) là bắt buộc. \newline - \textbf{BR-UC3.4-3 (V3):} Đặt cọc (nếu có) phải được thanh toán thành công để xác nhận. \newline - \textbf{BR-UC3.4-4 (V3):} Xác nhận đặt chỗ (mã, email/SMS) phải được gửi cho khách. \\
\hline
Non-Functional Requirement & (Kết hợp các NFR liên quan) \newline - \textbf{NFR-UC3.4-1 (V3) (Usability):} Toàn bộ quy trình từ xem tóm tắt, nhập thông tin đến thanh toán phải liền mạch, dễ hiểu. \newline - \textbf{NFR-UC3.4-2 (V3) (Security):} Thanh toán và xử lý thông tin cá nhân phải bảo mật. \newline - \textbf{NFR-UC3.4-3 (V3) (Performance):} Các bước phải nhanh chóng. \newline - \textbf{NFR-UC3.4-4 (V3) (Reliability):} Quy trình phải đáng tin cậy, đặc biệt là việc ghi nhận thanh toán và xác nhận đặt chỗ. \\
\hline
\end{longtable}

\subsubsection{Use Case UC-MD03-05: Xem Lịch sử Đặt chỗ Cá nhân Online}
% (Giữ nguyên nội dung chi tiết như UC-MD03-17 version cũ, chỉ đổi ID)
\begin{longtable}{|m{4cm}|p{11cm}|}
\caption{Đặc tả Use Case UC-MD03-05: Xem Lịch sử Đặt chỗ Cá nhân Online} \label{tab:uc_md03_05_revised_v3} \\
\hline
\multicolumn{2}{|c|}{\textbf{2.1. Tóm tắt (Summary)}} \\
\hline
\textbf{Mục} & \textbf{Nội dung} \\
\hline
\endhead % Header cho các trang tiếp theo
\hline
\endfoot % Footer cho bảng
\hline
\endlastfoot % Footer cho trang cuối cùng
Use Case Name & Xem Lịch sử Đặt chỗ Cá nhân Online \\
\hline
Use Case ID & UC-MD03-05 \\
\hline
Use Case Description & Cho phép Khách hàng (US-08) đã đăng nhập vào tài khoản trên website/app của nhà hàng xem lại danh sách các lượt đặt chỗ mà họ đã thực hiện trước đây và xem thông tin chi tiết của từng lượt đặt chỗ. \\
\hline
Actor & US-08 (Khách hàng) \\
\hline
Priority & Should Have \\
\hline
Trigger & Khách hàng muốn kiểm tra lại thông tin một lượt đặt chỗ sắp tới hoặc xem lại lịch sử các lần đặt chỗ trước đây. \\
\hline
Pre-Condition & - Khách hàng (US-08) có tài khoản trên website/app và đã đăng nhập thành công. \newline - Khách hàng đã thực hiện ít nhất một lượt đặt chỗ online thông qua tài khoản này trước đó. \\
\hline
Post-Condition & - Khách hàng xem được danh sách các lượt đặt chỗ của mình. \newline - Khách hàng xem được thông tin chi tiết của một lượt đặt chỗ cụ thể. \\
\hline
\multicolumn{2}{|c|}{\textbf{2.2. Luồng thực thi (Flow)}} \\
\hline
\textbf{Mục} & \textbf{Nội dung} \\
\hline
Basic Flow & 1. Khách hàng (US-08) đã đăng nhập vào tài khoản. \newline 2. US-08 điều hướng đến khu vực quản lý tài khoản cá nhân và chọn mục "Lịch sử Đặt chỗ", "Đặt chỗ của tôi" hoặc tương tự. \newline 3. Hệ thống truy vấn cơ sở dữ liệu để lấy danh sách các lượt đặt chỗ được liên kết với tài khoản của US-08. \newline 4. Hệ thống hiển thị danh sách các lượt đặt chỗ, bao gồm các thông tin cơ bản như: \newline    - Mã đặt chỗ. \newline    - Ngày giờ đặt. \newline    - Số lượng người. \newline    - Trạng thái (Đã xác nhận, Đã hủy, Đã hoàn thành...). \newline 5. US-08 nhấp vào một lượt đặt chỗ cụ thể trong danh sách để xem chi tiết. \newline 6. Hệ thống truy xuất và hiển thị thông tin chi tiết của lượt đặt chỗ đó. \newline 7. US-08 xem xét thông tin. \\
\hline
Alternative Flow & \textbf{4a. Lọc/Sắp xếp lịch sử.} \newline \textbf{6a. Hủy đặt chỗ từ màn hình chi tiết (nếu được phép - xem UC-MD03-06).} \\
\hline
Exception Flow & \textbf{3a. Lỗi tải lịch sử đặt chỗ.} \newline \textbf{3b. Không có lịch sử đặt chỗ.} \\
\hline
\multicolumn{2}{|c|}{\textbf{2.3. Thông tin bổ sung (Additional Information)}} \\
\hline
\textbf{Mục} & \textbf{Nội dung} \\
\hline
Business Rule & - \textbf{BR-UC3.5-1 (V3):} Khách hàng chỉ xem được lịch sử của chính mình. \newline - \textbf{BR-UC3.5-2 (V3):} Thông tin hiển thị phải chính xác. \\
\hline
Non-Functional Requirement & - \textbf{NFR-UC3.5-1 (V3) (Usability):} Dễ truy cập, dễ hiểu. \newline - \textbf{NFR-UC3.5-2 (V3) (Performance):} Tải nhanh. \newline - \textbf{NFR-UC3.5-3 (V3) (Security):} Đảm bảo bảo mật. \\
\hline
\end{longtable}

\subsubsection{Use Case UC-MD03-06: (Tùy chọn) Hủy Lượt Đặt chỗ Online}

\begin{longtable}{|m{4cm}|p{11cm}|}
\caption{Đặc tả Use Case UC-MD03-06: (Tùy chọn) Hủy Lượt Đặt chỗ Online} \label{tab:uc_md03_06_revised_v3} \\
\hline
\multicolumn{2}{|c|}{\textbf{2.1. Tóm tắt (Summary)}} \\
\hline
\textbf{Mục} & \textbf{Nội dung} \\
\hline
\endhead % Header cho các trang tiếp theo
\hline
\endfoot % Footer cho bảng
\hline
\endlastfoot % Footer cho trang cuối cùng
Use Case Name & (Tùy chọn) Hủy Lượt Đặt chỗ Online \\
\hline
Use Case ID & UC-MD03-06 \\
\hline
Use Case Description & Cho phép Khách hàng (US-08) đã đăng nhập, tự hủy một lượt đặt chỗ đã được xác nhận của mình thông qua giao diện website/app, nếu việc hủy đó tuân thủ chính sách và thời hạn cho phép của nhà hàng. \\
\hline
Actor & US-08 (Khách hàng) \\
\hline
Priority & Should Have \\
\hline
Trigger & Khách hàng có thay đổi kế hoạch và muốn hủy lượt đặt chỗ đã được xác nhận. \\
\hline
Pre-Condition & - Khách hàng đã đăng nhập và đang xem chi tiết một lượt đặt chỗ của mình (UC-MD03-05). \newline - Lượt đặt chỗ đang ở trạng thái "Đã xác nhận" (Confirmed). \newline - Việc hủy vẫn còn trong thời hạn cho phép theo chính sách của nhà hàng (ví dụ: hủy trước 24 giờ so với giờ đặt). \newline - Chức năng cho phép khách tự hủy online được kích hoạt. \\
\hline
Post-Condition & - Trạng thái của lượt đặt chỗ được cập nhật thành "Đã hủy bởi khách" (Cancelled by Customer). \newline - Bàn đã giữ cho đặt chỗ này được giải phóng. \newline - Tiền đặt cọc có thể được xử lý theo chính sách (hoàn lại một phần/toàn bộ hoặc không hoàn - quy trình hoàn tiền có thể là một UC riêng hoặc xử lý thủ công). \newline - Khách hàng nhận được thông báo xác nhận hủy. \\
\hline
\multicolumn{2}{|c|}{\textbf{2.2. Luồng thực thi (Flow)}} \\
\hline
\textbf{Mục} & \textbf{Nội dung} \\
\hline
Basic Flow & 1. US-08 đang xem chi tiết lượt đặt chỗ muốn hủy (từ UC-MD03-05). \newline 2. Giao diện hiển thị nút "Hủy Đặt chỗ" (Cancel Booking). \newline 3. US-08 nhấp vào nút "Hủy Đặt chỗ". \newline 4. Hệ thống hiển thị hộp thoại yêu cầu xác nhận việc hủy, có thể kèm thông tin về chính sách hoàn/mất cọc. \newline 5. US-08 xác nhận muốn hủy. \newline 6. Hệ thống (System) cập nhật trạng thái đặt chỗ thành "Cancelled by Customer". \newline 7. Hệ thống (System) giải phóng bàn đã giữ (nếu có). \newline 8. Hệ thống (System) xử lý tiền đặt cọc theo chính sách (ví dụ: tự động hoàn một phần nếu đủ điều kiện, hoặc ghi nhận mất cọc). \newline 9. Hệ thống (System) gửi email/SMS thông báo hủy đặt chỗ thành công cho khách hàng. \newline 10. Hệ thống hiển thị thông báo "Đặt chỗ của bạn đã được hủy thành công." \\
\hline
Alternative Flow & \textbf{8a. Hoàn cọc cần xử lý thủ công:} \newline    1. Nếu chính sách hoàn cọc phức tạp hoặc cần quản lý duyệt, hệ thống chỉ ghi nhận yêu cầu hoàn cọc. Việc hoàn tiền thực tế do nhân viên thực hiện sau. \\
\hline
Exception Flow & \textbf{2a. Không thể hủy online:} \newline    1. Đặt chỗ không đủ điều kiện để hủy online (ví dụ: quá sát giờ đặt, đã qua giờ đặt, loại đặt chỗ không cho hủy online). \newline    2. Nút "Hủy Đặt chỗ" bị vô hiệu hóa hoặc không hiển thị. Hệ thống có thể hiển thị thông báo yêu cầu liên hệ nhà hàng để hủy. \newline \textbf{6a. Lỗi hệ thống khi hủy:} \newline    1. Hệ thống gặp lỗi khi cập nhật trạng thái hoặc giải phóng bàn. \newline    2. Hệ thống báo lỗi. Việc hủy có thể chưa hoàn tất. \\
\hline
\multicolumn{2}{|c|}{\textbf{2.3. Thông tin bổ sung (Additional Information)}} \\
\hline
\textbf{Mục} & \textbf{Nội dung} \\
\hline
Business Rule & - \textbf{BR-UC3.6-1 (V3):} Chính sách hủy đặt chỗ (thời hạn cho phép hủy, điều kiện hoàn cọc) phải được cấu hình rõ ràng (FR-MD03-15) và thông báo cho khách hàng. \newline - \textbf{BR-UC3.6-2 (V3):} Việc giải phóng bàn là bắt buộc khi hủy thành công. \\
\hline
Non-Functional Requirement & - \textbf{NFR-UC3.6-1 (V3) (Usability):} Nút hủy dễ tìm, thông tin chính sách hủy rõ ràng. \newline - \textbf{NFR-UC3.6-2 (V3) (Reliability):} Quy trình hủy và xử lý cọc phải đáng tin cậy. \\
\hline
\end{longtable}

% Các Use Case cho Nhân viên (MD03-07 đến MD03-16) sẽ được giữ nguyên cấu trúc như đã làm
% với việc tách Tạo/Sửa và các hành động Quản lý Trạng thái.
% Do giới hạn độ dài, tôi sẽ tiếp tục với các UC này trong phản hồi tiếp theo nếu Giáo sư yêu cầu.

\subsubsection{Use Case UC-MD03-07: Xem Danh sách Tổng hợp các Lượt Đặt chỗ}
% Nội dung tương tự UC-MD03-12 (cũ)
\begin{longtable}{|m{4cm}|p{11cm}|}
\caption{Đặc tả Use Case UC-MD03-07: Xem Danh sách Tổng hợp các Lượt Đặt chỗ} \label{tab:uc_md03_07_revised_v3} \\
\hline
\multicolumn{2}{|c|}{\textbf{2.1. Tóm tắt (Summary)}} \\
\hline
\textbf{Mục} & \textbf{Nội dung} \\
\hline
\endhead % Header cho các trang tiếp theo
\hline
\endfoot % Footer cho bảng
\hline
\endlastfoot % Footer cho trang cuối cùng
Use Case Name & Xem Danh sách Tổng hợp các Lượt Đặt chỗ \\
\hline
Use Case ID & UC-MD03-07 \\
\hline
Use Case Description & Cho phép Nhân viên được phân quyền (Quản lý, Lễ tân) xem danh sách tổng hợp các lượt đặt chỗ đã được tạo trong hệ thống (bao gồm cả đặt online và nhập thủ công), với khả năng lọc và tìm kiếm theo các tiêu chí khác nhau. \\
\hline
Actor & US-01 (Quản lý nhà hàng), US-03 (Nhân viên lễ tân) \\
\hline
Priority & Must Have \\
\hline
Trigger & Nhân viên cần kiểm tra các lượt đặt chỗ sắp tới, xem tình hình đặt bàn chung, hoặc tìm kiếm một lượt đặt chỗ cụ thể. \\
\hline
Pre-Condition & - Người dùng (US-01 hoặc US-03) đã đăng nhập vào hệ thống với quyền xem đặt chỗ. \newline - Đã có ít nhất một lượt đặt chỗ được tạo trong hệ thống. \\
\hline
Post-Condition & - Danh sách các lượt đặt chỗ phù hợp với tiêu chí lọc/tìm kiếm được hiển thị cho người dùng. \newline - Người dùng có cái nhìn tổng quan về tình trạng đặt chỗ. \\
\hline
\multicolumn{2}{|c|}{\textbf{2.2. Luồng thực thi (Flow)}} \\
\hline
\textbf{Mục} & \textbf{Nội dung} \\
\hline
Basic Flow & 1. Người dùng (US-01/US-03) truy cập vào module quản lý Đặt chỗ (Reservations). \newline 2. Hệ thống mặc định hiển thị danh sách các lượt đặt chỗ, thường là các lượt đặt cho ngày hiện tại hoặc tương lai gần, ở dạng danh sách (List View) hoặc dạng lịch (Calendar View). \newline 3. Danh sách hiển thị các thông tin cơ bản của mỗi lượt đặt chỗ, ví dụ: \newline    - Mã đặt chỗ. \newline    - Tên khách hàng. \newline    - Ngày giờ đặt. \newline    - Số lượng người. \newline    - Bàn được gán (nếu có). \newline    - Trạng thái đặt chỗ (ví dụ: Đã xác nhận, Chờ xác nhận, Đã hủy, Đã đến). \newline    - Trạng thái thanh toán cọc (nếu có). \newline 4. Người dùng xem xét danh sách. \\
\hline
Alternative Flow & \textbf{4a. Lọc danh sách:} \newline    1. Người dùng sử dụng các bộ lọc có sẵn (Filters) để thu hẹp danh sách, ví dụ: lọc theo Ngày, theo Trạng thái, theo Bàn, theo Khách hàng. \newline    2. Hệ thống áp dụng bộ lọc và hiển thị lại danh sách kết quả. \newline    3. Use Case quay lại bước 4. \newline \textbf{4b. Tìm kiếm:} \newline    1. Người dùng nhập từ khóa (ví dụ: tên khách, SĐT, mã đặt chỗ) vào ô tìm kiếm. \newline    2. Hệ thống thực hiện tìm kiếm và hiển thị các lượt đặt chỗ khớp với từ khóa. \newline    3. Use Case quay lại bước 4. \newline \textbf{4c. Sắp xếp danh sách:} \newline    1. Người dùng nhấp vào tiêu đề cột (ví dụ: Ngày giờ đặt, Tên khách hàng) để sắp xếp danh sách tăng dần hoặc giảm dần theo cột đó. \newline    2. Hệ thống sắp xếp lại và hiển thị danh sách. \newline    3. Use Case quay lại bước 4. \newline \textbf{4d. Chuyển đổi dạng xem:} \newline    1. Người dùng chọn chuyển đổi sang dạng xem khác (ví dụ: từ List View sang Calendar View hoặc Kanban View nếu có). \newline    2. Hệ thống hiển thị dữ liệu đặt chỗ theo dạng xem mới. \newline    3. Use Case quay lại bước 4. \\
\hline
Exception Flow & \textbf{2a. Lỗi tải danh sách:} \newline    1. Hệ thống gặp lỗi khi truy vấn hoặc hiển thị danh sách đặt chỗ. \newline    2. Hệ thống hiển thị thông báo lỗi. \newline    3. Use Case kết thúc không thành công. \newline \textbf{2b. Không có đặt chỗ nào:} \newline    1. Nếu không có lượt đặt chỗ nào phù hợp với bộ lọc mặc định. \newline    2. Hệ thống hiển thị danh sách trống hoặc thông báo "Không có đặt chỗ nào". \\
\hline
\multicolumn{2}{|c|}{\textbf{2.3. Thông tin bổ sung (Additional Information)}} \\
\hline
\textbf{Mục} & \textbf{Nội dung} \\
\hline
Business Rule & - \textbf{BR-UC3.7-1 (V3):} Danh sách phải hiển thị đủ thông tin cơ bản để nhân viên có thể nhận diện nhanh lượt đặt chỗ. \newline - \textbf{BR-UC3.7-2 (V3):} Các bộ lọc và chức năng tìm kiếm phải hoạt động chính xác, giúp người dùng dễ dàng tìm thấy thông tin cần thiết. \newline - \textbf{BR-UC3.7-3 (V3):} Trạng thái đặt chỗ hiển thị phải là trạng thái mới nhất của lượt đặt chỗ đó. \\
\hline
Non-Functional Requirement & - \textbf{NFR-UC3.7-1 (V3) (Usability):} Giao diện danh sách phải rõ ràng, dễ đọc. Các chức năng lọc, tìm kiếm, sắp xếp phải dễ sử dụng. \newline - \textbf{NFR-UC3.7-2 (V3) (Performance):} Thời gian tải danh sách đặt chỗ (ví dụ: cho một ngày) phải nhanh chóng (dưới 3 giây). Việc lọc/tìm kiếm cũng cần phản hồi nhanh. \newline - \textbf{NFR-UC3.7-3 (V3) (Security):} Chỉ những người dùng có quyền hạn phù hợp mới được phép xem danh sách đặt chỗ. \newline - \textbf{NFR-UC3.7-4 (V3) (Accuracy):} Dữ liệu hiển thị trong danh sách phải chính xác và đồng bộ với dữ liệu gốc trong cơ sở dữ liệu. \\
\hline
\end{longtable}

\subsubsection{Use Case UC-MD03-08: Xem Thông tin Chi tiết một Lượt Đặt chỗ}
% Nội dung tương tự UC-MD03-13 (cũ)
\begin{longtable}{|m{4cm}|p{11cm}|}
\caption{Đặc tả Use Case UC-MD03-08: Xem Thông tin Chi tiết một Lượt Đặt chỗ} \label{tab:uc_md03_08_revised_v3} \\
\hline
\multicolumn{2}{|c|}{\textbf{2.1. Tóm tắt (Summary)}} \\
\hline
\textbf{Mục} & \textbf{Nội dung} \\
\hline
\endhead % Header cho các trang tiếp theo
\hline
\endfoot % Footer cho bảng
\hline
\endlastfoot % Footer cho trang cuối cùng
Use Case Name & Xem Thông tin Chi tiết một Lượt Đặt chỗ \\
\hline
Use Case ID & UC-MD03-08 \\
\hline
Use Case Description & Cho phép Nhân viên được phân quyền (Quản lý, Lễ tân) xem thông tin chi tiết đầy đủ của một lượt đặt chỗ cụ thể đã được chọn từ danh sách. \\
\hline
Actor & US-01 (Quản lý nhà hàng), US-03 (Nhân viên lễ tân) \\
\hline
Priority & Must Have \\
\hline
Trigger & Nhân viên nhấp vào một lượt đặt chỗ cụ thể từ danh sách đặt chỗ (UC-MD03-07) để xem thông tin chi tiết hơn. \\
\hline
Pre-Condition & - Người dùng đang xem danh sách đặt chỗ (UC-MD03-07 thành công). \newline - Người dùng có quyền xem chi tiết đặt chỗ. \\
\hline
Post-Condition & - Form/màn hình hiển thị chi tiết đầy đủ của lượt đặt chỗ đã chọn được hiển thị cho người dùng. \newline - Người dùng nắm được mọi thông tin liên quan đến lượt đặt chỗ đó. \\
\hline
\multicolumn{2}{|c|}{\textbf{2.2. Luồng thực thi (Flow)}} \\
\hline
\textbf{Mục} & \textbf{Nội dung} \\
\hline
Basic Flow & 1. Người dùng (US-01/US-03) đang xem danh sách đặt chỗ (UC-MD03-07). \newline 2. Người dùng nhấp vào mã đặt chỗ, tên khách hàng hoặc một vùng có thể nhấp được của một dòng đặt chỗ cụ thể. \newline 3. Hệ thống truy xuất toàn bộ thông tin chi tiết của lượt đặt chỗ đã chọn từ cơ sở dữ liệu. \newline 4. Hệ thống hiển thị Form/màn hình chi tiết đặt chỗ, bao gồm các thông tin: \newline    - Mã đặt chỗ. \newline    - Thông tin khách hàng (Tên, SĐT, Email). \newline    - Ngày giờ đặt. \newline    - Thời lượng đặt (ước tính). \newline    - Số lượng người. \newline    - Bàn được chỉ định (nếu có). \newline    - Trạng thái đặt chỗ hiện tại. \newline    - Thông tin thanh toán đặt cọc (Số tiền cọc, trạng thái thanh toán, phương thức thanh toán). \newline    - Danh sách chi tiết các món ăn đặt trước (tên món, biến thể, số lượng, đơn giá, thành tiền). \newline    - Tổng giá trị món ăn đặt trước. \newline    - Ghi chú của khách hàng (nếu có). \newline    - Ghi chú nội bộ (nếu có). \newline    - Lịch sử thay đổi trạng thái hoặc các thông tin quan trọng (nếu có). \newline 5. Người dùng xem xét các thông tin chi tiết. \\
\hline
Alternative Flow & \textbf{5a. Thực hiện hành động từ màn hình chi tiết:} \newline    1. Từ màn hình chi tiết, người dùng có thể truy cập các hành động khác như "Sửa" (UC-MD03-10), "Hủy" (UC-MD03-12), "Đánh dấu đã đến" (UC-MD03-13), "In thông tin"... tùy thuộc vào quyền hạn và trạng thái đặt chỗ. \\
\hline
Exception Flow & \textbf{3a. Lỗi truy xuất dữ liệu chi tiết:} \newline    1. Hệ thống gặp lỗi khi cố gắng lấy thông tin chi tiết của lượt đặt chỗ từ cơ sở dữ liệu. \newline    2. Hệ thống hiển thị thông báo lỗi. \newline    3. Use Case kết thúc không thành công. Người dùng có thể quay lại danh sách. \newline \textbf{3b. Đặt chỗ không tồn tại/không có quyền truy cập:} \newline    1. Do lỗi đồng bộ hoặc vấn đề phân quyền, người dùng nhấp vào một đặt chỗ mà họ không có quyền xem hoặc đã bị xóa. \newline    2. Hệ thống hiển thị thông báo lỗi "Không tìm thấy đặt chỗ" hoặc "Bạn không có quyền truy cập". \newline    3. Use Case kết thúc không thành công. \\
\hline
\multicolumn{2}{|c|}{\textbf{2.3. Thông tin bổ sung (Additional Information)}} \\
\hline
\textbf{Mục} & \textbf{Nội dung} \\
\hline
Business Rule & - \textbf{BR-UC3.8-1 (V3):} Màn hình chi tiết phải hiển thị tất cả các thông tin liên quan đến lượt đặt chỗ một cách đầy đủ và chính xác. \newline - \textbf{BR-UC3.8-2 (V3):} Các thông tin nhạy cảm (nếu có) cần được kiểm soát quyền truy cập. \\
\hline
Non-Functional Requirement & - \textbf{NFR-UC3.8-1 (V3) (Usability):} Thông tin chi tiết cần được trình bày một cách logic, dễ đọc. Các phần thông tin khác nhau (thông tin khách, chi tiết đặt bàn, món ăn, thanh toán) nên được phân tách rõ ràng. \newline - \textbf{NFR-UC3.8-2 (V3) (Performance):} Thời gian tải và hiển thị đầy đủ chi tiết của một lượt đặt chỗ phải nhanh chóng (dưới 2 giây). \newline - \textbf{NFR-UC3.8-3 (V3) (Accuracy):} Mọi thông tin hiển thị phải là dữ liệu mới nhất và chính xác nhất của lượt đặt chỗ đó. \\
\hline
\end{longtable}

\subsubsection{Use Case UC-MD03-09: Tạo mới Lượt Đặt chỗ Thủ công (Backend/POS)}
% (Tương ứng FR-MD03-14A)
\begin{longtable}{|m{4cm}|p{11cm}|}
\caption{Đặc tả Use Case UC-MD03-09: Tạo mới Lượt Đặt chỗ Thủ công (Backend/POS)} \label{tab:uc_md03_09_revised_v3} \\
\hline
\multicolumn{2}{|c|}{\textbf{2.1. Tóm tắt (Summary)}} \\
\hline
\textbf{Mục} & \textbf{Nội dung} \\
\hline
\endhead % Header cho các trang tiếp theo
\hline
\endfoot % Footer cho bảng
\hline
\endlastfoot % Footer cho trang cuối cùng
Use Case Name & Tạo mới Lượt Đặt chỗ Thủ công (Backend/POS) \\
\hline
Use Case ID & UC-MD03-09 \\
\hline
Use Case Description & Cho phép Nhân viên được phân quyền (Quản lý, Lễ tân) tạo một lượt đặt chỗ mới trực tiếp trong hệ thống (qua giao diện backend hoặc một giao diện POS được thiết kế cho quản lý đặt chỗ), thường dành cho các trường hợp khách đặt qua điện thoại, email, hoặc khách vãng lai muốn đặt trước. \\
\hline
Actor & US-01 (Quản lý nhà hàng), US-03 (Nhân viên lễ tân) \\
\hline
Priority & Must Have \\
\hline
Trigger & - Khách hàng liên hệ đặt bàn qua các kênh không trực tuyến (điện thoại, email). \newline - Nhân viên cần nhập một lượt đặt chỗ đặc biệt (ví dụ: cho khách VIP, cho sự kiện nội bộ). \\
\hline
Pre-Condition & - Người dùng (US-01 hoặc US-03) đã đăng nhập vào hệ thống với quyền tạo đặt chỗ. \newline - Module quản lý đặt chỗ đang hoạt động. \\
\hline
Post-Condition & - Một bản ghi đặt chỗ mới được tạo trong hệ thống với các thông tin do nhân viên nhập. \newline - Trạng thái ban đầu của đặt chỗ được thiết lập (thường là "Chờ xác nhận" hoặc "Đã xác nhận" tùy quy trình). \newline - Nếu bàn được chọn, trạng thái bàn có thể được cập nhật. \newline - Hệ thống ghi nhận hoạt động. \\
\hline
\multicolumn{2}{|c|}{\textbf{2.2. Luồng thực thi (Flow)}} \\
\hline
\textbf{Mục} & \textbf{Nội dung} \\
\hline
Basic Flow & 1. Người dùng (US-01/US-03) truy cập module quản lý Đặt chỗ và chọn hành động "Tạo mới". \newline 2. Hệ thống hiển thị form đặt chỗ trống. \newline 3. Người dùng tìm kiếm và chọn Khách hàng (nếu đã có) hoặc nhập thông tin khách hàng mới (Tên, SĐT, Email - BR-UC3.9-1). \newline 4. Người dùng chọn Ngày, Giờ đặt, và Số lượng người. \newline 5. Hệ thống (có thể) kiểm tra và hiển thị danh sách các Bàn còn trống phù hợp. Người dùng chọn một bàn (nếu muốn gán ngay). \newline 6. (Tùy chọn) Người dùng thêm các Món ăn đặt trước vào đặt chỗ (tương tự UC-MD03-03 nhưng trong giao diện backend/POS). \newline 7. (Tùy chọn) Người dùng ghi nhận thông tin về tiền Đặt cọc nếu khách đã thanh toán hoặc sẽ thanh toán qua kênh khác (ví dụ: tiền mặt, chuyển khoản sau). \newline 8. Người dùng chọn Trạng thái ban đầu cho đặt chỗ (ví dụ: "Chờ xác nhận", "Đã xác nhận"). \newline 9. Người dùng nhập Ghi chú nội bộ hoặc ghi chú của khách (nếu có). \newline 10. Người dùng chọn hành động "Lưu". \newline 11. Hệ thống kiểm tra tính hợp lệ của dữ liệu (thông tin khách, ngày giờ, bàn trống...). \newline 12. Hệ thống lưu bản ghi đặt chỗ mới. \newline 13. Hệ thống hiển thị thông báo tạo thành công. \newline 14. (Tùy chọn) Hệ thống có thể kích hoạt gửi email thông báo cho khách hàng (nếu email được cung cấp và cấu hình cho phép). \\
\hline
Alternative Flow & \textbf{5a. Không gán bàn ngay:} \newline    1. Người dùng có thể không chọn bàn cụ thể ngay lúc tạo, để bàn được gán sau hoặc khi khách đến. \\
\hline
Exception Flow & \textbf{11a. Lỗi Xác thực Dữ liệu:} \newline    1. Thiếu thông tin bắt buộc (khách, ngày giờ, số người) hoặc dữ liệu không hợp lệ (chọn bàn đã đặt). \newline    2. Hệ thống báo lỗi. Không lưu. Use Case quay lại bước nhập liệu. \newline \textbf{12a. Lỗi Hệ thống khi Lưu:} \newline    1. Hệ thống gặp lỗi kỹ thuật. \newline    2. Hệ thống báo lỗi chung. \\
\hline
\multicolumn{2}{|c|}{\textbf{2.3. Thông tin bổ sung (Additional Information)}} \\
\hline
\textbf{Mục} & \textbf{Nội dung} \\
\hline
Business Rule & - \textbf{BR-UC3.9-1 (V3):} Thông tin Khách hàng (Tên, SĐT) là bắt buộc. \newline - \textbf{BR-UC3.9-2 (V3):} Phải kiểm tra tính khả dụng của bàn/khung giờ khi tạo. \newline - \textbf{BR-UC3.9-3 (V3):} Quy trình xử lý đặt cọc cho đặt chỗ thủ công cần được định nghĩa. \\
\hline
Non-Functional Requirement & - \textbf{NFR-UC3.9-1 (V3) (Usability):} Form tạo đặt chỗ thủ công phải dễ sử dụng, cho phép nhập nhanh thông tin. \newline - \textbf{NFR-UC3.9-2 (V3) (Performance):} Kiểm tra bàn trống và lưu đặt chỗ phải nhanh. \\
\hline
\end{longtable}

\subsubsection{Use Case UC-MD03-10: Sửa Thông tin Lượt Đặt chỗ (Backend/POS)}
% (Tương ứng FR-MD03-14B)
\begin{longtable}{|m{4cm}|p{11cm}|}
\caption{Đặc tả Use Case UC-MD03-10: Sửa Thông tin Lượt Đặt chỗ (Backend/POS)} \label{tab:uc_md03_10_revised_v3} \\
\hline
\multicolumn{2}{|c|}{\textbf{2.1. Tóm tắt (Summary)}} \\
\hline
\textbf{Mục} & \textbf{Nội dung} \\
\hline
\endhead % Header cho các trang tiếp theo
\hline
\endfoot % Footer cho bảng
\hline
\endlastfoot % Footer cho trang cuối cùng
Use Case Name & Sửa Thông tin Lượt Đặt chỗ (Backend/POS) \\
\hline
Use Case ID & UC-MD03-10 \\
\hline
Use Case Description & Cho phép Nhân viên được phân quyền (Quản lý, Lễ tân) chỉnh sửa các thông tin của một lượt đặt chỗ đã tồn tại trong hệ thống, ví dụ: thay đổi ngày giờ, số lượng người, bàn được gán, danh sách món ăn đặt trước, hoặc thông tin liên hệ của khách hàng. \\
\hline
Actor & US-01 (Quản lý nhà hàng), US-03 (Nhân viên lễ tân) \\
\hline
Priority & Must Have \\
\hline
Trigger & - Khách hàng yêu cầu thay đổi thông tin đặt chỗ đã thực hiện trước đó. \newline - Nhân viên phát hiện có sai sót trong thông tin đặt chỗ cần được sửa lại. \\
\hline
Pre-Condition & - Người dùng (US-01 hoặc US-03) đã đăng nhập vào hệ thống với quyền sửa đặt chỗ. \newline - Lượt đặt chỗ cần sửa đã tồn tại trong hệ thống và đang ở trạng thái cho phép sửa đổi (ví dụ: chưa đến giờ, chưa hủy hoàn toàn). \\
\hline
Post-Condition & - Thông tin của lượt đặt chỗ được chọn đã được cập nhật trong cơ sở dữ liệu theo những thay đổi đã thực hiện. \newline - Nếu các thay đổi ảnh hưởng đến tính khả dụng (bàn, thời gian), hệ thống sẽ phản ánh điều này. \newline - (Tùy chọn) Thông báo về sự thay đổi có thể được gửi cho khách hàng. \newline - Hệ thống ghi nhận hoạt động. \\
\hline
\multicolumn{2}{|c|}{\textbf{2.2. Luồng thực thi (Flow)}} \\
\hline
\textbf{Mục} & \textbf{Nội dung} \\
\hline
Basic Flow & 1. Người dùng (US-01/US-03) tìm và mở chi tiết lượt đặt chỗ cần sửa (UC-MD03-08). \newline 2. Người dùng chọn hành động "Sửa" (Edit). \newline 3. Hệ thống cho phép chỉnh sửa các trường thông tin trên form đặt chỗ: \newline    - Thông tin khách hàng (Tên, SĐT, Email). \newline    - Ngày, Giờ đặt, Số lượng người. \newline    - Bàn được gán. \newline    - Danh sách Món ăn đặt trước (thêm/sửa/xóa món, thay đổi số lượng/biến thể). \newline    - Thông tin đặt cọc (nếu cần điều chỉnh thủ công). \newline    - Ghi chú. \newline 4. Người dùng thực hiện các thay đổi cần thiết. \newline 5. Người dùng chọn hành động "Lưu" (Save). \newline 6. Hệ thống kiểm tra tính hợp lệ của các dữ liệu đã thay đổi (ví dụ: nếu đổi giờ/bàn, kiểm tra xem còn trống không; nếu thay đổi món, tính lại tiền cọc món ăn nếu có). \newline 7. Hệ thống lưu lại các thay đổi vào bản ghi đặt chỗ. \newline 8. Hệ thống hiển thị thông báo cập nhật thành công. \newline 9. (Tùy chọn) Hệ thống có thể đề xuất hoặc tự động gửi thông báo cập nhật cho khách hàng về những thay đổi này. \\
\hline
Alternative Flow & Không có luồng thay thế đáng kể. \\
\hline
Exception Flow & \textbf{6a. Lỗi Xác thực Dữ liệu / Xung đột:} \newline    1. Hệ thống phát hiện dữ liệu sửa không hợp lệ (ví dụ: chọn bàn đã bị đặt vào giờ mới, số người vượt quá sức chứa bàn mới). \newline    2. Hệ thống báo lỗi cụ thể. \newline    3. Hệ thống không cho phép lưu. Use Case quay lại bước 4. \newline \textbf{7a. Lỗi Hệ thống khi Lưu:} \newline    1. Hệ thống gặp sự cố kỹ thuật khi lưu thay đổi. \newline    2. Hệ thống hiển thị thông báo lỗi chung. \\
\hline
\multicolumn{2}{|c|}{\textbf{2.3. Thông tin bổ sung (Additional Information)}} \\
\hline
\textbf{Mục} & \textbf{Nội dung} \\
\hline
Business Rule & - \textbf{BR-UC3.10-1 (V3):} Việc sửa đổi thông tin đặt chỗ phải tuân thủ các quy tắc về tính khả dụng của bàn/khung giờ. \newline - \textbf{BR-UC3.10-2 (V3):} Nếu thay đổi số lượng người hoặc món ăn đặt trước, tiền đặt cọc có thể cần được tính toán lại (hệ thống tự động hoặc nhân viên điều chỉnh). \newline - \textbf{BR-UC3.10-3 (V3):} Nên có chính sách rõ ràng về việc cho phép sửa đổi đặt chỗ gần giờ G và việc thông báo cho khách hàng. \\
\hline
Non-Functional Requirement & - \textbf{NFR-UC3.10-1 (V3) (Usability):} Form sửa đặt chỗ phải dễ dàng cho nhân viên cập nhật các loại thông tin khác nhau. \newline - \textbf{NFR-UC3.10-2 (V3) (Data Integrity):} Mọi thay đổi phải được lưu chính xác và đảm bảo tính nhất quán của dữ liệu đặt chỗ. \\
\hline
\end{longtable}

\subsubsection{Use Case UC-MD03-11: Xác nhận Thủ công Lượt Đặt chỗ}
% (Tương ứng FR-MD03-15A)
\begin{longtable}{|m{4cm}|p{11cm}|}
\caption{Đặc tả Use Case UC-MD03-11: Xác nhận Thủ công Lượt Đặt chỗ} \label{tab:uc_md03_11_revised_v3} \\
\hline
\multicolumn{2}{|c|}{\textbf{2.1. Tóm tắt (Summary)}} \\
\hline
\textbf{Mục} & \textbf{Nội dung} \\
\hline
\endhead % Header cho các trang tiếp theo
\hline
\endfoot % Footer cho bảng
\hline
\endlastfoot % Footer cho trang cuối cùng
Use Case Name & Xác nhận Thủ công Lượt Đặt chỗ \\
\hline
Use Case ID & UC-MD03-11 \\
\hline
Use Case Description & Cho phép Nhân viên được phân quyền (Quản lý, Lễ tân) thay đổi trạng thái của một lượt đặt chỗ từ "Chờ xác nhận" (Pending) hoặc một trạng thái tương tự sang "Đã xác nhận" (Confirmed), thường áp dụng cho các đặt chỗ được tạo thủ công không qua thanh toán cọc online hoặc các trường hợp đặc biệt. \\
\hline
Actor & US-01 (Quản lý nhà hàng), US-03 (Nhân viên lễ tân) \\
\hline
Priority & Must Have \\
\hline
Trigger & - Một lượt đặt chỗ được tạo thủ công (UC-MD03-09) đang ở trạng thái "Chờ xác nhận" và nhân viên muốn chính thức hóa nó. \newline - Một đặt chỗ online vì lý do nào đó chưa được tự động xác nhận và cần nhân viên can thiệp. \\
\hline
Pre-Condition & - Người dùng (US-01 hoặc US-03) đã đăng nhập với quyền quản lý trạng thái đặt chỗ. \newline - Lượt đặt chỗ cần xác nhận đang ở trạng thái cho phép chuyển sang "Đã xác nhận". \\
\hline
Post-Condition & - Trạng thái của lượt đặt chỗ được cập nhật thành "Đã xác nhận" (Confirmed). \newline - Bàn liên kết (nếu có) được chính thức giữ cho đặt chỗ này. \newline - (Tùy chọn) Email/SMS xác nhận được gửi cho khách hàng. \\
\hline
\multicolumn{2}{|c|}{\textbf{2.2. Luồng thực thi (Flow)}} \\
\hline
\textbf{Mục} & \textbf{Nội dung} \\
\hline
Basic Flow & 1. Người dùng (US-01/US-03) tìm và mở chi tiết lượt đặt chỗ cần xác nhận (UC-MD03-08). \newline 2. Người dùng xem xét thông tin đặt chỗ và đảm bảo mọi thứ hợp lệ. \newline 3. Người dùng tìm và nhấp vào nút/hành động "Xác nhận" (Confirm) trên form đặt chỗ. \newline 4. Hệ thống cập nhật trạng thái của bản ghi đặt chỗ thành "Confirmed". \newline 5. Nếu bàn chưa được giữ chính thức, hệ thống cập nhật trạng thái bàn tương ứng. \newline 6. (Tùy chọn) Hệ thống kích hoạt gửi email/SMS xác nhận cho khách hàng (nếu chưa gửi trước đó hoặc cần gửi lại). \newline 7. Hệ thống hiển thị thông báo "Đặt chỗ đã được xác nhận thành công." \\
\hline
Alternative Flow & \textbf{3a. Xác nhận từ List View:} \newline    1. Nhân viên có thể chọn một hoặc nhiều đặt chỗ "Chờ xác nhận" từ danh sách (UC-MD03-07) và chọn hành động "Xác nhận" hàng loạt. \\
\hline
Exception Flow & \textbf{4a. Lỗi cập nhật trạng thái/bàn:} \newline    1. Hệ thống gặp lỗi kỹ thuật khi lưu trạng thái mới hoặc cập nhật thông tin bàn. \newline    2. Hệ thống báo lỗi. Trạng thái có thể chưa được cập nhật. \\
\hline
\multicolumn{2}{|c|}{\textbf{2.3. Thông tin bổ sung (Additional Information)}} \\
\hline
\textbf{Mục} & \textbf{Nội dung} \\
\hline
Business Rule & - \textbf{BR-UC3.11-1 (V3):} Chỉ những đặt chỗ ở trạng thái phù hợp (ví dụ: "Pending", "Draft") mới có thể được xác nhận thủ công. \newline - \textbf{BR-UC3.11-2 (V3):} Việc xác nhận thủ công đồng nghĩa với việc nhà hàng cam kết giữ chỗ cho khách. \\
\hline
Non-Functional Requirement & - \textbf{NFR-UC3.11-1 (V3) (Usability):} Nút xác nhận phải rõ ràng. \newline - \textbf{NFR-UC3.11-2 (V3) (Performance):} Cập nhật trạng thái nhanh chóng. \\
\hline
\end{longtable}

\subsubsection{Use Case UC-MD03-12: Hủy bỏ Lượt Đặt chỗ (Backend/POS)}
% (Tương ứng FR-MD03-15B)
\begin{longtable}{|m{4cm}|p{11cm}|}
\caption{Đặc tả Use Case UC-MD03-12: Hủy bỏ Lượt Đặt chỗ (Backend/POS)} \label{tab:uc_md03_12_revised_v3} \\
\hline
\multicolumn{2}{|c|}{\textbf{2.1. Tóm tắt (Summary)}} \\
\hline
\textbf{Mục} & \textbf{Nội dung} \\
\hline
\endhead % Header cho các trang tiếp theo
\hline
\endfoot % Footer cho bảng
\hline
\endlastfoot % Footer cho trang cuối cùng
Use Case Name & Hủy bỏ Lượt Đặt chỗ (Backend/POS) \\
\hline
Use Case ID & UC-MD03-12 \\
\hline
Use Case Description & Cho phép Nhân viên được phân quyền (Quản lý, Lễ tân) hủy bỏ một lượt đặt chỗ đã tồn tại trong hệ thống, ví dụ khi khách hàng yêu cầu hủy qua điện thoại hoặc nhà hàng không thể đáp ứng đặt chỗ đó. \\
\hline
Actor & US-01 (Quản lý nhà hàng), US-03 (Nhân viên lễ tân) \\
\hline
Priority & Must Have \\
\hline
Trigger & - Khách hàng liên hệ yêu cầu hủy đặt chỗ. \newline - Nhà hàng cần hủy đặt chỗ do lý do bất khả kháng hoặc thay đổi kế hoạch. \\
\hline
Pre-Condition & - Người dùng (US-01 hoặc US-03) đã đăng nhập với quyền quản lý/hủy đặt chỗ. \newline - Lượt đặt chỗ cần hủy đang ở trạng thái cho phép hủy (ví dụ: "Confirmed", "Pending"). \\
\hline
Post-Condition & - Trạng thái của lượt đặt chỗ được cập nhật thành "Đã hủy" (Cancelled). \newline - Bàn đã được giữ cho đặt chỗ này (nếu có) được giải phóng. \newline - Tiền đặt cọc (nếu có) được xử lý theo chính sách (hoàn lại hoặc không). \newline - (Tùy chọn) Thông báo hủy được gửi cho khách hàng. \\
\hline
\multicolumn{2}{|c|}{\textbf{2.2. Luồng thực thi (Flow)}} \\
\hline
\textbf{Mục} & \textbf{Nội dung} \\
\hline
Basic Flow & 1. Người dùng (US-01/US-03) tìm và mở chi tiết lượt đặt chỗ cần hủy (UC-MD03-08). \newline 2. Người dùng chọn hành động "Hủy Đặt chỗ" (Cancel Booking). \newline 3. Hệ thống yêu cầu xác nhận việc hủy. \newline 4. (Tùy chọn) Hệ thống có thể yêu cầu nhập Lý do hủy. Người dùng nhập lý do. \newline 5. Người dùng xác nhận muốn hủy. \newline 6. Hệ thống cập nhật trạng thái đặt chỗ thành "Cancelled". \newline 7. Hệ thống giải phóng bàn đã được liên kết với đặt chỗ này. \newline 8. Hệ thống xử lý Tiền đặt cọc theo chính sách đã cấu hình (BR-UC3.12-1): \newline    a. Nếu đủ điều kiện hoàn cọc: Hệ thống có thể tạo một yêu cầu hoàn tiền hoặc nhân viên cần thực hiện quy trình hoàn tiền thủ công. \newline    b. Nếu không hoàn cọc: Tiền cọc được ghi nhận là doanh thu hoặc theo quy định kế toán. \newline 9. (Tùy chọn) Hệ thống gửi email/SMS thông báo hủy đặt chỗ cho khách hàng, có thể kèm lý do (nếu nhà hàng hủy) và thông tin về việc xử lý cọc. \newline 10. Hệ thống hiển thị thông báo "Đặt chỗ đã được hủy thành công." \\
\hline
Alternative Flow & \textbf{2a. Hủy từ List View:} Nhân viên có thể chọn hủy từ danh sách đặt chỗ. \\
\hline
Exception Flow & \textbf{5a. Không thể hủy do trạng thái không phù hợp:} \newline    1. Đặt chỗ đã ở trạng thái "Arrived" hoặc "Completed". \newline    2. Hệ thống báo lỗi "Không thể hủy đặt chỗ ở trạng thái này." \newline \textbf{6a. Lỗi cập nhật trạng thái/giải phóng bàn/xử lý cọc:} \newline    1. Hệ thống gặp lỗi kỹ thuật. \newline    2. Hệ thống báo lỗi. Việc hủy có thể chưa hoàn tất đúng cách. \\
\hline
\multicolumn{2}{|c|}{\textbf{2.3. Thông tin bổ sung (Additional Information)}} \\
\hline
\textbf{Mục} & \textbf{Nội dung} \\
\hline
Business Rule & - \textbf{BR-UC3.12-1 (V3):} Chính sách hoàn/mất tiền đặt cọc khi hủy phải được áp dụng đúng. \newline - \textbf{BR-UC3.12-2 (V3):} Việc giải phóng bàn là bắt buộc. \newline - \textbf{BR-UC3.12-3 (V3):} Nên ghi nhận lý do hủy để phục vụ báo cáo và cải thiện dịch vụ. \\
\hline
Non-Functional Requirement & - \textbf{NFR-UC3.12-1 (V3) (Usability):} Thao tác hủy phải rõ ràng. \newline - \textbf{NFR-UC3.12-2 (V3) (Reliability):} Xử lý hủy và cọc phải đáng tin cậy. \\
\hline
\end{longtable}

\subsubsection{Use Case UC-MD03-13: Đánh dấu Khách đã đến (Check-in) cho Lượt Đặt chỗ}
% (Tương ứng FR-MD03-15C)
\begin{longtable}{|m{4cm}|p{11cm}|}
\caption{Đặc tả Use Case UC-MD03-13: Đánh dấu Khách đã đến (Check-in) cho Lượt Đặt chỗ} \label{tab:uc_md03_13_revised_v3} \\
\hline
\multicolumn{2}{|c|}{\textbf{2.1. Tóm tắt (Summary)}} \\
\hline
\textbf{Mục} & \textbf{Nội dung} \\
\hline
\endhead % Header cho các trang tiếp theo
\hline
\endfoot % Footer cho bảng
\hline
\endlastfoot % Footer cho trang cuối cùng
Use Case Name & Đánh dấu Khách đã đến (Check-in) cho Lượt Đặt chỗ \\
\hline
Use Case ID & UC-MD03-13 \\
\hline
Use Case Description & Cho phép Nhân viên (Lễ tân, Phục vụ) đánh dấu trong hệ thống rằng khách hàng có đặt chỗ đã đến nhà hàng và nhận bàn. \\
\hline
Actor & US-03 (Nhân viên lễ tân), US-01 (Quản lý nhà hàng), US-02 (Nhân viên phục vụ - nếu có quyền) \\
\hline
Priority & Must Have \\
\hline
Trigger & Khách hàng có đặt chỗ đến nhà hàng vào đúng hoặc gần giờ đã đặt. \\
\hline
Pre-Condition & - Người dùng đã đăng nhập vào hệ thống. \newline - Lượt đặt chỗ của khách đang ở trạng thái "Đã xác nhận" (Confirmed). \newline - Nhân viên đã xác minh đúng là khách hàng của lượt đặt chỗ đó. \\
\hline
Post-Condition & - Trạng thái của lượt đặt chỗ được cập nhật thành "Đã đến" (Arrived / Seated). \newline - Bàn được gán cho đặt chỗ đó trên POS được chính thức chuyển sang trạng thái "Đang có khách" (Occupied) nếu chưa phải. \newline - Đơn hàng POS có thể được tự động mở cho bàn đó (liên kết với UC-MD05-03). \\
\hline
\multicolumn{2}{|c|}{\textbf{2.2. Luồng thực thi (Flow)}} \\
\hline
\textbf{Mục} & \textbf{Nội dung} \\
\hline
Basic Flow & 1. Khách hàng đến, thông báo có đặt chỗ. \newline 2. Nhân viên (US-03/US-01/US-02) tìm lượt đặt chỗ của khách trong danh sách (UC-MD03-07) hoặc trên Sơ đồ tầng POS (nếu bàn đã được gán và đánh dấu Reserved). \newline 3. Nhân viên mở chi tiết lượt đặt chỗ (UC-MD03-08). \newline 4. Nhân viên chọn hành động "Đánh dấu Đã đến" / "Check-in" / "Seat Customer". \newline 5. Hệ thống cập nhật trạng thái đặt chỗ thành "Arrived". \newline 6. Nếu đặt chỗ đã được gán bàn, hệ thống đảm bảo trạng thái bàn đó trên POS là "Occupied". Nếu chưa gán bàn, nhân viên có thể cần thực hiện gán bàn ngay lúc này (có thể là một phần của hành động check-in). \newline 7. Hệ thống hiển thị thông báo thành công. \newline 8. Nếu thao tác từ POS, hệ thống có thể tự động mở màn hình đơn hàng cho bàn đó (UC-MD05-03). \\
\hline
Alternative Flow & \textbf{4a. Check-in từ Sơ đồ tầng POS:} \newline    1. Nếu bàn đã được gán cho đặt chỗ và hiển thị là "Reserved" trên POS. \newline    2. Nhân viên nhấp vào bàn đó trên POS. \newline    3. Hệ thống có thể hiển thị thông tin đặt chỗ và có nút "Check-in" / "Seat". \newline    4. Nhân viên nhấn nút đó. Use Case tiếp tục từ bước 5. \\
\hline
Exception Flow & \textbf{6a. Lỗi cập nhật trạng thái đặt chỗ/bàn:} \newline    1. Hệ thống gặp lỗi kỹ thuật. \newline    2. Hệ thống báo lỗi. \\
\hline
\multicolumn{2}{|c|}{\textbf{2.3. Thông tin bổ sung (Additional Information)}} \\
\hline
\textbf{Mục} & \textbf{Nội dung} \\
\hline
Business Rule & - \textbf{BR-UC3.13-1 (V3):} Chỉ những đặt chỗ "Confirmed" mới có thể được check-in. \newline - \textbf{BR-UC3.13-2 (V3):} Việc check-in là cơ sở để theo dõi tình trạng no-show và quản lý bàn hiệu quả. \\
\hline
Non-Functional Requirement & - \textbf{NFR-UC3.13-1 (V3) (Usability):} Thao tác check-in phải nhanh chóng, dễ thực hiện. \newline - \textbf{NFR-UC3.13-2 (V3) (Integration):} Trạng thái check-in phải được đồng bộ giữa module Đặt chỗ và POS (nếu là module riêng). \\
\hline
\end{longtable}

\subsubsection{Use Case UC-MD03-14: Xem Báo cáo Tổng hợp Món ăn Cần chuẩn bị (Đặt trước)}
% (Giữ nguyên nội dung chi tiết như UC-MD03-16 version cũ, chỉ đổi ID)
\begin{longtable}{|m{4cm}|p{11cm}|}
\caption{Đặc tả Use Case UC-MD03-14: Xem Báo cáo Tổng hợp Món ăn Cần chuẩn bị (Đặt trước)} \label{tab:uc_md03_14_revised_v3} \\
\hline
\multicolumn{2}{|c|}{\textbf{2.1. Tóm tắt (Summary)}} \\
\hline
\textbf{Mục} & \textbf{Nội dung} \\
\hline
\endhead % Header cho các trang tiếp theo
\hline
\endfoot % Footer cho bảng
\hline
\endlastfoot % Footer cho trang cuối cùng
Use Case Name & Xem Báo cáo Tổng hợp Món ăn Cần chuẩn bị (Đặt trước) \\
\hline
Use Case ID & UC-MD03-14 \\
\hline
Use Case Description & Cung cấp cho bộ phận Bếp hoặc Quản lý một giao diện/báo cáo tổng hợp danh sách các món ăn và đồ uống đã được khách hàng đặt trước cho các lượt đặt chỗ sắp tới, giúp chuẩn bị nguyên liệu và lên kế hoạch chế biến hiệu quả. \\
\hline
Actor & US-04 (Nhân viên bếp), US-01 (Quản lý nhà hàng) \\
\hline
Priority & Should Have (Quan trọng nếu đặt món trước là phổ biến) \\
\hline
Trigger & Bộ phận bếp/quản lý cần biết trước các món ăn cần chuẩn bị cho các khách hàng đã đặt chỗ và đặt món trước. \\
\hline
Pre-Condition & - Người dùng (US-04 hoặc US-01) đã đăng nhập vào hệ thống với quyền truy cập báo cáo/danh sách món đặt trước. \newline - Có ít nhất một lượt đặt chỗ đã xác nhận và có chứa thông tin món ăn đặt trước. \\
\hline
Post-Condition & - Danh sách tổng hợp các món ăn cần chuẩn bị (tên món, biến thể, số lượng) cho một khoảng thời gian hoặc một ca làm việc cụ thể được hiển thị. \newline - Bộ phận bếp/quản lý có thông tin để chuẩn bị. \\
\hline
\multicolumn{2}{|c|}{\textbf{2.2. Luồng thực thi (Flow)}} \\
\hline
\textbf{Mục} & \textbf{Nội dung} \\
\hline
Basic Flow & 1. Người dùng (US-04/US-01) truy cập vào chức năng/báo cáo "Món ăn Đặt trước" (Pre-ordered Items Report/List). \newline 2. Hệ thống mặc định hiển thị danh sách các món ăn đã được đặt trước cho ngày hiện tại hoặc ca làm việc hiện tại. \newline 3. Danh sách này thường được tổng hợp theo từng món ăn/biến thể, hiển thị: \newline    - Tên món ăn / Biến thể. \newline    - Tổng số lượng cần chuẩn bị. \newline    - (Tùy chọn) Danh sách các lượt đặt chỗ liên quan đến món đó (Mã đặt chỗ, Giờ đến, Bàn). \newline    - (Tùy chọn) Ghi chú đặc biệt liên quan đến món ăn từ các lượt đặt chỗ. \newline 4. Người dùng xem xét danh sách để lên kế hoạch chuẩn bị. \\
\hline
Alternative Flow & \textbf{2a. Lọc theo khoảng thời gian:} \newline    1. Người dùng chọn một khoảng thời gian khác (ví dụ: ngày mai, tuần tới) hoặc một ca làm việc cụ thể. \newline    2. Hệ thống lọc và hiển thị lại danh sách món đặt trước cho khoảng thời gian đã chọn. \newline    3. Use Case quay lại bước 4. \newline \textbf{2b. Lọc theo trạng thái đặt chỗ:} \newline    1. Người dùng chỉ muốn xem món của các đặt chỗ "Đã xác nhận". \newline    2. Hệ thống áp dụng bộ lọc trạng thái. \newline    3. Use Case quay lại bước 4. \newline \textbf{3a. Xem chi tiết theo từng đặt chỗ:} \newline    1. Thay vì tổng hợp, giao diện hiển thị danh sách các lượt đặt chỗ sắp tới, và người dùng có thể nhấp vào từng lượt để xem danh sách món đặt trước của riêng lượt đó. \\
\hline
Exception Flow & \textbf{2a. Lỗi tải dữ liệu:} \newline    1. Hệ thống gặp lỗi khi truy vấn và tổng hợp dữ liệu món ăn đặt trước. \newline    2. Hệ thống hiển thị thông báo lỗi. \newline    3. Use Case kết thúc không thành công. \newline \textbf{2b. Không có món nào đặt trước:} \newline    1. Không có lượt đặt chỗ nào trong khoảng thời gian/bộ lọc có món đặt trước. \newline    2. Hệ thống hiển thị danh sách trống hoặc thông báo "Không có món ăn nào được đặt trước". \\
\hline
\multicolumn{2}{|c|}{\textbf{2.3. Thông tin bổ sung (Additional Information)}} \\
\hline
\textbf{Mục} & \textbf{Nội dung} \\
\hline
Business Rule & - \textbf{BR-UC3.14-1 (V3):} Danh sách chỉ nên bao gồm các món từ những lượt đặt chỗ đã được xác nhận (Confirmed) và chưa bị hủy (Not Cancelled). \newline - \textbf{BR-UC3.14-2 (V3):} Số lượng hiển thị phải là tổng số lượng của món ăn/biến thể đó từ tất cả các lượt đặt chỗ hợp lệ trong khoảng thời gian/bộ lọc được chọn. \newline - \textbf{BR-UC3.14-3 (V3):} Giao diện/báo cáo này nên dễ dàng truy cập đối với bộ phận bếp. Có thể cần in ra được. \\
\hline
Non-Functional Requirement & - \textbf{NFR-UC3.14-1 (V3) (Usability):} Giao diện/báo cáo phải rõ ràng, dễ đọc, dễ hiểu cho nhân viên bếp. Việc lọc theo thời gian/ca làm việc phải đơn giản. \newline - \textbf{NFR-UC3.14-2 (V3) (Performance):} Thời gian tải và tổng hợp danh sách món đặt trước cho một ngày phải nhanh chóng (dưới 5 giây). \newline - \textbf{NFR-UC3.14-3 (V3) (Accuracy):} Dữ liệu về tên món, biến thể, số lượng phải chính xác 100\%. \newline - \textbf{NFR-UC3.14-4 (V3) (Accessibility):} Nếu cần hiển thị trên màn hình trong bếp, giao diện cần có font chữ lớn, độ tương phản cao. \\
\hline
\end{longtable}

\subsubsection{Use Case UC-MD03-15: Cấu hình Quy tắc và Tham số Đặt chỗ}
% (Giữ nguyên nội dung chi tiết như UC-MD03-11 version cũ, chỉ đổi ID)
\begin{longtable}{|m{4cm}|p{11cm}|}
\caption{Đặc tả Use Case UC-MD03-15: Cấu hình Quy tắc và Tham số Đặt chỗ} \label{tab:uc_md03_15_revised_v3} \\
\hline
\multicolumn{2}{|c|}{\textbf{2.1. Tóm tắt (Summary)}} \\
\hline
\textbf{Mục} & \textbf{Nội dung} \\
\hline
\endhead % Header cho các trang tiếp theo
\hline
\endfoot % Footer cho bảng
\hline
\endlastfoot % Footer cho trang cuối cùng
Use Case Name & Cấu hình Quy tắc và Tham số Đặt chỗ \\
\hline
Use Case ID & UC-MD03-15 \\
\hline
Use Case Description & Cho phép Quản lý nhà hàng hoặc Quản trị viên hệ thống thiết lập các quy tắc và tham số vận hành cho chức năng đặt chỗ online và quản lý đặt chỗ nói chung. Bao gồm giờ hoạt động, khoảng thời gian đặt, giới hạn số khách, quy tắc đặt cọc, giá trị bàn, và các tùy chọn khác. \\
\hline
Actor & US-01 (Quản lý nhà hàng), US-10 (Quản trị viên Hệ thống) \\
\hline
Priority & Must Have \\
\hline
Trigger & Cần thiết lập ban đầu cho chức năng đặt chỗ hoặc cần thay đổi các quy tắc vận hành hiện tại. \\
\hline
Pre-Condition & - Người dùng (US-01 hoặc US-10) đã đăng nhập với quyền quản trị cấu hình module Đặt chỗ (Booking/Reservation) hoặc cấu hình Website/POS liên quan. \newline - Module Đặt chỗ (hoặc tương đương) đã được cài đặt. \\
\hline
Post-Condition & - Các quy tắc và tham số đặt chỗ được cập nhật trong cấu hình hệ thống. \newline - Chức năng đặt chỗ online (cho khách hàng) và quản lý đặt chỗ (cho nhân viên) sẽ hoạt động theo các quy tắc mới được thiết lập. \\
\hline
\multicolumn{2}{|c|}{\textbf{2.2. Luồng thực thi (Flow)}} \\
\hline
\textbf{Mục} & \textbf{Nội dung} \\
\hline
Basic Flow & 1. Người dùng (US-01/US-10) truy cập vào khu vực cấu hình của module Đặt chỗ (ví dụ: Reservations > Configuration > Settings). \newline 2. Hệ thống hiển thị giao diện cấu hình với nhiều tùy chọn được nhóm lại. \newline 3. Người dùng tìm đến các mục cấu hình cần thiết và thay đổi giá trị: \newline    - \textbf{Giờ hoạt động \& Khung giờ đặt chỗ:} Thiết lập giờ mở cửa, giờ đóng cửa cho phép đặt bàn, khoảng cách giữa các slot đặt (ví dụ: 15 phút), thời lượng mặc định của một lượt đặt. \newline    - \textbf{Giới hạn đặt chỗ:} Số ngày tối thiểu/tối đa cho phép đặt trước, số lượng khách tối thiểu/tối đa cho mỗi lượt đặt online. \newline    - \textbf{Quy tắc Đặt cọc:} Kích hoạt/Tắt yêu cầu đặt cọc, nhập Tỷ lệ phần trăm đặt cọc cho bàn, Tỷ lệ phần trăm đặt cọc cho món ăn. \newline    - \textbf{Giá trị Bàn:} Truy cập một khu vực riêng (ví dụ: quản lý tài nguyên bàn) để nhập giá trị tham chiếu cho từng bàn hoặc loại bàn (dùng để tính cọc bàn). \newline    - \textbf{Cho phép chọn bàn:} Kích hoạt/Tắt tùy chọn cho phép khách hàng tự chọn bàn cụ thể trên sơ đồ tầng khi đặt online. \newline    - \textbf{Thông báo \& Email Template:} Cấu hình nội dung các email/SMS xác nhận, nhắc nhở, hủy bỏ. \newline    - \textbf{Tích hợp Thanh toán:} Chọn và cấu hình cổng thanh toán sẽ sử dụng cho việc đặt cọc. \newline 4. Sau khi thực hiện các thay đổi mong muốn, Người dùng chọn hành động "Lưu" (Save). \newline 5. Hệ thống kiểm tra tính hợp lệ của các giá trị nhập vào (ví dụ: tỷ lệ phần trăm hợp lệ, giờ hợp lệ). \newline 6. Hệ thống lưu lại các cấu hình mới. \newline 7. Hệ thống hiển thị thông báo lưu thành công. \\
\hline
Alternative Flow & \textbf{3a. Cấu hình theo từng Điểm bán hàng (POS):} \newline    1. Nếu hệ thống hỗ trợ nhiều điểm bán hàng/nhà hàng, một số cấu hình (ví dụ: giờ hoạt động, giá bàn) có thể cần được thiết lập riêng cho từng điểm. Người dùng cần chọn đúng điểm bán hàng trước khi cấu hình. \\
\hline
Exception Flow & \textbf{5a. Lỗi Xác thực Dữ liệu:} \newline    1. Người dùng nhập giá trị không hợp lệ (ví dụ: tỷ lệ phần trăm > 100, giờ kết thúc trước giờ bắt đầu). \newline    2. Hệ thống báo lỗi, chỉ rõ trường bị sai. \newline    3. Hệ thống không lưu cấu hình. Use Case quay lại bước 3. \newline \textbf{6a. Lỗi Hệ thống khi Lưu:} \newline    1. Hệ thống gặp sự cố kỹ thuật khi cố gắng lưu cấu hình. \newline    2. Hệ thống hiển thị thông báo lỗi chung. \newline    3. Use Case kết thúc không thành công. \\
\hline
\multicolumn{2}{|c|}{\textbf{2.3. Thông tin bổ sung (Additional Information)}} \\
\hline
\textbf{Mục} & \textbf{Nội dung} \\
\hline
Business Rule & - \textbf{BR-UC3.15-1 (V3):} Các tham số cấu hình này ảnh hưởng trực tiếp đến luồng đặt chỗ của khách hàng và cách hệ thống quản lý đặt chỗ. \newline - \textbf{BR-UC3.15-2 (V3):} Giá trị bàn (Table Price) là giá tham chiếu để tính tiền cọc bàn, không nhất thiết là giá thuê bàn thực tế. Cần có cơ chế nhập giá trị này cho từng bàn hoặc loại bàn. \newline - \textbf{BR-UC3.15-3 (V3):} Việc thay đổi các cấu hình này (ví dụ: giờ hoạt động, tỷ lệ cọc) sẽ có hiệu lực cho các lượt đặt chỗ mới sau khi lưu. Các lượt đặt chỗ cũ không bị ảnh hưởng (trừ khi có cơ chế cập nhật lại). \\
\hline
Non-Functional Requirement & - \textbf{NFR-UC3.15-1 (V3) (Usability):} Giao diện cấu hình phải được tổ chức logic, dễ tìm các tùy chọn. Các thuật ngữ sử dụng phải rõ ràng. Nên có giải thích ngắn (tooltip) cho các tùy chọn phức tạp. \newline - \textbf{NFR-UC3.15-2 (V3) (Flexibility):} Hệ thống nên cung cấp đủ các tham số cấu hình cần thiết để đáp ứng các quy tắc kinh doanh phổ biến của nhà hàng về đặt chỗ. \newline - \textbf{NFR-UC3.15-3 (V3) (Security):} Chỉ những người dùng có quyền hạn cao (Quản lý, Admin) mới được phép thay đổi các cấu hình quan trọng này. \\
\hline
\end{longtable}



\subsection{Module MD-04: Xác nhận Tự động qua Bot}

\subsubsection{Use Case UC-MD04-01: Lên lịch và Kích hoạt Cuộc gọi Xác nhận}

\begin{longtable}{|m{4cm}|p{11cm}|}
\caption{Đặc tả Use Case UC-MD04-01: Lên lịch và Kích hoạt Cuộc gọi Xác nhận} \label{tab:uc_md04_01} \\
\hline

\endhead % Header cho các trang tiếp theo
\hline
\endfoot % Footer cho bảng
\hline
\endlastfoot % Footer cho trang cuối cùng
\multicolumn{2}{|c|}{\textbf{2.1. Tóm tắt (Summary)}} \\
\hline
\textbf{Mục} & \textbf{Nội dung} \\
\hline
Use Case Name & Lên lịch và Kích hoạt Cuộc gọi Xác nhận \\
\hline
Use Case ID & UC-MD04-01 \\
\hline
Use Case Description & Hệ thống tự động quét các lượt đặt chỗ có trạng thái "Đã xác nhận" và sắp đến ngày diễn ra, sau đó lên lịch và gửi yêu cầu thực hiện cuộc gọi xác nhận đến dịch vụ Bot Call bên ngoài vào thời điểm N ngày trước ngày đặt (N được cấu hình). \\
\hline
Actor & System (Scheduler, Backend Logic) \\
\hline
Priority & Must Have \\
\hline
Trigger & Một tác vụ tự động (Scheduled Action/Cron Job) trong  chạy định kỳ (ví dụ: hàng ngày vào một giờ cố định). \\
\hline
Pre-Condition & - Tác vụ tự động đã được kích hoạt và cấu hình tần suất chạy. \newline - Có các lượt đặt chỗ ở trạng thái "Đã xác nhận" (Confirmed). \newline - Tham số N (số ngày gọi trước) đã được cấu hình (FR-MD04-05). \newline - Thông tin tích hợp với dịch vụ Bot Call (API endpoint, credentials) đã được cấu hình (FR-MD04-05). \newline - Các đặt chỗ có đủ thông tin cần thiết (Số điện thoại khách hàng, Mã đặt chỗ). \\
\hline
Post-Condition & - Hệ thống xác định được danh sách các đặt chỗ cần gọi xác nhận trong ngày. \newline - Đối với mỗi đặt chỗ trong danh sách, hệ thống gửi thành công một yêu cầu (API call) đến dịch vụ Bot Call bên ngoài, bao gồm thông tin cần thiết (SĐT khách, Mã đặt chỗ, kịch bản gọi). \newline - Trạng thái của đặt chỗ có thể được cập nhật (ví dụ: "Đang chờ gọi xác nhận") hoặc một bản ghi yêu cầu gọi được tạo ra để theo dõi. \\
\hline
\multicolumn{2}{|c|}{\textbf{2.2. Luồng thực thi (Flow)}} \\
\hline
\textbf{Mục} & \textbf{Nội dung} \\
\hline
Basic Flow & 1. Tác vụ tự động trong hệ thống được kích hoạt theo lịch trình. \newline 2. Hệ thống xác định ngày mục tiêu để gọi xác nhận (Ngày Hiện Tại + N ngày, với N là số ngày gọi trước cấu hình). \newline 3. Hệ thống truy vấn cơ sở dữ liệu để tìm tất cả các lượt đặt chỗ thỏa mãn các điều kiện: \newline    - Có ngày đặt bàn (booking date) bằng Ngày Mục Tiêu. \newline    - Có trạng thái là "Đã xác nhận" (Confirmed). \newline    - Chưa được thực hiện gọi xác nhận (hoặc chưa có kết quả gọi). \newline    - Có số điện thoại khách hàng hợp lệ. \newline 4. Đối với mỗi lượt đặt chỗ tìm thấy: \newline    a. Hệ thống chuẩn bị dữ liệu cần thiết cho cuộc gọi: Số điện thoại khách hàng, Mã đặt chỗ, ID kịch bản thoại (đã cấu hình ở FR-MD04-05). \newline    b. Hệ thống thực hiện một lời gọi API đến dịch vụ Bot Call bên ngoài, truyền các dữ liệu đã chuẩn bị. \newline    c. Hệ thống nhận phản hồi từ API của Bot Call (ví dụ: xác nhận đã nhận yêu cầu thành công và trả về một Call ID của dịch vụ Bot Call). \newline    d. Hệ thống cập nhật trạng thái đặt chỗ (ví dụ: thêm cờ "Yêu cầu gọi đã gửi") hoặc lưu lại Call ID nhận được từ Bot Call vào bản ghi đặt chỗ hoặc một bảng log riêng để theo dõi kết quả sau này (FR-MD04-04). \newline 5. Tác vụ tự động kết thúc sau khi xử lý hết danh sách. \\
\hline
Alternative Flow & \textbf{3a. Không tìm thấy đặt chỗ nào cần gọi:} \newline    1. Hệ thống không tìm thấy lượt đặt chỗ nào thỏa mãn điều kiện ở bước 3. \newline    2. Tác vụ tự động kết thúc mà không thực hiện hành động nào khác. \\
\hline
Exception Flow & \textbf{3b. Lỗi truy vấn cơ sở dữ liệu:} \newline    1. Hệ thống gặp lỗi khi truy vấn danh sách đặt chỗ. \newline    2. Tác vụ tự động ghi nhận lỗi và kết thúc. Các cuộc gọi không được kích hoạt. Cần có cơ chế thông báo lỗi cho quản trị viên. \newline \textbf{4e. Lỗi gọi API đến dịch vụ Bot Call:} \newline    1. Hệ thống không thể kết nối đến dịch vụ Bot Call (lỗi mạng, sai endpoint) hoặc nhận được phản hồi lỗi từ API (sai credentials, sai định dạng dữ liệu, hết hạn mức dịch vụ...). \newline    2. Hệ thống ghi nhận lỗi chi tiết (bao gồm mã lỗi từ Bot Call nếu có) vào log hệ thống hoặc thông tin đặt chỗ. \newline    3. Hệ thống có thể thử lại việc gọi API sau (nếu cấu hình) hoặc bỏ qua đặt chỗ đó và tiếp tục với đặt chỗ tiếp theo. Cần có cơ chế thông báo lỗi cho quản trị viên. \newline \textbf{4f. Lỗi cập nhật trạng thái/lưu Call ID:} \newline    1. Hệ thống gặp lỗi khi cố gắng cập nhật trạng thái hoặc lưu Call ID vào cơ sở dữ liệu hệ thống sau khi gọi API thành công. \newline    2. Hệ thống ghi nhận lỗi. Có thể dẫn đến việc gọi lại không cần thiết vào lần chạy sau. \\
\hline
\multicolumn{2}{|c|}{\textbf{2.3. Thông tin bổ sung (Additional Information)}} \\
\hline
\textbf{Mục} & \textbf{Nội dung} \\
\hline
Business Rule & - \textbf{BR-UC4.1-1:} Hệ thống chỉ được kích hoạt cuộc gọi xác nhận cho các đặt chỗ ở trạng thái "Đã xác nhận" (Confirmed). \newline - \textbf{BR-UC4.1-2:} Cuộc gọi phải được kích hoạt đúng N ngày trước ngày khách đặt bàn, với N là số ngày cấu hình được (FR-MD04-05). \newline - \textbf{BR-UC4.1-3:} Hệ thống phải đảm bảo không gửi yêu cầu gọi trùng lặp cho cùng một đặt chỗ nếu yêu cầu trước đó đã được gửi thành công. \newline - \textbf{BR-UC4.1-4:} Cần có cơ chế xử lý lỗi khi giao tiếp với dịch vụ Bot Call để tránh bỏ sót việc xác nhận hoặc gây lỗi hệ thống. \\
\hline
Non-Functional Requirement & - \textbf{NFR-UC4.1-1 (Reliability):} Tác vụ tự động phải chạy ổn định theo lịch trình. Quá trình gửi yêu cầu API phải đáng tin cậy, có xử lý lỗi và thử lại (retry mechanism) nếu cần thiết và hợp lý. \newline - \textbf{NFR-UC4.1-2 (Performance):} Tác vụ tự động phải xử lý hiệu quả danh sách đặt chỗ, tránh làm quá tải hệ thống, đặc biệt nếu số lượng đặt chỗ lớn. Việc gọi API nên được thực hiện một cách tối ưu (ví dụ: bất đồng bộ nếu có thể). \newline - \textbf{NFR-UC4.1-3 (Scalability):} Giải pháp phải có khả năng xử lý số lượng lớn yêu cầu gọi khi nhà hàng phát triển. \newline - \textbf{NFR-UC4.1-4 (Monitoring):} Cần có cơ chế giám sát hoạt động của tác vụ tự động và kết quả việc gửi yêu cầu API (ví dụ: thông qua logs, dashboard) để quản trị viên có thể theo dõi và xử lý sự cố. \\
\hline
\end{longtable}

\subsubsection{Use Case UC-MD04-02: Thực hiện Cuộc gọi và Tương tác Khách hàng}

\begin{longtable}{|m{4cm}|p{11cm}|}
\caption{Đặc tả Use Case UC-MD04-02: Thực hiện Cuộc gọi và Tương tác Khách hàng} \label{tab:uc_md04_02} \\
\hline

\endhead % Header cho các trang tiếp theo
\hline
\endfoot % Footer cho bảng
\hline
\endlastfoot % Footer cho trang cuối cùng
\multicolumn{2}{|c|}{\textbf{2.1. Tóm tắt (Summary)}} \\
\hline
\textbf{Mục} & \textbf{Nội dung} \\
\hline
Use Case Name & Thực hiện Cuộc gọi và Tương tác Khách hàng \\
\hline
Use Case ID & UC-MD04-02 \\
\hline
Use Case Description & Dịch vụ Bot Call bên ngoài, sau khi nhận yêu cầu từ hệ thống (UC-MD04-01), thực hiện cuộc gọi đến số điện thoại của Khách hàng (US-08), phát kịch bản thoại đã được cấu hình, và ghi nhận lựa chọn (bấm phím 1, 0, hoặc 2) từ khách hàng. \\
\hline
Actor & Bot Call Service (Bên thứ ba - Thực hiện chính), US-08 (Khách hàng - Tương tác) \\
\hline
Priority & Must Have \\
\hline
Trigger & Dịch vụ Bot Call nhận được yêu cầu thực hiện cuộc gọi từ hệ thống hệ thống. \\
\hline
Pre-Condition & - Yêu cầu gọi từ hệ thống (UC-MD04-01) đã được gửi thành công đến Bot Call Service. \newline - Yêu cầu chứa đầy đủ thông tin (SĐT khách, kịch bản...). \newline - Dịch vụ Bot Call đang hoạt động. \newline - Khách hàng có điện thoại và trong vùng phủ sóng. \\
\hline
Post-Condition & - Cuộc gọi được thực hiện đến khách hàng. \newline - Kịch bản thoại được phát. \newline - Lựa chọn (bấm phím 1, 0, 2) của khách hàng được Bot Call Service ghi nhận (hoặc ghi nhận trạng thái không trả lời, máy bận...). \newline - Bot Call Service chuẩn bị gửi kết quả về cho hệ thống (UC-MD04-03). \\
\hline
\multicolumn{2}{|c|}{\textbf{2.2. Luồng thực thi (Flow)}} \\
\hline
\textbf{Mục} & \textbf{Nội dung} \\
\hline
Basic Flow (Khách hàng nghe máy và chọn 1) & 1. Bot Call Service nhận yêu cầu từ hệ thống và đưa vào hàng đợi xử lý của nó. \newline 2. Bot Call Service thực hiện cuộc gọi đến Số điện thoại của khách hàng (US-08) được cung cấp trong yêu cầu. \newline 3. Khách hàng US-08 nhấc máy. \newline 4. Bot Call Service phát kịch bản thoại đã được cấu hình (ví dụ: "Xin chào [Tên khách hàng], đây là nhà hàng ABC. Quý khách có đặt bàn vào [Giờ] ngày [Ngày]. Vui lòng bấm phím 1 để xác nhận, bấm phím 0 để hủy, bấm phím 2 để được hỗ trợ."). \newline 5. Khách hàng US-08 nghe và bấm phím 1. \newline 6. Bot Call Service ghi nhận lựa chọn là "1" (Xác nhận). \newline 7. Bot Call Service có thể phát lời thoại cảm ơn và kết thúc cuộc gọi (ví dụ: "Cảm ơn quý khách đã xác nhận. Hẹn gặp lại!"). \newline 8. Bot Call Service chuẩn bị dữ liệu kết quả (Call ID, SĐT, lựa chọn "1", trạng thái "Thành công") để gửi về hệ thống. \\
\hline
Alternative Flow & \textbf{5a. Khách hàng bấm phím 0 (Hủy):} \newline    1. Khách hàng US-08 bấm phím 0. \newline    2. Bot Call Service ghi nhận lựa chọn là "0" (Hủy). \newline    3. Bot Call Service có thể phát lời thoại xác nhận hủy (ví dụ: "Đặt chỗ của quý khách đã được hủy. Lưu ý quý khách sẽ mất tiền đặt cọc. Cảm ơn!"). \newline    4. Bot Call Service chuẩn bị dữ liệu kết quả (Call ID, SĐT, lựa chọn "0", trạng thái "Thành công"). \newline \textbf{5b. Khách hàng bấm phím 2 (Hỗ trợ):} \newline    1. Khách hàng US-08 bấm phím 2. \newline    2. Bot Call Service ghi nhận lựa chọn là "2" (Hỗ trợ). \newline    3. Bot Call Service thực hiện hành động chuyển hướng cuộc gọi đến số điện thoại hỗ trợ đã được cấu hình (FR-MD04-05). Cuộc gọi tiếp tục giữa khách hàng và nhân viên hỗ trợ (US-09). \newline    4. Bot Call Service chuẩn bị dữ liệu kết quả (Call ID, SĐT, lựa chọn "2", trạng thái "Đã chuyển hướng"). \newline \textbf{3a. Khách hàng không nghe máy / Máy bận:} \newline    1. Bot Call Service thực hiện gọi nhưng không kết nối được (máy bận, không trả lời sau số hồi chuông nhất định). \newline    2. Bot Call Service ghi nhận trạng thái cuộc gọi là "Không liên lạc được" hoặc "Máy bận". \newline    3. Bot Call Service có thể thử gọi lại sau một khoảng thời gian (tùy cấu hình dịch vụ Bot Call). Nếu sau số lần thử lại tối đa vẫn thất bại, dịch vụ sẽ chuẩn bị dữ liệu kết quả cuối cùng (Call ID, SĐT, lựa chọn: null, trạng thái: "Không liên lạc được"/"Máy bận"). \\
\hline
Exception Flow & \textbf{2a. Lỗi thực hiện cuộc gọi từ Bot Call Service:} \newline    1. Dịch vụ Bot Call gặp lỗi kỹ thuật nội bộ khi cố gắng thực hiện cuộc gọi (ví dụ: lỗi hạ tầng mạng viễn thông, lỗi tổng đài). \newline    2. Bot Call Service ghi nhận trạng thái cuộc gọi là "Lỗi hệ thống". \newline    3. Bot Call Service chuẩn bị dữ liệu kết quả (Call ID, SĐT, lựa chọn: null, trạng thái: "Lỗi hệ thống"). \newline \textbf{4a. Lỗi phát kịch bản thoại:} \newline    1. Bot Call Service kết nối được nhưng gặp lỗi khi phát file âm thanh/text-to-speech. \newline    2. Cuộc gọi có thể bị ngắt hoặc khách hàng không nghe được nội dung. \newline    3. Bot Call Service ghi nhận trạng thái "Lỗi phát thoại". \newline \textbf{6a. Khách hàng không bấm phím nào / Bấm phím không hợp lệ:} \newline    1. Khách hàng nghe xong nhưng không tương tác hoặc bấm phím khác (3, 4,...). \newline    2. Bot Call Service chờ một khoảng thời gian timeout. \newline    3. Bot Call Service ghi nhận trạng thái "Không phản hồi" hoặc "Lựa chọn không hợp lệ". \newline    4. Bot Call Service chuẩn bị dữ liệu kết quả tương ứng. \\
\hline
\multicolumn{2}{|c|}{\textbf{2.3. Thông tin bổ sung (Additional Information)}} \\
\hline
\textbf{Mục} & \textbf{Nội dung} \\
\hline
Business Rule & - \textbf{BR-UC4.2-1:} Kịch bản thoại phải rõ ràng, ngắn gọn, cung cấp đủ thông tin (tên nhà hàng, ngày giờ đặt) và hướng dẫn các phím bấm rõ ràng. \newline - \textbf{BR-UC4.2-2:} Dịch vụ Bot Call phải có khả năng nhận diện chính xác phím bấm của khách hàng (DTMF tone). \newline - \textbf{BR-UC4.2-3:} Nếu khách hàng chọn Hỗ trợ (phím 2), việc chuyển hướng cuộc gọi phải được thực hiện đến đúng số điện thoại hỗ trợ đã cấu hình. \newline - \textbf{BR-UC4.2-4:} Cần có giới hạn về số lần thử gọi lại nếu khách hàng không nghe máy hoặc máy bận để tránh làm phiền khách và tốn chi phí. \\
\hline
Non-Functional Requirement & - \textbf{NFR-UC4.2-1 (Voice Quality):} Chất lượng âm thanh của cuộc gọi (giọng đọc bot) phải rõ ràng, dễ nghe. \newline - \textbf{NFR-UC4.2-2 (Latency):} Độ trễ từ lúc khách hàng bấm phím đến lúc bot phản hồi (nếu có) hoặc ghi nhận phải thấp. \newline - \textbf{NFR-UC4.2-3 (Reliability):} Dịch vụ Bot Call phải có độ tin cậy cao, đảm bảo thực hiện cuộc gọi theo yêu cầu và xử lý các trường hợp lỗi (máy bận, không trả lời) một cách hợp lý. \newline - \textbf{NFR-UC4.2-4 (Integration):} Dịch vụ Bot Call phải cung cấp cơ chế (ví dụ: webhook) để gửi kết quả cuộc gọi về cho hệ thống hệ thống một cách kịp thời và đáng tin cậy (cho UC-MD04-03). \\
\hline
\end{longtable}

\subsubsection{Use Case UC-MD04-03: Xử lý Phản hồi Khách hàng từ Bot Call}

\begin{longtable}{|m{4cm}|p{11cm}|}
\caption{Đặc tả Use Case UC-MD04-03: Xử lý Phản hồi Khách hàng từ Bot Call} \label{tab:uc_md04_03} \\
\hline

\endhead % Header cho các trang tiếp theo
\hline
\endfoot % Footer cho bảng
\hline
\endlastfoot % Footer cho trang cuối cùng
\multicolumn{2}{|c|}{\textbf{2.1. Tóm tắt (Summary)}} \\
\hline
\textbf{Mục} & \textbf{Nội dung} \\
\hline
Use Case Name & Xử lý Phản hồi Khách hàng từ Bot Call \\
\hline
Use Case ID & UC-MD04-03 \\
\hline
Use Case Description & Hệ thống hệ thống nhận kết quả cuộc gọi (bao gồm lựa chọn của khách hàng: 1, 0, 2 hoặc trạng thái lỗi) từ dịch vụ Bot Call bên ngoài (thường qua webhook) và thực hiện các hành động tương ứng: cập nhật trạng thái đặt chỗ, giải phóng bàn (nếu hủy), hoặc không làm gì (nếu cần hỗ trợ hoặc lỗi). \\
\hline
Actor & System (hệ thống Backend Logic - Nhận và xử lý webhook), US-09 (Nhân viên hỗ trợ - Chỉ khi khách bấm phím 2) \\
\hline
Priority & Must Have \\
\hline
Trigger & Hệ thống hệ thống nhận được một yêu cầu HTTP (webhook callback) từ dịch vụ Bot Call chứa thông tin kết quả của một cuộc gọi xác nhận. \\
\hline
Pre-Condition & - Dịch vụ Bot Call đã thực hiện xong cuộc gọi (UC-MD04-02) và gửi kết quả về endpoint webhook của hệ thống. \newline - Endpoint webhook của hệ thống đã được cấu hình và sẵn sàng nhận yêu cầu. \newline - Dữ liệu gửi về chứa thông tin để định danh lượt đặt chỗ (ví dụ: Mã đặt chỗ hệ thống đã gửi đi, hoặc Call ID đã lưu) và kết quả cuộc gọi (lựa chọn của khách, trạng thái cuộc gọi). \\
\hline
Post-Condition & - Trạng thái của lượt đặt chỗ tương ứng trong hệ thống được cập nhật dựa trên kết quả cuộc gọi. \newline - Nếu khách hàng hủy (bấm 0), bàn liên quan (nếu có) được giải phóng. \newline - Nếu khách hàng cần hỗ trợ (bấm 2), không có thay đổi trạng thái tự động, chờ xử lý từ nhân viên hỗ trợ. \newline - Hoạt động được ghi nhận vào log hoặc lịch sử đặt chỗ. \\
\hline
\multicolumn{2}{|c|}{\textbf{2.2. Luồng thực thi (Flow)}} \\
\hline
\textbf{Mục} & \textbf{Nội dung} \\
\hline
Basic Flow (Xử lý kết quả thành công) & 1. Endpoint webhook của hệ thống nhận được yêu cầu POST/GET từ Bot Call Service chứa dữ liệu kết quả cuộc gọi (ví dụ: Call ID, lựa chọn khách hàng, trạng thái gọi). \newline 2. Hệ thống hệ thống xác thực yêu cầu (ví dụ: kiểm tra secret key/token nếu có). \newline 3. Hệ thống hệ thống phân tích dữ liệu nhận được, xác định lượt đặt chỗ tương ứng (dựa trên Mã đặt chỗ hoặc Call ID đã lưu). \newline 4. Hệ thống đọc lựa chọn của khách hàng (digit pressed) hoặc trạng thái cuộc gọi từ dữ liệu nhận được. \newline 5. \textbf{Nếu Lựa chọn = "1" (Xác nhận):} \newline    a. Hệ thống cập nhật trạng thái đặt chỗ thành "Đã xác nhận lại bởi khách" (hoặc một trạng thái tương tự) hoặc chỉ ghi nhận vào log là khách đã xác nhận. Trạng thái chính vẫn là "Confirmed". \newline 6. \textbf{Nếu Lựa chọn = "0" (Hủy):} \newline    a. Hệ thống cập nhật trạng thái đặt chỗ thành "Đã hủy bởi khách qua Bot" (hoặc "Cancelled"). \newline    b. Hệ thống kiểm tra xem đặt chỗ này có đang giữ bàn nào không. Nếu có, hệ thống thực hiện giải phóng bàn đó (cập nhật trạng thái bàn thành trống cho khung giờ đó). \newline    c. Hệ thống ghi nhận việc mất cọc (tiền cọc đã thanh toán không được hoàn lại). \newline    d. (Tùy chọn) Gửi thông báo nội bộ cho quản lý về việc hủy. \newline 7. \textbf{Nếu Lựa chọn = "2" (Hỗ trợ):} \newline    a. Hệ thống không thay đổi trạng thái đặt chỗ. \newline    b. Hệ thống ghi nhận vào log/ghi chú của đặt chỗ là "Khách hàng yêu cầu hỗ trợ qua Bot Call". \newline    c. Cuộc gọi đã được chuyển hướng bởi Bot Call Service đến Nhân viên hỗ trợ (US-09) để xử lý trực tiếp. Hành động tiếp theo (nếu có) trên đặt chỗ sẽ do US-09 thực hiện thủ công sau cuộc gọi. \newline 8. \textbf{Nếu Trạng thái cuộc gọi là "Không liên lạc được", "Máy bận", "Không phản hồi", "Lỗi"...:} \newline    a. Hệ thống không thay đổi trạng thái đặt chỗ ("Confirmed"). \newline    b. Hệ thống ghi nhận kết quả chi tiết vào log/ghi chú của đặt chỗ (ví dụ: "Bot Call thất bại: Không liên lạc được"). \newline    c. (Tùy chọn) Hệ thống có thể gắn cờ đặt chỗ này để nhân viên liên hệ xác nhận thủ công. \newline 9. Hệ thống gửi phản hồi HTTP 200 OK cho Bot Call Service để xác nhận đã nhận và xử lý webhook thành công. \newline 10. Hệ thống ghi nhận chi tiết kết quả và hành động đã thực hiện vào lịch sử đặt chỗ hoặc log hệ thống (FR-MD04-04). \\
\hline
Alternative Flow & Không có luồng thay thế đáng kể cho việc xử lý logic webhook này. \\
\hline
Exception Flow & \textbf{1a. Lỗi nhận Webhook / Xác thực thất bại:} \newline    1. Endpoint webhook gặp lỗi, không nhận được yêu cầu, hoặc yêu cầu nhận được không hợp lệ/không thể xác thực. \newline    2. Hệ thống không xử lý được kết quả cuộc gọi. \newline    3. Cần có cơ chế theo dõi lỗi webhook ở phía server hệ thống và có thể cần kiểm tra lại với Bot Call Service. Trạng thái đặt chỗ không được cập nhật. \newline \textbf{3a. Không tìm thấy Đặt chỗ tương ứng:} \newline    1. Hệ thống không tìm thấy bản ghi đặt chỗ nào khớp với thông tin (Mã đặt chỗ, Call ID) nhận được từ webhook. \newline    2. Hệ thống ghi nhận lỗi và gửi phản hồi lỗi cho Bot Call Service (nếu có thể). \newline    3. Trạng thái đặt chỗ không được cập nhật. \newline \textbf{6d/7d/8d. Lỗi cập nhật trạng thái/Giải phóng bàn:} \newline    1. Hệ thống gặp lỗi khi cố gắng cập nhật trạng thái đặt chỗ hoặc giải phóng bàn trong cơ sở dữ liệu hệ thống. \newline    2. Hệ thống ghi nhận lỗi nội bộ. \newline    3. Phản hồi cho Bot Call Service vẫn có thể là 200 OK (vì đã nhận được webhook), nhưng cần có cảnh báo cho quản trị viên hệ thống về lỗi xử lý nội bộ. \\
\hline
\multicolumn{2}{|c|}{\textbf{2.3. Thông tin bổ sung (Additional Information)}} \\
\hline
\textbf{Mục} & \textbf{Nội dung} \\
\hline
Business Rule & - \textbf{BR-UC4.3-1:} Trạng thái đặt chỗ phải được cập nhật chính xác dựa trên lựa chọn của khách hàng (1=Giữ Confirmed/Xác nhận lại; 0=Cancelled). \newline - \textbf{BR-UC4.3-2:} Nếu khách hàng hủy (bấm 0), bàn liên quan phải được tự động giải phóng để khách khác có thể đặt. \newline - \textbf{BR-UC4.3-3:} Nếu khách hàng yêu cầu hỗ trợ (bấm 2), hệ thống hệ thống không tự động thay đổi trạng thái đặt chỗ; trách nhiệm xử lý thuộc về nhân viên hỗ trợ nhận cuộc gọi. \newline - \textbf{BR-UC4.3-4:} Nếu cuộc gọi thất bại (không liên lạc được, lỗi...), trạng thái đặt chỗ nên giữ nguyên là "Confirmed" và cần có cơ chế để nhân viên biết và liên hệ thủ công nếu cần. \newline - \textbf{BR-UC4.3-5:} Phản hồi webhook từ hệ thống về Bot Call Service nên được thực hiện nhanh chóng để tránh Bot Call Service gửi lại yêu cầu không cần thiết. \\
\hline
Non-Functional Requirement & - \textbf{NFR-UC4.3-1 (Reliability):} Endpoint webhook phải luôn sẵn sàng và xử lý các yêu cầu một cách đáng tin cậy. Logic xử lý phải đúng đắn trong mọi trường hợp (1, 0, 2, lỗi). \newline - \textbf{NFR-UC4.3-2 (Performance):} Thời gian xử lý một yêu cầu webhook phải rất nhanh (dưới 1-2 giây) để tránh timeout từ phía Bot Call Service. \newline - \textbf{NFR-UC4.3-3 (Security):} Endpoint webhook cần có cơ chế xác thực (ví dụ: secret token, IP whitelist) để đảm bảo chỉ Bot Call Service mới có thể gửi dữ liệu đến. \newline - \textbf{NFR-UC4.3-4 (Atomicity):} Việc cập nhật trạng thái và giải phóng bàn (nếu hủy) nên được thực hiện trong cùng một giao dịch cơ sở dữ liệu để đảm bảo tính nhất quán. \\
\hline
\end{longtable}

\subsubsection{Use Case UC-MD04-04: Ghi nhận Kết quả Cuộc gọi}

\begin{longtable}{|m{4cm}|p{11cm}|}
\caption{Đặc tả Use Case UC-MD04-04: Ghi nhận Kết quả Cuộc gọi} \label{tab:uc_md04_04} \\
\hline

\endhead % Header cho các trang tiếp theo
\hline
\endfoot % Footer cho bảng
\hline
\endlastfoot % Footer cho trang cuối cùng
\multicolumn{2}{|c|}{\textbf{2.1. Tóm tắt (Summary)}} \\
\hline
\textbf{Mục} & \textbf{Nội dung} \\
\hline
Use Case Name & Ghi nhận Kết quả Cuộc gọi \\
\hline
Use Case ID & UC-MD04-04 \\
\hline
Use Case Description & Hệ thống hệ thống lưu trữ lại thông tin chi tiết về kết quả của mỗi cuộc gọi xác nhận tự động qua bot, bao gồm thời gian gọi, trạng thái cuộc gọi (thành công, thất bại, không liên lạc được), và lựa chọn của khách hàng (nếu có). Thông tin này có thể được lưu trực tiếp vào lịch sử/ghi chú của đặt chỗ hoặc vào một bảng log riêng. \\
\hline
Actor & System (hệ thống Backend Logic) \\
\hline
Priority & Must Have \\
\hline
Trigger & Sau khi hệ thống hệ thống xử lý xong phản hồi từ Bot Call Service qua webhook (kết thúc UC-MD04-03). \\
\hline
Pre-Condition & - Hệ thống đã xử lý xong webhook từ Bot Call Service (UC-MD04-03). \newline - Hệ thống có thông tin chi tiết về kết quả cuộc gọi (Call ID, trạng thái, lựa chọn khách hàng...). \newline - Đã xác định được bản ghi đặt chỗ tương ứng. \\
\hline
Post-Condition & - Thông tin về kết quả cuộc gọi Bot Call được lưu trữ bền vững trong hệ thống hệ thống, liên kết với lượt đặt chỗ cụ thể. \newline - Nhân viên có thể xem lại lịch sử và kết quả của cuộc gọi xác nhận khi xem chi tiết đặt chỗ. \\
\hline
\multicolumn{2}{|c|}{\textbf{2.2. Luồng thực thi (Flow)}} \\
\hline
\textbf{Mục} & \textbf{Nội dung} \\
\hline
Basic Flow & 1. Tiếp nối từ UC-MD04-03, sau khi hệ thống đã xác định hành động cần thực hiện dựa trên kết quả webhook. \newline 2. Hệ thống chuẩn bị nội dung log cần ghi nhận, bao gồm: \newline    - Thời gian nhận kết quả webhook. \newline    - Call ID từ Bot Call Service (nếu có). \newline    - Trạng thái cuộc gọi (ví dụ: "Thành công", "Không liên lạc được", "Lỗi hệ thống Bot Call"). \newline    - Lựa chọn của khách hàng (nếu có) (ví dụ: "Bấm phím 1 - Xác nhận", "Bấm phím 0 - Hủy", "Bấm phím 2 - Yêu cầu hỗ trợ", "Không phản hồi"). \newline 3. Hệ thống ghi nội dung log này vào: \newline    - Phần chatter/lịch sử/ghi chú (Messaging/Log Note) của bản ghi đặt chỗ tương ứng. \newline    - HOẶC vào một bảng dữ liệu (model) riêng được thiết kế để lưu trữ lịch sử Bot Call, có liên kết đến bản ghi đặt chỗ. \newline 4. Hệ thống lưu thay đổi vào cơ sở dữ liệu. \\
\hline
Alternative Flow & \textbf{3a. Lưu thông tin bổ sung:} \newline    1. Hệ thống có thể lưu thêm các thông tin khác từ webhook (nếu Bot Call Service cung cấp) như thời lượng cuộc gọi, chi phí cuộc gọi... \\
\hline
Exception Flow & \textbf{4a. Lỗi ghi log vào cơ sở dữ liệu:} \newline    1. Hệ thống gặp lỗi khi cố gắng ghi dữ liệu log vào chatter hoặc bảng log riêng. \newline    2. Hệ thống ghi nhận lỗi hệ thống nội bộ. \newline    3. Thông tin kết quả cuộc gọi có thể bị mất, ảnh hưởng đến khả năng truy vết và kiểm tra sau này. Cần có cảnh báo cho quản trị viên. \\
\hline
\multicolumn{2}{|c|}{\textbf{2.3. Thông tin bổ sung (Additional Information)}} \\
\hline
\textbf{Mục} & \textbf{Nội dung} \\
\hline
Business Rule & - \textbf{BR-UC4.4-1:} Mọi kết quả cuộc gọi Bot Call (thành công, thất bại, lỗi) đều phải được ghi nhận lại trong hệ thống hệ thống và liên kết với đúng lượt đặt chỗ. \newline - \textbf{BR-UC4.4-2:} Nội dung log cần đủ chi tiết để nhân viên hoặc quản trị viên hiểu được điều gì đã xảy ra với cuộc gọi xác nhận. \newline - \textbf{BR-UC4.4-3:} Dữ liệu log này chỉ nên hiển thị cho người dùng nội bộ có quyền hạn phù hợp, không hiển thị cho khách hàng. \\
\hline
Non-Functional Requirement & - \textbf{NFR-UC4.4-1 (Auditability):} Việc ghi nhận kết quả cuộc gọi là rất quan trọng cho mục đích kiểm tra, theo dõi và xử lý tranh chấp (nếu có). Dữ liệu log phải đầy đủ và không thể bị sửa đổi dễ dàng. \newline - \textbf{NFR-UC4.4-2 (Performance):} Việc ghi log không được làm chậm đáng kể quá trình xử lý webhook. \newline - \textbf{NFR-UC4.4-3 (Storage):} Cần ước tính dung lượng lưu trữ cần thiết cho dữ liệu log Bot Call, đặc biệt nếu số lượng đặt chỗ lớn. Có thể cần chính sách xóa log cũ định kỳ. \\
\hline
\end{longtable}

\subsubsection{Use Case UC-MD04-05: Cấu hình Dịch vụ Bot Call}

\begin{longtable}{|m{4cm}|p{11cm}|}
\caption{Đặc tả Use Case UC-MD04-05: Cấu hình Dịch vụ Bot Call} \label{tab:uc_md04_05} \\
\hline

\endhead % Header cho các trang tiếp theo
\hline
\endfoot % Footer cho bảng
\hline
\endlastfoot % Footer cho trang cuối cùng
\multicolumn{2}{|c|}{\textbf{2.1. Tóm tắt (Summary)}} \\
\hline
\textbf{Mục} & \textbf{Nội dung} \\
\hline
Use Case Name & Cấu hình Dịch vụ Bot Call \\
\hline
Use Case ID & UC-MD04-05 \\
\hline
Use Case Description & Cho phép Quản lý nhà hàng hoặc Quản trị viên hệ thống thiết lập các tham số cần thiết để tích hợp và vận hành chức năng gọi xác nhận tự động qua bot, bao gồm thông tin kết nối API, kịch bản thoại, thời gian gọi và số điện thoại hỗ trợ. \\
\hline
Actor & US-01 (Quản lý nhà hàng), US-10 (Quản trị viên Hệ thống) \\
\hline
Priority & Must Have \\
\hline
Trigger & - Thiết lập lần đầu cho chức năng Bot Call. \newline - Cần thay đổi nhà cung cấp dịch vụ Bot Call. \newline - Cần cập nhật kịch bản thoại, thời gian gọi hoặc số hỗ trợ. \\
\hline
Pre-Condition & - Người dùng (US-01 hoặc US-10) đã đăng nhập với quyền quản trị cấu hình hệ thống hoặc cấu hình module Đặt chỗ/Tích hợp. \newline - Nhà hàng đã đăng ký và có tài khoản với một nhà cung cấp dịch vụ Bot Call bên ngoài, có thông tin API cần thiết. \\
\hline
Post-Condition & - Các tham số cấu hình Bot Call được lưu lại trong hệ thống hệ thống. \newline - Hệ thống hệ thống có đủ thông tin để thực hiện việc gửi yêu cầu gọi (UC-MD04-01) và xử lý kết quả (UC-MD04-03) theo các cấu hình mới. \\
\hline
\multicolumn{2}{|c|}{\textbf{2.2. Luồng thực thi (Flow)}} \\
\hline
\textbf{Mục} & \textbf{Nội dung} \\
\hline
Basic Flow & 1. Người dùng (US-01/US-10) truy cập vào khu vực Cài đặt chung của hệ thống hoặc Cài đặt của module Đặt chỗ/Tích hợp. \newline 2. Người dùng tìm đến phần cấu hình liên quan đến "Bot Call Confirmation" hoặc "Voice Bot Integration". \newline 3. Hệ thống hiển thị form cấu hình với các trường: \newline    - \textbf{Kích hoạt Bot Call:} Ô kiểm để bật/tắt toàn bộ chức năng. \newline    - \textbf{Nhà cung cấp Bot Call:} Chọn nhà cung cấp từ danh sách hỗ trợ (nếu có nhiều) hoặc nhập thông tin API Endpoint. \newline    - \textbf{API Key / Secret Token:} Nhập thông tin xác thực do nhà cung cấp Bot Call cung cấp. \newline    - \textbf{Số ngày gọi trước (N):} Nhập số nguyên dương (ví dụ: 1, 2). \newline    - \textbf{ID Kịch bản thoại:} Nhập mã hoặc tên của kịch bản thoại sẽ sử dụng (kịch bản này thường được soạn thảo trên nền tảng của nhà cung cấp Bot Call). \newline    - \textbf{Số điện thoại Hỗ trợ:} Nhập số điện thoại đầy đủ sẽ nhận cuộc gọi khi khách bấm phím 2. \newline    - (Tùy chọn) Các tham số khác như Giờ bắt đầu/kết thúc cho phép gọi trong ngày, Số lần thử lại tối đa... \newline 4. Người dùng nhập hoặc cập nhật các giá trị cấu hình mong muốn. \newline 5. Người dùng chọn hành động "Lưu" (Save). \newline 6. Hệ thống kiểm tra tính hợp lệ cơ bản của dữ liệu (ví dụ: N là số nguyên dương, SĐT hỗ trợ có định dạng hợp lệ). \newline 7. Hệ thống lưu lại các cấu hình mới. \newline 8. Hệ thống hiển thị thông báo lưu thành công. \\
\hline
Alternative Flow & \textbf{3a. Cấu hình Kịch bản thoại trực tiếp (Nếu hệ thống hỗ trợ):} \newline    1. Thay vì nhập ID kịch bản, hệ thống cho phép soạn thảo trực tiếp nội dung kịch bản thoại (text-to-speech) và cấu hình các phím bấm ngay trong hệ thống. \newline    2. Người dùng soạn thảo/cập nhật kịch bản. \newline \textbf{3b. Kiểm tra kết nối API:} \newline    1. Giao diện cấu hình có nút "Kiểm tra kết nối" để xác thực thông tin API Key/Endpoint với nhà cung cấp Bot Call. \newline    2. Người dùng nhấp nút kiểm tra. \newline    3. Hệ thống gửi yêu cầu kiểm tra đến Bot Call Service và hiển thị kết quả (Thành công/Thất bại). \\
\hline
Exception Flow & \textbf{6a. Lỗi Xác thực Dữ liệu:} \newline    1. Hệ thống phát hiện giá trị nhập không hợp lệ (ví dụ: N không phải số, thiếu API Key). \newline    2. Hệ thống báo lỗi, chỉ rõ trường bị sai. \newline    3. Hệ thống không lưu cấu hình. Use Case quay lại bước 4. \newline \textbf{7a. Lỗi Hệ thống khi Lưu:} \newline    1. Hệ thống gặp sự cố kỹ thuật khi lưu cấu hình. \newline    2. Hệ thống hiển thị thông báo lỗi chung. \newline    3. Use Case kết thúc không thành công. \\
\hline
\multicolumn{2}{|c|}{\textbf{2.3. Thông tin bổ sung (Additional Information)}} \\
\hline
\textbf{Mục} & \textbf{Nội dung} \\
\hline
Business Rule & - \textbf{BR-UC4.5-1:} Thông tin API Key/Credentials phải chính xác và được giữ bí mật. \newline - \textbf{BR-UC4.5-2:} Số ngày gọi trước (N) phải là số nguyên dương. \newline - \textbf{BR-UC4.5-3:} Số điện thoại hỗ trợ phải là số điện thoại có thực và có người trực để tiếp nhận cuộc gọi từ khách hàng. \newline - \textbf{BR-UC4.5-4:} Kịch bản thoại (dù cấu hình bằng ID hay trực tiếp) phải tuân thủ các yêu cầu của UC-MD04-02 (rõ ràng, đủ thông tin, hướng dẫn phím bấm). \\
\hline
Non-Functional Requirement & - \textbf{NFR-UC4.5-1 (Usability):} Giao diện cấu hình phải tập trung các tham số liên quan đến Bot Call vào một nơi, dễ tìm và dễ hiểu. \newline - \textbf{NFR-UC4.5-2 (Security):} Các thông tin nhạy cảm như API Key/Secret Token phải được lưu trữ an toàn trong hệ thống (ví dụ: mã hóa hoặc lưu dưới dạng trường password). \newline - \textbf{NFR-UC4.5-3 (Flexibility):} Hệ thống nên được thiết kế để có thể tích hợp với các nhà cung cấp Bot Call khác nhau trong tương lai (có thể thông qua adapter pattern). \\
\hline
\end{longtable}


\subsection{Module MD-05: Quản lý Bán hàng Tại chỗ (POS - Eat-in)}

\subsubsection{Use Case UC-MD05-01: Mở phiên làm việc POS}

\begin{longtable}{|m{4cm}|p{11cm}|}
\caption{Đặc tả Use Case UC-MD05-01: Mở phiên làm việc POS} \label{tab:uc_md05_01} \\
\hline

\endhead % Header cho các trang tiếp theo
\hline
\endfoot % Footer cho bảng
\hline
\endlastfoot % Footer cho trang cuối cùng
\multicolumn{2}{|c|}{\textbf{2.1. Tóm tắt (Summary)}} \\
\hline
\textbf{Mục} & \textbf{Nội dung} \\
\hline
Use Case Name & Mở phiên làm việc POS \\
\hline
Use Case ID & UC-MD05-01 \\
\hline
Use Case Description & Cho phép Nhân viên được phân quyền (Thu ngân, Quản lý) bắt đầu một phiên làm việc mới trên giao diện Point of Sale. Nếu tính năng kiểm soát tiền mặt được bật, hệ thống yêu cầu nhập số dư tiền mặt đầu ca trong ngăn kéo. \\
\hline
Actor & US-05 (Nhân viên thu ngân), US-01 (Quản lý nhà hàng) \\
\hline
Priority & Must Have \\
\hline
Trigger & Bắt đầu một ca làm việc mới hoặc khi cần mở lại POS sau khi phiên trước đã đóng. \\
\hline
Pre-Condition & - Người dùng đã đăng nhập vào hệ thống Odoo với tài khoản được phép truy cập Point of Sale. \newline - Ít nhất một cấu hình Point of Sale (ví dụ: "Restaurant") đã được thiết lập. \newline - Phiên làm việc POS trước đó (nếu có) đã được đóng đúng cách. \\
\hline
Post-Condition & - Một phiên làm việc POS mới được tạo và ở trạng thái "Đang hoạt động" (In Progress). \newline - Giao diện chính của POS (ví dụ: sơ đồ tầng) được hiển thị, sẵn sàng cho việc nhận đơn hàng. \newline - Nếu có kiểm soát tiền mặt, số dư tiền mặt đầu ca được ghi nhận. \newline - Hệ thống bắt đầu ghi nhận các giao dịch thuộc về phiên làm việc này. \\
\hline
\multicolumn{2}{|c|}{\textbf{2.2. Luồng thực thi (Flow)}} \\
\hline
\textbf{Mục} & \textbf{Nội dung} \\
\hline
Basic Flow (Có kiểm soát tiền mặt) & 1. Người dùng (US-05/US-01) truy cập module Point of Sale. \newline 2. Người dùng chọn cấu hình POS muốn mở (ví dụ: "Restaurant"). \newline 3. Hệ thống kiểm tra xem có phiên nào đang mở cho cấu hình này không. Nếu không, hệ thống hiển thị tùy chọn "Mở phiên mới" (New Session) hoặc "Tiếp tục phiên cũ" (Resume) nếu có phiên chưa đóng đúng cách. Người dùng chọn "Mở phiên mới". \newline 4. Do có kiểm soát tiền mặt, hệ thống hiển thị hộp thoại yêu cầu nhập "Số dư tiền mặt đầu ca" (Opening Cash Balance). \newline 5. Người dùng đếm số tiền mặt thực tế có trong ngăn kéo và nhập số tiền đó vào hệ thống. \newline 6. Người dùng nhấn nút "Mở phiên" (Open Session) hoặc "Xác nhận". \newline 7. Hệ thống ghi nhận số dư tiền mặt đầu ca và tạo bản ghi phiên POS mới với trạng thái "In Progress". \newline 8. Hệ thống tải và hiển thị giao diện chính của POS (ví dụ: Sơ đồ tầng - FR-MD05-02). \\
\hline
Alternative Flow & \textbf{Basic Flow (Không kiểm soát tiền mặt):} \newline    1. Các bước 1-3 tương tự. \newline    2. Hệ thống bỏ qua bước 4, 5 (không yêu cầu nhập số dư đầu ca). \newline    3. Người dùng nhấn nút "Mở phiên" (Open Session) ở bước 6. \newline    4. Hệ thống tạo bản ghi phiên POS mới (bước 7) và hiển thị giao diện chính (bước 8). \newline \textbf{3a. Tiếp tục phiên cũ:} \newline    1. Nếu có phiên trước đó chưa được đóng đúng cách (ví dụ: do mất điện, lỗi trình duyệt), hệ thống hiển thị tùy chọn "Tiếp tục phiên cũ". \newline    2. Người dùng chọn "Tiếp tục phiên cũ". \newline    3. Hệ thống mở lại phiên làm việc đó và tải lại trạng thái gần nhất (các đơn hàng đang mở...). \newline    4. Use Case kết thúc, người dùng tiếp tục làm việc trên phiên cũ. \\
\hline
Exception Flow & \textbf{6a. Lỗi khi mở phiên:} \newline    1. Hệ thống gặp lỗi kỹ thuật khi cố gắng tạo bản ghi phiên mới hoặc ghi nhận số dư tiền mặt. \newline    2. Hệ thống hiển thị thông báo lỗi chung (ví dụ: "Không thể mở phiên làm việc. Vui lòng thử lại hoặc liên hệ quản trị viên."). \newline    3. Use Case kết thúc không thành công. \newline \textbf{6b. Số dư tiền mặt không hợp lệ:} \newline    1. Người dùng nhập giá trị không phải số hoặc số âm vào ô số dư tiền mặt. \newline    2. Hệ thống báo lỗi yêu cầu nhập số tiền hợp lệ. \newline    3. Use Case quay lại bước 5. \\
\hline
\multicolumn{2}{|c|}{\textbf{2.3. Thông tin bổ sung (Additional Information)}} \\
\hline
\textbf{Mục} & \textbf{Nội dung} \\
\hline
Business Rule & - \textbf{BR-UC5.1-1:} Mỗi cấu hình POS chỉ có thể có một phiên làm việc hoạt động (In Progress) tại một thời điểm. \newline - \textbf{BR-UC5.1-2:} Việc có yêu cầu nhập số dư tiền mặt đầu ca hay không phụ thuộc vào cấu hình "Cash Control" của Point of Sale đó. \newline - \textbf{BR-UC5.1-3:} Số dư tiền mặt đầu ca là cơ sở để đối chiếu tiền mặt cuối ca khi đóng phiên (UC-MD05-13). \\
\hline
Non-Functional Requirement & - \textbf{NFR-UC5.1-1 (Usability):} Quy trình mở phiên phải đơn giản. Nếu có kiểm soát tiền mặt, việc nhập số dư phải rõ ràng. \newline - \textbf{NFR-UC5.1-2 (Performance):} Thời gian từ lúc nhấn "Mở phiên" đến khi giao diện POS chính hiển thị phải nhanh chóng (dưới 5 giây). \newline - \textbf{NFR-UC5.1-3 (Security):} Chỉ người dùng được cấp quyền mới có thể mở phiên POS. Việc kiểm soát tiền mặt tăng cường tính bảo mật và trách nhiệm. \\
\hline
\end{longtable}

\subsubsection{Use Case UC-MD05-02: Truy cập Sơ đồ tầng \& Chọn bàn}

\begin{longtable}{|m{4cm}|p{11cm}|}
\caption{Đặc tả Use Case UC-MD05-02: Truy cập Sơ đồ tầng \& Chọn bàn} \label{tab:uc_md05_02} \\
\hline

\endhead % Header cho các trang tiếp theo
\hline
\endfoot % Footer cho bảng
\hline
\endlastfoot % Footer cho trang cuối cùng
\multicolumn{2}{|c|}{\textbf{2.1. Tóm tắt (Summary)}} \\
\hline
\textbf{Mục} & \textbf{Nội dung} \\
\hline
Use Case Name & Truy cập Sơ đồ tầng \& Chọn bàn \\
\hline
Use Case ID & UC-MD05-02 \\
\hline
Use Case Description & Cho phép Nhân viên Phục vụ hoặc Lễ tân xem sơ đồ mặt bằng trực quan của nhà hàng trên giao diện POS, nắm bắt trạng thái của từng bàn (trống, đang có khách, đã đặt trước) và chọn một bàn cụ thể để thực hiện hành động tiếp theo (xếp khách, nhận đơn hàng...). \\
\hline
Actor & US-02 (Nhân viên phục vụ), US-03 (Nhân viên lễ tân) \\
\hline
Priority & Must Have \\
\hline
Trigger & - Nhân viên vừa mở phiên POS (UC-MD05-01). \newline - Nhân viên hoàn thành một giao dịch và quay lại màn hình chính. \newline - Nhân viên cần xếp khách vào bàn hoặc kiểm tra tình trạng bàn. \\
\hline
Pre-Condition & - Phiên làm việc POS đang hoạt động (UC-MD05-01 thành công). \newline - Sơ đồ tầng (Floor Plan) với các bàn đã được cấu hình cho POS này trong backend (liên quan đến case study JBS). \newline - Hệ thống có thể lấy được trạng thái mới nhất của các bàn (ví dụ: thông tin từ module Đặt chỗ MD-03 về các bàn đã được đặt trước). \\
\hline
Post-Condition & - Sơ đồ tầng của nhà hàng được hiển thị trên màn hình POS. \newline - Trạng thái của từng bàn (trống, có khách, đặt trước) được hiển thị trực quan (ví dụ: qua màu sắc, biểu tượng). \newline - Nếu nhân viên chọn một bàn, hệ thống sẽ chuyển sang ngữ cảnh của bàn đó (ví dụ: mở đơn hàng - UC-MD05-03). \\
\hline
\multicolumn{2}{|c|}{\textbf{2.2. Luồng thực thi (Flow)}} \\
\hline
\textbf{Mục} & \textbf{Nội dung} \\
\hline
Basic Flow & 1. Sau khi mở phiên POS (UC-MD05-01) hoặc khi quay lại màn hình chính, hệ thống hiển thị giao diện Sơ đồ tầng (Floor Plan) mặc định. \newline 2. Giao diện hiển thị cách bố trí các bàn (hình vuông, tròn...) theo đúng cấu hình backend, có thể có hình nền là sơ đồ thực tế. \newline 3. Mỗi biểu tượng bàn hiển thị thông tin trạng thái: \newline    - \textbf{Trống (Available):} Màu sắc/hiển thị cho biết bàn sẵn sàng cho khách mới. \newline    - \textbf{Đang có khách (Occupied):} Màu sắc/hiển thị khác, có thể kèm thông tin thời gian khách ngồi hoặc số tiền tạm tính. \newline    - \textbf{Đã đặt trước (Reserved):} Màu sắc/hiển thị khác, có thể kèm thông tin giờ đặt và tên khách. \newline    - (Có thể có các trạng thái khác như: Chờ dọn, Chờ thanh toán...). \newline 4. Nhân viên (US-02/US-03) xem xét sơ đồ tầng để nắm tình hình. \newline 5. Nhân viên nhấp vào một biểu tượng bàn cụ thể. \\
\hline
Alternative Flow & \textbf{1a. Chuyển đổi giữa các tầng/khu vực:} \newline    1. Nếu nhà hàng có nhiều tầng hoặc khu vực (ví dụ: Trong nhà, Sân vườn) được cấu hình thành các Floor Plan riêng biệt. \newline    2. Giao diện POS có các nút/tab để chuyển đổi qua lại giữa các sơ đồ tầng này. \newline    3. Nhân viên chọn tầng/khu vực muốn xem. \newline    4. Hệ thống hiển thị sơ đồ tầng tương ứng. Use Case tiếp tục từ bước 2. \newline \textbf{4a. Xem thông tin nhanh của bàn (Hover):} \newline    1. Nhân viên di chuột (hoặc chạm giữ trên tablet) lên một biểu tượng bàn. \newline    2. Hệ thống hiển thị một tooltip/popup nhỏ chứa thông tin tóm tắt về bàn đó (số bàn, số ghế, trạng thái, tên khách nếu có...). \\
\hline
Exception Flow & \textbf{1a. Lỗi tải sơ đồ tầng/trạng thái bàn:} \newline    1. Hệ thống gặp lỗi khi lấy dữ liệu cấu hình sơ đồ tầng hoặc dữ liệu trạng thái bàn hiện tại. \newline    2. Hệ thống hiển thị thông báo lỗi hoặc không thể hiển thị sơ đồ đúng cách. \newline    3. Nhân viên không thể thao tác chọn bàn. Use Case kết thúc không thành công. Cần kiểm tra cấu hình hoặc kết nối. \\
\hline
\multicolumn{2}{|c|}{\textbf{2.3. Thông tin bổ sung (Additional Information)}} \\
\hline
\textbf{Mục} & \textbf{Nội dung} \\
\hline
Business Rule & - \textbf{BR-UC5.2-1:} Bố cục và thông tin các bàn (số bàn, số ghế) hiển thị trên POS phải khớp với cấu hình Floor Plan trong backend. \newline - \textbf{BR-UC5.2-2:} Trạng thái bàn phải được cập nhật gần thời gian thực nhất có thể, phản ánh đúng tình trạng khách ngồi, đặt trước, hoặc bàn trống. \newline - \textbf{BR-UC5.2-3:} Màu sắc hoặc biểu tượng sử dụng để biểu thị trạng thái bàn phải rõ ràng, dễ phân biệt và nhất quán. \newline - \textbf{BR-UC5.2-4:} Thông tin về đặt chỗ (Reserved) phải được đồng bộ từ module Đặt chỗ (MD-03) để hiển thị đúng trên sơ đồ tầng POS. \\
\hline
Non-Functional Requirement & - \textbf{NFR-UC5.2-1 (Usability):} Sơ đồ tầng phải trực quan, dễ nhìn, dễ thao tác (chạm/click). Việc chuyển đổi giữa các tầng (nếu có) phải thuận tiện. \newline - \textbf{NFR-UC5.2-2 (Performance):} Thời gian tải sơ đồ tầng và cập nhật trạng thái bàn phải nhanh chóng, không gây chậm trễ cho nhân viên. \newline - \textbf{NFR-UC5.2-3 (Accuracy):} Thông tin trạng thái bàn hiển thị phải chính xác. \newline - \textbf{NFR-UC5.2-4 (Responsiveness):} Sơ đồ tầng cần hiển thị tốt trên các thiết bị POS có kích thước màn hình khác nhau (PC, tablet). \\
\hline
\end{longtable}

\subsubsection{Use Case UC-MD05-03: Bắt đầu/Mở đơn hàng tại bàn}

\begin{longtable}{|m{4cm}|p{11cm}|}
\caption{Đặc tả Use Case UC-MD05-03: Bắt đầu/Mở đơn hàng tại bàn} \label{tab:uc_md05_03} \\
\hline

\endhead % Header cho các trang tiếp theo
\hline
\endfoot % Footer cho bảng
\hline
\endlastfoot % Footer cho trang cuối cùng
\multicolumn{2}{|c|}{\textbf{2.1. Tóm tắt (Summary)}} \\
\hline
\textbf{Mục} & \textbf{Nội dung} \\
\hline
Use Case Name & Bắt đầu/Mở đơn hàng tại bàn \\
\hline
Use Case ID & UC-MD05-03 \\
\hline
Use Case Description & Sau khi Nhân viên Phục vụ chọn một bàn từ sơ đồ tầng (UC-MD05-02), hệ thống sẽ mở giao diện đơn hàng cho bàn đó. Nếu bàn trống, hệ thống tạo đơn hàng mới. Nếu bàn đang có khách, hệ thống mở lại đơn hàng đang hoạt động của bàn đó. \\
\hline
Actor & US-02 (Nhân viên phục vụ) \\
\hline
Priority & Must Have \\
\hline
Trigger & Nhân viên chọn một bàn cụ thể trên sơ đồ tầng POS (UC-MD05-02). \\
\hline
Pre-Condition & - Nhân viên đang xem sơ đồ tầng POS (UC-MD05-02 thành công). \newline - Nhân viên chọn một bàn hợp lệ. \\
\hline
Post-Condition & - Giao diện đơn hàng (Order Screen) cho bàn đã chọn được hiển thị. \newline - Nếu là bàn mới, một bản ghi đơn hàng mới được tạo trong bộ nhớ hoặc cơ sở dữ liệu, liên kết với bàn đó và phiên POS hiện tại. \newline - Nếu là bàn đang có khách, đơn hàng hiện tại của bàn đó được tải lên giao diện. \newline - Hệ thống sẵn sàng để nhân viên thêm món (UC-MD05-05) hoặc tải món đặt trước (UC-MD05-04). \newline - Trạng thái của bàn trên sơ đồ tầng được cập nhật thành "Occupied" (nếu trước đó là trống hoặc đặt trước). \\
\hline
\multicolumn{2}{|c|}{\textbf{2.2. Luồng thực thi (Flow)}} \\
\hline
\textbf{Mục} & \textbf{Nội dung} \\
\hline
Basic Flow (Chọn bàn trống hoặc bàn đặt trước) & 1. Nhân viên (US-02) nhấp vào một bàn đang ở trạng thái "Trống" (Available) hoặc "Đã đặt trước" (Reserved) trên sơ đồ tầng (UC-MD05-02). \newline 2. Hệ thống kiểm tra xem bàn này có liên kết với một đặt chỗ sắp tới từ MD-03 hay không. \newline 3. Hệ thống tạo một bản ghi đơn hàng mới (POS Order) trong hệ thống, liên kết đơn hàng này với: \newline    - Bàn đã chọn. \newline    - Nhân viên đang đăng nhập POS. \newline    - Phiên POS hiện tại. \newline    - (Nếu có) Lượt đặt chỗ liên quan (từ bước 2). \newline 4. Hệ thống cập nhật trạng thái của bàn trên sơ đồ tầng thành "Đang có khách" (Occupied). \newline 5. Hệ thống hiển thị giao diện đơn hàng (Order Screen). Giao diện này bao gồm các khu vực chính: \newline    - Danh sách các món đã gọi (hiện tại đang trống). \newline    - Khu vực hiển thị các Danh mục POS (POS Categories). \newline    - Khu vực hiển thị các Sản phẩm (món ăn/đồ uống) thuộc danh mục đang chọn. \newline    - Các nút chức năng (Thanh toán, In bill, Gửi bếp...). \\
\hline
Alternative Flow & \textbf{1a. Chọn bàn đang có khách (Occupied):} \newline    1. Nhân viên (US-02) nhấp vào một bàn đang ở trạng thái "Đang có khách" trên sơ đồ tầng. \newline    2. Hệ thống tìm và tải lại bản ghi đơn hàng POS đang hoạt động (chưa thanh toán) của bàn đó. \newline    3. Hệ thống hiển thị giao diện đơn hàng với danh sách các món đã gọi trước đó. Use Case kết thúc, sẵn sàng để thêm món mới hoặc thanh toán. \\
\hline
Exception Flow & \textbf{3a. Lỗi tạo đơn hàng mới:} \newline    1. Hệ thống gặp lỗi kỹ thuật khi cố gắng tạo bản ghi đơn hàng mới trong cơ sở dữ liệu hoặc bộ nhớ. \newline    2. Hệ thống hiển thị thông báo lỗi "Không thể bắt đầu đơn hàng mới cho bàn này." \newline    3. Nhân viên không thể mở đơn hàng. Use Case kết thúc không thành công. \newline \textbf{Alternative Flow 2a. Lỗi tải lại đơn hàng cũ:} \newline    1. Hệ thống gặp lỗi khi cố gắng tải lại đơn hàng đang hoạt động của bàn đã chọn. \newline    2. Hệ thống hiển thị thông báo lỗi "Không thể mở lại đơn hàng của bàn này." \newline    3. Use Case kết thúc không thành công. \\
\hline
\multicolumn{2}{|c|}{\textbf{2.3. Thông tin bổ sung (Additional Information)}} \\
\hline
\textbf{Mục} & \textbf{Nội dung} \\
\hline
Business Rule & - \textbf{BR-UC5.3-1:} Mỗi bàn đang có khách chỉ được liên kết với một đơn hàng POS đang hoạt động tại một thời điểm. \newline - \textbf{BR-UC5.3-2:} Khi chọn bàn trống hoặc bàn đặt trước, hệ thống phải tự động tạo đơn hàng mới và cập nhật trạng thái bàn thành Occupied. \newline - \textbf{BR-UC5.3-3:} Khi chọn bàn Occupied, hệ thống phải mở lại đúng đơn hàng đang liên kết với bàn đó. \newline - \textbf{BR-UC5.3-4:} Nếu bàn được chọn có liên kết với đặt chỗ (từ MD-03), thông tin đặt chỗ đó (ID) phải được lưu vào đơn hàng POS để phục vụ các bước sau (tải món đặt trước, áp dụng cọc). \\
\hline
Non-Functional Requirement & - \textbf{NFR-UC5.3-1 (Performance):} Thời gian từ lúc chọn bàn đến khi giao diện đơn hàng hiển thị (dù là mới hay cũ) phải rất nhanh (dưới 1-2 giây). \newline - \textbf{NFR-UC5.3-2 (Usability):} Giao diện đơn hàng phải rõ ràng, dễ dàng phân biệt các khu vực (danh sách món gọi, danh mục, sản phẩm, nút chức năng). \newline - \textbf{NFR-UC5.3-3 (Data Integrity):} Việc liên kết đơn hàng với đúng bàn, nhân viên, phiên POS và đặt chỗ (nếu có) phải chính xác. \\
\hline
\end{longtable}

% ... (Continue with the rest of the Use Cases for MD-05 in the same format) ...

\subsubsection{Use Case UC-MD05-04: Tải và Xác nhận Món ăn Đặt trước}

\begin{longtable}{|m{4cm}|p{11cm}|}
\caption{Đặc tả Use Case UC-MD05-04: Tải và Xác nhận Món ăn Đặt trước} \label{tab:uc_md05_04} \\
\hline

\endhead % Header cho các trang tiếp theo
\hline
\endfoot % Footer cho bảng
\hline
\endlastfoot % Footer cho trang cuối cùng
\multicolumn{2}{|c|}{\textbf{2.1. Tóm tắt (Summary)}} \\
\hline
\textbf{Mục} & \textbf{Nội dung} \\
\hline
Use Case Name & Tải và Xác nhận Món ăn Đặt trước \\
\hline
Use Case ID & UC-MD05-04 \\
\hline
Use Case Description & Khi một đơn hàng POS được mở cho bàn có liên kết với một lượt đặt chỗ (từ MD-03) mà khách hàng đã đặt món trước, hệ thống tự động hiển thị danh sách các món ăn đó trên giao diện đơn hàng POS. Nhân viên phục vụ cần xác nhận các món này với khách và có thể gửi chúng xuống bếp. \\
\hline
Actor & US-02 (Nhân viên phục vụ), System (Tự động tải) \\
\hline
Priority & Must Have (nếu có chức năng đặt món trước) \\
\hline
Trigger & Đơn hàng POS được mở thành công cho một bàn có lượt đặt chỗ liên kết chứa thông tin món ăn đặt trước (sau UC-MD05-03). \\
\hline
Pre-Condition & - Đơn hàng POS đã được mở và liên kết với một bản ghi đặt chỗ (từ MD-03). \newline - Bản ghi đặt chỗ đó chứa danh sách các món ăn/đồ uống khách hàng đã chọn đặt trước (từ UC-MD03-05). \\
\hline
Post-Condition & - Danh sách các món ăn đặt trước (tên món, biến thể, số lượng) được hiển thị trên giao diện đơn hàng POS, có thể được đánh dấu đặc biệt (ví dụ: "Pre-ordered"). \newline - Nhân viên có thể xác nhận các món này và gửi xuống bếp (thông qua UC-MD05-07). \newline - Các món này được tính vào tổng giá trị đơn hàng. \\
\hline
\multicolumn{2}{|c|}{\textbf{2.2. Luồng thực thi (Flow)}} \\
\hline
\textbf{Mục} & \textbf{Nội dung} \\
\hline
Basic Flow & 1. Tiếp nối từ UC-MD05-03, sau khi giao diện đơn hàng POS cho bàn có đặt chỗ được hiển thị. \newline 2. Hệ thống tự động kiểm tra bản ghi đặt chỗ liên kết và truy xuất danh sách các món ăn/biến thể/số lượng đã được đặt trước. \newline 3. Hệ thống tự động thêm các món ăn đặt trước này vào danh sách các món đã gọi trên giao diện đơn hàng POS. \newline 4. Các món ăn đặt trước được hiển thị với đầy đủ thông tin (Tên, Biến thể nếu có, Số lượng, Đơn giá). Chúng có thể được đánh dấu hoặc có màu khác để phân biệt với các món gọi tại bàn sau này (BR-UC5.4-1). \newline 5. Nhân viên phục vụ (US-02) nhìn thấy danh sách các món đặt trước. \newline 6. US-02 xác nhận lại với khách hàng về các món đã đặt trước này. \newline 7. (Tùy chọn) US-02 có thể cần thực hiện một hành động để xác nhận gửi các món đặt trước này xuống bếp (ví dụ: nhấn nút "Gửi Món Đặt Trước" hoặc chúng được gửi cùng với lần gửi đơn hàng đầu tiên - UC-MD05-07). \\
\hline
Alternative Flow & \textbf{6a. Khách hàng muốn thay đổi/hủy món đặt trước:} \newline    1. Nếu khách hàng muốn thay đổi số lượng hoặc hủy một món đã đặt trước ngay tại thời điểm này. \newline    2. US-02 thực hiện thao tác sửa số lượng hoặc xóa món đó khỏi danh sách trên POS (tương tự như sửa/xóa món gọi tại bàn). \newline    3. Việc thay đổi này cần được ghi nhận và có thể ảnh hưởng đến tiền đặt cọc đã tính (cần xem xét logic nghiệp vụ xử lý thay đổi món đặt trước). \newline \textbf{3a. Hiển thị dưới dạng đề xuất:} \newline    1. Thay vì tự động thêm vào đơn hàng, hệ thống hiển thị danh sách món đặt trước ở một khu vực riêng biệt trên màn hình. \newline    2. US-02 cần nhấp vào nút "Thêm tất cả món đặt trước" hoặc chọn từng món để đưa vào đơn hàng chính thức. \\
\hline
Exception Flow & \textbf{2a. Lỗi truy xuất món đặt trước:} \newline    1. Hệ thống gặp lỗi khi cố gắng đọc danh sách món ăn từ bản ghi đặt chỗ liên kết. \newline    2. Hệ thống không thể tải các món đặt trước lên giao diện POS. \newline    3. Hệ thống có thể hiển thị thông báo lỗi "Không thể tải món ăn đặt trước." \newline    4. Nhân viên cần hỏi lại khách và nhập thủ công các món đó. \newline \textbf{3b. Lỗi thêm món đặt trước vào đơn hàng POS:} \newline    1. Hệ thống gặp lỗi kỹ thuật khi cố gắng thêm các dòng món ăn vào bản ghi đơn hàng POS. \newline    2. Hệ thống hiển thị thông báo lỗi. \newline    3. Các món đặt trước không xuất hiện trên đơn hàng. \\
\hline
\multicolumn{2}{|c|}{\textbf{2.3. Thông tin bổ sung (Additional Information)}} \\
\hline
Business Rule & - \textbf{BR-UC5.4-1:} Các món ăn được tải từ đặt chỗ trước phải được hiển thị rõ ràng trên đơn hàng POS, có thể phân biệt được với các món gọi thêm tại bàn. \newline - \textbf{BR-UC5.4-2:} Số lượng và biến thể của món ăn hiển thị phải chính xác theo những gì khách hàng đã đặt trước. \newline - \textbf{BR-UC5.4-3:} Giá của các món đặt trước được tính vào tổng hóa đơn như các món gọi tại bàn. \newline - \textbf{BR-UC5.4-4:} Cần có quy trình rõ ràng cho việc nhân viên xác nhận và gửi các món đặt trước này xuống bếp (có thể gửi ngay khi mở bàn hoặc chờ xác nhận của nhân viên). \newline - \textbf{BR-UC5.4-5:} Cần định nghĩa rõ quy trình xử lý khi khách hàng muốn thay đổi hoặc hủy món đã đặt trước và đã trả tiền cọc cho món đó. \\
\hline
Non-Functional Requirement & - \textbf{NFR-UC5.4-1 (Performance):} Việc tải và hiển thị các món đặt trước phải diễn ra nhanh chóng ngay khi mở đơn hàng. \newline - \textbf{NFR-UC5.4-2 (Accuracy):} Dữ liệu món ăn đặt trước (tên, biến thể, số lượng, giá) phải được tải lên chính xác. \newline - \textbf{NFR-UC5.4-3 (Usability):} Cách hiển thị món đặt trước phải rõ ràng cho nhân viên. Nếu cần hành động xác nhận gửi bếp, nút bấm phải dễ thấy. \\
\hline
\end{longtable}

\subsubsection{Use Case UC-MD05-05: Thêm món ăn/đồ uống vào đơn hàng}

\begin{longtable}{|m{4cm}|p{11cm}|}
\caption{Đặc tả Use Case UC-MD05-05: Thêm món ăn/đồ uống vào đơn hàng} \label{tab:uc_md05_05} \\
\hline

\endhead % Header cho các trang tiếp theo
\hline
\endfoot % Footer cho bảng
\hline
\endlastfoot % Footer cho trang cuối cùng
\multicolumn{2}{|c|}{\textbf{2.1. Tóm tắt (Summary)}} \\
\hline
\textbf{Mục} & \textbf{Nội dung} \\
\hline
Use Case Name & Thêm món ăn/đồ uống vào đơn hàng \\
\hline
Use Case ID & UC-MD05-05 \\
\hline
Use Case Description & Cho phép Nhân viên phục vụ (US-02) chọn các món ăn, đồ uống từ giao diện menu trực quan trên POS và thêm chúng vào đơn hàng hiện tại của bàn khách đang phục vụ. \\
\hline
Actor & US-02 (Nhân viên phục vụ) \\
\hline
Priority & Must Have \\
\hline
Trigger & Khách hàng tại bàn gọi món ăn hoặc đồ uống. \\
\hline
Pre-Condition & - Nhân viên đang ở màn hình đơn hàng của một bàn cụ thể (UC-MD05-03 thành công). \newline - Giao diện POS hiển thị các danh mục (FR-MD02-04) và sản phẩm (MD-02) được cấu hình "Available in POS" (FR-MD02-08). \\
\hline
Post-Condition & - Món ăn/đồ uống được chọn (cùng số lượng và biến thể nếu có) được thêm vào danh sách các món đã gọi của đơn hàng POS. \newline - Tổng tiền tạm tính của đơn hàng được cập nhật. \newline - Món ăn mới thêm sẵn sàng để được gửi xuống bếp/bar (UC-MD05-07). \\
\hline
\multicolumn{2}{|c|}{\textbf{2.2. Luồng thực thi (Flow)}} \\
\hline
\textbf{Mục} & \textbf{Nội dung} \\
\hline
Basic Flow & 1. Nhân viên (US-02) đang ở màn hình đơn hàng POS. \newline 2. US-02 chọn Danh mục POS (POS Category) chứa món ăn khách gọi (ví dụ: nhấp vào tab "Món chính"). \newline 3. Hệ thống hiển thị danh sách các sản phẩm thuộc danh mục đó, thường kèm hình ảnh (nếu có) và giá bán. \newline 4. US-02 tìm và nhấp vào sản phẩm (món ăn/đồ uống) mà khách hàng gọi. \newline 5. \textbf{Nếu sản phẩm không có biến thể:} \newline    a. Hệ thống thêm 1 đơn vị của sản phẩm đó vào danh sách món đã gọi ở bên trái (hoặc khu vực tương ứng). \newline    b. Giá của món ăn được cộng vào tổng tạm tính. \newline 6. \textbf{Nếu sản phẩm có biến thể (đã cấu hình ở FR-MD02-06):} \newline    a. Hệ thống hiển thị popup/dialog yêu cầu chọn các Giá trị Thuộc tính (ví dụ: Size, Độ chín...). \newline    b. US-02 chọn các giá trị theo yêu cầu của khách. \newline    c. US-02 xác nhận lựa chọn biến thể. \newline    d. Hệ thống thêm 1 đơn vị của biến thể sản phẩm cụ thể đó vào danh sách món đã gọi. \newline    e. Giá của biến thể (giá gốc + phụ thu nếu có) được cộng vào tổng tạm tính. \newline 7. Giao diện cập nhật danh sách món đã gọi và tổng tiền. \\
\hline
Alternative Flow & \textbf{4a. Tăng số lượng nhanh:} \newline    1. Thay vì nhấp 1 lần, US-02 nhấp nhiều lần vào cùng một sản phẩm để tăng số lượng nhanh chóng (ví dụ: nhấp 3 lần để gọi 3 ly Coca). \newline    2. Hoặc sau khi món được thêm vào danh sách, US-02 nhấp vào dòng món đó để tăng số lượng. \newline \textbf{4b. Sử dụng tìm kiếm sản phẩm:} \newline    1. Thay vì duyệt danh mục, US-02 sử dụng ô tìm kiếm trên giao diện POS. \newline    2. US-02 nhập tên hoặc mã món ăn. \newline    3. Hệ thống hiển thị các sản phẩm khớp với tìm kiếm. \newline    4. US-02 chọn sản phẩm từ kết quả tìm kiếm. Use Case tiếp tục từ bước 5 hoặc 6. \newline \textbf{4c. Chọn sản phẩm tùy chọn/phụ thu (Modifier):} \newline    1. Sau khi thêm một món chính, US-02 nhấp vào dòng món đó để mở các tùy chọn (nếu được cấu hình). \newline    2. Giao diện hiển thị danh sách các sản phẩm tùy chọn/phụ thu (đã tạo ở FR-MD02-11 và được cấu hình làm modifier). \newline    3. US-02 chọn các tùy chọn theo yêu cầu của khách (ví dụ: tick vào "Thêm Phô Mai"). \newline    4. Các tùy chọn này được thêm vào đơn hàng (có thể như một dòng riêng hoặc ghi chú kèm phụ thu) và giá được cập nhật. \\
\hline
Exception Flow & \textbf{4d. Chọn sản phẩm không khả dụng:} \newline    1. Sản phẩm hiển thị nhưng không thể chọn (ví dụ: do hết hàng nếu là Stockable và có kiểm tra tồn kho, hoặc sản phẩm bị vô hiệu hóa). \newline    2. Hệ thống hiển thị thông báo "Sản phẩm không khả dụng" hoặc không cho phép thêm vào đơn hàng. \newline \textbf{5c/6f. Lỗi thêm món vào đơn hàng:} \newline    1. Hệ thống gặp lỗi kỹ thuật khi cố gắng thêm dòng món ăn vào đơn hàng. \newline    2. Hệ thống hiển thị thông báo lỗi. \newline \textbf{6g. Lỗi chọn biến thể:} \newline    1. Popup chọn biến thể gặp lỗi hoặc không hiển thị đúng các tùy chọn. \newline    2. Nhân viên không thể chọn đúng biến thể. Cần báo lỗi. \\
\hline
\multicolumn{2}{|c|}{\textbf{2.3. Thông tin bổ sung (Additional Information)}} \\
\hline
\textbf{Mục} & \textbf{Nội dung} \\
\hline
Business Rule & - \textbf{BR-UC5.5-1:} Chỉ những sản phẩm được cấu hình "Available in POS" (FR-MD02-08) và thuộc về các Danh mục POS (FR-MD02-04) mới hiển thị trên giao diện chọn món. \newline - \textbf{BR-UC5.5-2:} Nếu sản phẩm có biến thể, hệ thống phải yêu cầu nhân viên chọn các giá trị thuộc tính bắt buộc trước khi thêm vào đơn hàng. \newline - \textbf{BR-UC5.5-3:} Giá và thông tin sản phẩm hiển thị trên POS phải được đồng bộ từ dữ liệu sản phẩm trong backend (MD-02). \\
\hline
Non-Functional Requirement & - \textbf{NFR-UC5.5-1 (Usability):} Giao diện chọn món phải cực kỳ nhanh và dễ sử dụng, đặc biệt trên màn hình cảm ứng. Việc duyệt danh mục, tìm kiếm, chọn món, chọn biến thể phải thuận tiện. \newline - \textbf{NFR-UC5.5-2 (Performance):} Thời gian phản hồi khi chọn danh mục, tìm kiếm, thêm món vào đơn hàng phải gần như tức thời (< 1 giây). \newline - \textbf{NFR-UC5.5-3 (Accuracy):} Món ăn, số lượng, biến thể và giá cả được thêm vào đơn hàng phải chính xác tuyệt đối. \\
\hline
\end{longtable}

% ... (Continue with the rest of the Use Cases for MD-05 in the same format) ...

\subsubsection{Use Case UC-MD05-06: Xử lý Yêu cầu đặc biệt/Ghi chú bếp}

\begin{longtable}{|m{4cm}|p{11cm}|}
\caption{Đặc tả Use Case UC-MD05-06: Xử lý Yêu cầu đặc biệt/Ghi chú bếp} \label{tab:uc_md05_06} \\
\hline

\endhead % Header cho các trang tiếp theo
\hline
\endfoot % Footer cho bảng
\hline
\endlastfoot % Footer cho trang cuối cùng
\multicolumn{2}{|c|}{\textbf{2.1. Tóm tắt (Summary)}} \\
\hline
\textbf{Mục} & \textbf{Nội dung} \\
\hline
Use Case Name & Xử lý Yêu cầu đặc biệt/Ghi chú bếp \\
\hline
Use Case ID & UC-MD05-06 \\
\hline
Use Case Description & Cho phép Nhân viên phục vụ (US-02) thêm các ghi chú hoặc yêu cầu đặc biệt của khách hàng vào một món ăn cụ thể hoặc toàn bộ đơn hàng trên POS, để thông tin này được truyền xuống bộ phận bếp/bar khi gửi đơn hàng. \\
\hline
Actor & US-02 (Nhân viên phục vụ) \\
\hline
Priority & Must Have \\
\hline
Trigger & - Khách hàng có yêu cầu đặc biệt về cách chế biến món ăn (ví dụ: ít cay, không hành, không bột ngọt). \newline - Khách hàng bị dị ứng với thành phần nào đó. \newline - Nhân viên cần ghi chú lại một yêu cầu đặc biệt khác (ví dụ: món này ra sau, làm cho trẻ em). \\
\hline
Pre-Condition & - Nhân viên đang ở màn hình đơn hàng POS (UC-MD05-03). \newline - Ít nhất một món ăn đã được thêm vào đơn hàng (UC-MD05-04 hoặc UC-MD05-05). \newline - (Tùy chọn) Quản lý đã cấu hình sẵn các ghi chú bếp phổ biến (Kitchen Notes) trong cài đặt POS. \\
\hline
Post-Condition & - Ghi chú đặc biệt được đính kèm vào dòng món ăn tương ứng hoặc vào toàn bộ đơn hàng trên giao diện POS. \newline - Khi đơn hàng được gửi đi (UC-MD05-07), ghi chú này sẽ được hiển thị trên phiếu in bếp hoặc màn hình KDS. \\
\hline
\multicolumn{2}{|c|}{\textbf{2.2. Luồng thực thi (Flow)}} \\
\hline
\textbf{Mục} & \textbf{Nội dung} \\
\hline
Basic Flow (Thêm ghi chú cho món ăn) & 1. Nhân viên (US-02) đang ở màn hình đơn hàng POS, đã thêm món ăn cần ghi chú. \newline 2. US-02 chọn (nhấp vào) dòng món ăn muốn thêm ghi chú trong danh sách các món đã gọi. \newline 3. Giao diện hiển thị các tùy chọn cho dòng món ăn đó, bao gồm nút/ô "Thêm ghi chú" (Add Note) hoặc tương tự. \newline 4. US-02 nhấp vào "Thêm ghi chú". \newline 5. Hệ thống hiển thị một hộp thoại hoặc bàn phím ảo cho phép nhập ghi chú. \newline 6. US-02 nhập nội dung ghi chú theo yêu cầu của khách (ví dụ: "Không hành, ít cay"). \newline 7. US-02 xác nhận (nhấn "OK", "Xong" hoặc tương tự). \newline 8. Ghi chú vừa nhập được hiển thị bên dưới hoặc bên cạnh dòng món ăn trên giao diện POS. \\
\hline
Alternative Flow & \textbf{5a. Chọn ghi chú có sẵn:} \newline    1. Thay vì nhập tự do, hệ thống hiển thị danh sách các ghi chú bếp phổ biến đã được cấu hình sẵn (ví dụ: "Ít đường", "Không đá", "Dị ứng đậu phộng", "Làm kỹ"). \newline    2. US-02 chọn một hoặc nhiều ghi chú từ danh sách. \newline    3. Use Case tiếp tục từ bước 8. \newline \textbf{1a. Thêm ghi chú cho toàn bộ đơn hàng:} \newline    1. Thay vì chọn một món cụ thể, US-02 tìm nút "Thêm ghi chú đơn hàng" (Add Order Note) ở khu vực tổng hợp của đơn hàng. \newline    2. US-02 thực hiện các bước 4-8 để thêm ghi chú áp dụng cho cả đơn (ví dụ: "Ưu tiên bàn này", "Khách VIP"). \\
\hline
Exception Flow & \textbf{8a. Lỗi lưu ghi chú:} \newline    1. Hệ thống gặp lỗi kỹ thuật khi cố gắng lưu ghi chú vào đơn hàng. \newline    2. Hệ thống hiển thị thông báo lỗi. \newline    3. Ghi chú có thể không được lưu. \\
\hline
\multicolumn{2}{|c|}{\textbf{2.3. Thông tin bổ sung (Additional Information)}} \\
\hline
\textbf{Mục} & \textbf{Nội dung} \\
\hline
Business Rule & - \textbf{BR-UC5.6-1:} Ghi chú đính kèm vào món ăn/đơn hàng phải được truyền tải chính xác và rõ ràng đến bộ phận bếp/bar thông qua phiếu in hoặc KDS. \newline - \textbf{BR-UC5.6-2:} Nên có khả năng cấu hình sẵn các ghi chú bếp phổ biến để nhân viên chọn nhanh, giảm thiểu việc gõ phím và đảm bảo tính nhất quán. \newline - \textbf{BR-UC5.6-3:} Các ghi chú quan trọng (như dị ứng) nên được làm nổi bật trên phiếu in/KDS (ví dụ: in đậm, màu đỏ - tùy khả năng của thiết bị và cấu hình). \\
\hline
Non-Functional Requirement & - \textbf{NFR-UC5.6-1 (Usability):} Việc thêm ghi chú (cả nhập tự do và chọn sẵn) phải nhanh chóng và dễ dàng trong quá trình nhận đơn. Hiển thị ghi chú trên đơn hàng POS phải rõ ràng. \newline - \textbf{NFR-UC5.6-2 (Accuracy):} Nội dung ghi chú phải được lưu và truyền đi chính xác. \newline - \textbf{NFR-UC5.6-3 (Integration):} Dữ liệu ghi chú phải được module In ấn/KDS đọc và hiển thị đúng cách. \\
\hline
\end{longtable}

\subsubsection{Use Case UC-MD05-07: Gửi đơn hàng xuống Bếp/Bar}

\begin{longtable}{|m{4cm}|p{11cm}|}
\caption{Đặc tả Use Case UC-MD05-07: Gửi đơn hàng xuống Bếp/Bar} \label{tab:uc_md05_07} \\
\hline

\endhead % Header cho các trang tiếp theo
\hline
\endfoot % Footer cho bảng
\hline
\endlastfoot % Footer cho trang cuối cùng
\multicolumn{2}{|c|}{\textbf{2.1. Tóm tắt (Summary)}} \\
\hline
\textbf{Mục} & \textbf{Nội dung} \\
\hline
Use Case Name & Gửi đơn hàng xuống Bếp/Bar \\
\hline
Use Case ID & UC-MD05-07 \\
\hline
Use Case Description & Cho phép Nhân viên phục vụ (US-02) gửi thông tin về các món ăn/đồ uống mới được thêm vào đơn hàng (hoặc các món đặt trước cần xác nhận chế biến) đến các máy in hoặc màn hình KDS tại bộ phận bếp và/hoặc bar tương ứng. \\
\hline
Actor & US-02 (Nhân viên phục vụ), System (Xử lý định tuyến và gửi lệnh in/hiển thị) \\
\hline
Priority & Must Have \\
\hline
Trigger & Nhân viên đã nhập xong một lượt gọi món của khách (hoặc cần gửi các món đặt trước) và muốn thông báo cho bếp/bar bắt đầu chuẩn bị. \\
\hline
Pre-Condition & - Nhân viên đang ở màn hình đơn hàng POS (UC-MD05-03). \newline - Có ít nhất một món ăn/đồ uống mới được thêm vào đơn hàng hoặc món đặt trước cần được gửi đi. \newline - Các máy in bếp/bar hoặc KDS đã được cấu hình và kết nối (liên quan MD-09). \newline - Quy tắc định tuyến theo danh mục sản phẩm đã được thiết lập (FR-MD02-10). \\
\hline
Post-Condition & - Thông tin về các món ăn/đồ uống cần chuẩn bị (bao gồm tên món, số lượng, biến thể, ghi chú đặc biệt, số bàn, tên nhân viên) được in ra hoặc hiển thị trên các thiết bị tại bếp/bar tương ứng. \newline - Trạng thái của các món ăn trên đơn hàng POS được cập nhật (ví dụ: đánh dấu là đã gửi bếp). \\
\hline
\multicolumn{2}{|c|}{\textbf{2.2. Luồng thực thi (Flow)}} \\
\hline
\textbf{Mục} & \textbf{Nội dung} \\
\hline
Basic Flow & 1. Nhân viên (US-02) đang ở màn hình đơn hàng POS, đã thêm các món mới hoặc xác nhận các món đặt trước. \newline 2. US-02 nhấn nút "Gửi Bếp" / "Order" / "Send" hoặc tương tự trên giao diện POS. \newline 3. Hệ thống xác định các món ăn/đồ uống trong đơn hàng chưa được gửi đi (hoặc các món đặt trước cần gửi). \newline 4. Đối với mỗi món ăn/đồ uống cần gửi: \newline    a. Hệ thống xác định Danh mục POS (POS Category) của món đó. \newline    b. Dựa vào quy tắc định tuyến đã cấu hình (FR-MD02-10), hệ thống xác định (các) Máy in hoặc KDS đích cần gửi thông tin món này đến. \newline 5. Hệ thống tạo các yêu cầu in/hiển thị riêng biệt cho từng Máy in/KDS đích, chỉ bao gồm các món ăn thuộc về đích đó. Yêu cầu chứa: \newline    - Thông tin bàn (số bàn). \newline    - Tên nhân viên phục vụ. \newline    - Thời gian gửi. \newline    - Danh sách các món (Tên món, Số lượng, Biến thể, Ghi chú đặc biệt). \newline 6. Hệ thống gửi các yêu cầu này đến IoT Box hoặc dịch vụ quản lý thiết bị tương ứng. \newline 7. IoT Box (hoặc dịch vụ) gửi lệnh in/hiển thị đến các thiết bị vật lý tại bếp/bar. \newline 8. Hệ thống cập nhật trạng thái các món ăn trên giao diện POS là "Đã gửi". \newline 9. Hệ thống có thể hiển thị thông báo gửi thành công cho nhân viên. \\
\hline
Alternative Flow & \textbf{2a. Tự động gửi khi thêm món (Nếu cấu hình):} \newline    1. Hệ thống có thể được cấu hình để tự động gửi món ăn xuống bếp/bar ngay khi nhân viên thêm món đó vào đơn hàng, thay vì chờ nhấn nút "Gửi Bếp" chung. \newline \textbf{3a. Chỉ gửi các món mới:} \newline    1. Nếu đơn hàng đã có món gửi đi trước đó, khi nhấn "Gửi Bếp" lần nữa, hệ thống chỉ gửi các món mới được thêm vào kể từ lần gửi trước. \\
\hline
Exception Flow & \textbf{6a. Lỗi gửi yêu cầu đến IoT Box/dịch vụ:} \newline    1. Hệ thống không thể kết nối hoặc gửi yêu cầu đến IoT Box/dịch vụ quản lý thiết bị. \newline    2. Hệ thống hiển thị thông báo lỗi cho nhân viên (ví dụ: "Lỗi kết nối máy in bếp. Vui lòng kiểm tra."). \newline    3. Các món ăn chưa được gửi đi, trạng thái trên POS không được cập nhật. Nhân viên cần báo bếp thủ công hoặc thử lại. \newline \textbf{7a. Lỗi tại Máy in/KDS vật lý:} \newline    1. IoT Box gửi lệnh thành công nhưng máy in hết giấy, hết mực, bị kẹt hoặc KDS bị lỗi, mất kết nối. \newline    2. Hệ thống Odoo có thể không nhận biết được lỗi này trực tiếp (trừ khi IoT Box có cơ chế phản hồi lỗi nâng cao). \newline    3. Nhân viên hoặc bộ phận bếp/bar cần phát hiện và xử lý sự cố tại thiết bị. \\
\hline
\multicolumn{2}{|c|}{\textbf{2.3. Thông tin bổ sung (Additional Information)}} \\
\hline
\textbf{Mục} & \textbf{Nội dung} \\
\hline
Business Rule & - \textbf{BR-UC5.7-1:} Việc gửi đơn hàng phải tuân thủ đúng quy tắc định tuyến đã cấu hình: món nào gửi đến máy in/KDS nào. \newline - \textbf{BR-UC5.7-2:} Thông tin trên phiếu in/KDS phải đầy đủ, rõ ràng, dễ đọc cho nhân viên bếp/bar (Tên món, SL, Biến thể, Ghi chú, Bàn, Nhân viên). \newline - \textbf{BR-UC5.7-3:} Hệ thống cần có cơ chế đánh dấu các món đã được gửi đi để tránh gửi lại nhầm lẫn. \\
\hline
Non-Functional Requirement & - \textbf{NFR-UC5.7-1 (Performance):} Thời gian từ lúc nhấn nút "Gửi Bếp" đến khi yêu cầu được gửi đi và trạng thái trên POS cập nhật phải nhanh chóng (dưới 2 giây). Thời gian thực tế để phiếu in ra hoặc hiển thị trên KDS phụ thuộc vào tốc độ mạng và thiết bị. \newline - \textbf{NFR-UC5.7-2 (Reliability):} Quá trình gửi đơn hàng phải đáng tin cậy. Cần có cơ chế xử lý lỗi kết nối hoặc thông báo rõ ràng cho nhân viên khi có sự cố. \newline - \textbf{NFR-UC5.7-3 (Integration):} Tích hợp giữa POS, Backend Odoo, IoT Box và các thiết bị phần cứng phải hoạt động trơn tru. \\
\hline
\end{longtable}

% ... (Continue with the rest of the Use Cases for MD-05 in the same format) ...

\subsubsection{Use Case UC-MD05-08: Yêu cầu/In Hóa đơn Tạm tính}

\begin{longtable}{|m{4cm}|p{11cm}|}
\caption{Đặc tả Use Case UC-MD05-08: Yêu cầu/In Hóa đơn Tạm tính} \label{tab:uc_md05_08} \\
\hline

\endhead % Header cho các trang tiếp theo
\hline
\endfoot % Footer cho bảng
\hline
\endlastfoot % Footer cho trang cuối cùng
\multicolumn{2}{|c|}{\textbf{2.1. Tóm tắt (Summary)}} \\
\hline
\textbf{Mục} & \textbf{Nội dung} \\
\hline
Use Case Name & Yêu cầu/In Hóa đơn Tạm tính \\
\hline
Use Case ID & UC-MD05-08 \\
\hline
Use Case Description & Cho phép Nhân viên phục vụ (US-02) tạo và in ra một bản hóa đơn tạm thời (bill, pro-forma invoice) liệt kê tất cả các món ăn, đồ uống khách hàng đã gọi tại bàn cùng với số lượng, đơn giá, thành tiền và tổng cộng (chưa bao gồm các khoản giảm giá cuối cùng hoặc tiền tip, nhưng CÓ THỂ đã trừ tiền đặt cọc nếu quy trình yêu cầu). Mục đích là để khách hàng kiểm tra lại trước khi yêu cầu thanh toán chính thức. \\
\hline
Actor & US-02 (Nhân viên phục vụ) \\
\hline
Priority & Must Have \\
\hline
Trigger & Khách hàng yêu cầu xem hóa đơn để kiểm tra hoặc chuẩn bị thanh toán. \\
\hline
Pre-Condition & - Nhân viên đang ở màn hình đơn hàng POS của bàn khách yêu cầu (UC-MD05-03). \newline - Đơn hàng có ít nhất một món đã gọi. \newline - Máy in hóa đơn (Receipt Printer) đã được cấu hình và kết nối với POS (liên quan MD-09). \\
\hline
Post-Condition & - Một bản hóa đơn tạm tính được in ra từ máy in hóa đơn. \newline - Đơn hàng trên POS vẫn ở trạng thái chờ thanh toán. \\
\hline
\multicolumn{2}{|c|}{\textbf{2.2. Luồng thực thi (Flow)}} \\
\hline
\textbf{Mục} & \textbf{Nội dung} \\
\hline
Basic Flow & 1. Nhân viên (US-02) đang ở màn hình đơn hàng POS của bàn khách. \newline 2. US-02 nhấn nút "In Bill" / "Print Bill" / "Hóa đơn tạm tính" hoặc tương tự. \newline 3. Hệ thống tổng hợp thông tin các món ăn/đồ uống đã gọi trong đơn hàng hiện tại (Tên món, SL, Đơn giá, Thành tiền). \newline 4. Hệ thống tính tổng tiền hàng (Subtotal). \newline 5. Hệ thống tính thuế (VAT/GST) nếu có cấu hình. \newline 6. Hệ thống kiểm tra xem có tiền đặt cọc liên quan đến đơn hàng này không (từ UC-MD05-09). \newline 7. Nếu CÓ tiền đặt cọc: Hệ thống tính toán Tổng cộng cuối cùng = Tổng tiền hàng + Thuế - Tiền đặt cọc (BR-UC5.8-1). \newline 8. Nếu KHÔNG có tiền đặt cọc: Hệ thống tính toán Tổng cộng cuối cùng = Tổng tiền hàng + Thuế. \newline 9. Hệ thống tạo dữ liệu định dạng hóa đơn tạm tính, bao gồm: \newline    - Thông tin nhà hàng (Tên, địa chỉ, SĐT). \newline    - Thông tin đơn hàng (Số bàn, Tên nhân viên, Ngày giờ). \newline    - Danh sách chi tiết các món đã gọi. \newline    - Tổng tiền hàng (Subtotal). \newline    - Thuế (VAT/GST). \newline    - (Nếu có) Số tiền đặt cọc đã trừ. \newline    - Tổng cộng cuối cùng (Amount Due). \newline    - Lời cảm ơn hoặc thông tin khác. \newline 10. Hệ thống gửi dữ liệu hóa đơn tạm tính đến máy in hóa đơn đã cấu hình. \newline 11. Máy in in ra hóa đơn tạm tính. \newline 12. Nhân viên lấy hóa đơn và đưa cho khách hàng. \\
\hline
Alternative Flow & \textbf{7a. Chưa áp dụng đặt cọc ở bước này (Tùy quy trình):} \newline    1. Nếu quy trình nghiệp vụ quy định tiền đặt cọc chỉ được trừ ở bước thanh toán cuối cùng (UC-MD05-11), thì ở bước 7 và 8, hệ thống không trừ tiền đặt cọc. Hóa đơn tạm tính sẽ hiển thị tổng tiền chưa trừ cọc, nhưng có thể có dòng ghi chú về số tiền cọc đã trả. \\
\hline
Exception Flow & \textbf{10a. Lỗi gửi lệnh in / Lỗi máy in:} \newline    1. Hệ thống không thể gửi lệnh in đến máy in (lỗi kết nối IoT Box, lỗi cấu hình máy in) hoặc máy in gặp sự cố (hết giấy, kẹt giấy...). \newline    2. Hệ thống hiển thị thông báo lỗi cho nhân viên (ví dụ: "Lỗi in hóa đơn. Vui lòng kiểm tra máy in."). \newline    3. Hóa đơn không được in ra. Nhân viên cần khắc phục sự cố máy in và thử lại hoặc báo cáo cho khách. \\
\hline
\multicolumn{2}{|c|}{\textbf{2.3. Thông tin bổ sung (Additional Information)}} \\
\hline
\textbf{Mục} & \textbf{Nội dung} \\
\hline
Business Rule & - \textbf{BR-UC5.8-1:} Hóa đơn tạm tính phải liệt kê chi tiết từng món khách đã gọi. \newline - \textbf{BR-UC5.8-2:} Việc có trừ tiền đặt cọc ngay trên hóa đơn tạm tính hay chỉ trừ ở bước thanh toán cuối cùng cần được quyết định dựa trên quy trình vận hành mong muốn của nhà hàng và phải nhất quán. Hiển thị rõ ràng số tiền cọc đã trừ (nếu có) là quan trọng. \newline - \textbf{BR-UC5.8-3:} Hóa đơn tạm tính không phải là hóa đơn tài chính chính thức (VAT invoice) trừ khi hệ thống được cấu hình đặc biệt và tuân thủ quy định pháp luật về hóa đơn điện tử/tài chính. \newline - \textbf{BR-UC5.8-4:} Việc in hóa đơn tạm tính không làm thay đổi trạng thái của đơn hàng trên POS (vẫn chờ thanh toán). \\
\hline
Non-Functional Requirement & - \textbf{NFR-UC5.8-1 (Usability):} Nút "In Bill" phải dễ tìm trên giao diện đơn hàng. Định dạng hóa đơn in ra phải rõ ràng, dễ đọc. \newline - \textbf{NFR-UC5.8-2 (Performance):} Thời gian từ lúc nhấn nút đến khi lệnh in được gửi đi phải nhanh (dưới 2 giây). \newline - \textbf{NFR-UC5.8-3 (Accuracy):} Mọi thông tin trên hóa đơn tạm tính (món ăn, số lượng, đơn giá, thành tiền, tổng cộng, thuế, tiền cọc đã trừ - nếu có) phải chính xác tuyệt đối. \newline - \textbf{NFR-UC5.8-4 (Reliability):} Việc gửi lệnh in phải đáng tin cậy. \\
\hline
\end{longtable}

\subsubsection{Use Case UC-MD05-09: Áp dụng Tiền Đặt cọc vào Hóa đơn}

\begin{longtable}{|m{4cm}|p{11cm}|}
\caption{Đặc tả Use Case UC-MD05-09: Áp dụng Tiền Đặt cọc vào Hóa đơn} \label{tab:uc_md05_09} \\
\hline

\endhead % Header cho các trang tiếp theo
\hline
\endfoot % Footer cho bảng
\hline
\endlastfoot % Footer cho trang cuối cùng
\multicolumn{2}{|c|}{\textbf{2.1. Tóm tắt (Summary)}} \\
\hline
\textbf{Mục} & \textbf{Nội dung} \\
\hline
Use Case Name & Áp dụng Tiền Đặt cọc vào Hóa đơn \\
\hline
Use Case ID & UC-MD05-09 \\
\hline
Use Case Description & Khi Nhân viên phục vụ chuẩn bị cho quá trình thanh toán cuối cùng, hệ thống tự động kiểm tra xem đơn hàng POS hiện tại có liên kết với một lượt đặt chỗ đã thanh toán tiền đặt cọc hay không. Nếu có, hệ thống sẽ tự động trừ số tiền đặt cọc đó vào tổng số tiền khách hàng cần phải trả. \\
\hline
Actor & System (Thực hiện chính), US-02 (Nhân viên phục vụ - Kích hoạt gián tiếp khi vào màn hình thanh toán) \\
\hline
Priority & Must Have \\
\hline
Trigger & Nhân viên phục vụ chọn hành động tiến tới màn hình thanh toán (Payment Screen) cho đơn hàng POS. \\
\hline
Pre-Condition & - Đơn hàng POS đang mở và được liên kết với một bản ghi đặt chỗ (từ UC-MD05-03). \newline - Bản ghi đặt chỗ liên kết có trạng thái thanh toán đặt cọc là "Đã thanh toán" và có lưu số tiền đặt cọc đã trả (từ UC-MD03-09, UC-MD03-10). \\
\hline
Post-Condition & - Số tiền đặt cọc được hệ thống xác định và ghi nhận là sẽ được trừ vào hóa đơn. \newline - Số tiền cuối cùng cần thanh toán (Amount Due) hiển thị trên màn hình thanh toán đã được giảm đi đúng bằng số tiền đặt cọc. \newline - Có thể có một dòng hiển thị riêng biệt trên màn hình thanh toán/hóa đơn ghi rõ số tiền đặt cọc đã được áp dụng. \\
\hline
\multicolumn{2}{|c|}{\textbf{2.2. Luồng thực thi (Flow)}} \\
\hline
\textbf{Mục} & \textbf{Nội dung} \\
\hline
Basic Flow & 1. Nhân viên (US-02) đang ở màn hình đơn hàng POS và nhấp vào nút "Thanh toán" (Payment). \newline 2. Hệ thống chuẩn bị chuyển sang màn hình thanh toán. \newline 3. Hệ thống kiểm tra xem bản ghi đơn hàng POS hiện tại có liên kết (ví dụ: qua trường `booking\_id`) đến một bản ghi Đặt chỗ (Reservation/Booking) hay không. \newline 4. Nếu có liên kết, hệ thống kiểm tra trạng thái thanh toán đặt cọc và lấy giá trị số tiền đặt cọc đã thanh toán (Deposit Amount) từ bản ghi Đặt chỗ đó. \newline 5. Hệ thống tính toán Tổng số tiền phải trả ban đầu (Total Amount = Subtotal + Taxes). \newline 6. Hệ thống tính toán Số tiền cần thanh toán cuối cùng (Amount Due): \newline    `Amount Due = Total Amount - Deposit Amount` \newline 7. Hệ thống hiển thị màn hình thanh toán (Payment Screen). \newline 8. Trên màn hình thanh toán, hệ thống hiển thị rõ ràng: \newline    - Tổng tiền ban đầu (Total Amount). \newline    - Số tiền đặt cọc đã áp dụng (Deposit Applied / Paid Deposit) với giá trị âm hoặc dưới dạng khoản trừ. \newline    - Số tiền cần thanh toán cuối cùng (Amount Due). \newline 9. Nhân viên và khách hàng nhìn thấy số tiền cuối cùng cần trả đã được giảm trừ. \newline 10. Hệ thống sẵn sàng cho việc nhập số tiền khách trả và chọn phương thức thanh toán (UC-MD05-11). \\
\hline
Alternative Flow & \textbf{3a. Đơn hàng không có đặt chỗ liên kết hoặc đặt chỗ không có cọc:} \newline    1. Hệ thống không tìm thấy liên kết đặt chỗ hoặc đặt chỗ liên kết chưa thanh toán cọc. \newline    2. `Deposit Amount = 0`. \newline    3. `Amount Due = Total Amount`. \newline    4. Màn hình thanh toán hiển thị tổng tiền bình thường, không có dòng trừ tiền đặt cọc. \newline \textbf{8a. Áp dụng đặt cọc dưới dạng một phương thức thanh toán riêng:} \newline    1. Thay vì trừ trực tiếp vào Amount Due, hệ thống coi tiền đặt cọc như một khoản đã thanh toán. \newline    2. Trên màn hình thanh toán, Total Amount vẫn giữ nguyên. \newline    3. Có một dòng "Đã thanh toán bằng Đặt cọc" với số tiền tương ứng. \newline    4. Số tiền "Còn lại phải trả" (Remaining Amount) bằng Total Amount - Deposit Amount. (Logic này tương đương nhưng cách hiển thị khác). \\
\hline
Exception Flow & \textbf{4a. Lỗi truy xuất thông tin đặt cọc:} \newline    1. Hệ thống tìm thấy liên kết đặt chỗ nhưng gặp lỗi khi đọc trạng thái thanh toán hoặc số tiền đặt cọc. \newline    2. Hệ thống không thể áp dụng tiền đặt cọc. \newline    3. Hệ thống nên hiển thị cảnh báo cho nhân viên "Không thể xác minh tiền đặt cọc. Vui lòng kiểm tra thủ công." và hiển thị Amount Due chưa trừ cọc. Nhân viên cần xử lý tình huống này (ví dụ: liên hệ quản lý, kiểm tra backend). \newline \textbf{6b. Lỗi tính toán số học:} \newline    1. Hệ thống gặp lỗi khi thực hiện phép trừ. \newline    2. Hệ thống báo lỗi và không hiển thị được số tiền cuối cùng chính xác. \\
\hline
\multicolumn{2}{|c|}{\textbf{2.3. Thông tin bổ sung (Additional Information)}} \\
\hline
\textbf{Mục} & \textbf{Nội dung} \\
\hline
Business Rule & - \textbf{BR-UC5.9-1:} Hệ thống phải tự động kiểm tra và áp dụng tiền đặt cọc khi nhân viên vào màn hình thanh toán cho đơn hàng có liên kết đặt chỗ đã trả cọc. \newline - \textbf{BR-UC5.9-2:} Số tiền đặt cọc được áp dụng phải chính xác bằng số tiền khách hàng đã thanh toán trước đó. \newline - \textbf{BR-UC5.9-3:} Việc áp dụng tiền đặt cọc phải được hiển thị rõ ràng trên màn hình thanh toán và trên hóa đơn cuối cùng để khách hàng và nhân viên đều thấy. \newline - \textbf{BR-UC5.9-4:} Sau khi tiền đặt cọc đã được áp dụng vào một đơn hàng POS, hệ thống phải đánh dấu để tránh việc áp dụng lại lần nữa (ví dụ: cập nhật trạng thái trên bản ghi đặt chỗ hoặc bản ghi thanh toán cọc). \\
\hline
Non-Functional Requirement & - \textbf{NFR-UC5.9-1 (Accuracy):} Việc xác định và áp dụng đúng số tiền đặt cọc là cực kỳ quan trọng, phải chính xác 100\%. \newline - \textbf{NFR-UC5.9-2 (Performance):} Quá trình kiểm tra và áp dụng đặt cọc phải diễn ra nhanh chóng, không làm chậm quá trình chuyển sang màn hình thanh toán. \newline - \textbf{NFR-UC5.9-3 (Transparency):} Cách hiển thị việc trừ tiền đặt cọc phải rõ ràng và dễ hiểu cho cả nhân viên và khách hàng. \newline - \textbf{NFR-UC5.9-4 (Reliability):} Logic kiểm tra và áp dụng đặt cọc phải hoạt động ổn định và đáng tin cậy. \\
\hline
\end{longtable}

\subsubsection{Use Case UC-MD05-10: Tách hóa đơn (Split Bill)}

\begin{longtable}{|m{4cm}|p{11cm}|}
\caption{Đặc tả Use Case UC-MD05-10: Tách hóa đơn (Split Bill)} \label{tab:uc_md05_10} \\
\hline

\endhead % Header cho các trang tiếp theo
\hline
\endfoot % Footer cho bảng
\hline
\endlastfoot % Footer cho trang cuối cùng
\multicolumn{2}{|c|}{\textbf{2.1. Tóm tắt (Summary)}} \\
\hline
\textbf{Mục} & \textbf{Nội dung} \\
\hline
Use Case Name & Tách hóa đơn (Split Bill) \\
\hline
Use Case ID & UC-MD05-10 \\
\hline
Use Case Description & Cung cấp chức năng cho phép Nhân viên phục vụ (US-02) chia một đơn hàng gốc của một bàn thành nhiều đơn hàng/hóa đơn nhỏ hơn để các khách hàng trong cùng bàn có thể thanh toán riêng lẻ phần của họ (theo món ăn hoặc chia đều). Chức năng này cần xem xét việc phân bổ tiền đặt cọc đã được áp dụng (nếu có). \\
\hline
Actor & US-02 (Nhân viên phục vụ) \\
\hline
Priority & Must Have \\
\hline
Trigger & Một nhóm khách hàng tại cùng bàn yêu cầu thanh toán riêng từng người hoặc theo nhóm nhỏ hơn. \\
\hline
Pre-Condition & - Nhân viên đang ở màn hình đơn hàng POS hoặc màn hình thanh toán của bàn cần tách hóa đơn. \newline - Đơn hàng có ít nhất hai món ăn hoặc có thể chia thành nhiều phần. \newline - Chức năng tách hóa đơn được bật trong cấu hình POS. \\
\hline
Post-Condition & - Đơn hàng gốc được chia thành hai hoặc nhiều đơn hàng con riêng biệt. \newline - Mỗi đơn hàng con chứa một phần các món ăn từ đơn hàng gốc. \newline - Tổng giá trị của các đơn hàng con (bao gồm thuế) bằng tổng giá trị của đơn hàng gốc. \newline - Tiền đặt cọc (nếu có) được phân bổ hợp lý cho các đơn hàng con (BR-UC5.10-2). \newline - Mỗi đơn hàng con có thể được thanh toán riêng lẻ (UC-MD05-11). \\
\hline
\multicolumn{2}{|c|}{\textbf{2.2. Luồng thực thi (Flow)}} \\
\hline
\textbf{Mục} & \textbf{Nội dung} \\
\hline
Basic Flow (Tách theo món ăn) & 1. Nhân viên (US-02) đang ở màn hình đơn hàng hoặc màn hình thanh toán của bàn cần tách. \newline 2. US-02 chọn chức năng "Tách hóa đơn" (Split Bill / Split). \newline 3. Hệ thống hiển thị giao diện tách hóa đơn, thường bao gồm hai cột (hoặc nhiều hơn): cột "Đơn hàng gốc" và cột(các) "Đơn hàng mới". \newline 4. Danh sách các món ăn từ đơn hàng gốc được hiển thị ở cột "Đơn hàng gốc". \newline 5. US-02 chọn (nhấp vào) các món ăn mà khách hàng thứ nhất muốn thanh toán từ cột "Đơn hàng gốc". \newline 6. Các món ăn được chọn sẽ di chuyển sang cột "Đơn hàng mới 1". \newline 7. US-02 lặp lại bước 5-6 để tạo "Đơn hàng mới 2" cho khách hàng thứ hai, hoặc các đơn hàng tiếp theo. \newline 8. Hệ thống tự động tính toán lại tổng tiền (bao gồm thuế) cho mỗi đơn hàng con. \newline 9. Hệ thống thực hiện phân bổ tiền đặt cọc đã áp dụng (từ UC-MD05-09) cho các đơn hàng con (theo logic BR-UC5.10-2). \newline 10. Giao diện hiển thị số tiền cần thanh toán cuối cùng cho mỗi đơn hàng con (đã trừ phần cọc được phân bổ). \newline 11. US-02 xác nhận việc tách hóa đơn. \newline 12. Hệ thống tạo ra các bản ghi đơn hàng con riêng biệt. \newline 13. Giao diện quay lại màn hình thanh toán, hiển thị các đơn hàng con sẵn sàng để thanh toán riêng lẻ. \\
\hline
Alternative Flow & \textbf{3a. Tách theo số người (Chia đều):} \newline    1. Thay vì chọn từng món, US-02 chọn tùy chọn "Chia đều" (Split by Guests/Evenly). \newline    2. US-02 nhập số lượng người/phần muốn chia (ví dụ: chia 3). \newline    3. Hệ thống tự động chia tổng số tiền của đơn hàng gốc (bao gồm thuế) thành số phần bằng nhau. \newline    4. Hệ thống cũng chia đều tiền đặt cọc đã áp dụng cho các phần. \newline    5. Hệ thống tạo ra các đơn hàng con với số tiền cần thanh toán bằng nhau. \newline    6. Use Case tiếp tục từ bước 12. \newline \textbf{5a. Tách một phần số lượng của món ăn:} \newline    1. Nếu một món ăn có số lượng lớn hơn 1 (ví dụ: 2 Pizza) và khách muốn chia đôi. \newline    2. Khi chọn món ăn đó (bước 5), hệ thống cho phép nhập số lượng muốn chuyển sang đơn hàng mới (ví dụ: chuyển 1 Pizza). \newline    3. Hệ thống cập nhật số lượng còn lại ở đơn hàng gốc và số lượng ở đơn hàng mới. \\
\hline
Exception Flow & \textbf{9a. Lỗi phân bổ tiền đặt cọc:} \newline    1. Hệ thống gặp lỗi logic khi cố gắng phân bổ tiền đặt cọc cho các đơn hàng con. \newline    2. Hệ thống báo lỗi hoặc việc phân bổ không chính xác. Cần kiểm tra lại cấu hình hoặc logic. \newline \textbf{11a. Hủy bỏ việc tách:} \newline    1. Trước khi xác nhận, US-02 chọn hủy bỏ thao tác tách hóa đơn. \newline    2. Hệ thống quay lại trạng thái đơn hàng gốc ban đầu. \newline \textbf{12a. Lỗi tạo đơn hàng con:} \newline    1. Hệ thống gặp lỗi kỹ thuật khi cố gắng tạo các bản ghi đơn hàng con. \newline    2. Hệ thống báo lỗi. Việc tách hóa đơn thất bại. \\
\hline
\multicolumn{2}{|c|}{\textbf{2.3. Thông tin bổ sung (Additional Information)}} \\
\hline
\textbf{Mục} & \textbf{Nội dung} \\
\hline
Business Rule & - \textbf{BR-UC5.10-1:} Hệ thống phải hỗ trợ ít nhất hai phương thức tách hóa đơn phổ biến: tách theo món ăn và tách chia đều theo số người/số phần. \newline - \textbf{BR-UC5.10-2 (Deposit Allocation):} Khi tách hóa đơn có áp dụng tiền đặt cọc, tiền đặt cọc phải được phân bổ cho các hóa đơn con một cách hợp lý. Có thể có các logic khác nhau: \newline    - \textit{Logic 1 (Ưu tiên):} Phân bổ cọc tỷ lệ thuận với giá trị của từng hóa đơn con. \newline    - \textit{Logic 2 (Tuần tự):} Trừ hết cọc vào hóa đơn con đầu tiên, nếu còn dư mới trừ tiếp vào hóa đơn con thứ hai... \newline    - \textit{Logic 3 (Chia đều - nếu tách đều):} Chia đều tiền cọc cho các hóa đơn con. \newline    Cần xác định và cấu hình logic mong muốn. Logic 1 thường là công bằng nhất. \newline - \textbf{BR-UC5.10-3:} Tổng số tiền cần thanh toán của tất cả các đơn hàng con sau khi tách và trừ cọc phải bằng tổng số tiền cần thanh toán của đơn hàng gốc sau khi trừ cọc. \newline - \textbf{BR-UC5.10-4:} Sau khi tách, mỗi đơn hàng con hoạt động độc lập cho việc thanh toán. \\
\hline
Non-Functional Requirement & - \textbf{NFR-UC5.10-1 (Usability):} Giao diện tách hóa đơn phải trực quan, dễ dàng cho nhân viên thao tác chọn món hoặc chia đều. Việc hiển thị tiền của từng hóa đơn con (bao gồm cả phần cọc được phân bổ) phải rõ ràng. \newline - \textbf{NFR-UC5.10-2 (Performance):} Thao tác tách hóa đơn và tính toán lại tiền phải diễn ra nhanh chóng. \newline - \textbf{NFR-UC5.10-3 (Accuracy):} Việc di chuyển món ăn và tính toán lại tổng tiền, thuế, phân bổ cọc cho từng hóa đơn con phải chính xác tuyệt đối. \\
\hline
\end{longtable}

\subsubsection{Use Case UC-MD05-11: Xử lý Thanh toán}

\begin{longtable}{|m{4cm}|p{11cm}|}
\caption{Đặc tả Use Case UC-MD05-11: Xử lý Thanh toán} \label{tab:uc_md05_11} \\
\hline

\endhead % Header cho các trang tiếp theo
\hline
\endfoot % Footer cho bảng
\hline
\endlastfoot % Footer cho trang cuối cùng
\multicolumn{2}{|c|}{\textbf{2.1. Tóm tắt (Summary)}} \\
\hline
\textbf{Mục} & \textbf{Nội dung} \\
\hline
Use Case Name & Xử lý Thanh toán \\
\hline
Use Case ID & UC-MD05-11 \\
\hline
Use Case Description & Cho phép Nhân viên (Phục vụ hoặc Thu ngân) nhận tiền thanh toán từ khách hàng cho một đơn hàng (hoặc một đơn hàng con sau khi tách), áp dụng các phương thức thanh toán khác nhau (tiền mặt, thẻ, ví...), xử lý tiền thừa (nếu trả tiền mặt), ghi nhận tiền boa (tip), và xác nhận hoàn tất giao dịch thanh toán trong hệ thống POS. \\
\hline
Actor & US-02 (Nhân viên phục vụ), US-05 (Nhân viên thu ngân) \\
\hline
Priority & Must Have \\
\hline
Trigger & Khách hàng sẵn sàng thanh toán hóa đơn sau khi đã kiểm tra (và hóa đơn đã được tách nếu cần, tiền cọc đã được áp dụng). Nhân viên đang ở màn hình thanh toán (Payment Screen). \\
\hline
Pre-Condition & - Nhân viên đang ở màn hình thanh toán cho một đơn hàng cụ thể. \newline - Số tiền cuối cùng cần thanh toán (Amount Due, đã trừ cọc nếu có) được hiển thị rõ ràng (từ UC-MD05-09). \newline - Các phương thức thanh toán (Payment Methods: Cash, Bank/Card, ví điện tử...) đã được cấu hình trong POS. \newline - Nếu thanh toán thẻ, thiết bị thanh toán thẻ (Payment Terminal) đã được kết nối và cấu hình (liên quan MD-09). \\
\hline
Post-Condition & - Giao dịch thanh toán được ghi nhận thành công trong hệ thống. \newline - Số tiền đã thanh toán và phương thức thanh toán được lưu lại. \newline - Trạng thái đơn hàng được cập nhật thành "Đã thanh toán" (Paid). \newline - Hóa đơn/Phiếu thu (Receipt) được in ra cho khách hàng. \newline - Số dư tiền mặt trong phiên POS được cập nhật (nếu thanh toán bằng tiền mặt). \newline - Đơn hàng sẵn sàng để đóng (UC-MD05-12). \\
\hline
\multicolumn{2}{|c|}{\textbf{2.2. Luồng thực thi (Flow)}} \\
\hline
\textbf{Mục} & \textbf{Nội dung} \\
\hline
Basic Flow (Thanh toán bằng một phương thức) & 1. Nhân viên (US-02/US-05) đang ở màn hình thanh toán, thấy rõ Số tiền cần thanh toán cuối cùng (Amount Due). \newline 2. Nhân viên hỏi khách hàng về phương thức thanh toán. \newline 3. Khách hàng chọn thanh toán bằng một phương thức (ví dụ: Tiền mặt). \newline 4. Nhân viên chọn phương thức "Tiền mặt" (Cash) trên giao diện POS. \newline 5. Nhân viên nhập số tiền khách đưa vào ô "Số tiền nhận" (Tendered Amount). \newline 6. Hệ thống tự động tính toán và hiển thị "Số tiền trả lại" (Change). \newline 7. Nhân viên nhận tiền từ khách, trả lại tiền thừa (nếu có). \newline 8. Nhân viên nhấn nút "Xác nhận thanh toán" (Validate / Confirm Payment). \newline 9. Hệ thống ghi nhận giao dịch thanh toán bằng tiền mặt với số tiền bằng Amount Due. \newline 10. Hệ thống cập nhật trạng thái đơn hàng thành "Paid". \newline 11. Hệ thống tự động gửi lệnh in hóa đơn/phiếu thu (Receipt) đến máy in hóa đơn. \newline 12. Hệ thống hiển thị màn hình xác nhận thanh toán thành công (thường có nút "Đơn hàng tiếp theo" - Next Order). \\
\hline
Alternative Flow & \textbf{3a. Thanh toán bằng Thẻ (Tích hợp Terminal):} \newline    1. Khách hàng chọn thanh toán thẻ. \newline    2. Nhân viên chọn phương thức "Thẻ ngân hàng" (Bank / Card) đã được cấu hình tích hợp với terminal. \newline    3. Hệ thống tự động gửi Số tiền cần thanh toán (Amount Due) đến máy POS quẹt thẻ (Payment Terminal). \newline    4. Nhân viên yêu cầu khách hàng sử dụng thẻ trên terminal (quẹt, cắm chip, chạm...). \newline    5. Khách hàng thực hiện theo tác trên terminal, có thể nhập mã PIN. \newline    6. Terminal xử lý giao dịch với ngân hàng. \newline    7. Terminal gửi kết quả (Thành công/Thất bại) về lại hệ thống POS Odoo. \newline    8. Nếu Thành công: Use Case tiếp tục từ bước 9 (ghi nhận thanh toán bằng thẻ). \newline    9. Nếu Thất bại: Hệ thống báo lỗi trên POS, yêu cầu thử lại hoặc đổi phương thức khác. Use Case quay lại bước 2. \newline \textbf{3b. Thanh toán bằng nhiều phương thức:} \newline    1. Khách hàng muốn trả một phần bằng tiền mặt, một phần bằng thẻ. \newline    2. Nhân viên chọn phương thức thứ nhất (ví dụ: Tiền mặt). \newline    3. Nhân viên nhập số tiền khách muốn trả bằng tiền mặt vào ô tiền nhận của phương thức đó. \newline    4. Hệ thống hiển thị số tiền còn lại cần thanh toán (Remaining Amount). \newline    5. Nhân viên chọn phương thức thứ hai (ví dụ: Thẻ). \newline    6. Hệ thống tự động điền số tiền còn lại vào phương thức thứ hai (hoặc nhân viên nhập). \newline    7. Nhân viên xử lý thanh toán cho phương thức thứ hai (ví dụ: qua terminal). \newline    8. Sau khi cả hai phần thanh toán thành công, tổng số tiền thanh toán bằng Amount Due. Use Case tiếp tục từ bước 8 (Xác nhận thanh toán). \newline \textbf{8a. Thêm tiền boa (Tip):} \newline    1. Trước khi nhấn "Xác nhận thanh toán", khách hàng muốn thêm tiền boa. \newline    2. Nhân viên nhấn nút "Tiền boa" (Tip) trên giao diện thanh toán. \newline    3. Nhân viên nhập số tiền boa khách muốn trả. \newline    4. Hệ thống cộng tiền boa vào tổng số tiền thanh toán cuối cùng. \newline    5. Nhân viên tiếp tục xử lý thanh toán cho tổng số tiền mới. \newline    6. Tiền boa được ghi nhận riêng trong giao dịch. \\
\hline
Exception Flow & \textbf{5a. Số tiền mặt nhận không đủ:} \newline    1. Nhân viên nhập số tiền mặt khách đưa nhỏ hơn Amount Due. \newline    2. Hệ thống báo lỗi hoặc không cho phép xác nhận thanh toán. Yêu cầu nhập lại hoặc thêm phương thức thanh toán khác. \newline \textbf{Alternative Flow 3a, step 8b. Lỗi xử lý thẻ/giao dịch thất bại:} Xem lại Exception Flow của Alternative Flow 3a. \newline \textbf{8b. Lỗi hệ thống khi xác nhận thanh toán:} \newline    1. Hệ thống gặp lỗi kỹ thuật khi cố gắng ghi nhận giao dịch thanh toán hoặc cập nhật trạng thái đơn hàng. \newline    2. Hệ thống hiển thị thông báo lỗi chung. \newline    3. Giao dịch có thể chưa được ghi nhận đúng. Cần kiểm tra và xử lý thủ công nếu cần. \newline \textbf{11a. Lỗi in hóa đơn:} \newline    1. Tương tự Exception Flow của UC-MD05-08, hệ thống không thể in hóa đơn cuối cùng. \newline    2. Hệ thống báo lỗi in. Giao dịch thanh toán vẫn được ghi nhận. Nhân viên có thể thử in lại sau. \\
\hline
\multicolumn{2}{|c|}{\textbf{2.3. Thông tin bổ sung (Additional Information)}} \\
\hline
\textbf{Mục} & \textbf{Nội dung} \\
\hline
Business Rule & - \textbf{BR-UC5.11-1:} Tổng số tiền thanh toán (từ một hoặc nhiều phương thức) phải bằng đúng Số tiền cần thanh toán cuối cùng (Amount Due, đã trừ cọc). \newline - \textbf{BR-UC5.11-2:} Hệ thống phải hỗ trợ các phương thức thanh toán phổ biến tại nhà hàng (Tiền mặt, Thẻ nội địa/quốc tế, có thể cả Ví điện tử nếu tích hợp). \newline - \textbf{BR-UC5.11-3:} Nếu tích hợp với terminal thanh toán thẻ, việc gửi số tiền và nhận kết quả phải diễn ra tự động và chính xác. \newline - \textbf{BR-UC5.11-4:} Tiền boa (Tip) phải được ghi nhận riêng biệt để phục vụ việc báo cáo và phân chia cho nhân viên (nếu có chính sách). \newline - \textbf{BR-UC5.11-5:} Hóa đơn/Phiếu thu cuối cùng phải thể hiện rõ các món đã gọi, tổng tiền, thuế, tiền đặt cọc đã trừ (nếu có), tiền boa (nếu có), tổng số tiền đã thanh toán và phương thức thanh toán. \\
\hline
Non-Functional Requirement & - \textbf{NFR-UC5.11-1 (Usability):} Màn hình thanh toán phải rõ ràng, dễ dàng chọn phương thức, nhập số tiền. Việc xử lý tiền thừa, tiền boa, thanh toán nhiều phương thức phải trực quan. \newline - \textbf{NFR-UC5.11-2 (Performance):} Quá trình xử lý thanh toán (ghi nhận, in hóa đơn) phải nhanh chóng để không làm khách hàng chờ đợi lâu. \newline - \textbf{NFR-UC5.11-3 (Accuracy):} Mọi tính toán (tiền thừa, tổng tiền, tiền còn lại) và ghi nhận giao dịch phải chính xác tuyệt đối. \newline - \textbf{NFR-UC5.11-4 (Security):} Nếu có tích hợp terminal, việc truyền dữ liệu phải an toàn. Hệ thống không lưu thông tin thẻ nhạy cảm. \newline - \textbf{NFR-UC5.11-5 (Reliability):} Quá trình thanh toán, đặc biệt là khi tích hợp với thiết bị ngoài, phải hoạt động ổn định. \\
\hline
\end{longtable}

\subsubsection{Use Case UC-MD05-12: Đóng Đơn hàng và Bàn}

\begin{longtable}{|m{4cm}|p{11cm}|}
\caption{Đặc tả Use Case UC-MD05-12: Đóng Đơn hàng và Bàn} \label{tab:uc_md05_12} \\
\hline

\endhead % Header cho các trang tiếp theo
\hline
\endfoot % Footer cho bảng
\hline
\endlastfoot % Footer cho trang cuối cùng
\multicolumn{2}{|c|}{\textbf{2.1. Tóm tắt (Summary)}} \\
\hline
\textbf{Mục} & \textbf{Nội dung} \\
\hline
Use Case Name & Đóng Đơn hàng và Bàn \\
\hline
Use Case ID & UC-MD05-12 \\
\hline
Use Case Description & Sau khi khách hàng đã thanh toán thành công toàn bộ hóa đơn (UC-MD05-11), Nhân viên phục vụ thực hiện hành động cuối cùng trên POS để chính thức đóng đơn hàng và giải phóng bàn, cập nhật trạng thái bàn thành trống trên sơ đồ tầng. \\
\hline
Actor & US-02 (Nhân viên phục vụ) \\
\hline
Priority & Must Have \\
\hline
Trigger & Giao dịch thanh toán cho đơn hàng đã hoàn tất (kết thúc thành công UC-MD05-11). Nhân viên nhìn thấy màn hình xác nhận thanh toán thành công. \\
\hline
Pre-Condition & - Đơn hàng POS đã ở trạng thái "Đã thanh toán" (Paid). \newline - Nhân viên đang ở màn hình xác nhận thanh toán thành công hoặc quay lại màn hình đơn hàng đã thanh toán. \\
\hline
Post-Condition & - Trạng thái cuối cùng của đơn hàng POS được cập nhật thành "Đã hoàn thành" (Done) hoặc tương đương. \newline - Trạng thái của bàn liên kết với đơn hàng trên sơ đồ tầng POS được cập nhật thành "Trống" (Available), sẵn sàng cho lượt khách tiếp theo. \newline - Nhân viên được chuyển về màn hình chính của POS (thường là sơ đồ tầng). \\
\hline
\multicolumn{2}{|c|}{\textbf{2.2. Luồng thực thi (Flow)}} \\
\hline
\textbf{Mục} & \textbf{Nội dung} \\
\hline
Basic Flow & 1. Sau khi hoàn tất thanh toán (UC-MD05-11), hệ thống hiển thị màn hình xác nhận thanh toán thành công, thường có nút "Đơn hàng tiếp theo" (Next Order) hoặc tương tự. \newline 2. Nhân viên (US-02) nhấp vào nút "Đơn hàng tiếp theo". \newline 3. Hệ thống thực hiện các hành động đóng đơn hàng cuối cùng: \newline    a. Cập nhật trạng thái của bản ghi đơn hàng POS thành "Done" hoặc "Completed". \newline    b. Tìm bàn đang liên kết với đơn hàng này. \newline    c. Cập nhật trạng thái của bàn đó trên sơ đồ tầng thành "Trống" (Available). \newline 4. Hệ thống chuyển hướng giao diện về màn hình chính của POS (Sơ đồ tầng - UC-MD05-02). \\
\hline
Alternative Flow & \textbf{1a. Đóng đơn hàng từ màn hình chi tiết:} \newline    1. Trong một số trường hợp, nhân viên có thể quay lại màn hình chi tiết đơn hàng sau khi thanh toán. \newline    2. Trên màn hình này có nút "Đóng đơn hàng" / "Close Order". \newline    3. Nhân viên nhấp vào nút đó. Use Case tiếp tục từ bước 3. \\
\hline
Exception Flow & \textbf{3d. Lỗi cập nhật trạng thái đơn hàng/bàn:} \newline    1. Hệ thống gặp lỗi kỹ thuật khi cố gắng cập nhật trạng thái cuối cùng cho đơn hàng hoặc trạng thái bàn. \newline    2. Hệ thống hiển thị thông báo lỗi. \newline    3. Trạng thái đơn hàng/bàn có thể không được cập nhật đúng. Nhân viên có thể cần báo quản lý hoặc thực hiện thao tác thủ công (nếu có) để giải phóng bàn. \\
\hline
\multicolumn{2}{|c|}{\textbf{2.3. Thông tin bổ sung (Additional Information)}} \\
\hline
\textbf{Mục} & \textbf{Nội dung} \\
\hline
Business Rule & - \textbf{BR-UC5.12-1:} Chỉ những đơn hàng đã được thanh toán đầy đủ (Paid) mới có thể được đóng. \newline - \textbf{BR-UC5.12-2:} Việc đóng đơn hàng phải đồng thời cập nhật trạng thái bàn liên quan thành "Trống" để đảm bảo sơ đồ tầng phản ánh đúng tình trạng thực tế. \newline - \textbf{BR-UC5.12-3:} Sau khi đóng, đơn hàng không thể được mở lại hoặc chỉnh sửa thêm trên giao diện POS thông thường (chỉ có thể xem lại hoặc xử lý nghiệp vụ đặc biệt trong backend nếu cần). \\
\hline
Non-Functional Requirement & - \textbf{NFR-UC5.12-1 (Performance):} Thao tác đóng đơn hàng và cập nhật trạng thái bàn phải diễn ra nhanh chóng (< 1 giây). \newline - \textbf{NFR-UC5.12-2 (Consistency):} Trạng thái bàn phải được cập nhật đồng bộ và chính xác trên sơ đồ tầng sau khi đóng đơn hàng. \newline - \textbf{NFR-UC5.12-3 (Usability):} Nút "Đơn hàng tiếp theo" hoặc hành động đóng đơn phải rõ ràng, giúp nhân viên nhanh chóng quay lại màn hình chính để phục vụ bàn khác. \\
\hline
\end{longtable}

\subsubsection{Use Case UC-MD05-13: Đóng Phiên làm việc POS}

\begin{longtable}{|m{4cm}|p{11cm}|}
\caption{Đặc tả Use Case UC-MD05-13: Đóng Phiên làm việc POS} \label{tab:uc_md05_13} \\
\hline

\endhead % Header cho các trang tiếp theo
\hline
\endfoot % Footer cho bảng
\hline
\endlastfoot % Footer cho trang cuối cùng
\multicolumn{2}{|c|}{\textbf{2.1. Tóm tắt (Summary)}} \\
\hline
\textbf{Mục} & \textbf{Nội dung} \\
\hline
Use Case Name & Đóng Phiên làm việc POS \\
\hline
Use Case ID & UC-MD05-13 \\
\hline
Use Case Description & Cho phép Nhân viên được phân quyền (Thu ngân, Quản lý) kết thúc phiên làm việc POS hiện tại. Hệ thống sẽ tổng kết tất cả các giao dịch đã xảy ra trong phiên, đối chiếu số tiền mặt thực tế với số tiền dự kiến (nếu có kiểm soát tiền mặt), và ghi nhận các bút toán liên quan vào hệ thống kế toán. \\
\hline
Actor & US-05 (Nhân viên thu ngân), US-01 (Quản lý nhà hàng) \\
\hline
Priority & Must Have \\
\hline
Trigger & Kết thúc ca làm việc hoặc cuối ngày kinh doanh, cần phải đóng phiên POS để tổng kết và bàn giao. \\
\hline
Pre-Condition & - Người dùng đã đăng nhập vào hệ thống Odoo với tài khoản được phép đóng phiên POS. \newline - Có một phiên làm việc POS đang ở trạng thái "Đang hoạt động" (In Progress). \newline - Tất cả các đơn hàng trong phiên nên đã được đóng (thanh toán hoặc hủy). Hệ thống có thể cảnh báo nếu còn đơn hàng mở. \\
\hline
Post-Condition & - Trạng thái của phiên POS được cập nhật thành "Đã đóng" (Closed). \newline - Tất cả các giao dịch thuộc phiên đó được tổng kết và ghi nhận cuối cùng. \newline - Nếu có kiểm soát tiền mặt, chênh lệch giữa tiền mặt thực tế và dự kiến được ghi nhận. \newline - Các bút toán kế toán liên quan đến doanh thu, thanh toán của phiên được tạo ra hoặc xác nhận trong module Kế toán. \newline - Không thể thực hiện thêm giao dịch nào trong phiên đã đóng. \\
\hline
\multicolumn{2}{|c|}{\textbf{2.2. Luồng thực thi (Flow)}} \\
\hline
\textbf{Mục} & \textbf{Nội dung} \\
\hline
Basic Flow (Có kiểm soát tiền mặt) & 1. Người dùng (US-05/US-01) truy cập module Point of Sale. \newline 2. Từ menu chính của POS hoặc khu vực quản lý phiên, Người dùng chọn hành động "Đóng phiên" (Close Session / Close Register). \newline 3. Hệ thống kiểm tra xem còn đơn hàng nào đang mở trong phiên không. Nếu có, hiển thị cảnh báo và yêu cầu đóng các đơn hàng đó trước khi tiếp tục (hoặc có tùy chọn buộc đóng). \newline 4. Hệ thống hiển thị màn hình tóm tắt phiên làm việc, bao gồm: \newline    - Số dư tiền mặt đầu ca (đã nhập ở UC-MD05-01). \newline    - Tổng doanh thu dự kiến theo từng phương thức thanh toán (Tiền mặt, Thẻ...). \newline    - Số tiền mặt dự kiến có trong ngăn kéo cuối ca (Expected Cash = Opening Cash + Cash Payments - Cash Refunds...). \newline 5. Hệ thống yêu cầu người dùng nhập "Số tiền mặt thực tế cuối ca" (Actual Closing Cash). \newline 6. Người dùng đếm tiền mặt trong ngăn kéo và nhập số tiền thực tế vào hệ thống. \newline 7. Hệ thống tính toán và hiển thị "Chênh lệch" (Difference = Actual Closing Cash - Expected Cash). \newline 8. Người dùng xem xét thông tin và nhấn nút "Xác nhận và Đóng phiên" (Validate Closing \& Post Entries). \newline 9. Hệ thống thực hiện các hành động cuối cùng: \newline    a. Cập nhật trạng thái phiên POS thành "Closed". \newline    b. Ghi nhận số tiền mặt thực tế và chênh lệch. \newline    c. Tạo/Xác nhận các bút toán kế toán liên quan đến doanh thu và thanh toán của phiên. \newline 10. Hệ thống hiển thị thông báo "Phiên đã được đóng thành công." \newline 11. Người dùng được chuyển về màn hình quản lý phiên hoặc màn hình chính Odoo. \\
\hline
Alternative Flow & \textbf{Basic Flow (Không kiểm soát tiền mặt):} \newline    1. Các bước 1-3 tương tự. \newline    2. Hệ thống hiển thị màn hình tóm tắt doanh thu theo phương thức thanh toán (bỏ qua các bước liên quan đến đối chiếu tiền mặt). \newline    3. Người dùng nhấn nút "Xác nhận và Đóng phiên". \newline    4. Hệ thống thực hiện bước 9a và 9c. \newline    5. Use Case tiếp tục từ bước 10. \newline \textbf{8a. Ghi chú chênh lệch:} \newline    1. Nếu có chênh lệch tiền mặt (bước 7), hệ thống có thể yêu cầu hoặc cho phép người dùng nhập lý do/ghi chú cho khoản chênh lệch đó trước khi xác nhận đóng phiên. \\
\hline
Exception Flow & \textbf{3a. Còn đơn hàng đang mở:} \newline    1. Hệ thống phát hiện vẫn còn đơn hàng chưa thanh toán hoặc chưa đóng. \newline    2. Hệ thống hiển thị danh sách các đơn hàng đó và yêu cầu người dùng xử lý trước. \newline    3. Việc đóng phiên bị tạm dừng cho đến khi tất cả đơn hàng được xử lý. \newline \textbf{9d. Lỗi tạo bút toán kế toán:} \newline    1. Hệ thống gặp lỗi khi cố gắng tạo hoặc xác nhận các bút toán trong module Kế toán (ví dụ: lỗi cấu hình tài khoản, lỗi ghi dữ liệu). \newline    2. Hệ thống báo lỗi chi tiết (thường hiển thị cho người dùng có quyền kế toán/quản trị). \newline    3. Phiên POS có thể vẫn được đóng nhưng các bút toán chưa được ghi nhận đúng. Cần sự can thiệp của kế toán/quản trị viên để khắc phục. \newline \textbf{9e. Lỗi hệ thống chung khi đóng phiên:} \newline    1. Hệ thống gặp sự cố kỹ thuật khác trong quá trình đóng phiên. \newline    2. Hệ thống báo lỗi chung. Phiên có thể vẫn ở trạng thái "In Progress". \\
\hline
\multicolumn{2}{|c|}{\textbf{2.3. Thông tin bổ sung (Additional Information)}} \\
\hline
\textbf{Mục} & \textbf{Nội dung} \\
\hline
Business Rule & - \textbf{BR-UC5.13-1:} Trước khi đóng phiên, tất cả các đơn hàng thuộc phiên đó phải ở trạng thái cuối cùng (Paid hoặc Cancelled). \newline - \textbf{BR-UC5.13-2:} Nếu sử dụng kiểm soát tiền mặt, việc đối chiếu tiền mặt cuối ca là bắt buộc. Khoản chênh lệch (nếu có) phải được ghi nhận. \newline - \textbf{BR-UC5.13-3:} Việc đóng phiên là hành động cuối cùng chốt lại các giao dịch và doanh thu của phiên đó. Sau khi đóng, không thể mở lại để sửa đổi giao dịch thuộc phiên đó. \newline - \textbf{BR-UC5.13-4:} Việc đóng phiên phải kích hoạt việc tạo/xác nhận các bút toán tương ứng trong module Kế toán để đảm bảo dữ liệu tài chính được cập nhật. \\
\hline
Non-Functional Requirement & - \textbf{NFR-UC5.13-1 (Usability):} Quy trình đóng phiên phải rõ ràng. Màn hình tóm tắt và đối chiếu tiền mặt (nếu có) phải dễ hiểu. \newline - \textbf{NFR-UC5.13-2 (Performance):} Việc tính toán tổng kết và đóng phiên (bao gồm cả tạo bút toán kế toán) nên được thực hiện trong thời gian hợp lý (ví dụ: dưới 10-15 giây tùy thuộc số lượng giao dịch). \newline - \textbf{NFR-UC5.13-3 (Accuracy):} Mọi số liệu tổng kết (doanh thu, tiền mặt dự kiến, chênh lệch) và các bút toán kế toán được tạo ra phải chính xác tuyệt đối. \newline - \textbf{NFR-UC5.13-4 (Auditability):} Mọi thông tin về phiên làm việc (số dư đầu/cuối, chênh lệch, người đóng phiên, thời gian đóng) phải được lưu trữ đầy đủ để phục vụ kiểm toán. \\
\hline
\end{longtable}

\subsubsection{Use Case UC-MD05-14: Chuyển bàn/Ghép bàn}

\begin{longtable}{|m{4cm}|p{11cm}|}
\caption{Đặc tả Use Case UC-MD05-14: Chuyển bàn/Ghép bàn} \label{tab:uc_md05_14} \\
\hline

\endhead % Header cho các trang tiếp theo
\hline
\endfoot % Footer cho bảng
\hline
\endlastfoot % Footer cho trang cuối cùng
\multicolumn{2}{|c|}{\textbf{2.1. Tóm tắt (Summary)}} \\
\hline
\textbf{Mục} & \textbf{Nội dung} \\
\hline
Use Case Name & Chuyển bàn/Ghép bàn \\
\hline
Use Case ID & UC-MD05-14 \\
\hline
Use Case Description & Cung cấp chức năng cho phép Nhân viên (Quản lý, Phục vụ) di chuyển một đơn hàng đang hoạt động từ bàn này sang một bàn khác (Chuyển bàn - Transfer) hoặc gộp nhiều đơn hàng từ nhiều bàn khác nhau vào một bàn duy nhất (Ghép bàn - Merge). \\
\hline
Actor & US-01 (Quản lý nhà hàng), US-02 (Nhân viên phục vụ) \\
\hline
Priority & Should Have \\
\hline
Trigger & - Khách hàng yêu cầu chuyển sang một bàn khác. \newline - Nhiều nhóm khách hàng ngồi riêng lẻ muốn gộp lại và thanh toán chung một hóa đơn. \newline - Nhân viên cần sắp xếp lại bàn để tối ưu không gian. \\
\hline
Pre-Condition & - Nhân viên đã đăng nhập và đang trong phiên POS hoạt động. \newline - Có ít nhất một đơn hàng đang hoạt động tại một bàn (trạng thái Occupied). \newline - \textit{(Cho Chuyển bàn):} Có một bàn khác đang trống (Available). \newline - \textit{(Cho Ghép bàn):} Có ít nhất hai đơn hàng đang hoạt động ở các bàn khác nhau. \\
\hline
Post-Condition & - \textbf{Chuyển bàn:} Đơn hàng được chuyển thành công sang bàn mới. Bàn cũ trở thành trống, bàn mới trở thành Occupied với đơn hàng đó. \newline - \textbf{Ghép bàn:} Tất cả các món ăn từ các đơn hàng nguồn được gộp vào đơn hàng của bàn đích. Các bàn nguồn trở thành trống. Đơn hàng tại bàn đích chứa tất cả món ăn đã gộp. \\
\hline
\multicolumn{2}{|c|}{\textbf{2.2. Luồng thực thi (Flow)}} \\
\hline
\textbf{Mục} & \textbf{Nội dung} \\
\hline
Basic Flow (Chuyển bàn - Transfer) & 1. Nhân viên (US-01/US-02) đang ở màn hình đơn hàng POS của bàn cần chuyển (Bàn Nguồn). \newline 2. Nhân viên chọn chức năng "Chuyển bàn" (Transfer). \newline 3. Hệ thống hiển thị lại Sơ đồ tầng (Floor Plan). \newline 4. Nhân viên chọn bàn đích (Bàn Đích) đang ở trạng thái "Trống". \newline 5. Hệ thống thực hiện chuyển đơn hàng từ Bàn Nguồn sang Bàn Đích. \newline 6. Hệ thống cập nhật trạng thái: Bàn Nguồn thành "Trống", Bàn Đích thành "Occupied". \newline 7. Hệ thống quay lại màn hình đơn hàng, giờ đây liên kết với Bàn Đích. \newline 8. Hệ thống hiển thị thông báo chuyển bàn thành công. \\
\hline
Alternative Flow & \textbf{Basic Flow (Ghép bàn - Merge):} \newline    1. Nhân viên đang ở màn hình Sơ đồ tầng. \newline    2. Nhân viên chọn chức năng "Ghép bàn" (Merge) (có thể cần vào chế độ Edit hoặc menu riêng). \newline    3. Nhân viên chọn (các) bàn nguồn có đơn hàng muốn ghép (Bàn Nguồn 1, Bàn Nguồn 2...). \newline    4. Nhân viên chọn bàn đích sẽ nhận tất cả các món (Bàn Đích - thường là một trong các bàn nguồn hoặc một bàn khác). \newline    5. Hệ thống yêu cầu xác nhận hành động ghép. Nhân viên xác nhận. \newline    6. Hệ thống di chuyển tất cả các món ăn từ đơn hàng của (các) Bàn Nguồn vào đơn hàng của Bàn Đích. \newline    7. Hệ thống đóng các đơn hàng ở (các) Bàn Nguồn (nếu cần) và cập nhật trạng thái (các) Bàn Nguồn thành "Trống". \newline    8. Đơn hàng tại Bàn Đích giờ chứa tất cả các món đã gộp. \newline    9. Hệ thống hiển thị thông báo ghép bàn thành công. Nhân viên có thể mở Bàn Đích để xem đơn hàng tổng hợp. \\
\hline
Exception Flow & \textbf{Chuyển bàn - 4a. Chọn bàn đích không hợp lệ:} \newline    1. Nhân viên chọn một bàn đích không trống (Occupied, Reserved). \newline    2. Hệ thống báo lỗi "Không thể chuyển đến bàn này. Bàn không trống." \newline    3. Use Case quay lại bước 4. \newline \textbf{Ghép bàn - 6a. Lỗi khi gộp món ăn:} \newline    1. Hệ thống gặp lỗi khi cố gắng di chuyển/gộp các dòng món ăn giữa các đơn hàng. \newline    2. Hệ thống báo lỗi. Hành động ghép có thể thất bại hoặc chỉ thành công một phần. Cần kiểm tra lại các đơn hàng. \newline \textbf{Chung - Lỗi cập nhật trạng thái bàn/đơn hàng:} \newline    1. Hệ thống gặp lỗi kỹ thuật khi cập nhật trạng thái bàn hoặc liên kết đơn hàng. \newline    2. Hệ thống báo lỗi. Trạng thái có thể không nhất quán. \\
\hline
\multicolumn{2}{|c|}{\textbf{2.3. Thông tin bổ sung (Additional Information)}} \\
\hline
\textbf{Mục} & \textbf{Nội dung} \\
\hline
Business Rule & - \textbf{BR-UC5.14-1:} Chỉ có thể chuyển đơn hàng đến một bàn đang trống. \newline - \textbf{BR-UC5.14-2:} Khi ghép bàn, tất cả các món ăn từ các bàn nguồn sẽ được cộng dồn vào bàn đích. Cần đảm bảo không mất mát dữ liệu món ăn. \newline - \textbf{BR-UC5.14-3:} Nếu các đơn hàng được ghép có liên kết với các lượt đặt chỗ khác nhau (có thể có tiền cọc khác nhau), hệ thống cần có logic xử lý việc gộp này một cách hợp lý (ví dụ: chỉ cho phép ghép nếu cùng một khách hàng, hoặc cần xử lý tiền cọc riêng). Trường hợp này phức tạp và cần làm rõ yêu cầu. Mặc định có thể chỉ cho ghép các đơn hàng không có đặt cọc hoặc cùng một đặt cọc. \newline - \textbf{BR-UC5.14-4:} Hành động chuyển/ghép bàn cần được ghi nhận vào lịch sử đơn hàng. \\
\hline
Non-Functional Requirement & - \textbf{NFR-UC5.14-1 (Usability):} Chức năng chuyển/ghép bàn phải dễ sử dụng, các bước chọn bàn nguồn/đích phải rõ ràng. \newline - \textbf{NFR-UC5.14-2 (Performance):} Thao tác chuyển/ghép phải diễn ra nhanh chóng. \newline - \textbf{NFR-UC5.14-3 (Data Integrity):} Phải đảm bảo tất cả các món ăn và thông tin liên quan được chuyển/gộp chính xác, không bị mất mát hoặc trùng lặp sai. Trạng thái bàn phải được cập nhật đúng. \\
\hline
\end{longtable}

\subsubsection{Use Case UC-MD05-15: Hủy món/Hủy đơn (Void)}

\begin{longtable}{|m{4cm}|p{11cm}|}
\caption{Đặc tả Use Case UC-MD05-15: Hủy món/Hủy đơn (Void)} \label{tab:uc_md05_15} \\
\hline

\endhead % Header cho các trang tiếp theo
\hline
\endfoot % Footer cho bảng
\hline
\endlastfoot % Footer cho trang cuối cùng
\multicolumn{2}{|c|}{\textbf{2.1. Tóm tắt (Summary)}} \\
\hline
\textbf{Mục} & \textbf{Nội dung} \\
\hline
Use Case Name & Hủy món/Hủy đơn (Void) \\
\hline
Use Case ID & UC-MD05-15 \\
\hline
Use Case Description & Cho phép Nhân viên (thường cần quyền Quản lý hoặc được cấp phép đặc biệt) hủy bỏ một món ăn cụ thể đã được thêm vào đơn hàng (Void Item) hoặc hủy bỏ toàn bộ đơn hàng đang hoạt động (Void Order), thường do lỗi nhập liệu hoặc khách hàng đổi ý. Hành động này cần được ghi nhận lại. \\
\hline
Actor & US-01 (Quản lý nhà hàng), US-02 (Nhân viên phục vụ - có thể cần quyền) \\
\hline
Priority & Must Have \\
\hline
Trigger & - Nhân viên nhập sai món hoặc sai số lượng và cần hủy bỏ món đó. \newline - Khách hàng đổi ý, không muốn lấy món đã gọi nữa. \newline - Cần hủy toàn bộ đơn hàng do một lý do đặc biệt (ví dụ: khách rời đi đột ngột, lỗi hệ thống...). \\
\hline
Pre-Condition & - Nhân viên đã đăng nhập và đang trong phiên POS hoạt động. \newline - Đang ở màn hình đơn hàng POS có món cần hủy hoặc đơn hàng cần hủy. \newline - Người dùng có quyền thực hiện hành động Void. \\
\hline
Post-Condition & - \textbf{Hủy món:} Món ăn được chọn bị loại bỏ khỏi đơn hàng hoặc được đánh dấu là đã hủy (với số lượng và giá trị bị điều chỉnh về 0 hoặc âm). Tổng tiền đơn hàng được cập nhật. \newline - \textbf{Hủy đơn:} Toàn bộ đơn hàng được đánh dấu là đã hủy (Cancelled/Voided). Bàn liên kết trở thành trống. \newline - Hành động hủy (món hoặc đơn) và lý do (nếu có) được ghi nhận vào hệ thống để kiểm soát và báo cáo. \\
\hline
\multicolumn{2}{|c|}{\textbf{2.2. Luồng thực thi (Flow)}} \\
\hline
\textbf{Mục} & \textbf{Nội dung} \\
\hline
Basic Flow (Hủy món - Void Item) & 1. Nhân viên (US-01/US-02) đang ở màn hình đơn hàng POS. \newline 2. Nhân viên chọn dòng món ăn cần hủy. \newline 3. Nhân viên chọn tùy chọn/nút "Hủy món" / "Void Item" / "Remove" (có thể cần nhập số lượng muốn hủy nếu SL > 1). \newline 4. Hệ thống (có thể) yêu cầu nhập lý do hủy hoặc yêu cầu xác nhận quyền quản lý (ví dụ: nhập mã PIN của quản lý). \newline 5. Người dùng cung cấp thông tin/xác nhận cần thiết. \newline 6. Hệ thống loại bỏ món ăn khỏi danh sách hoặc đánh dấu là đã hủy (ví dụ: gạch ngang, giá trị âm). \newline 7. Hệ thống cập nhật lại tổng tiền của đơn hàng. \newline 8. Hệ thống ghi nhận hành động hủy món và lý do (nếu có) vào log. \\
\hline
Alternative Flow & \textbf{Basic Flow (Hủy đơn - Void Order):} \newline    1. Nhân viên đang ở màn hình đơn hàng POS. \newline    2. Nhân viên chọn nút/tùy chọn "Hủy đơn hàng" / "Void Order" / "Delete Order". \newline    3. Hệ thống yêu cầu xác nhận hành động hủy toàn bộ đơn hàng, có thể yêu cầu lý do hoặc quyền quản lý. \newline    4. Người dùng cung cấp thông tin/xác nhận cần thiết. \newline    5. Hệ thống cập nhật trạng thái đơn hàng thành "Cancelled" hoặc "Voided". \newline    6. Hệ thống cập nhật trạng thái bàn liên kết thành "Trống". \newline    7. Hệ thống ghi nhận hành động hủy đơn và lý do vào log. \newline    8. Hệ thống thường chuyển về màn hình Sơ đồ tầng. \\
\hline
Exception Flow & \textbf{4a/3a-cancel. Không có quyền hủy:} \newline    1. Nhân viên không có quyền thực hiện hành động hủy món/đơn. \newline    2. Hệ thống báo lỗi "Bạn không có quyền thực hiện hành động này" hoặc yêu cầu xác thực của quản lý nhưng nhân viên không cung cấp được. \newline    3. Hành động hủy không được thực hiện. \newline \textbf{6a/5a-cancel. Lỗi hệ thống khi hủy:} \newline    1. Hệ thống gặp lỗi kỹ thuật khi cố gắng cập nhật trạng thái món ăn hoặc đơn hàng. \newline    2. Hệ thống báo lỗi chung. Hành động hủy có thể không thành công. \\
\hline
\multicolumn{2}{|c|}{\textbf{2.3. Thông tin bổ sung (Additional Information)}} \\
\hline
\textbf{Mục} & \textbf{Nội dung} \\
\hline
Business Rule & - \textbf{BR-UC5.15-1:} Hành động hủy món/đơn nên yêu cầu quyền hạn đặc biệt (quản lý hoặc nhân viên cấp cao) để tránh lạm dụng và kiểm soát thất thoát. \newline - \textbf{BR-UC5.15-2:} Việc hủy món/đơn phải được ghi log chi tiết: người thực hiện, thời gian, món/đơn bị hủy, lý do (nếu có). \newline - \textbf{BR-UC5.15-3:} Nếu hủy một món đã được gửi xuống bếp, cần có quy trình thông báo cho bếp biết để ngừng chế biến (có thể là thông báo tự động qua KDS/in phiếu hủy hoặc thông báo thủ công). \newline - \textbf{BR-UC5.15-4:} Việc hủy đơn hàng phải giải phóng bàn liên quan. \\
\hline
Non-Functional Requirement & - \textbf{NFR-UC5.15-1 (Security/Auditability):} Chức năng hủy phải được kiểm soát chặt chẽ về quyền hạn và mọi hành động phải được ghi log đầy đủ. \newline - \textbf{NFR-UC5.15-2 (Usability):} Thao tác hủy phải rõ ràng nhưng cũng cần có bước xác nhận để tránh hủy nhầm. Yêu cầu nhập lý do (nếu bắt buộc) cần thuận tiện. \newline - \textbf{NFR-UC5.15-3 (Performance):} Hành động hủy và cập nhật trạng thái phải nhanh chóng. \\
\hline
\end{longtable}


\subsection{Module MD-06: Quản lý Bán mang về (POS - Takeout)}

\subsubsection{Use Case UC-MD06-01: Chọn Chế độ Bán Mang về}
% (Nội dung được giữ nguyên từ phản hồi trước vì đã đúng, chỉ cần đảm bảo Actor là người dùng)
\begin{longtable}{|m{4cm}|p{11cm}|}
\caption{Đặc tả Use Case UC-MD06-01: Chọn Chế độ Bán Mang về} \label{tab:uc_md06_01_final_v3} \\
\hline
\multicolumn{2}{|c|}{\textbf{2.1. Tóm tắt (Summary)}} \\
\hline
\textbf{Mục} & \textbf{Nội dung} \\
\hline
\endhead % Header cho các trang tiếp theo
\hline
\endfoot % Footer cho bảng
\hline
\endlastfoot % Footer cho trang cuối cùng
Use Case Name & Chọn Chế độ Bán Mang về \\
\hline
Use Case ID & UC-MD06-01 \\
\hline
Use Case Description & Cho phép Nhân viên (US-02: Phục vụ hoặc US-05: Thu ngân) tại điểm bán hàng (POS) lựa chọn một chế độ hoạt động hoặc một giao diện được thiết kế riêng cho việc tiếp nhận và xử lý các đơn hàng khách mua mang đi (Takeout/Takeaway). \\
\hline
Actor & US-02 (Nhân viên phục vụ), US-05 (Nhân viên thu ngân) \\
\hline
Priority & Must Have \\
\hline
Trigger & Có khách hàng đến quầy để đặt món mang về. \\
\hline
Pre-Condition & - Nhân viên đã đăng nhập và đang trong phiên POS hoạt động (UC-MD05-01). \newline - Giao diện POS chính (ví dụ: Sơ đồ tầng hoặc màn hình chờ) đang hiển thị. \newline - Nút chức năng "Bán Mang về" (Takeout) đã được cấu hình và hiển thị trên giao diện POS. \\
\hline
Post-Condition & - Hệ thống chuyển sang giao diện hoặc chế độ dành riêng cho việc tạo đơn hàng mang về theo lựa chọn của Nhân viên. \newline - Giao diện này sẵn sàng để Nhân viên bắt đầu tạo đơn hàng mới (UC-MD06-02). \\
\hline
\multicolumn{2}{|c|}{\textbf{2.2. Luồng thực thi (Flow)}} \\
\hline
\textbf{Mục} & \textbf{Nội dung} \\
\hline
Basic Flow & 1. Nhân viên (US-02/US-05) đang ở giao diện POS chính. \newline 2. Nhân viên xác định vị trí nút hoặc tùy chọn "Bán Mang về" (Takeout / Takeaway) trên màn hình. \newline 3. Nhân viên nhấp vào nút "Bán Mang về". \newline 4. Hệ thống (System) phản hồi bằng cách chuyển đổi giao diện hoặc ngữ cảnh sang chế độ bán mang về. Giao diện này có thể: \newline    a. Tự động mở ngay một đơn hàng mới ở chế độ mang về. \newline    b. Hoặc hiển thị một danh sách các đơn hàng mang về đang chờ xử lý (nếu có) và nút "Tạo đơn mới". \\
\hline
Alternative Flow & \textbf{1a. Truy cập từ Menu chính/Dashboard của POS:} \newline    1. Nhân viên chọn tùy chọn "Bán Mang về" từ menu chính hoặc dashboard. \newline    2. Use Case tiếp tục từ bước 3. \\
\hline
Exception Flow & \textbf{3a. Nút/Chức năng bị vô hiệu hóa hoặc không tồn tại:} \newline    1. Nhân viên không tìm thấy hoặc không thể nhấp vào nút "Bán Mang về" (do lỗi cấu hình hoặc thiếu quyền). \newline    2. Hệ thống không thay đổi giao diện. Nhân viên cần báo quản lý. \newline \textbf{4c. Lỗi hệ thống khi chuyển đổi giao diện/chế độ:} \newline    1. Hệ thống gặp lỗi kỹ thuật khi tải giao diện bán mang về. \newline    2. Hệ thống hiển thị thông báo lỗi. Nhân viên không thể tiếp tục. \\
\hline
\multicolumn{2}{|c|}{\textbf{2.3. Thông tin bổ sung (Additional Information)}} \\
\hline
\textbf{Mục} & \textbf{Nội dung} \\
\hline
Business Rule & - \textbf{BR-UC6.1-1:} Phải có một cách thức rõ ràng (nút bấm, menu) để nhân viên chủ động chuyển sang chế độ xử lý đơn hàng mang về. \newline - \textbf{BR-UC6.1-2 (System):} Chế độ bán mang về phải bỏ qua hoàn toàn quy trình liên quan đến quản lý bàn. \\
\hline
Non-Functional Requirement & - \textbf{NFR-UC6.1-1 (Usability):} Nút/Tùy chọn "Bán Mang về" phải dễ dàng tìm thấy và nhận biết. Việc chuyển đổi chế độ phải nhanh chóng. \newline - \textbf{NFR-UC6.1-2 (Performance):} Thời gian hệ thống phản hồi và chuyển giao diện phải rất ngắn (< 1-2 giây). \\
\hline
\end{longtable}

\subsubsection{Use Case UC-MD06-02: Tạo Đơn hàng Mang về Mới}
% (Nội dung được giữ nguyên từ phản hồi trước vì đã đúng, chỉ cần đảm bảo Actor là người dùng)
\begin{longtable}{|m{4cm}|p{11cm}|}
\caption{Đặc tả Use Case UC-MD06-02: Tạo Đơn hàng Mang về Mới} \label{tab:uc_md06_02_final_v3} \\
\hline
\multicolumn{2}{|c|}{\textbf{2.1. Tóm tắt (Summary)}} \\
\hline
\textbf{Mục} & \textbf{Nội dung} \\
\hline
\endhead % Header cho các trang tiếp theo
\hline
\endfoot % Footer cho bảng
\hline
\endlastfoot % Footer cho trang cuối cùng
Use Case Name & Tạo Đơn hàng Mang về Mới \\
\hline
Use Case ID & UC-MD06-02 \\
\hline
Use Case Description & Cho phép Nhân viên (US-02/US-05) khởi tạo một đơn hàng Point of Sale mới, được hệ thống tự động đánh dấu là loại hình "Mang về" (Takeout), sau khi đã vào chế độ bán mang về. \\
\hline
Actor & US-02 (Nhân viên phục vụ), US-05 (Nhân viên thu ngân) \\
\hline
Priority & Must Have \\
\hline
Trigger & Nhân viên đã chọn chế độ Bán Mang về (UC-MD06-01) và cần tạo một đơn hàng mới cho khách (hoặc hệ thống tự động tạo khi vào chế độ). \\
\hline
Pre-Condition & - Nhân viên đang ở trong chế độ/giao diện Bán Mang về trên POS (UC-MD06-01 thành công). \\
\hline
Post-Condition & - Một bản ghi đơn hàng POS mới được hệ thống tạo ra với loại hình là "Mang về". \newline - Đơn hàng này không liên kết với bàn nào. \newline - Giao diện đơn hàng được hiển thị, sẵn sàng để Nhân viên thêm món (UC-MD06-04) và/hoặc liên kết khách hàng (UC-MD06-03). \\
\hline
\multicolumn{2}{|c|}{\textbf{2.2. Luồng thực thi (Flow)}} \\
\hline
\textbf{Mục} & \textbf{Nội dung} \\
\hline
Basic Flow & 1. Tiếp nối từ UC-MD06-01, Nhân viên (US-02/US-05) đang ở giao diện Bán Mang về. \newline 2. Nhân viên chọn hành động "Tạo đơn mới" (New Takeout Order) (nếu giao diện UC-MD06-01, bước 4b hiển thị màn hình quản lý và yêu cầu hành động này). \newline 3. Hệ thống (System) tạo một bản ghi đơn hàng POS mới và tự động gán loại hình là "Takeout". \newline 4. Hệ thống hiển thị giao diện đơn hàng cho Nhân viên, bao gồm khu vực danh sách món (trống), khu vực chọn món, và các nút chức năng. \\
\hline
Alternative Flow & \textbf{2a. Hệ thống tự động tạo đơn khi vào chế độ Mang về:} \newline    1. Nếu luồng UC-MD06-01 (bước 4a) được thiết kế để tự động mở đơn mới. \newline    2. Nhân viên không cần nhấn nút "Tạo đơn mới". Hệ thống trực tiếp thực hiện bước 3 và 4. \\
\hline
Exception Flow & \textbf{3a. Lỗi hệ thống khi tạo bản ghi đơn hàng mới:} \newline    1. Hệ thống gặp lỗi kỹ thuật khi tạo đơn hàng. \newline    2. Hệ thống hiển thị thông báo lỗi. Nhân viên không thể tiếp tục. \\
\hline
\multicolumn{2}{|c|}{\textbf{2.3. Thông tin bổ sung (Additional Information)}} \\
\hline
\textbf{Mục} & \textbf{Nội dung} \\
\hline
Business Rule & - \textbf{BR-UC6.2-1 (System):} Đơn hàng tạo ra từ chế độ "Bán Mang về" phải được hệ thống phân loại đúng là "Takeout". \newline - \textbf{BR-UC6.2-2 (System):} Đơn hàng mang về không được gắn với bàn. \\
\hline
Non-Functional Requirement & - \textbf{NFR-UC6.2-1 (Performance):} Tạo đơn mới và hiển thị giao diện phải tức thời (< 1 giây). \newline - \textbf{NFR-UC6.2-2 (Usability):} Giao diện đơn hàng mang về rõ ràng. \\
\hline
\end{longtable}

\subsubsection{Use Case UC-MD06-03: Gán Khách hàng vào Đơn Mang về}
% (Nội dung được giữ nguyên từ phản hồi trước vì đã đúng, chỉ cần đảm bảo Actor là người dùng)
\begin{longtable}{|m{4cm}|p{11cm}|}
\caption{Đặc tả Use Case UC-MD06-03:  Gán Khách hàng vào Đơn Mang về} \label{tab:uc_md06_03_final_v3} \\
\hline
\multicolumn{2}{|c|}{\textbf{2.1. Tóm tắt (Summary)}} \\
\hline
\textbf{Mục} & \textbf{Nội dung} \\
\hline
\endhead % Header cho các trang tiếp theo
\hline
\endfoot % Footer cho bảng
\hline
\endlastfoot % Footer cho trang cuối cùng
Use Case Name & (Tùy chọn) Gán Khách hàng vào Đơn Mang về \\
\hline
Use Case ID & UC-MD06-03 \\
\hline
Use Case Description & Cho phép Nhân viên (US-02: Phục vụ hoặc US-05: Thu ngân) tại POS tìm kiếm và chọn một khách hàng đã tồn tại trong cơ sở dữ liệu hoặc tạo nhanh thông tin khách hàng mới (Tên, SĐT) để liên kết với đơn hàng mang về đang được xử lý. \\
\hline
Actor & US-02 (Nhân viên phục vụ), US-05 (Nhân viên thu ngân) \\
\hline
Priority & Should Have \\
\hline
Trigger & - Nhân viên muốn gắn đơn hàng mang về với một khách hàng cụ thể. \newline - Khách hàng cung cấp thông tin. \\
\hline
Pre-Condition & - Nhân viên đang xử lý một đơn hàng mang về (UC-MD06-02 thành công). \newline - Giao diện POS có nút/chức năng để chọn/thêm khách hàng. \\
\hline
Post-Condition & - Đơn hàng mang về được liên kết với một bản ghi khách hàng. \newline - Tên khách hàng hiển thị trên giao diện đơn hàng. \\
\hline
\multicolumn{2}{|c|}{\textbf{2.2. Luồng thực thi (Flow)}} \\
\hline
\textbf{Mục} & \textbf{Nội dung} \\
\hline
Basic Flow (Chọn khách hàng đã có) & 1. Nhân viên (US-02/US-05) đang ở màn hình đơn hàng mang về. \newline 2. Nhân viên nhấp vào nút/khu vực "Chọn khách hàng". \newline 3. Hệ thống hiển thị giao diện tìm kiếm/chọn khách hàng. \newline 4. Nhân viên nhập tên, SĐT hoặc mã khách hàng. \newline 5. Hệ thống hiển thị danh sách khách hàng khớp. \newline 6. Nhân viên chọn đúng khách hàng. \newline 7. Hệ thống liên kết khách hàng với đơn hàng. \newline 8. Tên khách hàng hiển thị trên đơn hàng. \\
\hline
Alternative Flow & \textbf{4a. Tạo khách hàng mới nhanh chóng từ POS:} \newline    1. Nếu không tìm thấy, Nhân viên chọn "Tạo mới". \newline    2. Hệ thống hiển thị form nhập Tên, SĐT (bắt buộc). \newline    3. Nhân viên nhập thông tin. \newline    4. Nhân viên nhấn "Lưu". \newline    5. Hệ thống tạo khách hàng mới và tự động liên kết. Use Case tiếp tục từ bước 8. \newline \textbf{1a. Bỏ qua việc liên kết khách hàng:} \newline    1. Nhân viên bỏ qua và tiếp tục thêm món/thanh toán. Đơn hàng không có khách hàng cụ thể (hoặc gán khách vãng lai). \\
\hline
Exception Flow & \textbf{5a. Tìm thấy nhiều khách hàng trùng:} Nhân viên cần xác minh thêm. \newline \textbf{Alternative Flow 4a - Step 4a. Lỗi tạo khách hàng mới.} \newline \textbf{7a. Lỗi hệ thống khi liên kết khách hàng.} \\
\hline
\multicolumn{2}{|c|}{\textbf{2.3. Thông tin bổ sung (Additional Information)}} \\
\hline
\textbf{Mục} & \textbf{Nội dung} \\
\hline
Business Rule & - \textbf{BR-UC6.3-1:} Liên kết khách hàng là tùy chọn (trừ khi có chính sách đặc biệt). \newline - \textbf{BR-UC6.3-2:} Khi tạo mới, Tên và SĐT là tối thiểu. \newline - \textbf{BR-UC6.3-3 (System):} Dữ liệu khách hàng quản lý tập trung. \\
\hline
Non-Functional Requirement & - \textbf{NFR-UC6.3-1 (Usability):} Tìm kiếm/chọn/tạo khách hàng phải nhanh, dễ. \newline - \textbf{NFR-UC6.3-2 (Performance):} Phản hồi tìm kiếm/liên kết nhanh. \newline - \textbf{NFR-UC6.3-3 (Data Consistency):} Khuyến khích dùng lại hồ sơ cũ, tránh trùng lặp. \\
\hline
\end{longtable}

\subsubsection{Use Case UC-MD06-04: Thêm Món vào Đơn hàng Mang về}
\begin{longtable}{|m{4cm}|p{11cm}|}
\caption{Đặc tả Use Case UC-MD06-04: Thêm Món vào Đơn hàng Mang về} \label{tab:uc_md06_04_final_v3} \\
\hline
\multicolumn{2}{|c|}{\textbf{2.1. Tóm tắt (Summary)}} \\
\hline
\textbf{Mục} & \textbf{Nội dung} \\
\hline
\endhead % Header cho các trang tiếp theo
\hline
\endfoot % Footer cho bảng
\hline
\endlastfoot % Footer cho trang cuối cùng
Use Case Name & Thêm Món vào Đơn hàng Mang về \\
\hline
Use Case ID & UC-MD06-04 \\
\hline
Use Case Description & Cho phép Nhân viên (US-02/US-05) thêm các món ăn và đồ uống vào đơn hàng mang về đang mở, sử dụng giao diện chọn sản phẩm tương tự như khi xử lý đơn hàng tại bàn (UC-MD05-05). \\
\hline
Actor & US-02 (Nhân viên phục vụ), US-05 (Nhân viên thu ngân) \\
\hline
Priority & Must Have \\
\hline
Trigger & Khách hàng đang đặt món mang về tại quầy và yêu cầu thêm món. \\
\hline
Pre-Condition & - Nhân viên đang ở màn hình đơn hàng mang về (UC-MD06-02 thành công). \newline - Giao diện POS hiển thị các danh mục và sản phẩm phù hợp. \\
\hline
Post-Condition & - Món ăn/đồ uống được chọn (cùng số lượng và biến thể nếu có) được thêm vào danh sách các món đã gọi của đơn hàng mang về. \newline - Tổng tiền tạm tính của đơn hàng được cập nhật. \newline - Món ăn mới thêm sẵn sàng để được gửi xuống bếp/bar (UC-MD06-06). \\
\hline
\multicolumn{2}{|c|}{\textbf{2.2. Luồng thực thi (Flow)}} \\
\hline
\textbf{Mục} & \textbf{Nội dung} \\
\hline
Basic Flow, Alternative Flow, Exception Flow & Hành động của Nhân viên khi thêm món vào đơn hàng mang về (duyệt danh mục, chọn sản phẩm, chọn biến thể, tìm kiếm, thay đổi số lượng ban đầu) về cơ bản là **giống hệt** với **Use Case UC-MD05-05: Thêm Món mới vào Đơn hàng POS** và **Use Case UC-MD05-06: Điều chỉnh Số lượng Món trong Đơn hàng POS** (cho phần tăng số lượng khi thêm). \\
\hline
\multicolumn{2}{|c|}{\textbf{2.3. Thông tin bổ sung (Additional Information)}} \\
\hline
\textbf{Mục} & \textbf{Nội dung} \\
\hline
Business Rule & Các Business Rule về hiển thị sản phẩm, chọn biến thể, giá cả tương tự như BR-UC5.5-1, BR-UC5.5-2, BR-UC5.5-3. \\
\hline
Non-Functional Requirement & Các Non-Functional Requirement về Usability, Performance, Accuracy tương tự như NFR-UC5.5-1, NFR-UC5.5-2, NFR-UC5.5-3. \\
\hline
\end{longtable}

\subsubsection{Use Case UC-MD06-05: Thêm Ghi chú cho Đơn Mang về}
\begin{longtable}{|m{4cm}|p{11cm}|}
\caption{Đặc tả Use Case UC-MD06-05: Thêm Ghi chú cho Đơn Mang về} \label{tab:uc_md06_05_final_v3} \\
\hline
\multicolumn{2}{|c|}{\textbf{2.1. Tóm tắt (Summary)}} \\
\hline
\textbf{Mục} & \textbf{Nội dung} \\
\hline
\endhead % Header cho các trang tiếp theo
\hline
\endfoot % Footer cho bảng
\hline
\endlastfoot % Footer cho trang cuối cùng
Use Case Name & Thêm Ghi chú cho Đơn Mang về \\
\hline
Use Case ID & UC-MD06-05 \\
\hline
Use Case Description & Cho phép Nhân viên (US-02/US-05) thêm các ghi chú đặc biệt từ khách hàng (ví dụ: yêu cầu về đóng gói, khẩu vị) hoặc ghi chú nội bộ vào một món ăn cụ thể hoặc toàn bộ đơn hàng mang về. \\
\hline
Actor & US-02 (Nhân viên phục vụ), US-05 (Nhân viên thu ngân) \\
\hline
Priority & Must Have \\
\hline
Trigger & Khách hàng mua mang về có yêu cầu đặc biệt hoặc nhân viên cần ghi chú thông tin. \\
\hline
Pre-Condition & - Nhân viên đang ở màn hình đơn hàng mang về. \newline - Có ít nhất một món ăn đã được thêm vào đơn hàng (UC-MD06-04). \\
\hline
Post-Condition & - Ghi chú được đính kèm vào món ăn hoặc đơn hàng trên POS. \newline - Ghi chú sẽ được gửi cùng thông tin món ăn xuống bếp/bar (UC-MD06-06). \\
\hline
\multicolumn{2}{|c|}{\textbf{2.2. Luồng thực thi (Flow)}} \\
\hline
\textbf{Mục} & \textbf{Nội dung} \\
\hline
Basic Flow, Alternative Flow, Exception Flow & Hành động của Nhân viên khi thêm ghi chú cho đơn hàng mang về (chọn món/đơn, nhập ghi chú, chọn ghi chú có sẵn) về cơ bản là **giống hệt** với **Use Case UC-MD05-07: Thêm Ghi chú Bếp cho Món ăn/Đơn hàng POS**. \\
\hline
\multicolumn{2}{|c|}{\textbf{2.3. Thông tin bổ sung (Additional Information)}} \\
\hline
\textbf{Mục} & \textbf{Nội dung} \\
\hline
Business Rule & Các Business Rule về truyền tải ghi chú, cấu hình ghi chú sẵn tương tự như BR-UC5.7-1, BR-UC5.7-2, BR-UC5.7-3. Ghi chú có thể bao gồm yêu cầu đặc thù cho đơn mang về như "Gói riêng từng phần", "Thêm dụng cụ ăn uống". \\
\hline
Non-Functional Requirement & Các Non-Functional Requirement về Usability, Accuracy, Integration tương tự như NFR-UC5.7-1, NFR-UC5.7-2, NFR-UC5.7-3. \\
\hline
\end{longtable}

\subsubsection{Use Case UC-MD06-06: Gửi Yêu cầu Chuẩn bị Đơn Mang về (Bếp/Bar)}
\begin{longtable}{|m{4cm}|p{11cm}|}
\caption{Đặc tả Use Case UC-MD06-06: Gửi Yêu cầu Chuẩn bị Đơn Mang về (Bếp/Bar)} \label{tab:uc_md06_06_final_v3} \\
\hline
\multicolumn{2}{|c|}{\textbf{2.1. Tóm tắt (Summary)}} \\
\hline
\textbf{Mục} & \textbf{Nội dung} \\
\hline
\endhead % Header cho các trang tiếp theo
\hline
\endfoot % Footer cho bảng
\hline
\endlastfoot % Footer cho trang cuối cùng
Use Case Name & Gửi Yêu cầu Chuẩn bị Đơn Mang về (Bếp/Bar) \\
\hline
Use Case ID & UC-MD06-06 \\
\hline
Use Case Description & Cho phép Nhân viên (US-02/US-05) gửi thông tin các món ăn/đồ uống của đơn hàng mang về đến các máy in hoặc màn hình KDS tại bộ phận bếp/bar để bắt đầu chuẩn bị. Phiếu gửi đi cần chỉ rõ đây là đơn hàng mang về. \\
\hline
Actor & US-02 (Nhân viên phục vụ), US-05 (Nhân viên thu ngân) \\
\hline
Priority & Must Have \\
\hline
Trigger & Nhân viên đã nhập xong các món khách hàng mang về yêu cầu và cần thông báo cho bếp/bar. \\
\hline
Pre-Condition & - Nhân viên đang ở màn hình đơn hàng mang về. \newline - Có các món ăn chưa được gửi đi trong đơn hàng. \newline - Các thiết bị bếp/bar và quy tắc định tuyến đã được cấu hình (FR-MD02-20). \\
\hline
Post-Condition & - Hệ thống (System) gửi thông tin các món cần chuẩn bị đến đúng bộ phận bếp/bar, có đánh dấu là đơn "Takeout". \newline - Trạng thái các món trên POS được Nhân viên cập nhật (hoặc hệ thống tự động đánh dấu) là "Đã gửi bếp". \\
\hline
\multicolumn{2}{|c|}{\textbf{2.2. Luồng thực thi (Flow)}} \\
\hline
\textbf{Mục} & \textbf{Nội dung} \\
\hline
Basic Flow & Hành động của Nhân viên khi gửi yêu cầu chuẩn bị cho đơn hàng mang về (nhấn nút "Gửi bếp/Order") và các bước xử lý của hệ thống (xác định món, định tuyến, gửi lệnh in/KDS, cập nhật trạng thái món) về cơ bản là **giống hệt** với **Use Case UC-MD05-08: Gửi Các Món đã chọn xuống Bếp/Bar**. \newline Điểm khác biệt quan trọng là hệ thống (System) cần bao gồm thông tin nhận diện đây là "Đơn Mang về" (Takeout) trên dữ liệu gửi đi để bộ phận bếp/bar có thể chuẩn bị bao bì và quy trình đóng gói phù hợp. \\
\hline
Alternative Flow & Tương tự UC-MD05-08 (Tự động gửi khi thêm món, chỉ gửi món mới). \\
\hline
Exception Flow & Tương tự UC-MD05-08 (Lỗi gửi yêu cầu, Lỗi tại thiết bị vật lý). \\
\hline
\multicolumn{2}{|c|}{\textbf{2.3. Thông tin bổ sung (Additional Information)}} \\
\hline
\textbf{Mục} & \textbf{Nội dung} \\
\hline
Business Rule & Các Business Rule về định tuyến, thông tin trên phiếu/KDS, đánh dấu món đã gửi tương tự như BR-UC5.8-1, BR-UC5.8-2, BR-UC5.8-3. Bổ sung: \newline - \textbf{BR-UC6.6-1 (System):} Phiếu in/Hiển thị KDS cho đơn mang về phải có dấu hiệu rõ ràng để phân biệt với đơn ăn tại bàn (ví dụ: chữ "Takeout", "Mang về", có thể kèm tên khách nếu được liên kết ở UC-MD06-03). \\
\hline
Non-Functional Requirement & Các Non-Functional Requirement về Performance, Reliability, Integration tương tự như NFR-UC5.8-1, NFR-UC5.8-2, NFR-UC5.8-3. \\
\hline
\end{longtable}

\subsubsection{Use Case UC-MD06-07: Xác nhận và Tiến hành Thanh toán Đơn Mang về (có xem xét Cọc/Trả trước)}
\begin{longtable}{|m{4cm}|p{11cm}|}
\caption{Đặc tả Use Case UC-MD06-07: Xác nhận và Tiến hành Thanh toán Đơn Mang về (có xem xét Cọc/Trả trước)} \label{tab:uc_md06_07_final_v3} \\
\hline
\multicolumn{2}{|c|}{\textbf{2.1. Tóm tắt (Summary)}} \\
\hline
\textbf{Mục} & \textbf{Nội dung} \\
\hline
\endhead % Header cho các trang tiếp theo
\hline
\endfoot % Footer cho bảng
\hline
\endlastfoot % Footer cho trang cuối cùng
Use Case Name & Xác nhận và Tiến hành Thanh toán Đơn Mang về (có xem xét Cọc/Trả trước) \\
\hline
Use Case ID & UC-MD06-07 \\
\hline
Use Case Description & Cho phép Nhân viên (US-02/US-05) chuyển sang màn hình thanh toán cho một đơn hàng mang về. Trước khi hiển thị số tiền cuối cùng, hệ thống sẽ tự động kiểm tra và áp dụng (trừ đi) bất kỳ khoản tiền đặt cọc hoặc thanh toán trước nào mà khách hàng có thể đã thực hiện (ví dụ: khi đặt hàng mang về online). \\
\hline
Actor & US-02 (Nhân viên phục vụ), US-05 (Nhân viên thu ngân) \\
\hline
Priority & Should Have (Nếu có kênh đặt takeout online cho phép trả trước/cọc) \\
\hline
Trigger & Nhân viên đã hoàn tất việc nhập món cho đơn hàng mang về và khách hàng sẵn sàng thanh toán. \\
\hline
Pre-Condition & - Nhân viên đang ở màn hình đơn hàng mang về. \newline - Đơn hàng mang về có thể đã được liên kết với một đơn đặt hàng online có thông tin về tiền cọc/thanh toán trước. \\
\hline
Post-Condition & - Nhân viên được chuyển đến màn hình thanh toán. \newline - Số tiền cần thanh toán cuối cùng hiển thị trên màn hình đã được hệ thống tự động điều chỉnh (trừ đi) nếu có khoản cọc/trả trước hợp lệ. \\
\hline
\multicolumn{2}{|c|}{\textbf{2.2. Luồng thực thi (Flow)}} \\
\hline
\textbf{Mục} & \textbf{Nội dung} \\
\hline
Basic Flow & 1. Nhân viên (US-02/US-05) đang ở màn hình đơn hàng mang về và nhấp vào nút "Thanh toán" (Payment). \newline 2. Hệ thống (System) chuẩn bị dữ liệu cho màn hình thanh toán. \newline 3. Hệ thống (System) kiểm tra xem đơn hàng mang về hiện tại có được liên kết với một bản ghi đơn hàng đặt trước online nào không (ví dụ: qua mã đơn hàng online, SĐT khách hàng đã liên kết ở UC-MD06-03). \newline 4. \textbf{Nếu có liên kết và đơn hàng online đó có ghi nhận tiền đặt cọc/thanh toán trước đã thành công:} \newline    a. Hệ thống (System) lấy giá trị số tiền đã thanh toán trước (PaidDepositOrPrepaymentAmount). \newline 5. \textbf{Nếu không có liên kết hoặc không có thanh toán trước:} \newline    a. PaidDepositOrPrepaymentAmount = 0. \newline 6. Hệ thống (System) tính toán Tổng số tiền phải trả ban đầu của đơn hàng mang về (TotalTakeoutAmount = Subtotal + Taxes). \newline 7. Hệ thống (System) tính toán Số tiền cần thanh toán cuối cùng (AmountDueForTakeout): \newline    `AmountDueForTakeout = TotalTakeoutAmount - PaidDepositOrPrepaymentAmount` \newline 8. Hệ thống hiển thị màn hình thanh toán. Trên màn hình này, hệ thống hiển thị rõ ràng: \newline    - Tổng tiền ban đầu (TotalTakeoutAmount). \newline    - (Nếu có) Số tiền đặt cọc/thanh toán trước đã áp dụng với giá trị âm hoặc dưới dạng khoản trừ. \newline    - Số tiền cần thanh toán cuối cùng (AmountDueForTakeout). \newline 9. Nhân viên và khách hàng (nếu có màn hình khách) nhìn thấy số tiền cuối cùng cần trả. \newline 10. Hệ thống sẵn sàng để nhân viên thực hiện các UC thanh toán tiếp theo (UC-MD06-08, UC-MD06-09, UC-MD06-10). \\
\hline
Alternative Flow & \textbf{8a. Hiển thị đặt cọc/trả trước như một dòng thanh toán:} \newline    1. Tương tự Alternative Flow 8a của UC-MD05-09, hệ thống có thể hiển thị khoản trả trước như một dòng "đã thanh toán" thay vì trừ trực tiếp vào tổng. \\
\hline
Exception Flow & \textbf{4b. Lỗi truy xuất thông tin đặt cọc/trả trước từ đơn hàng online:} \newline    1. Hệ thống tìm thấy liên kết nhưng không thể đọc được thông tin thanh toán trước một cách tin cậy. \newline    2. Hệ thống không áp dụng được khoản đã trả trước. Nên hiển thị cảnh báo cho nhân viên "Không thể xác minh thanh toán trước. Vui lòng kiểm tra thủ công." và hiển thị AmountDueForTakeout chưa trừ. \\
\hline
\multicolumn{2}{|c|}{\textbf{2.3. Thông tin bổ sung (Additional Information)}} \\
\hline
\textbf{Mục} & \textbf{Nội dung} \\
\hline
Business Rule & - \textbf{BR-UC6.7-1 (System):} Hệ thống phải tự động kiểm tra và áp dụng các khoản thanh toán trước hợp lệ cho đơn hàng mang về khi nhân viên vào màn hình thanh toán. \newline - \textbf{BR-UC6.7-2 (System):} Việc áp dụng này phải được hiển thị rõ ràng. \newline - \textbf{BR-UC6.7-3 (System):} Khoản thanh toán trước đã áp dụng phải được đánh dấu để không bị áp dụng lại. \\
\hline
Non-Functional Requirement & - \textbf{NFR-UC6.7-1 (Accuracy):} Áp dụng đúng số tiền đã trả trước. \newline - \textbf{NFR-UC6.7-2 (Performance):} Kiểm tra và áp dụng nhanh chóng. \newline - \textbf{NFR-UC6.7-3 (Transparency):} Hiển thị rõ ràng. \\
\hline
\end{longtable}

\subsubsection{Use Case UC-MD06-08: Thực hiện Thanh toán Tiền mặt cho Đơn Mang về}
\begin{longtable}{|m{4cm}|p{11cm}|}
\caption{Đặc tả Use Case UC-MD06-08: Thực hiện Thanh toán Tiền mặt cho Đơn Mang về} \label{tab:uc_md06_08_final_v3} \\
\hline
\multicolumn{2}{|c|}{\textbf{2.1. Tóm tắt (Summary)}} \\
\hline
\textbf{Mục} & \textbf{Nội dung} \\
\hline
\endhead % Header cho các trang tiếp theo
\hline
\endfoot % Footer cho bảng
\hline
\endlastfoot % Footer cho trang cuối cùng
Use Case Name & Thực hiện Thanh toán Tiền mặt cho Đơn Mang về \\
\hline
Use Case ID & UC-MD06-08 \\
\hline
Use Case Description & Cho phép Nhân viên (US-02/US-05) nhận tiền mặt từ khách hàng tại quầy, nhập số tiền khách đưa vào hệ thống POS, để hệ thống tự động tính toán số tiền cần trả lại (nếu có) và ghi nhận thanh toán cho đơn hàng mang về. \\
\hline
Actor & US-02 (Nhân viên phục vụ), US-05 (Nhân viên thu ngân) \\
\hline
Priority & Must Have \\
\hline
Trigger & Khách hàng mua mang về chọn thanh toán bằng tiền mặt. Nhân viên đang ở màn hình thanh toán (sau UC-MD06-07). \\
\hline
Pre-Condition & - Nhân viên đang ở màn hình thanh toán cho đơn hàng mang về. \newline - Số tiền cần thanh toán cuối cùng (AmountDueForTakeout) được hiển thị. \newline - Phương thức "Tiền mặt" khả dụng. \\
\hline
Post-Condition & - Giao dịch tiền mặt được ghi nhận. \newline - Số tiền còn lại của đơn hàng được cập nhật (thường về 0 nếu thanh toán đủ). \\
\hline
\multicolumn{2}{|c|}{\textbf{2.2. Luồng thực thi (Flow)}} \\
\hline
\textbf{Mục} & \textbf{Nội dung} \\
\hline
Basic Flow, Alternative Flow, Exception Flow & Hành động của Nhân viên khi nhận và ghi nhận thanh toán tiền mặt cho đơn hàng mang về (chọn phương thức, nhập tiền nhận, hệ thống tính tiền thừa, xác nhận) về cơ bản là **giống hệt** với **Use Case UC-MD05-12: Thực hiện Thanh toán Tiền mặt**. \\
\hline
\multicolumn{2}{|c|}{\textbf{2.3. Thông tin bổ sung (Additional Information)}} \\
\hline
\textbf{Mục} & \textbf{Nội dung} \\
\hline
Business Rule & Các Business Rule về tính tiền thừa, cập nhật số dư tiền mặt phiên tương tự BR-UC5.12-1, BR-UC5.12-2. \\
\hline
Non-Functional Requirement & Các Non-Functional Requirement về Usability, Accuracy tương tự NFR-UC5.12-1, NFR-UC5.12-2. Tốc độ xử lý tại quầy rất quan trọng. \\
\hline
\end{longtable}

\subsubsection{Use Case UC-MD06-09: Ghi nhận Thanh toán bằng Phương thức Khác (Không Thẻ) cho Đơn Mang về}
\begin{longtable}{|m{4cm}|p{11cm}|}
\caption{Đặc tả Use Case UC-MD06-09: Ghi nhận Thanh toán bằng Phương thức Khác (Không Thẻ) cho Đơn Mang về} \label{tab:uc_md06_09_final_v3} \\
\hline
\multicolumn{2}{|c|}{\textbf{2.1. Tóm tắt (Summary)}} \\
\hline
\textbf{Mục} & \textbf{Nội dung} \\
\hline
\endhead % Header cho các trang tiếp theo
\hline
\endfoot % Footer cho bảng
\hline
\endlastfoot % Footer cho trang cuối cùng
Use Case Name & Ghi nhận Thanh toán bằng Phương thức Khác (Không Thẻ) cho Đơn Mang về \\
\hline
Use Case ID & UC-MD06-09 \\
\hline
Use Case Description & Cho phép Nhân viên (US-02/US-05) ghi nhận việc khách hàng thanh toán một phần hoặc toàn bộ đơn hàng mang về bằng một phương thức khác được hỗ trợ (ví dụ: Ví điện tử đã tích hợp, voucher, điểm thưởng...), không bao gồm thẻ ngân hàng. \\
\hline
Actor & US-02 (Nhân viên phục vụ), US-05 (Nhân viên thu ngân) \\
\hline
Priority & Should Have (Tùy thuộc vào các phương thức thanh toán nhà hàng chấp nhận) \\
\hline
Trigger & Khách hàng mua mang về chọn thanh toán bằng một phương thức khác tiền mặt và không phải thẻ. Nhân viên đang ở màn hình thanh toán. \\
\hline
Pre-Condition & - Nhân viên đang ở màn hình thanh toán cho đơn hàng mang về. \newline - Số tiền cần thanh toán được hiển thị. \newline - Các phương thức thanh toán khác (ví dụ: "MoMo", "VNPay QR", "Gift Card") đã được cấu hình và khả dụng trên POS. \\
\hline
Post-Condition & - Giao dịch thanh toán bằng phương thức đã chọn được ghi nhận. \newline - Số tiền còn lại của đơn hàng được cập nhật. \\
\hline
\multicolumn{2}{|c|}{\textbf{2.2. Luồng thực thi (Flow)}} \\
\hline
\textbf{Mục} & \textbf{Nội dung} \\
\hline
Basic Flow (Ví dụ: Thanh toán bằng Ví điện tử tích hợp) & 1. Nhân viên (US-02/US-05) đang ở màn hình thanh toán. \newline 2. Khách hàng yêu cầu thanh toán bằng Ví điện tử ABC. \newline 3. Nhân viên chọn phương thức "Ví điện tử ABC" trên POS. \newline 4. Hệ thống POS (nếu tích hợp) có thể hiển thị mã QR để khách quét hoặc yêu cầu nhập thông tin giao dịch từ ví. \newline 5. Khách hàng thực hiện thao tác thanh toán trên ứng dụng ví của họ. \newline 6. Hệ thống POS nhận được xác nhận thanh toán thành công từ cổng tích hợp ví điện tử. \newline 7. Hệ thống ghi nhận khoản thanh toán bằng "Ví điện tử ABC". \newline 8. Hệ thống cập nhật số tiền còn lại phải trả. \\
\hline
Alternative Flow & \textbf{4a. Ghi nhận thủ công cho phương thức không tích hợp trực tiếp:} \newline    1. Nếu phương thức (ví dụ: một loại voucher giấy) không tích hợp trực tiếp. \newline    2. Nhân viên chọn phương thức tương ứng trên POS (ví dụ: "Voucher XYZ"). \newline    3. Nhân viên nhập số tiền được thanh toán bằng voucher đó. \newline    4. Nhân viên thu lại voucher giấy (hoặc nhập mã voucher để hệ thống xác thực nếu có cơ chế riêng). \newline    5. Hệ thống ghi nhận khoản thanh toán. \\
\hline
Exception Flow & \textbf{6a. Thanh toán qua ví/voucher thất bại:} \newline    1. Giao dịch bị từ chối bởi hệ thống ví/voucher. \newline    2. Hệ thống POS báo lỗi. Nhân viên yêu cầu khách chọn phương thức khác. \newline \textbf{7a. Lỗi hệ thống khi ghi nhận thanh toán.} \\
\hline
\multicolumn{2}{|c|}{\textbf{2.3. Thông tin bổ sung (Additional Information)}} \\
\hline
\textbf{Mục} & \textbf{Nội dung} \\
\hline
Business Rule & - \textbf{BR-UC6.9-1 (V3):} Các phương thức thanh toán được hỗ trợ phải được cấu hình chính xác trong POS. \newline - \textbf{BR-UC6.9-2 (V3):} Đối với các phương thức tích hợp (ví dụ: ví điện tử qua API), việc xác nhận giao dịch thành công từ cổng tích hợp là bắt buộc trước khi ghi nhận. \\
\hline
Non-Functional Requirement & - \textbf{NFR-UC6.9-1 (V3) (Integration):} Nếu có tích hợp với ví điện tử hoặc hệ thống voucher, tích hợp phải ổn định và bảo mật. \newline - \textbf{NFR-UC6.9-2 (V3) (Usability):} Chọn và xử lý các phương thức khác phải dễ dàng cho nhân viên. \\
\hline
\end{longtable}

\subsubsection{Use Case UC-MD06-10: Thực hiện Thanh toán Đơn Mang về bằng Nhiều Phương thức (Không Thẻ)}
% (Trước đây là FR-MD06-10, giờ là UC tương ứng)
\begin{longtable}{|m{4cm}|p{11cm}|}
\caption{Đặc tả Use Case UC-MD06-10: Thực hiện Thanh toán Đơn Mang về bằng Nhiều Phương thức (Không Thẻ)} \label{tab:uc_md06_10_final_v3} \\
\hline
\multicolumn{2}{|c|}{\textbf{2.1. Tóm tắt (Summary)}} \\
\hline
\textbf{Mục} & \textbf{Nội dung} \\
\hline
\endhead % Header cho các trang tiếp theo
\hline
\endfoot % Footer cho bảng
\hline
\endlastfoot % Footer cho trang cuối cùng
Use Case Name & Thực hiện Thanh toán Đơn Mang về bằng Nhiều Phương thức (Không Thẻ) \\
\hline
Use Case ID & UC-MD06-10 \\
\hline
Use Case Description & Cho phép Nhân viên (US-02/US-05) nhận thanh toán cho một đơn hàng mang về bằng cách kết hợp nhiều phương thức thanh toán được hỗ trợ (ví dụ: một phần bằng Tiền mặt, một phần bằng Ví điện tử), không bao gồm thẻ ngân hàng. \\
\hline
Actor & US-02 (Nhân viên phục vụ), US-05 (Nhân viên thu ngân) \\
\hline
Priority & Should Have \\
\hline
Trigger & Khách hàng mua mang về muốn chia nhỏ khoản thanh toán của họ ra nhiều hình thức khác nhau. \\
\hline
Pre-Condition & - Nhân viên đang ở màn hình thanh toán của một đơn hàng mang về. \newline - Số tiền cần thanh toán (AmountDueForTakeout) được hiển thị. \newline - Có ít nhất hai phương thức thanh toán khác nhau (không phải Thẻ) được cấu hình và khả dụng trên POS. \\
\hline
Post-Condition & - Nhiều giao dịch thanh toán (tương ứng với từng phương thức) được ghi nhận cho cùng một đơn hàng mang về. \newline - Tổng số tiền từ tất cả các phương thức thanh toán bằng số tiền cần thanh toán của đơn hàng. \newline - Đơn hàng sẵn sàng để hoàn tất. \\
\hline
\multicolumn{2}{|c|}{\textbf{2.2. Luồng thực thi (Flow)}} \\
\hline
\textbf{Mục} & \textbf{Nội dung} \\
\hline
Basic Flow, Alternative Flow, Exception Flow & Hành động của Nhân viên khi xử lý thanh toán bằng nhiều phương thức (không bao gồm thẻ) cho đơn hàng mang về (chọn phương thức 1, nhập số tiền, chọn phương thức 2, nhập số tiền còn lại, xác nhận) về cơ bản là **giống hệt** với **Use Case UC-MD05-14: Thực hiện Thanh toán bằng Nhiều Phương thức (Không bao gồm Thẻ)**. \\
\hline
\multicolumn{2}{|c|}{\textbf{2.3. Thông tin bổ sung (Additional Information)}} \\
\hline
\textbf{Mục} & \textbf{Nội dung} \\
\hline
Business Rule & Các Business Rule về cho phép nhiều dòng thanh toán, tổng tiền phải khớp tương tự BR-UC5.14-1, BR-UC5.14-2. \\
\hline
Non-Functional Requirement & Các Non-Functional Requirement về Usability, Accuracy tương tự NFR-UC5.14-1, NFR-UC5.14-2. \\
\hline
\end{longtable}

\subsubsection{Use Case UC-MD06-11: In Hóa đơn/Biên lai cho Đơn Mang về}
% (Trước đây là FR-MD06-11, giờ là UC tương ứng, tương tự UC-MD05-16)
\begin{longtable}{|m{4cm}|p{11cm}|}
\caption{Đặc tả Use Case UC-MD06-11: In Hóa đơn/Biên lai cho Đơn Mang về} \label{tab:uc_md06_11_final_v3} \\
\hline
\multicolumn{2}{|c|}{\textbf{2.1. Tóm tắt (Summary)}} \\
\hline
\textbf{Mục} & \textbf{Nội dung} \\
\hline
\endhead % Header cho các trang tiếp theo
\hline
\endfoot % Footer cho bảng
\hline
\endlastfoot % Footer cho trang cuối cùng
Use Case Name & In Hóa đơn/Biên lai cho Đơn Mang về \\
\hline
Use Case ID & UC-MD06-11 \\
\hline
Use Case Description & Sau khi Nhân viên (US-02/US-05) đã xác nhận hoàn tất thanh toán cho đơn hàng mang về, cho phép Nhân viên kích hoạt (hoặc hệ thống tự động) in ra hóa đơn/biên lai cuối cùng cho khách hàng. Mẫu in có thể cần chỉ rõ đây là đơn hàng mang về. \\
\hline
Actor & US-02 (Nhân viên phục vụ), US-05 (Nhân viên thu ngân) \\
\hline
Priority & Must Have \\
\hline
Trigger & Giao dịch thanh toán đơn hàng mang về được xác nhận thành công (kết thúc UC-MD06-08, UC-MD06-09, hoặc UC-MD06-10). \\
\hline
Pre-Condition & - Đơn hàng mang về đã được thanh toán đủ. \newline - Máy in hóa đơn đã cấu hình và sẵn sàng. \newline - Mẫu in hóa đơn/biên lai POS đã được thiết lập. \\
\hline
Post-Condition & - Một bản hóa đơn/biên lai chi tiết về đơn hàng mang về được in ra. \\
\hline
\multicolumn{2}{|c|}{\textbf{2.2. Luồng thực thi (Flow)}} \\
\hline
\textbf{Mục} & \textbf{Nội dung} \\
\hline
Basic Flow, Alternative Flow, Exception Flow & Hành động của Nhân viên để hoàn tất thanh toán và kích hoạt in hóa đơn (hoặc hệ thống tự động in) cho đơn hàng mang về, cũng như các tùy chọn không in/gửi điện tử, và các lỗi có thể xảy ra, về cơ bản là **giống hệt** với **Use Case UC-MD05-16: Hoàn tất và In Hóa đơn Cuối cùng**. \newline Điểm khác biệt là nội dung hóa đơn có thể cần có chỉ dẫn "Đơn Mang về". \\
\hline
\multicolumn{2}{|c|}{\textbf{2.3. Thông tin bổ sung (Additional Information)}} \\
\hline
\textbf{Mục} & \textbf{Nội dung} \\
\hline
Business Rule & - \textbf{BR-UC6.11-1 (V3):} Hóa đơn phải được tạo sau khi thanh toán đủ. \newline - \textbf{BR-UC6.11-2 (V3):} Nên có cách phân biệt hóa đơn mang về trên mẫu in. \\
\hline
Non-Functional Requirement & - \textbf{NFR-UC6.11-1 (V3) (Reliability):} In hóa đơn đáng tin cậy. \newline - \textbf{NFR-UC6.11-2 (V3) (Clarity):} Hóa đơn rõ ràng. \\
\hline
\end{longtable}

\subsubsection{Use Case UC-MD06-12: Hoàn tất Đơn hàng Mang về}
% (Trước đây là FR-MD06-12, giờ là UC tương ứng, tương tự UC-MD05-17)
\begin{longtable}{|m{4cm}|p{11cm}|}
\caption{Đặc tả Use Case UC-MD06-12: Hoàn tất Đơn hàng Mang về} \label{tab:uc_md06_12_final_v3} \\
\hline
\multicolumn{2}{|c|}{\textbf{2.1. Tóm tắt (Summary)}} \\
\hline
\textbf{Mục} & \textbf{Nội dung} \\
\hline
\endhead % Header cho các trang tiếp theo
\hline
\endfoot % Footer cho bảng
\hline
\endlastfoot % Footer cho trang cuối cùng
Use Case Name & Hoàn tất Đơn hàng Mang về \\
\hline
Use Case ID & UC-MD06-12 \\
\hline
Use Case Description & Sau khi khách hàng đã thanh toán và nhận hàng mang về, Nhân viên (US-02/US-05) thực hiện hành động cuối cùng trên POS để chính thức đóng đơn hàng mang về đó trong hệ thống. \\
\hline
Actor & US-02 (Nhân viên phục vụ), US-05 (Nhân viên thu ngân) \\
\hline
Priority & Must Have \\
\hline
Trigger & Giao dịch thanh toán và giao hàng cho đơn mang về đã hoàn tất (sau UC-MD06-11). Nhân viên ở màn hình xác nhận thanh toán. \\
\hline
Pre-Condition & - Đơn hàng mang về đã ở trạng thái "Đã thanh toán" (Paid). \\
\hline
Post-Condition & - Trạng thái cuối cùng của đơn hàng POS mang về được cập nhật thành "Đã hoàn thành" (Done). \newline - Nhân viên được chuyển về màn hình POS sẵn sàng cho đơn hàng tiếp theo. \\
\hline
\multicolumn{2}{|c|}{\textbf{2.2. Luồng thực thi (Flow)}} \\
\hline
\textbf{Mục} & \textbf{Nội dung} \\
\hline
Basic Flow & 1. Sau khi hoàn tất thanh toán và in hóa đơn (UC-MD06-11), hệ thống hiển thị màn hình xác nhận thanh toán thành công, thường có nút "Đơn hàng tiếp theo" hoặc "Đơn mang về mới". \newline 2. Nhân viên (US-02/US-05) giao hàng cho khách và nhấp vào nút đó. \newline 3. Hệ thống (System) cập nhật trạng thái của bản ghi đơn hàng POS mang về thành "Done" hoặc "Completed". \newline 4. Hệ thống chuyển hướng giao diện về màn hình chờ của chế độ bán mang về (sẵn sàng cho UC-MD06-02) hoặc màn hình POS chính. \\
\hline
Alternative Flow & Tương tự UC-MD05-17 (Đóng đơn từ màn hình chi tiết nếu có). \\
\hline
Exception Flow & \textbf{3a. Lỗi cập nhật trạng thái đơn hàng:} Hệ thống báo lỗi. \\
\hline
\multicolumn{2}{|c|}{\textbf{2.3. Thông tin bổ sung (Additional Information)}} \\
\hline
\textbf{Mục} & \textbf{Nội dung} \\
\hline
Business Rule & - \textbf{BR-UC6.12-1 (V3):} Chỉ đơn hàng mang về đã thanh toán mới được đóng. \newline - \textbf{BR-UC6.12-2 (V3):} Sau khi đóng, đơn hàng không sửa đổi trên POS được nữa. \\
\hline
Non-Functional Requirement & - \textbf{NFR-UC6.12-1 (V3) (Performance):} Đóng đơn nhanh. \newline - \textbf{NFR-UC6.12-2 (V3) (Usability):} Chuyển tiếp mượt mà. \\
\hline
\end{longtable}


\subsection{Module MD-07: Quản lý Giao hàng (POS - Delivery)}

\subsubsection{Use Case UC-MD07-01: Chọn Chế độ Giao hàng}

\begin{longtable}{|m{4cm}|p{11cm}|}
\caption{Đặc tả Use Case UC-MD07-01: Chọn Chế độ Giao hàng} \label{tab:uc_md07_01} \\
\hline

\endhead % Header cho các trang tiếp theo
\hline
\endfoot % Footer cho bảng
\hline
\endlastfoot % Footer cho trang cuối cùng
\multicolumn{2}{|c|}{\textbf{2.1. Tóm tắt (Summary)}} \\
\hline
\textbf{Mục} & \textbf{Nội dung} \\
\hline
Use Case Name & Chọn Chế độ Giao hàng \\
\hline
Use Case ID & UC-MD07-01 \\
\hline
Use Case Description & Cho phép Nhân viên (Phục vụ hoặc Thu ngân) tại POS lựa chọn một chế độ hoạt động hoặc giao diện riêng biệt để tiếp nhận và quản lý các đơn hàng cần giao đến địa chỉ của khách hàng (Delivery). \\
\hline
Actor & US-02 (Nhân viên phục vụ), US-05 (Nhân viên thu ngân) \\
\hline
Priority & Must Have \\
\hline
Trigger & Có yêu cầu tạo đơn hàng giao đi (ví dụ: khách gọi điện đặt giao hàng, đơn hàng từ nền tảng online khác cần nhập vào POS để xử lý). \\
\hline
Pre-Condition & - Nhân viên đã đăng nhập và đang trong phiên POS hoạt động (UC-MD05-01). \newline - Giao diện POS chính đang hiển thị. \newline - Chế độ/Nút chức năng "Giao hàng" (Delivery) đã được cấu hình và hiển thị trên giao diện POS. \\
\hline
Post-Condition & - Hệ thống chuyển sang giao diện hoặc chế độ dành riêng cho việc tạo đơn hàng giao hàng. \newline - Giao diện này sẵn sàng để tạo đơn hàng mới (UC-MD07-02) và yêu cầu nhập thông tin khách hàng/địa chỉ giao. \\
\hline
\multicolumn{2}{|c|}{\textbf{2.2. Luồng thực thi (Flow)}} \\
\hline
\textbf{Mục} & \textbf{Nội dung} \\
\hline
Basic Flow & 1. Nhân viên (US-02/US-05) đang ở giao diện POS chính. \newline 2. Nhân viên xác định vị trí nút hoặc tùy chọn "Giao hàng" (Delivery) trên màn hình. \newline 3. Nhân viên nhấp vào nút "Giao hàng". \newline 4. Hệ thống chuyển đổi giao diện hoặc ngữ cảnh sang chế độ giao hàng. Giao diện này thường yêu cầu nhập/chọn thông tin khách hàng trước tiên hoặc mở ngay một đơn hàng mới với yêu cầu bắt buộc liên kết khách hàng. \\
\hline
Alternative Flow & Tương tự UC-MD06-01, nút "Giao hàng" có thể nằm ở menu chính hoặc dashboard. Hệ thống cũng có thể sử dụng cùng giao diện nhưng kích hoạt các quy tắc và trường thông tin đặc thù cho giao hàng. \\
\hline
Exception Flow & Tương tự UC-MD06-01 (Nút bị vô hiệu hóa, Lỗi chuyển đổi giao diện). \\
\hline
\multicolumn{2}{|c|}{\textbf{2.3. Thông tin bổ sung (Additional Information)}} \\
\hline
\textbf{Mục} & \textbf{Nội dung} \\
\hline
Business Rule & - \textbf{BR-UC7.1-1:} Phải có cách thức rõ ràng để nhân viên vào chế độ xử lý đơn hàng giao hàng. \newline - \textbf{BR-UC7.1-2:} Chế độ giao hàng phải yêu cầu thông tin khách hàng và địa chỉ giao hàng chi tiết. \\
\hline
Non-Functional Requirement & Tương tự UC-MD06-01 (Usability, Performance). \\
\hline
\end{longtable}

\subsubsection{Use Case UC-MD07-02: Tạo/Mở Đơn hàng Giao hàng}

\begin{longtable}{|m{4cm}|p{11cm}|}
\caption{Đặc tả Use Case UC-MD07-02: Tạo/Mở Đơn hàng Giao hàng} \label{tab:uc_md07_02} \\
\hline

\endhead % Header cho các trang tiếp theo
\hline
\endfoot % Footer cho bảng
\hline
\endlastfoot % Footer cho trang cuối cùng
\multicolumn{2}{|c|}{\textbf{2.1. Tóm tắt (Summary)}} \\
\hline
\textbf{Mục} & \textbf{Nội dung} \\
\hline
Use Case Name & Tạo/Mở Đơn hàng Giao hàng \\
\hline
Use Case ID & UC-MD07-02 \\
\hline
Use Case Description & Khởi tạo một bản ghi đơn hàng POS mới hoặc mở lại một đơn hàng giao hàng đang chờ xử lý, với yêu cầu bắt buộc phải liên kết với thông tin khách hàng và địa chỉ giao hàng. \\
\hline
Actor & US-02 (Nhân viên phục vụ), US-05 (Nhân viên thu ngân) \\
\hline
Priority & Must Have \\
\hline
Trigger & Nhân viên đã chọn chế độ Giao hàng (UC-MD07-01) và cần tạo đơn mới hoặc xử lý đơn đang chờ. \\
\hline
Pre-Condition & - Nhân viên đang ở trong chế độ/giao diện Giao hàng (UC-MD07-01 thành công). \\
\hline
Post-Condition & - Một bản ghi đơn hàng POS mới được tạo với loại hình "Giao hàng" (Delivery) HOẶC một đơn hàng giao hàng cũ được mở lại. \newline - Đơn hàng này bắt buộc phải được liên kết với khách hàng và địa chỉ giao (thông qua UC-MD07-03). \newline - Giao diện đơn hàng được hiển thị, sẵn sàng để thêm món hoặc xử lý tiếp. \\
\hline
\multicolumn{2}{|c|}{\textbf{2.2. Luồng thực thi (Flow)}} \\
\hline
\textbf{Mục} & \textbf{Nội dung} \\
\hline
Basic Flow (Tạo mới) & 1. Tiếp nối từ UC-MD07-01. \newline 2. Nhân viên chọn "Tạo đơn mới" hoặc hệ thống yêu cầu chọn/nhập khách hàng trước. \newline 3. Hệ thống chuyển sang giao diện yêu cầu thông tin khách hàng (UC-MD07-03). \newline 4. Sau khi khách hàng được chọn/tạo và địa chỉ giao được xác nhận (trong UC-MD07-03). \newline 5. Hệ thống tạo bản ghi đơn hàng POS mới, loại "Delivery", liên kết với khách hàng và địa chỉ đã chọn. \newline 6. Hệ thống hiển thị giao diện đơn hàng (tương tự UC-MD06-02 nhưng có hiển thị thông tin giao hàng). \\
\hline
Alternative Flow & \textbf{1a. Mở đơn hàng giao hàng đang chờ:} \newline    1. Giao diện chế độ giao hàng hiển thị danh sách các đơn hàng giao đi đang chờ xử lý (ví dụ: chờ gửi bếp, chờ gán tài xế). \newline    2. Nhân viên chọn một đơn hàng từ danh sách. \newline    3. Hệ thống mở lại chi tiết đơn hàng đó. \\
\hline
Exception Flow & \textbf{5a. Lỗi tạo đơn hàng mới:} Tương tự UC-MD06-02. \newline \textbf{Alternative Flow 1a - Step 3a. Lỗi mở lại đơn hàng cũ:} Tương tự UC-MD05-03 (Exception Flow 2a). \\
\hline
\multicolumn{2}{|c|}{\textbf{2.3. Thông tin bổ sung (Additional Information)}} \\
\hline
\textbf{Mục} & \textbf{Nội dung} \\
\hline
Business Rule & - \textbf{BR-UC7.2-1:} Đơn hàng loại "Delivery" bắt buộc phải có thông tin khách hàng và địa chỉ giao hàng hợp lệ được liên kết. \newline - \textbf{BR-UC7.2-2:} Hệ thống cần phân loại rõ ràng đơn hàng "Delivery" để tích hợp với Shipday và báo cáo. \\
\hline
Non-Functional Requirement & Tương tự UC-MD06-02 (Performance, Usability). \\
\hline
\end{longtable}

\subsubsection{Use Case UC-MD07-03: Liên kết/Nhập Thông tin Khách hàng Giao hàng}

\begin{longtable}{|m{4cm}|p{11cm}|}
\caption{Đặc tả Use Case UC-MD07-03: Liên kết/Nhập Thông tin Khách hàng Giao hàng} \label{tab:uc_md07_03} \\
\hline

\endhead % Header cho các trang tiếp theo
\hline
\endfoot % Footer cho bảng
\hline
\endlastfoot % Footer cho trang cuối cùng
\multicolumn{2}{|c|}{\textbf{2.1. Tóm tắt (Summary)}} \\
\hline
\textbf{Mục} & \textbf{Nội dung} \\
\hline
Use Case Name & Liên kết/Nhập Thông tin Khách hàng Giao hàng \\
\hline
Use Case ID & UC-MD07-03 \\
\hline
Use Case Description & Yêu cầu Nhân viên bắt buộc phải tìm kiếm và chọn một khách hàng đã có (với địa chỉ đã lưu) hoặc nhập thông tin cho khách hàng mới, bao gồm Tên, Số điện thoại và Địa chỉ giao hàng chi tiết (số nhà, đường, phường/xã, quận/huyện, tỉnh/thành phố), để liên kết với đơn hàng giao đi. \\
\hline
Actor & US-02 (Nhân viên phục vụ), US-05 (Nhân viên thu ngân) \\
\hline
Priority & Must Have \\
\hline
Trigger & Bắt đầu tạo đơn hàng giao hàng mới (UC-MD07-02). \\
\hline
Pre-Condition & - Nhân viên đang trong luồng tạo đơn hàng giao hàng. \newline - Hệ thống quản lý khách hàng (Contacts/CRM) đang hoạt động. \\
\hline
Post-Condition & - Một bản ghi khách hàng (cũ hoặc mới) với đầy đủ thông tin Tên, SĐT và Địa chỉ giao hàng hợp lệ được liên kết với đơn hàng giao đi. \newline - Hệ thống sẵn sàng để tạo bản ghi đơn hàng POS (bước 5 của UC-MD07-02). \\
\hline
\multicolumn{2}{|c|}{\textbf{2.2. Luồng thực thi (Flow)}} \\
\hline
\textbf{Mục} & \textbf{Nội dung} \\
\hline
Basic Flow (Chọn khách hàng đã có với địa chỉ) & 1. Hệ thống hiển thị giao diện yêu cầu chọn/nhập khách hàng. \newline 2. Nhân viên (US-02/US-05) tìm kiếm khách hàng theo Tên hoặc SĐT (tương tự UC-MD06-03). \newline 3. Hệ thống hiển thị kết quả tìm kiếm. \newline 4. Nhân viên chọn khách hàng đúng. \newline 5. Hệ thống kiểm tra xem khách hàng này đã có địa chỉ giao hàng được lưu chưa. \newline 6. \textbf{Nếu đã có địa chỉ (hoặc nhiều địa chỉ):} \newline    a. Hệ thống hiển thị (các) địa chỉ đã lưu. \newline    b. Nhân viên chọn/xác nhận địa chỉ giao hàng đúng. \newline 7. \textbf{Nếu chưa có địa chỉ hoặc cần nhập địa chỉ mới:} \newline    a. Hệ thống hiển thị form nhập địa chỉ chi tiết (Số nhà, Đường, Phường/Xã, Quận/Huyện, Tỉnh/Thành phố). \newline    b. Nhân viên nhập đầy đủ thông tin địa chỉ giao hàng. \newline    c. (Tùy chọn) Nhân viên có thể đánh dấu lưu địa chỉ này vào hồ sơ khách hàng cho lần sau. \newline 8. Nhân viên xác nhận thông tin khách hàng và địa chỉ giao hàng. \\
\hline
Alternative Flow & \textbf{2a. Tạo khách hàng mới:} \newline    1. Nếu không tìm thấy khách hàng, nhân viên chọn "Tạo mới". \newline    2. Hệ thống hiển thị form nhập thông tin khách hàng và địa chỉ. \newline    3. Nhân viên nhập Tên, SĐT (bắt buộc), Email (tùy chọn), và Địa chỉ giao hàng chi tiết (bắt buộc). \newline    4. Nhân viên nhấn "Lưu". \newline    5. Hệ thống tạo khách hàng mới và địa chỉ, sau đó tự động chọn khách hàng và địa chỉ này. Use Case tiếp tục từ bước 8. \\
\hline
Exception Flow & \textbf{8a. Thiếu thông tin bắt buộc / Địa chỉ không hợp lệ:} \newline    1. Nhân viên xác nhận nhưng thiếu Tên, SĐT, hoặc các thành phần bắt buộc của địa chỉ giao hàng. Hoặc định dạng SĐT không đúng. \newline    2. Hệ thống báo lỗi, yêu cầu nhập đầy đủ/chính xác thông tin. \newline    3. Use Case quay lại bước nhập liệu tương ứng. \newline \textbf{Alternative Flow 2a - Step 4a. Lỗi tạo khách hàng/địa chỉ mới:} \newline    1. Hệ thống gặp lỗi khi lưu khách hàng hoặc địa chỉ mới. \newline    2. Hệ thống báo lỗi. \newline \textbf{8b. Lỗi hệ thống khi liên kết khách hàng/địa chỉ:} \newline    1. Hệ thống gặp lỗi kỹ thuật khi lưu liên kết. \newline    2. Hệ thống báo lỗi. \\
\hline
\multicolumn{2}{|c|}{\textbf{2.3. Thông tin bổ sung (Additional Information)}} \\
\hline
\textbf{Mục} & \textbf{Nội dung} \\
\hline
Business Rule & - \textbf{BR-UC7.3-1:} Thông tin khách hàng (Tên, SĐT) và Địa chỉ giao hàng chi tiết là bắt buộc đối với đơn hàng loại "Delivery". \newline - \textbf{BR-UC7.3-2:} Địa chỉ giao hàng cần có cấu trúc rõ ràng (Số nhà, Đường, Phường/Xã, Quận/Huyện, Tỉnh/TP) để đảm bảo tính chính xác cho việc giao hàng và tích hợp với Shipday. \newline - \textbf{BR-UC7.3-3:} Hệ thống nên cho phép lưu nhiều địa chỉ giao hàng cho một khách hàng và cho phép chọn địa chỉ cụ thể cho từng đơn hàng. \\
\hline
Non-Functional Requirement & - \textbf{NFR-UC7.3-1 (Usability):} Việc tìm kiếm khách hàng, chọn/nhập địa chỉ phải thuận tiện. Form nhập địa chỉ nên có cấu trúc rõ ràng, có thể có gợi ý hoặc tích hợp bản đồ (nếu nâng cao). \newline - \textbf{NFR-UC7.3-2 (Data Validation):} Cần có kiểm tra định dạng cơ bản cho SĐT và các thành phần địa chỉ (ví dụ: không để trống trường bắt buộc). \newline - \textbf{NFR-UC7.3-3 (Integration):} Dữ liệu địa chỉ phải có cấu trúc phù hợp để gửi sang Shipday qua API một cách chính xác. \\
\hline
\end{longtable}

\subsubsection{Use Case UC-MD07-04: Thêm món vào Đơn hàng Giao hàng}

\begin{longtable}{|m{4cm}|p{11cm}|}
\caption{Đặc tả Use Case UC-MD07-04: Thêm món vào Đơn hàng Giao hàng} \label{tab:uc_md07_04} \\
\hline

\endhead % Header cho các trang tiếp theo
\hline
\endfoot % Footer cho bảng
\hline
\endlastfoot % Footer cho trang cuối cùng
\multicolumn{2}{|c|}{\textbf{2.1. Tóm tắt (Summary)}} \\
\hline
\textbf{Mục} & \textbf{Nội dung} \\
\hline
Use Case Name & Thêm món vào Đơn hàng Giao hàng \\
\hline
Use Case ID & UC-MD07-04 \\
\hline
Use Case Description & Cho phép Nhân viên thêm các món ăn và đồ uống vào đơn hàng giao đi đang mở, sử dụng giao diện chọn sản phẩm tương tự như các loại đơn hàng khác. \\
\hline
Actor & US-02 (Nhân viên phục vụ), US-05 (Nhân viên thu ngân) \\
\hline
Priority & Must Have \\
\hline
Trigger & Khách hàng (qua điện thoại hoặc kênh khác) đang đặt món cho đơn hàng giao đi. \\
\hline
Pre-Condition & - Nhân viên đang ở màn hình đơn hàng giao hàng đã liên kết khách hàng và địa chỉ (UC-MD07-02 và UC-MD07-03 thành công). \newline - Giao diện POS hiển thị các danh mục và sản phẩm. \\
\hline
Post-Condition & - Món ăn/đồ uống được chọn được thêm vào đơn hàng giao hàng. \newline - Tổng tiền tạm tính của đơn hàng được cập nhật. \\
\hline
\multicolumn{2}{|c|}{\textbf{2.2. Luồng thực thi (Flow)}} \\
\hline
\textbf{Mục} & \textbf{Nội dung} \\
\hline
Basic Flow, Alternative Flow, Exception Flow & Luồng thực thi, các luồng thay thế và ngoại lệ về cơ bản là giống hệt với Use Case UC-MD05-05: Thêm món ăn/đồ uống vào đơn hàng. \\
\hline
\multicolumn{2}{|c|}{\textbf{2.3. Thông tin bổ sung (Additional Information)}} \\
\hline
\textbf{Mục} & \textbf{Nội dung} \\
\hline
Business Rule & Các Business Rule tương tự như BR-UC5.5-1, BR-UC5.5-2, BR-UC5.5-3. \\
\hline
Non-Functional Requirement & Các Non-Functional Requirement tương tự như NFR-UC5.5-1, NFR-UC5.5-2, NFR-UC5.5-3. \\
\hline
\end{longtable}

\subsubsection{Use Case UC-MD07-05: Xử lý Ghi chú cho Đơn Giao hàng}

\begin{longtable}{|m{4cm}|p{11cm}|}
\caption{Đặc tả Use Case UC-MD07-05: Xử lý Ghi chú cho Đơn Giao hàng} \label{tab:uc_md07_05} \\
\hline

\endhead % Header cho các trang tiếp theo
\hline
\endfoot % Footer cho bảng
\hline
\endlastfoot % Footer cho trang cuối cùng
\multicolumn{2}{|c|}{\textbf{2.1. Tóm tắt (Summary)}} \\
\hline
\textbf{Mục} & \textbf{Nội dung} \\
\hline
Use Case Name & Xử lý Ghi chú cho Đơn Giao hàng \\
\hline
Use Case ID & UC-MD07-05 \\
\hline
Use Case Description & Cho phép Nhân viên thêm các ghi chú đặc biệt liên quan đến đơn hàng giao đi, bao gồm yêu cầu về món ăn, đóng gói, hoặc hướng dẫn cho tài xế giao hàng. \\
\hline
Actor & US-02 (Nhân viên phục vụ), US-05 (Nhân viên thu ngân) \\
\hline
Priority & Must Have \\
\hline
Trigger & Khách hàng có yêu cầu đặc biệt hoặc nhân viên cần ghi chú thông tin quan trọng cho bếp hoặc tài xế. \\
\hline
Pre-Condition & - Nhân viên đang ở màn hình đơn hàng giao hàng. \newline - Có thể đã thêm món hoặc chưa. \\
\hline
Post-Condition & - Ghi chú được đính kèm vào món ăn hoặc đơn hàng. \newline - Ghi chú liên quan đến món ăn sẽ được gửi xuống bếp/bar (UC-MD07-06). \newline - Ghi chú liên quan đến giao hàng sẽ được gửi sang Shipday (UC-MD07-08). \\
\hline
\multicolumn{2}{|c|}{\textbf{2.2. Luồng thực thi (Flow)}} \\
\hline
\textbf{Mục} & \textbf{Nội dung} \\
\hline
Basic Flow, Alternative Flow, Exception Flow & Luồng thực thi, các luồng thay thế và ngoại lệ về cơ bản là giống hệt với Use Case UC-MD05-06: Xử lý Yêu cầu đặc biệt/Ghi chú bếp. Tuy nhiên, cần phân biệt: \newline - Ghi chú cho món ăn (sẽ gửi bếp/bar). \newline - Ghi chú cho giao hàng (sẽ gửi cho tài xế qua Shipday). Giao diện có thể cần có ô ghi chú riêng cho việc giao hàng. \\
\hline
\multicolumn{2}{|c|}{\textbf{2.3. Thông tin bổ sung (Additional Information)}} \\
\hline
\textbf{Mục} & \textbf{Nội dung} \\
\hline
Business Rule & Các Business Rule tương tự như BR-UC5.6-1, BR-UC5.6-2, BR-UC5.6-3. Bổ sung: \newline - \textbf{BR-UC7.5-1:} Cần có cách phân biệt rõ ràng giữa ghi chú cho bếp/bar và ghi chú cho tài xế giao hàng để thông tin được gửi đúng nơi. \newline - \textbf{BR-UC7.5-2:} Ghi chú cho tài xế (ví dụ: "Gọi trước khi đến", "Để hàng ở cổng bảo vệ") phải được truyền sang hệ thống Shipday. \\
\hline
Non-Functional Requirement & Các Non-Functional Requirement tương tự như NFR-UC5.6-1, NFR-UC5.6-2, NFR-UC5.6-3. Việc phân biệt loại ghi chú cần được thiết kế rõ ràng (Usability). \\
\hline
\end{longtable}

\subsubsection{Use Case UC-MD07-06: Gửi đơn Giao hàng xuống Bếp/Bar}

\begin{longtable}{|m{4cm}|p{11cm}|}
\caption{Đặc tả Use Case UC-MD07-06: Gửi đơn Giao hàng xuống Bếp/Bar} \label{tab:uc_md07_06} \\
\hline

\endhead % Header cho các trang tiếp theo
\hline
\endfoot % Footer cho bảng
\hline
\endlastfoot % Footer cho trang cuối cùng
\multicolumn{2}{|c|}{\textbf{2.1. Tóm tắt (Summary)}} \\
\hline
\textbf{Mục} & \textbf{Nội dung} \\
\hline
Use Case Name & Gửi đơn Giao hàng xuống Bếp/Bar \\
\hline
Use Case ID & UC-MD07-06 \\
\hline
Use Case Description & Gửi thông tin các món ăn/đồ uống của đơn hàng giao đi đến các máy in hoặc màn hình KDS tại bộ phận bếp/bar để bắt đầu chuẩn bị. Phiếu gửi đi cần chỉ rõ đây là đơn hàng giao hàng. \\
\hline
Actor & US-02 (Nhân viên phục vụ), US-05 (Nhân viên thu ngân), System \\
\hline
Priority & Must Have \\
\hline
Trigger & Nhân viên đã nhập xong các món cho đơn hàng giao đi và cần thông báo cho bếp/bar. \\
\hline
Pre-Condition & - Nhân viên đang ở màn hình đơn hàng giao hàng. \newline - Có các món ăn chưa được gửi đi trong đơn hàng. \newline - Các thiết bị bếp/bar và quy tắc định tuyến đã được cấu hình. \\
\hline
Post-Condition & - Thông tin các món cần chuẩn bị được gửi đến đúng bộ phận bếp/bar, có đánh dấu là đơn "Delivery". \newline - Trạng thái các món trên POS được cập nhật là "Đã gửi". \\
\hline
\multicolumn{2}{|c|}{\textbf{2.2. Luồng thực thi (Flow)}} \\
\hline
\textbf{Mục} & \textbf{Nội dung} \\
\hline
Basic Flow, Alternative Flow, Exception Flow & Luồng thực thi, các luồng thay thế và ngoại lệ về cơ bản là giống hệt với Use Case UC-MD05-07: Gửi đơn hàng xuống Bếp/Bar. Điểm khác biệt quan trọng là: \newline - Hệ thống cần bao gồm thông tin "Delivery" hoặc "Giao hàng" và có thể cả thông tin khách hàng/địa chỉ tóm tắt (nếu cần) trên dữ liệu gửi đi (bước 5 của UC-MD05-07) để bếp/bar biết cách đóng gói phù hợp. \\
\hline
\multicolumn{2}{|c|}{\textbf{2.3. Thông tin bổ sung (Additional Information)}} \\
\hline
\textbf{Mục} & \textbf{Nội dung} \\
\hline
Business Rule & Các Business Rule tương tự như BR-UC5.7-1, BR-UC5.7-2, BR-UC5.7-3. Bổ sung: \newline - \textbf{BR-UC7.6-1:} Phiếu in/Hiển thị KDS cho đơn giao hàng phải có dấu hiệu rõ ràng (ví dụ: chữ "Delivery", "Giao hàng") và có thể kèm theo tên/SĐT khách hoặc mã đơn hàng để dễ đối chiếu khi đóng gói. \\
\hline
Non-Functional Requirement & Các Non-Functional Requirement tương tự như NFR-UC5.7-1, NFR-UC5.7-2, NFR-UC5.7-3. \\
\hline
\end{longtable}

\subsubsection{Use Case UC-MD07-07: Áp dụng Đặt cọc/Thanh toán Trước (Nếu có)}

\begin{longtable}{|m{4cm}|p{11cm}|}
\caption{Đặc tả Use Case UC-MD07-07: Áp dụng Đặt cọc/Thanh toán Trước (Nếu có)} \label{tab:uc_md07_07} \\
\hline

\endhead % Header cho các trang tiếp theo
\hline
\endfoot % Footer cho bảng
\hline
\endlastfoot % Footer cho trang cuối cùng
\multicolumn{2}{|c|}{\textbf{2.1. Tóm tắt (Summary)}} \\
\hline
\textbf{Mục} & \textbf{Nội dung} \\
\hline
Use Case Name & Áp dụng Đặt cọc/Thanh toán Trước (Nếu có) \\
\hline
Use Case ID & UC-MD07-07 \\
\hline
Use Case Description & Khi chuẩn bị thanh toán cho đơn hàng giao đi, nếu đơn hàng này bắt nguồn từ một kênh online (ví dụ: website/app) và khách hàng đã thanh toán trước một phần (đặt cọc) hoặc toàn bộ giá trị đơn hàng, hệ thống POS cần tự động nhận diện và áp dụng khoản đã thanh toán này vào hóa đơn. \\
\hline
Actor & System (Thực hiện chính), US-02, US-05 (Kích hoạt khi vào màn hình thanh toán) \\
\hline
Priority & Must Have (Nếu có kênh đặt hàng giao đi online và cho phép thanh toán trước/đặt cọc) \\
\hline
Trigger & Nhân viên tại POS mở màn hình thanh toán cho một đơn hàng giao hàng có liên kết với một bản ghi đơn hàng online đã thanh toán trước (một phần hoặc toàn bộ). \\
\hline
Pre-Condition & - Có một hệ thống/luồng cho phép khách hàng đặt hàng giao đi online và thanh toán trước/đặt cọc. \newline - Nhân viên POS đã mở đúng đơn hàng online đó trên giao diện POS. \newline - Đơn hàng online đó có ghi nhận số tiền đã thanh toán trước. \\
\hline
Post-Condition & - Số tiền đã thanh toán trước được tự động trừ vào tổng số tiền cần thanh toán trên màn hình thanh toán POS. \newline - Nếu số tiền đã thanh toán trước bằng hoặc lớn hơn tổng hóa đơn, số tiền cần thanh toán cuối cùng là 0. \newline - Nếu số tiền đã thanh toán trước nhỏ hơn tổng hóa đơn, số tiền cần thanh toán cuối cùng là phần còn lại (COD). \\
\hline
\multicolumn{2}{|c|}{\textbf{2.2. Luồng thực thi (Flow)}} \\
\hline
\textbf{Mục} & \textbf{Nội dung} \\
\hline
Basic Flow, Alternative Flow, Exception Flow & Logic kiểm tra và áp dụng khoản thanh toán trước về cơ bản là giống hệt với Use Case UC-MD05-09: Áp dụng Tiền Đặt cọc vào Hóa đơn. Hệ thống kiểm tra trường lưu số tiền đã thanh toán trước (Prepaid Amount) của đơn hàng liên kết và trừ vào tổng hóa đơn để ra số tiền còn lại cần thu (Amount Due / COD Amount). \\
\hline
\multicolumn{2}{|c|}{\textbf{2.3. Thông tin bổ sung (Additional Information)}} \\
\hline
\textbf{Mục} & \textbf{Nội dung} \\
\hline
Business Rule & Các Business Rule tương tự như BR-UC5.9-1, BR-UC5.9-2, BR-UC5.9-3, BR-UC5.9-4. Đảm bảo hệ thống phân biệt rõ giữa tiền đặt cọc và tiền thanh toán toàn bộ trước. \\
\hline
Non-Functional Requirement & Các Non-Functional Requirement tương tự như NFR-UC5.9-1, NFR-UC5.9-2, NFR-UC5.9-3, NFR-UC5.9-4. \\
\hline
\end{longtable}

\subsubsection{Use Case UC-MD07-08: Xác nhận và Gửi Đơn hàng sang Shipday}

\begin{longtable}{|m{4cm}|p{11cm}|}
\caption{Đặc tả Use Case UC-MD07-08: Xác nhận và Gửi Đơn hàng sang Shipday} \label{tab:uc_md07_08} \\
\hline

\endhead % Header cho các trang tiếp theo
\hline
\endfoot % Footer cho bảng
\hline
\endlastfoot % Footer cho trang cuối cùng
\multicolumn{2}{|c|}{\textbf{2.1. Tóm tắt (Summary)}} \\
\hline
\textbf{Mục} & \textbf{Nội dung} \\
\hline
Use Case Name & Xác nhận và Gửi Đơn hàng sang Shipday \\
\hline
Use Case ID & UC-MD07-08 \\
\hline
Use Case Description & Sau khi đơn hàng giao đi đã được chuẩn bị xong (hoặc gần xong) và thông tin đã đầy đủ, Nhân viên thực hiện hành động trên POS để gửi thông tin chi tiết của đơn hàng này (thông tin khách hàng, địa chỉ, danh sách món, số tiền COD nếu có) đến hệ thống quản lý giao hàng Shipday thông qua API tích hợp. \\
\hline
Actor & US-02 (Nhân viên phục vụ), US-05 (Nhân viên thu ngân), System (Thực hiện gọi API) \\
\hline
Priority & Must Have \\
\hline
Trigger & Đơn hàng giao đi đã sẵn sàng để được điều phối tài xế và giao cho khách. \\
\hline
Pre-Condition & - Nhân viên đang xem chi tiết đơn hàng giao đi trên POS. \newline - Đơn hàng đã có đầy đủ thông tin bắt buộc: khách hàng, SĐT, địa chỉ giao hàng hợp lệ, danh sách món ăn. \newline - Tích hợp API giữa Odoo và Shipday đã được cấu hình thành công (FR-MD07-13). \\
\hline
Post-Condition & - Yêu cầu tạo đơn hàng mới trên Shipday được gửi thành công qua API. \newline - Shipday nhận được thông tin và tạo một tác vụ giao hàng (delivery task) tương ứng. \newline - Đơn hàng trên POS Odoo có thể được cập nhật trạng thái (ví dụ: "Đã gửi Shipday", "Chờ gán tài xế") và/hoặc lưu lại ID đơn hàng từ Shipday để theo dõi. \\
\hline
\multicolumn{2}{|c|}{\textbf{2.2. Luồng thực thi (Flow)}} \\
\hline
\textbf{Mục} & \textbf{Nội dung} \\
\hline
Basic Flow & 1. Nhân viên (US-02/US-05) đang xem chi tiết đơn hàng giao đi đã sẵn sàng. \newline 2. Nhân viên nhấn nút "Gửi Giao hàng" / "Push to Shipday" / "Request Delivery" hoặc tương tự trên giao diện POS. \newline 3. Hệ thống Odoo thu thập các thông tin cần thiết từ đơn hàng POS: \newline    - Tên khách hàng. \newline    - Số điện thoại khách hàng. \newline    - Địa chỉ giao hàng chi tiết (đã được cấu trúc). \newline    - Danh sách các món ăn (có thể chỉ cần tổng số lượng hoặc mô tả chung). \newline    - Tổng giá trị đơn hàng. \newline    - Số tiền cần thu hộ (COD Amount = Amount Due sau khi trừ cọc/trả trước). \newline    - Ghi chú cho tài xế (từ UC-MD07-05). \newline    - Mã đơn hàng Odoo (để tham chiếu). \newline    - (Tùy chọn) Thời gian giao hàng mong muốn (nếu có). \newline 4. Hệ thống Odoo định dạng dữ liệu theo yêu cầu của API Shipday. \newline 5. Hệ thống Odoo thực hiện gọi API "Create Order" (hoặc tương đương) của Shipday, truyền dữ liệu đã chuẩn bị. \newline 6. Hệ thống Odoo chờ và nhận phản hồi từ API Shipday. \newline 7. \textbf{Nếu phản hồi thành công:} \newline    a. Phản hồi thường chứa ID đơn hàng trên Shipday (Shipday Order ID). \newline    b. Hệ thống Odoo lưu lại Shipday Order ID vào đơn hàng POS. \newline    c. Hệ thống Odoo cập nhật trạng thái đơn hàng POS thành "Đã gửi Shipday" hoặc tương đương. \newline    d. Hệ thống hiển thị thông báo "Đã gửi đơn hàng sang Shipday thành công." \newline 8. \textbf{Nếu phản hồi thất bại:} \newline    a. Phản hồi chứa mã lỗi và thông báo lỗi từ Shipday (ví dụ: địa chỉ không hợp lệ, thiếu thông tin bắt buộc...). \newline    b. Hệ thống Odoo hiển thị thông báo lỗi chi tiết cho nhân viên. \newline    c. Đơn hàng chưa được gửi sang Shipday. Nhân viên cần sửa lại thông tin và thử lại (quay lại bước 2). \\
\hline
Alternative Flow & \textbf{2a. Tự động gửi khi đạt trạng thái nhất định:} \newline    1. Hệ thống có thể được cấu hình để tự động gửi đơn hàng sang Shipday khi đơn hàng POS đạt một trạng thái cụ thể (ví dụ: "Sẵn sàng giao") mà không cần nhân viên nhấn nút thủ công. \\
\hline
Exception Flow & \textbf{5a. Lỗi kết nối hoặc gọi API Shipday:} \newline    1. Hệ thống Odoo không thể kết nối đến API của Shipday (lỗi mạng, sai endpoint, API key hết hạn...) hoặc API Shipday trả về lỗi hệ thống (5xx). \newline    2. Hệ thống Odoo ghi nhận lỗi kết nối/API. \newline    3. Hệ thống hiển thị thông báo lỗi chung cho nhân viên "Không thể gửi đơn hàng sang Shipday. Vui lòng thử lại sau hoặc kiểm tra cấu hình." \newline    4. Đơn hàng chưa được gửi. \newline \textbf{7e. Lỗi cập nhật trạng thái/lưu ID trong Odoo:} \newline    1. Sau khi nhận phản hồi thành công từ Shipday, hệ thống Odoo gặp lỗi khi lưu Shipday Order ID hoặc cập nhật trạng thái đơn hàng POS. \newline    2. Hệ thống Odoo ghi nhận lỗi nội bộ. Đơn hàng đã được tạo trên Shipday nhưng Odoo không ghi nhận đúng trạng thái, có thể gây nhầm lẫn. Cần cơ chế cảnh báo/xử lý. \\
\hline
\multicolumn{2}{|c|}{\textbf{2.3. Thông tin bổ sung (Additional Information)}} \\
\hline
\textbf{Mục} & \textbf{Nội dung} \\
\hline
Business Rule & - \textbf{BR-UC7.8-1:} Thông tin gửi sang Shipday phải đầy đủ và chính xác, đặc biệt là địa chỉ giao hàng và số tiền COD (nếu có). \newline - \textbf{BR-UC7.8-2:} Mỗi đơn hàng POS giao đi chỉ nên được gửi sang Shipday một lần. Cần có cơ chế kiểm tra trạng thái để tránh gửi trùng lặp. \newline - \textbf{BR-UC7.8-3:} Việc ánh xạ (mapping) dữ liệu giữa các trường của Odoo và các trường của API Shipday phải được định nghĩa chính xác trong quá trình tích hợp. \\
\hline
Non-Functional Requirement & - \textbf{NFR-UC7.8-1 (Reliability):} Tích hợp API phải ổn định và đáng tin cậy. Cần có xử lý lỗi mạng và lỗi từ API Shipday. \newline - \textbf{NFR-UC7.8-2 (Performance):} Thời gian gửi yêu cầu và nhận phản hồi từ API Shipday nên nhanh chóng để không làm gián đoạn quy trình của nhân viên. \newline - \textbf{NFR-UC7.8-3 (Accuracy):} Dữ liệu truyền đi phải chính xác 100%. \newline - \textbf{NFR-UC7.8-4 (Security):} Việc gọi API phải sử dụng phương thức xác thực an toàn (API Key/Token). \\
\hline
\end{longtable}

\subsubsection{Use Case UC-MD07-09: Nhận và Hiển thị Trạng thái Giao hàng từ Shipday}

\begin{longtable}{|m{4cm}|p{11cm}|}
\caption{Đặc tả Use Case UC-MD07-09: Nhận và Hiển thị Trạng thái Giao hàng từ Shipday} \label{tab:uc_md07_09} \\
\hline

\endhead % Header cho các trang tiếp theo
\hline
\endfoot % Footer cho bảng
\hline
\endlastfoot % Footer cho trang cuối cùng
\multicolumn{2}{|c|}{\textbf{2.1. Tóm tắt (Summary)}} \\
\hline
\textbf{Mục} & \textbf{Nội dung} \\
\hline
Use Case Name & Nhận và Hiển thị Trạng thái Giao hàng từ Shipday \\
\hline
Use Case ID & UC-MD07-09 \\
\hline
Use Case Description & Hệ thống Odoo tự động nhận các cập nhật về trạng thái của đơn hàng giao đi từ hệ thống Shipday (thường thông qua cơ chế Webhook) và hiển thị trạng thái này (ví dụ: Đã nhận đơn, Đã gán tài xế, Đang lấy hàng, Đang giao, Đã giao thành công, Giao thất bại) trên giao diện chi tiết đơn hàng trong Odoo (POS hoặc Backend). \\
\hline
Actor & System (Odoo Backend - Nhận Webhook, Shipday - Gửi Webhook) \\
\hline
Priority & Must Have \\
\hline
Trigger & Shipday cập nhật trạng thái của một tác vụ giao hàng (delivery task) và gửi thông báo cập nhật đó đến endpoint webhook đã đăng ký của Odoo. \\
\hline
Pre-Condition & - Đơn hàng đã được gửi thành công sang Shipday (UC-MD07-08). \newline - Odoo đã lưu lại ID đơn hàng Shipday tương ứng. \newline - Webhook endpoint của Odoo đã được đăng ký và cấu hình đúng trên Shipday để nhận cập nhật trạng thái. \newline - Webhook endpoint của Odoo đang hoạt động và sẵn sàng nhận yêu cầu. \\
\hline
Post-Condition & - Trạng thái giao hàng của đơn hàng POS tương ứng trong Odoo được cập nhật theo thông tin nhận được từ Shipday. \newline - Nhân viên có thể xem trạng thái giao hàng mới nhất khi xem chi tiết đơn hàng. \\
\hline
\multicolumn{2}{|c|}{\textbf{2.2. Luồng thực thi (Flow)}} \\
\hline
\textbf{Mục} & \textbf{Nội dung} \\
\hline
Basic Flow & 1. Shipday cập nhật trạng thái một đơn hàng (ví dụ: tài xế A chấp nhận đơn, tài xế B bắt đầu giao, tài xế C giao thành công). \newline 2. Shipday tự động gửi một yêu cầu HTTP (webhook) đến endpoint đã đăng ký của Odoo, chứa thông tin về đơn hàng (Shipday Order ID hoặc Mã đơn hàng Odoo) và trạng thái mới. \newline 3. Endpoint webhook của Odoo nhận được yêu cầu. \newline 4. Hệ thống Odoo xác thực yêu cầu webhook (nếu có cơ chế bảo mật). \newline 5. Hệ thống Odoo phân tích dữ liệu webhook, xác định đơn hàng POS tương ứng dựa trên ID. \newline 6. Hệ thống Odoo đọc trạng thái giao hàng mới từ dữ liệu webhook (ví dụ: "ASSIGNED", "STARTED", "PICKED\_UP", "DELIVERED", "FAILED"). \newline 7. Hệ thống Odoo cập nhật một trường trạng thái giao hàng (ví dụ: `delivery\_status`) trên bản ghi đơn hàng POS với giá trị mới nhận được. \newline 8. Hệ thống Odoo gửi phản hồi HTTP 200 OK cho Shipday để xác nhận đã nhận webhook. \newline 9. Khi nhân viên xem chi tiết đơn hàng POS đó trên Odoo (Backend hoặc có thể cả POS nếu thiết kế), họ sẽ thấy trường trạng thái giao hàng hiển thị giá trị mới nhất (ví dụ: "Đang giao", "Đã giao thành công"). \\
\hline
Alternative Flow & \textbf{7a. Ánh xạ trạng thái:} \newline    1. Trạng thái từ Shipday có thể cần được ánh xạ sang các trạng thái tương ứng trong Odoo nếu Odoo sử dụng bộ trạng thái riêng. \newline \textbf{9a. Thông báo nội bộ khi có cập nhật quan trọng:} \newline    1. Hệ thống Odoo có thể được cấu hình để gửi thông báo nội bộ (Odoo Discuss) cho nhân viên liên quan khi có cập nhật trạng thái quan trọng (ví dụ: "Đã giao thành công", "Giao thất bại"). \\
\hline
Exception Flow & \textbf{3a. Lỗi nhận Webhook / Xác thực thất bại:} Tương tự UC-MD04-03. \newline \textbf{5a. Không tìm thấy Đơn hàng POS tương ứng:} Tương tự UC-MD04-03. \newline \textbf{7a. Lỗi cập nhật trạng thái trong Odoo:} \newline    1. Hệ thống gặp lỗi khi cố gắng lưu trạng thái giao hàng mới vào đơn hàng POS. \newline    2. Hệ thống ghi nhận lỗi nội bộ. Trạng thái trên Odoo không được cập nhật dù Shipday đã gửi. Cần cơ chế cảnh báo/xử lý. \\
\hline
\multicolumn{2}{|c|}{\textbf{2.3. Thông tin bổ sung (Additional Information)}} \\
\hline
\textbf{Mục} & \textbf{Nội dung} \\
\hline
Business Rule & - \textbf{BR-UC7.9-1:} Hệ thống Odoo phải có khả năng nhận và xử lý webhook cập nhật trạng thái từ Shipday một cách đáng tin cậy. \newline - \textbf{BR-UC7.9-2:} Trạng thái giao hàng hiển thị trên Odoo phải phản ánh đúng trạng thái mới nhất nhận được từ Shipday. \newline - \textbf{BR-UC7.9-3:} Cần định nghĩa rõ các trạng thái giao hàng của Shipday sẽ được hiển thị/ánh xạ như thế nào trong Odoo. \\
\hline
Non-Functional Requirement & - \textbf{NFR-UC7.9-1 (Reliability):} Webhook endpoint phải có độ sẵn sàng cao. Quá trình xử lý webhook và cập nhật trạng thái phải đáng tin cậy. \newline - \textbf{NFR-UC7.9-2 (Performance):} Việc xử lý webhook và cập nhật trạng thái nên diễn ra nhanh chóng để thông tin trên Odoo không bị quá trễ so với thực tế. \newline - \textbf{NFR-UC7.9-3 (Security):} Webhook endpoint cần được bảo mật để tránh bị lạm dụng. \newline - \textbf{NFR-UC7.9-4 (Usability):} Việc hiển thị trạng thái giao hàng trên chi tiết đơn hàng Odoo phải rõ ràng và dễ hiểu cho nhân viên. \\
\hline
\end{longtable}

\subsubsection{Use Case UC-MD07-10: Xử lý Thanh toán Đơn hàng Giao hàng (Nếu COD)}

\begin{longtable}{|m{4cm}|p{11cm}|}
\caption{Đặc tả Use Case UC-MD07-10: Xử lý Thanh toán Đơn hàng Giao hàng (Nếu COD)} \label{tab:uc_md07_10} \\
\hline

\endhead % Header cho các trang tiếp theo
\hline
\endfoot % Footer cho bảng
\hline
\endlastfoot % Footer cho trang cuối cùng
\multicolumn{2}{|c|}{\textbf{2.1. Tóm tắt (Summary)}} \\
\hline
\textbf{Mục} & \textbf{Nội dung} \\
\hline
Use Case Name & Xử lý Thanh toán Đơn hàng Giao hàng (Nếu COD) \\
\hline
Use Case ID & UC-MD07-10 \\
\hline
Use Case Description & Đối với các đơn hàng giao đi thanh toán khi nhận hàng (Cash on Delivery - COD), cho phép Nhân viên (thường là Thu ngân hoặc Quản lý) ghi nhận việc đã nhận đủ số tiền COD từ tài xế giao hàng vào hệ thống POS/Odoo, từ đó hoàn tất quy trình thanh toán cho đơn hàng. \\
\hline
Actor & US-02 (Nhân viên phục vụ), US-05 (Nhân viên thu ngân), US-01 (Quản lý nhà hàng) \\
\hline
Priority & Must Have (Nếu hỗ trợ hình thức thanh toán COD) \\
\hline
Trigger & Tài xế giao hàng quay lại cửa hàng và nộp tiền COD đã thu từ khách cho đơn hàng giao đi. \\
\hline
Pre-Condition & - Đơn hàng giao đi đã được đánh dấu là COD và đã được giao thành công (trạng thái từ Shipday - UC-MD07-09 - là "DELIVERED"). \newline - Đơn hàng trên Odoo đang ở trạng thái chờ thanh toán (hoặc một trạng thái tương ứng với COD đã giao). \newline - Nhân viên có quyền ghi nhận thanh toán COD. \\
\hline
Post-Condition & - Giao dịch thanh toán COD được ghi nhận thành công trong hệ thống Odoo, liên kết với đơn hàng giao đi. \newline - Trạng thái thanh toán của đơn hàng được cập nhật thành "Đã thanh toán" (Paid). \newline - Số dư tiền mặt của phiên POS (nếu nhận tiền mặt) hoặc tài khoản đối ứng (nếu tài xế chuyển khoản) được cập nhật. \newline - Đơn hàng sẵn sàng để đóng cuối cùng (UC-MD07-12). \\
\hline
\multicolumn{2}{|c|}{\textbf{2.2. Luồng thực thi (Flow)}} \\
\hline
\textbf{Mục} & \textbf{Nội dung} \\
\hline
Basic Flow & 1. Nhân viên (US-02/US-05/US-01) nhận tiền COD từ tài xế giao hàng. \newline 2. Nhân viên tìm và mở lại đơn hàng giao đi tương ứng trên POS hoặc Backend Odoo (có thể lọc theo trạng thái "Đã giao, chờ thanh toán COD"). \newline 3. Nhân viên xác minh số tiền COD cần thu khớp với số tiền tài xế nộp. \newline 4. Nhân viên truy cập chức năng ghi nhận thanh toán cho đơn hàng này (có thể là nút "Register Payment" hoặc tương tự). \newline 5. Hệ thống hiển thị giao diện ghi nhận thanh toán, thường đã điền sẵn số tiền COD cần thu. \newline 6. Nhân viên chọn phương thức thanh toán mà tài xế nộp tiền (ví dụ: "Tiền mặt" nếu tài xế nộp tiền mặt, hoặc "Chuyển khoản" nếu tài xế chuyển khoản cho nhà hàng). \newline 7. Nhân viên xác nhận thông tin thanh toán. \newline 8. Hệ thống ghi nhận giao dịch thanh toán COD. \newline 9. Hệ thống cập nhật trạng thái đơn hàng thành "Paid". \newline 10. Hệ thống hiển thị thông báo ghi nhận thanh toán thành công. \\
\hline
Alternative Flow & \textbf{6a. Tài xế nộp thiếu/thừa tiền:} \newline    1. Nếu số tiền tài xế nộp khác với số tiền COD cần thu. \newline    2. Nhân viên cần xử lý theo quy trình của nhà hàng (ví dụ: ghi nhận số tiền thực nhận và tạo bút toán chênh lệch, yêu cầu tài xế nộp đủ...). Quy trình này có thể phức tạp và cần sự can thiệp của quản lý/kế toán. Hệ thống POS cơ bản có thể chỉ cho phép ghi nhận đúng số tiền COD. \newline \textbf{4a. Ghi nhận thanh toán hàng loạt:} \newline    1. Nếu có nhiều đơn COD cần ghi nhận cùng lúc, hệ thống có thể cung cấp chức năng chọn nhiều đơn và ghi nhận thanh toán hàng loạt (thường ở Backend). \\
\hline
Exception Flow & \textbf{8a. Lỗi ghi nhận thanh toán:} \newline    1. Hệ thống gặp lỗi kỹ thuật khi cố gắng tạo bản ghi thanh toán hoặc cập nhật trạng thái đơn hàng. \newline    2. Hệ thống báo lỗi. Thanh toán chưa được ghi nhận đúng. \newline \textbf{2a. Không tìm thấy đơn hàng / Đơn hàng sai trạng thái:} \newline    1. Nhân viên không tìm thấy đơn hàng hoặc đơn hàng không ở trạng thái phù hợp để ghi nhận thanh toán COD. \newline    2. Cần kiểm tra lại thông tin đơn hàng và trạng thái giao hàng từ Shipday. \\
\hline
\multicolumn{2}{|c|}{\textbf{2.3. Thông tin bổ sung (Additional Information)}} \\
\hline
\textbf{Mục} & \textbf{Nội dung} \\
\hline
Business Rule & - \textbf{BR-UC7.10-1:} Chỉ những đơn hàng được xác định là COD và đã có trạng thái giao hàng thành công từ Shipday mới được phép ghi nhận thanh toán theo luồng này. \newline - \textbf{BR-UC7.10-2:} Số tiền ghi nhận thanh toán phải khớp với số tiền COD cần thu của đơn hàng. Việc xử lý chênh lệch (nếu có) cần tuân theo quy định của nhà hàng. \newline - \textbf{BR-UC7.10-3:} Phương thức thanh toán ghi nhận phải phản ánh đúng cách thức tài xế nộp tiền (tiền mặt, chuyển khoản...). \\
\hline
Non-Functional Requirement & - \textbf{NFR-UC7.10-1 (Usability):} Việc tìm đơn hàng COD đã giao và thực hiện ghi nhận thanh toán phải đơn giản cho nhân viên. \newline - \textbf{NFR-UC7.10-2 (Accuracy):} Việc ghi nhận đúng số tiền và đúng phương thức thanh toán là rất quan trọng cho đối soát tài chính. \newline - \textbf{NFR-UC7.10-3 (Auditability):} Cần ghi log rõ ràng về việc ai đã ghi nhận thanh toán COD, thời gian, số tiền. \\
\hline
\end{longtable}

\subsubsection{Use Case UC-MD07-11: In Hóa đơn/Phiếu Giao hàng}

\begin{longtable}{|m{4cm}|p{11cm}|}
\caption{Đặc tả Use Case UC-MD07-11: In Hóa đơn/Phiếu Giao hàng} \label{tab:uc_md07_11} \\
\hline

\endhead % Header cho các trang tiếp theo
\hline
\endfoot % Footer cho bảng
\hline
\endlastfoot % Footer cho trang cuối cùng
\multicolumn{2}{|c|}{\textbf{2.1. Tóm tắt (Summary)}} \\
\hline
\textbf{Mục} & \textbf{Nội dung} \\
\hline
Use Case Name & In Hóa đơn/Phiếu Giao hàng \\
\hline
Use Case ID & UC-MD07-11 \\
\hline
Use Case Description & Cho phép Nhân viên in ra một hoặc nhiều bản hóa đơn hoặc phiếu giao hàng chi tiết cho đơn hàng giao đi. Phiếu này thường được đính kèm cùng gói hàng để tài xế tham khảo và giao cho khách hàng. \\
\hline
Actor & US-02 (Nhân viên phục vụ), US-05 (Nhân viên thu ngân) \\
\hline
Priority & Must Have \\
\hline
Trigger & - Sau khi đơn hàng giao đi được xác nhận và gửi bếp/bar (để chuẩn bị phiếu). \newline - Hoặc sau khi thanh toán thành công (nếu là hóa đơn cuối cùng). \newline - Hoặc khi tài xế đến lấy hàng. \\
\hline
Pre-Condition & - Nhân viên đang xem chi tiết đơn hàng giao đi trên POS hoặc Backend. \newline - Máy in hóa đơn/phiếu đã được cấu hình và kết nối. \newline - Mẫu in phù hợp cho đơn giao hàng đã được thiết lập. \\
\hline
Post-Condition & - Một hoặc nhiều bản hóa đơn/phiếu giao hàng được in ra. \\
\hline
\multicolumn{2}{|c|}{\textbf{2.2. Luồng thực thi (Flow)}} \\
\hline
\textbf{Mục} & \textbf{Nội dung} \\
\hline
Basic Flow & 1. Nhân viên (US-02/US-05) đang xem chi tiết đơn hàng giao đi. \newline 2. Nhân viên chọn nút "In Hóa đơn" / "In Phiếu Giao hàng" / "Print Receipt". \newline 3. Hệ thống (có thể) hỏi số lượng bản in mong muốn. Nhân viên nhập số lượng. \newline 4. Hệ thống tạo dữ liệu cần in, bao gồm: \newline    - Thông tin nhà hàng. \newline    - Thông tin khách hàng (Tên, SĐT, Địa chỉ giao hàng). \newline    - Mã đơn hàng. \newline    - Danh sách chi tiết các món ăn (SL, Tên, Đơn giá, Thành tiền). \newline    - Tổng tiền hàng, Thuế. \newline    - Tiền đặt cọc/đã trả trước (nếu có). \newline    - Số tiền COD cần thu (nếu có). \newline    - Ghi chú cho tài xế/khách hàng. \newline 5. Hệ thống gửi dữ liệu đến máy in hóa đơn/phiếu đã cấu hình. \newline 6. Máy in in ra (các) bản hóa đơn/phiếu giao hàng. \\
\hline
Alternative Flow & \textbf{4a. Mẫu in khác nhau:} \newline    1. Hệ thống có thể có các mẫu in khác nhau (ví dụ: Phiếu giao hàng cho tài xế chỉ có địa chỉ, SĐT, COD; Hóa đơn chi tiết cho khách). \newline    2. Nhân viên chọn đúng mẫu cần in. \newline \textbf{2a. Tự động in khi gửi Shipday:} \newline    1. Hệ thống có thể được cấu hình để tự động in phiếu giao hàng ngay sau khi gửi đơn thành công sang Shipday (UC-MD07-08). \\
\hline
Exception Flow & \textbf{5a. Lỗi gửi lệnh in / Lỗi máy in:} Tương tự UC-MD05-08. \\
\hline
\multicolumn{2}{|c|}{\textbf{2.3. Thông tin bổ sung (Additional Information)}} \\
\hline
\textbf{Mục} & \textbf{Nội dung} \\
\hline
Business Rule & - \textbf{BR-UC7.11-1:} Phiếu giao hàng/hóa đơn phải chứa đủ thông tin cần thiết cho tài xế (địa chỉ, SĐT khách, tiền COD) và cho khách hàng (chi tiết đơn hàng, tổng tiền). \newline - \textbf{BR-UC7.11-2:} Thông tin trên phiếu in phải khớp với thông tin đơn hàng trong hệ thống tại thời điểm in. \\
\hline
Non-Functional Requirement & - \textbf{NFR-UC7.11-1 (Usability):} Nút in phải dễ tìm. Nếu có nhiều mẫu in, việc lựa chọn phải rõ ràng. \newline - \textbf{NFR-UC7.11-2 (Clarity):} Định dạng phiếu in phải rõ ràng, dễ đọc, các thông tin quan trọng (địa chỉ, COD) phải nổi bật. \\
\hline
\end{longtable}

\subsubsection{Use Case UC-MD07-12: Đóng Đơn hàng Giao hàng}

\begin{longtable}{|m{4cm}|p{11cm}|}
\caption{Đặc tả Use Case UC-MD07-12: Đóng Đơn hàng Giao hàng} \label{tab:uc_md07_12} \\
\hline

\endhead % Header cho các trang tiếp theo
\hline
\endfoot % Footer cho bảng
\hline
\endlastfoot % Footer cho trang cuối cùng
\multicolumn{2}{|c|}{\textbf{2.1. Tóm tắt (Summary)}} \\
\hline
\textbf{Mục} & \textbf{Nội dung} \\
\hline
Use Case Name & Đóng Đơn hàng Giao hàng \\
\hline
Use Case ID & UC-MD07-12 \\
\hline
Use Case Description & Hoàn tất vòng đời của một đơn hàng giao đi trong hệ thống Odoo sau khi đã xác nhận giao hàng thành công (từ Shipday) và đã xử lý xong vấn đề thanh toán (khách trả trước hoặc đã thu COD). \\
\hline
Actor & US-02 (Nhân viên phục vụ), US-05 (Nhân viên thu ngân), System (Có thể tự động đóng) \\
\hline
Priority & Must Have \\
\hline
Trigger & - Đơn hàng giao đi có trạng thái giao hàng là "Đã giao thành công" (từ UC-MD07-09). \newline - Và trạng thái thanh toán là "Đã thanh toán" (Paid - do trả trước hoặc đã thu COD UC-MD07-10). \\
\hline
Pre-Condition & - Đơn hàng giao đi ở trạng thái "Đã giao thành công" VÀ "Đã thanh toán". \\
\hline
Post-Condition & - Trạng thái cuối cùng của đơn hàng POS giao đi được cập nhật thành "Đã hoàn thành" (Done) hoặc tương đương. \newline - Đơn hàng không còn xuất hiện trong danh sách các đơn hàng cần xử lý. \\
\hline
\multicolumn{2}{|c|}{\textbf{2.2. Luồng thực thi (Flow)}} \\
\hline
\textbf{Mục} & \textbf{Nội dung} \\
\hline
Basic Flow (Tự động đóng) & 1. Hệ thống (ví dụ: qua một tác vụ tự động hoặc ngay khi nhận đủ trạng thái) phát hiện một đơn hàng giao đi có trạng thái giao hàng = "DELIVERED" VÀ trạng thái thanh toán = "Paid". \newline 2. Hệ thống tự động cập nhật trạng thái của đơn hàng POS đó thành "Done" hoặc "Completed". \newline 3. Hệ thống ghi nhận hoạt động đóng đơn. \\
\hline
Alternative Flow & \textbf{Basic Flow (Đóng thủ công):} \newline    1. Nhân viên (US-02/US-05) xem chi tiết một đơn hàng giao đi đã thỏa mãn điều kiện đóng (đã giao, đã thanh toán). \newline    2. Nhân viên nhấn nút "Đóng đơn hàng" / "Mark as Done". \newline    3. Hệ thống cập nhật trạng thái thành "Done". \newline    4. Hệ thống ghi nhận hoạt động. \\
\hline
Exception Flow & \textbf{2a/3a. Lỗi cập nhật trạng thái cuối cùng:} \newline    1. Hệ thống gặp lỗi kỹ thuật khi cố gắng cập nhật trạng thái "Done". \newline    2. Hệ thống báo lỗi. Đơn hàng có thể vẫn ở trạng thái trước đó. \\
\hline
\multicolumn{2}{|c|}{\textbf{2.3. Thông tin bổ sung (Additional Information)}} \\
\hline
\textbf{Mục} & \textbf{Nội dung} \\
\hline
Business Rule & - \textbf{BR-UC7.12-1:} Một đơn hàng giao đi chỉ nên được đóng cuối cùng khi đã xác nhận cả việc giao hàng thành công và việc thanh toán hoàn tất. \newline - \textbf{BR-UC7.12-2:} Việc đóng đơn nên được tự động hóa dựa trên trạng thái giao hàng và thanh toán để giảm thao tác thủ công. \\
\hline
Non-Functional Requirement & - \textbf{NFR-UC7.12-1 (Automation):} Ưu tiên quy trình đóng đơn tự động để đảm bảo tính kịp thời và giảm sai sót. \newline - \textbf{NFR-UC7.12-2 (Reliability):} Logic kiểm tra điều kiện và cập nhật trạng thái đóng đơn phải đáng tin cậy. \\
\hline
\end{longtable}

\subsubsection{Use Case UC-MD07-13: Cấu hình Tích hợp Shipday}

\begin{longtable}{|m{4cm}|p{11cm}|}
\caption{Đặc tả Use Case UC-MD07-13: Cấu hình Tích hợp Shipday} \label{tab:uc_md07_13} \\
\hline

\endhead % Header cho các trang tiếp theo
\hline
\endfoot % Footer cho bảng
\hline
\endlastfoot % Footer cho trang cuối cùng
\multicolumn{2}{|c|}{\textbf{2.1. Tóm tắt (Summary)}} \\
\hline
\textbf{Mục} & \textbf{Nội dung} \\
\hline
Use Case Name & Cấu hình Tích hợp Shipday \\
\hline
Use Case ID & UC-MD07-13 \\
\hline
Use Case Description & Cho phép Quản lý nhà hàng hoặc Quản trị viên hệ thống thiết lập các tham số cần thiết để kết nối và trao đổi dữ liệu giữa hệ thống Odoo và nền tảng quản lý giao hàng Shipday, bao gồm thông tin xác thực API và các quy tắc đồng bộ dữ liệu cơ bản. \\
\hline
Actor & US-01 (Quản lý nhà hàng), US-10 (Quản trị viên Hệ thống) \\
\hline
Priority & Must Have \\
\hline
Trigger & - Thiết lập lần đầu cho việc tích hợp Odoo với Shipday. \newline - Cần cập nhật thông tin API Key hoặc các cài đặt tích hợp khác. \\
\hline
Pre-Condition & - Người dùng (US-01 hoặc US-10) đã đăng nhập với quyền quản trị cấu hình hệ thống hoặc cấu hình module Giao hàng/Tích hợp. \newline - Nhà hàng đã có tài khoản Shipday và đã lấy được API Key từ Shipday. \\
\hline
Post-Condition & - Thông tin kết nối API và các quy tắc tích hợp cơ bản giữa Odoo và Shipday được lưu lại trong cấu hình hệ thống Odoo. \newline - Hệ thống Odoo sẵn sàng để gửi đơn hàng sang Shipday (UC-MD07-08) và nhận cập nhật trạng thái từ Shipday (UC-MD07-09). \\
\hline
\multicolumn{2}{|c|}{\textbf{2.2. Luồng thực thi (Flow)}} \\
\hline
\textbf{Mục} & \textbf{Nội dung} \\
\hline
Basic Flow & 1. Người dùng (US-01/US-10) truy cập vào khu vực Cài đặt chung của hệ thống hoặc Cài đặt của module Giao hàng/Tích hợp bên thứ ba. \newline 2. Người dùng tìm đến phần cấu hình liên quan đến "Shipday Integration" hoặc tương tự. \newline 3. Hệ thống hiển thị form cấu hình với các trường: \newline    - \textbf{Kích hoạt Tích hợp Shipday:} Ô kiểm để bật/tắt. \newline    - \textbf{Shipday API Key:} Trường để nhập API Key do Shipday cung cấp. \newline    - (Tùy chọn) \textbf{Shipday API Endpoint:} URL của API Shipday (thường là cố định). \newline    - (Tùy chọn) \textbf{Webhook URL (của Odoo):} Hệ thống hiển thị URL webhook mà người dùng cần cấu hình bên phía Shipday để Shipday gửi cập nhật trạng thái về Odoo. \newline    - (Tùy chọn) Các cài đặt khác như: Tự động gửi đơn sang Shipday khi nào, Ánh xạ trạng thái Odoo-Shipday (nếu cần tùy chỉnh)... \newline 4. Người dùng nhập hoặc cập nhật các giá trị cấu hình, đặc biệt là API Key. \newline 5. (Tùy chọn) Người dùng sao chép Webhook URL để cấu hình trên tài khoản Shipday. \newline 6. Người dùng chọn hành động "Lưu" (Save). \newline 7. Hệ thống kiểm tra tính hợp lệ cơ bản (ví dụ: API Key không được trống). \newline 8. Hệ thống lưu lại các cấu hình mới. \newline 9. Hệ thống hiển thị thông báo lưu thành công. \\
\hline
Alternative Flow & \textbf{3a. Kiểm tra kết nối API:} \newline    1. Giao diện cấu hình có nút "Kiểm tra kết nối" / "Test Connection". \newline    2. Người dùng nhấp nút này sau khi nhập API Key. \newline    3. Hệ thống Odoo thực hiện một lời gọi API đơn giản đến Shipday (ví dụ: lấy thông tin tài khoản) để xác thực API Key. \newline    4. Hệ thống hiển thị kết quả kiểm tra (Thành công/Thất bại kèm lý do). \\
\hline
Exception Flow & \textbf{7a. Lỗi Xác thực Dữ liệu:} \newline    1. Hệ thống phát hiện thiếu API Key hoặc định dạng không hợp lệ. \newline    2. Hệ thống báo lỗi. \newline    3. Không lưu cấu hình. Use Case quay lại bước 4. \newline \textbf{8a. Lỗi Hệ thống khi Lưu:} \newline    1. Hệ thống gặp sự cố kỹ thuật khi lưu cấu hình. \newline    2. Hệ thống báo lỗi chung. \\
\hline
\multicolumn{2}{|c|}{\textbf{2.3. Thông tin bổ sung (Additional Information)}} \\
\hline
\textbf{Mục} & \textbf{Nội dung} \\
\hline
Business Rule & - \textbf{BR-UC7.13-1:} API Key của Shipday phải chính xác và còn hiệu lực. \newline - \textbf{BR-UC7.13-2:} Webhook URL do Odoo cung cấp phải được cấu hình đúng bên phía Shipday để đảm bảo Odoo nhận được cập nhật trạng thái. \newline - \textbf{BR-UC7.13-3:} Các quy tắc ánh xạ dữ liệu (nếu có) cần được thiết lập cẩn thận để đảm bảo thông tin được truyền đi và nhận về đúng cách. \\
\hline
Non-Functional Requirement & - \textbf{NFR-UC7.13-1 (Usability):} Giao diện cấu hình tích hợp phải rõ ràng, dễ hiểu các trường cần nhập. Việc kiểm tra kết nối API là một tính năng hữu ích. \newline - \textbf{NFR-UC7.13-2 (Security):} API Key phải được lưu trữ an toàn, không hiển thị trực tiếp sau khi lưu. \\
\hline
\end{longtable}


\subsection{Module MD-08: Tích hợp Bếp (Kitchen Integration)}

\subsubsection{Use Case UC-MD08-02: Xem Đơn hàng trên KDS}

\begin{longtable}{|m{4cm}|p{11cm}|}
\caption{Đặc tả Use Case UC-MD08-02: Xem Đơn hàng trên KDS} \label{tab:uc_md08_02} \\
\hline

\endhead % Header cho các trang tiếp theo
\hline
\endfoot % Footer cho bảng
\hline
\endlastfoot % Footer cho trang cuối cùng
\multicolumn{2}{|c|}{\textbf{2.1. Tóm tắt (Summary)}} \\
\hline
\textbf{Mục} & \textbf{Nội dung} \\
\hline
Use Case Name & Xem Đơn hàng trên KDS \\
\hline
Use Case ID & UC-MD08-02 \\
\hline
Use Case Description & Cho phép Nhân viên bếp (US-04) xem danh sách các đơn hàng (hoặc các phiếu/ticket) đang chờ xử lý, mới được gửi đến từ POS trên giao diện Màn hình Hiển thị Bếp (Kitchen Display System - KDS). \\
\hline
Actor & US-04 (Nhân viên bếp) \\
\hline
Priority & Must Have (Nếu sử dụng KDS thay vì/cùng với máy in bếp) \\
\hline
Trigger & - Nhân viên bếp bắt đầu ca làm việc và cần xem các đơn hàng đang chờ. \newline - Có đơn hàng mới được gửi từ POS đến KDS. \newline - Nhân viên bếp hoàn thành một món/đơn và cần xem đơn hàng tiếp theo. \\
\hline
Pre-Condition & - Màn hình KDS (thiết bị vật lý như tablet, màn hình cảm ứng) đã được cài đặt, kết nối mạng và chạy ứng dụng/giao diện KDS của Odoo. \newline - KDS đã được cấu hình trong POS để nhận đơn hàng từ các danh mục sản phẩm phù hợp (FR-MD02-10). \newline - Có ít nhất một đơn hàng đã được gửi từ POS đến KDS này (thông qua FR-MD08-01). \\
\hline
Post-Condition & - Giao diện KDS hiển thị danh sách các đơn hàng/phiếu đang chờ xử lý. \newline - Mỗi đơn hàng/phiếu hiển thị các thông tin cơ bản cần thiết để nhân viên bếp bắt đầu công việc. \\
\hline
\multicolumn{2}{|c|}{\textbf{2.2. Luồng thực thi (Flow)}} \\
\hline
\textbf{Mục} & \textbf{Nội dung} \\
\hline
Basic Flow & 1. Nhân viên bếp (US-04) nhìn vào màn hình KDS. \newline 2. Giao diện KDS hiển thị các đơn hàng/phiếu (tickets) dưới dạng các ô hoặc cột. Các đơn hàng mới đến thường xuất hiện ở vị trí đầu tiên hoặc có dấu hiệu mới. \newline 3. Mỗi ô/phiếu đại diện cho một phần hoặc toàn bộ đơn hàng từ POS, hiển thị các thông tin tóm tắt ban đầu: \newline    - Mã đơn hàng POS / Số bàn / Loại đơn (Eat-in, Takeout, Delivery). \newline    - Thời gian gửi đơn. \newline    - Danh sách các món ăn thuộc phạm vi xử lý của KDS này (Tên món, Số lượng). \newline    - (Tùy chọn) Thời gian chờ ước tính hoặc màu sắc chỉ thị độ khẩn cấp. \newline 4. Nhân viên bếp xem xét các đơn hàng đang chờ trên màn hình. \\
\hline
Alternative Flow & \textbf{2a. Chế độ xem khác nhau:} \newline    1. Giao diện KDS có thể cung cấp các chế độ xem khác nhau (ví dụ: xem theo món ăn thay vì theo đơn hàng, xem theo trạm chế biến nếu có). \newline    2. Nhân viên bếp chọn chế độ xem phù hợp với quy trình làm việc. \newline \textbf{3a. Thông báo đơn hàng mới:} \newline    1. Khi có đơn hàng mới được gửi đến, KDS phát ra âm thanh thông báo hoặc có hiệu ứng nhấp nháy để thu hút sự chú ý của nhân viên bếp. \\
\hline
Exception Flow & \textbf{2a. Lỗi hiển thị KDS:} \newline    1. Giao diện KDS gặp lỗi, không hiển thị được đơn hàng hoặc hiển thị sai thông tin (ví dụ: lỗi kết nối mạng, lỗi ứng dụng KDS). \newline    2. Nhân viên bếp không nhận được hoặc không xem được đơn hàng mới. Cần khắc phục sự cố thiết bị/ứng dụng. \newline \textbf{2b. Không có đơn hàng nào:} \newline    1. Nếu không có đơn hàng nào đang chờ xử lý cho KDS này, màn hình hiển thị trạng thái trống hoặc thông báo "Không có đơn hàng nào". \\
\hline
\multicolumn{2}{|c|}{\textbf{2.3. Thông tin bổ sung (Additional Information)}} \\
\hline
\textbf{Mục} & \textbf{Nội dung} \\
\hline
Business Rule & - \textbf{BR-UC8.2-1:} KDS phải hiển thị các đơn hàng/phiếu theo thứ tự ưu tiên hợp lý (ví dụ: thứ tự thời gian gửi đến, hoặc theo cấu hình ưu tiên khác). \newline - \textbf{BR-UC8.2-2:} Thông tin hiển thị ban đầu phải đủ để nhân viên bếp nắm bắt nhanh công việc cần làm. \newline - \textbf{BR-UC8.2-3:} KDS chỉ hiển thị các món ăn thuộc các Danh mục POS đã được cấu hình định tuyến đến KDS đó (FR-MD02-10). \\
\hline
Non-Functional Requirement & - \textbf{NFR-UC8.2-1 (Usability):} Giao diện KDS phải cực kỳ dễ đọc trong môi trường bếp (font chữ lớn, độ tương phản cao). Cách trình bày đơn hàng/phiếu phải rõ ràng, không gây nhầm lẫn. \newline - \textbf{NFR-UC8.2-2 (Performance):} Đơn hàng mới gửi từ POS phải xuất hiện trên KDS gần như tức thời (< 2-3 giây). Giao diện KDS phải hoạt động mượt mà. \newline - \textbf{NFR-UC8.2-3 (Reliability):} KDS phải hoạt động ổn định, đảm bảo không bỏ sót đơn hàng. Cần có cơ chế xử lý khi mất kết nối mạng tạm thời (ví dụ: lưu trữ offline và đồng bộ lại). \newline - \textbf{NFR-UC8.2-4 (Durability):} Thiết bị KDS sử dụng trong bếp cần có độ bền phù hợp với môi trường (chịu nhiệt, dầu mỡ, va đập nhẹ). \\
\hline
\end{longtable}

\subsubsection{Use Case UC-MD08-03: Thay đổi Trạng thái Món ăn/Đơn hàng trên KDS}

\begin{longtable}{|m{4cm}|p{11cm}|}
\caption{Đặc tả Use Case UC-MD08-03: Thay đổi Trạng thái Món ăn/Đơn hàng trên KDS} \label{tab:uc_md08_03} \\
\hline

\endhead % Header cho các trang tiếp theo
\hline
\endfoot % Footer cho bảng
\hline
\endlastfoot % Footer cho trang cuối cùng
\multicolumn{2}{|c|}{\textbf{2.1. Tóm tắt (Summary)}} \\
\hline
\textbf{Mục} & \textbf{Nội dung} \\
\hline
Use Case Name & Thay đổi Trạng thái Món ăn/Đơn hàng trên KDS \\
\hline
Use Case ID & UC-MD08-03 \\
\hline
Use Case Description & Cho phép Nhân viên bếp (US-04) tương tác trực tiếp với màn hình KDS (thường là cảm ứng) để cập nhật trạng thái của từng món ăn hoặc toàn bộ đơn hàng/phiếu, ví dụ: đánh dấu đang chế biến, đã hoàn thành. \\
\hline
Actor & US-04 (Nhân viên bếp) \\
\hline
Priority & Must Have (Nếu sử dụng KDS) \\
\hline
Trigger & - Nhân viên bếp bắt đầu chế biến một món ăn/đơn hàng. \newline - Nhân viên bếp hoàn thành việc chế biến một món ăn/đơn hàng. \\
\hline
Pre-Condition & - Nhân viên bếp đang xem danh sách đơn hàng trên KDS (UC-MD08-02). \newline - Đơn hàng/món ăn đang ở trạng thái có thể thay đổi (ví dụ: đang chờ xử lý). \\
\hline
Post-Condition & - Trạng thái của món ăn hoặc đơn hàng/phiếu trên KDS được cập nhật (ví dụ: chuyển màu, chuyển sang cột/khu vực khác, biến mất khỏi danh sách chờ). \newline - (Tùy chọn) Thông tin trạng thái mới được gửi về hệ thống Odoo backend và có thể cập nhật về POS (FR-MD08-06). \newline - Giúp theo dõi tiến độ công việc trong bếp. \\
\hline
\multicolumn{2}{|c|}{\textbf{2.2. Luồng thực thi (Flow)}} \\
\hline
\textbf{Mục} & \textbf{Nội dung} \\
\hline
Basic Flow (Đánh dấu hoàn thành đơn hàng/phiếu) & 1. Nhân viên bếp (US-04) đã hoàn thành tất cả các món trong một đơn hàng/phiếu hiển thị trên KDS. \newline 2. US-04 chạm vào đơn hàng/phiếu đó trên màn hình KDS. \newline 3. Giao diện KDS có thể hiển thị các tùy chọn hành động hoặc có một hành động mặc định khi chạm (ví dụ: đánh dấu hoàn thành). \newline 4. US-04 thực hiện hành động "Hoàn thành" / "Done" / "Ready". \newline 5. Đơn hàng/phiếu đó biến mất khỏi danh sách đang chờ hoặc chuyển sang khu vực/trạng thái "Đã xong". \newline 6. (Tùy chọn) Hệ thống KDS gửi thông tin cập nhật trạng thái về Odoo backend/POS. \\
\hline
Alternative Flow & \textbf{2a. Đánh dấu trạng thái "Đang làm":} \newline    1. Khi bắt đầu làm một đơn hàng/phiếu, US-04 chạm vào nó và chọn trạng thái "Đang làm" (In Progress / Cooking). \newline    2. Đơn hàng/phiếu có thể đổi màu hoặc di chuyển sang khu vực "Đang làm" trên KDS. \newline \textbf{2b. Đánh dấu hoàn thành từng món ăn:} \newline    1. Nếu KDS hỗ trợ quản lý theo từng món, US-04 chạm vào một món ăn cụ thể trong đơn hàng/phiếu. \newline    2. US-04 chọn hành động "Hoàn thành" cho món đó. \newline    3. Món ăn đó được đánh dấu là đã xong (ví dụ: gạch đi, đổi màu). \newline    4. Khi tất cả các món trong đơn hàng/phiếu đều hoàn thành, toàn bộ đơn hàng/phiếu có thể tự động chuyển sang trạng thái "Đã xong" hoặc yêu cầu xác nhận cuối cùng từ nhân viên. \newline \textbf{2c. Hoàn tác trạng thái:} \newline    1. Nếu đánh dấu nhầm, US-04 có thể có tùy chọn để hoàn tác hành động vừa thực hiện (ví dụ: chuyển món "Đã xong" về lại "Đang làm"). \\
\hline
Exception Flow & \textbf{4a. Lỗi khi cập nhật trạng thái trên KDS/Odoo:} \newline    1. Hệ thống KDS hoặc kết nối về backend Odoo gặp lỗi khi cố gắng lưu trạng thái mới. \newline    2. KDS có thể hiển thị thông báo lỗi hoặc trạng thái không được cập nhật đúng cách. \newline    3. Cần kiểm tra kết nối và thử lại. \\
\hline
\multicolumn{2}{|c|}{\textbf{2.3. Thông tin bổ sung (Additional Information)}} \\
\hline
\textbf{Mục} & \textbf{Nội dung} \\
\hline
Business Rule & - \textbf{BR-UC8.3-1:} Phải có cách thức rõ ràng và dễ dàng để nhân viên bếp cập nhật trạng thái trên KDS (ví dụ: chạm, vuốt). \newline - \textbf{BR-UC8.3-2:} Các trạng thái khả dụng (ví dụ: Chờ, Đang làm, Đã xong) và luồng chuyển đổi giữa chúng cần được định nghĩa phù hợp với quy trình làm việc của bếp. \newline - \textbf{BR-UC8.3-3:} Việc cập nhật trạng thái phải chính xác và phản ánh đúng tiến độ thực tế. \\
\hline
Non-Functional Requirement & - \textbf{NFR-UC8.3-1 (Usability):} Tương tác cảm ứng trên KDS phải nhạy và chính xác. Các nút/khu vực chạm phải đủ lớn và dễ thao tác trong môi trường bếp. \newline - \textbf{NFR-UC8.3-2 (Performance):} Phản hồi khi chạm và cập nhật trạng thái trên màn hình phải tức thời. \newline - \textbf{NFR-UC8.3-3 (Reliability):} Việc cập nhật trạng thái phải đáng tin cậy, không bị mất dữ liệu khi có sự cố tạm thời. \\
\hline
\end{longtable}

\subsubsection{Use Case UC-MD08-04: Xem Chi tiết Món ăn trên KDS}

\begin{longtable}{|m{4cm}|p{11cm}|}
\caption{Đặc tả Use Case UC-MD08-04: Xem Chi tiết Món ăn trên KDS} \label{tab:uc_md08_04} \\
\hline

\endhead % Header cho các trang tiếp theo
\hline
\endfoot % Footer cho bảng
\hline
\endlastfoot % Footer cho trang cuối cùng
\multicolumn{2}{|c|}{\textbf{2.1. Tóm tắt (Summary)}} \\
\hline
\textbf{Mục} & \textbf{Nội dung} \\
\hline
Use Case Name & Xem Chi tiết Món ăn trên KDS \\
\hline
Use Case ID & UC-MD08-04 \\
\hline
Use Case Description & Cho phép Nhân viên bếp (US-04) xem đầy đủ các thông tin chi tiết liên quan đến một món ăn cụ thể cần chế biến, bao gồm tên, số lượng, các tùy chọn biến thể đã chọn, và các ghi chú đặc biệt từ khách hàng hoặc nhân viên phục vụ. \\
\hline
Actor & US-04 (Nhân viên bếp) \\
\hline
Priority & Must Have (Nếu sử dụng KDS) \\
\hline
Trigger & Nhân viên bếp cần xem chi tiết một món ăn hiển thị trên KDS để bắt đầu chế biến hoặc kiểm tra lại yêu cầu. \\
\hline
Pre-Condition & - Nhân viên bếp đang xem danh sách đơn hàng/phiếu trên KDS (UC-MD08-02). \newline - Đơn hàng/phiếu chứa món ăn cần xem chi tiết. \\
\hline
Post-Condition & - Thông tin chi tiết của món ăn được hiển thị rõ ràng cho nhân viên bếp. \\
\hline
\multicolumn{2}{|c|}{\textbf{2.2. Luồng thực thi (Flow)}} \\
\hline
\textbf{Mục} & \textbf{Nội dung} \\
\hline
Basic Flow & 1. Nhân viên bếp (US-04) đang xem một đơn hàng/phiếu trên KDS (UC-MD08-02). \newline 2. Đơn hàng/phiếu hiển thị danh sách các món ăn cần làm. \newline 3. Với mỗi món ăn, KDS hiển thị các thông tin chi tiết quan trọng: \newline    - \textbf{Số lượng:} Số phần cần làm. \newline    - \textbf{Tên món ăn:} Tên đầy đủ của món. \newline    - \textbf{Biến thể:} Các tùy chọn biến thể khách đã chọn (ví dụ: "Size: L", "Độ chín: Medium Rare", "Không cay"). \newline    - \textbf{Ghi chú đặc biệt:} Các ghi chú do nhân viên phục vụ nhập (ví dụ: "Không hành", "Dị ứng hải sản", "Món này ra trước"). \newline 4. Nhân viên bếp đọc kỹ các thông tin này để đảm bảo chế biến đúng yêu cầu. \\
\hline
Alternative Flow & \textbf{1a. Nhấp để xem chi tiết hơn:} \newline    1. Nếu thông tin ban đầu bị ẩn bớt do giới hạn không gian, nhân viên có thể nhấp vào dòng món ăn để xem đầy đủ các biến thể và ghi chú trong một cửa sổ popup hoặc khu vực riêng. \\
\hline
Exception Flow & \textbf{3a. Thông tin bị thiếu hoặc không rõ ràng:} \newline    1. Do lỗi truyền dữ liệu từ POS hoặc lỗi hiển thị của KDS, một số thông tin chi tiết (biến thể, ghi chú) bị thiếu hoặc hiển thị không chính xác. \newline    2. Nhân viên bếp không có đủ thông tin để chế biến đúng. Cần liên hệ lại nhân viên phục vụ hoặc kiểm tra lại hệ thống. \\
\hline
\multicolumn{2}{|c|}{\textbf{2.3. Thông tin bổ sung (Additional Information)}} \\
\hline
\textbf{Mục} & \textbf{Nội dung} \\
\hline
Business Rule & - \textbf{BR-UC8.4-1:} KDS phải hiển thị đầy đủ và chính xác tất cả các thông tin liên quan đến việc chế biến món ăn: số lượng, tên, biến thể, ghi chú. \newline - \textbf{BR-UC8.4-2:} Các thông tin quan trọng như ghi chú dị ứng hoặc yêu cầu đặc biệt phải được làm nổi bật để nhân viên bếp dễ dàng nhận thấy. \\
\hline
Non-Functional Requirement & - \textbf{NFR-UC8.4-1 (Clarity/Readability):} Thông tin chi tiết món ăn trên KDS phải cực kỳ rõ ràng, dễ đọc, font chữ đủ lớn, bố cục hợp lý. \newline - \textbf{NFR-UC8.4-2 (Accuracy):} Dữ liệu hiển thị phải khớp 100\% với dữ liệu đã được gửi từ POS. \\
\hline
\end{longtable}

\subsubsection{Use Case UC-MD08-05: (Tùy chọn) Sắp xếp/Ưu tiên Đơn hàng trên KDS}

\begin{longtable}{|m{4cm}|p{11cm}|}
\caption{Đặc tả Use Case UC-MD08-05: (Tùy chọn) Sắp xếp/Ưu tiên Đơn hàng trên KDS} \label{tab:uc_md08_05} \\
\hline

\endhead % Header cho các trang tiếp theo
\hline
\endfoot % Footer cho bảng
\hline
\endlastfoot % Footer cho trang cuối cùng
\multicolumn{2}{|c|}{\textbf{2.1. Tóm tắt (Summary)}} \\
\hline
\textbf{Mục} & \textbf{Nội dung} \\
\hline
Use Case Name & (Tùy chọn) Sắp xếp/Ưu tiên Đơn hàng trên KDS \\
\hline
Use Case ID & UC-MD08-05 \\
\hline
Use Case Description & Cung cấp khả năng cho Nhân viên bếp (US-04) sắp xếp lại thứ tự hiển thị của các đơn hàng/phiếu trên KDS hoặc đánh dấu một số đơn hàng/phiếu là ưu tiên cần xử lý trước, dựa trên các yếu tố như thời gian chờ, yêu cầu đặc biệt hoặc chỉ đạo của quản lý. \\
\hline
Actor & US-04 (Nhân viên bếp) \\
\hline
Priority & Low / Nice to Have \\
\hline
Trigger & - Bếp nhận được nhiều đơn hàng cùng lúc và cần sắp xếp lại thứ tự làm việc. \newline - Có một đơn hàng cần được ưu tiên xử lý (ví dụ: khách VIP, khách đợi lâu). \\
\hline
Pre-Condition & - Nhân viên bếp đang xem danh sách đơn hàng trên KDS (UC-MD08-02). \newline - Giao diện KDS được thiết kế hỗ trợ chức năng sắp xếp hoặc đánh dấu ưu tiên. \\
\hline
Post-Condition & - Thứ tự hiển thị của các đơn hàng/phiếu trên KDS được thay đổi theo ý muốn của nhân viên. \newline - Các đơn hàng/phiếu được đánh dấu ưu tiên có dấu hiệu nhận biết rõ ràng. \\
\hline
\multicolumn{2}{|c|}{\textbf{2.2. Luồng thực thi (Flow)}} \\
\hline
\textbf{Mục} & \textbf{Nội dung} \\
\hline
Basic Flow (Đánh dấu ưu tiên) & 1. Nhân viên bếp (US-04) xác định đơn hàng/phiếu cần ưu tiên trên KDS. \newline 2. US-04 chạm vào đơn hàng/phiếu đó. \newline 3. Giao diện hiển thị tùy chọn "Đánh dấu Ưu tiên" (Prioritize) hoặc tương tự. \newline 4. US-04 chọn tùy chọn đó. \newline 5. Đơn hàng/phiếu được đánh dấu ưu tiên (ví dụ: đổi sang màu khác, có biểu tượng cờ đỏ, di chuyển lên đầu danh sách). \\
\hline
Alternative Flow & \textbf{1a. Sắp xếp lại thứ tự (Kéo thả):} \newline    1. Giao diện KDS cho phép kéo và thả các đơn hàng/phiếu để thay đổi vị trí của chúng trong danh sách chờ. \newline    2. US-04 nhấn giữ và kéo một đơn hàng/phiếu đến vị trí mong muốn. \newline    3. Hệ thống cập nhật lại thứ tự hiển thị. \newline \textbf{1b. Sắp xếp theo tiêu chí:} \newline    1. Giao diện KDS có các nút/tùy chọn để sắp xếp toàn bộ danh sách theo các tiêu chí khác nhau (ví dụ: Thời gian chờ lâu nhất, Thời gian gửi đơn mới nhất...). \newline    2. US-04 chọn tiêu chí sắp xếp. \newline    3. Hệ thống sắp xếp lại danh sách. \\
\hline
Exception Flow & \textbf{4a/2a-drag/2a-sort. Lỗi khi sắp xếp/đánh dấu:} \newline    1. Hệ thống KDS gặp lỗi khi cố gắng lưu lại thứ tự mới hoặc trạng thái ưu tiên. \newline    2. Thao tác có thể không thành công hoặc hiển thị không đúng. \\
\hline
\multicolumn{2}{|c|}{\textbf{2.3. Thông tin bổ sung (Additional Information)}} \\
\hline
\textbf{Mục} & \textbf{Nội dung} \\
\hline
Business Rule & - \textbf{BR-UC8.5-1:} Chức năng sắp xếp/ưu tiên là tùy chọn, không bắt buộc phải có. \newline - \textbf{BR-UC8.5-2:} Việc đánh dấu ưu tiên cần có biểu hiện trực quan rõ ràng để tất cả nhân viên bếp đều nhận biết được. \\
\hline
Non-Functional Requirement & - \textbf{NFR-UC8.5-1 (Usability):} Thao tác sắp xếp (kéo thả) hoặc đánh dấu ưu tiên phải dễ dàng thực hiện trên màn hình cảm ứng. \newline - \textbf{NFR-UC8.5-2 (Performance):} Việc sắp xếp lại danh sách hoặc đánh dấu ưu tiên phải có hiệu lực ngay lập tức trên màn hình. \\
\hline
\end{longtable}

\subsubsection{Use Case UC-MD08-06: Cập nhật Trạng thái Món ăn về POS (Tùy chọn)}

\begin{longtable}{|m{4cm}|p{11cm}|}
\caption{Đặc tả Use Case UC-MD08-06: Cập nhật Trạng thái Món ăn về POS (Tùy chọn)} \label{tab:uc_md08_06} \\
\hline

\endhead % Header cho các trang tiếp theo
\hline
\endfoot % Footer cho bảng
\hline
\endlastfoot % Footer cho trang cuối cùng
\multicolumn{2}{|c|}{\textbf{2.1. Tóm tắt (Summary)}} \\
\hline
\textbf{Mục} & \textbf{Nội dung} \\
\hline
Use Case Name & Cập nhật Trạng thái Món ăn về POS (Tùy chọn) \\
\hline
Use Case ID & UC-MD08-06 \\
\hline
Use Case Description & Khi Nhân viên bếp cập nhật trạng thái của một món ăn hoặc đơn hàng trên KDS (ví dụ: đánh dấu "Đã xong"), hệ thống KDS (nếu được cấu hình) sẽ tự động gửi thông tin cập nhật này về lại hệ thống POS, cho phép Nhân viên phục vụ biết được món nào đã sẵn sàng để mang ra cho khách. \\
\hline
Actor & System (KDS gửi cập nhật, POS nhận cập nhật) \\
\hline
Priority & Nice to Have \\
\hline
Trigger & Nhân viên bếp thay đổi trạng thái món ăn/đơn hàng trên KDS thành "Đã xong" (hoặc trạng thái tương đương) (UC-MD08-03). \\
\hline
Pre-Condition & - KDS và POS được kết nối cùng mạng và có cơ chế giao tiếp hai chiều (thường qua Odoo backend). \newline - Chức năng cập nhật trạng thái từ KDS về POS được kích hoạt trong cấu hình. \\
\hline
Post-Condition & - Trạng thái của món ăn tương ứng trên giao diện đơn hàng POS của Nhân viên phục vụ được cập nhật (ví dụ: hiển thị icon "Sẵn sàng", đổi màu). \newline - Nhân viên phục vụ nhận được thông báo (trực quan hoặc âm thanh - tùy thiết kế) về món ăn đã sẵn sàng. \\
\hline
\multicolumn{2}{|c|}{\textbf{2.2. Luồng thực thi (Flow)}} \\
\hline
\textbf{Mục} & \textbf{Nội dung} \\
\hline
Basic Flow & 1. Nhân viên bếp đánh dấu một món ăn/đơn hàng là "Đã xong" trên KDS (UC-MD08-03). \newline 2. Hệ thống KDS gửi thông tin cập nhật trạng thái (bao gồm ID món ăn/đơn hàng và trạng thái mới) đến Odoo backend. \newline 3. Odoo backend nhận thông tin và xác định đơn hàng POS tương ứng. \newline 4. Odoo backend cập nhật trạng thái của (các) dòng món ăn liên quan trong đơn hàng POS đó. \newline 5. Odoo backend gửi tín hiệu cập nhật đến giao diện POS client của nhân viên đang mở đơn hàng đó (thường qua longpolling hoặc websocket). \newline 6. Giao diện POS client nhận được cập nhật. \newline 7. Giao diện POS cập nhật hiển thị của dòng món ăn đó (ví dụ: thêm icon "✓", đổi màu nền). \newline 8. (Tùy chọn) Giao diện POS phát ra âm thanh thông báo ngắn hoặc hiển thị một popup nhỏ báo "Món [Tên món] của bàn [Số bàn] đã sẵn sàng". \\
\hline
Alternative Flow & Không có luồng thay thế đáng kể. \\
\hline
Exception Flow & \textbf{2a. Lỗi gửi cập nhật từ KDS:} \newline    1. KDS không thể gửi thông tin cập nhật về backend (lỗi mạng...). \newline    2. Trạng thái trên POS không được cập nhật. \newline \textbf{5a. Lỗi gửi tín hiệu đến POS client:} \newline    1. Backend không thể gửi tín hiệu cập nhật đến POS client (client offline, lỗi kết nối...). \newline    2. Trạng thái trên POS không được cập nhật tức thời, chỉ cập nhật khi làm mới đơn hàng. \\
\hline
\multicolumn{2}{|c|}{\textbf{2.3. Thông tin bổ sung (Additional Information)}} \\
\hline
\textbf{Mục} & \textbf{Nội dung} \\
\hline
Business Rule & - \textbf{BR-UC8.6-1:} Chức năng cập nhật trạng thái 2 chiều này là tùy chọn và cần được cấu hình. \newline - \textbf{BR-UC8.6-2:} Chỉ trạng thái "Đã xong" (hoặc tương đương) mới nên kích hoạt thông báo/cập nhật về POS để tránh làm phiền nhân viên phục vụ với các trạng thái trung gian. \newline - \textbf{BR-UC8.6-3:} Cơ chế thông báo trên POS cho nhân viên phục vụ cần rõ ràng nhưng không quá gây mất tập trung. \\
\hline
Non-Functional Requirement & - \textbf{NFR-UC8.6-1 (Performance/Real-time):} Việc cập nhật trạng thái từ KDS về POS nên diễn ra gần như thời gian thực để nhân viên phục vụ có thông tin kịp thời. \newline - \textbf{NFR-UC8.6-2 (Reliability):} Cơ chế giao tiếp giữa KDS, backend và POS client phải đáng tin cậy. \\
\hline
\end{longtable}

\subsubsection{Use Case UC-MD08-07: Nhận và Xử lý Phiếu in Bếp}

\begin{longtable}{|m{4cm}|p{11cm}|}
\caption{Đặc tả Use Case UC-MD08-07: Nhận và Xử lý Phiếu in Bếp} \label{tab:uc_md08_07} \\
\hline

\endhead % Header cho các trang tiếp theo
\hline
\endfoot % Footer cho bảng
\hline
\endlastfoot % Footer cho trang cuối cùng
\multicolumn{2}{|c|}{\textbf{2.1. Tóm tắt (Summary)}} \\
\hline
\textbf{Mục} & \textbf{Nội dung} \\
\hline
Use Case Name & Nhận và Xử lý Phiếu in Bếp \\
\hline
Use Case ID & UC-MD08-07 \\
\hline
Use Case Description & Mô tả quy trình thủ công của Nhân viên bếp (US-04) khi nhận được phiếu in đơn hàng từ máy in bếp, đọc thông tin trên phiếu và thực hiện chế biến các món ăn theo yêu cầu. \\
\hline
Actor & US-04 (Nhân viên bếp) \\
\hline
Priority & Must Have (Nếu sử dụng máy in bếp thay vì/cùng với KDS) \\
\hline
Trigger & Máy in bếp in ra một phiếu đơn hàng mới (kích hoạt bởi FR-MD08-01). \\
\hline
Pre-Condition & - Máy in bếp đang hoạt động và có giấy. \newline - Một phiếu đơn hàng hợp lệ vừa được in ra. \\
\hline
Post-Condition & - Nhân viên bếp nắm được các món cần chế biến cho đơn hàng đó. \newline - Nhân viên bếp bắt đầu quá trình chế biến. (Hệ thống không tự động biết được điều này). \\
\hline
\multicolumn{2}{|c|}{\textbf{2.2. Luồng thực thi (Flow)}} \\
\hline
\textbf{Mục} & \textbf{Nội dung} \\
\hline
Basic Flow & 1. Máy in bếp in ra phiếu đơn hàng. \newline 2. Nhân viên bếp (US-04) lấy phiếu in. \newline 3. US-04 đọc các thông tin trên phiếu: \newline    - Số bàn / Loại đơn (Takeout/Delivery) / Tên khách (nếu có). \newline    - Thời gian gửi đơn. \newline    - Tên nhân viên phục vụ. \newline    - Danh sách các món ăn cần chế biến (Tên món, Số lượng, Biến thể, Ghi chú đặc biệt). \newline 4. US-04 xác định các món cần làm và bắt đầu quy trình chế biến theo thứ tự ưu tiên hoặc quy trình của bếp. \newline 5. US-04 giữ lại phiếu in để tham chiếu trong quá trình làm và đối chiếu khi món ăn hoàn thành. \\
\hline
Alternative Flow & \textbf{3a. Phiếu in bị mờ/rách/không rõ ràng:} \newline    1. Nhân viên bếp không đọc được rõ thông tin trên phiếu. \newline    2. Nhân viên cần liên hệ lại nhân viên phục vụ đã gửi đơn để xác nhận lại thông tin. \\
\hline
Exception Flow & \textbf{1a. Máy in lỗi (Hết giấy, kẹt...):} \newline    1. Phiếu không được in ra hoặc in không hoàn chỉnh. \newline    2. Nhân viên bếp không nhận được đơn hàng. Cần xử lý sự cố máy in và yêu cầu gửi lại đơn từ POS (nếu có chức năng in lại). \\
\hline
\multicolumn{2}{|c|}{\textbf{2.3. Thông tin bổ sung (Additional Information)}} \\
\hline
\textbf{Mục} & \textbf{Nội dung} \\
\hline
Business Rule & - \textbf{BR-UC8.7-1:} Định dạng phiếu in bếp phải rõ ràng, dễ đọc, font chữ đủ lớn, các thông tin quan trọng (số lượng, ghi chú) phải nổi bật. \newline - \textbf{BR-UC8.7-2:} Thông tin in ra phải khớp chính xác với những gì đã được gửi từ POS. \newline - \textbf{BR-UC8.7-3:} Cần có quy trình trong bếp để quản lý các phiếu in (ví dụ: treo lên bảng theo thứ tự, đánh dấu khi hoàn thành) để tránh nhầm lẫn hoặc bỏ sót. \\
\hline
Non-Functional Requirement & - \textbf{NFR-UC8.7-1 (Clarity/Readability):} Yêu cầu quan trọng nhất là phiếu in phải dễ đọc trong môi trường bếp. \newline - \textbf{NFR-UC8.7-2 (Reliability):} Máy in bếp cần hoạt động ổn định, ít gặp sự cố. \\
\hline
\end{longtable}



\subsection{Module MD-09: Quản lý Phiên \& Báo cáo}

\subsubsection{Use Case UC-MD09-03: Xem Báo cáo Doanh thu Phiên POS}

\begin{longtable}{|m{4cm}|p{11cm}|}
\caption{Đặc tả Use Case UC-MD09-03: Xem Báo cáo Doanh thu Phiên POS} \label{tab:uc_md09_03} \\
\hline

\endhead % Header cho các trang tiếp theo
\hline
\endfoot % Footer cho bảng
\hline
\endlastfoot % Footer cho trang cuối cùng
\multicolumn{2}{|c|}{\textbf{2.1. Tóm tắt (Summary)}} \\
\hline
\textbf{Mục} & \textbf{Nội dung} \\
\hline
Use Case Name & Xem Báo cáo Doanh thu Phiên POS \\
\hline
Use Case ID & UC-MD09-03 \\
\hline
Use Case Description & Cho phép Quản lý nhà hàng hoặc Kế toán xem lại thông tin tổng kết chi tiết của một hoặc nhiều phiên làm việc POS đã đóng, bao gồm doanh thu, số lượng đơn hàng, chi tiết thanh toán theo từng phương thức, tiền mặt đối chiếu (nếu có), và tiền boa. \\
\hline
Actor & US-01 (Quản lý nhà hàng), US-06 (Kế toán) \\
\hline
Priority & Must Have \\
\hline
Trigger & Cần xem lại kết quả kinh doanh của một ngày/ca làm việc cụ thể, đối soát doanh thu, hoặc kiểm tra thông tin của một phiên POS đã đóng. \\
\hline
Pre-Condition & - Người dùng (US-01 hoặc US-06) đã đăng nhập vào hệ thống Odoo với quyền truy cập báo cáo POS. \newline - Có ít nhất một phiên làm việc POS đã được đóng (UC-MD05-13 thành công). \\
\hline
Post-Condition & - Báo cáo chi tiết của (các) phiên POS đã chọn được hiển thị. \newline - Người dùng nắm được thông tin tổng kết về hiệu quả hoạt động của phiên đó. \\
\hline
\multicolumn{2}{|c|}{\textbf{2.2. Luồng thực thi (Flow)}} \\
\hline
\textbf{Mục} & \textbf{Nội dung} \\
\hline
Basic Flow & 1. Người dùng (US-01/US-06) truy cập vào module Point of Sale. \newline 2. Người dùng chọn mục "Báo cáo" (Reporting) > "Phiên làm việc" (Sessions) hoặc tương tự. \newline 3. Hệ thống hiển thị danh sách các phiên POS đã đóng, thường sắp xếp theo ngày giờ đóng phiên. Danh sách hiển thị thông tin tóm tắt như ID phiên, Người mở/đóng, Thời gian mở/đóng, Tổng doanh thu. \newline 4. Người dùng chọn (nhấp vào) một phiên cụ thể muốn xem chi tiết. \newline 5. Hệ thống hiển thị báo cáo chi tiết cho phiên đã chọn, bao gồm các thông tin: \newline    - Thông tin chung: ID phiên, POS, Người mở/đóng, Thời gian mở/đóng. \newline    - Tóm tắt Doanh thu: Tổng doanh thu (đã bao gồm thuế), Tổng thuế, Doanh thu thuần (chưa thuế). \newline    - Chi tiết Thanh toán: Tổng số tiền nhận được theo từng phương thức thanh toán (Tiền mặt, Thẻ, Ví...). \newline    - (Nếu có kiểm soát tiền mặt): Số dư đầu ca, Tiền mặt dự kiến cuối ca, Tiền mặt thực tế cuối ca, Chênh lệch. \newline    - Số lượng đơn hàng đã xử lý trong phiên. \newline    - Tổng tiền boa (Tip) thu được (nếu có). \newline    - (Tùy chọn) Danh sách các đơn hàng thuộc phiên đó. \newline    - (Tùy chọn) Các thông tin khác như giảm giá, hủy đơn... \newline 6. Người dùng xem xét báo cáo. \\
\hline
Alternative Flow & \textbf{3a. Lọc/Tìm kiếm phiên:} \newline    1. Người dùng sử dụng bộ lọc (theo ngày, theo POS, theo trạng thái) hoặc tìm kiếm (theo ID phiên, theo người dùng) để tìm phiên mong muốn. \newline    2. Hệ thống hiển thị kết quả lọc/tìm kiếm. \newline    3. Use Case tiếp tục từ bước 4. \newline \textbf{3b. Xem báo cáo tổng hợp nhiều phiên:} \newline    1. Thay vì chọn một phiên, người dùng chọn xem báo cáo tổng hợp cho một khoảng thời gian (ví dụ: theo ngày, theo tuần). \newline    2. Hệ thống tổng hợp dữ liệu từ tất cả các phiên trong khoảng thời gian đó và hiển thị báo cáo tổng kết. \newline \textbf{6a. In báo cáo:} \newline    1. Giao diện báo cáo có nút "In" (Print). \newline    2. Người dùng nhấn nút In. \newline    3. Hệ thống tạo bản PDF hoặc gửi lệnh in trực tiếp báo cáo chi tiết phiên. \\
\hline
Exception Flow & \textbf{3a. Không có phiên nào đã đóng:} \newline    1. Nếu chưa có phiên nào được đóng, danh sách ở bước 3 sẽ trống. \newline    2. Hệ thống hiển thị thông báo "Không có dữ liệu phiên." \newline \textbf{4a. Lỗi tải chi tiết phiên:} \newline    1. Hệ thống gặp lỗi khi truy vấn hoặc tính toán dữ liệu chi tiết của phiên đã chọn. \newline    2. Hệ thống hiển thị thông báo lỗi. \\
\hline
\multicolumn{2}{|c|}{\textbf{2.3. Thông tin bổ sung (Additional Information)}} \\
\hline
\textbf{Mục} & \textbf{Nội dung} \\
\hline
Business Rule & - \textbf{BR-UC9.3-1:} Báo cáo phải hiển thị chính xác các số liệu tài chính đã được tổng kết khi đóng phiên. \newline - \textbf{BR-UC9.3-2:} Dữ liệu thanh toán theo từng phương thức phải khớp với tổng số tiền đã thu qua các phương thức đó trong suốt phiên. \newline - \textbf{BR-UC9.3-3:} Nếu có kiểm soát tiền mặt, báo cáo phải hiển thị rõ ràng số tiền đối chiếu và khoản chênh lệch (nếu có). \\
\hline
Non-Functional Requirement & - \textbf{NFR-UC9.3-1 (Usability):} Giao diện báo cáo phải rõ ràng, dễ đọc, các số liệu quan trọng phải dễ nhận biết. Việc lọc và tìm kiếm phiên phải thuận tiện. \newline - \textbf{NFR-UC9.3-2 (Performance):} Thời gian tải danh sách phiên và chi tiết một phiên phải nhanh chóng. Việc tổng hợp báo cáo cho khoảng thời gian dài cần có hiệu năng chấp nhận được. \newline - \textbf{NFR-UC9.3-3 (Accuracy):} Mọi số liệu trong báo cáo phải chính xác và nhất quán với dữ liệu giao dịch gốc. \newline - \textbf{NFR-UC9.3-4 (Security):} Chỉ người dùng có quyền hạn (Quản lý, Kế toán) mới được phép truy cập các báo cáo tài chính này. \\
\hline
\end{longtable}

\subsubsection{Use Case UC-MD09-04: Xem Báo cáo Bán hàng theo Sản phẩm/Danh mục}

\begin{longtable}{|m{4cm}|p{11cm}|}
\caption{Đặc tả Use Case UC-MD09-04: Xem Báo cáo Bán hàng theo Sản phẩm/Danh mục} \label{tab:uc_md09_04} \\
\hline

\endhead % Header cho các trang tiếp theo
\hline
\endfoot % Footer cho bảng
\hline
\endlastfoot % Footer cho trang cuối cùng
\multicolumn{2}{|c|}{\textbf{2.1. Tóm tắt (Summary)}} \\
\hline
\textbf{Mục} & \textbf{Nội dung} \\
\hline
Use Case Name & Xem Báo cáo Bán hàng theo Sản phẩm/Danh mục \\
\hline
Use Case ID & UC-MD09-04 \\
\hline
Use Case Description & Cung cấp báo cáo thống kê chi tiết về số lượng bán ra và doanh thu (trước và sau thuế, trước và sau giảm giá) của từng sản phẩm (món ăn/đồ uống) hoặc nhóm theo Danh mục Sản phẩm POS trong một khoảng thời gian do người dùng lựa chọn. \\
\hline
Actor & US-01 (Quản lý nhà hàng), US-06 (Kế toán) \\
\hline
Priority & Must Have \\
\hline
Trigger & Cần phân tích hiệu quả bán hàng của từng món ăn, xác định món bán chạy/chậm, hoặc xem xét cơ cấu doanh thu theo từng nhóm món. \\
\hline
Pre-Condition & - Người dùng (US-01 hoặc US-06) đã đăng nhập với quyền truy cập báo cáo POS/Bán hàng. \newline - Đã có dữ liệu giao dịch bán hàng từ các phiên POS đã đóng. \newline - Các sản phẩm và danh mục POS đã được định nghĩa. \\
\hline
Post-Condition & - Báo cáo thống kê bán hàng theo sản phẩm hoặc danh mục được hiển thị cho khoảng thời gian đã chọn. \newline - Người dùng có thông tin để đánh giá hiệu quả kinh doanh của từng mặt hàng. \\
\hline
\multicolumn{2}{|c|}{\textbf{2.2. Luồng thực thi (Flow)}} \\
\hline
\textbf{Mục} & \textbf{Nội dung} \\
\hline
Basic Flow (Xem theo Sản phẩm) & 1. Người dùng (US-01/US-06) truy cập vào mục Báo cáo (Reporting) của POS hoặc Sales. \newline 2. Người dùng chọn loại báo cáo "Bán hàng theo Sản phẩm" (Sales by Product) hoặc tương tự. \newline 3. Hệ thống yêu cầu hoặc cho phép người dùng chọn Khoảng thời gian (Date Range) muốn xem báo cáo (ví dụ: Hôm nay, Tuần này, Tháng này, Tùy chọn). \newline 4. Người dùng chọn khoảng thời gian và nhấn "Xem báo cáo" / "Apply". \newline 5. Hệ thống truy vấn và tổng hợp dữ liệu từ các dòng đơn hàng (POS Order Lines) trong các phiên POS đã đóng thuộc khoảng thời gian đã chọn. \newline 6. Hệ thống hiển thị báo cáo dưới dạng bảng, mỗi dòng là một Sản phẩm (hoặc Biến thể sản phẩm), với các cột thông tin như: \newline    - Tên Sản phẩm/Biến thể. \newline    - Số lượng đã bán (Quantity Sold). \newline    - Doanh thu thuần (Untaxed Total / Net Sales). \newline    - Tổng giảm giá (Total Discount - nếu có). \newline    - Tổng doanh thu (bao gồm thuế - Total Price / Gross Sales). \newline 7. Báo cáo thường có dòng tổng cộng ở cuối. \newline 8. Người dùng xem xét và phân tích báo cáo. \\
\hline
Alternative Flow & \textbf{2a. Xem theo Danh mục POS:} \newline    1. Người dùng chọn loại báo cáo "Bán hàng theo Danh mục POS" (Sales by POS Category). \newline    2. Các bước chọn thời gian và xem báo cáo tương tự, nhưng bảng kết quả sẽ nhóm theo từng Danh mục POS, hiển thị tổng số lượng và doanh thu cho mỗi danh mục. \newline \textbf{6a. Sắp xếp/Lọc báo cáo:} \newline    1. Giao diện báo cáo cho phép nhấp vào tiêu đề cột để sắp xếp (ví dụ: sắp xếp theo Số lượng bán giảm dần để xem món bán chạy nhất). \newline    2. Có thể có các bộ lọc bổ sung (ví dụ: lọc theo một POS cụ thể nếu có nhiều điểm bán). \newline \textbf{6b. Xem dạng biểu đồ:} \newline    1. Giao diện báo cáo có thể cung cấp tùy chọn xem dữ liệu dưới dạng biểu đồ (tròn, cột...) để trực quan hóa cơ cấu doanh thu hoặc top sản phẩm bán chạy. \\
\hline
Exception Flow & \textbf{5a. Lỗi truy vấn/tổng hợp dữ liệu:} \newline    1. Hệ thống gặp lỗi khi lấy hoặc tính toán dữ liệu bán hàng. \newline    2. Hệ thống hiển thị thông báo lỗi. \newline \textbf{5b. Không có dữ liệu bán hàng:} \newline    1. Không có giao dịch bán hàng nào trong khoảng thời gian được chọn. \newline    2. Hệ thống hiển thị báo cáo trống hoặc thông báo "Không có dữ liệu". \\
\hline
\multicolumn{2}{|c|}{\textbf{2.3. Thông tin bổ sung (Additional Information)}} \\
\hline
\textbf{Mục} & \textbf{Nội dung} \\
\hline
Business Rule & - \textbf{BR-UC9.4-1:} Báo cáo phải tổng hợp dữ liệu từ tất cả các đơn hàng đã được thanh toán và đóng trong khoảng thời gian và phạm vi (POS) được chọn. \newline - \textbf{BR-UC9.4-2:} Doanh thu hiển thị cần rõ ràng là doanh thu trước thuế (thuần) hay sau thuế, trước hay sau giảm giá, để phục vụ đúng mục đích phân tích. \newline - \textbf{BR-UC9.4-3:} Nếu sản phẩm có biến thể, báo cáo nên có tùy chọn xem chi tiết doanh thu theo từng biến thể cụ thể hoặc tổng hợp theo sản phẩm gốc. \\
\hline
Non-Functional Requirement & - \textbf{NFR-UC9.4-1 (Usability):} Giao diện báo cáo cần dễ sử dụng, cho phép chọn khoảng thời gian linh hoạt, dễ dàng sắp xếp và lọc dữ liệu. Biểu đồ (nếu có) cần rõ ràng. \newline - \textbf{NFR-UC9.4-2 (Performance):} Thời gian tạo báo cáo phải hợp lý, ngay cả khi xử lý lượng lớn dữ liệu giao dịch (ví dụ: báo cáo tháng). Cần tối ưu hóa truy vấn cơ sở dữ liệu. \newline - \textbf{NFR-UC9.4-3 (Accuracy):} Số liệu thống kê (số lượng, doanh thu) phải chính xác 100%. \\
\hline
\end{longtable}

\subsubsection{Use Case UC-MD09-05: Xem Báo cáo Hiệu suất Nhân viên (POS)}

\begin{longtable}{|m{4cm}|p{11cm}|}
\caption{Đặc tả Use Case UC-MD09-05: Xem Báo cáo Hiệu suất Nhân viên (POS)} \label{tab:uc_md09_05} \\
\hline

\endhead % Header cho các trang tiếp theo
\hline
\endfoot % Footer cho bảng
\hline
\endlastfoot % Footer cho trang cuối cùng
\multicolumn{2}{|c|}{\textbf{2.1. Tóm tắt (Summary)}} \\
\hline
\textbf{Mục} & \textbf{Nội dung} \\
\hline
Use Case Name & Xem Báo cáo Hiệu suất Nhân viên (POS) \\
\hline
Use Case ID & UC-MD09-05 \\
\hline
Use Case Description & Cung cấp báo cáo thống kê về hoạt động bán hàng trên POS của từng nhân viên (phục vụ hoặc thu ngân) trong một khoảng thời gian do người dùng lựa chọn, thường bao gồm tổng doanh thu, số lượng đơn hàng đã xử lý, và có thể cả tiền boa nhận được. \\
\hline
Actor & US-01 (Quản lý nhà hàng) \\
\hline
Priority & Should Have \\
\hline
Trigger & Quản lý muốn đánh giá hiệu suất làm việc của từng nhân viên tại điểm bán hàng, theo dõi doanh số hoặc tính toán hoa hồng/thưởng (nếu có). \\
\hline
Pre-Condition & - Người dùng (US-01) đã đăng nhập với quyền truy cập báo cáo POS/Nhân viên. \newline - Dữ liệu giao dịch POS đã được ghi nhận và liên kết đúng với nhân viên thực hiện (ví dụ: người đăng nhập POS khi tạo/thanh toán đơn hàng). \\
\hline
Post-Condition & - Báo cáo thống kê hiệu suất theo từng nhân viên được hiển thị. \newline - Quản lý có thông tin để đánh giá và đưa ra quyết định liên quan đến nhân sự. \\
\hline
\multicolumn{2}{|c|}{\textbf{2.2. Luồng thực thi (Flow)}} \\
\hline
\textbf{Mục} & \textbf{Nội dung} \\
\hline
Basic Flow & 1. Người dùng (US-01) truy cập vào mục Báo cáo (Reporting) của POS. \newline 2. Người dùng chọn loại báo cáo "Doanh thu theo Nhân viên" (Sales by Employee/Salesperson) hoặc tương tự. \newline 3. Hệ thống yêu cầu hoặc cho phép người dùng chọn Khoảng thời gian muốn xem báo cáo. \newline 4. Người dùng chọn khoảng thời gian và nhấn "Xem báo cáo" / "Apply". \newline 5. Hệ thống truy vấn và tổng hợp dữ liệu từ các đơn hàng POS đã đóng trong khoảng thời gian đó, nhóm theo nhân viên đã thực hiện (ví dụ: nhân viên đăng nhập vào POS). \newline 6. Hệ thống hiển thị báo cáo dưới dạng bảng, mỗi dòng là một Nhân viên, với các cột thông tin như: \newline    - Tên Nhân viên. \newline    - Số lượng đơn hàng đã xử lý. \newline    - Tổng doanh thu (trước/sau thuế - tùy cấu hình báo cáo). \newline    - (Tùy chọn) Tổng tiền boa nhận được. \newline 7. Báo cáo có thể có dòng tổng cộng. \newline 8. Người dùng xem xét báo cáo. \\
\hline
Alternative Flow & \textbf{6a. Xem chi tiết đơn hàng của nhân viên:} \newline    1. Từ báo cáo tổng hợp, người dùng có thể nhấp vào tên nhân viên để xem danh sách các đơn hàng cụ thể mà nhân viên đó đã xử lý trong kỳ. \newline \textbf{6b. Lọc theo nhân viên/vai trò:} \newline    1. Báo cáo cho phép lọc để chỉ xem một hoặc một nhóm nhân viên cụ thể. \\
\hline
Exception Flow & Tương tự UC-MD09-04 (Lỗi truy vấn/tổng hợp, Không có dữ liệu). \\
\hline
\multicolumn{2}{|c|}{\textbf{2.3. Thông tin bổ sung (Additional Information)}} \\
\hline
\textbf{Mục} & \textbf{Nội dung} \\
\hline
Business Rule & - \textbf{BR-UC9.5-1:} Dữ liệu báo cáo phải được tổng hợp dựa trên việc ghi nhận chính xác nhân viên nào đã xử lý đơn hàng nào trên POS (thường là người đăng nhập vào phiên hoặc người được gán cho đơn hàng). \newline - \textbf{BR-UC9.5-2:} Các chỉ số hiệu suất (doanh thu, số đơn) cần được định nghĩa rõ ràng (tính trước hay sau thuế, có bao gồm đơn hủy không...). \\
\hline
Non-Functional Requirement & - \textbf{NFR-UC9.5-1 (Usability):} Báo cáo dễ đọc, dễ so sánh hiệu suất giữa các nhân viên. \newline - \textbf{NFR-UC9.5-2 (Performance):} Thời gian tạo báo cáo phải hợp lý. \newline - \textbf{NFR-UC9.5-3 (Accuracy):} Số liệu phải chính xác. \newline - \textbf{NFR-UC9.5-4 (Security):} Thông tin hiệu suất nhân viên có thể nhạy cảm, cần kiểm soát quyền truy cập báo cáo này. \\
\hline
\end{longtable}

\subsubsection{Use Case UC-MD09-06: Xem Báo cáo Tiền đặt cọc}

\begin{longtable}{|m{4cm}|p{11cm}|}
\caption{Đặc tả Use Case UC-MD09-06: Xem Báo cáo Tiền đặt cọc} \label{tab:uc_md09_06} \\
\hline

\endhead % Header cho các trang tiếp theo
\hline
\endfoot % Footer cho bảng
\hline
\endlastfoot % Footer cho trang cuối cùng
\multicolumn{2}{|c|}{\textbf{2.1. Tóm tắt (Summary)}} \\
\hline
\textbf{Mục} & \textbf{Nội dung} \\
\hline
Use Case Name & Xem Báo cáo Tiền đặt cọc \\
\hline
Use Case ID & UC-MD09-06 \\
\hline
Use Case Description & Cung cấp báo cáo tổng hợp về tình hình thu và sử dụng tiền đặt cọc từ các lượt đặt chỗ trong một khoảng thời gian, bao gồm tổng tiền cọc đã thu, tổng tiền cọc đã được áp dụng (trừ vào hóa đơn thanh toán), và tổng tiền cọc bị mất (do khách hủy và không được hoàn). \\
\hline
Actor & US-01 (Quản lý nhà hàng), US-06 (Kế toán) \\
\hline
Priority & Must Have \\
\hline
Trigger & Cần theo dõi dòng tiền đặt cọc, đối soát doanh thu từ cọc, hoặc phân tích tỷ lệ khách hủy đặt chỗ sau khi đã đặt cọc. \\
\hline
Pre-Condition & - Người dùng (US-01 hoặc US-06) đã đăng nhập với quyền truy cập báo cáo Đặt chỗ/Tài chính. \newline - Hệ thống có chức năng đặt chỗ với yêu cầu đặt cọc và ghi nhận trạng thái thanh toán cọc (MD-03). \newline - Có dữ liệu về các lượt đặt chỗ đã thanh toán cọc, đã được sử dụng hoặc đã bị hủy. \\
\hline
Post-Condition & - Báo cáo tổng hợp về tình hình tiền đặt cọc được hiển thị. \newline - Người dùng có thông tin để quản lý và đối soát dòng tiền này. \\
\hline
\multicolumn{2}{|c|}{\textbf{2.2. Luồng thực thi (Flow)}} \\
\hline
\textbf{Mục} & \textbf{Nội dung} \\
\hline
Basic Flow & 1. Người dùng (US-01/US-06) truy cập vào khu vực Báo cáo của module Đặt chỗ hoặc Kế toán. \newline 2. Người dùng chọn loại báo cáo "Tiền đặt cọc" (Deposits Report) hoặc tương tự. \newline 3. Hệ thống yêu cầu hoặc cho phép chọn Khoảng thời gian báo cáo. \newline 4. Người dùng chọn khoảng thời gian và nhấn "Xem báo cáo". \newline 5. Hệ thống truy vấn dữ liệu từ các bản ghi Đặt chỗ và các giao dịch thanh toán/áp dụng cọc liên quan trong khoảng thời gian đó. \newline 6. Hệ thống hiển thị báo cáo tổng hợp, bao gồm các số liệu chính: \newline    - Tổng số tiền đặt cọc đã thu (Total Deposits Received). \newline    - Tổng số tiền đặt cọc đã áp dụng vào hóa đơn (Total Deposits Applied). \newline    - Tổng số tiền đặt cọc bị mất/không hoàn lại (Total Forfeited Deposits - ví dụ từ các đặt chỗ bị hủy không hoàn cọc). \newline    - (Tùy chọn) Số dư tiền đặt cọc chưa sử dụng (nếu có trường hợp này). \newline    - (Tùy chọn) Danh sách chi tiết các giao dịch đặt cọc trong kỳ. \newline 7. Người dùng xem xét báo cáo. \\
\hline
Alternative Flow & \textbf{6a. Phân tích chi tiết hơn:} \newline    1. Báo cáo có thể cho phép xem chi tiết các lượt đặt chỗ ứng với từng loại giao dịch cọc (đã thu, đã áp dụng, đã mất). \\
\hline
Exception Flow & Tương tự UC-MD09-04 (Lỗi truy vấn/tổng hợp, Không có dữ liệu). \\
\hline
\multicolumn{2}{|c|}{\textbf{2.3. Thông tin bổ sung (Additional Information)}} \\
\hline
\textbf{Mục} & \textbf{Nội dung} \\
\hline
Business Rule & - \textbf{BR-UC9.6-1:} Báo cáo phải phân biệt rõ ràng giữa các trạng thái của tiền đặt cọc (đã thu, đã áp dụng, đã mất). \newline - \textbf{BR-UC9.6-2:} Dữ liệu phải được tổng hợp chính xác từ các bản ghi đặt chỗ và trạng thái thanh toán/hủy tương ứng. \newline - \textbf{BR-UC9.6-3:} Logic xác định tiền cọc bị mất phải dựa trên trạng thái hủy đặt chỗ và chính sách hoàn cọc của nhà hàng. \\
\hline
Non-Functional Requirement & - \textbf{NFR-UC9.6-1 (Accuracy):} Số liệu báo cáo tiền đặt cọc phải tuyệt đối chính xác để phục vụ đối soát tài chính. \newline - \textbf{NFR-UC9.6-2 (Usability):} Báo cáo cần trình bày các số liệu một cách rõ ràng, dễ hiểu. \newline - \textbf{NFR-UC9.6-3 (Performance):} Tốc độ tạo báo cáo cần chấp nhận được. \\
\hline
\end{longtable}

\subsubsection{Use Case UC-MD09-07: Xem Báo cáo Doanh thu theo Loại hình (Eat-in, Takeout, Delivery)}

\begin{longtable}{|m{4cm}|p{11cm}|}
\caption{Đặc tả Use Case UC-MD09-07: Xem Báo cáo Doanh thu theo Loại hình (Eat-in, Takeout, Delivery)} \label{tab:uc_md09_07} \\
\hline

\endhead % Header cho các trang tiếp theo
\hline
\endfoot % Footer cho bảng
\hline
\endlastfoot % Footer cho trang cuối cùng
\multicolumn{2}{|c|}{\textbf{2.1. Tóm tắt (Summary)}} \\
\hline
\textbf{Mục} & \textbf{Nội dung} \\
\hline
Use Case Name & Xem Báo cáo Doanh thu theo Loại hình (Eat-in, Takeout, Delivery) \\
\hline
Use Case ID & UC-MD09-07 \\
\hline
Use Case Description & Cung cấp báo cáo phân tích tổng doanh thu (và có thể cả số lượng đơn hàng) được tạo ra từ mỗi loại hình phục vụ khác nhau mà nhà hàng hỗ trợ: Ăn tại chỗ (Eat-in), Mang về (Takeout), và Giao hàng (Delivery) trong một khoảng thời gian. \\
\hline
Actor & US-01 (Quản lý nhà hàng), US-06 (Kế toán) \\
\hline
Priority & Must Have \\
\hline
Trigger & Cần phân tích hiệu quả kinh doanh và đóng góp doanh thu của từng kênh/loại hình phục vụ. \\
\hline
Pre-Condition & - Người dùng (US-01 hoặc US-06) đã đăng nhập với quyền truy cập báo cáo POS/Bán hàng. \newline - Hệ thống POS có khả năng phân loại và ghi nhận loại hình cho mỗi đơn hàng (Eat-in, Takeout, Delivery - từ MD-05, MD-06, MD-07). \newline - Đã có dữ liệu giao dịch từ các loại hình đơn hàng khác nhau. \\
\hline
Post-Condition & - Báo cáo phân tích doanh thu theo từng loại hình được hiển thị. \newline - Người dùng có thông tin để so sánh hiệu quả giữa các kênh bán hàng. \\
\hline
\multicolumn{2}{|c|}{\textbf{2.2. Luồng thực thi (Flow)}} \\
\hline
\textbf{Mục} & \textbf{Nội dung} \\
\hline
Basic Flow & 1. Người dùng (US-01/US-06) truy cập vào mục Báo cáo (Reporting) của POS hoặc Sales. \newline 2. Người dùng chọn loại báo cáo "Doanh thu theo Loại hình" (Sales by Order Type / Channel) hoặc tương tự. \newline 3. Hệ thống yêu cầu hoặc cho phép chọn Khoảng thời gian báo cáo. \newline 4. Người dùng chọn khoảng thời gian và nhấn "Xem báo cáo". \newline 5. Hệ thống truy vấn dữ liệu từ các đơn hàng POS đã đóng trong khoảng thời gian đó, nhóm theo trường "Loại hình đơn hàng" (Order Type). \newline 6. Hệ thống hiển thị báo cáo, thường dưới dạng bảng hoặc biểu đồ, thể hiện: \newline    - Loại hình (Eat-in, Takeout, Delivery). \newline    - Tổng Doanh thu (trước/sau thuế) cho từng loại hình. \newline    - (Tùy chọn) Số lượng đơn hàng cho từng loại hình. \newline    - (Tùy chọn) Tỷ trọng đóng góp doanh thu của từng loại hình. \newline 7. Người dùng xem xét báo cáo. \\
\hline
Alternative Flow & \textbf{6a. Xem chi tiết hơn:} \newline    1. Báo cáo có thể cho phép nhấp vào một loại hình để xem chi tiết hơn (ví dụ: xem danh sách sản phẩm bán chạy nhất của kênh Delivery). \\
\hline
Exception Flow & Tương tự UC-MD09-04 (Lỗi truy vấn/tổng hợp, Không có dữ liệu). \\
\hline
\multicolumn{2}{|c|}{\textbf{2.3. Thông tin bổ sung (Additional Information)}} \\
\hline
\textbf{Mục} & \textbf{Nội dung} \\
\hline
Business Rule & - \textbf{BR-UC9.7-1:} Hệ thống phải có khả năng ghi nhận và phân loại chính xác loại hình (Eat-in, Takeout, Delivery) cho mỗi đơn hàng POS. \newline - \textbf{BR-UC9.7-2:} Báo cáo phải tổng hợp đúng doanh thu và số lượng đơn cho từng loại hình dựa trên dữ liệu đã ghi nhận. \\
\hline
Non-Functional Requirement & - \textbf{NFR-UC9.7-1 (Usability):} Báo cáo cần trình bày dữ liệu so sánh giữa các loại hình một cách trực quan (biểu đồ tròn/cột thường hữu ích). \newline - \textbf{NFR-UC9.7-2 (Performance):} Việc tạo báo cáo phân tích theo loại hình cần có hiệu năng tốt. \newline - \textbf{NFR-UC9.7-3 (Accuracy):} Số liệu phân loại phải chính xác. \\
\hline
\end{longtable}

\subsubsection{Use Case UC-MD09-08: Xuất dữ liệu Báo cáo}

\begin{longtable}{|m{4cm}|p{11cm}|}
\caption{Đặc tả Use Case UC-MD09-08: Xuất dữ liệu Báo cáo} \label{tab:uc_md09_08} \\
\hline

\endhead % Header cho các trang tiếp theo
\hline
\endfoot % Footer cho bảng
\hline
\endlastfoot % Footer cho trang cuối cùng
\multicolumn{2}{|c|}{\textbf{2.1. Tóm tắt (Summary)}} \\
\hline
\textbf{Mục} & \textbf{Nội dung} \\
\hline
Use Case Name & Xuất dữ liệu Báo cáo \\
\hline
Use Case ID & UC-MD09-08 \\
\hline
Use Case Description & Cho phép Người dùng (Quản lý, Kế toán) đang xem một báo cáo trong hệ thống Odoo (ví dụ: báo cáo doanh thu phiên, báo cáo bán hàng theo sản phẩm...) xuất (export) dữ liệu của báo cáo đó ra một tệp tin theo định dạng phổ biến như Excel (.xlsx) hoặc CSV (.csv). \\
\hline
Actor & US-01 (Quản lý nhà hàng), US-06 (Kế toán) \\
\hline
Priority & Should Have \\
\hline
Trigger & Người dùng muốn lưu trữ dữ liệu báo cáo, chia sẻ với người khác, hoặc thực hiện các phân tích phức tạp hơn bằng công cụ bên ngoài (như Microsoft Excel). \\
\hline
Pre-Condition & - Người dùng đang xem một giao diện báo cáo hoặc danh sách dữ liệu trong Odoo (ví dụ: kết quả của UC-MD09-03, UC-MD09-04...). \newline - Giao diện đó có chức năng/nút "Xuất" (Export). \\
\hline
Post-Condition & - Một tệp tin chứa dữ liệu của báo cáo được tạo ra và tải về máy tính của người dùng theo định dạng đã chọn (Excel/CSV). \\
\hline
\multicolumn{2}{|c|}{\textbf{2.2. Luồng thực thi (Flow)}} \\
\hline
\textbf{Mục} & \textbf{Nội dung} \\
\hline
Basic Flow & 1. Người dùng (US-01/US-06) đang xem một báo cáo hoặc danh sách dữ liệu muốn xuất. \newline 2. Người dùng tìm và nhấp vào nút/liên kết "Xuất" (Export) hoặc biểu tượng tương ứng. \newline 3. Hệ thống (có thể) hiển thị một hộp thoại/tùy chọn cho phép: \newline    - Chọn các trường dữ liệu muốn xuất (mặc định là các cột đang hiển thị). \newline    - Chọn định dạng tệp xuất (Excel hoặc CSV). \newline    - (Tùy chọn) Đặt tên tệp. \newline 4. Người dùng lựa chọn các tùy chọn mong muốn (hoặc giữ mặc định) và nhấn nút "Xuất" / "Export". \newline 5. Hệ thống xử lý yêu cầu, truy xuất dữ liệu tương ứng từ cơ sở dữ liệu. \newline 6. Hệ thống tạo ra tệp tin theo định dạng đã chọn (Excel/CSV) chứa dữ liệu đó. \newline 7. Hệ thống kích hoạt trình duyệt của người dùng để tải tệp tin về máy. \newline 8. Người dùng lưu tệp tin vào vị trí mong muốn trên máy tính. \\
\hline
Alternative Flow & \textbf{3a. Xuất nhanh với định dạng mặc định:} \newline    1. Nút "Xuất" có thể xuất trực tiếp ra định dạng mặc định (ví dụ: Excel) mà không cần qua bước chọn tùy chọn. \\
\hline
Exception Flow & \textbf{5a. Lỗi truy xuất dữ liệu / tạo tệp:} \newline    1. Hệ thống gặp lỗi khi lấy dữ liệu hoặc khi tạo tệp tin xuất ra (ví dụ: dữ liệu quá lớn, lỗi định dạng, lỗi ghi tệp). \newline    2. Hệ thống hiển thị thông báo lỗi cho người dùng. \newline    3. Tệp tin không được tạo ra hoặc bị lỗi. \\
\hline
\multicolumn{2}{|c|}{\textbf{2.3. Thông tin bổ sung (Additional Information)}} \\
\hline
\textbf{Mục} & \textbf{Nội dung} \\
\hline
Business Rule & - \textbf{BR-UC9.8-1:} Hệ thống phải hỗ trợ xuất dữ liệu ra ít nhất hai định dạng phổ biến là Excel (.xlsx) và CSV (.csv). \newline - \textbf{BR-UC9.8-2:} Dữ liệu trong tệp xuất ra phải khớp với dữ liệu đang hiển thị trên báo cáo/danh sách tại thời điểm xuất. \newline - \textbf{BR-UC9.8-3:} Cấu trúc cột trong tệp Excel/CSV nên tương ứng với các cột đang hiển thị trên giao diện Odoo. \\
\hline
Non-Functional Requirement & - \textbf{NFR-UC9.8-1 (Usability):} Chức năng xuất dữ liệu phải dễ dàng tìm thấy và sử dụng. \newline - \textbf{NFR-UC9.8-2 (Performance):} Thời gian tạo và tải tệp xuất phải hợp lý, tùy thuộc vào khối lượng dữ liệu. Đối với dữ liệu rất lớn, có thể cần cơ chế xử lý nền (background job). \newline - \textbf{NFR-UC9.8-3 (Compatibility):} Tệp Excel/CSV xuất ra phải tương thích và mở được bằng các phần mềm bảng tính phổ biến (Microsoft Excel, Google Sheets, LibreOffice Calc). \newline - \textbf{NFR-UC9.8-4 (Security):} Quyền xuất dữ liệu từ các báo cáo nhạy cảm cần được kiểm soát. \\
\hline
\end{longtable}


\subsection{Module MD-10: Quản lý Hệ thống \& Người dùng}

\subsubsection{Use Case UC-MD10-01: Tạo mới Tài khoản Người dùng (Nhân viên)}
\begin{longtable}{|m{4cm}|p{11cm}|}
\caption{Đặc tả Use Case UC-MD10-01: Tạo mới Tài khoản Người dùng (Nhân viên)} \label{tab:uc_md10_01_full_v2_latex_fixed_in_codeblock} \\
\hline
\multicolumn{2}{|c|}{\textbf{2.1. Tóm tắt (Summary)}} \\
\hline
\textbf{Mục} & \textbf{Nội dung} \\
\hline
\endhead % Header cho các trang tiếp theo
\midrule
\endfoot % Footer cho bảng
\bottomrule
\endlastfoot % Footer cho trang cuối cùng
Use Case Name & Tạo mới Tài khoản Người dùng (Nhân viên) \\
\hline
Use Case ID & UC-MD10-01 \\
\hline
Use Case Description & Cho phép Quản trị viên hệ thống (US-10) tạo một tài khoản đăng nhập mới trong hệ thống cho một nhân viên của nhà hàng, bao gồm việc nhập thông tin cơ bản và chuẩn bị cho việc gán các quyền truy cập ban đầu. \\
\hline
Actor & US-10 (Quản trị viên Hệ thống) \\
\hline
Priority & Must Have \\
\hline
Trigger & - Có nhân viên mới gia nhập nhà hàng và cần được cấp tài khoản để truy cập và sử dụng hệ thống theo vai trò công việc. \\
\hline
Pre-Condition & - Người dùng US-10 đã đăng nhập vào hệ thống với quyền quản trị người dùng (thường là quyền "Administration / Settings" hoặc tương đương). \\
\hline
Post-Condition & - Một tài khoản người dùng mới cho nhân viên được tạo thành công trong cơ sở dữ liệu của hệ thống. \newline - Tài khoản này được liên kết với các thông tin cơ bản do US-10 nhập (Tên, Địa chỉ Email đăng nhập). \newline - Tài khoản người dùng mới này sẵn sàng để được US-10 gán các Nhóm Quyền truy cập cụ thể (thông qua UC-MD10-04) nhằm xác định phạm vi hoạt động của nhân viên đó trong hệ thống. \newline - Nhân viên có thể nhận được thông tin đăng nhập ban đầu (ví dụ: qua email mời được hệ thống tự động gửi hoặc do US-10 cung cấp trực tiếp) để bắt đầu sử dụng hệ thống. \\
\hline
\multicolumn{2}{|c|}{\textbf{2.2. Luồng thực thi (Flow)}} \\
\hline
\textbf{Mục} & \textbf{Nội dung} \\
\hline
Basic Flow & 1. US-10 truy cập vào mục "Cài đặt" (Settings) trên giao diện chính của hệ thống. \newline 2. US-10 điều hướng đến khu vực "Quản lý Người dùng \& Công ty" (Users \& Companies) và chọn mục "Người dùng" (Users). \newline 3. Hệ thống hiển thị danh sách các tài khoản người dùng hiện có trong hệ thống (tham chiếu UC-MD10-02). \newline 4. US-10 chọn hành động "Tạo mới" (Create) trên giao diện danh sách người dùng. \newline 5. Hệ thống hiển thị một form trống để US-10 nhập thông tin cho người dùng mới. \newline 6. US-10 nhập Tên đầy đủ của người dùng (ví dụ: "Nguyễn Văn B") vào trường "Tên" (Name) (Bắt buộc). \newline 7. US-10 nhập Địa chỉ Email sẽ được sử dụng làm tên đăng nhập (Login / Email Address) cho người dùng này (Bắt buộc, và địa chỉ email này phải là duy nhất trong toàn hệ thống - BR-UC10.1-1). \newline 8. (Tùy chọn) US-10 có thể chọn liên kết tài khoản người dùng này với một bản ghi "Nhân viên" (Employee record) đã tồn tại trong module Quản lý Nhân sự (HR) (nếu module này được sử dụng và hồ sơ nhân viên đã được tạo trước). \newline 9. (Tùy chọn) US-10 có thể cấu hình các tùy chọn khác trên form người dùng như Ngôn ngữ ưu tiên, Múi giờ, Ảnh đại diện, hoặc các thông tin liên hệ khác. \newline 10. (Quan trọng) US-10 sẽ cần thực hiện việc gán các Nhóm Quyền truy cập phù hợp cho người dùng này sau khi lưu hoặc trong cùng bước này nếu giao diện cho phép (hành động gán quyền chi tiết được mô tả ở UC-MD10-04). Ví dụ, nếu đây là nhân viên thu ngân, US-10 cần chuẩn bị để gán quyền "Point of Sale / User". \newline 11. (Tùy chọn) US-10 chọn phương thức thiết lập mật khẩu ban đầu cho người dùng mới: \newline     a. Chọn tùy chọn "Gửi email mời" (Send password reset instructions / Send invitation email): Hệ thống sẽ tự động gửi một email đến địa chỉ email đã nhập ở bước 7, chứa một liên kết cho phép người dùng tự đặt mật khẩu cho lần đăng nhập đầu tiên. \newline     b. Hoặc, US-10 có thể bỏ qua việc gửi email mời và sẽ đặt mật khẩu thủ công cho người dùng sau khi bản ghi người dùng được lưu (sử dụng chức năng "Đặt lại mật khẩu" - UC-MD10-07). \newline 12. US-10 chọn hành động "Lưu" (Save) trên form tạo người dùng. \newline 13. Hệ thống kiểm tra tính hợp lệ của các thông tin đã nhập, đặc biệt là tính duy nhất của Địa chỉ Email đăng nhập và việc các trường bắt buộc đã được điền. \newline 14. Nếu tất cả thông tin hợp lệ, hệ thống tạo một bản ghi người dùng mới trong cơ sở dữ liệu. Theo mặc định, tài khoản người dùng mới này thường ở trạng thái hoạt động (Active). \newline 15. Hệ thống hiển thị thông báo "Người dùng <Tên người dùng> đã được tạo thành công." và có thể chuyển sang giao diện chi tiết của người dùng vừa tạo hoặc quay lại danh sách người dùng. \\
\hline
Alternative Flow & \textbf{11c. Không thực hiện hành động thiết lập mật khẩu ngay:} \newline    1. US-10 có thể chọn lưu người dùng mà không gửi email mời hoặc đặt mật khẩu ngay. Trong trường hợp này, tài khoản được tạo nhưng người dùng chưa thể đăng nhập cho đến khi mật khẩu được US-10 thiết lập sau đó (UC-MD10-07). \\
\hline
Exception Flow & \textbf{13a. Lỗi xác thực dữ liệu khi lưu:} \newline    1. Hệ thống phát hiện Địa chỉ Email đăng nhập mà US-10 nhập đã được sử dụng bởi một tài khoản người dùng khác trong hệ thống, hoặc một trường bắt buộc (như Tên người dùng) bị bỏ trống. \newline    2. Hệ thống hiển thị một thông báo lỗi cụ thể cho US-10, chỉ rõ trường thông tin không hợp lệ (ví dụ: "Địa chỉ email này đã được sử dụng. Vui lòng chọn một địa chỉ email khác." hoặc "Trường Tên không được để trống."). \newline    3. Hệ thống không cho phép lưu bản ghi người dùng. US-10 cần phải sửa lại các thông tin không hợp lệ. Use Case quay lại bước 6 hoặc 7. \newline \textbf{14a. Lỗi hệ thống trong quá trình tạo người dùng:} \newline    1. Hệ thống gặp phải một lỗi kỹ thuật không mong muốn (ví dụ: lỗi kết nối đến cơ sở dữ liệu, lỗi logic nội bộ) khi cố gắng thực hiện hành động lưu bản ghi người dùng mới. \newline    2. Hệ thống hiển thị một thông báo lỗi chung cho US-10 (ví dụ: "Không thể tạo người dùng. Đã xảy ra lỗi hệ thống. Vui lòng thử lại sau."). \newline    3. Tài khoản người dùng có thể chưa được tạo thành công hoặc được tạo không đầy đủ. US-10 có thể cần thử lại hoặc liên hệ bộ phận hỗ trợ kỹ thuật. \\
\hline
\multicolumn{2}{|c|}{\textbf{2.3. Thông tin bổ sung (Additional Information)}} \\
\hline
\textbf{Mục} & \textbf{Nội dung} \\
\hline
Business Rule & - \textbf{BR-UC10.1-1:} Địa chỉ Email được sử dụng để đăng nhập của mỗi người dùng phải là duy nhất trên toàn bộ hệ thống để đảm bảo tính định danh. \newline - \textbf{BR-UC10.1-2:} Khi tạo người dùng mới, việc gán các Nhóm Quyền truy cập phù hợp (thông qua UC-MD10-04) là bước cực kỳ quan trọng để đảm bảo người dùng chỉ có thể truy cập những chức năng và dữ liệu mà họ được phép theo vai trò công việc của mình. \newline - \textbf{BR-UC10.1-3:} Mật khẩu ban đầu cho người dùng mới (nếu được Quản trị viên đặt trực tiếp) nên tuân thủ các chính sách bảo mật của nhà hàng (ví dụ: độ dài tối thiểu, yêu cầu ký tự đặc biệt) và người dùng nên được khuyến khích hoặc bắt buộc thay đổi mật khẩu này ngay sau lần đăng nhập đầu tiên. \\
\hline
Non-Functional Requirement & - \textbf{NFR-UC10.1-1 (Usability):} Giao diện (form) tạo người dùng mới phải được thiết kế rõ ràng, các trường thông tin bắt buộc phải dễ nhận biết. Các tùy chọn liên quan đến thiết lập mật khẩu và gán quyền ban đầu cần trực quan. \newline - \textbf{NFR-UC10.1-2 (Security):} Quy trình tạo tài khoản người dùng và thiết lập mật khẩu ban đầu phải được thực hiện một cách an toàn, đảm bảo thông tin đăng nhập không bị lộ. \newline - \textbf{NFR-UC10.1-3 (Performance):} Thời gian hệ thống cần để tạo và lưu một bản ghi người dùng mới phải nhanh chóng, không gây chờ đợi lâu cho Quản trị viên. \\
\hline
\end{longtable}

\subsubsection{Use Case UC-MD10-02: Xem Danh sách Người dùng}
\begin{longtable}{|m{4cm}|p{11cm}|}
\caption{Đặc tả Use Case UC-MD10-02: Xem Danh sách Người dùng} \label{tab:uc_md10_02_full_v2_latex_fixed_in_codeblock} \\
\hline
\multicolumn{2}{|c|}{\textbf{2.1. Tóm tắt (Summary)}} \\
\hline
\textbf{Mục} & \textbf{Nội dung} \\
\hline
\endhead % Header cho các trang tiếp theo
\midrule
\endfoot % Footer cho bảng
\bottomrule
\endlastfoot % Footer cho trang cuối cùng
Use Case Name & Xem Danh sách Người dùng \\
\hline
Use Case ID & UC-MD10-02 \\
\hline
Use Case Description & Cho phép Quản trị viên hệ thống (US-10) xem danh sách tất cả các tài khoản người dùng (bao gồm cả nhân viên và có thể cả các loại người dùng khác nếu có) đã được tạo trong hệ thống, với các thông tin tóm tắt của mỗi tài khoản. \\
\hline
Actor & US-10 (Quản trị viên Hệ thống) \\
\hline
Priority & Must Have \\
\hline
Trigger & Quản trị viên cần kiểm tra danh sách các người dùng hiện có trong hệ thống, tìm kiếm một người dùng cụ thể, hoặc chuẩn bị cho các thao tác quản lý khác như sửa đổi thông tin, phân quyền, hoặc vô hiệu hóa tài khoản. \\
\hline
Pre-Condition & - US-10 đã đăng nhập vào hệ thống với quyền quản trị người dùng. \\
\hline
Post-Condition & - Danh sách các tài khoản người dùng (theo bộ lọc mặc định, thường là các tài khoản đang hoạt động) được hiển thị trên giao diện cho US-10. \newline - US-10 có thể xem các thông tin cơ bản của từng người dùng trong danh sách và có thể chọn một người dùng cụ thể để xem chi tiết hoặc thực hiện các hành động quản lý khác. \\
\hline
\multicolumn{2}{|c|}{\textbf{2.2. Luồng thực thi (Flow)}} \\
\hline
\textbf{Mục} & \textbf{Nội dung} \\
\hline
Basic Flow & 1. US-10 truy cập vào mục "Cài đặt" (Settings) trên giao diện chính. \newline 2. US-10 điều hướng đến khu vực "Quản lý Người dùng \& Công ty" (Users \& Companies) và chọn mục "Người dùng" (Users). \newline 3. Hệ thống truy vấn cơ sở dữ liệu và hiển thị danh sách các tài khoản người dùng hiện có. Theo mặc định, danh sách này thường chỉ bao gồm các người dùng đang ở trạng thái hoạt động (Active=True). \newline 4. Với mỗi người dùng trong danh sách (thường được hiển thị ở dạng List View hoặc Kanban View), hệ thống hiển thị các thông tin tóm tắt cơ bản như: \newline    - Tên người dùng (Name). \newline    - Địa chỉ Email đăng nhập (Login / Email Address). \newline    - (Tùy chọn) Ảnh đại diện (Avatar/Image) nếu có. \newline    - (Tùy chọn) Tên Công ty mà người dùng thuộc về (nếu hệ thống được cấu hình để quản lý đa công ty). \newline    - (Tùy chọn) Thời gian đăng nhập lần cuối (Last Login). \newline 5. US-10 xem xét danh sách các người dùng. \\
\hline
Alternative Flow & \textbf{3a. Tìm kiếm người dùng trong danh sách:} \newline    1. Giao diện danh sách người dùng cung cấp một ô tìm kiếm. \newline    2. US-10 nhập tên, địa chỉ email, hoặc một phần thông tin của người dùng cần tìm vào ô tìm kiếm. \newline    3. US-10 nhấn Enter hoặc hệ thống tự động lọc khi nhập. \newline    4. Hệ thống hiển thị danh sách các người dùng khớp với từ khóa tìm kiếm. \newline \textbf{3b. Lọc danh sách người dùng theo các tiêu chí:} \newline    1. Giao diện danh sách người dùng cung cấp các bộ lọc (Filters) có sẵn hoặc tùy chỉnh. \newline    2. US-10 có thể chọn các bộ lọc như: \newline       - Lọc theo Trạng thái tài khoản (ví dụ: "Hoạt động" - Active, "Đã lưu trữ/Vô hiệu hóa" - Archived). \newline       - Lọc theo một hoặc nhiều Nhóm Quyền truy cập (ví dụ: hiển thị tất cả người dùng thuộc nhóm "Point of Sale / User"). \newline       - (Nếu có) Lọc theo Công ty (trong môi trường đa công ty). \newline    3. Hệ thống áp dụng các bộ lọc đã chọn và hiển thị lại danh sách kết quả. \newline \textbf{3c. Sắp xếp danh sách người dùng:} \newline    1. US-10 nhấp vào tiêu đề của các cột trong danh sách (ví dụ: cột "Tên người dùng", cột "Email") để sắp xếp danh sách theo thứ tự tăng dần hoặc giảm dần theo nội dung của cột đó. \\
\hline
Exception Flow & \textbf{3d. Lỗi hệ thống khi tải danh sách người dùng:} \newline    1. Hệ thống gặp lỗi kỹ thuật (ví dụ: lỗi truy vấn cơ sở dữ liệu, lỗi kết nối máy chủ) khi cố gắng tải danh sách người dùng. \newline    2. Hệ thống hiển thị một thông báo lỗi chung cho US-10. \newline    3. US-10 không thể xem được danh sách người dùng. Use Case kết thúc không thành công. \newline \textbf{3e. Không có người dùng nào (ngoại trừ tài khoản admin ban đầu):} \newline    1. Nếu hệ thống vừa được cài đặt và chưa có bất kỳ tài khoản người dùng nào khác được tạo (ngoài tài khoản quản trị viên mặc định). \newline    2. Hệ thống hiển thị danh sách chỉ chứa một hoặc một vài tài khoản quản trị viên mặc định, hoặc có thể hiển thị thông báo "Chưa có người dùng nào khác được tạo." \\
\hline
\multicolumn{2}{|c|}{\textbf{2.3. Thông tin bổ sung (Additional Information)}} \\
\hline
\textbf{Mục} & \textbf{Nội dung} \\
\hline
Business Rule & - \textbf{BR-UC10.2-1 (V2):} Theo mặc định, giao diện danh sách người dùng nên chỉ hiển thị các tài khoản đang ở trạng thái hoạt động (Active=True). Quản trị viên phải có khả năng dễ dàng sử dụng bộ lọc để xem cả các tài khoản đã bị vô hiệu hóa (Archived/Active=False). \\
\hline
Non-Functional Requirement & - \textbf{NFR-UC10.2-1 (V2 - Usability):} Giao diện hiển thị danh sách người dùng phải rõ ràng, dễ đọc và dễ dàng điều hướng. Các chức năng tìm kiếm, lọc, và sắp xếp phải hoạt động hiệu quả và trực quan. \newline - \textbf{NFR-UC10.2-2 (V2 - Performance):} Thời gian hệ thống cần để tải và hiển thị danh sách người dùng (ngay cả khi có số lượng lớn tài khoản trong hệ thống) phải nhanh chóng, không gây chờ đợi cho Quản trị viên. \\
\hline
\end{longtable}

\subsubsection{Use Case UC-MD10-03: Sửa Thông tin Tài khoản Người dùng}
\begin{longtable}{|m{4cm}|p{11cm}|}
\caption{Đặc tả Use Case UC-MD10-03: Sửa Thông tin Tài khoản Người dùng} \label{tab:uc_md10_03_full_v2_latex_fixed_in_codeblock} \\
\hline
\multicolumn{2}{|c|}{\textbf{2.1. Tóm tắt (Summary)}} \\
\hline
\textbf{Mục} & \textbf{Nội dung} \\
\hline
\endhead % Header cho các trang tiếp theo
\midrule
\endfoot % Footer cho bảng
\bottomrule
\endlastfoot % Footer cho trang cuối cùng
Use Case Name & Sửa Thông tin Tài khoản Người dùng \\
\hline
Use Case ID & UC-MD10-03 \\
\hline
Use Case Description & Cho phép Quản trị viên hệ thống (US-10) cập nhật các thông tin liên quan đến một tài khoản người dùng đã tồn tại trong hệ thống, ví dụ như thay đổi tên hiển thị, địa chỉ email đăng nhập (nếu được phép và đảm bảo tính duy nhất), ảnh đại diện, ngôn ngữ ưu tiên, múi giờ, hoặc các thông tin liên hệ khác. (Việc thay đổi nhóm quyền được xử lý riêng ở UC-MD10-04). \\
\hline
Actor & US-10 (Quản trị viên Hệ thống) \\
\hline
Priority & Must Have \\
\hline
Trigger & - Thông tin cá nhân hoặc thông tin liên hệ của một nhân viên thay đổi (ví dụ: nhân viên đổi họ sau khi kết hôn, cập nhật địa chỉ email hoặc số điện thoại mới). \newline - Cần sửa lại các lỗi nhập liệu đã xảy ra khi tạo tài khoản người dùng ban đầu. \newline - Cần cập nhật ảnh đại diện mới cho người dùng hoặc thay đổi các cài đặt ưu tiên cá nhân của họ. \\
\hline
Pre-Condition & - US-10 đã đăng nhập vào hệ thống với quyền quản trị người dùng. \newline - Tài khoản Người dùng cần được sửa thông tin đã tồn tại trong hệ thống (đã được tạo qua UC-MD10-01). \\
\hline
Post-Condition & - Các thông tin của tài khoản Người dùng được chọn đã được cập nhật thành công và lưu lại trong cơ sở dữ liệu của hệ thống. \newline - Nếu thông tin Địa chỉ Email đăng nhập bị thay đổi, người dùng đó sẽ cần phải sử dụng địa chỉ email mới cho các lần đăng nhập tiếp theo. \newline - Các thay đổi về Ngôn ngữ hoặc Múi giờ sẽ ảnh hưởng đến trải nghiệm hiển thị của người dùng đó khi họ sử dụng hệ thống. \\
\hline
\multicolumn{2}{|c|}{\textbf{2.2. Luồng thực thi (Flow)}} \\
\hline
\textbf{Mục} & \textbf{Nội dung} \\
\hline
Basic Flow & 1. US-10 truy cập vào danh sách Người dùng hiện có trong hệ thống (UC-MD10-02). \newline 2. US-10 tìm kiếm (nếu cần) và chọn (nhấp vào) tài khoản Người dùng cụ thể mà mình muốn sửa thông tin. \newline 3. Hệ thống hiển thị form chi tiết thông tin của Người dùng đã chọn (thường ở chế độ chỉ xem - read-only). \newline 4. US-10 chọn hành động "Sửa" (Edit) trên form chi tiết Người dùng. \newline 5. Hệ thống chuyển form sang chế độ cho phép chỉnh sửa, các trường thông tin trở nên có thể thay đổi được. US-10 có thể chỉnh sửa các thông tin như: \newline    - Tên người dùng (Name). \newline    - Địa chỉ Email đăng nhập (Login / Email Address) (Lưu ý Business Rule BR-UC10.1-1 về tính duy nhất). \newline    - Ảnh đại diện (Avatar/Image) - có thể tải lên ảnh mới. \newline    - Ngôn ngữ ưu tiên (Language) cho giao diện của người dùng đó. \newline    - Múi giờ (Timezone) của người dùng đó. \newline    - (Nếu có liên kết) Thông tin liên quan đến Hồ sơ Nhân viên (Employee record) trong module HR. \newline    - Các thông tin liên hệ khác như Số điện thoại, Địa chỉ công tác (nếu có). \newline 6. US-10 thực hiện các thay đổi mong muốn trên các trường thông tin. \newline 7. Sau khi hoàn tất việc chỉnh sửa, US-10 chọn hành động "Lưu" (Save) trên form. \newline 8. Hệ thống kiểm tra tính hợp lệ của các dữ liệu đã được thay đổi (ví dụ: nếu Địa chỉ Email đăng nhập được thay đổi, hệ thống phải kiểm tra xem email mới này đã tồn tại trong hệ thống hay chưa). \newline 9. Nếu tất cả các thay đổi là hợp lệ, hệ thống cập nhật thông tin mới cho bản ghi người dùng trong cơ sở dữ liệu. \newline 10. Hệ thống chuyển form trở lại chế độ chỉ xem, hiển thị các thông tin đã được cập nhật, và có thể kèm theo một thông báo "Thông tin người dùng đã được cập nhật thành công." \\
\hline
Alternative Flow & Không có luồng thay thế đáng kể cho hành động sửa thông tin cơ bản này. Việc sửa đổi các Nhóm Quyền được thực hiện trong UC-MD10-04. \\
\hline
Exception Flow & \textbf{8a. Lỗi Xác thực Dữ liệu khi Lưu:} \newline    1. Hệ thống phát hiện rằng Địa chỉ Email đăng nhập mới (nếu US-10 đã thay đổi) đã được sử dụng bởi một tài khoản người dùng khác trong hệ thống, hoặc một trường thông tin bắt buộc (như Tên người dùng) bị xóa trắng. \newline    2. Hệ thống hiển thị một thông báo lỗi cụ thể, chỉ rõ trường thông tin gây ra lỗi. \newline    3. Hệ thống không cho phép lưu các thay đổi. US-10 cần phải sửa lại các thông tin không hợp lệ đó. Use Case quay lại bước 6. \newline \textbf{9a. Lỗi Hệ thống trong quá trình Cập nhật:} \newline    1. Hệ thống gặp phải một lỗi kỹ thuật không mong muốn (ví dụ: lỗi kết nối cơ sở dữ liệu) khi cố gắng thực hiện hành động lưu các thay đổi vào bản ghi người dùng. \newline    2. Hệ thống hiển thị một thông báo lỗi chung. \newline    3. Các thay đổi thông tin người dùng có thể chưa được lưu thành công. US-10 có thể cần thử lại thao tác. \\
\hline
\multicolumn{2}{|c|}{\textbf{2.3. Thông tin bổ sung (Additional Information)}} \\
\hline
\textbf{Mục} & \textbf{Nội dung} \\
\hline
Business Rule & - \textbf{BR-UC10.3-1 (V2):} Nếu Quản trị viên thay đổi Địa chỉ Email đăng nhập của một người dùng, địa chỉ email mới đó vẫn phải đảm bảo tính duy nhất trên toàn bộ hệ thống (không trùng với email đăng nhập của bất kỳ người dùng nào khác). \newline - \textbf{BR-UC10.3-2 (V2):} Việc thay đổi các thông tin cá nhân như Ngôn ngữ ưu tiên hoặc Múi giờ sẽ trực tiếp ảnh hưởng đến cách giao diện hệ thống được hiển thị cho người dùng đó trong các lần đăng nhập tiếp theo của họ. \\
\hline
Non-Functional Requirement & - \textbf{NFR-UC10.3-1 (V2 - Usability):} Form sửa thông tin người dùng phải cho phép Quản trị viên dễ dàng tìm thấy và cập nhật các trường thông tin cần thiết. Việc tải lên hoặc thay đổi ảnh đại diện cần đơn giản. \newline - \textbf{NFR-UC10.3-2 (V2 - Performance):} Thời gian hệ thống cần để lưu các thay đổi thông tin người dùng phải nhanh chóng, không gây chờ đợi. \\
\hline
\end{longtable}

\subsubsection{Use Case UC-MD10-04: Gán/Gỡ bỏ Nhóm Quyền cho Người dùng}
\begin{longtable}{|m{4cm}|p{11cm}|}
\caption{Đặc tả Use Case UC-MD10-04: Gán/Gỡ bỏ Nhóm Quyền cho Người dùng} \label{tab:uc_md10_04_full_v2_latex_fixed_in_codeblock} \\
\hline
\multicolumn{2}{|c|}{\textbf{2.1. Tóm tắt (Summary)}} \\
\hline
\textbf{Mục} & \textbf{Nội dung} \\
\hline
\endhead % Header cho các trang tiếp theo
\midrule
\endfoot % Footer cho bảng
\bottomrule
\endlastfoot % Footer cho trang cuối cùng
Use Case Name & Gán/Gỡ bỏ Nhóm Quyền cho Người dùng \\
\hline
Use Case ID & UC-MD10-04 \\
\hline
Use Case Description & Cho phép Quản trị viên hệ thống (US-10) thay đổi các Nhóm Quyền truy cập (Access Groups) mà một người dùng (nhân viên) cụ thể thuộc về. Hành động này trực tiếp xác định những ứng dụng, menu, chức năng và dữ liệu mà người dùng đó có thể truy cập và thao tác trong toàn bộ hệ thống. \\
\hline
Actor & US-10 (Quản trị viên Hệ thống) \\
\hline
Priority & Must Have \\
\hline
Trigger & - Khi một người dùng mới được tạo (UC-MD10-01), cần phải gán các quyền truy cập ban đầu. \newline - Khi vai trò công việc hoặc trách nhiệm của một nhân viên thay đổi, cần phải cập nhật lại các nhóm quyền của họ cho phù hợp. \newline - Khi cần cấp thêm quyền truy cập vào một ứng dụng hoặc chức năng mới cho một nhân viên. \newline - Khi cần thu hồi bớt quyền truy cập của một nhân viên vì lý do bảo mật hoặc thay đổi nhiệm vụ. \\
\hline
Pre-Condition & - US-10 đã đăng nhập vào hệ thống với quyền quản trị người dùng. \newline - Tài khoản Người dùng cần được phân quyền đã tồn tại trong hệ thống. \newline - Các Nhóm Quyền truy cập phù hợp với các vai trò công việc trong nhà hàng đã tồn tại trong hệ thống (do hệ thống cung cấp sẵn hoặc đã được tùy chỉnh - UC-MD10-08). \\
\hline
Post-Condition & - Danh sách các Nhóm Quyền mà Người dùng được chọn thuộc về được cập nhật trong cơ sở dữ liệu. \newline - Quyền truy cập thực tế của Người dùng đó sẽ thay đổi theo các nhóm quyền mới được gán hoặc gỡ bỏ (thường có hiệu lực ngay lập tức, nhưng đôi khi người dùng có thể cần đăng xuất và đăng nhập lại để thấy đầy đủ các thay đổi về giao diện menu). \\
\hline
\multicolumn{2}{|c|}{\textbf{2.2. Luồng thực thi (Flow)}} \\
\hline
\textbf{Mục} & \textbf{Nội dung} \\
\hline
Basic Flow & 1. US-10 truy cập form chi tiết của Người dùng cần phân quyền (thông qua UC-MD10-02 rồi chọn một người dùng). \newline 2. US-10 chọn hành động "Sửa" (Edit) trên form Người dùng. \newline 3. US-10 tìm đến phần "Quyền Truy cập" (Access Rights) trên form. Phần này thường được tổ chức theo từng Ứng dụng (Application) chính của hệ thống. \newline 4. Đối với mỗi Ứng dụng, US-10 xem xét các tùy chọn Nhóm Quyền có sẵn (thường là dạng checkbox hoặc danh sách thả xuống như "User", "Manager", "Administrator" cho ứng dụng đó). \newline 5. US-10 đánh dấu (tick) vào các ô checkbox của những Nhóm Quyền mà mình muốn gán cho Người dùng này cho từng ứng dụng, hoặc bỏ dấu tick khỏi những Nhóm Quyền muốn gỡ bỏ. \newline 6. US-10 chọn hành động "Lưu" (Save) trên form Người dùng. \newline 7. Hệ thống lưu lại các thay đổi về việc gán/gỡ bỏ Nhóm Quyền cho Người dùng này. \newline 8. Hệ thống hiển thị thông báo "Thông tin người dùng đã được cập nhật thành công." \\
\hline
Alternative Flow & \textbf{1a. Phân quyền bằng cách thêm Người dùng vào một Nhóm Quyền cụ thể:} \newline    1. US-10 truy cập vào danh sách các Nhóm Quyền (UC-MD10-08). \newline    2. US-10 chọn một Nhóm Quyền cụ thể. \newline    3. Hệ thống hiển thị form chi tiết của Nhóm Quyền. US-10 chọn tab "Người dùng" (Users). \newline    4. US-10 chọn "Sửa", sau đó "Thêm một dòng". \newline    5. US-10 tìm và chọn (các) Người dùng muốn thêm vào Nhóm Quyền này. \newline    6. US-10 chọn "Lưu" trên form Nhóm Quyền. \\
\hline
Exception Flow & \textbf{7a. Lỗi hệ thống khi lưu thay đổi phân quyền:} \newline    1. Hệ thống gặp lỗi kỹ thuật khi cố gắng lưu các thay đổi. \newline    2. Hệ thống hiển thị thông báo lỗi chung. Thay đổi có thể không được lưu. \\
\hline
\multicolumn{2}{|c|}{\textbf{2.3. Thông tin bổ sung (Additional Information)}} \\
\hline
\textbf{Mục} & \textbf{Nội dung} \\
\hline
Business Rule & - \textbf{BR-UC10.4-1 (V2):} Quản trị viên phải tuân theo nguyên tắc phân quyền tối thiểu. \newline - \textbf{BR-UC10.4-2 (V2):} Quản trị viên cần hiểu rõ ý nghĩa của từng Nhóm Quyền. \newline - \textbf{BR-UC10.4-3 (V2):} Quyền hạn mới thường có hiệu lực sau khi người dùng đăng xuất và đăng nhập lại. \\
\hline
Non-Functional Requirement & - \textbf{NFR-UC10.4-1 (V2 - Usability):} Giao diện gán quyền phải trực quan. \newline - \textbf{NFR-UC10.4-2 (V2 - Security):} Phân quyền chính xác là yếu tố then chốt. \newline - \textbf{NFR-UC10.4-3 (V2 - Maintainability):} Phân quyền theo nhóm giúp dễ quản lý. \\
\hline
\end{longtable}

\subsubsection{Use Case UC-MD10-05: Vô hiệu hóa Tài khoản Người dùng}
\begin{longtable}{|m{4cm}|p{11cm}|}
\caption{Đặc tả Use Case UC-MD10-05: Vô hiệu hóa Tài khoản Người dùng} \label{tab:uc_md10_05_full_v2_latex_fixed_in_codeblock} \\
\hline
\multicolumn{2}{|c|}{\textbf{2.1. Tóm tắt (Summary)}} \\
\hline
\textbf{Mục} & \textbf{Nội dung} \\
\hline
\endhead % Header cho các trang tiếp theo
\midrule
\endfoot % Footer cho bảng
\bottomrule
\endlastfoot % Footer cho trang cuối cùng
Use Case Name & Vô hiệu hóa Tài khoản Người dùng \\
\hline
Use Case ID & UC-MD10-05 \\
\hline
Use Case Description & Cho phép Quản trị viên hệ thống (US-10) tạm thời hoặc vĩnh viễn khóa khả năng đăng nhập vào hệ thống của một tài khoản người dùng (nhân viên). \\
\hline
Actor & US-10 (Quản trị viên Hệ thống) \\
\hline
Priority & Must Have \\
\hline
Trigger & - Một nhân viên nghỉ việc. \newline - Một nhân viên tạm nghỉ dài hạn. \newline - Phát hiện hành vi đáng ngờ từ một tài khoản. \\
\hline
Pre-Condition & - US-10 đã đăng nhập với quyền quản trị người dùng. \newline - Tài khoản Người dùng cần vô hiệu hóa đang ở trạng thái hoạt động (Active=True). \\
\hline
Post-Condition & - Trạng thái của tài khoản Người dùng được cập nhật thành "Không hoạt động" (Active=False) / "Đã lưu trữ" (Archived). \newline - Người dùng đó không thể đăng nhập vào hệ thống nữa. \newline - Dữ liệu lịch sử của người dùng vẫn được giữ lại. \\
\hline
\multicolumn{2}{|c|}{\textbf{2.2. Luồng thực thi (Flow)}} \\
\hline
\textbf{Mục} & \textbf{Nội dung} \\
\hline
Basic Flow (Từ Form chi tiết Người dùng) & 1. US-10 truy cập danh sách Người dùng (UC-MD10-02). \newline 2. US-10 tìm và chọn người dùng muốn vô hiệu hóa. \newline 3. Hệ thống hiển thị form chi tiết người dùng. \newline 4. US-10 tìm menu "Hành động" (Action). \newline 5. US-10 chọn "Lưu trữ" (Archive). \newline 6. Hệ thống (có thể) yêu cầu xác nhận. US-10 xác nhận. \newline 7. Hệ thống cập nhật trạng thái `active` của người dùng thành `False`. \newline 8. Hệ thống báo "Người dùng đã được lưu trữ thành công." \newline 9. Người dùng đó không còn xuất hiện trong danh sách người dùng hoạt động mặc định. \\
\hline
Alternative Flow & \textbf{Basic Flow (Từ Danh sách Người dùng - List View):} \newline    1. US-10 đang xem danh sách Người dùng. \newline    2. US-10 chọn (tick) một hoặc nhiều người dùng muốn vô hiệu hóa. \newline    3. US-10 chọn menu "Hành động" chung của danh sách. \newline    4. US-10 chọn "Lưu trữ". \newline    5. Hệ thống (có thể) yêu cầu xác nhận. US-10 xác nhận. \newline    6. Hệ thống cập nhật `active = False` cho các người dùng đã chọn. \newline    7. Hệ thống làm mới danh sách. \newline    8. Hệ thống báo thành công. \\
\hline
Exception Flow & \textbf{7a. Lỗi hệ thống khi cập nhật trạng thái:} \newline    1. Hệ thống gặp lỗi khi cập nhật trường `active`. \newline    2. Hệ thống báo lỗi chung. Trạng thái có thể không thay đổi. \\
\hline
\multicolumn{2}{|c|}{\textbf{2.3. Thông tin bổ sung (Additional Information)}} \\
\hline
\textbf{Mục} & \textbf{Nội dung} \\
\hline
Business Rule & - \textbf{BR-UC10.5-1 (V2):} Vô hiệu hóa (Archive) là phương pháp được khuyến nghị khi nhân viên nghỉ việc, thay vì xóa hẳn, để bảo toàn lịch sử dữ liệu. \\
\hline
Non-Functional Requirement & - \textbf{NFR-UC10.5-1 (V2 - Usability):} Hành động Vô hiệu hóa phải dễ dàng thực hiện. \newline - \textbf{NFR-UC10.5-2 (V2 - Security):} Đảm bảo người dùng bị vô hiệu hóa không thể đăng nhập được nữa. \\
\hline
\end{longtable}

\subsubsection{Use Case UC-MD10-06: Kích hoạt lại Tài khoản Người dùng đã Vô hiệu hóa}
\begin{longtable}{|m{4cm}|p{11cm}|}
\caption{Đặc tả Use Case UC-MD10-06: Kích hoạt lại Tài khoản Người dùng đã Vô hiệu hóa} \label{tab:uc_md10_06_full_v2_latex_fixed_in_codeblock} \\
\hline
\multicolumn{2}{|c|}{\textbf{2.1. Tóm tắt (Summary)}} \\
\hline
\textbf{Mục} & \textbf{Nội dung} \\
\hline
\endhead % Header cho các trang tiếp theo
\midrule
\endfoot % Footer cho bảng
\bottomrule
\endlastfoot % Footer cho trang cuối cùng
Use Case Name & Kích hoạt lại Tài khoản Người dùng đã Vô hiệu hóa \\
\hline
Use Case ID & UC-MD10-06 \\
\hline
Use Case Description & Cho phép Quản trị viên hệ thống (US-10) mở lại khả năng đăng nhập cho một tài khoản người dùng (nhân viên) đã bị vô hiệu hóa (lưu trữ) trước đó. \\
\hline
Actor & US-10 (Quản trị viên Hệ thống) \\
\hline
Priority & Should Have \\
\hline
Trigger & - Một nhân viên quay trở lại làm việc sau thời gian tạm nghỉ. \newline - Một tài khoản bị vô hiệu hóa nhầm cần được kích hoạt lại. \\
\hline
Pre-Condition & - US-10 đã đăng nhập với quyền quản trị người dùng. \newline - Tài khoản Người dùng cần kích hoạt lại đang ở trạng thái "Không hoạt động" (Active=False) / "Đã lưu trữ" (Archived). \\
\hline
Post-Condition & - Trạng thái của tài khoản Người dùng được cập nhật thành "Hoạt động" (Active=True). \newline - Người dùng đó có thể đăng nhập lại vào hệ thống bằng thông tin đăng nhập cũ (trừ khi mật khẩu cũng cần đặt lại). \\
\hline
\multicolumn{2}{|c|}{\textbf{2.2. Luồng thực thi (Flow)}} \\
\hline
\textbf{Mục} & \textbf{Nội dung} \\
\hline
Basic Flow (Từ Form chi tiết Người dùng) & 1. US-10 truy cập danh sách Người dùng (UC-MD10-02) và sử dụng bộ lọc để hiển thị các người dùng đã bị lưu trữ (ví dụ: bỏ bộ lọc "Active=True" hoặc thêm bộ lọc "Archived=True"). \newline 2. US-10 tìm và chọn người dùng muốn kích hoạt lại. \newline 3. Hệ thống hiển thị form chi tiết người dùng. \newline 4. US-10 tìm menu "Hành động" (Action). \newline 5. US-10 chọn tùy chọn "Hủy lưu trữ" (Unarchive). \newline 6. Hệ thống (có thể) yêu cầu xác nhận. US-10 xác nhận. \newline 7. Hệ thống cập nhật trạng thái `active` của người dùng thành `True`. \newline 8. Hệ thống báo "Người dùng đã được hủy lưu trữ thành công." \newline 9. Người dùng đó sẽ xuất hiện trở lại trong danh sách người dùng hoạt động mặc định. \\
\hline
Alternative Flow & \textbf{Basic Flow (Từ Danh sách Người dùng - List View):} \newline    1. US-10 lọc danh sách Người dùng để hiển thị các tài khoản đã lưu trữ. \newline    2. US-10 chọn (tick) một hoặc nhiều người dùng muốn kích hoạt lại. \newline    3. US-10 chọn menu "Hành động" chung của danh sách. \newline    4. US-10 chọn "Hủy lưu trữ". \newline    5. Hệ thống (có thể) yêu cầu xác nhận. US-10 xác nhận. \newline    6. Hệ thống cập nhật `active = True` cho các người dùng đã chọn. \newline    7. Hệ thống làm mới danh sách. \newline    8. Hệ thống báo thành công. \\
\hline
Exception Flow & \textbf{7a. Lỗi hệ thống khi cập nhật trạng thái:} \newline    1. Hệ thống gặp lỗi khi cập nhật trường `active`. \newline    2. Hệ thống báo lỗi chung. Trạng thái có thể không thay đổi. \\
\hline
\multicolumn{2}{|c|}{\textbf{2.3. Thông tin bổ sung (Additional Information)}} \\
\hline
\textbf{Mục} & \textbf{Nội dung} \\
\hline
Business Rule & - \textbf{BR-UC10.6-1 (V2):} Chỉ những tài khoản đang ở trạng thái "Lưu trữ" / "Không hoạt động" mới có thể được Hủy lưu trữ / Kích hoạt lại. \\
\hline
Non-Functional Requirement & - \textbf{NFR-UC10.6-1 (V2 - Usability):} Việc tìm và kích hoạt lại tài khoản phải dễ dàng. \\
\hline
\end{longtable}

\subsubsection{Use Case UC-MD10-07: Đặt lại Mật khẩu cho Người dùng (bởi Admin)}
\begin{longtable}{|m{4cm}|p{11cm}|}
\caption{Đặc tả Use Case UC-MD10-07: Đặt lại Mật khẩu cho Người dùng (bởi Admin)} \label{tab:uc_md10_07_full_v2_latex_fixed_in_codeblock} \\
\hline
\multicolumn{2}{|c|}{\textbf{2.1. Tóm tắt (Summary)}} \\
\hline
\textbf{Mục} & \textbf{Nội dung} \\
\hline
\endhead % Header cho các trang tiếp theo
\midrule
\endfoot % Footer cho bảng
\bottomrule
\endlastfoot % Footer cho trang cuối cùng
Use Case Name & Đặt lại Mật khẩu cho Người dùng (bởi Admin) \\
\hline
Use Case ID & UC-MD10-07 \\
\hline
Use Case Description & Cho phép Quản trị viên hệ thống (US-10) hỗ trợ một người dùng (nhân viên) đặt lại mật khẩu đăng nhập vào hệ thống khi người dùng đó quên mật khẩu hoặc cần thay đổi mật khẩu vì lý do bảo mật. Admin có thể gửi link tự đặt lại hoặc đặt mật khẩu mới trực tiếp. \\
\hline
Actor & US-10 (Quản trị viên Hệ thống) \\
\hline
Priority & Must Have \\
\hline
Trigger & - Một nhân viên thông báo cho US-10 rằng họ đã quên mật khẩu đăng nhập. \newline - Cần chủ động đặt lại mật khẩu cho một tài khoản vì lý do bảo mật. \\
\hline
Pre-Condition & - US-10 đã đăng nhập vào hệ thống với quyền quản trị người dùng. \newline - Tài khoản Người dùng cần đặt lại mật khẩu đã tồn tại trong hệ thống. \\
\hline
Post-Condition & - Mật khẩu của người dùng được chọn đã được đặt lại. \newline - Người dùng có thể nhận được mật khẩu mới (nếu admin đặt) hoặc một liên kết để tự đặt mật khẩu mới qua email. \\
\hline
\multicolumn{2}{|c|}{\textbf{2.2. Luồng thực thi (Flow)}} \\
\hline
\textbf{Mục} & \textbf{Nội dung} \\
\hline
Basic Flow (Gửi email hướng dẫn đặt lại mật khẩu) & 1. US-10 truy cập danh sách Người dùng (UC-MD10-02). \newline 2. US-10 tìm và chọn người dùng cần đặt lại mật khẩu. \newline 3. Hệ thống hiển thị form chi tiết người dùng. \newline 4. US-10 tìm menu "Hành động" (Action) trên form. \newline 5. US-10 chọn tùy chọn "Gửi Hướng dẫn Đặt lại Mật khẩu" (Send Password Reset Instructions) hoặc "Đặt lại Mật khẩu" (sau đó hệ thống sẽ có tùy chọn gửi email). \newline 6. Hệ thống (có thể) yêu cầu US-10 xác nhận hành động. US-10 xác nhận. \newline 7. Hệ thống tự động tạo một liên kết đặt lại mật khẩu dùng một lần và gửi một email đến địa chỉ email đã đăng ký của người dùng đó, chứa liên kết này cùng hướng dẫn. \newline 8. Hệ thống hiển thị thông báo "Email hướng dẫn đặt lại mật khẩu đã được gửi đến [Email người dùng]." \\
\hline
Alternative Flow & \textbf{5a. Quản trị viên đặt mật khẩu mới trực tiếp (Nếu hệ thống cho phép và chính sách bảo mật chấp nhận):} \newline    1. Từ form chi tiết người dùng (chế độ Sửa), US-10 chọn nút "Thay đổi Mật khẩu" (Change Password). \newline    2. Hệ thống hiển thị hộp thoại yêu cầu nhập Mật khẩu mới và Xác nhận Mật khẩu mới. \newline    3. US-10 nhập mật khẩu mới (tuân thủ chính sách độ phức tạp nếu có). \newline    4. US-10 chọn "Thay đổi Mật khẩu". \newline    5. Hệ thống cập nhật mật khẩu mới cho người dùng. \newline    6. US-10 cần thông báo mật khẩu mới này cho người dùng một cách an toàn. \\
\hline
Exception Flow & \textbf{7a. Lỗi hệ thống khi gửi email / Lỗi máy chủ email:} \newline    1. Hệ thống không thể gửi email đặt lại mật khẩu (do lỗi cấu hình Outgoing Email Server - UC-MD10-10, hoặc địa chỉ email người dùng không hợp lệ). \newline    2. Hệ thống báo lỗi "Không thể gửi email." US-10 có thể cần thử phương án đặt mật khẩu trực tiếp (Alternative Flow 5a) hoặc kiểm tra lại cấu hình email. \newline \textbf{Alternative Flow 5a - Step 5a. Lỗi hệ thống khi thay đổi mật khẩu trực tiếp:} \newline    1. Hệ thống gặp lỗi khi cố gắng lưu mật khẩu mới. \newline    2. Hệ thống báo lỗi. \\
\hline
\multicolumn{2}{|c|}{\textbf{2.3. Thông tin bổ sung (Additional Information)}} \\
\hline
\textbf{Mục} & \textbf{Nội dung} \\
\hline
Business Rule & - \textbf{BR-UC10.7-1 (V2):} Phương thức gửi email cho người dùng tự đặt lại mật khẩu thường được ưu tiên hơn vì lý do bảo mật (admin không cần biết mật khẩu của người dùng). \newline - \textbf{BR-UC10.7-2 (V2):} Liên kết đặt lại mật khẩu gửi qua email phải có thời hạn sử dụng và chỉ dùng được một lần. \newline - \textbf{BR-UC10.7-3 (V2):} Mật khẩu mới (dù do người dùng hay admin đặt) nên tuân thủ chính sách độ mạnh mật khẩu của hệ thống (nếu có). \\
\hline
Non-Functional Requirement & - \textbf{NFR-UC10.7-1 (V2 - Security):} Quy trình đặt lại mật khẩu phải được thiết kế an toàn, tránh việc người không có thẩm quyền có thể chiếm quyền tài khoản. \newline - \textbf{NFR-UC10.7-2 (V2 - Usability):} Hướng dẫn đặt lại mật khẩu cho người dùng cuối (qua email) phải rõ ràng và dễ thực hiện. Giao diện cho admin đặt lại mật khẩu cũng phải đơn giản. \\
\hline
\end{longtable}

\subsubsection{Use Case UC-MD10-08: Xem Chi tiết một Nhóm Quyền Truy cập}
\begin{longtable}{|m{4cm}|p{11cm}|}
\caption{Đặc tả Use Case UC-MD10-08: Xem Chi tiết một Nhóm Quyền Truy cập} \label{tab:uc_md10_08_full_v2_latex_fixed_in_codeblock} \\
\hline
\multicolumn{2}{|c|}{\textbf{2.1. Tóm tắt (Summary)}} \\
\hline
\textbf{Mục} & \textbf{Nội dung} \\
\hline
\endhead % Header cho các trang tiếp theo
\midrule
\endfoot % Footer cho bảng
\bottomrule
\endlastfoot % Footer cho trang cuối cùng
Use Case Name & Xem Chi tiết một Nhóm Quyền Truy cập \\
\hline
Use Case ID & UC-MD10-08 \\
\hline
Use Case Description & Cho phép Quản trị viên hệ thống (US-10) xem chi tiết cấu hình và các quyền hạn cụ thể được định nghĩa trong một Nhóm Quyền truy cập (Access Group) đã tồn tại trong hệ thống. Điều này giúp Quản trị viên hiểu rõ phạm vi tác động của nhóm quyền trước khi gán cho người dùng. \\
\hline
Actor & US-10 (Quản trị viên Hệ thống) \\
\hline
Priority & Should Have (Để hiểu rõ hệ thống phân quyền) \\
\hline
Trigger & - Quản trị viên cần tìm hiểu xem một Nhóm Quyền cụ thể cho phép người dùng làm những gì trước khi thực hiện gán quyền cho một tài khoản (UC-MD10-04). \newline - Khi cần rà soát lại cấu trúc phân quyền của hệ thống. \\
\hline
Pre-Condition & - US-10 đã đăng nhập vào hệ thống với quyền quản trị hệ thống cao nhất. \newline - US-10 đã kích hoạt Chế độ Nhà phát triển (Developer Mode) trong hệ thống để có thể thấy các menu kỹ thuật. \\
\hline
Post-Condition & - Quản trị viên nắm được thông tin chi tiết về các quyền hạn (truy cập menu, quyền CRUD trên đối tượng, quy tắc bản ghi...) được định nghĩa trong Nhóm Quyền đã chọn. \\
\hline
\multicolumn{2}{|c|}{\textbf{2.2. Luồng thực thi (Flow)}} \\
\hline
\textbf{Mục} & \textbf{Nội dung} \\
\hline
Basic Flow & 1. US-10 (đã kích hoạt Developer Mode) truy cập vào "Cài đặt" (Settings). \newline 2. US-10 điều hướng đến menu "Kỹ thuật" (Technical) > "Bảo mật" (Security) > "Nhóm" (Groups). \newline 3. Hệ thống hiển thị danh sách tất cả các Nhóm Quyền hiện có. \newline 4. US-10 tìm và chọn (nhấp vào) một Nhóm Quyền cụ thể muốn xem chi tiết. \newline 5. Hệ thống hiển thị form chi tiết của Nhóm Quyền đã chọn, bao gồm các tab/phần thông tin như: \newline    - Tên Nhóm, Ứng dụng liên quan. \newline    - Các nhóm được kế thừa (Implied Groups). \newline    - Danh sách Người dùng hiện thuộc nhóm này. \newline    - Các Menu mà nhóm này được phép truy cập. \newline    - Các Quyền Truy cập đối tượng (Access Rights: Read, Write, Create, Delete trên các Models). \newline    - Các Quy tắc Bản ghi (Record Rules) áp dụng. \newline    - Các Chế độ xem (Views) được phép. \newline 6. US-10 xem xét các thông tin chi tiết này. \\
\hline
Alternative Flow & \textbf{4a. Lọc/Tìm kiếm Nhóm Quyền:} \newline    1. US-10 sử dụng các bộ lọc (theo Ứng dụng) hoặc ô tìm kiếm để nhanh chóng tìm thấy Nhóm Quyền cần xem. \\
\hline
Exception Flow & \textbf{2a. Chưa kích hoạt Developer Mode:} \newline    1. Menu "Kỹ thuật" và "Nhóm" không hiển thị. US-10 không thể truy cập. \newline \textbf{5a. Lỗi hệ thống khi tải chi tiết Nhóm Quyền:} \newline    1. Hệ thống gặp lỗi kỹ thuật. \newline    2. Hệ thống báo lỗi chung. \\
\hline
\multicolumn{2}{|c|}{\textbf{2.3. Thông tin bổ sung (Additional Information)}} \\
\hline
\textbf{Mục} & \textbf{Nội dung} \\
\hline
Business Rule & - \textbf{BR-UC10.8-1 (V2 - System):} Hệ thống phân quyền là một cơ chế phức tạp và mạnh mẽ, dựa trên sự kết hợp của các Nhóm Quyền, quyền truy cập mô hình, và quy tắc bản ghi. \newline - \textbf{BR-UC10.8-2 (V2):} Việc hiểu rõ cách các nhóm quyền được định nghĩa và kế thừa là rất quan trọng cho việc phân quyền chính xác và an toàn. \\
\hline
Non-Functional Requirement & - \textbf{NFR-UC10.8-1 (V2 - Usability):} Mặc dù là chức năng kỹ thuật, giao diện hiển thị chi tiết nhóm quyền nên cố gắng trình bày thông tin một cách có cấu trúc và dễ theo dõi nhất có thể. \newline - \textbf{NFR-UC10.8-2 (V2 - Security):} Quyền truy cập để xem (và đặc biệt là sửa) Nhóm Quyền phải được giới hạn ở mức quản trị cao nhất. \\
\hline
\end{longtable}

\subsubsection{Use Case UC-MD10-09: Cấu hình Thông tin Chung của Nhà hàng}
\begin{longtable}{|m{4cm}|p{11cm}|}
\caption{Đặc tả Use Case UC-MD10-09: Cấu hình Thông tin Chung của Nhà hàng} \label{tab:uc_md10_09_full_v2_latex_fixed_in_codeblock} \\
\hline
\multicolumn{2}{|c|}{\textbf{2.1. Tóm tắt (Summary)}} \\
\hline
\textbf{Mục} & \textbf{Nội dung} \\
\hline
\endhead % Header cho các trang tiếp theo
\midrule
\endfoot % Footer cho bảng
\bottomrule
\endlastfoot % Footer cho trang cuối cùng
Use Case Name & Cấu hình Thông tin Chung của Nhà hàng \\
\hline
Use Case ID & UC-MD10-09 \\
\hline
Use Case Description & Cho phép Quản trị viên hệ thống (US-10) hoặc Quản lý nhà hàng (US-01) thiết lập và cập nhật các thông tin cơ bản và chung nhất của nhà hàng/công ty, ví dụ như tên đầy đủ của nhà hàng, địa chỉ, số điện thoại, địa chỉ email liên hệ, website, logo, và đơn vị tiền tệ mặc định sẽ được sử dụng trong toàn bộ hệ thống. \\
\hline
Actor & US-10 (Quản trị viên Hệ thống), US-01 (Quản lý nhà hàng) \\
\hline
Priority & Must Have \\
\hline
Trigger & - Khi thiết lập hệ thống lần đầu tiên cho nhà hàng. \newline - Khi có sự thay đổi về thông tin pháp lý hoặc thông tin liên hệ của nhà hàng (ví dụ: đổi địa chỉ, cập nhật logo mới, thay đổi SĐT). \newline - Khi cần thay đổi đơn vị tiền tệ chính mà hệ thống sử dụng. \\
\hline
Pre-Condition & - Người dùng (US-10 hoặc US-01 có quyền) đã đăng nhập vào hệ thống với quyền truy cập vào phần Cài đặt Chung (General Settings). \\
\hline
Post-Condition & - Các thông tin chung của nhà hàng (tên, địa chỉ, logo, tiền tệ...) được cập nhật và lưu trữ trong cấu hình hệ thống. \newline - Các thông tin này sẽ được hệ thống tự động sử dụng và hiển thị trên các tài liệu chính thức (ví dụ: hóa đơn, báo cáo, email gửi đi), cũng như ảnh hưởng đến cách các giao dịch tài chính được ghi nhận (ví dụ: theo đơn vị tiền tệ đã chọn). \\
\hline
\multicolumn{2}{|c|}{\textbf{2.2. Luồng thực thi (Flow)}} \\
\hline
\textbf{Mục} & \textbf{Nội dung} \\
\hline
Basic Flow & 1. Người dùng (US-10 hoặc US-01) truy cập vào mục "Cài đặt" (Settings) trên giao diện chính. \newline 2. Hệ thống hiển thị trang Cài đặt Chung (General Settings). Người dùng có thể cần tìm đến phần "Công ty" (Companies) hoặc "Thông tin Công ty" (Company Information). \newline 3. US-10/US-01 chọn công ty/nhà hàng hiện tại để chỉnh sửa (trong trường hợp hệ thống quản lý đa công ty, nếu không thì sẽ là công ty mặc định). \newline 4. Hệ thống hiển thị form cho phép US-10/US-01 nhập hoặc cập nhật các thông tin sau: \newline    - Tên Công ty/Nhà hàng (Company Name). \newline    - Địa chỉ chi tiết (Address). \newline    - Số điện thoại (Phone). \newline    - Địa chỉ Email (Email). \newline    - Trang web (Website). \newline    - Mã số thuế (VAT/Tax ID). \newline    - Logo của nhà hàng (cho phép tải lên tệp hình ảnh). \newline    - Đơn vị tiền tệ mặc định của công ty (Company Currency). \newline    - (Tùy chọn) Các thông tin khác như tài khoản mạng xã hội, thông tin ngân hàng... \newline 5. US-10/US-01 thực hiện nhập mới hoặc chỉnh sửa các thông tin cần thiết. \newline 6. Sau khi hoàn tất, US-10/US-01 chọn hành động "Lưu" (Save) để áp dụng các thay đổi. \newline 7. Hệ thống kiểm tra tính hợp lệ cơ bản của dữ liệu (ví dụ: định dạng email, SĐT nếu có). \newline 8. Hệ thống lưu lại các thông tin cấu hình chung mới. \newline 9. Hệ thống hiển thị thông báo "Cài đặt đã được lưu thành công." \\
\hline
Alternative Flow & \textbf{4a. Quản lý nhiều công ty (nếu có):} \newline    1. Nếu hệ thống được cấu hình cho nhiều công ty, US-10/US-01 có thể cần chọn đúng bản ghi công ty từ danh sách trước khi chỉnh sửa. \\
\hline
Exception Flow & \textbf{7a. Lỗi xác thực dữ liệu khi lưu:} \newline    1. Người dùng nhập giá trị không hợp lệ cho một trường nào đó (ví dụ: định dạng email không đúng, thiếu tên công ty). \newline    2. Hệ thống hiển thị thông báo lỗi, chỉ rõ trường bị sai. \newline    3. Hệ thống không lưu các thay đổi. US-10/US-01 cần sửa lại. \newline \textbf{8a. Lỗi hệ thống trong quá trình lưu cấu hình:} \newline    1. Hệ thống gặp lỗi kỹ thuật khi cố gắng lưu các thông tin cấu hình chung. \newline    2. Hệ thống hiển thị thông báo lỗi chung. Các thay đổi có thể không được lưu. \\
\hline
\multicolumn{2}{|c|}{\textbf{2.3. Thông tin bổ sung (Additional Information)}} \\
\hline
\textbf{Mục} & \textbf{Nội dung} \\
\hline
Business Rule & - \textbf{BR-UC10.9-1 (V2):} Thông tin Công ty/Nhà hàng (đặc biệt là Tên, Địa chỉ, Mã số thuế, Logo) sẽ được hệ thống tự động sử dụng để hiển thị trên các tài liệu chính thức như hóa đơn bán hàng, phiếu đặt chỗ, báo cáo tài chính, và các giao tiếp khác với khách hàng. \newline - \textbf{BR-UC10.9-2 (V2):} Đơn vị tiền tệ mặc định của công ty (Company Currency) là đơn vị tiền tệ chính được sử dụng cho tất cả các giao dịch tài chính và báo cáo trong hệ thống. Việc thay đổi đơn vị tiền tệ này sau khi đã có giao dịch là một thao tác phức tạp và cần cân nhắc kỹ. \\
\hline
Non-Functional Requirement & - \textbf{NFR-UC10.9-1 (V2 - Usability):} Giao diện cấu hình thông tin chung phải dễ dàng cho người dùng tìm thấy và cập nhật các thông tin cần thiết. Việc tải lên logo cần đơn giản. \newline - \textbf{NFR-UC10.9-2 (V2 - Impact):} Quản trị viên/Quản lý cần ý thức rằng các thay đổi trong phần Cài đặt chung này có thể có ảnh hưởng trên toàn bộ hệ thống và cách hệ thống hiển thị thông tin ra bên ngoài. \\
\hline
\end{longtable}

\subsubsection{Use Case UC-MD10-10: Cấu hình Máy chủ Gửi Email (Outgoing Email Server)}
\begin{longtable}{|m{4cm}|p{11cm}|}
\caption{Đặc tả Use Case UC-MD10-10: Cấu hình Máy chủ Gửi Email (Outgoing Email Server)} \label{tab:uc_md10_10_full_v2_latex_fixed_in_codeblock} \\
\hline
\multicolumn{2}{|c|}{\textbf{2.1. Tóm tắt (Summary)}} \\
\hline
\textbf{Mục} & \textbf{Nội dung} \\
\hline
\endhead % Header cho các trang tiếp theo
\midrule
\endfoot % Footer cho bảng
\bottomrule
\endlastfoot % Footer cho trang cuối cùng
Use Case Name & Cấu hình Máy chủ Gửi Email (Outgoing Email Server) \\
\hline
Use Case ID & UC-MD10-10 \\
\hline
Use Case Description & Cho phép Quản trị viên hệ thống (US-10) hoặc Quản lý nhà hàng (US-01) thiết lập và quản lý thông tin cấu hình cho máy chủ gửi email (SMTP server). Cấu hình này là bắt buộc để hệ thống có thể tự động gửi đi các email giao dịch như xác nhận đặt chỗ, thông báo lịch làm việc, email đặt lại mật khẩu, v.v. \\
\hline
Actor & US-10 (Quản trị viên Hệ thống), US-01 (Quản lý nhà hàng) \\
\hline
Priority & Must Have (Để các tính năng gửi email tự động hoạt động) \\
\hline
Trigger & - Khi thiết lập hệ thống lần đầu tiên và cần kích hoạt khả năng gửi email. \newline - Khi nhà hàng thay đổi nhà cung cấp dịch vụ email hoặc thông tin máy chủ SMTP. \newline - Khi cần khắc phục sự cố liên quan đến việc gửi email từ hệ thống. \\
\hline
Pre-Condition & - Người dùng (US-10 hoặc US-01 có quyền) đã đăng nhập vào hệ thống với quyền truy cập vào phần Cài đặt Kỹ thuật hoặc Cài đặt Email. \newline - Nhà hàng đã có thông tin chi tiết về máy chủ SMTP muốn sử dụng (ví dụ: từ Gmail, Microsoft 365, hoặc một dịch vụ SMTP relay chuyên dụng như SendGrid, Mailgun), bao gồm: địa chỉ máy chủ, cổng, thông tin đăng nhập (tên người dùng, mật khẩu/API key), và loại mã hóa. \\
\hline
Post-Condition & - Thông tin cấu hình máy chủ gửi email (Outgoing Email Server) được lưu trữ chính xác trong hệ thống. \newline - Nếu cấu hình đúng và máy chủ SMTP hoạt động, hệ thống có thể gửi email đi thành công. \\
\hline
\multicolumn{2}{|c|}{\textbf{2.2. Luồng thực thi (Flow)}} \\
\hline
\textbf{Mục} & \textbf{Nội dung} \\
\hline
Basic Flow & 1. US-10/US-01 (đã kích hoạt Developer Mode nếu cần) truy cập vào "Cài đặt" (Settings). \newline 2. US-10/US-01 điều hướng đến menu "Kỹ thuật" (Technical) > "Email" > "Máy chủ Thư đi" (Outgoing Email Servers). \newline 3. Hệ thống hiển thị danh sách các máy chủ gửi email đã được cấu hình (nếu có). \newline 4. US-10/US-01 chọn "Tạo mới" (Create) để thêm một cấu hình máy chủ mới, hoặc chọn một cấu hình hiện có để "Sửa" (Edit). \newline 5. Hệ thống hiển thị form để nhập thông tin cấu hình máy chủ gửi email. Các trường chính bao gồm: \newline    - \textbf{Mô tả (Description):} Một tên gợi nhớ cho cấu hình này (ví dụ: "Gmail SMTP Server"). \newline    - \textbf{Ưu tiên (Priority):} Nếu có nhiều máy chủ, máy chủ có ưu tiên thấp hơn (số nhỏ hơn) sẽ được thử trước. \newline    - \textbf{Máy chủ SMTP (SMTP Server):} Địa chỉ của máy chủ SMTP (ví dụ: smtp.gmail.com). \newline    - \textbf{Cổng SMTP (SMTP Port):} Số cổng SMTP (ví dụ: 587 hoặc 465). \newline    - \textbf{Bảo mật Kết nối (Connection Security):} Chọn loại mã hóa (ví dụ: None, SSL/TLS, STARTTLS). \newline    - \textbf{Tên người dùng (Username):} Địa chỉ email hoặc tên người dùng để xác thực với máy chủ SMTP. \newline    - \textbf{Mật khẩu (Password):} Mật khẩu tương ứng (hoặc App Password nếu dùng Gmail với 2FA). \newline 6. US-10/US-01 nhập đầy đủ và chính xác các thông tin cấu hình này. \newline 7. (Rất quan trọng) US-10/US-01 sử dụng nút "Kiểm tra Kết nối" (Test Connection) trên form. \newline 8. Hệ thống cố gắng thiết lập một kết nối thử nghiệm đến máy chủ SMTP với các thông tin đã cung cấp. \newline 9. Hệ thống hiển thị kết quả kiểm tra: \newline    a. \textbf{Nếu thành công:} Thông báo "Kết nối thành công!" hoặc tương tự. \newline    b. \textbf{Nếu thất bại:} Thông báo lỗi chi tiết (ví dụ: "Xác thực thất bại", "Không thể kết nối đến máy chủ", "Cổng sai"...). US-10/US-01 cần sửa lại thông tin cấu hình (quay lại bước 6). \newline 10. Sau khi kiểm tra kết nối thành công (bước 9a), US-10/US-01 chọn "Lưu" (Save). \newline 11. Hệ thống lưu lại cấu hình máy chủ gửi email. \\
\hline
Alternative Flow & Không có luồng thay thế đáng kể cho việc cấu hình trực tiếp này. \\
\hline
Exception Flow & \textbf{10a. Lỗi hệ thống khi lưu cấu hình:} \newline    1. Hệ thống gặp lỗi kỹ thuật khi cố gắng lưu thông tin cấu hình. \newline    2. Hệ thống hiển thị thông báo lỗi chung. Cấu hình có thể chưa được lưu. \\
\hline
\multicolumn{2}{|c|}{\textbf{2.3. Thông tin bổ sung (Additional Information)}} \\
\hline
\textbf{Mục} & \textbf{Nội dung} \\
\hline
Business Rule & - \textbf{BR-UC10.10-1 (V2):} Việc cấu hình chính xác ít nhất một Máy chủ Gửi Email đang hoạt động là điều kiện tiên quyết để hệ thống có thể gửi đi bất kỳ email nào (ví dụ: xác nhận đặt chỗ cho khách, thông báo lịch làm việc cho nhân viên, email đặt lại mật khẩu). \newline - \textbf{BR-UC10.10-2 (V2):} Thông tin xác thực (Tên người dùng, Mật khẩu/API Key) cho máy chủ SMTP phải chính xác và còn hiệu lực. \newline - \textbf{BR-UC10.10-3 (V2):} Cài đặt về Bảo mật Kết nối (SSL/TLS, STARTTLS) và Cổng SMTP phải khớp với yêu cầu của nhà cung cấp dịch vụ SMTP. \\
\hline
Non-Functional Requirement & - \textbf{NFR-UC10.10-1 (V2 - Security):} Mật khẩu hoặc API Key của máy chủ SMTP là thông tin nhạy cảm và phải được hệ thống lưu trữ một cách an toàn (ví dụ: được mã hóa trong cơ sở dữ liệu và không hiển thị dạng text thuần trên giao diện sau khi lưu). \newline - \textbf{NFR-UC10.10-2 (V2 - Usability):} Giao diện cấu hình Máy chủ Gửi Email phải cung cấp đủ các trường cần thiết. Chức năng "Kiểm tra Kết nối" là cực kỳ quan trọng và hữu ích để người dùng xác thực cấu hình trước khi lưu. \newline - \textbf{NFR-UC10.10-3 (V2 - Reliability):} Sau khi được cấu hình đúng, hệ thống gửi email của hệ thống phải hoạt động đáng tin cậy. \\
\hline
\end{longtable}

\subsubsection{Use Case UC-MD10-11: Cấu hình Tích hợp Cổng Thanh toán}
\begin{longtable}{|m{4cm}|p{11cm}|}
\caption{Đặc tả Use Case UC-MD10-11: Cấu hình Tích hợp Cổng Thanh toán} \label{tab:uc_md10_11_full_v2_latex_fixed_in_codeblock} \\
\hline
\multicolumn{2}{|c|}{\textbf{2.1. Tóm tắt (Summary)}} \\
\hline
\textbf{Mục} & \textbf{Nội dung} \\
\hline
\endhead % Header cho các trang tiếp theo
\midrule
\endfoot % Footer cho bảng
\bottomrule
\endlastfoot % Footer cho trang cuối cùng
Use Case Name & Cấu hình Tích hợp Cổng Thanh toán \\
\hline
Use Case ID & UC-MD10-11 \\
\hline
Use Case Description & Cho phép Quản trị viên hệ thống (US-10) hoặc Quản lý nhà hàng (US-01) thiết lập và quản lý thông tin cấu hình để kết nối hệ thống với một hoặc nhiều Cổng thanh toán trực tuyến (Payment Acquirers) bên ngoài (ví dụ: Stripe, PayPal, VNPay, MoMo...). Cấu hình này cần thiết để khách hàng có thể thanh toán tiền đặt cọc online (UC-MD03-04). \\
\hline
Actor & US-10 (Quản trị viên Hệ thống), US-01 (Quản lý nhà hàng) \\
\hline
Priority & Must Have (Nếu có yêu cầu thanh toán đặt cọc online) \\
\hline
Trigger & - Khi nhà hàng quyết định sử dụng một cổng thanh toán trực tuyến mới để nhận tiền đặt cọc. \newline - Khi cần cập nhật thông tin cấu hình (ví dụ: API Keys, Secret Keys) của một cổng thanh toán đã tích hợp. \newline - Khi cần kích hoạt hoặc vô hiệu hóa một cổng thanh toán nào đó. \\
\hline
Pre-Condition & - Người dùng (US-10 hoặc US-01 có quyền) đã đăng nhập vào hệ thống với quyền quản trị cài đặt Kế toán/Thanh toán hoặc cài đặt Website/eCommerce (nơi thường quản lý các cổng thanh toán). \newline - Nhà hàng đã đăng ký và có tài khoản với nhà cung cấp dịch vụ cổng thanh toán và đã nhận được các thông tin API cần thiết (API Key, Secret Key, Merchant ID...). \\
\hline
Post-Condition & - Thông tin cấu hình cho (các) cổng thanh toán trực tuyến được lưu trữ an toàn và chính xác trong hệ thống. \newline - Nếu cổng thanh toán được kích hoạt và cấu hình đúng, khách hàng sẽ có thể chọn và sử dụng cổng thanh toán đó để thực hiện giao dịch thanh toán tiền đặt cọc trên giao diện đặt chỗ online. \\
\hline
\multicolumn{2}{|c|}{\textbf{2.2. Luồng thực thi (Flow)}} \\
\hline
\textbf{Mục} & \textbf{Nội dung} \\
\hline
Basic Flow & 1. US-10/US-01 truy cập vào mục "Cài đặt" (Settings) hoặc module "Kế toán" (Accounting) hoặc "Website" / "eCommerce" (tùy theo nơi hệ thôngs quản lý Payment Acquirers). \newline 2. US-10/US-01 tìm đến phần "Cổng thanh toán" (Payment Acquirers) hoặc "Phương thức thanh toán trực tuyến". \newline 3. Hệ thống hiển thị danh sách các cổng thanh toán đã được cài đặt hoặc hỗ trợ. \newline 4. US-10/US-01 chọn một cổng thanh toán muốn cấu hình (ví dụ: "Stripe", "VNPay") và chọn "Cấu hình" (Configure) hoặc "Sửa" (Edit). (Nếu cổng chưa có, có thể cần "Cài đặt" - Install - cổng đó trước nếu là một module riêng). \newline 5. Hệ thống hiển thị form cấu hình riêng cho cổng thanh toán đã chọn. Các trường thông tin sẽ khác nhau tùy thuộc vào từng cổng, nhưng thường bao gồm: \newline    - \textbf{Tên hiển thị cho khách hàng} (Displayed as). \newline    - \textbf{Trạng thái (State):} "Disabled", "Enabled in Test Mode", "Enabled in Production Mode". \newline    - \textbf{API Keys/Credentials:} Các trường để nhập API Key, Secret Key, Merchant ID, Public Key... do cổng thanh toán cung cấp. \newline    - (Tùy chọn) \textbf{Callback/Webhook URL:} URL mà cổng thanh toán sẽ gửi thông báo kết quả giao dịch. \newline    - (Tùy chọn) Các cài đặt khác như loại thẻ chấp nhận, quốc gia áp dụng... \newline 6. US-10/US-01 nhập hoặc cập nhật các thông tin cấu hình chính xác. \newline 7. (Rất quan trọng) US-10/US-01 chọn đúng Trạng thái hoạt động (ví dụ: "Enabled in Production Mode" khi muốn chạy thật). \newline 8. US-10/US-01 chọn "Lưu" (Save) để áp dụng cấu hình. \newline 9. Hệ thống lưu lại cấu hình cho cổng thanh toán. \\
\hline
Alternative Flow & \textbf{4a. Cài đặt cổng thanh toán mới (nếu là module):} \newline    1. Nếu cổng thanh toán mong muốn là một module cần cài đặt, US-10/US-01 vào mục Apps, tìm và cài đặt module đó trước. Sau đó mới thực hiện cấu hình. \\
\hline
Exception Flow & \textbf{6a. Thiếu thông tin API Key/Credentials bắt buộc:} \newline    1. US-10/US-01 cố gắng lưu cấu hình mà chưa nhập đủ các thông tin API Key cần thiết. \newline    2. Hệ thống báo lỗi yêu cầu nhập đủ. \newline \textbf{9a. Lỗi hệ thống khi lưu cấu hình cổng thanh toán:} \newline    1. Hệ thống gặp lỗi kỹ thuật khi lưu. \newline    2. Hệ thống báo lỗi chung. \\
\hline
\multicolumn{2}{|c|}{\textbf{2.3. Thông tin bổ sung (Additional Information)}} \\
\hline
\textbf{Mục} & \textbf{Nội dung} \\
\hline
Business Rule & - \textbf{BR-UC10.11-1 (V2):} Thông tin API Keys/Credentials phải chính xác và tương ứng với môi trường (Test hoặc Production) của cổng thanh toán. \newline - \textbf{BR-UC10.11-2 (V2):} Cổng thanh toán phải được đặt ở trạng thái "Enabled in Production Mode" thì khách hàng mới có thể sử dụng để thanh toán thật. Chế độ "Test Mode" dùng để thử nghiệm. \newline - \textbf{BR-UC10.11-3 (V2 - System):} Hệ thống phải xử lý chính xác các Callback/Webhook từ cổng thanh toán để cập nhật trạng thái giao dịch (thành công/thất bại). \\
\hline
Non-Functional Requirement & - \textbf{NFR-UC10.11-1 (V2 - Security):} API Keys, Secret Keys của cổng thanh toán là thông tin cực kỳ nhạy cảm, phải được hệ thống lưu trữ mã hóa và không hiển thị đầy đủ sau khi đã lưu. Quyền truy cập cấu hình này phải được giới hạn nghiêm ngặt. \newline - \textbf{NFR-UC10.11-2 (V2 - Usability):} Giao diện cấu hình cho từng cổng thanh toán nên rõ ràng, hướng dẫn người dùng nhập đúng các thông tin cần thiết. \newline - \textbf{NFR-UC10.11-3 (V2 - Reliability):} Tích hợp với cổng thanh toán phải ổn định và xử lý giao dịch một cách đáng tin cậy. \\
\hline
\end{longtable}

\subsubsection{Use Case UC-MD10-12: Cấu hình Tích hợp Dịch vụ Bot Call}
\begin{longtable}{|m{4cm}|p{11cm}|}
\caption{Đặc tả Use Case UC-MD10-12: Cấu hình Tích hợp Dịch vụ Bot Call} \label{tab:uc_md10_12_full_v2_latex_fixed_in_codeblock} \\
\hline
\multicolumn{2}{|c|}{\textbf{2.1. Tóm tắt (Summary)}} \\
\hline
\textbf{Mục} & \textbf{Nội dung} \\
\hline
\endhead % Header cho các trang tiếp theo
\midrule
\endfoot % Footer cho bảng
\bottomrule
\endlastfoot % Footer cho trang cuối cùng
Use Case Name & Cấu hình Tích hợp Dịch vụ Bot Call \\
\hline
Use Case ID & UC-MD10-12 \\
\hline
Use Case Description & Cho phép Quản trị viên hệ thống (US-10) hoặc Quản lý nhà hàng (US-01) thiết lập các tham số cần thiết để kết nối hệ thống với một dịch vụ Bot Call bên ngoài. Cấu hình này bao gồm thông tin API, và các tham số vận hành như kịch bản thoại sẽ sử dụng, số ngày gọi trước. (Một phần của FR-MD04-05). \\
\hline
Actor & US-10 (Quản trị viên Hệ thống), US-01 (Quản lý nhà hàng) \\
\hline
Priority & Must Have (Để chức năng Bot Call tự động hoạt động) \\
\hline
Trigger & - Khi nhà hàng quyết định triển khai chức năng gọi bot tự động xác nhận đặt chỗ. \newline - Khi cần thay đổi nhà cung cấp dịch vụ Bot Call hoặc cập nhật thông tin kết nối API. \newline - Khi cần điều chỉnh các tham số vận hành của Bot Call. \\
\hline
Pre-Condition & - Người dùng (US-10 hoặc US-01 có quyền) đã đăng nhập vào với quyền quản trị cài đặt. \newline - Nhà hàng đã chọn và đăng ký tài khoản với một nhà cung cấp dịch vụ Bot Call bên ngoài, và có được các thông tin cần thiết (API Key, API Endpoint, ID Kịch bản thoại...). \\
\hline
Post-Condition & - Các tham số cấu hình cho việc tích hợp và vận hành dịch vụ Bot Call được lưu trữ trong hệ thống. \newline - Hệ thống có đủ thông tin để có thể gửi yêu cầu thực hiện cuộc gọi đến dịch vụ Bot Call (UC-MD04-01) và để dịch vụ Bot Call có thể gọi lại webhook của hệ thống. \\
\hline
\multicolumn{2}{|c|}{\textbf{2.2. Luồng thực thi (Flow)}} \\
\hline
\textbf{Mục} & \textbf{Nội dung} \\
\hline
Basic Flow & 1. US-10/US-01 truy cập vào khu vực "Cài đặt" (Settings) của hệ thống hoặc của module Đặt chỗ/Tích hợp. \newline 2. US-10/US-01 tìm đến phần cấu hình có tên liên quan đến "Xác nhận Đặt chỗ bằng Bot Call" (Bot Call Confirmation Settings) hoặc "Tích hợp Voice Bot". \newline 3. Hệ thống hiển thị form cấu hình với các trường cần thiết. Các trường này đã được liệt kê trong UC-MD04-05 (Basic Flow, bước 3), bao gồm: \newline    - Kích hoạt/Tắt chức năng Bot Call. \newline    - Thông tin API của Nhà cung cấp Bot Call (Endpoint, API Key/Token). \newline    - ID hoặc tên của Kịch bản thoại sẽ được Bot sử dụng. \newline    - Số ngày N sẽ gọi trước ngày khách đặt bàn. \newline    - Số điện thoại của nhà hàng sẽ nhận cuộc gọi khi khách bấm phím "Hỗ trợ". \newline 4. US-10/US-01 nhập hoặc cập nhật các giá trị cấu hình này một cách chính xác. \newline 5. (Tùy chọn) US-10/US-01 có thể sử dụng nút "Kiểm tra Kết nối API" (nếu có) để xác thực thông tin API với dịch vụ Bot Call. \newline 6. US-10/US-01 chọn hành động "Lưu" (Save) để áp dụng các thay đổi cấu hình. \newline 7. Hệ thống kiểm tra tính hợp lệ cơ bản của dữ liệu (ví dụ: N là số dương, API Key không trống nếu kích hoạt). \newline 8. Hệ thống lưu lại các thông tin cấu hình mới. \newline 9. Hệ thống hiển thị thông báo "Cấu hình Bot Call đã được lưu thành công." \\
\hline
Alternative Flow & \textbf{3a. Cấu hình kịch bản thoại trực tiếp trong hệ thống:} \newline    1. Thay vì chỉ nhập ID kịch bản, hệ thống có thể cung cấp một giao diện cho phép US-10/US-01 tự soạn thảo hoặc tùy chỉnh nội dung văn bản của kịch bản thoại (sẽ được Bot sử dụng qua text-to-speech) và định nghĩa các hành động tương ứng với phím bấm của khách. \\
\hline
Exception Flow & \textbf{7a. Lỗi xác thực dữ liệu khi lưu:} \newline    1. Hệ thống phát hiện giá trị cấu hình không hợp lệ (ví dụ: Số ngày N không phải số, thiếu API Key khi đang kích hoạt). \newline    2. Hệ thống hiển thị thông báo lỗi. Cấu hình không được lưu. \newline \textbf{8a. Lỗi hệ thống khi lưu cấu hình:} \newline    1. Hệ thống gặp lỗi kỹ thuật khi lưu. \newline    2. Hệ thống báo lỗi chung. \\
\hline
\multicolumn{2}{|c|}{\textbf{2.3. Thông tin bổ sung (Additional Information)}} \\
\hline
\textbf{Mục} & \textbf{Nội dung} \\
\hline
Business Rule & - \textbf{BR-UC10.12-1 (V2):} Thông tin API Key/Credentials của dịch vụ Bot Call phải được cung cấp chính xác để hệ thống có thể gửi yêu cầu gọi đi. \newline - \textbf{BR-UC10.12-2 (V2):} ID Kịch bản thoại phải khớp với một kịch bản đã được tạo và cấu hình trên nền tảng của nhà cung cấp dịch vụ Bot Call (trừ khi kịch bản được cấu hình trực tiếp trong hệ thống). \newline - \textbf{BR-UC10.12-3 (V2):} Số điện thoại Hỗ trợ phải là số điện thoại thực tế có người trực để tiếp nhận cuộc gọi từ khách hàng khi họ cần hỗ trợ. \\
\hline
Non-Functional Requirement & - \textbf{NFR-UC10.12-1 (V2 - Security):} Thông tin API Key/Token của dịch vụ Bot Call cần được lưu trữ an toàn. \newline - \textbf{NFR-UC10.12-2 (V2 - Usability):} Giao diện cấu hình Bot Call nên rõ ràng, các trường dễ hiểu. Nếu có tùy chọn kiểm tra kết nối API thì rất hữu ích. \newline - \textbf{NFR-UC10.12-3 (V2 - Flexibility):} Hệ thống nên được thiết kế để có thể dễ dàng thay đổi hoặc cập nhật thông tin cấu hình khi cần (ví dụ: đổi nhà cung cấp Bot Call, cập nhật kịch bản). \\
\hline
\end{longtable}

\subsubsection{Use Case UC-MD10-13: Cấu hình Tích hợp Dịch vụ Giao hàng (Shipday)}
\begin{longtable}{|m{4cm}|p{11cm}|}
\caption{Đặc tả Use Case UC-MD10-13: Cấu hình Tích hợp Dịch vụ Giao hàng (Shipday)} \label{tab:uc_md10_13_full_v2_latex_fixed_in_codeblock} \\
\hline
\multicolumn{2}{|c|}{\textbf{2.1. Tóm tắt (Summary)}} \\
\hline
\textbf{Mục} & \textbf{Nội dung} \\
\hline
\endhead % Header cho các trang tiếp theo
\midrule
\endfoot % Footer cho bảng
\bottomrule
\endlastfoot % Footer cho trang cuối cùng
Use Case Name & Cấu hình Tích hợp Dịch vụ Giao hàng (Shipday) \\
\hline
Use Case ID & UC-MD10-13 \\
\hline
Use Case Description & Cho phép Quản trị viên hệ thống (US-10) hoặc Quản lý nhà hàng (US-01) thiết lập các tham số cần thiết để kết nối và trao đổi dữ liệu giữa hệ thống và nền tảng quản lý giao hàng Shipday. Điều này bao gồm việc nhập thông tin xác thực API (API Key) và các cài đặt liên quan khác để đảm bảo luồng gửi đơn hàng và nhận cập nhật trạng thái hoạt động chính xác. (Tương ứng FR-MD07-15). \\
\hline
Actor & US-10 (Quản trị viên Hệ thống), US-01 (Quản lý nhà hàng) \\
\hline
Priority & Must Have (Để chức năng giao hàng qua Shipday hoạt động) \\
\hline
Trigger & - Khi nhà hàng bắt đầu triển khai tích hợp với Shipday để quản lý đơn hàng giao đi. \newline - Khi cần cập nhật thông tin API Key của Shipday (ví dụ: do API Key cũ hết hạn hoặc được Shipday cấp mới). \newline - Khi cần thay đổi các cài đặt khác liên quan đến việc đồng bộ dữ liệu giữa hai hệ thống. \\
\hline
Pre-Condition & - Người dùng (US-10 hoặc US-01 có quyền) đã đăng nhập với quyền quản trị cài đặt. \newline - Nhà hàng đã có tài khoản Shipday và đã lấy được API Key từ trang quản trị Shipday của mình. \\
\hline
Post-Condition & - Thông tin cấu hình tích hợp với Shipday (đặc biệt là API Key) được lưu trữ an toàn trong hệ thống. \newline - Hệ thống sẵn sàng để gửi thông tin đơn hàng giao đi sang Shipday (UC-MD07-08) và nhận lại các cập nhật trạng thái giao hàng từ Shipday (thông qua webhook đã được cấu hình ở phía Shipday để trỏ về - UC-MD07-09). \\
\hline
\multicolumn{2}{|c|}{\textbf{2.2. Luồng thực thi (Flow)}} \\
\hline
\textbf{Mục} & \textbf{Nội dung} \\
\hline
Basic Flow & 1. US-10/US-01 truy cập vào khu vực "Cài đặt" (Settings) của hệ thống, có thể trong mục cài đặt chung hoặc cài đặt của module Giao hàng (Delivery) hoặc Tích hợp (Integrations). \newline 2. US-10/US-01 tìm đến phần cấu hình có tên "Shipday Integration", "Delivery Service Configuration" hoặc tương tự. \newline 3. Hệ thống hiển thị form cấu hình tích hợp Shipday. Các trường chính thường bao gồm: \newline    - Ô kiểm "Kích hoạt Tích hợp Shipday" (Enable Shipday Integration). \newline    - Trường nhập "Shipday API Key". \newline    - (Có thể có) URL API Endpoint của Shipday (thường cố định và có thể điền sẵn). \newline    - (Hiển thị) URL của Webhook Endpoint mà người dùng cần sao chép để cấu hình trên Shipday. \newline    - (Tùy chọn) Các cài đặt về ánh xạ trường dữ liệu, tự động gửi đơn, v.v. \newline 4. US-10/US-01 đánh dấu "Kích hoạt" và nhập chính xác API Key của Shipday. \newline 5. US-10/US-01 sao chép Webhook URL của hệ thống (nếu được hiển thị) để thực hiện cấu hình tương ứng trên tài khoản Shipday. \newline 6. (Tùy chọn) US-10/US-01 có thể nhấn nút "Kiểm tra Kết nối API" (nếu có) để xác thực API Key với Shipday. \newline 7. US-10/US-01 chọn "Lưu" (Save) trên form cấu hình. \newline 8. Hệ thống kiểm tra và lưu lại cấu hình. \newline 9. Hệ thống báo "Cấu hình Shipday đã được lưu." \\
\hline
Alternative Flow & \textbf{6a. Kết quả Kiểm tra Kết nối API:} \newline    1. Nếu kiểm tra thành công, hệ thống báo "Kết nối thành công." \newline    2. Nếu thất bại, hệ thống báo "Kết nối thất bại: [Lý do]" và US-10/US-01 cần sửa lại API Key hoặc kiểm tra kết nối mạng. \\
\hline
Exception Flow & \textbf{8a. Lỗi xác thực/lưu cấu hình:} \newline    1. Thiếu API Key khi đã kích hoạt, hoặc lỗi hệ thống khi lưu. \newline    2. Hệ thống báo lỗi. Cấu hình không được lưu. \\
\hline
\multicolumn{2}{|c|}{\textbf{2.3. Thông tin bổ sung (Additional Information)}} \\
\hline
\textbf{Mục} & \textbf{Nội dung} \\
\hline
Business Rule & - \textbf{BR-UC10.13-1 (V2):} API Key của Shipday phải chính xác và còn hiệu lực. \newline - \textbf{BR-UC10.13-2 (V2):} Webhook URL phải được cấu hình đúng trên Shipday để hệ thống nhận được cập nhật trạng thái đơn hàng. \newline - \textbf{BR-UC10.13-3 (V2):} Việc ánh xạ các trường dữ liệu (ví dụ: các thành phần địa chỉ) giữa hệ thống và Shipday phải chính xác. \\
\hline
Non-Functional Requirement & - \textbf{NFR-UC10.13-1 (V2 - Security):} API Key của Shipday phải được lưu trữ an toàn trong hệ thống. \newline - \textbf{NFR-UC10.13-2 (V2 - Usability):} Giao diện cấu hình phải rõ ràng. Tính năng kiểm tra kết nối là hữu ích. \\
\hline
\end{longtable}

\subsubsection{Use Case UC-MD10-14: Cấu hình Tham số Nghiệp vụ Đặc thù cho Đặt chỗ}
\begin{longtable}{|m{4cm}|p{11cm}|}
\caption{Đặc tả Use Case UC-MD10-14: Cấu hình Tham số Nghiệp vụ Đặc thù cho Đặt chỗ} \label{tab:uc_md10_14_full_v2_latex_fixed_in_codeblock} \\
\hline
\multicolumn{2}{|c|}{\textbf{2.1. Tóm tắt (Summary)}} \\
\hline
\textbf{Mục} & \textbf{Nội dung} \\
\hline
\endhead % Header cho các trang tiếp theo
\midrule
\endfoot % Footer cho bảng
\bottomrule
\endlastfoot % Footer cho trang cuối cùng
Use Case Name & Cấu hình Tham số Nghiệp vụ Đặc thù cho Đặt chỗ \\
\hline
Use Case ID & UC-MD10-14 \\
\hline
Use Case Description & Cho phép Quản trị viên hệ thống (US-10) hoặc Quản lý nhà hàng (US-01) thiết lập và tùy chỉnh các tham số, quy tắc kinh doanh riêng của nhà hàng liên quan trực tiếp đến chức năng Đặt chỗ (Reservations / Bookings). Điều này bao gồm các tham số như tỷ lệ phần trăm đặt cọc cho bàn, tỷ lệ phần trăm đặt cọc cho món ăn đặt trước, giá trị tham chiếu của từng bàn (để tính cọc), và số ngày Bot Call sẽ tự động gọi xác nhận trước ngày khách đặt. (Tương ứng FR-MD03-15 và một phần FR-MD04-05). \\
\hline
Actor & US-10 (Quản trị viên Hệ thống), US-01 (Quản lý nhà hàng) \\
\hline
Priority & Must Have (Để chức năng đặt chỗ và các tính năng liên quan hoạt động đúng theo chính sách nhà hàng) \\
\hline
Trigger & - Khi cần thiết lập các quy tắc kinh doanh ban đầu cho hệ thống đặt chỗ của nhà hàng. \newline - Khi nhà hàng có sự thay đổi về chính sách đặt cọc, cách tính giá trị bàn cho việc cọc, hoặc quy trình gọi bot xác nhận. \\
\hline
Pre-Condition & - Người dùng (US-10 hoặc US-01 có quyền) đã đăng nhập vào hệ thống với quyền quản trị cấu hình của module Đặt chỗ (Reservations) hoặc một module cài đặt tùy chỉnh liên quan đến các tham số này. \\
\hline
Post-Condition & - Các quy tắc và tham số nghiệp vụ đặc thù cho chức năng đặt chỗ của nhà hàng được cập nhật và lưu trữ trong cấu hình hệ thống. \newline - Các module và chức năng khác của hệ thống (ví dụ: module Đặt chỗ MD-03 khi khách hàng đặt online, module Bot Call MD-04 khi lên lịch gọi) sẽ đọc và hoạt động dựa trên các tham số mới này. \\
\hline
\multicolumn{2}{|c|}{\textbf{2.2. Luồng thực thi (Flow)}} \\
\hline
\textbf{Mục} & \textbf{Nội dung} \\
\hline
Basic Flow & 1. Người dùng (US-10 hoặc US-01) truy cập vào khu vực "Cài đặt" (Settings) của module Đặt chỗ (Reservations) hoặc một khu vực cấu hình nghiệp vụ tùy chỉnh riêng của nhà hàng (nếu có). \newline 2. Hệ thống hiển thị một form hoặc một nhóm các trường cấu hình cho phép người dùng nhập hoặc thay đổi các tham số nghiệp vụ đặc thù liên quan đến đặt chỗ. Các trường này có thể bao gồm: \newline    a. \textbf{Tỷ lệ Đặt cọc cho Bàn (\%):} Trường nhập số, cho phép nhập giá trị phần trăm (ví dụ: 15 cho 15\%). \newline    b. \textbf{Tỷ lệ Đặt cọc cho Món ăn Đặt trước (\%):} Trường nhập số, cho phép nhập giá trị phần trăm (ví dụ: 15 cho 15\%). \newline    c. \textbf{Số ngày Bot Call gọi xác nhận trước:} Trường nhập số nguyên dương (ví dụ: 1, 2, 3 ngày). \newline    d. (Có thể có) Các trường để thiết lập giá trị tham chiếu cho từng loại bàn hoặc từng bàn cụ thể (Table Value for Deposit Calculation). Việc nhập giá trị cụ thể cho từng bàn có thể được thực hiện ở một giao diện quản lý tài nguyên bàn riêng biệt, và ở đây chỉ là bật/tắt hoặc chọn phương thức tính. \newline    e. (Có thể có) Các cấu hình khác liên quan đến chính sách hủy đặt chỗ, điều kiện hoàn cọc, thời gian tối thiểu/tối đa cho phép đặt trước, giới hạn số lượng khách mỗi đặt chỗ online, v.v. (Các cấu hình này cũng đã được đề cập trong UC-MD03-15 nhưng có thể được quản lý tập trung hơn ở đây nếu thiết kế module theo hướng đó). \newline 3. US-10/US-01 nhập hoặc cập nhật các giá trị mong muốn cho các tham số này theo đúng chính sách của nhà hàng. \newline 4. Sau khi hoàn tất việc nhập liệu, US-10/US-01 chọn hành động "Lưu" (Save) trên giao diện cấu hình. \newline 5. Hệ thống kiểm tra tính hợp lệ của các dữ liệu đã nhập (ví dụ: tỷ lệ phần trăm phải là số, số ngày phải là số nguyên dương, các giá trị nằm trong khoảng cho phép nếu có). \newline 6. Nếu dữ liệu hợp lệ, hệ thống lưu lại các thông tin cấu hình mới này. \newline 7. Hệ thống hiển thị thông báo "Các tham số nghiệp vụ đã được lưu thành công." \\
\hline
Alternative Flow & \textbf{2f. Quản lý Giá trị Bàn ở một giao diện riêng:} \newline    1. Nếu việc thiết lập giá trị cụ thể cho từng bàn (để tính cọc) được thực hiện ở một menu hoặc giao diện khác (ví dụ: trong quản lý Sơ đồ tầng hoặc quản lý Tài sản/Bàn). \newline    2. Trong trường hợp đó, Use Case này chỉ tập trung vào việc thiết lập các tỷ lệ phần trăm và các tham số chung khác. Việc nhập giá trị cho từng bàn sẽ là một Use Case riêng (có thể thuộc MD-02 hoặc một phần của cấu hình POS). \\
\hline
Exception Flow & \textbf{5a. Lỗi Xác thực Dữ liệu khi Lưu:} \newline    1. Hệ thống phát hiện một hoặc nhiều giá trị cấu hình mà người dùng nhập không hợp lệ (ví dụ: tỷ lệ đặt cọc không phải là số, số ngày gọi bot là số âm). \newline    2. Hệ thống hiển thị thông báo lỗi cụ thể, chỉ rõ trường hoặc giá trị không hợp lệ. \newline    3. Hệ thống không lưu các thay đổi. US-10/US-01 cần sửa lại các thông tin không hợp lệ. \newline \textbf{6a. Lỗi Hệ thống trong quá trình Lưu Cấu hình:} \newline    1. Hệ thống gặp lỗi kỹ thuật khi cố gắng lưu các tham số cấu hình. \newline    2. Hệ thống hiển thị một thông báo lỗi chung. Các thay đổi có thể không được lưu. \\
\hline
\multicolumn{2}{|c|}{\textbf{2.3. Thông tin bổ sung (Additional Information)}} \\
\hline
\textbf{Mục} & \textbf{Nội dung} \\
\hline
Business Rule & - \textbf{BR-UC10.14-1 (V2):} Các tham số nghiệp vụ được cấu hình ở đây (như tỷ lệ đặt cọc, giá trị bàn, số ngày gọi bot) sẽ ảnh hưởng trực tiếp đến logic hoạt động của các module khác trong hệ thống, bao gồm: \newline    - Module Đặt chỗ (MD-03): Cách hệ thống tự động tính toán tiền đặt cọc (UC-MD03-04). \newline    - Module Xác nhận qua Bot (MD-04): Thời điểm hệ thống lên lịch và kích hoạt cuộc gọi xác nhận (UC-MD04-01). \newline - \textbf{BR-UC10.14-2 (V2):} Các giá trị cấu hình phải được nhập một cách chính xác để phản ánh đúng chính sách kinh doanh của nhà hàng. Sai sót trong cấu hình có thể dẫn đến tính toán sai tiền đặt cọc hoặc quy trình xác nhận không hiệu quả. \\
\hline
Non-Functional Requirement & - \textbf{NFR-UC10.14-1 (V2 - Usability):} Giao diện cấu hình các tham số nghiệp vụ đặc thù này nên được nhóm lại một cách logic, dễ dàng cho Quản trị viên hoặc Quản lý tìm thấy và hiểu rõ ý nghĩa của từng tham số. Nên có các gợi ý hoặc giải thích ngắn (tooltip) cho mỗi trường cấu hình. \newline - \textbf{NFR-UC10.14-2 (V2 - Flexibility):} Hệ thống nên cho phép dễ dàng thay đổi và cập nhật các tham số này khi chính sách kinh doanh của nhà hàng có sự điều chỉnh, mà không cần can thiệp vào mã nguồn. \\
\hline
\end{longtable}

\subsubsection{Use Case UC-MD10-15: Xem Nhật ký Hoạt động Hệ thống (Logs)}
\begin{longtable}{|m{4cm}|p{11cm}|}
\caption{Đặc tả Use Case UC-MD10-15: Xem Nhật ký Hoạt động Hệ thống (Logs)} \label{tab:uc_md10_15_full_v2_latex_fixed_in_codeblock} \\
\hline
\multicolumn{2}{|c|}{\textbf{2.1. Tóm tắt (Summary)}} \\
\hline
\textbf{Mục} & \textbf{Nội dung} \\
\hline
\endhead % Header cho các trang tiếp theo
\midrule
\endfoot % Footer cho bảng
\bottomrule
\endlastfoot % Footer cho trang cuối cùng
Use Case Name & Xem Nhật ký Hoạt động Hệ thống (Logs) \\
\hline
Use Case ID & UC-MD10-15 \\
\hline
Use Case Description & Cho phép Quản trị viên hệ thống (US-10) truy cập và xem xét các bản ghi nhật ký (logs) do hệ thống tự động tạo ra trong quá trình hoạt động. Các log này có thể bao gồm thông tin về các lỗi kỹ thuật đã xảy ra, các cảnh báo hệ thống, và có thể cả các hoạt động quan trọng của người dùng (nếu tính năng audit log được kích hoạt và cấu hình). Mục đích là để theo dõi tình trạng hệ thống, chẩn đoán nguyên nhân sự cố và hỗ trợ khắc phục lỗi. \\
\hline
Actor & US-10 (Quản trị viên Hệ thống) \\
\hline
Priority & Should Have (Rất quan trọng cho việc vận hành, bảo trì và khắc phục sự cố hệ thống) \\
\hline
Trigger & - Khi hệ thống gặp một lỗi hoặc hoạt động không như mong đợi, US-10 cần điều tra nguyên nhân. \newline - Khi cần theo dõi hoạt động của một tính năng cụ thể hoặc kiểm tra các sự kiện hệ thống đã xảy ra. \newline - Thực hiện kiểm tra định kỳ tình trạng hoạt động và các vấn đề tiềm ẩn của hệ thống. \\
\hline
Pre-Condition & - US-10 đã đăng nhập vào hệ thống với quyền quản trị hệ thống cao nhất. \newline - Hệ thống đang hoạt động và đã được cấu hình để ghi nhận nhật ký ở một mức độ chi tiết nhất định (log level). \\
\hline
Post-Condition & - Quản trị viên xem được danh sách các bản ghi nhật ký hệ thống, được sắp xếp theo thời gian. \newline - Quản trị viên có thể sử dụng các công cụ lọc, tìm kiếm để tìm các log cụ thể và xem chi tiết nội dung của từng bản ghi log, bao gồm cả thông điệp lỗi và (nếu có) dấu vết ngăn xếp (stack trace) để phục vụ việc chẩn đoán sự cố. \\
\hline
\multicolumn{2}{|c|}{\textbf{2.2. Luồng thực thi (Flow)}} \\
\hline
\textbf{Mục} & \textbf{Nội dung} \\
\hline
Basic Flow (Xem log qua giao diện, nếu có) & 1. US-10 (thường cần kích hoạt Developer Mode) truy cập vào khu vực kỹ thuật của hệ thống trong menu "Cài đặt" (Settings). \newline 2. US-10 tìm đến mục "Nhật ký" (Logging), "System Logs", hoặc một mục tương tự quản lý các bản ghi `ir.logging` của hệ thống. \newline 3. Hệ thống hiển thị danh sách các bản ghi nhật ký, thường được sắp xếp theo thời gian tạo giảm dần (log mới nhất ở trên cùng). \newline 4. Mỗi bản ghi trong danh sách thường hiển thị các thông tin tóm tắt như: \newline    - Thời gian ghi log (Timestamp). \newline    - Mức độ của log (Log Level: INFO, WARNING, ERROR, CRITICAL, DEBUG...). \newline    - Tên của thành phần/module ghi log (Logger Name). \newline    - Một phần nội dung của thông điệp log (Message). \newline 5. US-10 xem xét danh sách các log. US-10 có thể sử dụng các công cụ lọc (ví dụ: lọc theo Mức độ log để chỉ xem các lỗi ERROR, CRITICAL; lọc theo Logger Name để xem log của một module cụ thể; lọc theo khoảng Thời gian) hoặc sử dụng ô tìm kiếm để nhập từ khóa liên quan đến vấn đề đang điều tra. \newline 6. US-10 nhấp vào một bản ghi log cụ thể để xem thông tin chi tiết đầy đủ của nó, bao gồm toàn bộ thông điệp log và, nếu đó là một log lỗi, có thể cả thông tin dấu vết ngăn xếp (stack trace) chi tiết để giúp xác định vị trí gây lỗi trong mã nguồn. \\
\hline
Alternative Flow & \textbf{Basic Flow (Xem file log trực tiếp trên máy chủ):} \newline    1. Nếu hệ thống được triển khai trên một máy chủ mà US-10 có quyền truy cập. \newline    2. US-10 sử dụng các công cụ như SSH để đăng nhập vào máy chủ hệ thống. \newline    3. US-10 điều hướng đến thư mục chứa file log của hệ thống (vị trí file log này được xác định trong file cấu hình của hệ thống, ví dụ: `abc.conf`, và tên file thường là `abc.log` hoặc tương tự). \newline    4. US-10 sử dụng các công cụ dòng lệnh của hệ điều hành (ví dụ: `tail -f` để xem log theo thời gian thực, `grep` để tìm kiếm từ khóa, `less` hoặc `cat` để xem nội dung file) hoặc mở file log bằng một trình soạn thảo văn bản để xem, lọc và tìm kiếm nội dung log. \newline \textbf{5a. Xuất log:} \newline    1. Giao diện xem log của hệ thống có thể cung cấp chức năng cho phép US-10 xuất một phần hoặc toàn bộ các log đang hiển thị ra một tệp (ví dụ: CSV, TXT) để lưu trữ hoặc phân tích ngoại tuyến. \\
\hline
Exception Flow & \textbf{3a. Lỗi hệ thống khi tải hoặc hiển thị log qua giao diện:} \newline    1. Hệ thống gặp lỗi kỹ thuật (ví dụ: do lượng log quá lớn không thể hiển thị hết qua giao diện, lỗi truy vấn bảng log) khi US-10 cố gắng xem log. \newline    2. Hệ thống hiển thị một thông báo lỗi chung. Việc xem log qua giao diện có thể bị gián đoạn hoặc không thực hiện được. Trong trường hợp này, US-10 có thể cần phải chuyển sang xem file log trực tiếp trên máy chủ (Alternative Flow). \newline \textbf{Alternative Flow - Step 2a. Không có quyền truy cập máy chủ hoặc file log:} \newline    1. US-10 không có thông tin đăng nhập hoặc không được cấp quyền truy cập vào máy chủ hệ thống, hoặc file log của hệ thống bị lỗi, bị xóa, hoặc không được cấu hình để ghi. \newline    2. US-10 không thể xem được log theo phương pháp này. \\
\hline
\multicolumn{2}{|c|}{\textbf{2.3. Thông tin bổ sung (Additional Information)}} \\
\hline
\textbf{Mục} & \textbf{Nội dung} \\
\hline
Business Rule & - \textbf{BR-UC10.15-1 (V2 - System):} Hệ thống cần được cấu hình để ghi nhận nhật ký ở một mức độ chi tiết phù hợp (log level). Ví dụ: trong môi trường phát triển hoặc thử nghiệm (development/staging), log level có thể được đặt là INFO hoặc DEBUG để có nhiều thông tin. Trong môi trường sản phẩm thực tế (production), log level thường được đặt là WARNING hoặc ERROR để giảm dung lượng log và tập trung vào các vấn đề quan trọng. \newline - \textbf{BR-UC10.15-2 (V2 - System):} Các bản ghi log lỗi (đặc biệt là ERROR, CRITICAL) cần cung cấp đủ thông tin, bao gồm cả dấu vết ngăn xếp (stack trace) nếu có, để các nhà phát triển hoặc quản trị viên có thể dễ dàng xác định nguyên nhân và vị trí gây ra sự cố trong mã nguồn. \newline - \textbf{BR-UC10.15-3:} Cần có chính sách và cơ chế quản lý vòng đời của các file log trên máy chủ (ví dụ: xoay vòng log - log rotation, nén log cũ, giới hạn dung lượng tối đa) để tránh việc file log phát triển quá lớn, chiếm hết dung lượng đĩa của máy chủ và ảnh hưởng đến hoạt động của hệ thống. \\
\hline
Non-Functional Requirement & - \textbf{NFR-UC10.15-1 (V2 - Security):} Quyền truy cập vào nhật ký hệ thống, đặc biệt là quyền truy cập vào file log trực tiếp trên máy chủ, phải được kiểm soát cực kỳ chặt chẽ và chỉ được cấp cho những Quản trị viên hệ thống có thẩm quyền và trách nhiệm. Log có thể chứa thông tin nhạy cảm. \newline - \textbf{NFR-UC10.15-2 (V2 - Performance):} Quá trình ghi log của hệ thống không được làm ảnh hưởng đáng kể đến hiệu năng chung của ứng dụng. Việc truy vấn và hiển thị log cũng cần được tối ưu để hoạt động hiệu quả, ngay cả khi có lượng lớn bản ghi log. \newline - \textbf{NFR-UC10.15-3 (V2 - Maintainability \& Diagnostics):} Nhật ký hệ thống là một công cụ không thể thiếu cho việc bảo trì, theo dõi hoạt động, chẩn đoán và khắc phục sự cố của hệ thống. Chất lượng và mức độ chi tiết của log ảnh hưởng trực tiếp đến khả năng giải quyết vấn đề. \\
\hline
\end{longtable}

\subsubsection{Use Case UC-MD10-16: Đăng nhập vào Hệ thống}
\begin{longtable}{|m{4cm}|p{11cm}|}
\caption{Đặc tả Use Case UC-MD10-16: Đăng nhập vào Hệ thống} \label{tab:uc_md10_16_login_in_codeblock} \\
\hline
\multicolumn{2}{|c|}{\textbf{2.1. Tóm tắt (Summary)}} \\
\hline
\textbf{Mục} & \textbf{Nội dung} \\
\hline
\endhead % Header cho các trang tiếp theo
\midrule
\endfoot % Footer cho bảng
\bottomrule
\endlastfoot % Footer cho trang cuối cùng
Use Case Name & Đăng nhập vào Hệ thống \\
\hline
Use Case ID & UC-MD10-16 \\
\hline
Use Case Description & Cho phép một Người dùng đã có tài khoản (Nhân viên các cấp, Quản lý, Quản trị viên) cung cấp thông tin xác thực (tên đăng nhập/email và mật khẩu) để truy cập vào các chức năng và dữ liệu của hệ thống theo quyền hạn đã được cấp. \\
\hline
Actor & US-01, US-02, US-03, US-04, US-05, US-06, US-07, US-09, US-10 (Bất kỳ Người dùng nội bộ nào có tài khoản) \\
\hline
Priority & Must Have \\
\hline
Trigger & Người dùng muốn bắt đầu một phiên làm việc và truy cập vào các tài nguyên của hệ thống. \\
\hline
Pre-Condition & - Người dùng đã có một tài khoản hợp lệ trong hệ thống (đã được tạo qua UC-MD10-01). \newline - Người dùng biết Tên đăng nhập (thường là địa chỉ email) và Mật khẩu của mình. \newline - Giao diện trang Đăng nhập của hệ thống đang được hiển thị cho người dùng. \\
\hline
Post-Condition & - \textbf{Thành công:} Người dùng được xác thực thành công. Hệ thống tạo một phiên làm việc (session) cho người dùng. Người dùng được chuyển hướng đến giao diện chính của hệ thống (dashboard hoặc màn hình làm việc mặc định) tương ứng với vai trò và quyền hạn của họ. \newline - \textbf{Thất bại:} Người dùng không được xác thực. Hệ thống hiển thị thông báo lỗi (ví dụ: "Tên đăng nhập hoặc mật khẩu không đúng") và vẫn giữ người dùng ở trang Đăng nhập. \\
\hline
\multicolumn{2}{|c|}{\textbf{2.2. Luồng thực thi (Flow)}} \\
\hline
\textbf{Mục} & \textbf{Nội dung} \\
\hline
Basic Flow & 1. Người dùng mở trình duyệt web và truy cập vào địa chỉ URL của hệ thống. \newline 2. Hệ thống hiển thị trang Đăng nhập, yêu cầu nhập Tên đăng nhập (Email) và Mật khẩu. \newline 3. Người dùng nhập Tên đăng nhập (Email) của mình vào trường tương ứng. \newline 4. Người dùng nhập Mật khẩu của mình vào trường Mật khẩu. \newline 5. Người dùng nhấp vào nút "Đăng nhập" (Login). \newline 6. Hệ thống (System) kiểm tra thông tin Tên đăng nhập và Mật khẩu mà người dùng cung cấp so với dữ liệu đã lưu trong cơ sở dữ liệu người dùng. \newline 7. \textbf{Nếu thông tin xác thực là chính xác và tài khoản đang hoạt động:} \newline    a. Hệ thống tạo một phiên làm việc (session) mới cho người dùng. \newline    b. Hệ thống ghi nhận thời gian đăng nhập. \newline    c. Hệ thống chuyển hướng người dùng đến trang làm việc mặc định của họ (ví dụ: dashboard, màn hình POS, module Đặt chỗ...). \newline 8. \textbf{Nếu thông tin xác thực KHÔNG chính xác hoặc tài khoản đã bị vô hiệu hóa:} \newline    a. Hệ thống hiển thị một thông báo lỗi trên trang Đăng nhập (ví dụ: "Tên đăng nhập hoặc mật khẩu không đúng. Vui lòng thử lại." hoặc "Tài khoản của bạn đã bị khóa."). \newline    b. Hệ thống không cho phép truy cập và giữ nguyên trang Đăng nhập để người dùng thử lại. \\
\hline
Alternative Flow & \textbf{4a. Người dùng chọn tùy chọn "Ghi nhớ tôi" (Remember me):} \newline    1. Nếu giao diện đăng nhập có tùy chọn này, người dùng có thể đánh dấu vào. \newline    2. Nếu đăng nhập thành công, hệ thống có thể lưu một cookie để tự động điền tên đăng nhập hoặc duy trì phiên đăng nhập lâu hơn (tùy cấu hình bảo mật). \newline \textbf{8c. Tài khoản bị khóa tạm thời do nhập sai mật khẩu nhiều lần:} \newline    1. Nếu hệ thống có chính sách khóa tài khoản sau một số lần nhập sai mật khẩu liên tiếp. \newline    2. Sau N lần nhập sai, hệ thống hiển thị thông báo "Tài khoản của bạn đã bị khóa tạm thời. Vui lòng thử lại sau X phút hoặc liên hệ quản trị viên." \\
\hline
Exception Flow & \textbf{6a. Lỗi hệ thống trong quá trình xác thực:} \newline    1. Hệ thống gặp lỗi kỹ thuật (ví dụ: lỗi kết nối cơ sở dữ liệu) khi đang kiểm tra thông tin đăng nhập. \newline    2. Hệ thống hiển thị một thông báo lỗi chung (ví dụ: "Đã xảy ra lỗi. Vui lòng thử lại sau."). \\
\hline
\multicolumn{2}{|c|}{\textbf{2.3. Thông tin bổ sung (Additional Information)}} \\
\hline
\textbf{Mục} & \textbf{Nội dung} \\
\hline
Business Rule & - \textbf{BR-UC10.16-1 (V2):} Mật khẩu phải được lưu trữ trong cơ sở dữ liệu dưới dạng mã hóa (hashed) an toàn, không lưu dạng text thuần. \newline - \textbf{BR-UC10.16-2 (V2):} Hệ thống nên có chính sách về độ mạnh mật khẩu và có thể yêu cầu người dùng thay đổi mật khẩu định kỳ hoặc sau lần đăng nhập đầu tiên (nếu mật khẩu ban đầu do admin cấp). \newline - \textbf{BR-UC10.16-3 (V2):} Cần có cơ chế xử lý trường hợp nhập sai mật khẩu nhiều lần (ví dụ: khóa tài khoản tạm thời, yêu cầu captcha) để chống tấn công brute-force. \\
\hline
Non-Functional Requirement & - \textbf{NFR-UC10.16-1 (V2 - Security):} Quá trình xác thực phải được thực hiện qua kết nối an toàn (HTTPS). Mật khẩu không được truyền đi dưới dạng text thuần. \newline - \textbf{NFR-UC10.16-2 (V2 - Performance):} Thời gian phản hồi của hệ thống khi người dùng nhấn nút "Đăng nhập" phải nhanh chóng (dưới 2-3 giây). \newline - \textbf{NFR-UC10.16-3 (V2 - Usability):} Giao diện đăng nhập phải đơn giản, rõ ràng. Thông báo lỗi phải dễ hiểu và có hướng dẫn cho người dùng. \\
\hline
\end{longtable}

\subsubsection{Use Case UC-MD10-17: Đăng xuất khỏi Hệ thống}
\begin{longtable}{|m{4cm}|p{11cm}|}
\caption{Đặc tả Use Case UC-MD10-17: Đăng xuất khỏi Hệ thống} \label{tab:uc_md10_17_logout_in_codeblock} \\
\hline
\multicolumn{2}{|c|}{\textbf{2.1. Tóm tắt (Summary)}} \\
\hline
\textbf{Mục} & \textbf{Nội dung} \\
\hline
\endhead % Header cho các trang tiếp theo
\midrule
\endfoot % Footer cho bảng
\bottomrule
\endlastfoot % Footer cho trang cuối cùng
Use Case Name & Đăng xuất khỏi Hệ thống \\
\hline
Use Case ID & UC-MD10-17 \\
\hline
Use Case Description & Cho phép một Người dùng đang đăng nhập (Nhân viên các cấp, Quản lý, Quản trị viên) kết thúc phiên làm việc hiện tại của mình và thoát khỏi hệ thống một cách an toàn. \\
\hline
Actor & US-01, US-02, US-03, US-04, US-05, US-06, US-07, US-09, US-10 (Bất kỳ Người dùng nội bộ nào đã đăng nhập) \\
\hline
Priority & Must Have \\
\hline
Trigger & Người dùng muốn kết thúc công việc với hệ thống, đặc biệt khi sử dụng máy tính chung hoặc muốn bảo vệ tài khoản của mình. \\
\hline
Pre-Condition & - Người dùng đã đăng nhập thành công vào hệ thống (đã thực hiện UC-MD10-16). \newline - Giao diện hệ thống đang hiển thị, và có một nút/liên kết "Đăng xuất" (Logout) rõ ràng. \\
\hline
Post-Condition & - Phiên làm việc (session) hiện tại của người dùng trên máy chủ bị hủy bỏ hoặc vô hiệu hóa. \newline - Mọi cookie hoặc token liên quan đến phiên đăng nhập trên trình duyệt của người dùng được xóa hoặc làm cho không hợp lệ. \newline - Người dùng được chuyển hướng về trang Đăng nhập của hệ thống (UC-MD10-16) hoặc một trang chủ công khai (nếu có). \newline - Để tiếp tục sử dụng các chức năng yêu cầu đăng nhập, người dùng sẽ phải thực hiện lại quy trình Đăng nhập. \\
\hline
\multicolumn{2}{|c|}{\textbf{2.2. Luồng thực thi (Flow)}} \\
\hline
\textbf{Mục} & \textbf{Nội dung} \\
\hline
Basic Flow & 1. Người dùng đang ở một trang bất kỳ trong hệ thống sau khi đã đăng nhập. \newline 2. Người dùng tìm đến nút, biểu tượng hoặc mục menu có nhãn "Đăng xuất" (Logout), "Thoát", hoặc tương tự (thường nằm ở góc trên bên phải giao diện, trong menu người dùng). \newline 3. Người dùng nhấp vào nút/liên kết "Đăng xuất". \newline 4. Hệ thống (System) nhận được yêu cầu đăng xuất. \newline 5. Hệ thống thực hiện các hành động sau: \newline    a. Hủy bỏ hoặc vô hiệu hóa phiên làm việc (session) hiện tại của người dùng trên máy chủ. \newline    b. Xóa các cookie hoặc token xác thực liên quan khỏi trình duyệt của người dùng. \newline 6. Hệ thống chuyển hướng người dùng về trang Đăng nhập của hệ thống. \\
\hline
Alternative Flow & \textbf{3a. Hệ thống yêu cầu xác nhận trước khi đăng xuất:} \newline    1. Sau khi người dùng nhấp "Đăng xuất", hệ thống hiển thị một hộp thoại hỏi "Bạn có chắc chắn muốn đăng xuất không?". \newline    2. Người dùng chọn "Có" / "Đồng ý". \newline    3. Use Case tiếp tục từ bước 4. \newline    4. Nếu người dùng chọn "Không" / "Hủy", họ vẫn ở lại trang hiện tại. \\
\hline
Exception Flow & \textbf{5c. Lỗi hệ thống trong quá trình đăng xuất:} \newline    1. Hệ thống gặp lỗi kỹ thuật (ví dụ: lỗi máy chủ) khi cố gắng hủy phiên làm việc. \newline    2. Hệ thống có thể hiển thị một thông báo lỗi chung. \newline    3. Trong trường hợp xấu nhất, phiên làm việc có thể không được hủy hoàn toàn trên máy chủ, nhưng người dùng vẫn có thể bị chuyển về trang đăng nhập. (Việc này ít khi xảy ra với các hệ thống hiện đại). \\
\hline
\multicolumn{2}{|c|}{\textbf{2.3. Thông tin bổ sung (Additional Information)}} \\
\hline
\textbf{Mục} & \textbf{Nội dung} \\
\hline
Business Rule & - \textbf{BR-UC10.17-1 (V2):} Chức năng đăng xuất phải luôn sẵn có cho người dùng đã đăng nhập. \newline - \textbf{BR-UC10.17-2 (V2):} Việc đăng xuất phải đảm bảo kết thúc hoàn toàn phiên làm việc, không cho phép truy cập lại các trang yêu cầu xác thực bằng cách sử dụng nút "Back" của trình duyệt (trừ khi trang đó được cache và không thực sự yêu cầu session mới). \\
\hline
Non-Functional Requirement & - \textbf{NFR-UC10.17-1 (V2 - Security):} Quá trình đăng xuất phải hiệu quả trong việc vô hiệu hóa phiên làm việc để ngăn chặn truy cập trái phép sau đó. \newline - \textbf{NFR-UC10.17-2 (V2 - Usability):} Nút/liên kết đăng xuất phải dễ dàng tìm thấy. \newline - \textbf{NFR-UC10.17-3 (V2 - Performance):} Hành động đăng xuất và chuyển hướng phải diễn ra nhanh chóng. \\
\hline
\end{longtable}

\subsubsection{Use Case UC-MD10-18: (Tùy chọn) Người dùng Yêu cầu Đặt lại Mật khẩu (Quên Mật khẩu)}
\begin{longtable}{|m{4cm}|p{11cm}|}
\caption{Đặc tả Use Case UC-MD10-18: (Tùy chọn) Người dùng Yêu cầu Đặt lại Mật khẩu (Quên Mật khẩu)} \label{tab:uc_md10_18_forgot_password_in_codeblock} \\
\hline
\multicolumn{2}{|c|}{\textbf{2.1. Tóm tắt (Summary)}} \\
\hline
\textbf{Mục} & \textbf{Nội dung} \\
\hline
\endhead % Header cho các trang tiếp theo
\midrule
\endfoot % Footer cho bảng
\bottomrule
\endlastfoot % Footer cho trang cuối cùng
Use Case Name & (Tùy chọn) Người dùng Yêu cầu Đặt lại Mật khẩu (Quên Mật khẩu) \\
\hline
Use Case ID & UC-MD10-18 \\
\hline
Use Case Description & Cho phép một Người dùng (Nhân viên các cấp, Quản lý, Quản trị viên) không thể nhớ mật khẩu của mình, tự yêu cầu hệ thống gửi hướng dẫn đặt lại mật khẩu đến địa chỉ email đã đăng ký với tài khoản của họ. \\
\hline
Actor & US-01, US-02, US-03, US-04, US-05, US-06, US-07, US-09, US-10 (Bất kỳ Người dùng nội bộ nào có tài khoản và quên mật khẩu) \\
\hline
Priority & Must Have (Nếu không, người dùng quên mật khẩu sẽ hoàn toàn phụ thuộc vào Admin - UC-MD10-07) \\
\hline
Trigger & Người dùng nhập sai mật khẩu khi cố gắng đăng nhập (UC-MD10-16) hoặc chủ động nhận ra mình đã quên mật khẩu. \\
\hline
Pre-Condition & - Người dùng đang ở trang Đăng nhập của hệ thống. \newline - Giao diện trang Đăng nhập có một liên kết/nút "Quên mật khẩu?" (Forgot Password?) hoặc tương tự. \newline - Người dùng nhớ địa chỉ email đã sử dụng để đăng ký tài khoản. \newline - Hệ thống đã được cấu hình Máy chủ Gửi Email (Outgoing Email Server - UC-MD10-10) và có thể gửi email đi. \\
\hline
Post-Condition & - \textbf{Thành công:} Nếu địa chỉ email người dùng cung cấp khớp với một tài khoản hiện có trong hệ thống, hệ thống sẽ gửi một email đến địa chỉ đó. Email này chứa một liên kết đặc biệt (có thời hạn và dùng một lần) cho phép người dùng truy cập vào một trang để tự đặt mật khẩu mới. \newline - \textbf{Thất bại (Email không tồn tại):} Hệ thống có thể hiển thị thông báo "Nếu email của bạn tồn tại trong hệ thống, bạn sẽ nhận được hướng dẫn đặt lại mật khẩu." (để tránh tiết lộ email nào có/không có trong hệ thống) hoặc một thông báo lỗi cụ thể hơn tùy chính sách. \\
\hline
\multicolumn{2}{|c|}{\textbf{2.2. Luồng thực thi (Flow)}} \\
\hline
\textbf{Mục} & \textbf{Nội dung} \\
\hline
Basic Flow & 1. Người dùng đang ở trang Đăng nhập của hệ thống. \newline 2. Người dùng nhấp vào liên kết/nút "Quên mật khẩu?" / "Đặt lại mật khẩu". \newline 3. Hệ thống chuyển hướng người dùng đến một trang/form yêu cầu nhập Địa chỉ Email đã đăng ký tài khoản. \newline 4. Người dùng nhập địa chỉ email của mình vào trường cung cấp. \newline 5. Người dùng nhấp vào nút "Gửi hướng dẫn" / "Đặt lại mật khẩu". \newline 6. Hệ thống (System) kiểm tra xem địa chỉ email người dùng cung cấp có tồn tại trong cơ sở dữ liệu người dùng và tài khoản đó có đang hoạt động hay không. \newline 7. \textbf{Nếu địa chỉ email tồn tại và hợp lệ:} \newline    a. Hệ thống tạo một token/liên kết đặt lại mật khẩu duy nhất, có thời hạn sử dụng ngắn. \newline    b. Hệ thống soạn và gửi một email đến địa chỉ email đó. Nội dung email bao gồm liên kết đặt lại mật khẩu và hướng dẫn người dùng nhấp vào liên kết để đặt mật khẩu mới. \newline    c. Hệ thống hiển thị một thông báo cho người dùng trên trang web (ví dụ: "Hướng dẫn đặt lại mật khẩu đã được gửi đến địa chỉ email của bạn. Vui lòng kiểm tra hộp thư đến (và cả mục Spam/Junk)."). \newline 8. \textbf{Nếu địa chỉ email không tồn tại trong hệ thống hoặc tài khoản không hợp lệ:} \newline    a. Hệ thống (để bảo mật, tránh xác nhận email nào có/không có) vẫn có thể hiển thị thông báo tương tự như bước 7c. Hoặc, tùy theo cấu hình, có thể hiển thị "Địa chỉ email không được tìm thấy." \\
\hline
Alternative Flow & \textbf{Bước tiếp theo (Người dùng nhận email và đặt lại mật khẩu - thường là một luồng riêng nhưng liên quan chặt chẽ):} \newline    1. Người dùng kiểm tra email và nhấp vào liên kết đặt lại mật khẩu. \newline    2. Hệ thống mở một trang yêu cầu người dùng nhập Mật khẩu mới và Xác nhận Mật khẩu mới. \newline    3. Người dùng nhập mật khẩu mới (tuân thủ chính sách độ mạnh). \newline    4. Người dùng nhấn "Lưu mật khẩu mới" / "Đặt lại". \newline    5. Hệ thống xác thực token, cập nhật mật khẩu mới cho tài khoản người dùng. \newline    6. Hệ thống báo "Mật khẩu của bạn đã được đặt lại thành công. Bạn có thể đăng nhập ngay bây giờ." và có thể chuyển hướng về trang Đăng nhập. \\
\hline
Exception Flow & \textbf{7d. Lỗi hệ thống khi gửi email:} \newline    1. Hệ thống không thể gửi email (do lỗi cấu hình Máy chủ Gửi Email - UC-MD10-10, hoặc lỗi dịch vụ email). \newline    2. Hệ thống hiển thị thông báo lỗi chung "Không thể gửi email hướng dẫn. Vui lòng liên hệ quản trị viên." Người dùng không nhận được email. \newline \textbf{Alternative Flow - Step 5a. Token đặt lại mật khẩu không hợp lệ/hết hạn:} \newline    1. Người dùng nhấp vào liên kết trong email nhưng token đã hết hạn hoặc không hợp lệ. \newline    2. Hệ thống báo lỗi "Liên kết đặt lại mật khẩu không hợp lệ hoặc đã hết hạn. Vui lòng yêu cầu lại." \\
\hline
\multicolumn{2}{|c|}{\textbf{2.3. Thông tin bổ sung (Additional Information)}} \\
\hline
\textbf{Mục} & \textbf{Nội dung} \\
\hline
Business Rule & - \textbf{BR-UC10.18-1 (V2):} Liên kết đặt lại mật khẩu phải là duy nhất, chỉ sử dụng được một lần và có thời hạn hiệu lực ngắn (ví dụ: 1 giờ, 24 giờ) để tăng cường bảo mật. \newline - \textbf{BR-UC10.18-2 (V2):} Mật khẩu mới do người dùng tự đặt phải tuân thủ các chính sách về độ mạnh mật khẩu của hệ thống (nếu được cấu hình). \newline - \textbf{BR-UC10.18-3 (V2):} Chức năng này chỉ nên áp dụng cho các tài khoản đang hoạt động hoặc ít nhất là không bị khóa vĩnh viễn. \\
\hline
Non-Functional Requirement & - \textbf{NFR-UC10.18-1 (V2 - Security):} Toàn bộ quy trình phải được thiết kế an toàn để ngăn chặn việc chiếm đoạt tài khoản. Email gửi đi phải sử dụng HTTPS cho liên kết. \newline - \textbf{NFR-UC10.18-2 (V2 - Usability):} Quy trình phải đơn giản và dễ thực hiện cho người dùng cuối. Hướng dẫn trong email và trên trang đặt lại mật khẩu phải rõ ràng. \newline - \textbf{NFR-UC10.18-3 (V2 - Reliability):} Hệ thống gửi email và xử lý token đặt lại mật khẩu phải đáng tin cậy. \\
\hline
\end{longtable}

\subsubsection{Use Case UC-MD10-19: Khách hàng Tự Đăng ký Tài khoản}
\begin{longtable}{|m{4cm}|p{11cm}|}
\caption{Đặc tả Use Case UC-MD10-19: Khách hàng Tự Đăng ký Tài khoản} \label{tab:uc_md10_19_customer_registration_in_codeblock} \\
\hline
\multicolumn{2}{|c|}{\textbf{2.1. Tóm tắt (Summary)}} \\
\hline
\textbf{Mục} & \textbf{Nội dung} \\
\hline
\endhead % Header cho các trang tiếp theo
\midrule
\endfoot % Footer cho bảng
\bottomrule
\endlastfoot % Footer cho trang cuối cùng
Use Case Name & Khách hàng Tự Đăng ký Tài khoản \\
\hline
Use Case ID & UC-MD10-19 \\
\hline
Use Case Description & Cho phép một Khách hàng mới (US-08) chưa có tài khoản, tự tạo một tài khoản người dùng trên giao diện web/app của nhà hàng để có thể đặt chỗ, quản lý các lượt đặt chỗ cá nhân và các thông tin khác. \\
\hline
Actor & US-08 (Khách hàng) \\
\hline
Priority & Must Have (Nếu muốn khách hàng quản lý đặt chỗ online) \\
\hline
Trigger & Khách hàng mới muốn tạo tài khoản để sử dụng các dịch vụ online của nhà hàng. \\
\hline
Pre-Condition & - Giao diện web/app của nhà hàng có chức năng "Đăng ký" (Sign Up / Register). \newline - Hệ thống đã được cấu hình Máy chủ Gửi Email (UC-MD10-10) để gửi email xác thực (nếu có). \\
\hline
Post-Condition & - \textbf{Thành công (và có xác thực email):} Một tài khoản người dùng mới cho khách hàng được tạo trong hệ thống (có thể ở trạng thái "Chờ xác thực"). Một email xác thực được gửi đến địa chỉ email khách hàng cung cấp. Sau khi khách hàng xác thực qua email, tài khoản được kích hoạt và khách hàng có thể đăng nhập. \newline - \textbf{Thành công (không cần xác thực email):} Tài khoản người dùng mới cho khách hàng được tạo và kích hoạt ngay. Khách hàng có thể đăng nhập ngay. \newline - \textbf{Thất bại:} Tài khoản không được tạo do lỗi nhập liệu hoặc lỗi hệ thống. \\
\hline
\multicolumn{2}{|c|}{\textbf{2.2. Luồng thực thi (Flow)}} \\
\hline
\textbf{Mục} & \textbf{Nội dung} \\
\hline
Basic Flow (Có xác thực email) & 1. Khách hàng (US-08) truy cập giao diện web/app của nhà hàng và chọn nút/liên kết "Đăng ký". \newline 2. Hệ thống hiển thị form đăng ký, yêu cầu nhập các thông tin: \newline    - Họ và Tên (Full Name). \newline    - Địa chỉ Email (Email Address - sẽ dùng làm tên đăng nhập). \newline    - Mật khẩu (Password). \newline    - Xác nhận Mật khẩu (Confirm Password). \newline    - (Tùy chọn) Số điện thoại. \newline    - (Tùy chọn) Đồng ý với Điều khoản Dịch vụ. \newline 3. US-08 nhập đầy đủ các thông tin yêu cầu. \newline 4. US-08 nhấn nút "Đăng ký" / "Tạo tài khoản". \newline 5. Hệ thống (System) kiểm tra tính hợp lệ của dữ liệu: \newline    - Các trường bắt buộc đã được điền. \newline    - Định dạng email hợp lệ. \newline    - Mật khẩu và Xác nhận mật khẩu khớp nhau. \newline    - Độ mạnh mật khẩu (nếu có chính sách). \newline    - Địa chỉ Email chưa được sử dụng bởi một tài khoản khác. \newline 6. \textbf{Nếu dữ liệu hợp lệ:} \newline    a. Hệ thống tạo một bản ghi người dùng mới cho khách hàng (thường là loại "Portal User" hoặc "Public User" trong hệ thống), có thể với trạng thái "Chờ xác thực" hoặc "Không hoạt động". \newline    b. Hệ thống tạo một token xác thực email duy nhất, có thời hạn. \newline    c. Hệ thống gửi một email đến địa chỉ email US-08 đã cung cấp, chứa một liên kết xác thực kèm token. \newline    d. Hệ thống hiển thị thông báo cho US-08: "Đăng ký gần hoàn tất. Vui lòng kiểm tra email của bạn để xác thực tài khoản." \newline 7. \textbf{Nếu dữ liệu không hợp lệ (bước 5):} \newline    a. Hệ thống hiển thị thông báo lỗi cụ thể (ví dụ: "Email đã được sử dụng", "Mật khẩu không khớp", "Vui lòng nhập tên"). \newline    b. Hệ thống giữ nguyên form để US-08 sửa lại. Use Case quay lại bước 3. \\
\hline
Alternative Flow & \textbf{Luồng Xác thực Email (Tiếp nối từ Basic Flow bước 6d):} \newline    1. US-08 kiểm tra hộp thư đến và nhấp vào liên kết xác thực trong email. \newline    2. Hệ thống nhận yêu cầu xác thực, kiểm tra tính hợp lệ của token (còn hạn, đúng token). \newline    3. Nếu token hợp lệ, hệ thống kích hoạt tài khoản người dùng (ví dụ: đặt trạng thái "Hoạt động"). \newline    4. Hệ thống hiển thị thông báo "Tài khoản của bạn đã được xác thực thành công. Bạn có thể đăng nhập ngay bây giờ." và có thể chuyển hướng đến trang đăng nhập. \newline \textbf{Basic Flow (Không yêu cầu xác thực email):} \newline    1. Các bước 1-5 tương tự Basic Flow. \newline    2. Nếu dữ liệu hợp lệ (bước 6), hệ thống tạo bản ghi người dùng mới và kích hoạt ngay. \newline    3. Hệ thống hiển thị thông báo "Đăng ký thành công. Bạn có thể đăng nhập ngay." và có thể chuyển hướng đến trang đăng nhập. \\
\hline
Exception Flow & \textbf{6e. Lỗi hệ thống khi tạo người dùng hoặc gửi email xác thực:} \newline    1. Hệ thống gặp lỗi kỹ thuật. \newline    2. Hệ thống hiển thị thông báo lỗi chung "Đã xảy ra lỗi trong quá trình đăng ký. Vui lòng thử lại sau." \newline \textbf{Alternative Flow - Step 3a (Xác thực Email). Token không hợp lệ/hết hạn:} \newline    1. US-08 nhấp vào liên kết nhưng token đã hết hạn hoặc không đúng. \newline    2. Hệ thống báo lỗi "Liên kết xác thực không hợp lệ hoặc đã hết hạn. Vui lòng thử đăng ký lại hoặc yêu cầu gửi lại email xác thực (nếu có chức năng)." \\
\hline
\multicolumn{2}{|c|}{\textbf{2.3. Thông tin bổ sung (Additional Information)}} \\
\hline
\textbf{Mục} & \textbf{Nội dung} \\
\hline
Business Rule & - \textbf{BR-UC10.19-1 (V2):} Địa chỉ Email khách hàng cung cấp phải là duy nhất trong hệ thống đối với các tài khoản khách hàng. \newline - \textbf{BR-UC10.19-2 (V2):} Chính sách mật khẩu (độ dài, độ phức tạp) nên được áp dụng khi khách hàng tạo mật khẩu. \newline - \textbf{BR-UC10.19-3 (V2):} Việc yêu cầu xác thực email là một biện pháp bảo mật tốt để đảm bảo email là của khách hàng và giảm thiểu tài khoản ảo. Liên kết xác thực phải có thời hạn. \\
\hline
Non-Functional Requirement & - \textbf{NFR-UC10.19-1 (V2 - Security):} Quá trình đăng ký phải được thực hiện qua HTTPS. Mật khẩu phải được hash trước khi lưu. \newline - \textbf{NFR-UC10.19-2 (V2 - Usability):} Form đăng ký phải đơn giản, dễ hiểu. Thông báo lỗi và hướng dẫn phải rõ ràng. \newline - \textbf{NFR-UC10.19-3 (V2 - Performance):} Tốc độ xử lý đăng ký và gửi email phải nhanh. \\
\hline
\end{longtable}
\subsection{Module MD-11: Quản lý Quan hệ Khách hàng (CRM)}

% === Quản lý Hồ sơ Khách hàng (CRM View) - Tách nhỏ ===
\subsubsection{Use Case UC-MD11-01: Tạo mới Hồ sơ Khách hàng (CRM)}
\begin{longtable}{|m{4cm}|p{11cm}|}
\caption{Đặc tả Use Case UC-MD11-01: Tạo mới Hồ sơ Khách hàng (CRM)} \label{tab:uc_md11_01_create_customer_crm} \\
\hline
\multicolumn{2}{|c|}{\textbf{2.1. Tóm tắt (Summary)}} \\
\hline
\textbf{Mục} & \textbf{Nội dung} \\
\hline
\endhead
\midrule
\endfoot
\bottomrule
\endlastfoot
Use Case Name & Tạo mới Hồ sơ Khách hàng (CRM) \\
\hline
Use Case ID & UC-MD11-01 \\
\hline
Use Case Description & Cho phép Người dùng được phân quyền (US-01: Quản lý, US-03: Lễ tân, US-10: Quản trị viên) tạo một bản ghi hồ sơ khách hàng mới trong hệ thống CRM của nhà hàng. Hồ sơ này bao gồm các thông tin liên hệ cơ bản và có thể được bổ sung các thông tin chi tiết khác để phục vụ việc quản lý quan hệ và chăm sóc khách hàng. \\
\hline
Actor & US-01 (Quản lý nhà hàng), US-03 (Nhân viên lễ tân), US-10 (Quản trị viên Hệ thống) \\
\hline
Priority & Must Have \\
\hline
Trigger & - Có một khách hàng mới đến nhà hàng hoặc liên hệ (ví dụ: đặt bàn qua điện thoại lần đầu) và nhân viên muốn tạo hồ sơ để lưu trữ thông tin. \newline - Nhân viên nhập thông tin khách hàng từ một nguồn khác (ví dụ: danh sách khách mời sự kiện). \\
\hline
Pre-Condition & - Người dùng đã đăng nhập vào hệ thống với quyền tạo hồ sơ khách hàng trong module CRM. \\
\hline
Post-Condition & - Một bản ghi hồ sơ khách hàng mới được tạo và lưu thành công trong cơ sở dữ liệu. \newline - Hồ sơ mới này chứa các thông tin cơ bản do người dùng nhập (ví dụ: Tên, Số điện thoại, Email). \newline - Hồ sơ sẵn sàng để được liên kết với các giao dịch (đặt chỗ, đơn hàng) hoặc được cập nhật thêm thông tin chi tiết (sở thích, lịch sử...). \\
\hline
\multicolumn{2}{|c|}{\textbf{2.2. Luồng thực thi (Flow)}} \\
\hline
\textbf{Mục} & \textbf{Nội dung} \\
\hline
Basic Flow & 1. Người dùng (US-01/US-03/US-10) truy cập module CRM (hoặc Contacts). \newline 2. Người dùng chọn hành động "Tạo mới" (Create) Khách hàng. \newline 3. Hệ thống hiển thị form nhập thông tin khách hàng mới. \newline 4. Người dùng nhập các thông tin bắt buộc: \newline    - Tên Khách hàng (Customer Name). \newline    - Số Điện thoại (Phone Number) (BR-UC11.1-1). \newline 5. Người dùng nhập các thông tin tùy chọn (nếu có): \newline    - Địa chỉ Email. \newline    - Địa chỉ nhà/công ty. \newline    - Ngày sinh. \newline    - Giới tính. \newline    - Công ty (nếu là khách hàng doanh nghiệp). \newline    - Ghi chú ban đầu về khách hàng. \newline 6. Người dùng chọn "Lưu" (Save). \newline 7. Hệ thống kiểm tra tính hợp lệ của dữ liệu (ví dụ: SĐT không được để trống, có thể kiểm tra trùng SĐT/Email - BR-UC11.1-2). \newline 8. Hệ thống lưu bản ghi hồ sơ khách hàng mới. \newline 9. Hệ thống hiển thị thông báo tạo thành công và có thể chuyển sang giao diện chi tiết của khách hàng vừa tạo. \\
\hline
Alternative Flow & \textbf{6a. Lưu và Tạo mới (Save \& New):} \newline    1. Người dùng chọn "Lưu và Tạo mới" để tiếp tục nhập hồ sơ khách hàng khác. \\
\hline
Exception Flow & \textbf{7a. Dữ liệu không hợp lệ:} \newline    1. Hệ thống phát hiện thiếu thông tin bắt buộc hoặc dữ liệu trùng lặp (nếu có kiểm tra). \newline    2. Hệ thống báo lỗi, yêu cầu người dùng sửa. Use Case quay lại bước 4. \newline \textbf{8a. Lỗi hệ thống khi lưu:} \newline    1. Hệ thống gặp lỗi kỹ thuật khi lưu. \newline    2. Hệ thống báo lỗi chung. \\
\hline
\multicolumn{2}{|c|}{\textbf{2.3. Thông tin bổ sung (Additional Information)}} \\
\hline
\textbf{Mục} & \textbf{Nội dung} \\
\hline
Business Rule & - \textbf{BR-UC11.1-1:} Số điện thoại của khách hàng nên là thông tin bắt buộc để liên hệ. \newline - \textbf{BR-UC11.1-2:} Hệ thống nên có cơ chế kiểm tra trùng lặp khách hàng dựa trên Số điện thoại hoặc Email để tránh tạo nhiều hồ sơ cho cùng một người. \newline - \textbf{BR-UC11.1-3:} Khách hàng tạo qua CRM sẽ có loại là "Contact" hoặc "Customer", khác với "User" nội bộ. \\
\hline
Non-Functional Requirement & - \textbf{NFR-UC11.1-1 (Usability):} Form tạo khách hàng phải đơn giản, rõ ràng. \newline - \textbf{NFR-UC11.1-2 (Performance):} Lưu hồ sơ mới phải nhanh. \newline - \textbf{NFR-UC11.1-3 (Data Integrity):} Dữ liệu khách hàng phải được lưu chính xác. \\
\hline
\end{longtable}

\subsubsection{Use Case UC-MD11-02: Xem Danh sách Hồ sơ Khách hàng (CRM)}
\begin{longtable}{|m{4cm}|p{11cm}|}
\caption{Đặc tả Use Case UC-MD11-02: Xem Danh sách Hồ sơ Khách hàng (CRM)} \label{tab:uc_md11_02_view_customer_list_crm} \\
\hline
\multicolumn{2}{|c|}{\textbf{2.1. Tóm tắt (Summary)}} \\
\hline
\textbf{Mục} & \textbf{Nội dung} \\
\hline
\endhead
\midrule
\endfoot
\bottomrule
\endlastfoot
Use Case Name & Xem Danh sách Hồ sơ Khách hàng (CRM) \\
\hline
Use Case ID & UC-MD11-02 \\
\hline
Use Case Description & Cho phép Người dùng được phân quyền (US-01, US-03, US-06, US-10) xem danh sách tất cả các hồ sơ khách hàng đã được tạo và lưu trữ trong hệ thống CRM, với các thông tin tóm tắt và khả năng tìm kiếm, lọc. \\
\hline
Actor & US-01 (Quản lý nhà hàng), US-03 (Nhân viên lễ tân), US-06 (Kế toán), US-10 (Quản trị viên Hệ thống) \\
\hline
Priority & Must Have \\
\hline
Trigger & - Cần tra cứu thông tin của một nhóm khách hàng. \newline - Cần tìm kiếm một khách hàng cụ thể. \newline - Chuẩn bị cho các thao tác quản lý khác (xem chi tiết, sửa, phân loại). \\
\hline
Pre-Condition & - Người dùng đã đăng nhập vào hệ thống với quyền xem hồ sơ khách hàng trong module CRM. \\
\hline
Post-Condition & - Danh sách các hồ sơ khách hàng (theo bộ lọc mặc định) được hiển thị. \newline - Người dùng có thể xem thông tin tóm tắt và thực hiện các hành động tiếp theo. \\
\hline
\multicolumn{2}{|c|}{\textbf{2.2. Luồng thực thi (Flow)}} \\
\hline
\textbf{Mục} & \textbf{Nội dung} \\
\hline
Basic Flow & 1. Người dùng truy cập module CRM (hoặc Contacts). \newline 2. Hệ thống mặc định hiển thị danh sách các khách hàng (có thể là tất cả hoặc theo một bộ lọc mặc định). \newline 3. Với mỗi khách hàng, hiển thị thông tin tóm tắt: Tên, SĐT, Email (nếu có), Công ty (nếu có). \newline 4. Người dùng xem xét danh sách. \\
\hline
Alternative Flow & \textbf{4a. Tìm kiếm khách hàng:} \newline    1. Người dùng nhập tên, SĐT, email, hoặc mã khách hàng vào ô tìm kiếm. \newline    2. Hệ thống lọc và hiển thị kết quả. \newline \textbf{4b. Lọc khách hàng:} \newline    1. Người dùng sử dụng các bộ lọc có sẵn (ví dụ: theo Loại khách hàng, theo Thẻ tag, theo Nhân viên phụ trách - nếu có). \newline    2. Hệ thống áp dụng bộ lọc. \newline \textbf{4c. Sắp xếp danh sách:} \newline    1. Người dùng nhấp vào tiêu đề cột để sắp xếp. \\
\hline
Exception Flow & \textbf{2a. Lỗi tải danh sách:} Hệ thống báo lỗi. \newline \textbf{2b. Không có khách hàng nào:} Hiển thị danh sách trống. \\
\hline
\multicolumn{2}{|c|}{\textbf{2.3. Thông tin bổ sung (Additional Information)}} \\
\hline
\textbf{Mục} & \textbf{Nội dung} \\
\hline
Business Rule & - \textbf{BR-UC11.2-1:} Danh sách phải hiển thị chính xác các hồ sơ khách hàng. \\
\hline
Non-Functional Requirement & - \textbf{NFR-UC11.2-1 (Usability):} Giao diện dễ đọc, tìm kiếm/lọc hiệu quả. \newline - \textbf{NFR-UC11.2-2 (Performance):} Tải danh sách nhanh, kể cả với số lượng lớn. \\
\hline
\end{longtable}

\subsubsection{Use Case UC-MD11-03: Xem Chi tiết Hồ sơ Khách hàng (CRM)}
\begin{longtable}{|m{4cm}|p{11cm}|}
\caption{Đặc tả Use Case UC-MD11-03: Xem Chi tiết Hồ sơ Khách hàng (CRM)} \label{tab:uc_md11_03_view_customer_detail_crm} \\
\hline
\multicolumn{2}{|c|}{\textbf{2.1. Tóm tắt (Summary)}} \\
\hline
\textbf{Mục} & \textbf{Nội dung} \\
\hline
\endhead
\midrule
\endfoot
\bottomrule
\endlastfoot
Use Case Name & Xem Chi tiết Hồ sơ Khách hàng (CRM) \\
\hline
Use Case ID & UC-MD11-03 \\
\hline
Use Case Description & Cho phép Người dùng được phân quyền (US-01, US-03, US-06, US-10) xem thông tin chi tiết đầy đủ của một hồ sơ khách hàng cụ thể đã được chọn từ danh sách, bao gồm thông tin liên hệ, lịch sử giao dịch, các ghi chú, thẻ tag, và các thông tin CRM khác. \\
\hline
Actor & US-01 (Quản lý nhà hàng), US-03 (Nhân viên lễ tân), US-06 (Kế toán), US-10 (Quản trị viên Hệ thống) \\
\hline
Priority & Must Have \\
\hline
Trigger & Người dùng nhấp vào một khách hàng từ danh sách (UC-MD11-02) để xem hoặc chuẩn bị sửa đổi thông tin. \\
\hline
Pre-Condition & - Người dùng đang xem danh sách hồ sơ khách hàng (UC-MD11-02 thành công). \newline - Người dùng có quyền xem chi tiết hồ sơ khách hàng. \\
\hline
Post-Condition & - Form chi tiết (Contact Form View) của khách hàng được chọn được hiển thị. \newline - Người dùng nắm được mọi thông tin đã cấu hình và lịch sử liên quan đến khách hàng đó. \\
\hline
\multicolumn{2}{|c|}{\textbf{2.2. Luồng thực thi (Flow)}} \\
\hline
\textbf{Mục} & \textbf{Nội dung} \\
\hline
Basic Flow & 1. Người dùng đang xem danh sách hồ sơ khách hàng (UC-MD11-02). \newline 2. Người dùng nhấp vào tên hoặc một vùng có thể nhấp được của dòng khách hàng muốn xem chi tiết. \newline 3. Hệ thống truy xuất toàn bộ thông tin của hồ sơ khách hàng đó. \newline 4. Hệ thống hiển thị Form chi tiết khách hàng, thường được tổ chức thành nhiều tab hoặc phần thông tin: \newline    - \textbf{Thông tin liên hệ:} Tên, SĐT, Email, Địa chỉ, Công ty... \newline    - \textbf{Thông tin nội bộ/CRM:} Loại khách hàng, Thẻ tag, Nhân viên phụ trách, Ghi chú nội bộ... \newline    - \textbf{Lịch sử Giao dịch (Smart Buttons/Tabs):} Số lượt đặt chỗ, Tổng doanh thu, Đơn hàng gần nhất, Cơ hội (nếu dùng Sales), Hóa đơn... \newline    - (Tùy chọn) Thông tin sở thích, ngày kỷ niệm... \newline 5. Người dùng xem xét các thông tin chi tiết. \\
\hline
Alternative Flow & \textbf{5a. Nhấp vào các nút thông minh (Smart Buttons):} \newline    1. US-01/US-03/US-06/US-10 nhấp vào một nút thông minh (ví dụ: "X Đặt chỗ", "Y Hóa đơn") để xem danh sách các bản ghi liên quan. \\
\hline
Exception Flow & \textbf{3a. Lỗi tải chi tiết hồ sơ khách hàng.} \newline \textbf{3b. Khách hàng không tồn tại/không có quyền xem.} \\
\hline
\multicolumn{2}{|c|}{\textbf{2.3. Thông tin bổ sung (Additional Information)}} \\
\hline
\textbf{Mục} & \textbf{Nội dung} \\
\hline
Business Rule & - \textbf{BR-UC11.3-1:} Form chi tiết phải hiển thị đầy đủ và chính xác thông tin. \\
\hline
Non-Functional Requirement & - \textbf{NFR-UC11.3-1 (Usability):} Thông tin trên form phải được tổ chức logic. \newline - \textbf{NFR-UC11.3-2 (Performance):} Thời gian tải form chi tiết phải nhanh. \\
\hline
\end{longtable}

\subsubsection{Use Case UC-MD11-04: Sửa Thông tin Hồ sơ Khách hàng (CRM)}
\begin{longtable}{|m{4cm}|p{11cm}|}
\caption{Đặc tả Use Case UC-MD11-04: Sửa Thông tin Hồ sơ Khách hàng (CRM)} \label{tab:uc_md11_04_edit_customer_crm} \\
\hline
\multicolumn{2}{|c|}{\textbf{2.1. Tóm tắt (Summary)}} \\
\hline
\textbf{Mục} & \textbf{Nội dung} \\
\hline
\endhead
\midrule
\endfoot
\bottomrule
\endlastfoot
Use Case Name & Sửa Thông tin Hồ sơ Khách hàng (CRM) \\
\hline
Use Case ID & UC-MD11-04 \\
\hline
Use Case Description & Cho phép Người dùng được phân quyền (US-01, US-03, US-10) cập nhật các thông tin trong một hồ sơ khách hàng đã tồn tại, ví dụ: thay đổi SĐT, email, địa chỉ, thêm ghi chú, cập nhật sở thích, hoặc thay đổi phân loại. \\
\hline
Actor & US-01 (Quản lý nhà hàng), US-03 (Nhân viên lễ tân), US-10 (Quản trị viên Hệ thống) \\
\hline
Priority & Must Have \\
\hline
Trigger & - Thông tin liên hệ của khách hàng thay đổi. \newline - Cần bổ sung thông tin mới vào hồ sơ khách hàng (ví dụ: sở thích, ghi chú chăm sóc). \newline - Cần sửa lỗi nhập liệu trước đó. \\
\hline
Pre-Condition & - Người dùng đã đăng nhập với quyền sửa hồ sơ khách hàng. \newline - Hồ sơ khách hàng cần sửa đã tồn tại và người dùng đang xem form chi tiết của khách hàng đó (UC-MD11-03). \\
\hline
Post-Condition & - Các thông tin của hồ sơ khách hàng được cập nhật thành công. \newline - Thay đổi sẽ được phản ánh trong các giao dịch và báo cáo liên quan. \\
\hline
\multicolumn{2}{|c|}{\textbf{2.2. Luồng thực thi (Flow)}} \\
\hline
\textbf{Mục} & \textbf{Nội dung} \\
\hline
Basic Flow & 1. Người dùng đang xem form chi tiết hồ sơ khách hàng (UC-MD11-03). \newline 2. Người dùng chọn hành động "Sửa" (Edit). \newline 3. Hệ thống cho phép chỉnh sửa các trường thông tin trên form. \newline 4. Người dùng thực hiện các thay đổi mong muốn (Tên, SĐT, Email, Địa chỉ, Ghi chú, Thẻ tag...). \newline 5. Người dùng chọn hành động "Lưu" (Save). \newline 6. Hệ thống kiểm tra tính hợp lệ của dữ liệu (ví dụ: SĐT/Email có thể cần duy nhất nếu thay đổi). \newline 7. Hệ thống lưu các thay đổi. \newline 8. Hệ thống chuyển form về chế độ xem với thông tin đã cập nhật. \\
\hline
Alternative Flow & Không có. \\
\hline
Exception Flow & \textbf{6a. Lỗi Xác thực Dữ liệu:} Hệ thống báo lỗi (ví dụ: SĐT/Email mới đã tồn tại). \newline \textbf{7a. Lỗi Hệ thống khi Cập nhật.} \\
\hline
\multicolumn{2}{|c|}{\textbf{2.3. Thông tin bổ sung (Additional Information)}} \\
\hline
\textbf{Mục} & \textbf{Nội dung} \\
\hline
Business Rule & - \textbf{BR-UC11.4-1:} Thông tin SĐT/Email sau khi sửa (nếu thay đổi) nên được kiểm tra tính duy nhất. \\
\hline
Non-Functional Requirement & - \textbf{NFR-UC11.4-1 (Usability):} Form sửa dễ sử dụng. \newline - \textbf{NFR-UC11.4-2 (Performance):} Lưu thay đổi nhanh. \\
\hline
\end{longtable}

\subsubsection{Use Case UC-MD11-05: Xóa/Lưu trữ Hồ sơ Khách hàng (CRM)}
\begin{longtable}{|m{4cm}|p{11cm}|}
\caption{Đặc tả Use Case UC-MD11-05: Xóa/Lưu trữ Hồ sơ Khách hàng (CRM)} \label{tab:uc_md11_05_delete_archive_customer_crm} \\
\hline
\multicolumn{2}{|c|}{\textbf{2.1. Tóm tắt (Summary)}} \\
\hline
\textbf{Mục} & \textbf{Nội dung} \\
\hline
\endhead
\midrule
\endfoot
\bottomrule
\endlastfoot
Use Case Name & Xóa/Lưu trữ Hồ sơ Khách hàng (CRM) \\
\hline
Use Case ID & UC-MD11-05 \\
\hline
Use Case Description & Cho phép Người dùng được phân quyền (US-01, US-10) xóa vĩnh viễn một hồ sơ khách hàng (nếu khách hàng đó chưa có bất kỳ giao dịch nào liên quan) hoặc lưu trữ (ẩn đi) hồ sơ khách hàng không còn tương tác hoặc không còn phù hợp, trong khi vẫn giữ lại dữ liệu lịch sử. \\
\hline
Actor & US-01 (Quản lý nhà hàng), US-10 (Quản trị viên Hệ thống) \\
\hline
Priority & Should Have \\
\hline
Trigger & - Cần dọn dẹp dữ liệu khách hàng không còn liên quan. \newline - Một hồ sơ khách hàng được tạo nhầm. \\
\hline
Pre-Condition & - Người dùng đã đăng nhập với quyền xóa/lưu trữ hồ sơ khách hàng. \newline - Hồ sơ khách hàng cần xử lý đã tồn tại. \\
\hline
Post-Condition & - \textbf{Nếu Xóa thành công:} Bản ghi hồ sơ khách hàng bị xóa khỏi cơ sở dữ liệu. \newline - \textbf{Nếu Lưu trữ thành công:} Hồ sơ khách hàng được đánh dấu là "đã lưu trữ" (archived), không hiển thị trong danh sách mặc định nhưng dữ liệu vẫn còn. \newline - \textbf{Nếu không thể Xóa (do có giao dịch):} Hệ thống báo lỗi, người dùng có thể chọn Lưu trữ. \\
\hline
\multicolumn{2}{|c|}{\textbf{2.2. Luồng thực thi (Flow)}} \\
\hline
\textbf{Mục} & \textbf{Nội dung} \\
\hline
Basic Flow (Lưu trữ) & 1. Người dùng đang xem chi tiết hồ sơ khách hàng (UC-MD11-03) hoặc đã chọn khách hàng từ danh sách (UC-MD11-02). \newline 2. Người dùng chọn hành động "Lưu trữ" (Archive) từ menu Hành động. \newline 3. Hệ thống (có thể) yêu cầu xác nhận. Người dùng xác nhận. \newline 4. Hệ thống cập nhật trạng thái `active = False` (hoặc tương đương) cho hồ sơ khách hàng. \newline 5. Hệ thống báo thành công. Hồ sơ biến mất khỏi danh sách mặc định. \\
\hline
Alternative Flow & \textbf{Basic Flow (Xóa - Nếu được phép và không có ràng buộc):} \newline    1. Người dùng chọn hành động "Xóa" (Delete). \newline    2. Hệ thống kiểm tra xem khách hàng có giao dịch nào liên quan không (BR-UC11.5-1). \newline    3. \textbf{Nếu không có giao dịch:} Hệ thống yêu cầu xác nhận xóa. Người dùng xác nhận. Hệ thống xóa bản ghi. \newline    4. \textbf{Nếu có giao dịch:} Hệ thống báo lỗi "Không thể xóa khách hàng này vì đã có giao dịch. Bạn có muốn Lưu trữ thay thế không?". Người dùng có thể chọn Lưu trữ (chuyển sang Basic Flow). \\
\hline
Exception Flow & \textbf{Lỗi hệ thống khi xóa/lưu trữ.} \\
\hline
\multicolumn{2}{|c|}{\textbf{2.3. Thông tin bổ sung (Additional Information)}} \\
\hline
\textbf{Mục} & \textbf{Nội dung} \\
\hline
Business Rule & - \textbf{BR-UC11.5-1:} Không thể xóa vĩnh viễn hồ sơ khách hàng nếu đã có các bản ghi liên quan (đặt chỗ, hóa đơn...). Nên sử dụng Lưu trữ. \newline - \textbf{BR-UC11.5-2:} Lưu trữ giúp ẩn khách hàng nhưng vẫn giữ lại lịch sử. \\
\hline
Non-Functional Requirement & - \textbf{NFR-UC11.5-1 (Data Integrity):} Ràng buộc không cho xóa khi có liên kết là quan trọng. \newline - \textbf{NFR-UC11.5-2 (Usability):} Thông báo lỗi/hướng dẫn rõ ràng. \\
\hline
\end{longtable}

\subsubsection{Use Case UC-MD11-06: Phân loại/Gắn thẻ Khách hàng}
\begin{longtable}{|m{4cm}|p{11cm}|}
\caption{Đặc tả Use Case UC-MD11-06: Phân loại/Gắn thẻ Khách hàng} \label{tab:uc_md11_06_categorize_tag_customer} \\
\hline
\multicolumn{2}{|c|}{\textbf{2.1. Tóm tắt (Summary)}} \\
\hline
\textbf{Mục} & \textbf{Nội dung} \\
\hline
\endhead
\midrule
\endfoot
\bottomrule
\endlastfoot
Use Case Name & Phân loại/Gắn thẻ Khách hàng \\
\hline
Use Case ID & UC-MD11-06 \\
\hline
Use Case Description & Cho phép Người dùng được phân quyền (US-01, US-06, US-10) gán các thẻ (tags) hoặc phân loại (ví dụ: VIP, Thường xuyên, Khách công ty) cho một hoặc nhiều hồ sơ khách hàng. Việc này giúp nhóm khách hàng cho các mục đích marketing, chăm sóc đặc biệt, hoặc phân tích. \\
\hline
Actor & US-01 (Quản lý nhà hàng), US-06 (Kế toán), US-10 (Quản trị viên Hệ thống) \\
\hline
Priority & Should Have \\
\hline
Trigger & - Cần nhóm các khách hàng có đặc điểm chung. \newline - Chuẩn bị cho một chiến dịch marketing nhắm mục tiêu. \newline - Đánh dấu các khách hàng quan trọng. \\
\hline
Pre-Condition & - Người dùng đã đăng nhập với quyền sửa hồ sơ khách hàng/gắn thẻ. \newline - Các thẻ/phân loại đã được tạo sẵn trong hệ thống (thường là một chức năng cấu hình riêng của CRM/Contacts). \newline - Hồ sơ khách hàng cần gắn thẻ đã tồn tại. \\
\hline
Post-Condition & - Hồ sơ khách hàng được chọn được liên kết với (các) thẻ/phân loại đã chọn. \newline - Thông tin này có thể được sử dụng để lọc, tìm kiếm, hoặc trong các quy trình tự động khác. \\
\hline
\multicolumn{2}{|c|}{\textbf{2.2. Luồng thực thi (Flow)}} \\
\hline
\textbf{Mục} & \textbf{Nội dung} \\
\hline
Basic Flow (Gắn thẻ cho một khách hàng) & 1. Người dùng đang xem chi tiết hồ sơ khách hàng (UC-MD11-03), ở chế độ Sửa. \newline 2. Người dùng tìm đến trường "Thẻ" (Tags) hoặc "Phân loại" (Category). \newline 3. Người dùng nhấp vào trường đó. Hệ thống hiển thị danh sách các thẻ/phân loại có sẵn và/hoặc cho phép nhập tên thẻ mới. \newline 4. Người dùng chọn (tick) vào (các) thẻ muốn gán hoặc nhập tên thẻ mới rồi chọn tạo. \newline 5. Người dùng chọn "Lưu". \newline 6. Hệ thống lưu liên kết thẻ với khách hàng. \\
\hline
Alternative Flow & \textbf{1a. Gắn thẻ hàng loạt từ danh sách khách hàng:} \newline    1. Người dùng đang xem danh sách khách hàng (UC-MD11-02). \newline    2. Người dùng chọn (tick) một hoặc nhiều khách hàng. \newline    3. Người dùng chọn hành động "Gắn thẻ" / "Add Tags" từ menu chung. \newline    4. Hệ thống hiển thị popup chọn thẻ. Người dùng chọn thẻ. \newline    5. Hệ thống áp dụng thẻ cho tất cả khách hàng đã chọn. \\
\hline
Exception Flow & \textbf{6a. Lỗi hệ thống khi lưu thẻ.} \\
\hline
\multicolumn{2}{|c|}{\textbf{2.3. Thông tin bổ sung (Additional Information)}} \\
\hline
\textbf{Mục} & \textbf{Nội dung} \\
\hline
Business Rule & - \textbf{BR-UC11.6-1:} Một khách hàng có thể được gán nhiều thẻ. \newline - \textbf{BR-UC11.6-2:} Danh sách thẻ/phân loại nên được quản lý tập trung. \\
\hline
Non-Functional Requirement & - \textbf{NFR-UC11.6-1 (Usability):} Việc chọn/gắn thẻ phải dễ dàng. \newline - \textbf{NFR-UC11.6-2 (Performance):} Gắn thẻ hàng loạt (nếu có) phải hiệu quả. \\
\hline
\end{longtable}

\subsubsection{Use Case UC-MD11-07: Xem Lịch sử Tương tác/Đặt chỗ của Khách hàng}
\begin{longtable}{|m{4cm}|p{11cm}|}
\caption{Đặc tả Use Case UC-MD11-07: Xem Lịch sử Tương tác/Đặt chỗ của Khách hàng} \label{tab:uc_md11_07_view_customer_history} \\
\hline
\multicolumn{2}{|c|}{\textbf{2.1. Tóm tắt (Summary)}} \\
\hline
\textbf{Mục} & \textbf{Nội dung} \\
\hline
\endhead
\midrule
\endfoot
\bottomrule
\endlastfoot
Use Case Name & Xem Lịch sử Tương tác/Đặt chỗ của Khách hàng \\
\hline
Use Case ID & UC-MD11-07 \\
\hline
Use Case Description & Cho phép Người dùng được phân quyền (US-01, US-06, US-10) truy cập và xem lại toàn bộ lịch sử các hoạt động và giao dịch liên quan đến một khách hàng cụ thể, bao gồm các lượt đặt chỗ đã thực hiện, các hóa đơn đã thanh toán, các phản hồi/đánh giá đã gửi (nếu có), và các ghi chú tương tác khác. \\
\hline
Actor & US-01 (Quản lý nhà hàng), US-06 (Kế toán), US-10 (Quản trị viên Hệ thống) \\
\hline
Priority & Must Have \\
\hline
Trigger & - Cần hiểu rõ hơn về một khách hàng trước khi tương tác hoặc đưa ra quyết định chăm sóc. \newline - Kiểm tra lại một giao dịch cũ của khách hàng. \newline - Phân tích hành vi và sở thích của khách hàng. \\
\hline
Pre-Condition & - Người dùng đang xem chi tiết hồ sơ của một khách hàng cụ thể (UC-MD11-03). \\
\hline
Post-Condition & - Người dùng thấy được danh sách hoặc các tab thông tin chi tiết về lịch sử hoạt động của khách hàng đó với nhà hàng. \\
\hline
\multicolumn{2}{|c|}{\textbf{2.2. Luồng thực thi (Flow)}} \\
\hline
\textbf{Mục} & \textbf{Nội dung} \\
\hline
Basic Flow & 1. Người dùng đang xem form chi tiết hồ sơ khách hàng (UC-MD11-03). \newline 2. Trên form này, có các nút thông minh (smart buttons) hoặc các tab riêng biệt hiển thị số lượng các bản ghi liên quan, ví dụ: "X Đặt chỗ", "Y Hóa đơn", "Z Đánh giá". \newline 3. Người dùng nhấp vào một nút thông minh hoặc một tab (ví dụ: nhấp vào "X Đặt chỗ"). \newline 4. Hệ thống chuyển hướng người dùng đến danh sách tất cả các lượt đặt chỗ của khách hàng đó, được lọc sẵn. \newline 5. Tương tự, người dùng có thể nhấp vào các nút/tab khác để xem lịch sử Hóa đơn, Đánh giá, Ghi chú liên hệ, v.v. \\
\hline
Alternative Flow & \textbf{2a. Hiển thị lịch sử dạng dòng thời gian (timeline):} \newline    1. Một số hệ thống CRM có thể hiển thị lịch sử tương tác dưới dạng một dòng thời gian trực quan trong hồ sơ khách hàng. \\
\hline
Exception Flow & \textbf{4a. Lỗi tải danh sách lịch sử.} \\
\hline
\multicolumn{2}{|c|}{\textbf{2.3. Thông tin bổ sung (Additional Information)}} \\
\hline
\textbf{Mục} & \textbf{Nội dung} \\
\hline
Business Rule & - \textbf{BR-UC11.7-1:} Lịch sử phải bao gồm tất cả các giao dịch và tương tác quan trọng. \newline - \textbf{BR-UC11.7-2:} Dữ liệu lịch sử phải được liên kết chính xác với đúng khách hàng. \\
\hline
Non-Functional Requirement & - \textbf{NFR-UC11.7-1 (Usability):} Việc truy cập và xem lịch sử phải dễ dàng, thông tin trình bày rõ ràng. \newline - \textbf{NFR-UC11.7-2 (Performance):} Tải lịch sử (kể cả khi nhiều) phải nhanh. \\
\hline
\end{longtable}

% === Quản lý Voucher/Khuyến mãi ===
\subsubsection{Use Case UC-MD11-08: Tạo mới Chương trình Khuyến mãi/Voucher}
\begin{longtable}{|m{4cm}|p{11cm}|}
\caption{Đặc tả Use Case UC-MD11-08: Tạo mới Chương trình Khuyến mãi/Voucher} \label{tab:uc_md11_08_create_promo_voucher} \\
\hline
\multicolumn{2}{|c|}{\textbf{2.1. Tóm tắt (Summary)}} \\
\hline
\textbf{Mục} & \textbf{Nội dung} \\
\hline
\endhead
\midrule
\endfoot
\bottomrule
\endlastfoot
Use Case Name & Tạo mới Chương trình Khuyến mãi/Voucher \\
\hline
Use Case ID & UC-MD11-08 \\
\hline
Use Case Description & Cho phép Người dùng được phân quyền (US-01: Quản lý, US-10: Quản trị viên) định nghĩa một chương trình khuyến mãi mới hoặc một lô mã voucher mới trong hệ thống. Bao gồm việc đặt tên, mô tả, và loại hình khuyến mãi (ví dụ: giảm giá theo phần trăm, giảm một số tiền cố định, mua X tặng Y, hoặc tạo các mã giảm giá riêng lẻ/hàng loạt). \\
\hline
Actor & US-01 (Quản lý nhà hàng), US-10 (Quản trị viên Hệ thống) \\
\hline
Priority & Must Have \\
\hline
Trigger & - Nhà hàng muốn triển khai một chương trình khuyến mãi mới để thu hút khách hàng. \newline - Cần tạo các mã voucher để tặng khách hoặc sử dụng trong các chiến dịch marketing. \\
\hline
Pre-Condition & - Người dùng đã đăng nhập vào hệ thống với quyền quản lý chương trình khuyến mãi/voucher (thường trong module Sales, Point of Sale, hoặc một module Marketing riêng). \\
\hline
Post-Condition & - Một bản ghi chương trình khuyến mãi mới hoặc một lô mã voucher mới được tạo và lưu trong hệ thống (có thể ở trạng thái "Nháp" hoặc "Chờ kích hoạt"). \newline - Chương trình/voucher này sẵn sàng để được cấu hình các điều kiện áp dụng chi tiết (UC-MD11-09). \\
\hline
\multicolumn{2}{|c|}{\textbf{2.2. Luồng thực thi (Flow)}} \\
\hline
\textbf{Mục} & \textbf{Nội dung} \\
\hline
Basic Flow (Tạo chương trình khuyến mãi chung) & 1. Người dùng (US-01/US-10) truy cập module quản lý Khuyến mãi/Giá (ví dụ: Sales > Promotions hoặc POS > Promotions). \newline 2. Người dùng chọn hành động "Tạo mới" (Create) Chương trình Khuyến mãi. \newline 3. Hệ thống hiển thị form nhập thông tin. \newline 4. Người dùng nhập Tên Chương trình (Program Name) (ví dụ: "Giảm 20\% Thứ Ba Vui Vẻ"). \newline 5. Người dùng chọn Loại Khuyến mãi (Discount Type): \newline    - Giảm giá phần trăm (Percentage Discount). \newline    - Giảm giá số tiền cố định (Fixed Amount Discount). \newline    - Mua X Tặng Y (Buy X Get Y Free). \newline    - (Có thể có) Các loại khác như Miễn phí vận chuyển, Tặng điểm thưởng... \newline 6. Nếu là giảm giá, người dùng nhập Giá trị giảm (ví dụ: 20 cho 20\%, hoặc 50000 cho 50,000 VNĐ). \newline 7. (Tùy chọn) Người dùng nhập Mô tả chi tiết về chương trình. \newline 8. Người dùng chọn "Lưu". \newline 9. Hệ thống lưu chương trình khuyến mãi mới. \\
\hline
Alternative Flow & \textbf{Basic Flow (Tạo Voucher/Mã giảm giá):} \newline    1. Người dùng chọn loại "Chương trình Voucher" hoặc "Tạo Mã Giảm giá". \newline    2. Người dùng nhập thông tin chung cho lô voucher (Tên, Loại giảm giá, Giá trị). \newline    3. (Tùy chọn) Người dùng cấu hình cách tạo mã: \newline       a. Tạo một mã cụ thể (ví dụ: "WELCOME20"). \newline       b. Tạo hàng loạt mã ngẫu nhiên (ví dụ: nhập số lượng mã cần tạo, tiền tố/hậu tố nếu có). \newline    4. Hệ thống tạo ra các mã voucher tương ứng. \newline    5. Use Case tiếp tục từ bước 8 của Basic Flow. \\
\hline
Exception Flow & \textbf{8a. Lỗi lưu chương trình/voucher.} \\
\hline
\multicolumn{2}{|c|}{\textbf{2.3. Thông tin bổ sung (Additional Information)}} \\
\hline
\textbf{Mục} & \textbf{Nội dung} \\
\hline
Business Rule & - \textbf{BR-UC11.8-1:} Tên chương trình/voucher phải rõ ràng. \newline - \textbf{BR-UC11.8-2:} Loại và giá trị khuyến mãi phải được định nghĩa chính xác. \newline - \textbf{BR-UC11.8-3:} Mã voucher (nếu tạo) phải đảm bảo tính duy nhất. \\
\hline
Non-Functional Requirement & - \textbf{NFR-UC11.8-1 (Usability):} Giao diện tạo khuyến mãi/voucher phải linh hoạt. \newline - \textbf{NFR-UC11.8-2 (Performance):} Tạo mã voucher hàng loạt (nếu có) phải hiệu quả. \\
\hline
\end{longtable}

\subsubsection{Use Case UC-MD11-09: Thiết lập Điều kiện Áp dụng Khuyến mãi/Voucher}
\begin{longtable}{|m{4cm}|p{11cm}|}
\caption{Đặc tả Use Case UC-MD11-09: Thiết lập Điều kiện Áp dụng Khuyến mãi/Voucher} \label{tab:uc_md11_09_set_promo_conditions} \\
\hline
\multicolumn{2}{|c|}{\textbf{2.1. Tóm tắt (Summary)}} \\
\hline
\textbf{Mục} & \textbf{Nội dung} \\
\hline
\endhead
\midrule
\endfoot
\bottomrule
\endlastfoot
Use Case Name & Thiết lập Điều kiện Áp dụng Khuyến mãi/Voucher \\
\hline
Use Case ID & UC-MD11-09 \\
\hline
Use Case Description & Sau khi một chương trình khuyến mãi hoặc lô voucher đã được tạo (UC-MD11-08), cho phép Người dùng được phân quyền (US-01, US-10) cấu hình các điều kiện và quy tắc chi tiết để chương trình/voucher đó có thể được áp dụng. Các điều kiện có thể bao gồm: khoảng thời gian hiệu lực, giá trị đơn hàng tối thiểu/tối đa, giới hạn số lần sử dụng, các sản phẩm hoặc danh mục sản phẩm cụ thể được áp dụng, hoặc đối tượng khách hàng cụ thể. \\
\hline
Actor & US-01 (Quản lý nhà hàng), US-10 (Quản trị viên Hệ thống) \\
\hline
Priority & Must Have \\
\hline
Trigger & - Một chương trình khuyến mãi/voucher vừa được tạo và cần được hoàn thiện các quy tắc. \newline - Cần thay đổi điều kiện áp dụng của một chương trình/voucher hiện có. \\
\hline
Pre-Condition & - Người dùng đã đăng nhập với quyền quản lý khuyến mãi/voucher. \newline - Chương trình khuyến mãi hoặc lô voucher đã tồn tại trong hệ thống (đã tạo ở UC-MD11-08). \\
\hline
Post-Condition & - Các điều kiện và quy tắc áp dụng cho chương trình khuyến mãi/voucher được cập nhật và lưu lại. \newline - Hệ thống (POS, Đặt chỗ Online) sẽ sử dụng các điều kiện này để xác định xem một chương trình/voucher có hợp lệ để áp dụng cho một đơn hàng cụ thể hay không. \\
\hline
\multicolumn{2}{|c|}{\textbf{2.2. Luồng thực thi (Flow)}} \\
\hline
\textbf{Mục} & \textbf{Nội dung} \\
\hline
Basic Flow & 1. Người dùng (US-01/US-10) truy cập vào danh sách các chương trình khuyến mãi/voucher. \newline 2. Người dùng chọn chương trình/voucher muốn cấu hình điều kiện và mở chi tiết (hoặc vào chế độ Sửa). \newline 3. Hệ thống hiển thị các trường/tab cho phép thiết lập điều kiện: \newline    - \textbf{Thời gian hiệu lực:} Ngày bắt đầu, Ngày kết thúc. \newline    - \textbf{Điều kiện Đơn hàng:} Giá trị đơn hàng tối thiểu, Giá trị đơn hàng tối đa. \newline    - \textbf{Giới hạn Sử dụng:} Tổng số lần sử dụng cho toàn chương trình, Số lần sử dụng cho mỗi khách hàng, Số lần sử dụng cho mỗi mã voucher (nếu là voucher). \newline    - \textbf{Sản phẩm Áp dụng:} Chọn tất cả sản phẩm, hoặc chỉ một số sản phẩm cụ thể, hoặc các sản phẩm thuộc một/nhiều Danh mục Sản phẩm POS. \newline    - \textbf{Khách hàng Áp dụng:} Cho tất cả khách hàng, hoặc chỉ một số khách hàng cụ thể, hoặc khách hàng thuộc một/nhiều Thẻ tag/Phân loại khách hàng. \newline    - (Tùy chọn) Các điều kiện khác như kênh bán hàng (POS, Online)... \newline 4. Người dùng nhập/chọn các giá trị cho các điều kiện mong muốn. \newline 5. Người dùng chọn "Lưu". \newline 6. Hệ thống lưu lại các điều kiện. \\
\hline
Alternative Flow & Không có luồng thay thế đáng kể. \\
\hline
Exception Flow & \textbf{4a. Nhập điều kiện không hợp lệ/mâu thuẫn:} \newline    1. Ví dụ: Ngày kết thúc trước ngày bắt đầu. \newline    2. Hệ thống báo lỗi. \newline \textbf{6a. Lỗi hệ thống khi lưu điều kiện.} \\
\hline
\multicolumn{2}{|c|}{\textbf{2.3. Thông tin bổ sung (Additional Information)}} \\
\hline
\textbf{Mục} & \textbf{Nội dung} \\
\hline
Business Rule & - \textbf{BR-UC11.9-1:} Các điều kiện phải được định nghĩa rõ ràng và không mâu thuẫn. \newline - \textbf{BR-UC11.9-2:} Hệ thống phải kiểm tra tất cả các điều kiện khi có yêu cầu áp dụng khuyến mãi/voucher (ở POS hoặc online). \\
\hline
Non-Functional Requirement & - \textbf{NFR-UC11.9-1 (Usability):} Giao diện thiết lập điều kiện phải linh hoạt nhưng dễ hiểu. \newline - \textbf{NFR-UC11.9-2 (Flexibility):} Hệ thống nên hỗ trợ nhiều loại điều kiện phổ biến. \\
\hline
\end{longtable}

\subsubsection{Use Case UC-MD11-10: Quản lý Danh sách Mã Voucher}
\begin{longtable}{|m{4cm}|p{11cm}|}
\caption{Đặc tả Use Case UC-MD11-10: Quản lý Danh sách Mã Voucher} \label{tab:uc_md11_10_manage_voucher_codes} \\
\hline
\multicolumn{2}{|c|}{\textbf{2.1. Tóm tắt (Summary)}} \\
\hline
\textbf{Mục} & \textbf{Nội dung} \\
\hline
\endhead
\midrule
\endfoot
\bottomrule
\endlastfoot
Use Case Name & Quản lý Danh sách Mã Voucher \\
\hline
Use Case ID & UC-MD11-10 \\
\hline
Use Case Description & Cho phép Người dùng được phân quyền (US-01, US-10) xem danh sách các mã voucher đã được tạo trong một chương trình voucher cụ thể, kiểm tra trạng thái của từng mã (ví dụ: còn hiệu lực, đã sử dụng, hết hạn), số lần đã sử dụng (nếu cho phép nhiều lần), và có thể thực hiện các hành động như xuất danh sách mã ra file, hoặc vô hiệu hóa thủ công một số mã voucher. \\
\hline
Actor & US-01 (Quản lý nhà hàng), US-10 (Quản trị viên Hệ thống) \\
\hline
Priority & Should Have (Nếu sử dụng nhiều mã voucher riêng lẻ) \\
\hline
Trigger & - Cần kiểm tra tình trạng sử dụng của các mã voucher đã phát hành. \newline - Cần xuất danh sách mã voucher để gửi cho đối tác hoặc cho mục đích marketing. \newline - Cần vô hiệu hóa một số mã voucher vì lý do nào đó. \\
\hline
Pre-Condition & - Người dùng đã đăng nhập với quyền quản lý voucher. \newline - Đã có ít nhất một chương trình voucher với các mã đã được tạo (từ UC-MD11-08). \\
\hline
Post-Condition & - Người dùng xem được danh sách và trạng thái các mã voucher. \newline - (Nếu thực hiện) Danh sách mã được xuất ra file hoặc một số mã được cập nhật trạng thái (ví dụ: vô hiệu hóa). \\
\hline
\multicolumn{2}{|c|}{\textbf{2.2. Luồng thực thi (Flow)}} \\
\hline
\textbf{Mục} & \textbf{Nội dung} \\
\hline
Basic Flow (Xem danh sách mã voucher) & 1. Người dùng (US-01/US-10) truy cập vào chi tiết một Chương trình Voucher cụ thể. \newline 2. Trong giao diện của chương trình voucher, có một tab hoặc một nút "Danh sách Mã" (Coupon Codes / Vouchers). Người dùng chọn vào đó. \newline 3. Hệ thống hiển thị danh sách các mã voucher thuộc chương trình đó, với các cột thông tin: Mã Voucher, Trạng thái (Active, Used, Expired, Disabled), Ngày tạo, Ngày hết hạn (nếu có), Số lần đã sử dụng / Số lần cho phép. \newline 4. Người dùng xem xét danh sách. \\
\hline
Alternative Flow & \textbf{4a. Xuất danh sách mã voucher:} \newline    1. Giao diện danh sách có nút "Xuất Excel" / "Export CSV". \newline    2. Người dùng nhấp vào. Hệ thống tạo và cho tải về file chứa danh sách mã. \newline \textbf{4b. Vô hiệu hóa/Kích hoạt lại mã voucher:} \newline    1. Người dùng chọn một hoặc nhiều mã voucher. \newline    2. Người dùng chọn hành động "Vô hiệu hóa" (Disable) hoặc "Kích hoạt lại" (Enable) (nếu đã bị vô hiệu hóa). \newline    3. Hệ thống cập nhật trạng thái của các mã đã chọn. \\
\hline
Exception Flow & \textbf{3a. Lỗi tải danh sách mã.} \newline \textbf{Alternative Flow 4b - Step 3a. Lỗi cập nhật trạng thái mã.} \\
\hline
\multicolumn{2}{|c|}{\textbf{2.3. Thông tin bổ sung (Additional Information)}} \\
\hline
\textbf{Mục} & \textbf{Nội dung} \\
\hline
Business Rule & - \textbf{BR-UC11.10-1:} Trạng thái của mã voucher phải được cập nhật tự động khi mã được sử dụng hoặc hết hạn. \newline - \textbf{BR-UC11.10-2:} Cần kiểm soát quyền được phép vô hiệu hóa/kích hoạt lại mã. \\
\hline
Non-Functional Requirement & - \textbf{NFR-UC11.10-1 (Usability):} Danh sách mã dễ quản lý, dễ lọc/tìm kiếm. \newline - \textbf{NFR-UC11.10-2 (Performance):} Hiển thị/xuất danh sách mã lớn phải hiệu quả. \\
\hline
\end{longtable}

\subsubsection{Use Case UC-MD11-11: Áp dụng Khuyến mãi/Voucher vào Đơn hàng POS}
\begin{longtable}{|m{4cm}|p{11cm}|}
\caption{Đặc tả Use Case UC-MD11-11: Áp dụng Khuyến mãi/Voucher vào Đơn hàng POS} \label{tab:uc_md11_11_apply_promo_pos} \\
\hline
\multicolumn{2}{|c|}{\textbf{2.1. Tóm tắt (Summary)}} \\
\hline
\textbf{Mục} & \textbf{Nội dung} \\
\hline
\endhead
\midrule
\endfoot
\bottomrule
\endlastfoot
Use Case Name & Áp dụng Khuyến mãi/Voucher vào Đơn hàng POS \\
\hline
Use Case ID & UC-MD11-11 \\
\hline
Use Case Description & Cho phép Nhân viên (US-02: Phục vụ, US-05: Thu ngân) tại điểm bán hàng (POS) áp dụng một chương trình khuyến mãi hợp lệ hoặc nhập một mã voucher hợp lệ vào đơn hàng hiện tại của khách, để hệ thống tự động tính toán lại tổng giá trị đơn hàng sau khi đã trừ đi khoản giảm giá. \\
\hline
Actor & US-02 (Nhân viên phục vụ), US-05 (Nhân viên thu ngân) \\
\hline
Priority & Must Have \\
\hline
Trigger & - Khách hàng cung cấp mã voucher khi thanh toán. \newline - Nhân viên xác định đơn hàng của khách đủ điều kiện để áp dụng một chương trình khuyến mãi đang diễn ra. \\
\hline
Pre-Condition & - Nhân viên đang ở màn hình đơn hàng POS hoặc màn hình thanh toán. \newline - Đã có ít nhất một món trong đơn hàng. \newline - Các chương trình khuyến mãi/voucher và điều kiện áp dụng đã được cấu hình (UC-MD11-08, UC-MD11-09). \\
\hline
Post-Condition & - Khoản giảm giá từ chương trình khuyến mãi/voucher được áp dụng thành công vào đơn hàng. \newline - Tổng số tiền cần thanh toán của đơn hàng được cập nhật (giảm đi). \newline - Nếu là voucher, trạng thái của mã voucher đó được cập nhật (ví dụ: số lần sử dụng tăng lên, hoặc chuyển sang "Đã sử dụng" nếu chỉ dùng 1 lần). \\
\hline
\multicolumn{2}{|c|}{\textbf{2.2. Luồng thực thi (Flow)}} \\
\hline
\textbf{Mục} & \textbf{Nội dung} \\
\hline
Basic Flow (Áp dụng Voucher) & 1. Nhân viên (US-02/US-05) đang ở màn hình đơn hàng hoặc thanh toán. \newline 2. Khách hàng cung cấp mã voucher. \newline 3. Nhân viên tìm và nhấp vào nút "Nhập Mã Voucher" / "Apply Coupon" trên giao diện. \newline 4. Hệ thống hiển thị ô để nhập mã voucher. Nhân viên nhập mã. \newline 5. Nhân viên nhấn "Áp dụng" / "Enter". \newline 6. Hệ thống (System) kiểm tra tính hợp lệ của mã voucher: \newline    - Mã có tồn tại không? \newline    - Mã còn hiệu lực không (chưa hết hạn, chưa bị vô hiệu hóa)? \newline    - Mã còn lượt sử dụng không? \newline    - Đơn hàng hiện tại có thỏa mãn các điều kiện áp dụng của voucher không (giá trị tối thiểu, sản phẩm áp dụng...)? \newline 7. \textbf{Nếu mã hợp lệ và đơn hàng đủ điều kiện:} \newline    a. Hệ thống tính toán số tiền được giảm dựa trên loại và giá trị của voucher. \newline    b. Hệ thống áp dụng khoản giảm giá vào đơn hàng (hiển thị một dòng giảm giá). \newline    c. Hệ thống cập nhật lại tổng tiền cần thanh toán. \newline    d. Hệ thống cập nhật trạng thái/số lần sử dụng của mã voucher. \newline    e. Hệ thống báo "Áp dụng voucher thành công." \newline 8. \textbf{Nếu mã không hợp lệ hoặc đơn hàng không đủ điều kiện:} \newline    a. Hệ thống báo lỗi cụ thể (ví dụ: "Mã voucher không hợp lệ", "Đơn hàng chưa đủ điều kiện", "Voucher đã hết lượt sử dụng"). \\
\hline
Alternative Flow & \textbf{Basic Flow (Áp dụng chương trình khuyến mãi tự động/chọn từ danh sách):} \newline    1. Hệ thống POS (có thể) tự động kiểm tra xem đơn hàng hiện tại có đủ điều kiện cho bất kỳ chương trình khuyến mãi nào đang hoạt động không. \newline    2. Nếu có, hệ thống có thể tự động áp dụng (nếu cấu hình) hoặc hiển thị danh sách các khuyến mãi khả dụng cho nhân viên chọn. \newline    3. Nhân viên chọn một chương trình khuyến mãi từ danh sách. \newline    4. Hệ thống thực hiện các bước 7a-7e. \newline \textbf{3a. Xóa bỏ khuyến mãi/voucher đã áp dụng:} \newline    1. Nếu nhân viên áp dụng nhầm hoặc khách đổi ý. \newline    2. Nhân viên chọn dòng giảm giá đã áp dụng và chọn "Xóa" / "Remove Discount". \newline    3. Hệ thống hoàn tác lại việc áp dụng, cập nhật lại tổng tiền và trạng thái voucher (nếu có). \\
\hline
Exception Flow & \textbf{6a. Lỗi hệ thống khi kiểm tra voucher/khuyến mãi.} \newline \textbf{7f. Lỗi hệ thống khi áp dụng giảm giá/cập nhật trạng thái voucher.} \\
\hline
\multicolumn{2}{|c|}{\textbf{2.3. Thông tin bổ sung (Additional Information)}} \\
\hline
\textbf{Mục} & \textbf{Nội dung} \\
\hline
Business Rule & - \textbf{BR-UC11.11-1 (System):} Hệ thống phải kiểm tra nghiêm ngặt tất cả các điều kiện trước khi cho phép áp dụng. \newline - \textbf{BR-UC11.11-2 (System):} Một đơn hàng có thể được áp dụng nhiều loại khuyến mãi khác nhau hay không tùy thuộc vào cấu hình "Stackable promotions". \newline - \textbf{BR-UC11.11-3 (System):} Việc cập nhật trạng thái voucher sau khi sử dụng là bắt buộc. \\
\hline
Non-Functional Requirement & - \textbf{NFR-UC11.11-1 (Usability):} Thao tác áp dụng/nhập voucher phải nhanh và đơn giản. \newline - \textbf{NFR-UC11.11-2 (Performance):} Kiểm tra điều kiện và áp dụng phải nhanh. \newline - \textbf{NFR-UC11.11-3 (Accuracy):} Tính toán giảm giá phải chính xác. \\
\hline
\end{longtable}

\subsubsection{Use Case UC-MD11-12: Khách hàng Sử dụng Voucher khi Đặt chỗ Online}
\begin{longtable}{|m{4cm}|p{11cm}|}
\caption{Đặc tả Use Case UC-MD11-12: Khách hàng Sử dụng Voucher khi Đặt chỗ Online} \label{tab:uc_md11_12_customer_uses_voucher_online} \\
\hline
\multicolumn{2}{|c|}{\textbf{2.1. Tóm tắt (Summary)}} \\
\hline
\textbf{Mục} & \textbf{Nội dung} \\
\hline
\endhead
\midrule
\endfoot
\bottomrule
\endlastfoot
Use Case Name & Khách hàng Sử dụng Voucher khi Đặt chỗ Online \\
\hline
Use Case ID & UC-MD11-12 \\
\hline
Use Case Description & Cho phép Khách hàng (US-08) khi đang thực hiện quy trình đặt chỗ hoặc đặt món trước trực tuyến (qua website/app của nhà hàng), nhập một mã voucher hợp lệ để được hưởng ưu đãi/giảm giá trên tổng giá trị đơn đặt chỗ hoặc tiền đặt cọc. \\
\hline
Actor & US-08 (Khách hàng) \\
\hline
Priority & Should Have (Nếu muốn tăng cường đặt chỗ online) \\
\hline
Trigger & Khách hàng có một mã voucher và muốn sử dụng nó để được giảm giá khi đặt chỗ/đặt món online. \\
\hline
Pre-Condition & - Khách hàng đang trong luồng đặt chỗ online (ví dụ: ở bước xem tóm tắt đơn hàng trước khi thanh toán cọc - UC-MD03-04). \newline - Giao diện đặt chỗ online có một trường để nhập mã voucher/khuyến mãi. \newline - Các mã voucher hợp lệ và điều kiện áp dụng đã được cấu hình trong hệ thống (UC-MD11-08, UC-MD11-09). \\
\hline
Post-Condition & - Nếu mã voucher hợp lệ và đơn đặt chỗ đủ điều kiện, khoản giảm giá được áp dụng. \newline - Tổng giá trị đơn đặt chỗ hoặc số tiền đặt cọc cần thanh toán được cập nhật (giảm đi). \newline - Trạng thái của mã voucher được cập nhật. \newline - Khách hàng có thể tiếp tục hoàn tất đặt chỗ với giá đã được giảm. \\
\hline
\multicolumn{2}{|c|}{\textbf{2.2. Luồng thực thi (Flow)}} \\
\hline
\textbf{Mục} & \textbf{Nội dung} \\
\hline
Basic Flow & 1. Khách hàng (US-08) đang ở trang tóm tắt đơn đặt chỗ/giỏ hàng online, trước bước thanh toán. \newline 2. Giao diện hiển thị một ô "Nhập mã giảm giá" / "Voucher Code". \newline 3. US-08 nhập mã voucher mình có vào ô đó. \newline 4. US-08 nhấn nút "Áp dụng" / "Apply". \newline 5. Hệ thống (System) thực hiện kiểm tra tính hợp lệ của mã voucher và đơn hàng (tương tự bước 6 của UC-MD11-11). \newline 6. \textbf{Nếu mã hợp lệ và đủ điều kiện:} \newline    a. Hệ thống tính toán và áp dụng khoản giảm giá vào tổng giá trị đơn đặt chỗ hoặc tiền đặt cọc. \newline    b. Giao diện cập nhật lại các số tiền, hiển thị rõ khoản giảm giá. \newline    c. Hệ thống cập nhật trạng thái/số lần sử dụng của mã voucher. \newline    d. Hệ thống báo "Áp dụng mã giảm giá thành công." \newline 7. \textbf{Nếu mã không hợp lệ hoặc không đủ điều kiện:} \newline    a. Hệ thống hiển thị thông báo lỗi cụ thể cho khách hàng. \\
\hline
Alternative Flow & \textbf{3a. Xóa mã voucher đã nhập:} \newline    1. Nếu khách hàng muốn xóa mã voucher đã áp dụng để dùng mã khác hoặc không dùng nữa. \newline    2. Giao diện có nút "Xóa mã" / "Remove voucher". Khách hàng nhấp vào. \newline    3. Hệ thống hoàn tác giảm giá, cập nhật lại số tiền và trạng thái voucher. \\
\hline
Exception Flow & \textbf{5a. Lỗi hệ thống khi kiểm tra/áp dụng voucher online.} \\
\hline
\multicolumn{2}{|c|}{\textbf{2.3. Thông tin bổ sung (Additional Information)}} \\
\hline
\textbf{Mục} & \textbf{Nội dung} \\
\hline
Business Rule & - \textbf{BR-UC11.12-1 (System):} Quy trình kiểm tra và áp dụng voucher online phải đồng bộ và nhất quán với quy trình áp dụng tại POS. \newline - \textbf{BR-UC11.12-2 (System):} Giảm giá từ voucher có thể ảnh hưởng đến số tiền đặt cọc cần thanh toán. \\
\hline
Non-Functional Requirement & - \textbf{NFR-UC11.12-1 (Usability):} Ô nhập voucher dễ thấy, thông báo rõ ràng. \newline - \textbf{NFR-UC11.12-2 (Performance):} Phản hồi khi áp dụng voucher phải nhanh. \\
\hline
\end{longtable}

\subsubsection{Use Case UC-MD11-13: Xem Báo cáo Hiệu quả Khuyến mãi/Voucher}
\begin{longtable}{|m{4cm}|p{11cm}|}
\caption{Đặc tả Use Case UC-MD11-13: Xem Báo cáo Hiệu quả Khuyến mãi/Voucher} \label{tab:uc_md11_13_report_promo_effectiveness} \\
\hline
\multicolumn{2}{|c|}{\textbf{2.1. Tóm tắt (Summary)}} \\
\hline
\textbf{Mục} & \textbf{Nội dung} \\
\hline
\endhead
\midrule
\endfoot
\bottomrule
\endlastfoot
Use Case Name & Xem Báo cáo Hiệu quả Khuyến mãi/Voucher \\
\hline
Use Case ID & UC-MD11-13 \\
\hline
Use Case Description & Cung cấp cho Quản lý nhà hàng (US-01) báo cáo thống kê chi tiết về tình hình sử dụng và hiệu quả của các chương trình khuyến mãi hoặc các lô mã voucher đã được triển khai. Báo cáo bao gồm số lần áp dụng, tổng giá trị giảm giá, và có thể cả doanh thu từ các đơn hàng có áp dụng khuyến mãi, trong một khoảng thời gian. (Đây là chức năng được mở rộng và chi tiết hóa từ UC-MD09-09, tập trung vào góc độ CRM/Marketing). \\
\hline
Actor & US-01 (Quản lý nhà hàng) \\
\hline
Priority & Must Have \\
\hline
Trigger & - Cần đánh giá hiệu quả thực tế của một hoặc nhiều chương trình khuyến mãi/voucher. \newline - Cần so sánh hiệu quả giữa các chương trình khác nhau. \newline - Cần dữ liệu để lập kế hoạch cho các chiến dịch khuyến mãi trong tương lai. \\
\hline
Pre-Condition & - US-01 đã đăng nhập với quyền truy cập báo cáo Khuyến mãi/Marketing. \newline - Đã có dữ liệu về các đơn hàng đã áp dụng khuyến mãi/voucher. \\
\hline
Post-Condition & - Báo cáo thống kê chi tiết về hiệu quả của các chương trình khuyến mãi/voucher được hiển thị. \newline - US-01 có thông tin để đưa ra các quyết định kinh doanh và marketing. \\
\hline
\multicolumn{2}{|c|}{\textbf{2.2. Luồng thực thi (Flow)}} \\
\hline
\textbf{Mục} & \textbf{Nội dung} \\
\hline
Basic Flow & 1. US-01 truy cập vào mục "Báo cáo" (Reporting) của module CRM/Marketing hoặc Sales. \newline 2. US-01 chọn loại báo cáo "Hiệu quả Khuyến mãi" / "Voucher Usage Report". \newline 3. US-01 chọn Khoảng thời gian muốn xem báo cáo. \newline 4. (Tùy chọn) US-01 có thể lọc theo một Chương trình Khuyến mãi cụ thể hoặc một Lô Voucher cụ thể. \newline 5. US-01 nhấn "Xem báo cáo". \newline 6. Hệ thống truy vấn và tổng hợp dữ liệu từ các đơn hàng đã áp dụng khuyến mãi/voucher trong khoảng thời gian và theo bộ lọc đã chọn. \newline 7. Hệ thống hiển thị báo cáo, có thể bao gồm các thông tin cho mỗi chương trình/lô voucher: \newline    - Tên Chương trình/Lô Voucher. \newline    - Số lần được áp dụng (Số đơn hàng/Số mã đã dùng). \newline    - Tổng giá trị đã giảm giá. \newline    - (Tùy chọn) Tổng doanh thu từ các đơn hàng có áp dụng. \newline    - (Tùy chọn) Tỷ lệ chuyển đổi (nếu có dữ liệu về số người thấy/nhận). \newline    - (Tùy chọn) Danh sách chi tiết các đơn hàng/mã voucher đã áp dụng. \newline 8. US-01 xem xét và phân tích báo cáo. \\
\hline
Alternative Flow & \textbf{7a. Xem biểu đồ so sánh hiệu quả:} Giao diện có thể cung cấp biểu đồ để so sánh các chỉ số giữa các chương trình khác nhau. \\
\hline
Exception Flow & Tương tự UC-MD09-03 (Lỗi truy vấn/tổng hợp, Không có dữ liệu). \\
\hline
\multicolumn{2}{|c|}{\textbf{2.3. Thông tin bổ sung (Additional Information)}} \\
\hline
\textbf{Mục} & \textbf{Nội dung} \\
\hline
Business Rule & - \textbf{BR-UC11.13-1:} Báo cáo phải phân tách rõ ràng dữ liệu của từng chương trình/lô voucher. \newline - \textbf{BR-UC11.13-2:} Các chỉ số tính toán (ví dụ: tổng giảm giá, doanh thu liên quan) phải chính xác. \\
\hline
Non-Functional Requirement & - \textbf{NFR-UC11.13-1 (Usability):} Báo cáo dễ hiểu, có khả năng drill-down xem chi tiết. \newline - \textbf{NFR-UC11.13-2 (Performance):} Tạo báo cáo nhanh. \\
\hline
\end{longtable}

% === Thu thập & Quản lý Đánh giá/Phản hồi ===
\subsubsection{Use Case UC-MD11-14: Khách hàng Gửi Đánh giá/Review sau Khi sử dụng Dịch vụ}
\begin{longtable}{|m{4cm}|p{11cm}|}
\caption{Đặc tả Use Case UC-MD11-14: Khách hàng Gửi Đánh giá/Review sau Khi sử dụng Dịch vụ} \label{tab:uc_md11_14_customer_submit_review} \\
\hline
\multicolumn{2}{|c|}{\textbf{2.1. Tóm tắt (Summary)}} \\
\hline
\textbf{Mục} & \textbf{Nội dung} \\
\hline
\endhead
\midrule
\endfoot
\bottomrule
\endlastfoot
Use Case Name & Khách hàng Gửi Đánh giá/Review sau Khi sử dụng Dịch vụ \\
\hline
Use Case ID & UC-MD11-14 \\
\hline
Use Case Description & Cho phép Khách hàng (US-08), sau khi đã trải nghiệm dịch vụ tại nhà hàng (ăn tại chỗ, mang về, hoặc nhận giao hàng), gửi các ý kiến đánh giá, nhận xét, hoặc xếp hạng (rating) về chất lượng món ăn, thái độ phục vụ, không gian nhà hàng, và các khía cạnh khác. Việc này có thể được thực hiện thông qua một liên kết trong email mời đánh giá tự động, một mã QR trên hóa đơn, hoặc một mục "Gửi phản hồi" trên website/app. \\
\hline
Actor & US-08 (Khách hàng) \\
\hline
Priority & Should Have \\
\hline
Trigger & - Khách hàng nhận được email mời gửi đánh giá sau khi hoàn tất một giao dịch. \newline - Khách hàng chủ động muốn chia sẻ ý kiến của mình về trải nghiệm tại nhà hàng. \\
\hline
Pre-Condition & - Khách hàng có kết nối internet để truy cập form đánh giá. \newline - Hệ thống có một form/giao diện trực tuyến cho phép khách hàng nhập và gửi đánh giá. \newline - (Nếu qua email mời) Hệ thống đã gửi email mời đánh giá có chứa liên kết duy nhất đến form (có thể liên kết với đơn hàng/đặt chỗ cụ thể). \\
\hline
Post-Condition & - Đánh giá của khách hàng (bao gồm xếp hạng sao, nhận xét bằng văn bản, và có thể cả các lựa chọn cho từng tiêu chí) được ghi nhận vào hệ thống. \newline - Đánh giá này được liên kết với hồ sơ khách hàng (nếu khách hàng đăng nhập hoặc email được định danh) và/hoặc với đơn hàng/đặt chỗ cụ thể (nếu gửi qua link mời). \newline - Đánh giá mới sẵn sàng để được Quản lý xem xét (UC-MD11-15). \\
\hline
\multicolumn{2}{|c|}{\textbf{2.2. Luồng thực thi (Flow)}} \\
\hline
\textbf{Mục} & \textbf{Nội dung} \\
\hline
Basic Flow & 1. Khách hàng (US-08) truy cập vào form đánh giá trực tuyến của nhà hàng (qua email, QR code, hoặc link trên website). \newline 2. Hệ thống hiển thị form đánh giá, có thể bao gồm các phần: \newline    - Xếp hạng tổng thể (ví dụ: 1-5 sao). \newline    - Xếp hạng cho các tiêu chí cụ thể (Món ăn, Phục vụ, Không gian, Giá cả...). \newline    - Ô văn bản để nhập nhận xét chi tiết. \newline    - (Tùy chọn) Thông tin đơn hàng/đặt chỗ liên quan (nếu gửi qua link mời). \newline    - (Tùy chọn) Yêu cầu nhập thông tin liên hệ (Tên, Email, SĐT) nếu đánh giá ẩn danh hoặc không qua link mời. \newline 3. US-08 thực hiện đánh giá: chọn số sao, nhập nhận xét. \newline 4. US-08 nhấn nút "Gửi Đánh giá" / "Submit Review". \newline 5. Hệ thống (System) kiểm tra tính hợp lệ cơ bản của dữ liệu (ví dụ: các trường bắt buộc như xếp hạng tổng thể phải được chọn). \newline 6. Hệ thống lưu bản ghi đánh giá mới vào cơ sở dữ liệu. \newline 7. Hệ thống hiển thị thông báo cảm ơn khách hàng đã gửi đánh giá. \\
\hline
Alternative Flow & \textbf{2a. Đăng nhập để gửi đánh giá:} \newline    1. Nếu khách hàng đã có tài khoản và đăng nhập, form có thể tự động điền một số thông tin hoặc liên kết đánh giá với tài khoản của họ. \newline \textbf{3a. Tải lên hình ảnh kèm theo đánh giá (nếu hỗ trợ):} \newline    1. Form cho phép khách hàng tải lên một hoặc nhiều hình ảnh liên quan đến trải nghiệm của họ. \\
\hline
Exception Flow & \textbf{5a. Lỗi xác thực dữ liệu trên form đánh giá:} \newline    1. Khách hàng bỏ trống trường bắt buộc. \newline    2. Hệ thống báo lỗi, yêu cầu hoàn thiện. \newline \textbf{6a. Lỗi hệ thống khi lưu đánh giá:} \newline    1. Hệ thống gặp lỗi kỹ thuật. \newline    2. Hệ thống báo lỗi chung. Đánh giá có thể không được lưu. \\
\hline
\multicolumn{2}{|c|}{\textbf{2.3. Thông tin bổ sung (Additional Information)}} \\
\hline
\textbf{Mục} & \textbf{Nội dung} \\
\hline
Business Rule & - \textbf{BR-UC11.14-1:} Form đánh giá nên được thiết kế để thu thập được cả thông tin định lượng (xếp hạng sao) và định tính (nhận xét văn bản). \newline - \textbf{BR-UC11.14-2:} Cần có cơ chế để liên kết đánh giá với một đơn hàng/đặt chỗ cụ thể (nếu có thể) để dễ dàng đối chiếu và xử lý. \newline - \textbf{BR-UC11.14-3:} Nhà hàng cần có chính sách về việc xử lý và phản hồi các đánh giá của khách hàng. \\
\hline
Non-Functional Requirement & - \textbf{NFR-UC11.14-1 (Usability):} Form đánh giá phải đơn giản, dễ sử dụng, và không quá dài dòng để khuyến khích khách hàng hoàn thành. Tương thích tốt trên thiết bị di động. \newline - \textbf{NFR-UC11.14-2 (Accessibility):} Form đánh giá dễ dàng truy cập từ nhiều kênh (email, web, QR). \\
\hline
\end{longtable}

% Phần tiếp theo sẽ là các Use Case còn lại của  Module MD-11 mà bạn đã yêu cầu.



\end{document}